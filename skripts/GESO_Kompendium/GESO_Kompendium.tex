%%
%% Meta: Master Document
%% Kompendium GESO
%%

\input{bbwLayoutDoc}

\renewcommand{\author}{Philipp G. Freimann}
\renewcommand{\grafikautor}{Ph. G. Freimann}
\renewcommand{\authoremail}{philipp.freimann@bbw.ch}
\renewcommand{\erstellungsdatum}{14. Nov. 2023}
\renewcommand{\docversion}{0.0.2 (\LaTeX{})}

\renewcommand{\doctitel}{Kompendium Mathematik}
\renewcommand{\untertitel}{Fachbereich Gesundheit / Soziale Arbeit [GESO]}
\renewcommand{\fachthema}{Aufgaben}
\renewcommand{\rechte}{\textbf{Nutzungsrechte} Creative Commons CC-NB-CY}

%%%%
%%Makros


%%%%%%%%%%%%%%%%%%%%%%%%%%%%%%%%%%%%%%%%%%%%555
%% Englisch / deutsch
%% Variable: «isEnglisch»
\newif\ifisEnglisch%%
\newif\ifisLoesungsteil%%

%% Flags

%% use
%% deu for deutsch
%% eng for english
\newcommand{\kLanguage}{eng}

%% nur Lösungen (ohne Aufgabentext)
%% true: nur lösungen
%% (ohne Lösungen: schreibe 'false'
\newcommand{\kLoesungen}{false}

%% nur Niveau Aufgaben
%% falls "nurBMP", werden nur die Aufgaben mit
%% Prüfungsniveau ins Kompendium aufgenommen
%% ansonsten "auchTraining", dann werden alle Aufgaben angezeigt.
\newcommand{\kAufgabenNiveau}{auchTraining}

%% mit mm-Papier
%% true
%% (ohen mmMapier schreibe 'false'
\newcommand{\kMMPapier}{false}

%% Versionsbezeichnung
\newcommand{\kVersion}{V0.0.1 Nov. 2023}


%% Ändern auf <true> für Englisch; ändern auf «false» für Deutsch.
%%\isEnglischfalse%%
\newcommand{\eng}[1]{\ifisEnglisch{#1}\else{}\fi{}}
\newcommand{\deu}[1]{\ifisEnglisch{}\else{#1}\fi{}}

%%%%%%%%%%%%%%%%%%%%%%%%%%%%%%%%%%%%%%%%%%%%555
%% Lösungsteil
%% Variable: «isLoesungsteil»
%% Ändern auf <true> für mit Lösungen, Ändern auf <false> für ohne Lösungen
%%\isLoesungsteilfalse%% 


%%%%%%%%%%%%%%%%%%%   Aufgabenblock %%%%%%%%%%%%%%%%%%%%%%%%%%%%%%%%%5
\newcounter{aufgabenNummer}
\setcounter{aufgabenNummer}{1}

\newcommand{\printAufgabentext}[1]{%%
\ifdefined\trainingsAufgabe%%
%%{\color{blue}«Training»}:
\vspace{-5mm}\begin{tcolorbox}[colback=blue!20,enhanced,sharp corners,frame hidden,halign=left]
#1
\end{tcolorbox}
\else
%%{\color{red}«Niveau»}:
\vspace{-5mm}\begin{tcolorbox}[colback=red!30,enhanced,sharp corners,frame hidden,halign=left]
#1
\end{tcolorbox}
\fi
}%% end command PrintAufgabenText


\newcommand{\mkaufgabe}[2]{\arabic{aufgabenNummer}.
\ifisLoesungsteil{{\color{blue} #2}

}%%
\else{%%
\printAufgabentext{#1}

%%\mmPapier{4}%%
}%%
\fi%%
\stepcounter{aufgabenNummer}%%
}%% end makro mkaufgabe


\newcommand{\kaufgabe}[2]{%%
\undef\trainingsAufgabe%%
\mkaufgabe{#1}{#2}}%% end makro kaufgabe


\newcommand{\kaufgabeTRAINING}[2]{%%
\def\trainingsAufgabe{true}
\mkaufgabe{#1}{#2}}%%




%%%%%%%%%%%%%%%%%%%%%%%%%%%%%%%%%%%%%%%%%%%%%%%%%%%%%%%%%%%%%%%%%%%
%%%%%%%%%%%%%% A r i t h m e t i k   u n d   A l g e b r a  I
\begin{document}

\ptitlepage

%%\newpage
\newpage
%%%%%%%%%%%%%%%%%%%%%%%%%%% Arithmetik und Algebra %%%%%%%%%%%%%%%%%%%%%
\section{\deu{Arithmetik und Algebra}\eng{Aritmetics and Algebra}}
\subsection{\deu{Zahlen}\eng{Numbers}}\deu{\index{Zahlmengen}}\eng{\index{numbers}}

\kaufgabeTRAINING{
\deu{Ordnen Sie die folgenden Zahlen der am weitesten links stehenden
Zahlmenge zu}\eng{Find the least powefull set for the given number
(the rightmost ist the most powerfull set of numbers)}:

$$\mathbb{N} \subset \mathbb{Z} \subset \mathbb{Q} \subset \mathbb{R}$$

\begin{multicols}{2}
\begin{enumerate}[label=\alph*)]
 \item$-\sqrt{\frac{16}{2}}$
 \item$|-\pi|$
 \item$1.\overline{25}$
 \item$\frac{\sqrt{8}}{\sqrt{2}}$
\end{enumerate}
\end{multicols}

}{%% Lösungsteil
\begin{multicols}{2}
\begin{enumerate}[label=\alph*)]
 \item$-\sqrt{\frac{16}{2}}$ : $\mathbb{R}$
 \item$|-\pi|$  : $\mathbb{R}$
 \item$1.\overline{25}$ : $\mathbb{Q}$
 \item$\frac{\sqrt{8}}{\sqrt{2}}$ : $\mathbb{N}$
\end{enumerate}
\end{multicols}
}%% end Kaufgabe


\kaufgabeTRAINING{%%
Setzen Sie zwischen die Terme (an die Stelle der drei Punkte) die
richtigen Relationszeichen ($<$, $=$, $>$):

\begin{multicols}{2}
\begin{enumerate}[label=\alph*)]
 \item$|5-2| ... |2-5|$
 \item$-\sqrt{3} ... -\sqrt{2}$ 
 \item$-3 ... |-3|$ 
 \item$\frac{1}{3} ... \frac{1}{4}$ 
 \item$\frac{2}{3} ... \frac{3}{4}$
 \item$2.\overline{9}$ ... 3
 \item$-\frac{1}{3} ... -\frac{1}{4}$ 
\end{enumerate}
\end{multicols}
}{%% Lösungsteil
\begin{multicols}{2}
\begin{enumerate}[label=\alph*)]
 \item$|5-2| =  |2-5|$
 \item$-\sqrt{3} < -\sqrt{2}$ 
 \item$-3 < |-3|$ 
 \item$\frac{1}{3} > \frac{1}{4}$
 \item$\frac{2}{3} < \frac{3}{4}$
 \item$2.\overline{9} = 3$
 \item$-\frac{1}{3} <-\frac{1}{4}$
\end{enumerate}
\end{multicols}
}%% end Kaufgabe Training


\subsection{Terme}

\kaufgabeTRAINING{
Benennen Sie die folgenden Terme mit dem richtigen Begriff
(Jeweils einer aus: \textit{Summe}, \textit{Differenz}, \textit{Produkt},
\textit{Quotient}, \textit{Potenz}):

\begin{multicols}{3}
\begin{enumerate}[label=\alph*)]
 \item $(3+x)\cdot{}2 + y$ 
 \item $(4a)^x$
 \item$4a^x$
 \item$(x-2y^4z)^2$
 \item$2a^7 - 4bc(z-1)^3$
 \item$(2a -b) : x + z : 4$
 \item$(a-4)(z+3)$ 
 \item$(2a-b):(x+z:4)$ 
\end{enumerate}
\end{multicols}
}{
\begin{multicols}{3}
\begin{enumerate}[label=\alph*)]
 \item $(3+x)\cdot{}2 + y$ :Summe
 \item $(4a)^x$ : Potenz
 \item$4a^x$ : Produkt
 \item$(x-2y^4z)^2$ : Potenz
 \item$2a^7 - 4bc(z-1)^3$ : Differenz
 \item$(2a -b) : x + z : 4$  : Summe
 \item$(a-4)(z+3)$   :Produkt
 \item$(2a-b):(x+z:4)$ :Quotient
\end{enumerate}
\end{multicols}
}%% end kaufgabe

\subsection{Betrag}
\subsection{Runden}
\subsection{Addition / Subtraktion}
\subsection{Multiplikation / Division}
\subsubsection{Faktorisieren}
\subsubsection{Bruchterme}



%%%%%%%%%%%%%%%%%%%%%%%%%%%%%%%%%%%% GLeichungen %%%%%%%%%%%%%%%%%%%%%%%%%%

\section{\eng{Equations}\deu{Gleichungen}}

\deu{Vereinfachen Sie}\eng{simplify}:

\kaufgabeTRAINING{$|-4|$}{$4$}
\kaufgabeTRAINING{$-\big| -5-|-8|  \big|$}{$-13$}

\deu{Terme}\eng{terms}:

\kaufgabe{$T(x) = (-4)\cdot{}|x-8|\cdot{}(x^2)$\\
\deu{Berechnen Sie}\eng{calculate}: $$T(-2)$$ }{$80$}%% end kaufgabe


\kaufgabeTRAINING{}{}

\subsection{\deu{lineare Gleichungen}\eng{linear
equations}}\deu{\index{lineare Gleichungen}\index{Gleichungen!lineare}}\eng{\index{linear equations}\index{equations!linear}}

\subsection{\deu{quadratische Gleichungen}\eng{quadratic equations}}
\subsection{\deu{Bruchgleichungen}\eng{übersetzte: equations with
fractions}}
\subsection{\deu{Potenzgleichungen}\eng{translate: Potenzgleichungen}}
\subsection{\deu{Exponentialgleichungen}\eng{exponential equations}}


\section{\deu{Funktionen}\eng{Functions}}
\subsection{\deu{lineare Funtkonen}\eng{linear functions}}
\subsection{\deu{Potenzfunktionen}\eng{Powerfunctions}}
\subsection{\deu{Exponentialfunktionen}\eng{exponential functions}}
\subsubsection{\deu{Wachstum und Zerfall}\eng{translate «Wachstum und Zerfall»}}


\section{\deu{Datenanalyse}\eng{Data analysis}}
\subsection{\deu{Grundlagen}\eng{Basics}}
\subsection{\deu{Maßzahlen}\eng{translate: Maßzahlen}}
\subsection{\deu{Diagramme}\eng{Diagrams}}



\section{\deu{Stochastik}\eng{Stochastics}}
\subsection{\deu{Kombinatorik}\eng{«Combinatorics»}}
\subsection{\deu{Wahrscheinlichkeitsrechnung}\eng{Probability Calculations}}
\subsubsection{\deu{Kontingenztafeln}\eng{translate: Kontingenztafeln}}
\subsection{\deu{Statistisches Schließen}\eng{translate: Statistisches Schließen}}
%%%%%%%%%%%%%%%%%%%%%%%%%%%%%%%%%%%%%%%%%%%%%%%%%%%%%%%%%%%%%%%%%%%
%%\newpage\mbox{}
%%\blankOddPage{}%
%\include{texlife-bibtex-extra}
%% \bibliography{bibAll}{}\label{literatur}

 \printindex
%%%%%%%%%%%%%%%%%%%%%%%%%%%%%%%%%%%%%%%%%%%%%%%%%%%%%%%%%%%%%%%%%%%

\end{document}
