%%
%% 2019 07 04 Ph. G. Freimann
%%
\section{Quadratische Gleichungen I}\index{Gleichungen!quadratische}
\sectuntertitel{
al-Kit\={a}b al-muhta\d{s}ar f\={i} \d{h}is\={a}b al-gabr wa-\'{}l-muq\={a}bala
}

\theorieGESO{165}{10}
\theorieTALS{93}{2.3}

Einführendes Youtube Video: \texttt{youtu.be/ZywdPuXR0S0}

%%%%%%%%%%%%%%%%%%%%%%%%%%%%%%%%%%%%%%%%%%%%%%%%%%%%%%%%%%%%%%%%%%%%%%%%%%%%%%%%%
\subsection*{Lernziele}

\begin{itemize}
\item Normalform
\item Lösungsmethoden
  \begin{itemize}
    \item{Fallunterscheidung}
    \item{Zerlegung in Linearfaktoren}
    \item{Substitution}
  \end{itemize}
\item Allgemeine Form, ABC-Formel (Mitternachtsformel)
\item Rationale Gleichungen
\item Definitionsbereich
\item Taschenrechner: Ohne Parameter
\item Mögliche Lösungsverfahren und sinnvolle Anwendung
\TALS{\item Taschenrechner: Visualisierung, Interpretation}
\TALS{\item Linarfaktoren}
\TALS{\item Substitution}
\end{itemize}

Al-Chwarizmis Buch (
al-Kit\={a}b al-muhta\d{s}ar f\={i} \d{h}is\={a}b al-gabr wa-\'{}l-muq\={a}bala) \textit{über die Rechenverfahren durch Ergänzen und
Ausgleichen} trägt zum Namen «Algebra» bei. In seinem Buch wird die
Quadratische Gleichung in allen Varianten gelöst.
\newpage

\subsection{Einstiegsbeispiele}
$$x^2=50$$

\TNT{2}{$x=\pm\sqrt{50} = \pm 5\sqrt2$}

$$(x+5)^2=50$$

\TNT{4}{$x+5=\pm\sqrt{50} \Rightarrow x = \pm\sqrt{50} - 5$, danach unbedingt Probe vorzeigen.}

%%\TRAINER{Normalform $x^2 +5x +6 = 0 \rightarrow (x+2)(x+3)=0$}
%%\newpage

\TALS{%%
%% 2020 03 27 Ph. G. Freimann
%%
\subsection{Spezialfälle}\index{Spezialfälle!quadratische Gleichungen}

\theorieTALS{95 ff}{2.3.3.1}

\subsubsection{$c=0$}
Wenn eine quadratische Gleichung der Form
$$ax^2 +bx = 0$$
gegeben ist, so kann man einfach ein $x$ ausklammern:

$$x(ax+b)=0$$
 Die Gleichung ist erfüllt, wenn nun entweder $x$ selbst oder aber
 der Klammerausdruck $(ax+b)$ Null werden. Somit haben wir sofort zwei
 Lösungen gefunden:
 $$\mathbb{L}_x=\{0, \frac{-b}{a}\}$$

 \TALSAadB{95ff}{265. a) b) c) e) g)}

 \subsubsection{$b=0$}
 Ist die quadratische Gleichung in der Form
 $$ax^2 + c = 0$$
 gegeben, so gibt es ebenfalls eine einfache Lösungsformel. Es folgt:
 $$ax^2 = -c$$
 und daraus:
 $$x^2 = \frac{-c}{a}$$

 Die Lösungsmenge ist also schlicht
 $$\mathbb{L}_x=\left\{ + \sqrt{\frac{-c}{a}}, -\sqrt{\frac{-c}{a}} \right\}.$$

\TALSAadB{95ff}{266. b) c) d)}
}

\aufgabenfarbe{Einstieg mit separatem Aufgabenblatt}
\newpage


\subsection{Quadratische Ergänzung}\index{quadratische Ergänzung}\index{Ergänzung!quadratische}
Normalform:
$$3x^2 - 11x - 4 = 0$$

\TNT{13.2}{
$+4$, dann $:3 \Rightarrow$
%%
  $$x^2 - {\color{red}\frac{11}{3}}x  = \frac{4}{3}$$
  quadratisch ergänzen:
  $$x^2 - {\color{red}\frac{11}{3}}x + ({\color{green}\frac{1}{2}}\cdot{\color{red}\frac{11}{3}})^2 = \frac{4}{3} + ({\color{green}\frac{1}{2}}\cdot{\color{red}\frac{11}{3}})^2$$
%%
faktorisieren mit binomischer Formel:
  $$(x - \frac{1}{2}\cdot\frac{11}{3})^2 = \frac{4}{3}+ \frac{121}{36}$$
links Brüche multiplizieren, rechts ersten Bruch mit 12 erweitern:
  $$(x - \frac{11}{6})^2 = \frac{48}{36}+ \frac{121}{36}$$
%%
  $$(x - \frac{11}{6})^2 = \frac{169}{36}$$
%%
Wurzel ziehen (Achtung, es kann zwei Lösungen geben)
%%
   $$x - \frac{11}{6} = \pm\sqrt{ \frac{169}{36}}$$
%%
   $$x - \frac{11}{6} = \pm \frac{13}{6}$$
%%
   $$x  = \frac{11}{6} \pm \frac{13}{6}$$
%%
   Erste Lösung $x=\frac{24}{6} = +4$ und zweite Lösung $x =
   -\frac{2}{6} = -\frac{1}{3}$.
%%
   Probe: 4 und $-\frac{1}{3}$ einsetzen.
}%%
\newpage




\subsection{$a$-$b$-$c$ -- Formel}\index{abc-Formel!quadratische Gleichung}\index{Mitternachtsformel}
(Auch Mitternachtsformel)

Normalform ($a\ne 0$):
$$ax^2 + bx + c = 0$$
Lösung:

$$x_{1,2}=\frac{-b\pm\sqrt{b^2-4ac}}{2a}$$


\subsubsection{Beweis der abc-Formel}\index{Mitternachtsformel!Beweis}\index{abc-Formel!Beweis}
\TRAINER{nach dorfuchs: [$-c$] [$\cdot{}4a$] [$+b^2$] [TU: bin. Formel] [$\pm\sqrt{\,\,}$] [$-b$] [$/2a$]}

\TNT{16}{%%
\begin{tabular}{c|c|c}
Zahlenbeispiel       & Vorgehen                     &  Allgemein           \\
\hline
$2x^2-14=4x$         &                              &  Normalform:        \\
                     & in Grundform bringen         &                     \\
$2x^2-4x-14 = 0$     &                              &  $ax^2 + bx + c =0$  \\
                     & $x$ separieren               &                      \\
$2x^2-4x=14$         &                              &  $ax^2 + bx = -c$    \\
                     & Durch «$a$» teilen          &                      \\
$x^2 - 2x = 7$       &                              &  $x^2 + \frac{b}{a} = -\frac{c}{a}$ \\
                     & quadratisch ergänzen         &
                                         \\
$x^2 - 2x + 1 = 7+1$ &                              &  $x^2 + \frac{b}{a} + \left(\frac{b}{2a}\right)^2= -\frac{c}{a}+ \left(\frac{b}{2a}\right)^2$ \\
                      & Binomische Formel            &
                                                                                                   \\

$(x-1)^2 = 8$        &                              &  $(x+\frac{b}{2a})^2= -\frac{c}{a}+ \left(\frac{b}{2a}\right)^2$ \\
                     & $\pm$ Wurzel ziehen          &
                                                \\
$x-1 = \pm\sqrt{8}$  &                              &  $x+\frac{b}{2a}= \pm\sqrt{-\frac{c}{a}+ \left(\frac{b}{2a}\right)^2}$ \\
                     & nochmals $x$ separieren
&
  \\
$x   = 1 \pm \sqrt{8}$  &     &  $x= -\frac{b}{2a} \pm \sqrt{-\frac{c}{a}+ \left(\frac{b}{2a}\right)^2}$ \\
\hline
& schön darstellen&\\
\hline
 & & $x_{1,2} = -\frac{b}{2a} \pm \sqrt{\frac{b^2}{4a^2} - \frac{4ac}{4a^2}}$ \\
$x_{1,2} = 1 \pm\sqrt{2\cdot{}4}$ & & $x_{1,2} = \frac{-b\pm \sqrt{b^2-4ac}}{2a}$ \\
$\mathbb{L}_x = \{1 - 2\sqrt{2}; 1+2\sqrt{2}\}$ & & $\mathbb{L}_x = \{\frac{-b-\sqrt{b^2-4ac}}{2a}; \frac{-b+\sqrt{b^2-4ac}}{2a}\}$ \\
\end{tabular}
\vspace{3cm}
}%% END TRAINER
\newpage

\subsubsection{Anwenden der $a$-$b$-$c$---Formel}
\begin{rezept}{$a$-$b$-$c$-Formel}{rezept_abc_formel}

  $$3x^2 -9x + 6 = 0$$

\TNT{8}{%%
  Lösung:

  $a = 3$

  $b = -9$

  $c= 6$

  Einsetzen in

  $\frac{-b \pm \sqrt{b^2-4ac}}{2a}$

  $$\frac{-(-9) \pm \sqrt{(-9)^2-4\cdot{}3\cdot{}6}}{2\cdot{}3} = \frac{+9 \pm \sqrt{81 - 72}}{6}$$

  $\mathbb{L}_x = \{1; 2\}$
  }%% END TRAINER
\end{rezept}
\newpage


\subsection{Diskriminante}\index{Diskriminante}
Die \textbf{Diskriminante} ist das, was unter der Wurzel steht (= Radikand)

$D := b^2 - 4ac$

«diskriminieren = unterscheiden»

Mit der Diskriminante ist eine einfache \textbf{Fallunterscheidung} möglich:
\begin{itemize}
\item $D > 0 \Rightarrow $ zwei Lösungen:
  $$x_{1,2} = \frac{-b \pm \sqrt{D}}{2a} = \frac{-b \pm \sqrt{b^2 -
    4ac}}{2a}$$

\item $D = 0 \Rightarrow $ eine Lösung:
  $$ x = \frac{-b}{2a}$$

\item $D < 0 \Rightarrow $ keine Lösung in $\mathbb{R}$

\end{itemize}

\subsubsection{Anwendung}

\begin{tabular}{c|c|c}
  $x^2 + 3x +1 = 0$ & $b^2-4ac = 9 -4 > 0$ & zwei Lösungen \\
  \hline
  $x^2 + 2x +1 = 0$ & $b^2-4ac = 4 -4 = 0$ & eine Lösung \\
  \hline
  $x^2 + 1x +1 = 0$ & $b^2-4ac = 1 -4 < 0$ & keine Lösungen \\
\end{tabular}



\GESO{
  \subsection*{Aufgaben}
Vorzeigeaufgabe: S.181: 13. b)%%
\GESOAadB{182}{3. a) e) 4. e) [mit Probe] 5. a) e) 6. d) 7. a) e)
  8. a) b) d) e) i) 9. c) 10. b) e) 11. c) f) 13. a) [falls 13. b) vorgezeigt]}
}%% END GESO

\newpage


%%
%% 2020 02 fp@bbw.ch
%%

%% Quadratische Gleichungen: Substitution
\subsection{Substitution}\index{Substitution!quadratische Gleichungen}

\TRAINER{
Zur Erinnerung beim Faktorisieren:\\
$a(2s-r) + b(2s-r) {\text{Subst. }T:=(2s-r) \over =} aT + bT = (a+b)T {\text{Rücksusbst. }2s-r=T\over = } (a+b)(2s-r)$
}

Oft können kompliziertere Gleichungen mittels einer geeigneten \textbf{Substitution} (= Ersetzung\footnote{lat. \textbf{substitu\=o} = an die Stelle \textit{jmds. o. einer Sache} setzen.
})
in eine quadratische oder lineare Gleichung verwandelt werden. Wie \zB hier:

\newcommand\tmpPart{\left(\frac{1}{3}(5x+0.5)\right)}

$$-6\tmpPart{} + 8.75 = -{\tmpPart{}}^2$$

%%
\TRAINER{«Mit Farben geht alles besser.»\\}%%
\TNT{2.4}{$$-6{\color{ForestGreen}\tmpPart{}} + 8.75 = -{\color{ForestGreen}{\tmpPart{}}^{\color{black}2}}$$}

\textbf{Substitution}

\TNT{1.6}{Wir ersetzne: $y := {\color{ForestGreen}\tmpPart{}}$}

und die ursprüngliche Gleichung ist nun äquivalent zu
\TNT{1.6}{$$-6{\color{ForestGreen}y} + 8.75 = -{\color{ForestGreen}y}^2.$$}

Substituierte Gleichung \GESO{mit TR }lösen:


$$\mathbb{L}_y=\LoesungsRaumLang{\{2.5; 3.5\}}$$

\platzFuerBerechnungen{7.2}%% Bis Ende Seite nicht möglich wegen Fussnote

\TALS{\TRAINER{
  Erst in Grundform bringen $$y^2 - 6y + 8.75 = 0$$
  und danach mit der $a$-$b$-$c$ -- Formel auflösen ($a=1, b=-6, c=8.75$).
}}%%

\TRAINER{\textbf{Nomenklatur:}\\
  Substitutionsbasis: $-6\tmpPart{} + 8.75 = -\tmpPart{}^2$\\
  Substituendum: $\tmpPart{}$\\
  Substituens: $y$
}%%

\newpage


\textbf{Rücksubstitution}\index{Rücksubstitution}\\
Nun setzen wir die Lösungen anstelle der substituierten Variable ein:


\TNTeop{
$$1.: y_1 = 2.5 = \tmpPart{} \Rightarrow 7.5 = 5x+0.5 \Rightarrow x = \frac{7}{5} = 1.4$$

$$2.: y_2 = 3.5 = \tmpPart{} \Rightarrow 10.5=5x+0.5 \Rightarrow x = 2$$

$$\lx=\{1.4; 2\}$$}%% END TNTeop

%%%%%%%%%%%%%%%%%%%%%%%%%%%%%%%%%%%55555


\subsection*{Aufgaben}

\GESO{
  \subsubsection{Nullserie}
  Die folgende Aufgabe stammt aus der \textit{Nullserie}-Maturaprüfung:

  Lösen Sie mit Hilfe einer geeigneten Substitution \TRAINER{Tipp: $Z^{10} = \left(Z^5\right)^2$ }:
  $$\left(\frac{x}{3}-3.5 \right)^{10} -2\left( \frac{x}{3} - 3.5 \right)^5 =-1$$
  
  \TNTeop{Substituiere $z = \left(\frac{x}{3} - 3.5\right)^5$. Somit lautet die Gleichung
$$z^2 - 2z +1 = 0$$
    Was uns zu $z = 1$ bringt. Damit ist $1 = \left(\frac{x}{3} - 3.5 \right)^5$ und somit auch $1 = \frac{x}{3} - 3.5$. Auf beiden Seiten 3.5 addieren und danach mit 3 multiplizieren liefert $x = 13.5$.
 }%% End TNT

  %%%%%%%%%%%%%%%%%%%%%%%%%%%%%%%%%%%%
  
\aufgabenFarbe{Berufsmaturitätsprüfung 2020: Aufgabe 4 von Serie 2:
  $$(x-5)^4 - \frac{(x-5)^2}{3} = 8$$}%% END Aufgabenfarbe
\TNT{8}{Lösung der Substituierten $y=(x-5)^2)$ ist $y=3$ bzw. $y=-\frac83$. Die zweite Lösung kommt jedoch nur als Scheinlösung vor.\vspace{45mm}}%% END TNT
    $$\lx = \LoesungsRaumLang{\left\{5-\sqrt{3}; 5+\sqrt{3}\right\}}$$
  
}%% END GESO

\AadBMTA{184}{28. a) d) 29. a) c) 30. a) b)}
%%\TALSAadBMTA{99ff (Substitution)}{283. a) b) f) 282. a) f) 284. a)}


\olatLinkGESOKompendium{2.3.3.}{17}{52. bis 54.}

\newpage



\newpage
\subsection{Welches Verfahren}
Wann soll welches Verfahren eingesetzt werden? Dies ist eine
individuelle Fragestellung. Je mehr Erfahrung man hat, umso eher sieht
man die Spezialfälle. Die $a$-$b$-$c$-Formel funktioniert immer, doch
die anderen Verfahren (Binome, weitere Spezialfälle, Taschenrechner)
sind oft sinnvoller. Hier ein Versuch eines Überblicks:

\GESO{
\begin{tabular}{|p{44mm}|p{53mm}|p{64mm}|}
	\hline
	Spezialfall $b=0$               & $5x^2 = 3$                   & Durch 5 dividieren, dann positive und negative Wurzel ziehen.\\
	\hline
	Spezialfall $c=0$               & $3x^2 = 5x$                   & Fallunterscheidung $x$ = 0 und $x \ne 0$\\
	\hline
	Bereits faktorisiert       & $(x-4.6 + b)\cdot{}(x+a) = 0$ & Lösungen hier $x_1=4.6-b$ und $x_2 = -a$\\
	\hline
	Einfache Zahlen            & $x^2 -2x + 1= 0$           & $a$-$b$-$c$--Formel (Mitternachtsformel) oder faktorisieren $x^2-2x+1=(x-1)^2$ und danach jeden einzelnen Faktor $=0$ setzen.\\
	\hline
	Komplizierte Zahlen        & $7.3x^2 - 8x - 3.4 = 0$       & Taschenrechner \tiprobutton{cos_poly-solv}             \\
	\hline
	Variable (Parameter)       & $7.3x^2 - cx + 2.6=0$         & $a$-$b$-$c$-Formel (Mitternachtsformel) \\
	\hline
	Komplexe Terme im Quadrat  & $3(x-6)^2 - 17(x-6)  = 0$     & Substitution                            \\
	\hline
\end{tabular}
\newpage
}

\GESO{
  \subsubsection{Taschenrechner Tipps}
  Der TI-30X Pro MathPrint kann quadratische Gleichungen mit Zahlen direkt auflösen.
  Doch dazu einige Tipps:
  
  \begin{tabular}{|c|p{13cm}|}
    \hline
    Tasten & Bemerkung \\
    \hline
    \tiprobutton{2nd}\tiprobutton{cos_poly-solv} & Starte das Lösen von quadratischen Gleichungen.\\
    \hline
    \tiprobutton{2nd}\tiprobutton{mode_quit}      & Beende den Poly-Solver.\\
    \hline
    \tiprobutton{neg} & Negative Zahlen nicht mit dem Subtraktionsoperator sondern mit der Vorzeichen-Taste eingeben.\\
    \hline
    \textbf{ACHTUNG} & Das Resultat lässt beim TI-30X Pro negative Wurzeln zu! Dies wird in der Lösung mit $i$ angegeben ($i$ steht für die Imaginäre Einheit: $i=\sqrt{-1}$). Das $i$ kann aber je nach Modus erst gesehen werden, wenn mit der Pfeiltaste ganz nach rechts \textit{gescrollt} wurde.
    
    Steht ein $i$ in der Lösung des Taschenrechners, so gibt es keine reellen Lösungen und wir schreiben

    $\mathbb{L}_x=\{\}$\\
    \hline
    \end{tabular} 
}

\TALS{
\begin{tabular}{|p{49mm}|p{53mm}|p{60mm}|}
	\hline
	Spezialfall $b=0$               & $5x^2 = 3$                   & Durch 5 dividieren, dann positive und negative Wurzel ziehen.\\
	\hline
	Spezialfall $c=0$               & $3x^2 = 5x$                   & Fallunterscheidung $x$ = 0 und $x \ne 0$\\
  \hline
	Bereits faktorisiert & $(x-4.6 + b)\cdot{}(x+a) = 0$ & Lösungen hier $x_1=4.6-b$ und $x_2 = -a$\\
	\hline
	Einfache Zahlen      & $6x^2 -3x - 18 = 0$               & $a$-$b$-$c$-Formel (Mitternachtsformel) \\
	\hline
	Komplizierte Zahlen  & $7.3x^2 - 8x - 3.4 = 0$           & Taschenrechner (solve())                \\
	\hline
	Variable             & $7.3x^2 - cx + 2.6=0$         & Taschenrechner                          \\
	\hline
	Gesucht Anzahl Lösungen (auf 1)
           beschränken & $7.3x^2 - cx + 2.6 = 	0$     & 	Diskriminante mit Taschenrechner Null setzen.\\
	\hline
\end{tabular} 
}
\newpage

\TALS{\subsection*{Aufgaben}
\TALSAadB{95ff}{265. a) c) e) 266. a) d) 267. e) 269. a) 275. a)
276. a) b) f)}%%
\TALSAadB{99ff (Substitution)}{283. a) b) f) 282. a) f) 284. a)}
}
\newpage
