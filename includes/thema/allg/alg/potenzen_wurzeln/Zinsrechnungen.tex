%%
%% 2019 07 04 Ph. G. Freimann
%%

\newpage
\section{Textaufgaben/Zinsrechnung}\index{Textaufgaben zur Zinsrechnung}
\sectuntertitel{``Klar hab' ich Probleme --- ich bin Mathelehrer.''}
%%%%%%%%%%%%%%%%%%%%%%%%%%%%%%%%%%%%%%%%%%%%%%%%%%%%%%%%%%%%%%%%%%%%%%%%%%%%%%%%%
\subsection*{Lernziele}

\begin{itemize}
  \item Der Zins ist eine  Multiplikation, der Zinseszins ist eine Potenz
\item Textaufgaben mit Zins und Zinseszins
\end{itemize}

\subsection{Der Zins als Faktor}\index{Zins}
Einstiegsaufgabe:
Ein Händler gibt Ihnen auf eine Ware von 234.50 CHF einen Rabatt von
5\%, wenn Sie sofort bezahlen. Wenn Sie in bar bezahlen, erhalten Sie
einen weiteren Rabatt von 3\%. Ist es für Sie nun schlauer, zuerst den
Sofort-Rabatt (5\%) und danach den Bar-Rabatt(\%) einzufordern oder
ist die umgekehrte Reihenfolge schlauer? \TRAINER{Lsg: 216.10}

\subsubsection{100\% = 1}
Um von einer Ausgangsgröße 100\% zu berechnen, kann einfach der Faktor
1.0 genommen werden. Ebenso kann der Faktor 0.03 genommen werden, um
3\% der Größe zu berechnen. Ein Anwachsen eines Kapitals um 3\% ist
also nichts anderes als das multiplizieren mit dem Faktor 1.03.

\subsubsection{Zinseszins}\index{Zinseszins}
Beim Zinseszins, wird bei jeder weiteren Verzinsung der bereits
erhaltene Zins weiterverzinst. Beispiel 2\%:

CHF 100.- $\rightarrow$ 102.-- $\rightarrow$ 104.04 $\rightarrow$
106.1208

Bei 2\% kann also jedesmal mit einem \textbf{Verzinsungsfaktor} von
1.02 multipliziert werden.
Nach 1000 Jahren wachsen unsere CHF 100.-- also auf $100 *
1.02^{1000}$ an\footnote{Dies entspricht einem Faktor von fast 400 Millionen}. 
\newpage


\subsection{Zinsformel}\index{Zins und Zinseszins}
\TRAINER{Hinweis an die Lehrperson:Insbesondere BM2 gut behandeln, denn der Stoff ist ev. in der Sekundarschule nicht behandelt worden (Sek B) oder es liegt generell zu weit zurück.}

Bei gegebenem Startkapital $K_0$ und gegebenem Zinsfuß $p$ (in \%) kann das Endkapital $K_n$ nach $n$ Jahren wie folgt berechnet werden:

\begin{center}\fbox{$K_n = K_0 \cdot{} f^n$}\end{center}

mit

\begin{center}\fbox{$f := 1 + \frac{p}{100}.$}\end{center}

\begin{tabular}{lcl|l}
  \textit{Zeichen}  & &   \textit{Bedeutung}   & \textit{Beispiel}\\%%
\hline%%
 $K_0$             &=&   Startkapital         & 100.-  [CHF]\\
 $n$               &=&   Laufzeit in Jahren   & 30  [Jahre]\\
 $p$               &=&   Zinsfuß              & 2  [\%]\\
 $f:= 100\% + p\%$ = $1+\frac{p}{100}$ &=&   Zinsfaktor          & hier: $f=1.02$\\
$K_n = K_0\cdot{}f^n$     &=&   Kapital nach $n$ Jahren         & $100 \cdot{} 1.02^{30}\approx{} 181.14$ [CHF]\\%%
\hline\\%%
\end{tabular}

Zeigen Sie, dass gilt

$$K_1 = K_0 + \frac{K_0}{100}\cdot{} p = K_0 \cdot{} \left(1+\frac{p}{100}\right) = K_0 \cdot{} f$$

... und ...:

$$K_2 = K_0 \cdot{} f^2$$

\TNT{4.0}{Beweis: $K_0$ ausklammern. \vspace{3cm}}

Bemerkung: Die Zinseszinsformel beschreibt ein exponentielles Wachstum.
\newpage

\subsubsection{Zinsbeispiele}

Berechnen Sie:

\begin{itemize}
  \item Berechnen Sie das Endkapital nach 20 Jahren bei einem
  Startkapital von CHF 15\,000.- und einem Zins von jährlich
  2.5\%.\\%%

\TNT{2.4}{Endkapital = $15\,000\cdot{} 1.025^{20}\approx 24\,579.25$ CHF\vspace{2cm}}%%
\item In einem Wald werden 200 Fichten für Weihnachten gepflanzt. Wegen der hohen Nachfrage kommen jedes Jahr 3\% Fichten dazu. Wie viele Fichten werden nach fünf Jahren gepfanzt?


  \TNT{2.4}{Endkapital = $200\cdot{} (1.03)^{5} \approx 231$ Fichten.\vspace{2cm}}%%
\item Ein Auto hat einen Neupreis von CHF 40\,000.-. Jedes Jahr müssen wegen Abnutzung ein Wertverlust von 3\% in Kauf
  genommen werden. Wie viel Wert hat das Auto noch nach 15 Jahren? (Achtung, hier ist der Zins negativ!)

\TNT{2.4}{Endkapital = $40\,000\cdot{} (1-0.03)^{15} = 40\,000\cdot{} 0.97^{15}\approx 25\, 330.-$ CHF\vspace{2cm}}
\end{itemize}
\newpage

\subsubsection{Momentanverzinsung}\index{Momentanverzinsung}
(auch stetige Verzinsung)\index{Verzinsung!stetige}
Wird ein Kapital zu 100\% verzinst, so wächst unser Kapital auf 200\%
an. Wenn wir das Kapital jedoch zweimal zu 50\%
verzinsen\footnote{Sprich: Wir lassen uns nach 6 Monaten den Zins
auszahlen, heben das Konto auf und bezahlen sofort wieder mit Zins in
ein neues Konto ein.}, erhalten wir den folgenden, besseren Verzinsungsfaktor:

$1.50^2  = (1 + \frac12)^2 = 2.25$

Bei 4-maliger Verzinsung steigt der Faktor weiter an:
$1.25^4 = (1 + \frac14)^4 \approx 2.44 $

Füllen Sie die folgende Tabelle aus:

\begin{tabular}{c|c|c|c} 
  Anzahl Teile  & Faktor                      & Formel          & Endkapital \\ \hline
  $2$           & $1.5^2$                      & $(1+\frac12)^2$ & $= K_0 \cdot{} 2.25 $ \\ \hline
  $4$           & $1.25^4$                  & $(1+\frac14)^4$ & $\approx K_0 \cdot{} 2.4414 $ \\ \hline
  $5$           & $\LoesungsRaum{1.2^5}$  & $\LoesungsRaum{(1+\frac14)^2}$ & $\LoesungsRaum{= K_0 \cdot{} 2.48832} $ \\ \hline
  $10$           & $\LoesungsRaum{1.1^{10}}$  & $\LoesungsRaum{(1+\frac{1}{10})^{10}}$ & $\LoesungsRaum{\approx K_0 \cdot{} 2.5937} $ \\ \hline
  $100$           & $\LoesungsRaum{1.01^{100}}$  & $\LoesungsRaum{(1+\frac{1}{100})^{100}}$ & $\LoesungsRaum{\approx K_0 \cdot{} 2.7048 }$ \\ \hline
  $1000$           & $\LoesungsRaum{1.001^{1000}}$  & $\LoesungsRaum{(1+\frac{1}{1000})^{1000}}$ & $\LoesungsRaum{\approx K_0 \cdot{} 2.7169 }$ \\ \hline
  Großes $n$           & ****  & $\LoesungsRaum{(1+\frac{1}{n})^{n}}$ & $\LoesungsRaum{\approx K_0 \cdot{} e }$ \\ \hline
\end{tabular} 

Dieses maximal erreichbare Kapital entspricht etwa dem Faktor 2.71828 und
wird als Eulersche Konstante\index{Eulersche Zahl} bezeichnet.

\bbwCenterGraphic{10cm}{allg/img/euler_banknote.jpg}
Bildlegende: Leonhard Euler (1707-1783) auf der Schweizer Zehnernote.

\begin{definition}{Eulersche Zahl}{}
$$e \approx 2.71828182746$$
\end{definition}
\newpage

\GESO{\subsection*{Aufgaben}}
\TALSAadB{???}{???}

\GESOAadB{207}{10. a)} %% die anderen setzen Lograithmen voraus! und 12.}

\GESOAadB{207}{9. a) und 11. a)}
\GESOAadB{355}{13.}
