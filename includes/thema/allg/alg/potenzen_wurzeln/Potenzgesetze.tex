\newcommand{\aaaa}{a\cdot a \cdot a \cdot{} ... \cdot a}
\newcommand{\bbbb}{b\cdot b \cdot b \cdot{} ... \cdot b}

\newpage
\subsection{Potenzen, Definitionen und Gesetze}\index{Potenzen}

\subsubsection{Einführungsbeispiele}
Die sieben verschiedenen schweizer Münzen werden nacheinander geworfen
und es wird notiert, welcher Wurf Zahl oder «Kopf» aufweist. Dieses
Experiment wird einige Male wiederholt, bis man sich die Frage stellt:
Wie viele mögliche Ausgänge hat das Experiment?

\TNT{4.0}{$$2^7 = 128$$}

Berechen Sie
$$5^3$$

\TNT{2.8}{125}

Vereinfachen Sie:

$$a^5\cdot(ab)^3\cdot{}(2c)^{2+3}$$

\TNT{3.2}{$$32a^8b^3c^5$$}

\newpage

\begin{definition}{Potenz}{}\index{Potenz}
Unter der $n$-ten \textbf{Potenz} verstehen wir eine $n$-malige Multiplikaiton mit demselben Faktor:
$$a^2 = a\cdot a$$
$$a^n = \underbrace{\aaaa}_{n\, \textrm{Faktoren}}$$
\end{definition}
Dabei bezeichnen


\bbwCenterGraphic{9cm}{allg/alg/potenzen_wurzeln/img/Potenzbegriff.png}

%\begin{itemize}
% \item Potenz\index{Potenz}: $a^n$
% \item Exponent\index{Exponent}: $n$
% \item Basis\index{Basis}: $a$
%\end{itemize}


\subsubsection{gleiche Basis}\index{Potenzen!Gesetze}

\begin{itemize}
 \item \fbox{$a^m\cdot a^n = a^{m+n}$}

   Begründung:

   \TNT{2.4}{
   \TALS{$a^m\cdot a^n =
    \underbrace{{\underbrace{\aaaa}_{m\textrm{-mal}}}\cdot{\underbrace{\aaaa}_{n\textrm{-mal}}}}_{m+n\textrm{-mal}}
    = a^{m+n}$}
   \GESO{$7^3\cdot{} 7^5 = (7\cdot{}7\cdot{}7)\cdot{}(7\cdot{}7\cdot{}7\cdot{}7\cdot{}7)=7^8$}%%
}%% END TNT

   
   \item \fbox{$a^m : a^n = \frac{a^m}{a^n}= a^{m-n}$}
     
     Begründung:

     \TNT{2.4}{%
     \TALS{$a^m :  a^n =
    \underbrace{{\underbrace{\aaaa}_{m\textrm{-mal}}} : {\underbrace{\aaaa}_{n\textrm{-mal}}}}_{m+n\textrm{-mal}}
    = a^{m-n}$}
     \GESO{
  Kürzen: 
   $7^5 :  7^3 = \frac{7^5}{7^3} =
  \frac{7\cdot{}7\cdot{}7\cdot{}7\cdot{}7}{7\cdot{}7\cdot{}7} = 7^2 =
  7^{5-3}$}
     }%% END TNT

  
\item \fbox{$(a^n)^m = a^{n\cdot{}m} = (a^m)^n$}

  Begründung:

  \TNT{2.4}{Beispiel $(a^4)^3$ vorrechnen \vspace{12mm}}%%
  
\end{itemize}
\newpage

%%%%%%%%%%%%%%%%%%%%%%%%%%%%%%%%%%%%%%%%%%%%%%%%%%%%%%%%%%%

\subsubsection{gleiche Exponenten}

\begin{itemize}
\item \fbox{$a^n \cdot{} b^n  = (ab)^n$}
  
  Begründung

  \TNT{2.4}{
    \TALS{$a^n\cdot b^n
= \underbrace{\aaaa}_{n\textrm{-mal}}\underbrace{\bbbb}_{n\textrm{-mal}}
= \underbrace{ab\cdot ab\cdot ... \cdot ab}_{n\textrm{-mal}} =
(ab)^n$}%%
   \GESO{\zB $7^3 \cdot 5^3 = (7\cdot{}7\cdot{}7) \cdot{}
     5\cdot{}5\cdot{}5 = (7\cdot{}5)\cdot{} (7\cdot{}5)\cdot{} (7\cdot{}5) =(7\cdot{}5)^3$}
}%% end TRAINER
   
\item \fbox{$\frac{a^n}{b^n} = \left(\frac{a}{b}\right)^n$} (Gilt ganz analog.)
\end{itemize}

%%%%%%%%%%%%%%%%%%%%%%%%%%%%%%%%%%%%%%%%%%%%%%%%%%%%%%%%%%%%%%

\textbf{Aber Vorsicht}

$$a^2 + b^2 \ne ???$$

