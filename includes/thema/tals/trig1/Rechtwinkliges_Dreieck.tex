%%
%% 2019 07 04 Ph. G. Freimann
%%

\section{Rechtwinkliges Dreieck}\index{Dreieck!rechtwinkliges}\index{rechtwinkliges Dreieck}
\sectuntertitel{... nein, wir sind nicht Asterix und Obelix: Wir sind
  Römer, wir sind Sinus und Cosinus ...\footnote{S. Asterix -- Tour de France -- Seite 40}}

\theorieTALSGeom{84}{2.1}
%%\theorieGESO ????
%%%%%%%%%%%%%%%%%%%%%%%%%%%%%%%%%%%%%%%%%%%%%%%%%%%%%%%%%%%%%%%%%%%%%%%%%%%%%%%%%
\subsection*{Lernziele}

\begin{itemize}
 \item Satz des Pythagoras Formel
 \item Höhensatz
 \item sinus/cosinus/tangens
\end{itemize}



%% include Allgemeine Form und Höhensatz
%%
%% 2019 07 04 Ph. G. Freimann
%%

\section{Satz des Pythagoras}\index{Pythagoras!Planimetrie}\index{Satz des Pythagoras!Planimetrie}
\sectuntertitel{Warum sitzen die Mathematiker im Winter immer in eine
  Ecke? --- Weil dort bestimmt 90 Grad ist.}
%%%%%%%%%%%%%%%%%%%%%%%%%%%%%%%%%%%%%%%%%%%%%%%%%%%%%%%%%%%%%%%%%%%%%%%%%%%%%%%%%
\TadBMTG{29}{2.3.2}

\subsection*{Lernziele}

\begin{itemize}
  \item Formel des Pythagoras
\end{itemize}

Repetition \ifisALLINONE{\totalref{satzDesPythagoras}}\fi
\begin{bemerkung}{Pythagoras}{}
  Im rechtwinkligen Dreieck gilt:
  $$a^2+b^2=c^2$$
\end{bemerkung}

%%\TALSAadBFWG{23}{56. 57. 58. 60. 63. }
\subsection*{Aufgaben}
\AadBMTG{37}{11., 12., 13. und 31.}
%%\GESOAadBMTA{???}{???}
\newpage
\subsection{Spezielle Dreiecke}
Repetition: Halbes Quadrat und halbes gleichseitiges Dreieck \ifisALLINONE{\totalref{spezielleDreiecke}}\fi

\bbwCenterGraphic{8cm}{tals/plani/img/regulaer.png}
\TNT{12}{
  Beziehungen im Quadrat:

  $a$ = Seitenlänge

  $d$ = Diagonale

  $\frac{d}2$ = halbe Diagonale

  Beziehungen:  $a = \frac{d}2 \cdot{} \sqrt{2}; d = a\cdot{} \sqrt{2}$

  \vspace{35mm}

  Beziehungen im gleichseitigen (regulären) Dreieck:

  $a$ = Seitenlänge

  $h$ = Höhe

  $\frac{a}2$ = halbe Seitenlänge

  Beziehungen:

 $h = \frac{a}2 \cdot{} \sqrt{3};  a=2 \cdot{} \frac{h}{\sqrt{3}} $

  \vspace{35mm}

}%% END TNT

\subsection*{Aufgaben}
\AadBMTG{38ff}{19., 20., 24., 26., 28., 29., 38., 40. und 41.}
\newpage

\subsection{Kreisberührung}\index{Kreisberührung!Planimetrie}


\bbwCenterGraphic{8cm}{tals/plani/img/KleinsterKreis.png}
In obigem Kreis sind zwei kleinere Kreise einbeschrieben. Berechnen
Sie den Radius $k$ des kleinsten Kreises aus dem gegebenen Radius $r$ des
großen Kreises.
\TNT{12}{
  \bbwCenterGraphic{8cm}{tals/plani/img/KleinsterKreisLoesung.png}
  $$\Delta A: x^2 + k^2 = (r-k)^2$$
  $$\Delta B: x^2 + (\frac{r}2 - k)^2 = (\frac{r}2 + k)^2$$
  Ausmultiplizieren und 2. Gleichung von 1. Gleichung subtrahieren:
  $$k^2 - \frac{r^2}4 +rk - k^2  =r^2  -2rk - \frac{r^2}4 - rk$$\
  $$-\frac{r^2}4 + rk = r^2 - 3rk - \frac{r^2}4$$
  $$rk = r^2 - 3rk$$
  $$4k = r$$
  $$k = \frac{r}4$$
}%% END TNT
\newpage


\begin{rezept}{Kreisberührung}{}
  Wählen Sie zwei Kreise, die sich \textbf{berühren}.

  Von sich berührenden Kreisen werden die \textbf{Mittlpunkte miteinander verbunden}.

  Meist sind die Strecken zwischen den beiden Mittelpunkten
  \textbf{Hypothenusen} in rechtwinkligen Dreiecken. Dann verwenden Sie den
  \textbf{Satz des Pythagoras}.
\end{rezept}

\subsection*{Aufgaben}

\AadBMTG{39ff}{28., 33., 34.}
\newpage



\subsection{Rechtwinkliges Dreieck}

\definecolor{qqwuqq}{rgb}{0,0.39,0}
\definecolor{xdxdff}{rgb}{0.49,0.49,1}
\definecolor{qqqqff}{rgb}{0,0,1}
\begin{tikzpicture}[line cap=round,line join=round,>=triangle 45,x=1.0cm,y=1.0cm]
\clip(-0.54,0.17) rectangle (5.83,4.51);
\draw [shift={(3.35,3.58)},color=qqwuqq,fill=qqwuqq,fill opacity=0.1] (0,0) -- (-154.75:0.95) arc (-154.75:-64.75:0.95) -- cycle;
\draw (0,2)-- (4.56,1.02);
\draw (0,2)-- (3.35,3.58);
\draw (3.35,3.58)-- (4.56,1.02);
\fill[color=qqwuqq,fill=qqwuqq,fill opacity=0.1] (3.16,3.05) circle (0.03);
\begin{scriptsize}
\fill [color=qqqqff] (0,2) circle (1.5pt);
\draw[color=qqqqff] (0.25,2.42) node {$A$};
\fill [color=qqqqff] (4.56,1.02) circle (1.5pt);
\draw[color=qqqqff] (4.81,1.44) node {$B$};
\fill [color=xdxdff] (3.35,3.58) circle (1.5pt);
\draw[color=xdxdff] (3.61,4.01) node {$C$};
\draw[color=qqwuqq] (2.37,2.49) node {$90\textrm{\degre}$};
\end{scriptsize}
\end{tikzpicture}


\subsection{$\sin{}$, $\cos{}$ und $\tan{}$}
\TALS{(Theorie S. 84 Kap. 2.1 \cite{frommenwiler18geom})}


\newpage
\subsection{Fläche im allgemeinen Dreieck}\index{Fl\"ache!Dreieck}
\TALS{(S. 96 Kap. 2.1.3 \cite{frommenwiler18geom})}

\bbwCenterGraphic{6cm}{tals/trig1/img/dreieck_sws.png}

Im allgemeinen Dreieck kann die Fläche A wie folgt berechnet werden:
$$A = \frac{1}{2}\cdot{}b\cdot{}c\cdot{}sin(\epsilon)$$
Also die Hälfte von Seite mal Seite mal Sinus des Zwischenwinkels.
\newpage


\subsection{Steigung vs. Winkel}\index{Steigung}

\bbwCenterGraphic{4.5cm}{tals/trig1/img/starkeSteigung.jpg}

Steigungen werden üblicherweise in \% angegeben. Das obige
Verkehrsschild «starke Steigung» gibt 10\% an. Das heißt:

\TNT{3.6}{Auf einen horizontalen Meter, steigt die Straße um 10cm (=
  10\%).
\vspace{3cm}}

Um den Winkel zu berechnen, kann der Tangens verwendet werden:
$$\frac{0.10m}{1.00m} = \tan(\sigma)$$
In obigem Beispiel ist der Steigungswinkel also durch den $\arctan()$
berechenbar:

$$\sigma = \arctan(\frac{0.1m}{1.0m}) = \arctan(0.1) \approx 5.71\degre$$

\subsection*{Aufgaben}
\TALSGeomAadB{84ff}{2. alle 3. a) b) e) 4. a) c) 6. a) 8. 9. 15. 23. 26. a) b) c) 30.}%
\TALSGeomAadB{96}{59}
\GESOAadB{????}{????}
\newpage
