%% Trigonometrie III
%% Arkusfunktionen
%% 2020 - 12 - 21 φ@bbw.ch

\section{Arkusfunktionen}\index{Arkusfunktionen}



\subsection{Arcussinus}
Auf die Umkehrungen der Sinus-, Cosinus- und Tangensbeziehungen können
als Funktionen aufgefasst werden.

Berechnen Sie von einigen ArcSin-Verhältnissen den zugehörigen Winkel
(also \zB $\arcsin(0.5)$...) und tragen Sie die Werte ins folgende
Koordinatensystem im Bogenmaß ein:

\noTRAINER{\bbwCenterGraphic{8cm}{tals/trig3/img/py_system.png}}
\TRAINER{\bbwCenterGraphic{8cm}{tals/trig3/img/arcsin.png}}

Geben Sie den Definitionsbereich Arcus-Sinus-Funktion an:

$$\mathbb{D} = \LoesungsRaum{[-1; 1]}$$

Geben Sie den Wertebereich der Arcus-Sinus-Funktion einmal im Bogenmaß,
einmal im Gradmaß an:
$$\mathbb{W} = \LoesungsRaum{[-\frac{\pi}{2}; \frac{\pi}{2}]}  =  \LoesungsRaum{[-90\degre; 90\degre]}$$
\newpage

\subsection{Arccuscosinus}
Zeichnen Sie die Umkehrfunktion des Cosinus und geben Sie Definitionsbereich an.

\noTRAINER{\bbwCenterGraphic{8cm}{tals/trig3/img/py_system.png}}
\TRAINER{\bbwCenterGraphic{8cm}{tals/trig3/img/arcsin.png}}


$$\mathbb{D} = \LoesungsRaum{[-1; 1]}$$

Geben Sie den Wertebereich der Arcus-Sinus-Funktion einmal im Bogenmaß,
einmal im Gradmaß an:
$$\mathbb{W} = \LoesungsRaum{[0; \pi]}  =  \LoesungsRaum{[0; 180\degre]}$$
\newpage


\subsection{Arcustangens}
Zeichnen Sie die Umkehrfunktion des Tangens (im Bogenmaß) und geben Sie Definitionsbereich an.

\noTRAINER{\bbwCenterGraphic{8cm}{tals/trig3/img/py_system.png}}
\TRAINER{\bbwCenterGraphic{8cm}{tals/trig3/img/arcsin.png}}


$$\mathbb{D} = \LoesungsRaum{\mathbb{R}}$$

Geben Sie den Wertebereich der Arcus-Sinus-Funktion einmal im Bogenmaß,
einmal im Gradmaß an:
$$\mathbb{W} = \LoesungsRaum{]-\frac{\pi}{2}; \frac{\pi}{2}[}  =
\LoesungsRaum{]-90\degre; 90\degre[}$$

Beachten Sie, dass im Wertebereich $\mathbb{W}$ die Grenzen ($-90\degre$
und $+90\degre$) nicht enthalten sind. Warum ist das so?
\newpage


\subsubsection{Für Applikationsentwickler\LoesungsRaum{\,($\mathrm{atan2}()$)}}

Wir betrachten ein kleines Computerspiel, bei dem von einer
Abschussrampe im Punkt $O=(0|0)$ in einem Winkel $\varphi$ ein Geschoss
abgeschossen wird (siehe folgende beiden Graphiken). Die Längen sind
in Einheiten von \texttt{px}\footnote{Ein \texttt{px} ist ein
  sog. \textit{picture element} und ist ursprünglich die kleinste adressierbare
  Einheit auf dem Computerdisplay.} gegeben.

\begin{tabular}{p{11cm}c}
\textbf{Problem I}: Gegeben ist ein Winkel von $107\degre$ von der
Abschussrampe aus gesehen und gesucht sind die Koordinaten des Punktes
$P=(x_P|y_P)$, der in der Entfernung von 350 [px].
&
\raisebox{-4cm}{\includegraphics[width=5cm]{tals/trig3/img/atan2PunktGesucht.jpg}}\\
\end{tabular}

Lösung: $P=( \LoesungsRaumLang{350\cdot{}\cos(107\degre)\approx{}-102} | \LoesungsRaumLang{350\cdot{}\sin(107\degre)\approx{}335})$


\begin{tabular}{p{11cm}c}
\textbf{Problem II}: Gegeben ist neben dem Ursprung $O = (0 | 0)$ ein
weiterer Punkt $P(x_P|y_P)$. In welchem Winkel $\varphi$ befindet sich
$P$ von $O$ aus gesehen? Gesucht ist der Winkel $\varphi$ im
mathematisch positiven Sinne.

&
\raisebox{-4cm}{\includegraphics[width=5cm]{tals/trig3/img/atan2WinkelGesucht.jpg}}\\
\end{tabular}

Beispiele und Lösung:

\begin{tabular}{l|l|l}
  Punkt          & Winkel                  & allgemeine Formel? \\ \hline
  $P(+10 |  60)$ & $\varphi = \LoesungsRaum{80.5\degre}$  & $\varphi=\LoesungsRaum{\arctan(\frac{y}{x})}$ \\ \hline
  $P(-10 |  60)$ & $\varphi = \LoesungsRaum{99.46\degre}$ & $\varphi=\LoesungsRaum{90\degre - \arctan(\frac{y}{x})}$ \\ \hline
  $P(  0 |  70)$ & $\varphi = \LoesungsRaum{90\degre}$    & $\varphi=$\LoesungsRaum{Sonderfall, sonst Division durch 0} \\ \hline
  $P(-20 |   0)$ & $\varphi = \LoesungsRaum{180\degre}$   & $\varphi=\LoesungsRaum{\arctan(\frac{y}{x}) + 180\degre}$ \\ \hline
  $P(x_P | y_P)$ & allgemeiner Fall?      & $\varphi=\LoesungsRaum{\textrm{arctan2}(y, x)}$\\ \hline
\end{tabular}



