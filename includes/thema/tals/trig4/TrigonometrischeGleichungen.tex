%% Trigonometrie IV
%% Trigonometrische Gleichungen
%% 2020 - 12 - 21 φ@bbw.ch

\section{Trigonometrische Gleichungen}\index{trigonometrische Gleichungen}\index{Gleichungen!trigonometrische}


\subsection{Trigonometrische Beziehungen}
Es gelten diverse Beziehungen zwischen $\sin$, $\cos$ und $\tan$
.
\subsubsection{$\sin$ und $\cos$}
Eine Beziehung kennen wir bereits von der verallgemeinerten
Sinus-Funktion:

\begin{bemerkung}{Sinus und Cosinus}{}
  Cosinus ist ein verschobener Sinus.
  $$\sin\left(\varphi+\frac{\pi}{2}\right) = \sin(\varphi+90\degre)= \cos(\varphi)$$
\end{bemerkung}

\subsubsection{Periodizität}
Auch aus den Funktionsgraphen von $\sin$, $\cos$ und $\tan$ kennen wir die
folgenden Beziehungen für alle ganzen Zahlen $z\in\mathbb{Z}$:
\begin{bemerkung}{Periodizität}{}


  Für alle Winkel $\varphi$ gilt:
  $$\sin(\varphi) = \sin(\varphi + z\cdot2\pi) = \sin(\varphi + z\cdot360\degre)$$
  $$\cos(\varphi) = \cos(\varphi + z\cdot2\pi) = \sin(\varphi + z\cdot360\degre)$$
  Für alle Winkel $\varphi$ mit $\cos(\varphi)\ne 0$ gilt:
  $$\tan(\varphi) = \tan(\varphi + z\cdot\pi) =\tan(\varphi + z\cdot{}180\degre)$$
  \end{bemerkung}
\newpage

\subsubsection{Tangens}
Eine weitere wichtige Beziehung folgt direkt aus den Definitionen:

\begin{gesetz}{Tangens}{}
  Für alle Winkel $\varphi$, deren $\cos$ nicht Null ergibt, gilt:

  $$\tan(\varphi) = \frac{\sin(\varphi)}{\cos(\varphi)}$$
  \end{gesetz}
Zur Herleitung dient einfach die Definition des Tangens mit
anschließendem Erweitern:

$$\tan(\varphi) = \frac{\textrm{Gegenkathete von
  }\varphi}{\textrm{Ankathete von }\varphi} =
\frac{\frac{\textrm{Gegenkathete von }
    \varphi}{\textrm{Hypotenuse}}}{\frac{\textrm{Ankathete von }\varphi}{\textrm{Hypotenuse}}}
=\frac{\sin(\varphi)}{\cos(\varphi)}$$

\newpage


\subsubsection{$\sin^2 + \cos^2$}
Betrachten wir ein rechtwinkliges Dreieck mit dem Winkel $\varphi$
gegenüber der Kathete $g$. $a$ sei die Ankathete und $h$ die
Hypotenuse:

\bbwCenterGraphic{6cm}{tals/trig4/img/pythagoras.png}

\TNT{10.4}{
  $\frac{g}{h} = \sin(\varphi)$ und $\frac{a}{h} = \cos{\varphi}$.

  Daraus folgt:

  $$g= h\cdot{}\sin(\varphi) \textrm{ und } a = h\cdot{} \cos(\varphi)$$
  Ebenso gilt nach Pythagoras:

  $$g^2 + a^2 = h^2$$

  Duch Einsetzen von $g$ und $h$ ergibt sich:

  $$(h\cdot{}\sin(\varphi))^2 + (h\cdot{} \cos(\varphi))^2 = h^2$$

  somit

  $$h^2 \cdot{} \sin(\varphi) \cdot{} \sin(\varphi) + h^2 \cdot{} \cos(\varphi)\cdot{}\cos(\varphi) = h^2$$
  teilen durch $h^2$ ergibt:
  
}%% END TNT
\newpage



\begin{gesetz}{}{}
  $$\sin(\varphi)\cdot \sin(\varphi) + \cos(\varphi) \cdot
  \cos(\varphi) = 1$$
\end{gesetz}
Oder am Einheitskreis:
\TNT{4.4}{\bbwCenterGraphic{8cm}{tals/trig4/img/ssccEinheitskreis.png}}%% END TNT

\begin{bemerkung}{}{}
  Es sind die folgenden Notationen üblich:
  $$\sin^2(\varphi) = \sin(\varphi)\cdot\sin(\varphi)$$
  $$\cos^2(\varphi) = \cos(\varphi)\cdot\cos(\varphi)$$
\end{bemerkung}
Daher merken wir uns:
\begin{bemerkung}{}{}
  $\sin^2+\cos^2 = 1$
\end{bemerkung}

\begin{bemerkung}{}{}

  \begin{tabular}{p{6cm}p{2cm}p{6cm}}
    \fbox{$\sin^2(\varphi) = \sin(\varphi)\cdot\sin(\varphi)$} & Aber: & \fbox{$\sin(\varphi^2) = \sin(\varphi\cdot\varphi)$}\\
    \end{tabular}
  
\end{bemerkung}
\newpage


\subsubsection{Weitere trigonometrische Beziehungen}
Des weiteren gelten die folgenden Beziehungen:
\begin{gesetz}{Sinus}{}
  Sinus ist ``ungerade''.\index{ungerade Funktion!Sinus}
  $$sin(\varphi) = - sin(-\varphi)$$
  \end{gesetz}

\begin{gesetz}{Cosinus}{}
  Cosinus ist ``gerade''.\index{gerade Funktion!Cosinus}
  $$cos(\varphi) = cos(-\varphi)$$
  \end{gesetz}

\newpage

\subsection*{Aufgaben}
\aufgabenfarbe{Strukturaufgaben S. 3 Aufg 3 $k$-$x$}
\TALSGeomAadB{124}{183. a) b) c) und d), 185. a) und mit TR (solve)
  185. b), 188. a) b) c) d) und e)}
\TALSAadB{234ff}{885.ff}
