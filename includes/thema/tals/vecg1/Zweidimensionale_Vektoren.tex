%% 2020 - 12 - 25 ph. g. freimann @bbw.ch
%% Vektoren im Zweidimensionalen

\section{Zweidimensionale Vektoren}\index{Vektoren!zweidimensionale}\index{zweidimensionale Vektoren}
\sectuntertitel{Kommt ein Nullvektor zum Psychiater: ``... ich bin so orientierungslos!''}

\theorieTALSGeom{175}{4}

\subsection*{Lernziele}
\begin{itemize}
\item Komponenten / Zerlegung
\item elementare Operationen
\item Addition, Subtraktion (Gegenvektor)
\end{itemize}
\newpage

\subsection{Einführung}

Betrachten Sie die beiden folgenden Vektoren ${\color{blue} \vec{a}}$ und
${\color{red}\vec{b}}$, welche beide durch mehrere Repräsentanten eingezeichnet sind:

\bbwGraph{-4}{7}{-3}{3}{
\bbwLetter{3.5,3}{\vec{a}}{blue}
\draw [->,blue] (1,1) -- (4,2);
\draw [->,blue] (2,2) -- (5,3);
\draw [->,blue] (-3.5,-1) -- (-0.5,0);
\draw [->,blue] (2,0.5) -- (5,1.5);
\bbwLetter{-1,3}{\vec{b}}{red}
\draw [->,red] (-1,1) --(-2,3);
\draw [->,red] (-1,-2) --(-2,0);
\draw [->,red] (2.5,-0.5) --(1.5,1.5);
\draw [->,red] (5,-3) --(4,-1);

}%% END bbwGraph

Tragen Sie die fehlenden Werte in die Tabelle ein\footnote{Der
  mathematisch positive Winkel wird ab der $x$-Achse im
  Gegenuhrzeigersinn gemessen.}:

\begin{tabular}{|c|c|c|}\hline
                 & ${\color{blue}\vec{a}}$   & ${\color{red}\vec{b}}$   \\\hline
  $x$-Komponente & \LoesungsRaumLang{3}      & \LoesungsRaumLang{-1}    \\\hline
  $y$-Komponente & \TRAINER{1}               & \TRAINER{2}              \\\hline
  Länge          & \TRAINER{$\sqrt{10}$}     & \TRAINER{$\sqrt{5}$}     \\\hline
  math. pos. Winkel  & \TRAINER{$\arctan{}\left(\frac13\right)\approx
    18.43\degre$} & \TRAINER{$90\degre +
    \arctan{}\left(\frac12\right)\approx 116.6\degre$}               \\\hline
\end{tabular}
\platzFuerBerechnungen{3.2}
\newpage
