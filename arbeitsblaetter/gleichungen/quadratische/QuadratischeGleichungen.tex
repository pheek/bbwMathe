%
%% Meta: TI nSpire Einführung
%%       Ziel: Damit die Grundoperationen damit durchgeführt werden können.
%%             Damit man sich an den Rechner gewöhnt.
%%

\input{bmsLayoutPage}
\renewcommand{\bbwAufgabenBlockID}{GlQuad}

\ifisNURAUFGABEN
\newcommand{\LoesungsBlock}[1]{\TRAINER{#1%%
\vspace{1mm}
\hrule}}%% end new Command "LoesungsBlock"
\else
\newcommand{\LoesungsBlock}[1]{\noTRAINER{\TNTeop{}}\TRAINER{#1%%
\vspace{1mm}
\hrule}}%% end new Command "LoesungsBlock"
\fi
%%%%%%%%%%%%%%%%%%%%%%%%%%%%%%%%%%%%%%%%%%%%%%%%%%%%%%%%%%%%%%%%%%

\renewcommand{\metaHeaderLine}{Arbeitsblatt Quadratische Gleichungen}
\renewcommand{\arbeitsblattTitel}{Quadratische Gleichungen: Aufgaben}

\newcommand{\TNTeopS}[1]{\TRAINER{#1}\noTRAINER{\TNTeop{}}}

\begin{document}%%
\arbeitsblattHeader{}

\begin{center}\textit{\tiny{V 0.99 --- 3. März 2025}}\end{center}

\tableofcontents{}

\newpage

\section{Lösungsformel (von a bis z)}
Lösen Sie diese Aufgaben der Reihe nach. Wenn Sie eine Aufgabe nicht lösen können, so versuchen Sie, die Aufgabe auf eine bekannte vorangehende zurückzuführen.


%%\nextBbwAufgabenNummer{}

\begin{bbwAufgabenBlock}
\item $x^3=8$
\LoesungsBlock{$\lx = \{2\}$}

\item $x^2 = 9$
\LoesungsBlock{$\lx = \{-3; 3  \}$}

\item $x^2-25 = 0$
\LoesungsBlock{$\lx=\{-5; 5$\} }

\item $x^2 = 12$
\LoesungsBlock{$\lx=\{-\sqrt{12}; \sqrt{12}\} = \{-2\sqrt{3}; 2\sqrt{3}\} \approx \{-3.464; +3.464\}$}

\item $x^2 - 6 = 0$
\LoesungsBlock{$\lx=\{\-\sqrt{6};\sqrt{6}  \}$}

\item $x^2=-3$
\LoesungsBlock{$\lx=\{ \}$}

\item $2x^2-32 = 0$
\LoesungsBlock{$\lx = \{-4; 4\}$}


\item $|x-3| = 5$
\LoesungsBlock{$\lx=\left\{ -2; 8\right\}$}


\item $(x-3)^2 = 25 $
\LoesungsBlock{$\lx = \left\{ -2; 8 \right\}$}

\item $x^2-6x+9 = 25$
\LoesungsBlock{$\lx = \left\{ -2; 8 \right\}$}

\item $(x-6)^2 = -25 $
\LoesungsBlock{$\lx = \left\{ \right\}$}

\item $(x+6)^2 = 49$
\LoesungsBlock{$\lx = \left\{-13; 1 \right\}$}

\item $x^2+12x+36 = 9 $
\LoesungsBlock{$\lx = \left\{-9; 3 \right\}$}

\item $x^2-10x+25 = 0  $
\LoesungsBlock{$\lx = \left\{ -5 \right\}$}

\item $x^2 + 4x +4 = 36$
\LoesungsBlock{$\lx = \left\{ -8; 4 \right\}$}

\item $x^2+4x+5 = 37$
\LoesungsBlock{$\lx = \left\{ -8; 4\right\}$}

\item $x^2 -12 x + 36 = 7$
\LoesungsBlock{$\lx = \left\{6+\sqrt{7}; 6-\sqrt{7} \right\}$}

\item $ x^2 - 8x + 16= -5$
\LoesungsBlock{$\lx = \left\{ \right\}$}

\item $ x^2 -8x + 18 = 1$
\LoesungsBlock{$\lx = \left\{ \right\}$}

\item $x^2-8x+20 = 50  $
\LoesungsBlock{$\lx = \left\{4-\sqrt{46} ; 4+ \sqrt{46} \right\}$}

\item $ 2x^2 -16x + 40 = 100$
\LoesungsBlock{$\lx = \left\{4-\sqrt{46} ; 4+ \sqrt{46} \right\}$}

\item $ 2x^2 - 20x + 50 = \frac{49}{8} $
\LoesungsBlock{$\lx = \left\{\frac{13}{4} ; \frac{27}{4} \right\}$}

\item $ 2x^2 - 16 x= \frac{-207}{8}$
\LoesungsBlock{$\lx = \left\{\frac{9}4 ; \frac{23}4 \right\}$}

\item $ x^2 + px + \frac{p^2}{4}= q$
\LoesungsBlock{$\lx = \left\{-\frac{p}{2} - \sqrt{q};-\frac{p}{2} + \sqrt{q} \right\}$}

\item $ x^2 + px =  q$
\LoesungsBlock{$\lx = \left\{ \frac12\left( -p-\sqrt{4q+p^2}\right); \frac12\left( -p+\sqrt{4q+p^2}\right) \right\}$}

\item $ ax^2+bx+c=0 $
\LoesungsBlock{$\lx = \left\{ \frac{-b-\sqrt{b^2-4ac}}{2a} ;  \frac{-b+\sqrt{b^2-4ac}}{2a}\right\}$}


\end{bbwAufgabenBlock}
\newpage

%%%%%%%%%%%%%%%%%%%%%%%%%%%%%%%%%%%%%%%%%%%%%%%%%%%%%%%%%%%%%%5

\section{Lösungsmengsformel}
Lösen Sie mit der $abc$-Formel (Mitternachtsformel):

$$ax^2+bx+c \Longleftrightarrow x_{1,2}=\frac{-b\pm\sqrt{b^2-4ac}}{2a}$$

\begin{bbwAufgabenBlock}
\item $2x^2+2x -4 = 0$
\LoesungsBlock{$\lx=\left\{ -2; 1 \right\}$}

\item $2x^2 - x - 6 = 0$
\LoesungsBlock{$\lx=\left\{ -1; \frac32 \right\}$}

\item $2x^2-3x-7 = 0$
\LoesungsBlock{$\lx=\left\{\frac34 - \frac{\sqrt{65}}4 ; \frac34 + \frac{\sqrt{65}}4 \right\}$}

\item $4x^2+\frac43 x + \frac1{12} =0 $
\LoesungsBlock{$\lx=\left\{\frac{-1}4 ;\frac{-1}{12} \right\}$}

\item $ 2x^2 -20x + 50 = 0 $
\LoesungsBlock{$\lx=\left\{ 5 \right\}$}

\item $2x^2 -8x +9 =0 $
\LoesungsBlock{$\lx=\left\{  \right\}$}

\item $ x^2-7x= 0$
\LoesungsBlock{$\lx=\left\{0 ;7 \right\}$}

\item $ x^2-25 =0 $
\LoesungsBlock{$\lx=\left\{-5 ; 5\right\}$}

\item $11x=2x^2+15 $
\LoesungsBlock{$\lx=\left\{ \frac52; 3 \right\}$}


\end{bbwAufgabenBlock}
\newpage
%%%%%%%%%%%%%%%%%%%%%%%%%%%%%%%%%%%%%%%%%%%%%%%%%%%%%%%%%%%%%%%%%%%%5
\section{Erst in Grundform}
Bringen Sie die folgenden Aufgaben erst in die Grundform und lösen Sie anschließend mit der Lösungsformel.

$$a\cdot{}x^2 + b\cdot{}x + c = 0$$

\begin{bbwAufgabenBlock}
\item $ x^2+x= 6$
\LoesungsBlock{$\lx=\left\{ -3; 2   \right\}$}

\item $4x^2-x =12x-3 $
\LoesungsBlock{$\lx=\left\{ \frac14; 3   \right\}$}

\item $ 2x^2+2x+5 = 3-5x-x^2 $
\LoesungsBlock{$\lx=\left\{ -2 ; \frac{-1}3    \right\}$}

\item $ 8x + 5 = -3x^2 $
\LoesungsBlock{$\lx=\left\{  \frac{-5}3 ; -1  \right\}$}

\item $x^2 + 70 = -3x$
\LoesungsBlock{$\lx=\left\{    \right\}$}

\item $ x^2+3= 13x-3x^2$
\LoesungsBlock{$\lx=\left\{\frac14 ; 3    \right\}$}

\end{bbwAufgabenBlock}
\newpage
%%%%%%%%%%%%%%%%%%%%%%%%%%%%%%%%%%%%%%%%%%%%%%%%%%%%%%%%%%%%%%%%%%%%5
\section{$a$, $b$ und $c$ finden}
Bestimmen Sie die Parameter $a$, $b$ und $c$ der folgenden quadratischen Gleichungen für die Grundform $ax^2 +bx +c = 0$. Eine Lösung der Gleichung wird nicht verlangt.
\begin{bbwAufgabenBlock}
\item $5x^2+3x+7 = 0$
\LoesungsBlock{$a= 5;   b= 3;   c= 7$}

\item $5x^2-3x+7 = 0$
\LoesungsBlock{$a=5 ;   b=-3 ;   c= 7$}

\item $5x^2 + 3x = $
\LoesungsBlock{$a= 5;   b= 3;   c= 0$}

\item $ x^2+3x-4= 0$
\LoesungsBlock{$a=1 ;   b=3 ;   c=-4 $}

\item $3x^2 =16 $
\LoesungsBlock{$a= 3;   b= 0;   c=-16 $}

\item $3\mu x = 5z - rx^2 $
\LoesungsBlock{$a= r;   b=3\mu ;   c= -5z$}

\item $ x^2+x+2x= 0$
\LoesungsBlock{$a= 1;   b= 3;   c= 0$}

\item $ax^2 + c = bx $
\LoesungsBlock{$A= a;   B= -b;   C= c$}

\item $bx^2 -axc = -b^2$
\LoesungsBlock{$A= b;   B= -ac;   C= b^2$}

\item $ux^2 -x+w = vx^2 $
\LoesungsBlock{$a= u-v;   b= -1;   c= w$}

\item $\frac12\cdot{}x^2 -2x+3 = \frac12 x + b$
\LoesungsBlock{$A= \frac12;   B=\frac{-5}2 ;   C= 3-b$}

\item $rx^2-x+5 = -t^2x +r$
\LoesungsBlock{$a= r;   b= t^2-1;   c= s-r$}

\item $ 3-7x+(2-5x^2) = 3\cdot{}(5-x)(2x+3k) $
\LoesungsBlock{$a= 1;   b= 9k-37 ;   c= 5-45k$}

\item $ \sqrt2x^2 - \frac13x + \pi = \mu x^2 - \log(5)\cdot{}x - 0.111 $
\LoesungsBlock{$a=\sqrt{2} - \mu ;   b= \log(5) -\frac13;   c= \pi   + 0.111$}

\end{bbwAufgabenBlock}
\newpage
%%%%%%%%%%%%%%%%%%%%%%%%%%%%%%%%%%%%%%%%%%%%%%%%%%%%%%%%%%%%%%%%%%%%5
\section{Nicht nur einfache Zahlen}
Die folgenden Aufgaben beinhalten kompliziertere Zahlen. Auch hier
soll die Gleichung zuerst in die Grundform gebracht werden und danach
mit der $abc$-Formel gelöst werden.

\begin{bbwAufgabenBlock}
\item $ x^2 - 2\sqrt{2}x = -2$
\LoesungsBlock{$\lx= \left\{ \sqrt{2}  \right\}$}

\item $ x^2 = -\pi x $
\LoesungsBlock{$\lx= \left\{ 0; \pi  \right\}$}

Challenges...

Die folgenden Aufgaben können allenfalls mit Faktorisieren rascher
gelöst werden. Sie dienen jedoch der Übung der $abc$-Formel.

\item $ x^2 + 2x = 2\pi + \pi x$
\LoesungsBlock{$\lx= \left\{-2 ; \pi  \right\}$}

\item $ 2x^2 +\sqrt{3} x= \pi \sqrt{3} + 2\pi x$
\LoesungsBlock{$\lx= \left\{ \frac{-\sqrt{3}}2 ; \pi \right\}$}

\item $ 2x^2 -2\sqrt{3} x = 2\sqrt{3}- 2x $
\LoesungsBlock{$\lx= \left\{-1; \sqrt{3} \right\}$}

\end{bbwAufgabenBlock}
\newpage
%%%%%%%%%%%%%%%%%%%%%%%%%%%%%%%%%%%%%%%%%%%%%%%%%%%%%%%%%%%%%%%%%%%%5
\section{Spezialfälle}
Die Lösungsformel ist nicht immer die beste Wahl.

\begin{bbwAufgabenBlock}
\item $x^2 = 0.04$
\LoesungsBlock{$\lx= \left\{ -0.2; 0.2 \right\}$}

\item $ x^2 + 4= 0$
\LoesungsBlock{$\lx= \left\{ \right\}$}

\item $ 5x^2= 18$
\LoesungsBlock{$\lx= \left\{ -\sqrt{\frac{18}5}  ; \sqrt{\frac{18}5}\right\}$}

\item $ 7x^2= \sqrt{2} x$
\LoesungsBlock{$\lx= \left\{ 0 ; \frac{\sqrt{2}}7 \right\}$}

\item $ (x+2)(x-8) = (x+4)(x-3)$
\LoesungsBlock{$\lx= \left\{ \frac{-4}7  \right\}$}

\item $ (x-5) (x+6) (x-8)= 0$
\LoesungsBlock{$\lx= \left\{ -6 ; 5 ; 8 \right\}$}

\item $ (x-\frac14)(x+\pi)= 0$
\LoesungsBlock{$\lx= \left\{ -\pi ; \frac14 \right\}$}

\item $ (x-3.5)(x^2-9)= 0$
\LoesungsBlock{$\lx= \left\{-3 ; 3; 3.5 \right\}$}

\item $ x^3 -14x^2 + 49x=  0$
\LoesungsBlock{$\lx= \left\{0  ; 7 \right\}$}

\item $ (x-2)^2= \pi$
\LoesungsBlock{$\lx= \left\{ 2-\sqrt{\pi}; 2+\sqrt{\pi} \right\}$}

\item $x^2 + 6x + 9 = 19$
\LoesungsBlock{$\lx= \left\{-3 -\sqrt{19} ; -3+\sqrt{19} \right\}$}

\item $ (7+x)(7-x)= (3x+2)^2 - (2x + 3)^2$
\LoesungsBlock{$\lx= \left\{-3 ; 3 \right\}$}

\end{bbwAufgabenBlock}
\newpage

%%%%%%%%%%%%%%%%%%%%%%%%%%%%%%%%%%%%%%%%%%%%%%%%%%%%%%%%%%%%%%%%%%%%5
\section{Faktorisieren (TALS)}
Lösen Sie durch Faktorisieren (TALS)

\begin{bbwAufgabenBlock}
\item $ x^2 + 3x - 10= 0$
\LoesungsBlock{$\lx= \left\{-5 ; 2 \right\}$}

\item $ x^2 - 6x= 0$
\LoesungsBlock{$\lx= \left\{ 0; 6 \right\}$}

\item $ x^2 + 7x -\pi x= 7x$
\LoesungsBlock{$\lx= \left\{-7 ;\pi  \right\}$}

\item $ \sqrt{6}x^2 - 4\sqrt{3}x + 5\sqrt{2}x - 20= 0$
\LoesungsBlock{$\lx= \left\{ ;  \right\}$}


\end{bbwAufgabenBlock}
\newpage



%%%%%%%%%%%%%%%%%%%%%%%%%%%%%%%%%%%%%%%%%%%%%%%%%%%%%%%%%%%%%%%%%%%%5
\section{Diskriminante}
Bestimmen Sie die Diskriminante und geben Sie damit an, wie viele
Lösungen die GLeichung aufweist. Eine Lösung der Gleichung wird nicht
verlangt.


\begin{bbwAufgabenBlock}
\item $ x^2 + 3x - 10= 0$
\LoesungsBlock{zwei Lösungen}

\item $ x^2 - 12= 0$
\LoesungsBlock{zwei Lösungen}

\item $ x-4= x^2 + 16$
\LoesungsBlock{keine Lösungen}

\item $ x^2 - \sqrt{3} = 5(x-\frac13)$
\LoesungsBlock{zwei Lösungen}

\item $ x^2 - 20x + 50 = 1 - 6x$
\LoesungsBlock{eine Lösung}




\end{bbwAufgabenBlock}
\newpage



%%%%%%%%%%%%%%%%%%%%%%%%%%%%%%%%%%%%%%%%%%%%%%%%%%%%%%%%%%%%%%%%%%%%5
\section{mit Parametern (TALS)}
Lösen Sie nach der Variable $x$ auf:

\begin{bbwAufgabenBlock}
\item $ x^2 - 2sx + s^2 -1 = 0$
\LoesungsBlock{$\lx= \left\{\frac{2s-1}2 ; \frac{2s+1}2 \right\}$}

\item $ x^2 - x= m(m+1)$
\LoesungsBlock{$\lx= \left\{ m; m+1  \right\}$}

\item $ x^2 + rx = 2r^2 $
\LoesungsBlock{$\lx= \left\{ -2r ; r \right\}$}

\item $ ax^2 + axp - pc = cx$
\LoesungsBlock{$\lx= \left\{ -p; \frac{c}{a} \right\}$}


\end{bbwAufgabenBlock}
\newpage
%%%%%%%%%%%%%%%%%%%%%%%%%%%%%%%%%%%%%%%%%%%%%%%%%%%%%%%%%%%%%%%%%%%%5
\section{Parameter finden (TALS)}
Bestimmen Sie den Parameter so, dass die Gleichung genau eine Lösung
(in $x$) aufweist.

\begin{bbwAufgabenBlock}
\item $ x^2 + px + 3= 0$
\LoesungsBlock{$\mathbb{L}_p= \left\{ -\sqrt{12}; \sqrt{12} \right\}$}

\item $ ax^2 - 4x + 5= 0$
\LoesungsBlock{$\mathbb{L}_a= \left\{ \frac45  \right\}$}

\item $bx^2 + bx + 1 =8x $
\LoesungsBlock{$\mathbb{L}_b= \left\{ 4; 20 \right\}$}

\item $ \lambda (x^2+x)=2(2x+1) $
\LoesungsBlock{$\mathbb{L}_\lambda = \left\{   \right\}$}


\end{bbwAufgabenBlock}
\newpage
%%%%%%%%%%%%%%%%%%%%%%%%%%%%%%%%%%%%%%%%%%%%%%%%%%%%%%%%%%%%%%%%%%%%5
\section{Vermischte Aufgaben}

\begin{bbwAufgabenBlock}
\item $ 5xJ^2 + 4x - 1= 0$
\LoesungsBlock{$\lx= \left\{ -1; \frac15 \right\}$}

\item $ \varphi^2 + \varphi = 1$
\LoesungsBlock{$\lx= \left\{ \frac{1-\sqrt5}2 ; \frac{1+\sqrt5}2 \right\}$}

\item $ x^2= 5x$
\LoesungsBlock{$\lx= \left\{ 0 ; 5 \right\}$}

\item $ 5x^2 = -15 - x^2 $
\LoesungsBlock{$\lx= \left\{   \right\}$}

\item $ 7x^2 = 18$
\LoesungsBlock{$\lx= \left\{ -\sqrt{\frac{18}7}; \sqrt{\frac{18}7} \right\}$}


\end{bbwAufgabenBlock}
\newpage

%%%%%%%%%%%%%%%%%%%%%%%%%%%%%%%%%%%%%%%%%%%%%%%%%%%%%%%%%%%%%%%%%%%%5
\section{Taschenrechner (GSEO)}
Lösen Sie mit der $abc$-Formel im Taschenrechner.


\begin{bbwAufgabenBlock}
\item $ 5x^2-17x+3= 0$
\LoesungsBlock{$\lx= \left\{ \frac{17\pm\sqrt{229}}{10}  \right\} \approx
\{3.213; 0.1867\}$}

\item $ 3x^2+\pi x + 2= 0$
\LoesungsBlock{$\lx= \left\{  \right\}$}

\item $ 7x^2 + 2\cdot{}\sqrt{77} + 11x= 0$
\LoesungsBlock{$\lx= \left\{ \frac{\sqrt{77}}7  \right\} \approx 1.25357$}

Bringen Sie die folgenden Gleichungen erst in die Grundform und lösen
Sie danach mit dem Taschenrechner.

\item $ 10x^2 -9.5x = 5.1 - 7x $
\LoesungsBlock{$\lx= \left\{ -0.6; 0.85 \right\}$}

\item $ 3x\cdot{}(7x-11)= 20(1-x)$
\LoesungsBlock{$\lx= \left\{ \frac{-5}7;  \frac43\right\}$}

\item $ 2x\cdot{}(x+\frac12) + 2 = x\cdot{}(x+1) -5 $
\LoesungsBlock{$\lx= \left\{   \right\}$}

\item $ 4x^2 + 2x = \frac27 x - \frac9{49} $
\LoesungsBlock{$\lx= \left\{ \frac{-3}4  \right\}$}


\end{bbwAufgabenBlock}
\newpage

%%%%%%%%%%%%%%%%%%%%%%%%%%%%%%%%%%%%%%%%%%%%%%%%%%%%%%%%%%%%%%%%%%%%5
\section{Fallunterscheidung (TALS)}

Je nach Wahl der Parameter haben die Aufgaben eine andere Anzahl von
Lösungen. Machen Sie jeweils eine vollständige Fallunterscheidung.


\begin{bbwAufgabenBlock}
\item $ x^2 + x= a$
\LoesungsBlock{Keine Lösung, falls $a < \frac{-1}4$. Ansonten ist
$x_{1,2} = \frac{-1\pm\sqrt{1+4a}}2$. Wenn $a=\frac{-1}4$, gibt es
genau eine Lösung.}

\item $ 2s\cdot{}(x^2 -2x) = \frac{-3}2 s $
\LoesungsBlock{$\lx= \left\{ \frac12;  \frac32 \right\}$. Ist $s=0$,
so ist $\lx = \mathbb{R}$.}

\item $ 3x^2 = x-4a$
\LoesungsBlock{$\lx= \left\{ \frac{-1\pm\sqrt{1-48a}}2  \right\}$. Dies
gilt für $a\le \frac{1}{48}$. Falls $a$ jedoch größer als $\frac1{48}$,
so gibt es keine Lösung(en). Bei $a=\frac1{48}$ erhalten wir $x=\frac{-1}2$.}

\item $ px^2 - 5x + 4 = 0 $
\LoesungsBlock{Für $p \ge \frac{-25}{16}$ gilt
$x=\frac{s\pm\sqrt{25-16p}}{2p}$. Die Ausnahme ist noch $p=0$, dann
gilt die Lösungsformel nicht (Division durch Null). Im Fall $p=0$ wird
die Gleichung linear und die Lösung ist $x=\frac45$.}

\item $ x^2 -mn= m^2 +nx$
\LoesungsBlock{$\lx= \left\{ -m; n+m \right\}$. Speizalfall $D=0$,
wenn also $(n+2m)^2 = 0$ dann gibt es nur eine Lösung: $x=\frac{n}2$.}


\end{bbwAufgabenBlock}
\newpage



%%%%%%%%%%%%%%%%%%%%%%%%%%%%%%%%%%%%%%%%%%%%%%%%%%%%%%%%%%%%%%%%%%%%5
\section{Substitution}
Potenzgleichungen höherer Ordnung, die durch Substitution auf
quadratische Gleichungen führen (\zB Biquadratische Gleichungen):


... coming soon ...
\begin{bbwAufgabenBlock}
\item $...$
\LoesungsBlock{$...$}


\end{bbwAufgabenBlock}
\newpage




%%%%%%%%%%%%%%%%%%%%%%%%%%%%%%%%%%%%%%%%%%%%%%%%%%%%%%%%%%%%%%%%%%%%5
\section{Textaufgaben}


\begin{bbwAufgabenBlock}
\item Subtrahiert man vom Produkt zweier aufeinanderfolgender, ganzer
Zahlen 62, so erhält man das 30-Fache der kleineren Zahl.

\LoesungsBlock{Die beiden Zahlen heißen 31 und 32 bzw. (2. Lösung) -2
und -1.}

\item Das Produkt aus dem vierten und dem fünften Teil einer Zahl gibt
fünfhundert. Wie heißt die Zahl?

\LoesungsBlock{Die Zahl heißt 100 (bzw. 2. Lösung: Die Zahl heißt -100)}

\item Das Produkt der beiden kleinsten von sechs aufeinander folgenden
natürlichen Zahlen ist dreimal so groß wie die Summe der vier übrigen
Zahlen. Wie heißt die kleinste Zahl?

\LoesungsBlock{Die kleinste Zahl heißt 14}

\item Das um 100 verminderte Quadrat einer gesuchten Zahl übertrifft
die Zahl 200 um so viel, wie die gesuchte Zahl über 300 liegt.

\LoesungsBlock{Möglich sind $x=0$ oder $x=1$.}

\item Dividiert man die Summe zweier positiver Zahlen durch ihre
Differenz, so erhält man 12.5 \% der größeren Zahl. Berechnen Sie die
größere Zahl, wenn die kleinere 42 ist.

\LoesungsBlock{Die größere Zahl ist 56.}

\item Der Nenner eines Bruches ist um vier größer als der
Zähler. Vermindert man den Zähler um drei und vermehrt man den Nenner
um dieselbe Zahl, so ist der entstehende Bruch nur halb so groß wie
der ursprüngliche. Wie lautet der ursprüngliche Bruch?

\LoesungsBlock{Der urpsüngliche Bruch lautet $\frac8{12}$
(bzw. 2. Lösung: Der ursprüngliche Bruch lautet $\frac{-3}{1}$.)}

\item Eine Praxiseinrichtung mit einem Anschaffungswert von CHF
24\,000.- wurde zweimal mit dem gleichen Prozentsatz abgeschrieben und
hat zur Zeit einen Wert von CHF 16\,335.-. Wie viele Prozente beträgt
der Abschreibungssatz?

\LoesungsBlock{Der Prozentsatz der Abschreibung beträgt 17.5\%}

\item Würde man den Umfang eines Rades um 1 m vergrößern, so würde es
sich beim Abrollen auf einer 546 m langen Strecke 25-ma weniger
drehen. Berechnen Sie den ursprünglichen Radumfang.

\LoesungsBlock{Der Radumfang war ursprünglich 4.2 m}

\item Jemand verkauft eine Uhr für CHF 144.- und gewinnt dabei so
viele Prozente, wie die Uhr CHF gekostet hat. Wie viele Prozente sind
es?

\LoesungsBlock{Antwort: Es sind 80 \% und der ursprüngliche Einkaufspreis war somit CHF 80.-.}

\item Eine  Schulklasse fährt mit dem Zug in die Berufsmaturitätswoche.
Die Kosten von CHF 300.- werden gleichmäßig aufgeteilt. Da ein Schüler
krankheitshalber nicht mitfahren kann, ist der Kostenanteil für die
übrigen Teilnehmenden um CHF 0.50 größer. Wie viele sind mitgefahren?

\LoesungsBlock{Effektiv nehmen 24 Personen teil.}

\item  Jemand hat ein Gefäß mit 125 Litern reinen Weins. Er zapft eine
gewisse Quantität ab und ersetzt die abgezapfte Flüssigkeit durch
Wasser.

Nachdem er dies 3-mal wiederholt hat, sind im Gefäß noch 27 Liter des
reinen Weins übrig. Wie viel Liter hat er jedesmal abgezapft?

\LoesungsBlock{Die jeweils abgezapfte Quantität sind 50 Liter.}


\item In einem Rechteck ist eine Seite um 18 m länger als die andere.
Der Flächeninhalt des Rechtecks misst 1440 $\text{m}^2$. Wie lang ist das Rechteck?

\LoesungsBlock{Das Rechteck ist 48 m lang.}

\item (Bhâskara, 12. Jahrhundert) Der achte Teil einer Herde Affen,
ins Quadrat erhoben, hüpfte in einem Haine herum und erfreute sich dem
Spiele; die übrigen 12 sah man auf einem Hügel miteinander
schwatzen. Wie stark war die Herde?

\LoesungsBlock{Die Herde besteht entweder aus 48 oder aus 16 Affen.}


\item Zwei Männer ($A$ und $B$) haben zusammen in zwanzig Tagen eine Mauer
ausgeführt.

Wie lange hätte jeder alleine daran arbeiten müssen, wenn $B$ noch neun Tage
länger als $A$ gebraucht hätte?

\LoesungsBlock{$A$ hätte alleine 36 Tage und $B$ alleine 45 Tage daran gearbeitet.}

\item Um einen Tank zu füllen, braucht Pumpe $A$ eine Stunde länger als
Pumpe $B$. Sind beide Pumpen gleichzeitig in Betrie, dauert die
Füllung sechs Stunden. Wie lange dauert es, wenn nur die Pumpe $B$ zur
Verfügung steht?

\LoesungsBlock{Mit Pumpe $B$ alleine dauert es 11.52 Stunden (=11:31)
bis der Tank voll ist.}

\item Um 1000 Schrauben herzustellen braucht die alte Maschen 9
Minuten länger als die neue. Zusammen benötigen sie 20 Minuten. Wie
lange dauert es, wenn wegen eines Defekts der neuen Maschine nur die
alte eingesetzt werden kann?

\LoesungsBlock{Die alte Maschine alleine braucht 45 Minuten, um 1000
Schrauben herzustellen.}

\item  Zwei Kräfte unterscheiden sich um 5 N. Sie wirken senkrecht
zueinander auf denselben Körper. Die resultierende Kraft misst 8
N. Wie groß sind die beiden Kräfte?

\LoesungsBlock{Die Kräfte messen 2.57 N und 7.57 N.}

\item Die Linsenlgleichung lautet
$$\frac1g + \frac1b = \frac1f$$
Dabei sind $g$, bzw. $b$ die Entfernung des Gegenstands ($g$) bzw. des
Bildes ($b$) von der  Linse. $f$ ist die sog. Brennweite.
Wo entsteht das Bild (Abstand zur Linse), wenn das Bild  ($b$)  64 cm 
näher an der Linse ist als der Gegenstand ($g$) bei einer Brennweite
$f$ von 1.44 cm?

\LoesungsBlock{Das Bild entstehet bei 1.472 cm.}

\end{bbwAufgabenBlock}
\newpage


\end{document}
