%%
%% Meta: TI nSpire Einführung
%%       Ziel: Damit die Grundoperationen damit durchgeführt werden können.
%%             Damit man sich an den Rechner gewöhnt.
%%

\input{bmsLayoutPage}


%%%%%%%%%%%%%%%%%%%%%%%%%%%%%%%%%%%%%%%%%%%%%%%%%%%%%%%%%%%%%%%%%%

\usepackage{amssymb} %% für \blacktriangleright
\renewcommand{\metaHeaderLine}{Arbeitsblatt}
\renewcommand{\arbeitsblattTitel}{Bruchgleichungen (alte GESO Maturaaufgaben)}

\begin{document}%%
\arbeitsblattHeader{}

\renewcommand{\TNTeop}[1]{\noTRAINER{\mmPapier{16}\newpage}\TRAINER{#1}}

%%%%%%%%%%%%%%%%%%%%%%%%%%%%%%%%%%%%%%%%%%%%%%%%%%%%%%%%%%%%%%%%%%%%%%%%%%%%%%%%
(2016 Serie 1 Aufgabe 3) Lösen Sie die Gleichung nach $x$ auf und schreiben Sie das
Resultat möglichst einfach:

$$a(2a-2b) - ax = - \frac{b(x+b)}{2}$$

\TNTeop{$x=2a-b$
}

%%%%%%%%%%%%%%%%%%%%%%%%%%%%%%%%%%%%%%%%%%%%%%%%%%%%%%%%%%%%%%%%%%%%%%%%
(2017 Aufgabe 4) Bestimmen Sie den Definitionsbereich des folgenden Terms bezüglich der Grundmenge $\mathbb{R}$.
$$\frac{5x}{x^2 - 9x - 36}$$

\TNTeop{$\DefinitionsMenge{}_x=\mathbb{R}\backslash\{-3;12\}$}

%%%%%%%%%%%%%%%%%%%%%%%%%%%%%%%%%%%%%%%%%%%%%%%%%%%%%%%%%%%%%%%%%%%%%%%%%%%%%%%%%%
(2017 Aufgabe 8)
Bestimmen Sie den Definitionsbereich und die Lösungsmenge der
folgenden Gleichung:

$$\frac{x+1}{x-3} = \frac{10-2x}{x^2-3x}$$


\TNTeop{$\DefinitionsMenge{} = \mathbb{R}\backslash{}\{0, 3\}; \LoesungsMenge{}=\{-5, 2\}$}
%%%%%%%%%%%%%%%%%%%%%%%%%%%%%%%%%%%%%%%%%%%%%%%%%%%%%%%%%%%%%%%%%%%%%%%%%%%%%%%%%%

(2018 Serie 1)
Bestimmen Sie die Definitionsmenge und die Lösungsmenge der folgenden Gleichung:

$$\frac{6x-24}{3-x} + x - 2 = \frac{6}{x-3}$$

\TNTeop{$ \DefinitionsMenge{} = \mathbb{R}\backslash{}\{3\}, \LoesungsMenge{} = \{8\}$}

%%%%%%%%%%%%%%%%%%%%%%%%%%%%%%%%%%%%%%%%%%%%%%%%%%%%%%%%%%%%%%%%%%%%%%%%%%%%%%%
(2018 Serie 2)
Bestimmen Sie den Definitionsbereich und die Lösungsmenge der Gleichung.
Die Gleichung soll auf die \textbf{Grundform} $ax^2 + bx + c = 0$ gebracht werden und
kann dann mit dem entsprechenden Taschenrechnermodus gelöst werden.
$$\frac{x^2}{x-2} + \frac{4}{2-x} = 3$$
\TNTeop{$\lx=\{1\}$}
%%%%%%%%%%%%%%%%%%%%%%%%%%%%%%%%%%%%%%%%%%%%%%%%%%%%%%%%%%%%%%%%%%%%%%%%%%%%%%%%


%% Das folgende, nun auskommentierte,  war die Einstiegsaufgabe oben, daher raus:
%% Bestimmen Sie den Definitionsbereich und die Lösungsmenge der Gleichung.
%% Die Gleichung ist auf \textbf{Grundform} $ax^2 + bx + c = 0$ zu bringen und
%% kann dann mit dem entsprechenden Taschenrechnermodus gelöst werden.
%% $$\frac{6x-24}{3-x} +x -2 = \frac{6}{x-3}$$
(2018 Serie 3)
Bestimmen Sie den Definitionsbereich und die Lösungsmenge der Gleichung.
Die Gleichung ist auf \textbf{Grundform} $ax^2 + bx + c = 0$ zu bringen und kann dann mit
dem entsprechenden Taschenrechnermodus gelöst werden.
$$\frac{x^2-16x}{x-3} + 1 = \frac{39}{3-x}$$
\TNTeop{$\lx=\{12\}$}
%%%%%%%%%%%%%%%%%%%%%%%%%%%%%%%%%%%%%%%%%%%%%%%%%%%%%%%%%%%%%%%%%%%%%%%%%%%%%
(2018 Serie 4)

Bestimmen Sie den Definitionsbereich und die Lösungsmenge der Gleichung.
Die Gleichung ist auf \textbf{Grundform}
$ax^2 + bx + c = 0$ zu bringen und kann dann
mit dem entsprechenden Taschenrechnermodus gelöst werden.
$$\frac{x^2-10x}{x-4} + 1 = \frac{24}{4-x}$$

\TNTeop{$\lx=\{5\}$}

%%%%%%%%%%%%%%%%%%%%%%%%%%%%%%%%%%%%%%%%%%%%%%%%%%%%%%%%%%%%%%%%%%%%%%%%%%%%%%%%
(2022 Aufgabe 6)

Bestimmen Sie die Definitionsmenge und die Lösungsmenge der Gleichung.

Grundmenge = $\mathbb{R}$


$$\frac5{x-3}   = \frac{7x-1}{x^2-9}   -    \frac4{x+3}$$

\TNTeop{$$\DefinitionsMenge{} = \mathbb{R} \backslash{}  \{-3; 3\}$$
$$\lx = \{ -2 \}$$}%% end TNT       

\end{document}
