%%
%% Meta: TI nSpire Einführung
%%       Ziel: Damit die Grundoperationen damit durchgeführt werden können.
%%             Damit man sich an den Rechner gewöhnt.
%%

\input{bbwLayoutPage}


%%%%%%%%%%%%%%%%%%%%%%%%%%%%%%%%%%%%%%%%%%%%%%%%%%%%%%%%%%%%%%%%%%

\usepackage{amssymb} %% für \blacktriangleright
\renewcommand{\metaHeaderLine}{Arbeitsblatt}
\renewcommand{\arbeitsblattTitel}{Bruchgleichungen (alte GESO Maturaaufgaben)}

\begin{document}%%
\arbeitsblattHeader{}

%%%%%%%%%%%%%%%%%%%%%%%%%%%%%%%%%%%%%%%%%%%%%%%%%%%%%%%%%%%%%%%%%%%%%%%%
Bestimmen Sie den Definitionsbereich des folgenden Terms bezüglich der Grundmenge $\mathbb{R}$.
$$\frac{5x}{x^2 - 9x - 36}$$

\TNT{8}{$\mathcal{D}_x=\mathbb{R}\backslash\{-3;12\}$}
\vspace{2mm}
\hrule

Bestimmen Sie den Definitionsbereich und die Lösungsmenge der Gleichung.
Die Gleichung ist auf \textbf{Grundform}
$ax^2 + bx + c = 0$ zu bringen und kann dann
mit dem entsprechenden Taschenrechnermodus gelöst werden.
$$\frac{x^2-10x}{x-4} + 1 = \frac{24}{4-x}$$

\TNT{8}{$\lx=\{5\}$}

%% Das folgende, nun auskommentierte,  war die Einstiegsaufgabe oben, daher raus:
%% Bestimmen Sie den Definitionsbereich und die Lösungsmenge der Gleichung.
%% Die Gleichung ist auf \textbf{Grundform} $ax^2 + bx + c = 0$ zu bringen und
%% kann dann mit dem entsprechenden Taschenrechnermodus gelöst werden.
%% $$\frac{6x-24}{3-x} +x -2 = \frac{6}{x-3}$$
\noTRAINER{\newpage}

Bestimmen Sie den Definitionsbereich und die Lösungsmenge der Gleichung.
Die Gleichung ist auf \textbf{Grundform} $ax^2 + bx + c = 0$ zu bringen und kann dann mit
dem entsprechenden Taschenrechnermodus gelöst werden.
$$\frac{x^2-16x}{x-3} + 1 = \frac{39}{3-x}$$
\TNT{8}{$\lx=\{12\}$}
\vspace{2mm}

\hrule

Bestimmen Sie den Definitionsbereich und die Lösungsmenge der Gleichung.
Die Gleichung soll auf die \textbf{Grundform} $ax^2 + bx + c = 0$ gebracht werden und
kann dann mit dem entsprechenden Taschenrechnermodus gelöst werden.
$$\frac{x^2}{x-2} + \frac{4}{2-x} = 3$$
\TNT{8}{$\lx=\{1\}$}


\noTRAINER{\newpage}

Bestimmen Sie die Lösung(en) der Gleichung in der Grundmenge $\mathbb{R}$.

$$\frac{x+1}{x-3} = \frac{10-2x}{x^2-3x}$$

\TNT{8}{$\mathcal{D}=\mathbb{R}\backslash{}\{0,3\}; \lx=\{-5, 2\}  $}
\end{document}
