%%
%% Meta: TI nSpire Einführung
%%       Ziel: Damit die Grundoperationen damit durchgeführt werden können.
%%             Damit man sich an den Rechner gewöhnt.
%%

\input{bmsLayoutPage}
\renewcommand{\bbwAufgabenBlockID}{GL\_Br}



%%%%%%%%%%%%%%%%%%%%%%%%%%%%%%%%%%%%%%%%%%%%%%%%%%%%%%%%%%%%%%%%%%

\usepackage{amssymb} %% für \blacktriangleright
\renewcommand{\metaHeaderLine}{Arbeitsblatt}
\renewcommand{\arbeitsblattTitel}{Bruchgleichungen}

\begin{document}%%
\arbeitsblattHeader{}

\begin{center}Version 1.0 (Dez. 2023)\end{center}
%%%%%%%%%%%%%%%%%%%%%%%%%%%%%%%%%%%%%%%%%%%%%%%%%%%%%%%%%%%%%%%%%%
\section{Voraussetzungen / Repetition}

Werten Sie die Terme für die gegebenen $x$-Argumente aus:

\begin{bbwAufgabenBlock}

\item $T(x) = -5.3\cdot{}(x-2.8)$  \\
\vspace{3mm}\\
$T(1)=\LoesungsRaumLang{9.54}$ 
und $T(2.8)=\LoesungsRaumLang{0}$\abplz{4}



\item $T(x) = (x-5)\cdot{}3x\cdot{}(x+18.3)$  \\
\vspace{3mm}\\
$T(0)=\LoesungsRaumLang{0}$ 
und $T(5)=\LoesungsRaumLang{0}$\abplz{4}

\end{bbwAufgabenBlock}

\noTRAINER{\newpage}

Lösen Sie die folgenden Gleichungen nach $x$ auf und geben Sie die
Lösungsmenge $\lx$ an. Tipp: Wann ist ein Produkt gleich Null?

\begin{bbwAufgabenBlock}

\item $(x-4)(x+5)=0$  \hspace{10mm}  $\lx = \LoesungsRaumLang{\{-5; 4\}}$\abplz{4}

\item $7.3x(x+2.84)(x-3)(x+\frac{77}{12}) = 0$  \hspace{10mm}  $\lx = \LoesungsRaumLang{\{-\frac{77}{12};-2.84;0;3\}}$\abplz{4}

\item $x^2 -6x + 9 = 0$ \TRAINER{$\Longleftrightarrow (x-3)^2=0$} \hspace{10mm}  $\lx = \LoesungsRaumLang{\{3\}}$\abplz{4}

\item $x^2 +3x - 28= 0$ \TRAINER{$\Longleftrightarrow (x-4)(x+7)=0$} \hspace{10mm}  $\lx = \LoesungsRaumLang{\{-7; 4\}}$\abplz{4}

\end{bbwAufgabenBlock}

\newpage


%%%%%%%%%%%%%%%%%%%%%%%%%%%%%%%%%%%%%%%%%%%%%%%%%%%%%%%%%%%%%%%%%%
\section{Definitionsbereich}

Füllen Sie die folgende Tabelle für Bruchterme bzw. Bruchgleichungen
aus. Sie müssen die Gleichungen nicht lösen, sondern lediglich jeweils
den Definitionsbreich angeben:

\begin{bbwFillInTabular}{|l|c|l|}\hline
\textbf{\textit{Term/Gleichung}} & \textbf{\textit{Was darf nicht eingesetzt
werden?}}& Definitionsmenge $\DefinitionsMenge{}_x$\\\hline

$\frac{3.8}x$   & $0$   & $\mathbb{R} \backslash \{0\}$\\\hline

$\frac{2}x$     & \TRAINER{$0$}   & \TRAINER{$\mathbb{R} \backslash \{0\}$}\\\hline

$\frac{x}3$     & \TRAINER{alles darf eingesetzt werden}   & \TRAINER{$\mathbb{R}$}\\\hline

$\frac{x+2}{x-1}$     & \TRAINER{$1$} & \TRAINER{$\mathbb{R} \backslash \{1\}$}\\\hline

$\frac{x+3}{2x}$     & \TRAINER{$0$} & \TRAINER{$\mathbb{R} \backslash \{0\}$}\\\hline

$\frac{(x-3)(x+4)}{(x-5)(x+6)}$     & \TRAINER{$-6\text{ und }5$} & \TRAINER{$\mathbb{R} \backslash \{-6;5\}$}\\\hline

$\frac{ax-4}{x+a}$     & \TRAINER{$-a$} & \TRAINER{$\mathbb{R} \backslash \{-a\}$}\\\hline

$\frac1x = \frac{3-x}{2}$     & \TRAINER{$0$} & \TRAINER{$\mathbb{R} \backslash \{0\}$}\\\hline

$\frac1{(x+3)(x-4)}$     & \TRAINER{$-3\text{ und }4$} & \TRAINER{$\mathbb{R} \backslash \{-3; 4\}$}\\\hline

$\frac1{(x-3)(x-3)}$     & \TRAINER{$3$} & \TRAINER{$\mathbb{R} \backslash \{3\}$}\\\hline

$\frac1{x^2-6x+9}$     & \TRAINER{$3$} & \TRAINER{$\mathbb{R} \backslash \{3\}$}\\\hline

$\frac{5x-3.4}{(x-4)(x+7)}$     & \TRAINER{$-7\text{ und }4$} & \TRAINER{$\mathbb{R} \backslash \{-7; 4\}$}\\\hline

$\frac{5x-3.4}{x^2+3x-28}$     & \TRAINER{$-7\text{ und }4$} & \TRAINER{$\mathbb{R} \backslash \{-7; 4\}$}\\\hline

$\frac{ax+3}{x^2+3x-28} = \frac{5(x-6)}{3x - 5.7}$ & \TRAINER{$-7\text{ und }4\text{ und } 1.9$} & \TRAINER{$\mathbb{R} \backslash \{-7; 1.9; 4\}$}\\\hline


\end{bbwFillInTabular} 

\newpage
%%%%%%%%%%%%%%%%%%%%%%%%%%%%%%%%%%%%%%%%%%%%%%%%%%%%%%%%%%%%%%%%%%
\section{Bruchgleichungen...}
\subsection{... die lineare Gleichungen werden ...}

Bruchgleichungen, die auf lineare Gleichungen führen

Die Lösungsvariable ist $x$, sofern nichts anderes vermekrt wurde.


\begin{bbwAufgabenBlock}
\item $\frac1x = \frac53$     \hspace{10mm}              $\DefinitionsMenge{}= \LoesungsRaumLang{\mathbb{R}\backslash\{0\}}$    $\LoesungsMenge{}_x = \LoesungsRaumLang{\{\frac35\}}$\abplz{6}
\item $\frac1x + \frac1{2x} = \frac1{3x}$ \hspace{10mm}   $\DefinitionsMenge{}= \LoesungsRaumLang{\mathbb{R}\backslash\{0\}}$    $\LoesungsMenge{}_x = \LoesungsRaumLang{\{\}}$\abplz{6}\noTRAINER{\newpage}
\item $\frac1x + \frac2x = \frac3x$  \hspace{10mm}       $\DefinitionsMenge{}= \LoesungsRaumLang{\mathbb{R}\backslash\{0\}}$    $\LoesungsMenge{}_x = \LoesungsRaumLang{\DefinitionsMenge{}=\mathbb{R}\backslash\{0\}}$\abplz{6}
\item $\frac1x + \frac1{x+1} = \frac3x$ \hspace{10mm}    $\DefinitionsMenge{}= \LoesungsRaumLang{\mathbb{R}\backslash\{-1;0\}}$ $\LoesungsMenge{}_x = \LoesungsRaumLang{\{-2\}}$\abplz{6}
\item $\frac1{x-1} + \frac{2x}{x-1} = \frac{4-x}{1-x}$ \hspace{10mm}    $\DefinitionsMenge{}= \LoesungsRaumLang{\mathbb{R}\backslash\{1\}}$ $\LoesungsMenge{}_x = \LoesungsRaumLang{\{-5\}}$\abplz{6}\noTRAINER{\newpage}
\item $\frac1{x-2} + \frac{2}{x-3} = \frac3{3-x} + \frac{4}{2-x}$ \hspace{10mm}    $\DefinitionsMenge{}= \LoesungsRaumLang{\mathbb{R}\backslash\{2;3\}}$ $\LoesungsMenge{}_x = \LoesungsRaumLang{\{\frac{5}{2}=2.5\}}$\abplz{6}
\item $\frac{x+10}{12x} + \frac{3+x}{4x} = 1 + \frac{4-5x}{6x}$ \hspace{10mm}    $\DefinitionsMenge{}= \LoesungsRaumLang{\mathbb{R}\backslash\{0\}}$ $\LoesungsMenge{}_x = \LoesungsRaumLang{\{-5.5\}}$
\end{bbwAufgabenBlock}

\platzFuerBerechnungenBisEndeSeite{}
%%%%%%%%%%%%%%%%%%%%%%%%%%%%%%%%%%%%%%%%%%%%%%%%%%%%%%%%%%%%%%%%%%%%%%%%%%%%%%%%%%%%%%%%%%%%%%%%%%%%%

Bei den folgenden Bruchgleichungen erst faktorisieren und kürzen...

\begin{bbwAufgabenBlock}
\item $\frac7{x^2-9} = \frac4{x^2-6x+9}$ \hspace{10mm}    $\DefinitionsMenge{}= \LoesungsRaumLang{\mathbb{R}\backslash\{-3; 3\}}$ $\LoesungsMenge{}_x = \LoesungsRaumLang{\{11\}}$\abplz{6}
\item $1 + \frac{12x+3}{x^2-9} = \frac{2x-15}{2x-6} + \frac{6}{x+3}$ \hspace{10mm}    $\DefinitionsMenge{}= \LoesungsRaumLang{\mathbb{R}\backslash\{-3; 3\}}$ $\LoesungsMenge{}_x = \LoesungsRaumLang{\{\frac{-23}{7}\}}$\abplz{6}
\item $- \frac{3x+4}{5-x} = \frac{2x^2-13x+27}{x^2-12x+35} + \frac{9-x}{7-x}$ \hspace{10mm}    $\DefinitionsMenge{}= \LoesungsRaumLang{\mathbb{R}\backslash\{5;7\}}$ $\LoesungsMenge{}_x = \LoesungsRaumLang{\{10\}}$\abplz{6}\noTRAINER{\newpage}
\item $\frac5{x-4} - \frac1{x-5} = \frac{9x-1}{x^2-9x+20}$ \hspace{10mm}    $\DefinitionsMenge{}= \LoesungsRaumLang{\mathbb{R}\backslash\{4;5\}}$ $\LoesungsMenge{}_x = \LoesungsRaumLang{\{-4\}}$
\end{bbwAufgabenBlock}



\platzFuerBerechnungenBisEndeSeite{}





%%%%%%%%%%%%%%%%%%%%%%%%%%%%%%%%%%%%%%%%%%%%%%%%%%%%%%%%%%%%%%%%%%%%%%%%


Bruchgleichungen mit Parametern. Wenn nichts vermerkt, ist die
Lösungsvariable $x$.

\begin{bbwAufgabenBlock}
\item $\frac{a-1}x = \frac{c+2}d$ \hspace{10mm}    $\DefinitionsMenge{}= \LoesungsRaumLang{\mathbb{R}\backslash\{0\}}$ $\LoesungsMenge{}_x = \LoesungsRaumLang{\{\frac{d(a-1)}{c+2}\}}$\abplz{6}
\item $\frac{a}x + \frac{b}c = \frac{d}e$ \hspace{10mm}    $\DefinitionsMenge{}= \LoesungsRaumLang{\mathbb{R}\backslash\{0\}}$ $\LoesungsMenge{}_x = \LoesungsRaumLang{\{\frac{ace}{dc-eb}\}}$\abplz{6}
\item $\frac{1+x}{1-x} = a$ \hspace{10mm}    $\DefinitionsMenge{}= \LoesungsRaumLang{\mathbb{R}\backslash\{1\}}$ $\LoesungsMenge{}_x = \LoesungsRaumLang{\{\frac{a-1}{a+1}\}}$\abplz{6}\noTRAINER{\newpage}
\item Lösen Sie die Gleichung nach dem Widerstand $R_1$ auf:
      $$R_T = \frac1{\frac1{R_1} + \frac1{R_2}}$$    $R_1 = \LoesungsRaumLang{\frac1{\frac1{R_T}-\frac1{R_2}}}$
\end{bbwAufgabenBlock}


\platzFuerBerechnungenBisEndeSeite{}

%%%%%%%%%%%%%%%%%%%%%%%%%%%%%%%%%%%%%%%%%%%%%%%%%%%%%%%%%%%%%%%%%%%%%%%%%%%5
\subsection{... die quadratische Gleichungen werden}
Die folgenden Bruchgleichungen führen auf quadratische Gleichungen.


\begin{bbwAufgabenBlock}
\item $\frac{12}x = \frac{x}3$ \hspace{10mm}   $\DefinitionsMenge{}= \LoesungsRaumLang{\mathbb{R}\backslash\{0\}}$  $\LoesungsMenge{}_x = \LoesungsRaumLang{\{-6; 6\}}$\abplz{6}
\item $\frac13 + \frac1x + \frac2x = \frac{1+x}3$ \hspace{10mm}   $\DefinitionsMenge{}= \LoesungsRaumLang{\mathbb{R}\backslash\{0\}}$  $\LoesungsMenge{}_x = \LoesungsRaumLang{\{-3; 3\}}$\abplz{6}
\item $\frac{4.5}{3-x} = \frac{3x-x^2}{2x}$ \hspace{10mm}   $\DefinitionsMenge{}= \LoesungsRaumLang{\mathbb{R}\backslash\{0;3\}}$  $\LoesungsMenge{}_x = \LoesungsRaumLang{\{6\}}$\abplz{6}\noTRAINER{\newpage}
\item $\frac{4.5}{3-x} = \frac{x-3}{2}$ \hspace{10mm}   $\DefinitionsMenge{}= \LoesungsRaumLang{\mathbb{R}\backslash\{3\}}$  $\LoesungsMenge{}_x = \LoesungsRaumLang{\{\}}$\abplz{6}
\item $\frac{5x^2-8x+2}{3x^2-7x-6}=\frac{4x^2+x-16}{3x^2-7x-6}$ \hspace{10mm}   $\DefinitionsMenge{}= \LoesungsRaumLang{\mathbb{R}\backslash\{3;\frac{-2}3\}}$  $\LoesungsMenge{}_x = \LoesungsRaumLang{\{6\}}$\abplz{6}
\item $\frac1{x-2} + \frac3{x+4} = 5$ \hspace{10mm}         $\DefinitionsMenge{}= \LoesungsRaumLang{\mathbb{R}\backslash\{-4;2\}}$          $\LoesungsMenge{}_x = \LoesungsRaumLang{\{\frac{-3-\sqrt{199}}{5};\frac{-3+\sqrt{199}}{5} \}}$

\end{bbwAufgabenBlock}

\platzFuerBerechnungenBisEndeSeite{}


%%%%%%%%%%%%%%%%%%%%%%%%%%%%%%%%%%%%%%%%%%%%%%%%%%%%%%%%%%%%%%%%%%%%%%%%%%%5

... mit Parametern ...


\begin{bbwAufgabenBlock}
\item $\frac{x+r}x + \frac{x^2+r^2}{x^2} = 16$ \hspace{10mm}         $\DefinitionsMenge{}= \LoesungsRaumLang{\mathbb{R}\backslash\{0\}}$          $\LoesungsMenge{}_x = \LoesungsRaumLang{\{r\cdot{}\frac{1\pm\sqrt{57}}{28} \}}$\abplz{6}
\item $\frac1{x+2s}+ \frac1{x+s} + \frac1{x}=0$ \hspace{10mm}
$\DefinitionsMenge{}= \LoesungsRaumLang{\mathbb{R}\backslash\{-2s;-s;0\}}$
$\LoesungsMenge{}_x = \LoesungsRaumLang{\{-s\pm s\frac{\sqrt{3}}{3} \}}$

\end{bbwAufgabenBlock}

\platzFuerBerechnungenBisEndeSeite{}


%%%%%%%%%%%%%%%%%%%%%%%%%%%%%%%%%%%%%%%%%%%%%%%%%%%%%%%%%%%%%%%%%%%%%%%%%%%5


Vermischte Aufgaben


\begin{bbwAufgabenBlock}
\item $\frac{x}{x+3} + \frac{3}{x+3} = \frac{10}{x+4}$ \hspace{10mm}
$\DefinitionsMenge{}= \LoesungsRaumLang{\mathbb{R}\backslash\{-4;-3\}}$
$\LoesungsMenge{}_x = \LoesungsRaumLang{\{  6  \}}$
\abplz{6}
\item $\frac{x-4}{x-5}  = \frac{30-x^2}{x^2-5x}$ \hspace{10mm}
$\DefinitionsMenge{}= \LoesungsRaumLang{\mathbb{R}\backslash\{0;5 \}}$
$\LoesungsMenge{}_x = \LoesungsRaumLang{\{  -3  \}}$
\abplz{6}
\item $\frac{7x+1}{x-7}   - \frac{7x-1}{7-x} = 7$ \hspace{10mm}
$\DefinitionsMenge{}= \LoesungsRaumLang{\mathbb{R}\backslash\{7 \}}$
$\LoesungsMenge{}_x = \LoesungsRaumLang{\{  -7  \}}$

\end{bbwAufgabenBlock}

\platzFuerBerechnungenBisEndeSeite{}


%%%%%%%%%%%%%%%%%%%%%%%%%%%%%%%%%%%%%%%%%%%%%%%%%%%%%%%%%%%%%%%%%%%%%%%%%%%5

\end{document}
