%%
%% Meta: TI nSpire Einführung
%%       Ziel: Damit die Grundoperationen damit durchgeführt werden können.
%%             Damit man sich an den Rechner gewöhnt.
%%

\input{bbwLayoutPage}

%%%%%%%%%%%%%%%%%%%%%%%%%%%%%%%%%%%%%%%%%%%%%%%%%%%%%%%%%%%%%%%%%%

\usepackage{amssymb} 
\renewcommand{\metaHeaderLine}{Arbeitsblatt}
\renewcommand{\arbeitsblattTitel}{Einführung zur quadratischen Ergänzung}

\begin{document}%%
\arbeitsblattHeader{}

\textbf{Hinführung zur Lösungsformel $ax^2+bx+c=0$}

Bestimmen Sie die Lösungsmenge jeder Gleichung. Generell kann jede Gleichung zwei Lösungen aufweisen.

Versuchen Sie, bei jeder Gleichung zuerst herauszufinden, was sich gegenüber der vorangehenden Gleichung geändert hat.

\newcounter{nuemmerli}
%\newcommand{\noteSpace}{\noTRAINER{\rule{0pt}{6.5ex}}}
\newcommand{\noteSpace}{\noTRAINER{\mmPapier{4}}}

\newcommand{\quadGleichung}[4]{
\TRAINER{\hrulefill\\}%%
\par
\needspace{4\baselineskip}
\begin{samepage}
\stepcounter{nuemmerli}(\thenuemmerli. )\,\,\,\, #1\\
\noTRAINER{\mmPapier{#2}}\TRAINER{{\color{black}#3}
{\\(Vorgehen: \color{blue}#4})}\\
\end{samepage}

}%% end quadGleichnug

\quadGleichung{$x^3=8$}{4}{$x=2$}{3. Wurzel ziehen}
\quadGleichung{$x^2=9$}{4}{$\lx=\{-3; 3\}$}{Wurzel ziehen, negative Lösung beachten}
\quadGleichung{$x^2-25=0$}{4}{$\lx=\{-5; 5\}$}{Erst 25 auf die andere Seite bringen, dann Wurzel ziehen, negative Lösung beachten}
\quadGleichung{$x^2=12$}{4}{$\lx=\{-2\sqrt{3}; 2\sqrt{3}\}=\{-\sqrt{12};\sqrt{12}\}$}{Erst 25 auf die andere Seite bringen, dann Wurzel ziehen, negative Lösung beachten}
\quadGleichung{$x^2-6=0$}{4}{$\lx=\{-\sqrt{6}; \sqrt{6}\}$}{Erst 6 auf die andere Seite bringen, dann Wurzel ziehen, negative Lösung beachten}
\quadGleichung{$x^2=-3$}{4}{$\lx=\{ \}$}{Kein Quadrat ist negativ in $\mathbb{R}$}
\quadGleichung{$2x^2-32=0$}{4}{$\lx=\{ -4;4\}$}{durch 2 teilen; $x^2$ separieren}
\quadGleichung{$|x-3|=5$}{4}{$\lx=\{ -2;8\}$}{Zwei Fälle: $x-3=5$ und $x-3=-5$}
\quadGleichung{$(x-3)^2=25$}{4}{$\lx=\{ -2;8\}$}{Zwei Fälle: $x-3=5$ und $x-3=-5$}
\quadGleichung{$(x-6)^2=-25$}{4}{$\lx=\{ \}$}{Kein Quadrat ist negativ in $\mathbb{R}$}
\quadGleichung{$x^2+10x+25=0$}{4}{$\lx=\{-5\}$}{binomische Formel anwenden: $(x+5)^2=0$}
\quadGleichung{$x^2-4x+4=0$}{4}{$\lx=\{2\}$}{binomische Formel anwenden: $(x-2)^2=0$}
\quadGleichung{$6x^2-24x=-24$}{4}{$\lx=\{2\}$}{Erst durch 6 teilen: $x^2-4x=-4$, danach Grundform $x^2-4x+4=0$, danach Binom $(x-2)^2=0$}
\quadGleichung{$x^2+10x+25=49$}{4}{$\lx=\{2;12\}$}{Binom: $(x+5)^2=49$; Wurzel ziehen: $x+5=\pm 7$; Minus 5: $x=-5\pm7$}
\quadGleichung{$x^2+12x+36=7$}{4}{$\lx=\{-5-\sqrt7;-5+\sqrt7\}$}{Binom: $(x+6)^2=7$; Wurzel ziehen: $x+5=\pm \sqrt{7}$; Minus 5: $x=-5\pm\sqrt{7}$}
\quadGleichung{$x^2-8x+16=-8$}{4}{$\lx=\{\}$}{Binom: $(x-4)^2=-8$; Kein Quadrat wird negitiv in $\mathbb{R}$}
\quadGleichung{$x^2+12x+35=6$}{4}{$\lx=\{-5-\sqrt7;-5+\sqrt7\}$}{Ergänzen $x^2+12x+36=7$; dann wie zweitletzte Aufgabe: Binom: $(x+6)^2=7$; Wurzel ziehen: $x+5=\pm \sqrt{7}$; Minus 5: $x=-5\pm\sqrt{7}$}


Challenge:

\quadGleichung{$2x^2-16x+32=\frac{49}{8}$}{4}{
$\lx=\{\frac94; \frac{23}4\}$}{
Halbieren: $x^2-8x+16 = \frac{49}{16}$\\
Binom: $(x-4)^2 = \frac{49}{16} | \pm\sqrt{}$\\
Wurzel: $x-4 = \frac74$\\
$\lx = \{4-\frac74; x+\frac74\}$\\
}

Tipp bei der folgenden Aufgabe: Versuchen Sie zunächst die Aufgabe so
umzuformen, dass Sie etwas bekanntes (Aufgabe vorhin) erhalten:

\quadGleichung{$2x^2-16x=\frac{-207}{8}$}{4}{$\lx=\{\frac94; \frac{23}4\}$}{
Halbieren: $x^2-8x=\frac{-207}{16}$\\
Ergänzen: $x^2-8x+16 = \frac{-207}{16}+16 = \frac{49}{16}$\\
Binom: $(x-4)^2 = \frac{49}{16} | \pm\sqrt{}$\\
Wurzel: $x-4 = \frac74$\\
$\lx = \{4-\frac74; x+\frac74\}$\\
}
\end{document}
