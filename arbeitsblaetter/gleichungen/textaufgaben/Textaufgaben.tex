%%
%% Meta: TI nSpire Einführung
%%       Ziel: Damit die Grundoperationen damit durchgeführt werden können.
%%             Damit man sich an den Rechner gewöhnt.
%%

\input{bmsLayoutPage}

%%%%%%%%%%%%%%%%%%%%%%%%%%%%%%%%%%%%%%%%%%%%%%%%%%%%%%%%%%%%%%%%%%

\usepackage{amssymb} 
\renewcommand{\metaHeaderLine}{Theorieblatt}
\renewcommand{\arbeitsblattTitel}{Systematisches Lösen von Textaufgaben}

\begin{document}%%
\arbeitsblattHeader{}
(Beispiel Mathematik Marthaler S. 131 Aufgabe 24)

\textit{«Zieht man von 45 das Dreifache einer gesuchten Zahl ab, so
erhält man die um 10 verminderte Zahl.»}


\begin{enumerate}
\item \textbf{Aufgabe verstehen:} Es hat eine gesuchte Zahl und ein
Dreifaches. Vermindert heißt hier abgezogen (subtrahiert).

\item \textbf{Variable einführen:} $z\in \mathbb{R}$ ist die gesuchte
Zahl.
Hier \textbf{ausnahmsweise ohne} Maßeinheit, denn es handelt sich nur
um eine blanke Zahl. Ansonsten müsste hier cm, kg, m, km, Stunden,
... angegeben werden.

\item \textbf{Terme bilden:}
$3z$: Das Dreifache der gesuchten Zahl

$45-3z$: 45 um das Dreifache vermindert

$z-10$: Die um zehn verminderte Zahl

\item \textbf{Gleichung(en) aufstellen:} Der Text \textit{«... so
erhält man... »} heißt eigentlich «gleich» und wird mit « $=$ » in der
Mathematik gekennzeichnet.
$$45 - 3z = z - 10$$

\item \textbf{Gleichung(en) lösen:}

\begin{tabular}{rcl|l}

 45-3z &=& z-10 & +3z\\

 45 &=& 4z-10 & + 10 \\

 55 &=& 4z  & :4\\

  13.75 &=& z & (fertig)
 \end{tabular} 

\item \textbf{Kontrolle/Probe:}

Zieht man ab: $45-3z = 45-41.25 = 3.75$

Das Verminderte: $z-10 = 13.75 - 10 = 3.75$

Die Probe stimmt.

\item \textbf{Antwort:} Die gesuchte Zahl lautet \textbf{13.75}.

\end{enumerate}

\end{document}
