%%
%% Meta: TI nSpire Einführung
%%       Ziel: Damit die Grundoperationen damit durchgeführt werden können.
%%             Damit man sich an den Rechner gewöhnt.
%%

\input{bmsLayoutPage}
\renewcommand{\bbwAufgabenBlockID}{Gl1Li}

\ifisNURAUFGABEN{
\newcommand{\LoesungsBlock}[1]{\TRAINER{#1
\vspace{1mm}
\hrule}}%% end new Command "LoesungsBlock"
\else
\newcommand{\LoesungsBlock}[1]{\noTRAINER{\TNTeop{}}\TRAINER{#1
\vspace{1mm}
\hrule}}%% end new Command "LoesungsBlock"
\fi
%%%%%%%%%%%%%%%%%%%%%%%%%%%%%%%%%%%%%%%%%%%%%%%%%%%%%%%%%%%%%%%%%%

\usepackage{amssymb} %% für \blacktriangleright
\renewcommand{\metaHeaderLine}{Arbeitsblatt Lineare Gleichungen}
\renewcommand{\arbeitsblattTitel}{Lineare Gleichungen Trainingsaufgaben}

\newcommand{\TNTeopS}[1]{\TRAINER{#1}\noTRAINER{\TNTeop{}}}

\begin{document}%%
\arbeitsblattHeader{}

\begin{center}\textit{\tiny{V 0.99 17. Okt. 2024}}\end{center}

\tableofcontents{}

\newpage

\section{Nach $x$ auflösen}
Lösen Sie nach der Variable $x$ auf:


%%\nextBbwAufgabenNummer{}

\begin{bbwAufgabenBlock}
\item $$5x + 2 = 7x -4$$
\LoesungsBlock{$\lx = \{3\}$}

\item $$3x-(5-6x)=2-x$$
\LoesungsBlock{$\lx=\{0.u7\} = \{\frac7{10}\}$}

\item $$\sqrt{2}\cdot{}x + 3 = \sqrt{7} - x$$
\LoesungsBlock{$\lx=\{\frac{\sqrt{7}-3}{\sqrt{2}+1}\}
= \{\sqrt{14}-\sqrt{7} -3\sqrt{2}+3 \} \approx -0.146735$}

\item $$5-x(x-4) = 5x-x^2+6$$
\LoesungsBlock{$\lx=\{-1\}$}

\item $$\frac15 + \frac{3x(1-x)}{8} = \frac14 - \frac{(3x-3)(x-2)}{8}$$
\LoesungsBlock{$\lx=\{\frac{14}{15}\}$}

\item $$(3x-2)^2 -x(x-3(x-4))  + 5 = 11x^2$$
\LoesungsBlock{$\lx=\{\frac38\}$}

\end{bbwAufgabenBlock}
\newpage

%%%%%%%%%%%%%%%%%%%%%%%%%%%%%%%%%%%%%%%%%%%%%%%%%%%%%%%%%%%%%%5

\section{Lösungsmenge}
Geben Sie die Lösungsmenge $\lx$ an:

\begin{bbwAufgabenBlock}

\item $$5x+3 = 5x-2$$
\LoesungsBlock{$\lx=\{\}$}

\item $$3(x-1)-x = 2x-3$$
\LoesungsBlock{$\lx=\mathbb{R}$}

\item $$2(2x-1)-x+1 = 3(x+1)$$
\LoesungsBlock{$\lx=\{\}$}

\item $$7x+2 = -(18x - 2)$$
\LoesungsBlock{$\lx=\{0\}$}

\item $$\frac{3(x+2)}2 - 1 = \frac32 \cdot{} x + 2$$
\LoesungsBlock{$\lx=\mathbb{R}$}

\item $$5x^2 + 2 = -5(x+2)(-x+2)$$
\LoesungsBlock{$\lx=\{\}$}

\item $$\frac{x}{10} + \frac{75}{77} = \frac{7x+6}{14} - \frac{22x-30}{55}$$
\LoesungsBlock{$\lx=\mathbb{R}$}

\end{bbwAufgabenBlock}
\newpage
%%%%%%%%%%%%%%%%%%%%%%%%%%%%%%%%%%%%%%%%%%%%%%%%%%%%%%%%%%%%%%%%%%%%5
\section{Parameter}
Lösen Sie jeweils nach $x$ auf. (Eine Fallunterscheidung ist nicht verlangt.)
\begin{bbwAufgabenBlock}
\item $$4x-4t = x+ 11t$$
\LoesungsBlock{$\lx=\{5t\}$}

\item $$5x-3a = 3x+2a-4$$
\LoesungsBlock{$\lx=\{\frac{5a}{2} - 2\} = \{\frac{5a-4}{2}\}$}

\item $$a^2x+a = a^3x -a$$
\LoesungsBlock{$\lx=\{\frac{-2}{a(1-a)}\}$}

\item $$ax + 2b = 3cx +4d$$
\LoesungsBlock{$\lx=\{\frac{4d-2b}{a-3c}\}$}

\item $$ax = -x-a-1$$
\LoesungsBlock{$\lx=\{-1\}$}

\item $$\sqrt{7}\left(\sqrt{28}ax+3b\right) = -6-bx\left(1-\frac17a\right)$$
\LoesungsBlock{$\lx=\left\{\frac{-6 -3\sqrt{7}b}{14a+b-\frac{ab}7}\right\}
= \left\{\frac{-42 -21\sqrt{7}b}{98a+7b-ab}\right\}$}

\item $$(x+2s)^2 = 2x(x+3s)-x(x+3)$$
\LoesungsBlock{$\lx=\left\{\frac{4s^2}{2s-3}\right\}$}

\item $$\frac13(a+x) = \frac12 a + \frac{x}5$$
\LoesungsBlock{$\lx=\{\frac54\cdot{}a\}$}

\item $$$$
\LoesungsBlock{$\lx=\{\}$}
\end{bbwAufgabenBlock}
\newpage
\section{Andere Variable als $x$}
Lösen Sie nach anderen Variablen auf:

\begin{bbwAufgabenBlock}

\item Lösen Sie nach $r$ auf:
 $$S= 2G + 2r\pi\cdot{}h$$
\LoesungsBlock{$r = \frac{S-2G}{2\pi h}$}

\item
Lösen Sie nach $p$ auf:
$$K_1 = K_0 + K_0 \cdot{}\frac{p}{100}$$
\LoesungsBlock{$p = \frac{100}{K_0} \cdot{} (K_1 - K_0 )$}

\item Lösen SIe nach $a$ auf:
$$T = \frac13 (ab + ac + bc)$$
\LoesungsBlock{$a = \frac{3T - bc}{b+c}$}

\end{bbwAufgabenBlock}
\newpage
%%%%%%%%%%%%%%%%%%%%%%%%%%%%%%%%%%%%%%%%%%%%%%%%%%%%%%%%%%%%%%%%55
\section{TALS: Mit Fallunterscheidung}
Lösen Sie nach $x$ auf und diskutieren Sie die Sonderfälle, die durch
die gegebenen Parameter entstehen können:
\begin{bbwAufgabenBlock}
\item $$(a-5)x = 0$$
\LoesungsBlock{$\lx=\{0\}$. Soderfall $a=5$, dann: $\lx = \mathbb{R}$.}

\item $$a^2x+a = a^3x - a$$
\LoesungsBlock{$\lx=\{\frac{2a}{a^3-a^2}\}$. Sonderfälle $a=0$, dann:
$\lx = \mathbb{R}$ und $a=1$, dann: $\lx = \{\}$.}

\item $$(s^2 + 3s - 10)x = s-2$$
\LoesungsBlock{$\lx=\left\{\frac{1}{s+5}\right\}$. Sonderfälle $s=2$,
dann: $\lx = \mathbb{R}$ und $s=-5$, dann $\lx = \{\}$.}

\item $$ax+b^2 = a^2 - bx$$
\LoesungsBlock{$\lx=\{a-b\}$. Sonderfall $a=-b$, dann $\lx = \mathbb{R}$.}

\item $$a^2x-b = a+b^2x$$
\LoesungsBlock{$\lx=\left\{\frac1{a-b}\right\}$. Sonderfall $a=b=0$,
dann $\lx = \mathbb{R}$; Sonderfall $a=b\ne 0$, dann $\lx = \{\}$;
Sonderfall $a=-b$, dann $\lx=\mathbb{R}$.}
\end{bbwAufgabenBlock}
\newpage
%%%%%%%%%%%%%%%%%%%%%%%%%%%%%%%%%%%%%%%%%%%%%%%%%%%%%%%%%%%%%%%%%%%%%%
\section{Kontrolle (Testen Sie Ihr Können)}

Aus alten Aufnahme und Abschlussprüfungen

\subsection{Aufnahmeprüfung 2023 Serie D}
Aufgabe 4.

$$2b = \frac{ab+8}{7}$$

a) Lösen Sie die (obige) Gleichung nach der Variable $a$ auf.

\LoesungsBlock{$$a = \frac{14b-8}{b}$$ (Falls $b\ne 0$)}

b) Lösen Sie die (obige) Gleichung nach der Variable $b$ auf.

\LoesungsBlock{$$b = \frac8{14-a}$$ (Falls $a\ne 14$)}

\subsection{Aufnahmeprüfung 2021 Serie A1}
Aufgabe 2. b)

Berechnen Sie die Lösung der Gleichung

$$\frac{21x}{4} - \frac{18(4-x)}{8} = 6x$$

\LoesungsBlock{$x=6$}

\subsection{Abschlussprüfung GESO 2023 Serie 1}
Aufgabe 4

Lösen Sie die Gleichung nach $x$ auf. Notieren Sie das Ergebnis so
einfach wie möglich.

$$ax + 50a = 2a^3 + 5x$$

\LoesungsBlock{$x = 2a(a+5)$}

\subsection{Abschulssprüfung GESO 2022 Serie 1}
Aufgabe 4

Lösen Sie die Gleichung nach $x$ auf. Notieren Sie das Ergebnis so
einfach wie möglich.

$$5ax - c = a(x+1) = 2(c+x)$$

\LoesungsBlock{$x = \frac{a-c}{4a+2}$}

subsection{Abschulssprüfung GESO 2017 Serie 1}
Aufgabe 7

$$7a -x -c = \frac{7b^2 -ac +bx}{a}$$

\LoesungsBlock{$x = 7(a+b)$}

\subsection{Abschulssprüfung TALS GLF 2023 Serie 1}
(GLF = Grundlagenfach)

Aufgabe 2. b)

Berechnen Sie $x$ in Abhängigkeit von $p$:

$$2px + 1 = (p+2)(x+3)$$

\LoesungsBlock{$x = \frac{3p + 5}{p-2}$}

\end{document}
