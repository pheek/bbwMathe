%%
%% Meta: Gleichungssysteme
%%

\input{bmsLayoutPage}
\renewcommand{\bbwAufgabenBlockID}{GLS}

\ifisNURAUFGABEN
\newcommand{\LoesungsBlock}[1]{\TRAINER{Lösung:#1
\vspace{1mm}
\hrule}}%% end new Command "LoesungsBlock"
\else
\newcommand{\LoesungsBlock}[1]{\noTRAINER{\TNTeop{}}\TRAINER{Lösung:\,\,\,\,#1
\vspace{1mm}
\hrule}}%% end new Command "LoesungsBlock"
\fi
%%%%%%%%%%%%%%%%%%%%%%%%%%%%%%%%%%%%%%%%%%%%%%%%%%%%%%%%%%%%%%%%%%

\usepackage{amssymb} %% für \blacktriangleright
\renewcommand{\metaHeaderLine}{Arbeitsblatt Gleichungssysteme}
\renewcommand{\arbeitsblattTitel}{Lineare Gleichungssysteme}

\newcommand{\TNTeopS}[1]{\TRAINER{#1}\noTRAINER{\TNTeop{}}}

\begin{document}%%
\arbeitsblattHeader{}

\begin{center}\textit{\tiny{V 0.0 - 13. Okt. 2025}}\end{center}

\tableofcontents{}

\newpage


\textbf{Vorgehen}


\begin{enumerate}
\item 
Lösen Sie von jedem Kapitel 1-2 Aufgaben. Wenn Sie Mühe haben, lösen
Sie auch die anderen Übungen ansonsten probieren Sie noch die letzte
Übung des Kapitels, bevor Sie zum nächsten Kapitel übergehen.
\end{enumerate}

\newpage

\section{Einsetzverfahren}
\begin{enumerate}[label=\alph*)]
\item

\gleichungZZ{2x+11y}{57}{x-3y}{-14}


\LoesungsBlock{$\mathbb{L}_{(x;y)} = \left\{ (1;5) \right\}  $}

%%%%%%%%%%%%%%%%%

\item
\gleichungZZ{4a+2b}{22}{3a-5b}{36}

\LoesungsBlock{$\mathbb{L}_{(a;b)} = \left\{ (7;-3) \right\}  $}

%%%%%%%%%%%%%%%%%%%%%%%
\item 
\gleichungZZ{s+3t}{4}{5s-32}{21t}

\LoesungsBlock{$\mathbb{L}_{(s;t)} = \left\{ \left(5;\frac{-1}{3}\right) \right\}  $}

%%%%%%%%%%%%%%%%%%%%%%%
\item 
\gleichungZZ{x+y}{-5}{x-y}{10}

\LoesungsBlock{$\mathbb{L}_{(x;y)} = \left\{ \left( 2.5;-7.5 \right) \right\}  $}

%%%%%%%%%%%%%%%%%%%%%%%
\item 
\gleichungZZ{y}{2x-3}{y}{-0.5x+1}

\LoesungsBlock{$\mathbb{L}_{(x;y)} = \left\{ \left( 1.6; 0.2 \right)= \left( \frac85 ; \frac15 \right) \right\}  $}

%%%%%%%%%%%%%%%%%%%%%%%
\item 
\gleichungZZ{y}{8x+4}{y}{-7x+2}

\LoesungsBlock{$\mathbb{L}_{(x;y)} = \left\{ \left( \frac{-2}{15} ; \frac{44}{15} \right) \right\}  $}

%%%%%%%%%%%%%%%%%%%%%%%
\item 
\gleichungZZ{3(2x+y)}{18}{9(4x+2y)+20y}{28}

\LoesungsBlock{$\mathbb{L}_{(x;y)} = \left\{ \left( 5; -4 \right) \right\}  $}

%%%%%%%%%%%%%%%%%%%%%%%
\item 
\gleichungZZ{3+2(y-3)}{-\frac{3(x+5)}{2}}{2x+7}{\frac15\left( 6(2y-1)+11\right)}

\LoesungsBlock{$\mathbb{L}_{(x;y)} = \left\{ \left( -3; 0 \right) \right\}  $}



\end{enumerate}

\newpage
%%%%%%%%%%%%%%%%%%%%%%%%%%%%%%%%%%%%%%%%%%%%%%%%%%%%%%%%%%%%%%%%%%%%%%%%
\section{nicht lineare Systeme}
(Optional bei GESO.)

\begin{enumerate}[label=\alph*)]
%%%%%%%%%%%%%%%%%%%%%%%
\item 
\gleichungZZ{x^3}{y+107}{3x^3+2y}{411}

\LoesungsBlock{$\mathbb{L}_{(x;y)} = \left\{ \left( 5;18 \right) \right\}  $}

%%%%%%%%%%%%%%%%%%%%%%%

\item 
\gleichungZZ{x^5-y}{27}{x^5+4y}{52}

\LoesungsBlock{$\mathbb{L}_{(x;y)} = \left\{ \left( 2;5 \right) \right\}  $}

%%%%%%%%%%%%%%%%%%%%%%%

\item 
\gleichungZZ{a^2+b^2}{20}{a+b}{3}

\LoesungsBlock{$\mathbb{L}_{(a;b)}
= \left\{ \left(\frac{3-\sqrt{31}}2; \frac{3+\sqrt{31}}2  \right), \left( \frac{3+\sqrt{31}}2; \frac{3-\sqrt{31}}2\right) \right\}  $}

%%%%%%%%%%%%%%%%%%%%%%%

\item 
\gleichungZZ{\sqrt{x} + y}{13.5}{x-7y}{103.5}

\LoesungsBlock{$\mathbb{L}_{(x;y)}
= \left\{ \left( 121; 2.5 \right) \right\}  $}

%%%%%%%%%%%%%%%%%%%%%%%%%%%%
\item
Die Summe, das Produkt und der Quotient $\left(\frac{a}{b}\right)$ von
zwei Zahlen $a$ und $b$ sind gleich. Berechnen Sie $a$ und $b$

\LoesungsBlock{$\mathbb{L}_{(a;b)}
= \left\{ \left( \frac12; -1 \right) \right\}  $}

\end{enumerate}

%%%%%%%%%%%%%%%%%%%%%%%%%%%%%%%%%%%%%%%%%%%%%%%%%%%%%%%%%%%%%%%%%%%%%%%%%%%%%
\newpage
%%%%%%%%%%%%%%%%%%%%%%%%%%%%%%%%%%%%%%%%%%%%%%%%%%%%%%%%%%%%%%%%%%%%%%%%
\section{Additions- bzw. Subtraktionsverfahren}

\begin{enumerate}[label=\alph*)]
\item 
\gleichungZZ{7x+6y}{32}{11x-4y}{10}

\LoesungsBlock{$\mathbb{L}_{(x;y)}
= \left\{ \left( 2; 3 \right) \right\}  $}
%%%%%%%%%%%%%%%%%%%%%%
\item 
\gleichungZZ{7x-11y}{-6}{9x+12y}{18}

\LoesungsBlock{$\mathbb{L}_{(x;y)}
= \left\{ \left( \frac{42}{61}; \frac{60}{61} \right) \right\}  $}
%%%%%%%%%%%%%%%%%%%%%%
\item 
\gleichungZZ{4x-9y}{11}{-x+8y}{3}

\LoesungsBlock{$\mathbb{L}_{(x;y)}
= \left\{ \left( 5;1 \right) \right\}  $}
%%%%%%%%%%%%%%%%%%%%%%
\item 
\gleichungZZ{2s+5t}{1}{6s+7t}{3}

\LoesungsBlock{$\mathbb{L}_{(s;t)}
= \left\{ \left( \frac12; 0 \right) \right\}  $}
%%%%%%%%%%%%%%%%%%%%%%
\item 
\gleichungZZ{2x+30y}{150-4y}{6x+7y}{110-10y}

\LoesungsBlock{$\mathbb{L}_{(x;y)}
= \left\{ \left( 7; 4 \right) \right\}  $}
%%%%%%%%%%%%%%%%%%%%%%
\item 
\gleichungZZ{\frac{c}2 + \frac{d}3}{7}{\frac{c}3 + \frac{d}2}{8}

\LoesungsBlock{$\mathbb{L}_{(c;d)}
= \left\{ \left( 6; 12 \right) \right\}  $}
%%%%%%%%%%%%%%%%%%%%%%
\item 
\gleichungZZ{0.16a - 0.04b}{1}{0.38a - 0.22b}{2}

\LoesungsBlock{$\mathbb{L}_{(a; b)}
= \left\{ \left( 7; 3 \right) \right\}  $}
%%%%%%%%%%%%%%%%%%%%%%
\item 
\gleichungZZ{x \cdot{} \sqrt{2}}{y+\sqrt{2}}{2}{x+y}

\LoesungsBlock{$\mathbb{L}_{(x;y)}
= \left\{ \left( \sqrt{2} ;
2-\sqrt{2} \right) \right\}  \approx \{(1.41; 0.5858)\}$}
%%%%%%%%%%%%%%%%%%%%%%
\item
Erstellen Sie ein lineares Gleichungssystem und lösen Sie dieses.

$$a-1 = 2a-3b+4 = b$$

\LoesungsBlock{$\mathbb{L}_{(a; b)}
= \left\{ \left( 4; 3 \right) \right\}  $}

\end{enumerate}
\newpage
%%%%%%%%%%%%%%%%%%%%%%%%%%%%%%%%%%%%%%%%%%%%%%%%%%%%%%%%%%%%%%%%%%%%%%%
\section{Taschenrechner}
Bringen Sie jeweils zunächst in die Grundform.


\begin{enumerate}[label=\alph*)]
\item
\gleichungZZ{-231x + 1155y}{1386}{\frac{-5}3 x + \frac{25y}3}{10}


\LoesungsBlock{Es gibt unendlich viele Lösungen: Die beiden
Gleichungen sind linear abhängig. Die zugehörigen Geraden sind
zusammenfallend.

$\mathbb{L}_{(x;y)} =
 \left\{ (x|y) \in \mathbb{R}\times\mathbb{R} \middle| y = \frac{6+x}{5} \right\}
=
\left\{ (x|y)\in \mathbb{R}\times\mathbb{R} \middle| x = 5y -6 \right\}
$}
%%%%%%%%%%%%%%%%%%%%%%%%%%%%%%%%%%%%%%%%%%%%%%%
\item

\gleichungZZ{x\cdot{}\frac23 + \frac{2y}9}{400}{0.66 x  + 0.22y}{400}

\LoesungsBlock{$\mathbb{L}_{(x;y)} = \left\{ \right\}$}

%%%%%%%%%%%%%%%%%%%%%%%%%%%%%%%%%%%%%%%%%%%%%%%
\item

\gleichungZZ{3x}{18+4y}{5y}{7x+2}

\LoesungsBlock{$\mathbb{L}_{(x;y)} = \left\{ \left(\frac{98}{13} ; \frac{15}{13}\right)\right\}$}
%%%%%%%%%%%%%%%%%%%%%%%%%%%%%%%%%%%%%%%%%%%%%%%
\item

\gleichungZZ{6x}{27-3y}{\frac{14x}3 - 14}{\frac{-7}{3}\cdot{}y}

\LoesungsBlock{$\mathbb{L}_{(x;y)} = \left\{ \right\}$}
%%%%%%%%%%%%%%%%%%%%%%%%%%%%%%%

\item
\gleichungZZ{5x+3}{7y}{10x-14y}{-6}


\LoesungsBlock{Es gibt unendlich viele Lösungen: Die beiden
Gleichungen sind linear abhängig. Die zugehörigen Geraden sind
zusammenfallend.

$\mathbb{L}_{(x;y)} =
 \left\{ (x|y) \in \mathbb{R}\times\mathbb{R} \middle| y = \frac57 x
 + \frac37 \right\}
=
\left\{ (x|y)\in \mathbb{R}\times\mathbb{R} \middle| x = \frac75 y - \frac35 \right\}
$}
%%%%%%%%%%%%%%%%%%%%%%%%%%%%%%%%%%%%%%%%%%%%%%%
\item
\gleichungZZ{200x + 5y}{116}{400x}{81y+177}

\LoesungsBlock{$
\mathbb{L}_{(x;y)} =
 \left\{ \left(0.555 ; \frac59 \right) \right\} =
 \left\{ \left(\frac{111}{200} ; \frac59 \right) \right\} 
$}

\end{enumerate}
\newpage
%%%%%%%%%%%%%%%%%%%%%%%%%%%%%%%%%%%%%%%%%%%%%%%%%%%%%%%%%%%%%%%%%
\section{Graphische Interpretation}
\begin{enumerate}[label=\alph*)]
\item Betrachten Sie die Gleichungen als
 Funktionsgleichungen. Skizzieren Sie die Graphen und bestimmen Sie
 die Lösungen \textit{rein graphisch}.


\gleichungZZ{2x+y}{4}{x+y}{3}


\noTRAINER{\bbwCenterGraphic{120mm}{img/ksys6666.png}}
\TRAINER{\bbwCenterGraphic{60mm}{img/ksys6666Lsg.png}}
\LoesungsBlock{$x=1; y=2$}
\newpage
%%%%%%%%%%%%%%%%%%%%%%%%%%%%%%%%%%
\item

Lösen Sie das Gleichungssystem graphisch für $-20 \le x \le 80$ und
$-50\le y \le 100$.

\gleichungZZ{10y}{7.5x}{18.75x + 25y}{1500}


\noTRAINER{\bbwCenterGraphic{120mm}{img/ksys208040100.png}}
\TRAINER{\bbwCenterGraphic{60mm}{img/ksys208040100Lsg.png}}

\LoesungsBlock{$x=40; y=30$}
\newpage
%%%%%%%%%%%%%%%%%%%%%%%%%%%%%%%%%%%
\item
Lösen Sie das Gleichungssystem graphisch, indem Sie jede Gleichung in
die Form $y=ax+b$ bringen.

Skizzieren Sie im Bereich $-10 \le x \l 10$ und $-10 \le y \le 10$.

\gleichungZZ{3x + y -9.5}{4y - \frac{28}7}{\frac{-4y}{5} + x}{-5 + 0.2x}

\noTRAINER{\bbwCenterGraphic{120mm}{img/ksys10101010.png}}
\TRAINER{\bbwCenterGraphic{60mm}{img/ksys10101010Lsg.png}}

\LoesungsBlock{$\mathbb{L} = \{\}$}
\newpage

\item
Lösen Sie das Gleichungssystem graphisch für $-50 \le x \le 50$ und
$-50 \le y \le 50$.

\gleichungZZ{40}{-2y-x}{3.5x -5y + 82}{4\cdot{}\left(x-y+\frac{51}2\right)}

Skizzieren Sie im Bereich $-10 \le x \l 10$ und $-10 \le y \le 10$.

\noTRAINER{\bbwCenterGraphic{120mm}{img/ksys50505050.png}}
\TRAINER{\bbwCenterGraphic{60mm}{img/ksys50505050Lsg.png}}

\LoesungsBlock{Die Geraden sind zusammenfallend:

$\mathbb{L}_{(x;y)}
= \left\{ (x;y) \in \mathbb{R}\times\mathbb{R} \middle|
y= \frac{-1}2x-20 \Longleftrightarrow x= -2y-40 \right\}$}
\newpage

\end{enumerate}

\end{document}
