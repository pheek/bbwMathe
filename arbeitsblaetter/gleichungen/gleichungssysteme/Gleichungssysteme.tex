%%
%% Meta: Gleichungssysteme
%%

\input{bmsLayoutPage}
\renewcommand{\bbwAufgabenBlockID}{GLS}

\ifisNURAUFGABEN
\newcommand{\LoesungsBlock}[1]{\TRAINER{Lösung:#1
\vspace{1mm}
\hrule}}%% end new Command "LoesungsBlock"
\else
\newcommand{\LoesungsBlock}[1]{\noTRAINER{\TNTeop{}}\TRAINER{Lösung:\,\,\,\,#1
\vspace{1mm}
\hrule}}%% end new Command "LoesungsBlock"
\fi
%%%%%%%%%%%%%%%%%%%%%%%%%%%%%%%%%%%%%%%%%%%%%%%%%%%%%%%%%%%%%%%%%%

\usepackage{amssymb} %% für \blacktriangleright
\renewcommand{\metaHeaderLine}{Arbeitsblatt Gleichungssysteme}
\renewcommand{\arbeitsblattTitel}{Lineare Gleichungssysteme}

\newcommand{\TNTeopS}[1]{\TRAINER{#1}\noTRAINER{\TNTeop{}}}

\begin{document}%%
\arbeitsblattHeader{}

\begin{center}\textit{\tiny{V 0.1 - 14. Okt. 2025}}\end{center}

\tableofcontents{}

\newpage


\textbf{Vorgehen}


\begin{enumerate}
\item 
Lösen Sie von jedem Kapitel 1-2 Aufgaben. Wenn Sie Mühe haben, lösen
Sie auch die anderen Übungen ansonsten probieren Sie noch die letzte
Übung des Kapitels, bevor Sie zum nächsten Kapitel übergehen.
\end{enumerate}

\newpage

\section{Einsetzverfahren}
\begin{enumerate}[label=\alph*)]
\item

\gleichungZZ{2x+11y}{57}{x-3y}{-14}


\LoesungsBlock{$\mathbb{L}_{(x;y)} = \left\{ (1;5) \right\}  $}

%%%%%%%%%%%%%%%%%

\item
\gleichungZZ{4a+2b}{22}{3a-5b}{36}

\LoesungsBlock{$\mathbb{L}_{(a;b)} = \left\{ (7;-3) \right\}  $}

%%%%%%%%%%%%%%%%%%%%%%%
\item 
\gleichungZZ{s+3t}{4}{5s-32}{21t}

\LoesungsBlock{$\mathbb{L}_{(s;t)} = \left\{ \left(5;\frac{-1}{3}\right) \right\}  $}

%%%%%%%%%%%%%%%%%%%%%%%
\item 
\gleichungZZ{x+y}{-5}{x-y}{10}

\LoesungsBlock{$\mathbb{L}_{(x;y)} = \left\{ \left( 2.5;-7.5 \right) \right\}  $}

%%%%%%%%%%%%%%%%%%%%%%%
\item 
\gleichungZZ{y}{2x-3}{y}{-0.5x+1}

\LoesungsBlock{$\mathbb{L}_{(x;y)} = \left\{ \left( 1.6; 0.2 \right)= \left( \frac85 ; \frac15 \right) \right\}  $}

%%%%%%%%%%%%%%%%%%%%%%%
\item 
\gleichungZZ{y}{8x+4}{y}{-7x+2}

\LoesungsBlock{$\mathbb{L}_{(x;y)} = \left\{ \left( \frac{-2}{15} ; \frac{44}{15} \right) \right\}  $}

%%%%%%%%%%%%%%%%%%%%%%%
\item 
\gleichungZZ{3(2x+y)}{18}{9(4x+2y)+20y}{28}

\LoesungsBlock{$\mathbb{L}_{(x;y)} = \left\{ \left( 5; -4 \right) \right\}  $}

%%%%%%%%%%%%%%%%%%%%%%%
\item 
\gleichungZZ{3+2(y-3)}{-\frac{3(x+5)}{2}}{2x+7}{\frac15\left( 6(2y-1)+11\right)}

\LoesungsBlock{$\mathbb{L}_{(x;y)} = \left\{ \left( -3; 0 \right) \right\}  $}



\end{enumerate}

\newpage
%%%%%%%%%%%%%%%%%%%%%%%%%%%%%%%%%%%%%%%%%%%%%%%%%%%%%%%%%%%%%%%%%%%%%%%%
\section{nicht lineare Systeme}
(Optional bei GESO.)

\begin{enumerate}[label=\alph*)]
%%%%%%%%%%%%%%%%%%%%%%%
\item 
\gleichungZZ{x^3}{y+107}{3x^3+2y}{411}

\LoesungsBlock{$\mathbb{L}_{(x;y)} = \left\{ \left( 5;18 \right) \right\}  $}

%%%%%%%%%%%%%%%%%%%%%%%

\item 
\gleichungZZ{x^5-y}{27}{x^5+4y}{52}

\LoesungsBlock{$\mathbb{L}_{(x;y)} = \left\{ \left( 2;5 \right) \right\}  $}

%%%%%%%%%%%%%%%%%%%%%%%

\item 
\gleichungZZ{a^2+b^2}{20}{a+b}{3}

\LoesungsBlock{$\mathbb{L}_{(a;b)}
= \left\{ \left(\frac{3-\sqrt{31}}2; \frac{3+\sqrt{31}}2  \right), \left( \frac{3+\sqrt{31}}2; \frac{3-\sqrt{31}}2\right) \right\}  $}

%%%%%%%%%%%%%%%%%%%%%%%

\item 
\gleichungZZ{\sqrt{x} + y}{13.5}{x-7y}{103.5}

\LoesungsBlock{$\mathbb{L}_{(x;y)}
= \left\{ \left( 121; 2.5 \right) \right\}  $}

%%%%%%%%%%%%%%%%%%%%%%%%%%%%
\item
Die Summe, das Produkt und der Quotient $\left(\frac{a}{b}\right)$ von
zwei Zahlen $a$ und $b$ sind gleich. Berechnen Sie $a$ und $b$

\LoesungsBlock{$\mathbb{L}_{(a;b)}
= \left\{ \left( \frac12; -1 \right) \right\}  $}

\end{enumerate}

%%%%%%%%%%%%%%%%%%%%%%%%%%%%%%%%%%%%%%%%%%%%%%%%%%%%%%%%%%%%%%%%%%%%%%%%%%%%%
\newpage
%%%%%%%%%%%%%%%%%%%%%%%%%%%%%%%%%%%%%%%%%%%%%%%%%%%%%%%%%%%%%%%%%%%%%%%%
\section{Additions- bzw. Subtraktionsverfahren}

\begin{enumerate}[label=\alph*)]
\item 
\gleichungZZ{7x+6y}{32}{11x-4y}{10}

\LoesungsBlock{$\mathbb{L}_{(x;y)}
= \left\{ \left( 2; 3 \right) \right\}  $}
%%%%%%%%%%%%%%%%%%%%%%
\item 
\gleichungZZ{7x-11y}{-6}{9x+12y}{18}

\LoesungsBlock{$\mathbb{L}_{(x;y)}
= \left\{ \left( \frac{42}{61}; \frac{60}{61} \right) \right\}  $}
%%%%%%%%%%%%%%%%%%%%%%
\item 
\gleichungZZ{4x-9y}{11}{-x+8y}{3}

\LoesungsBlock{$\mathbb{L}_{(x;y)}
= \left\{ \left( 5;1 \right) \right\}  $}
%%%%%%%%%%%%%%%%%%%%%%
\item 
\gleichungZZ{2s+5t}{1}{6s+7t}{3}

\LoesungsBlock{$\mathbb{L}_{(s;t)}
= \left\{ \left( \frac12; 0 \right) \right\}  $}
%%%%%%%%%%%%%%%%%%%%%%
\item 
\gleichungZZ{2x+30y}{150-4y}{6x+7y}{110-10y}

\LoesungsBlock{$\mathbb{L}_{(x;y)}
= \left\{ \left( 7; 4 \right) \right\}  $}
%%%%%%%%%%%%%%%%%%%%%%
\item 
\gleichungZZ{\frac{c}2 + \frac{d}3}{7}{\frac{c}3 + \frac{d}2}{8}

\LoesungsBlock{$\mathbb{L}_{(c;d)}
= \left\{ \left( 6; 12 \right) \right\}  $}
%%%%%%%%%%%%%%%%%%%%%%
\item 
\gleichungZZ{0.16a - 0.04b}{1}{0.38a - 0.22b}{2}

\LoesungsBlock{$\mathbb{L}_{(a; b)}
= \left\{ \left( 7; 3 \right) \right\}  $}
%%%%%%%%%%%%%%%%%%%%%%
\item 
\gleichungZZ{x \cdot{} \sqrt{2}}{y+\sqrt{2}}{2}{x+y}

\LoesungsBlock{$\mathbb{L}_{(x;y)}
= \left\{ \left( \sqrt{2} ;
2-\sqrt{2} \right) \right\}  \approx \{(1.41; 0.5858)\}$}
%%%%%%%%%%%%%%%%%%%%%%
\item
Erstellen Sie ein lineares Gleichungssystem und lösen Sie dieses.

$$a-1 = 2a-3b+4 = b$$

\LoesungsBlock{$\mathbb{L}_{(a; b)}
= \left\{ \left( 4; 3 \right) \right\}  $}

\end{enumerate}
\newpage
%%%%%%%%%%%%%%%%%%%%%%%%%%%%%%%%%%%%%%%%%%%%%%%%%%%%%%%%%%%%%%%%%%%%%%%
\section{Gleichsetzungsverfahren}
\begin{enumerate}[label=\alph*)]
\item
\gleichungZZ{3y+2}{4x+12}{3y+2}{8x-4}

\LoesungsBlock{$\mathbb{L}_{(x;y)} = \left\{ \left(4 ; \frac{26}3 \right)  \right\}$}
%%%%%%%%%%%%%%%%%%%%

\item
\gleichungZZ{2x-5y}{6}{3x-5y}{-1}

\LoesungsBlock{$\mathbb{L}_{(x;y)} = \left\{ \left(-7 ;  -4  \right)  \right\}$}
%%%%%%%%%%%%%%%%%%%%

\item
\gleichungZZ{-2x-y}{3+4x+y}{-2x-y}{3-2x+y+12}

\LoesungsBlock{$\mathbb{L}_{(x;y)} = \left\{ \left(2 ; -7.5  \right)  \right\}$}

\end{enumerate}
\newpage


%%%%%%%%%%%%%%%%%%%%%%%%%%%%%%%%%%%%%%%%%%%%%%%%%%%%%%%%%%%%%%%%%%%%%%%
\section{Graphische Interpretation}
\begin{enumerate}[label=\alph*)]
\item Betrachten Sie die Gleichungen als
 Funktionsgleichungen. Skizzieren Sie die Graphen und bestimmen Sie
 die Lösungen \textit{rein graphisch}.


\gleichungZZ{2x+y}{4}{x+y}{3}


\noTRAINER{\bbwCenterGraphic{120mm}{img/ksys6666.png}}
\TRAINER{\bbwCenterGraphic{60mm}{img/ksys6666Lsg.png}}
\LoesungsBlock{$x=1; y=2$}
\newpage
%%%%%%%%%%%%%%%%%%%%%%%%%%%%%%%%%%
\item

Lösen Sie das Gleichungssystem graphisch für $-20 \le x \le 80$ und
$-50\le y \le 100$.

\gleichungZZ{10y}{7.5x}{18.75x + 25y}{1500}


\noTRAINER{\bbwCenterGraphic{120mm}{img/ksys208040100.png}}
\TRAINER{\bbwCenterGraphic{60mm}{img/ksys208040100Lsg.png}}

\LoesungsBlock{$x=40; y=30$}
\newpage
%%%%%%%%%%%%%%%%%%%%%%%%%%%%%%%%%%%
\item
Lösen Sie das Gleichungssystem graphisch, indem Sie jede Gleichung in
die Form $y=ax+b$ bringen.

Skizzieren Sie im Bereich $-10 \le x \le 10$ und $-10 \le y \le 10$.

\gleichungZZ{3x + y -9.5}{4y - \frac{28}7}{\frac{-4y}{5} + x}{-5 + 0.2x}

\noTRAINER{\bbwCenterGraphic{120mm}{img/ksys10101010.png}}
\TRAINER{\bbwCenterGraphic{60mm}{img/ksys10101010Lsg.png}}

\LoesungsBlock{$\mathbb{L} = \{\}$}
\newpage

\item
Lösen Sie das Gleichungssystem graphisch für $-50 \le x \le 50$ und
$-50 \le y \le 50$.

\gleichungZZ{40}{-2y-x}{3.5x -5y + 82}{4\cdot{}\left(x-y+\frac{51}2\right)}

Skizzieren Sie im Bereich $-10 \le x \le 10$ und $-10 \le y \le 10$.

\noTRAINER{\bbwCenterGraphic{120mm}{img/ksys50505050.png}}
\TRAINER{\bbwCenterGraphic{60mm}{img/ksys50505050Lsg.png}}

\LoesungsBlock{Die Geraden sind zusammenfallend:

$\mathbb{L}_{(x;y)}
= \left\{ (x;y) \in \mathbb{R}\times\mathbb{R} \middle|
y= \frac{-1}2x-20 \Longleftrightarrow x= -2y-40 \right\}$}
\newpage
%%%%%%%%%%%%%%%%%%%%%%%%%%%%%%%%
\item Bestimmen Sie die Koordinaten der Schnittpunkte der Geraden $g$
mit den Koordinatenachsen.

$$g: \,\,\,  y= -5x + 3$$

\LoesungsBlock{$$S_x = \left(\frac35 ; 0\right); S_y = (0; 3)$$}

%%%%%%%%%%%%%%%%%%%%%%%%%%%%%%%%%%%%%
\item Bestimmen Sie die Koordinaten des Schnittpunktes zwischen den
Geraden $g$ und $h$.

$$g: \,\,\,  y= 2x - 3$$
$$h: \,\,\,  y= \frac{-x}2 +1$$

\LoesungsBlock{$$S = \left(\frac85 ; \frac15\right)$$}
%%%%%%%%%%%%%%%%%%%%%%%%%%%%%%%%%%%%%
\item Bestimmen Sie den Schnittpunkt zwischen den Geraden AB und CD.


$$A=(1|-3); B=(3|-1); C=(0|0); D=(15|-9)$$

\LoesungsBlock{$$S = \left(\frac52 ; \frac{-3}2 \right)$$}

\end{enumerate}
\newpage

%%%%%%%%%%%%%%%%%%%%%%%%%%%%%%%%%%%%%%%%%%%%%%%%%%%%%%%%%%%%%%%%%%%%%%%
\section{Taschenrechner}
Bringen Sie jeweils zunächst in die Grundform.


\begin{enumerate}[label=\alph*)]
\item
\gleichungZZ{-231x + 1155y}{1386}{\frac{-5}3 x + \frac{25y}3}{10}


\LoesungsBlock{Es gibt unendlich viele Lösungen: Die beiden
Gleichungen sind linear abhängig. Die zugehörigen Geraden sind
zusammenfallend.

$\mathbb{L}_{(x;y)} =
 \left\{ (x|y) \in \mathbb{R}\times\mathbb{R} \middle| y = \frac{6+x}{5} \right\}
=
\left\{ (x|y)\in \mathbb{R}\times\mathbb{R} \middle| x = 5y -6 \right\}
$}
%%%%%%%%%%%%%%%%%%%%%%%%%%%%%%%%%%%%%%%%%%%%%%%
\item

\gleichungZZ{x\cdot{}\frac23 + \frac{2y}9}{400}{0.66 x  + 0.22y}{400}

\LoesungsBlock{$\mathbb{L}_{(x;y)} = \left\{ \right\}$}

%%%%%%%%%%%%%%%%%%%%%%%%%%%%%%%%%%%%%%%%%%%%%%%
\item

\gleichungZZ{3x}{18+4y}{5y}{7x+2}

\LoesungsBlock{$\mathbb{L}_{(x;y)} = \left\{ \left(\frac{-98}{13} ; \frac{-132}{13}\right)\right\}$}
%%%%%%%%%%%%%%%%%%%%%%%%%%%%%%%%%%%%%%%%%%%%%%%
\item

\gleichungZZ{6x}{27-3y}{\frac{14x}3 - 14}{\frac{-7}{3}\cdot{}y}

\LoesungsBlock{$\mathbb{L}_{(x;y)} = \left\{ \right\}$}
%%%%%%%%%%%%%%%%%%%%%%%%%%%%%%%

\item
\gleichungZZ{5x+3}{7y}{10x-14y}{-6}


\LoesungsBlock{Es gibt unendlich viele Lösungen: Die beiden
Gleichungen sind linear abhängig. Die zugehörigen Geraden sind
zusammenfallend.

$\mathbb{L}_{(x;y)} =
 \left\{ (x|y) \in \mathbb{R}\times\mathbb{R} \middle| y = \frac57 x
 + \frac37 \right\}
=
\left\{ (x|y)\in \mathbb{R}\times\mathbb{R} \middle| x = \frac75 y - \frac35 \right\}
$}
%%%%%%%%%%%%%%%%%%%%%%%%%%%%%%%%%%%%%%%%%%%%%%%
\item
\gleichungZZ{200x + 5y}{116}{400x}{81y+177}

\LoesungsBlock{$
\mathbb{L}_{(x;y)} =
 \left\{ \left(0.555 ; \frac59 \right) \right\} =
 \left\{ \left(\frac{111}{200} ; \frac59 \right) \right\} 
$}

\end{enumerate}
\newpage
%%%%%%%%%%%%%%%%%%%%%%%%%%%%%%%%%%%%%%%%%%%%%%%%%%%%%%%%%%%%%%%%%
\section{$3x3$- und höhere Gleichungssysteme}

\begin{enumerate}[label=\alph*)]
\item
\gleichungDD{4x+7y}{10}{3x+8z}{13}{3y+5z}{16}
\LoesungsBlock{$\mathbb{L}_{(x;y;z)} = \left\{ \left( -1; 2; 2 \right) \right\}  $}
%%%%%%%%%%%%%%%%%%%%%%%%%%%%%%%%
\item
Eine quadratische Funktion $y=ax^2 + bx + c$ geht durch die Punkte
$P=(-4|8)$, $Q=(0|0)$ und $R=(10|15)$. Bestimmen Sie die Koeffizienten
$a$, $b$ und $c$ von Hand, indem Sie jeden der drei Punkte in die
Funktionsgleichung einsetzen und das entstehende Gleichungssystem
auf\/lösen.

\LoesungsBlock{Die Gleichung mit $a=\frac14$, $b=-1$ und $c=0$ lautet $y = \frac14x^2-x$}
%%%%%%%%%%%%%%%%%%%%%%%%%%%%%%%%
\item
Eine quadratische Funktion $y=ax^2 + bx + c$ geht durch die Punkte
$P=(1|-1)$, $Q=(2|4)$ und $R=(4|8)$. Bestimmen Sie die Koeffizienten
$a$, $b$ und $c$ von Hand.

\LoesungsBlock{Die Gleichung mit $a=-1$, $b=8$ und $c=-8$ lautet $y = -x^2 + 8x -8$}
%%%%%%%%%%%%%%%%%%%%%%%%%%%%%%%%%%
\item
Im folgenden ist ein elektronischer Schaltplan gegeben.

\bbwCenterGraphic{9cm}{img/URISchaltplan.png}

Es gilt die allgemeingültige Formel $U=R\cdot{}I$ mit $U$ = Spannung,
$R$=elektrischer Widerstand und $I$ = Stromstärke.

Gegeben sind
$U_0 = 30\,\text{V}$, 
$R_1 = 12\,\Omega$,
$R_2 = 20\,\Omega$,
$R_3 = 16\,\Omega$,
$R_4 = 8\,\Omega$ und
$R_5 = 24\,\Omega$.

Berechnen Sie die Stromstärken $I_0$ bis $I_5$ anhand der folgenden
Kirchhoffschen Regeln:

1. Knotenregel: An jedem Punkt sind die Summen der «herausfließenden»
Ströme gleich der Summe der «ankommenden» Ströme. Beispiel: $I_0 = I_1
+ I+2$.

2. Maschenregel: In jedem geschlossenen Kreis (Masche) heben sich alle
Spannungen auf. Beispiel $U_1 + U_4 - U_0 = 0$. Mit
$U_0 = R_0 \cdot{} I_0$,
$U_1 = R_1 \cdot{} I_1$ bzw. 
$U_4 = R_4 \cdot{} I_4$.

\LoesungsBlock{In Ampere ergeben sich die folgenden Ströme:

$I_0 = \frac{240}{109}\approx 2.202; I_1 = \frac{315}{218} \approx 1.445;
I_2 = \frac{165}{218}\approx 0.7569; I_3 = \frac{15}{109}\approx
0.1376; I_4 = \frac{345}{218}\approx 1.583; I_5
= \frac{135}{218}\approx 0.6193$}%% end Loesungsblock
\newpage

%%%%%%%%%%%%%%%%%%%%%%%%%%%%%%%%%
\item
In einem Kreisel sind die folgenden Verkehrsflüsse festgestellt
worden:

\bbwCenterGraphic{90mm}{img/Strassen.png}

a) Bestimmen Sie die Verkehrsflüsse $x_1, x_2, ..., x_6$

b) Welcher Verkehrsfluss ist am kleinsten? Welcher am größten?


\LoesungsBlock{
a) Die Verkehrsflüsse können nur relativ zu einander
angegeben werden: $x_6 \ge 0$, $x_1 \ge x_6 + 120$, $x_2=x_6+20$,
$x_3= x_6 + 100$, $x_4 = x_6 + 40$, $x_5 = x_6+110$


b) $x_6$ hat am wenigsten Verkehr, $x_1$ hat am meisten.}%% end Loesungsblock



\end{enumerate}
\newpage
%%%%%%%%%%%%%%%%%%%%%%%%%%%%%%%%%%%%%%%%%%%%%%%%%%%%%%%%%%%%%%%%%%%%%%%%%%%%%
\section{Substitution}

\begin{enumerate}[label=\alph*)]
\item
\gleichungZZ{5\cdot{}\frac{x}y - (x - 13)}{13}{\frac{7x}y + 5\cdot{}(x-13)}{63}

\LoesungsBlock{$\mathbb{L}_{(x;y)} = \left\{ \left(20;5\right) \right\}$}%% end LoesungsBlock


%%%%%%%%%%%%%%%%%%%%%%%%%%
\item
\gleichungZZ{\frac3{a+b}}{5-\frac4{a-b}}{\frac{-9}{a+b}}{-1-\frac2{a-b}}


\LoesungsBlock{$\mathbb{L}_{(a;b)} = \left\{ \left(2; 1\right) \right\}$}%% end LoesungsBlock

%%%%%%%%%%%%%%%%%%%%%%%%%%%%%%%%%%%%%%%%%
\item %% c)

\gleichungZZ{5\cdot{}\frac{x-1}{y} - 3\cdot{}\frac{x+y}{x}}{-6}{-10\cdot{}\frac{1-x}{y} + 12\cdot{}\frac{x+y}{x}}{36}

\LoesungsBlock{$\mathbb{L}_{(x;y)} = \left\{ \left(3; 5\right) \right\}$}%% end LoesungsBlock


%%%%%%%%%%%%%%%%%%%%%%%%%%%%%%%%%%%%%%%
\item
\gleichungZZ{\sqrt{m^2 - n^2}}{6-\sqrt{m-n}}{\sqrt{m-n}}{\sqrt{m^2-n^2}-4}

\LoesungsBlock{$\mathbb{L}_{(m; n)} = \left\{ \left(13; 12\right) \right\}$}%% end LoesungsBlock

%%%%%%%%%%%%%%%%%%%%%%%%%%%%%%%
\item
Suchen Sie unter «vermischte Aufgaben» (Kap. \ref{vermischteAufgaben}
auf Seite \pageref{vermischteAufgaben})
weitere Aufgaben, die sinnvollerweise mittels Substitution gelöst
werden.

\end{enumerate}
\newpage
%%%%%%%%%%%%%%%%%%%%%%%%%%%%%%%%%%%%%%%%%%%%%%%%%%%%%%%%%%%%%%%%%%%%%%%
\section{Vermischte Aufgaben}\label{vermischteAufgaben}
Lösen Sie mit einem geeigneten Verfahren.

\begin{enumerate}[label=\alph*)]
\item
\gleichungZZ{\frac{-x}2}{2y-1-x}{\frac{x}2+3}{1-y}

\LoesungsBlock{$\mathbb{L}_{(x; y)} = \left\{ \left(10; -7\right) \right\}$}%% end LoesungsBlock
%%%%%%%%%%%%%%%%%%%%%%%%%%%%%%%%%%%%%%%%%%%%%%%%%%%%%

\item
\gleichungZZ{\frac{a+b}6}{\frac{a}2-5}{a+8}{\frac{b+a}6}

\LoesungsBlock{$\mathbb{L}_{(a; b)} = \left\{ \left(-26; -82\right) \right\}$}%% end LoesungsBlock


%%%%%%%%%%%%%%%%%%%%%%%%%%%%%%%%%%%%%%%%%%%%%%%%%%%%%

\item
\gleichungZZ{5x-8y}1{20x}{7+32y}

\LoesungsBlock{$\mathbb{L}_{(x; y)} = \left\{  \right\}$}%% end LoesungsBlock
%%%%%%%%%%%%%%%%%%%%%%%%%%%%%%%%%%%%%%%%%%%%%%%%%%%%%

\item
\gleichungZZ{\frac{m}2 + \frac{n}3}6{3m + 2n}{36}

\LoesungsBlock{$\mathbb{L}_{(m; n)} = \left\{ (m;n)\in\mathbb{R} \middle| m=12 - \frac{2n}3 \right\}$}%% end LoesungsBlock

%%%%%%%%%%%%%%%%%%%%%%%%%%%%%%%%%%%%%%%%%%%%%%%%%%%%%

\item
$$2x+7y = -5+8y = 24x + 7$$

\LoesungsBlock{$\mathbb{L}_{(x; y)} = \left\{ \left(\frac72 ; 12 \right) \right\}$}%% end LoesungsBlock


%%%%%%%%%%%%%%%%%%%%%%%%%%%%%%%%%%%%%%%%%%%%%%%%%%%%%

\item
\gleichungZZ{\frac{4x-1}5-x}{y-\frac{3y-4}2}{\frac{x+y-3}{2x-y+7}}{\frac34}

\LoesungsBlock{$\mathbb{L}_{(x; y)} = \left\{ \left(\frac{11}4 ; \frac{11}2 \right) \right\}$}%% end LoesungsBlock

%%%%%%%%%%%%%%%%%%%%%%%%%%%%%%%%%%
\item
\gleichungZZ{3x}{2+5y}{6}{\frac{24}{3x} + \frac{10y}{x}}

\LoesungsBlock{$\mathbb{L}_{(x; y)} = \left\{  \right\}$}%% end LoesungsBlock
%%%%%%%%%%%%%%%%%%%%%%%%%%%%%%%%%%%
\item
\gleichungZZ{\frac{b}{a+b}}{1-\frac3{a-b}}{\frac{2b}{a+b}}{1-\frac{2}{a-b}}

\LoesungsBlock{$\mathbb{L}_{(a; b)} = \left\{ (6;2) \right\}$}%% end LoesungsBlock
%%%%%%%%%%%%%%%%%%%%%%%%%%%%%%%%%%%%
\item
\gleichungZZ{\frac{2y}{4x-1} + 9}{\frac5{x+2y}}{\frac{10}{x+2y}-12}{\frac{y}{4x-1}}

\LoesungsBlock{$\mathbb{L}_{(x; y)} = \left\{ \left( \frac15; \frac25 \right) \right\}$}%% end LoesungsBlock

\end{enumerate}
\newpage
%%%%%%%%%%%%%%%%%%%%%%%%%%%%%%%%%%%%%%%%%%%%%%%%%%%%%%%%%%%%%%%%%%%%%
%%%%%%%%%%%%%%%%%%%%%%%%%%%%%%%%%%%%%%%%%%%%%%%%%%%%%%%%%%%%%%%%%%%%%

\section{Gleichungssysteme mit Parametern}

Lösen Sie jeweils nach den Variablen $x$ und $y$ auf.

\begin{enumerate}[label=\alph*)]
\item
\gleichungZZ{x+y}{-a}{x-4y}{2a}

\LoesungsBlock{$\mathbb{L}_{(x; y)} = \left\{ \left(\frac{-2a}5 ; \frac{-3a}5\right) \right\}$}%% end LoesungsBlock
%%%%%%%%%%%%%%%%%%%%%%%%%%%%%%%%%%%%%%%%%%%%%%%%%%%%%%%%
\item
\gleichungZZ{3x-7y}{-4y}{4x+3y}{7c}

\LoesungsBlock{$\mathbb{L}_{(x; y)} = \left\{  \left( 4c-3b;
4b-3c \right) \right\}$}

%%%%%%%%%%%%%%%%%%%%%%%%%%%%%%%%%%%%%%%%%%%%%%%
\item
\gleichungZZ{7x=5m}{2n-3y}{7y+2n}{5m-3x}

\LoesungsBlock{$\mathbb{L}_{(x; y)} = \left\{  \left( \frac{m+n}2;
\frac{m-n}n \right) \right\}$}

%%%%%%%%%%%%%%%%%%%%%%%%%%%%%%%%%%%%%%%%%%%%%
\item
\gleichungZZ{y}{u+x}{x}{2y-5}

\LoesungsBlock{$\mathbb{L}_{(x; y)} = \left\{  \left( v-2u ; v-u \right) \right\}$}

\end{enumerate}
\newpage
%%%%%%%%%%%%%%%%%%%%%%%%%%%%%%%%%%%%%%%%%%%%%%%%%%%%%%%%%%%%%%%%
\subsection{Fallunterscheidung}
Lösen Sie nach $(x;y)$ auf und führen Sie eine Fallunterscheidung je
nach Werten der Parameter durch.

\begin{enumerate}[label=\alph*)]
\item
\gleichungZZ{2mx+y}3{-x-y}n

\LoesungsBlock{$\mathbb{L}_{(x; y)} = \left\{  \left( \frac{n+3}{2m-1}
; \frac{3+2mn}{1-2m} \right) \right\}$

1. Sonderfall: $n\ne -3$ und $m=\frac12$
$\Longrightarrow \mathbb{L}=\{\}$

2. Sonderfall: $n=-3$. So gibt es zwei Unterfälle:

2.1 $x=0 \Longrightarrow \mathbb{L} = \{(0;3)\}$

2.2 $m=\frac12 \Longrightarrow \mathbb{L} = \left\{ (x;y) \middle| y=3-x \right\}$
}%% end LoesungsBlock
%%%%%%%%%%%%%%%%%%%%%%%%%%%%%%%%%%%%%%%%%%%%%%

\item
\gleichungZZ{ax+2y}2{x+y}{2a}

\LoesungsBlock{Für $a\ne2$ gilt:

$\mathbb{L}_{(x; y)} = \left\{  \left( \frac{4a-2}{2-a}
; \frac{2-2a^2}{2-a} \right) \right\}$

Ist $a=2$ so gilt:

$\mathbb{L}_{(x; y)} = \left\{   \right\}$

}%% end LoesungsBlock

%%%%%%%%%%%%%%%%%%%%%%%%%%%%%%%%%%%%%%%%
\item

\gleichungZZ{-4x+5y}4{6x -7.5y}{\vartheta}

\LoesungsBlock{Für $\vartheta \ne -6$ gilt:

$\mathbb{L}_{(x; y)} = \left\{  \right\}$

Ist hingegen $\vartheta = -6$ so gilt:

$\mathbb{L}_{(x; y)} = \left\{  (x;y) \in \mathbb{R}\times\mathbb{R}
| y=\frac54x + \frac54  \right\}$

}%% end LoesungsBlock
%%%%%%%%%%%%%%%%%%%%%%%%%%%%%%%%%%%%%%%%%%%%%%

\item
\gleichungZZ{mx+y}0{-x-my}0

\LoesungsBlock{$m=1$:

$\mathbb{L}_{(x; y)} = \left\{ (x;y) | y=-x \right\}$

$m=-1$:

$\mathbb{L}_{(x; y)} = \left\{ (x;y) | y=x \right\}$

sonst:

$\mathbb{L}_{(x; y)} = \left\{ (0;0) \right\}$

}%% end LoesungsBlock

%%%%%%%%%%%%%%%%%%%%%%%%%%%%%%%%%%%%%%%%%%%%%%%%%%
\item

\gleichungZZ{y}{4x-5}{y}{fx+g}

\LoesungsBlock{
Fall $f\ne 4$:

$\mathbb{L}_{(x; y)} = \left\{ \left( \frac{g+5}{4-f} ;\frac{5f+4g}{4-f} \right) \right\}$

Fall $f=4$ und $g=-5$:

$\mathbb{L}_{(x; y)} = \left\{ (x;y) | y=4x-5 \right\}$

Fall $f=4$ und $g\ne -5$:

$\mathbb{L}_{(x; y)} = \left\{  \right\}$

}%% end LoesungsBlock

%%%%%%%%%%%%%%%%%%%%%%%%%%%%%%%%%%%%%%%%%%%%%%%%%%%%%%%%%%%%%%%%%%%%5
\item

\gleichungZZ{(\delta -9)x -y}{2}{2\delta x + y}{1}

\LoesungsBlock{
Fall $\delta = 3$:

$\mathbb{L}_{(x; y)} = \left\{  \right\}$

Fall $\delta \ne 3$:


$\mathbb{L}_{(x; y)} = \left\{ \left( \frac1{\delta - 3} ;\frac{3+\delta}{3-\delta} \right) \right\}$

}%% end LoesungsBlock
\end{enumerate}
\newpage
%%%%%%%%%%%%%%%%%%%%%%%%%%%%%%%%%%%%%%%%%%%%%%%%%%%%%%%%%%%%%%%%%%%%
%%%%%%%%%%%%%%%%%%%%%%%%%%%%%%%%%%%%%%%%%%%%%%%%%%%%%%%%%%%%%%%%%%%
\section{Aufgaben in Textform}
\subsection{Misch-Aufgaben}

\begin{enumerate}[label=\alph*)]
\item Zwei Pralinensorten, die pro 100 g CHF 8.20 bzw. CHF 10.40
kosten, werden gemischt und in Säcklein für CHF 22.50 verkauft. Wie
viel Gramm jeder Sorte sind darin enthalten?

\LoesungsBlock{Von der ersten Sorte sind 159.1 g und von der anderen
Sorte 90.91 g enthalten.}
%%%%%%%%%%%%%%%%%%%%%%%%%%%%%%%%%%%%%%%%%%%%%%%%%%%%%%%%%%%
\item

Zwei Kaffeesorten unterschiedlicher Qualität werden gemischt. Die
Mischung $A$ enthält 36 kg der besseren (teureren) und 24 kg der
schlechteren Sorte, was einen Kilopreis von CHF 20.40 ergibt. Die
Mischung $B$ enthält 24 kg der besseren und 36 kg der schlechteren
Sorte, ihr Kilopreis beträgt CHF 18.60. Welchen Kilopreis haben die
beiden Sorten, welche zum Mischen verwendet wurden?

\LoesungsBlock{Kilopreis der besseren Sorte = $24$ Fr/kg; Kilopreis
der schlechteren Sorte = $15$ Fr/kg.}
%%%%%%%%%%%%%%%%%%%%%%%%%%%%%%%%%%%%%%%%%%%
\item
Chemische Mischung:

Aus einer 20\%igen und einer
65\%igen Salzlösung (jeweils Gewichtsprozente) sollen durch mischen 40
kg einer neuen Lösung mit einem Salzgehalt von 45\% hergestellt
werden. Wie viel kg jeder Salzlösung werden benötigt?

\LoesungsBlock{In der Mischung sind $\frac{160}{9}\approx 17.78$ kg
der 20\%igen Salzlösung und $\frac{200}9 \approx 22.22$ kg der
65\%igen Salzlösung enthalten.}
%%%%%%%%%%%%%%%%%%%%%%%%%%%%%%%%%%%%%%%%%%%%%%%%5
\item
Metall-Legierung (Feingehalt)

Werden 1.2 kg einer Silbersorte mit 2.4 kg einer zweiten Silbersorte
legiert, so hat die Legierung eine Feingehalt von 800 (dies bedeutet
800 Promille, was 80\% Silberanteil gleich kommt).

Legiert man jedoch 2.4 kg der ersten mit 1.2 kg der zweiten Sorte, so
beträgt der Feingehalt 750 (= 75\% Silberanteil).

Welchen Feingehalt (in Promille \textperthousand) haben die beiden ursprünglichen
Silbersorten?

\LoesungsBlock{Die erste Silbersorte hat einen Feingehalt von 700 (\textperthousand) und
die zweite Sorte einen von 850 (\textperthousand).}
%%%%%%%%%%%%%%%%%%%%%%%%%%%%%%%%%%%%%%%%%%%%%%%%%%%%%%%%%55
\item Nuss-Mischung

Numa Nussbaumer handelt mit Nüssen. Numa möchte eine Nussmischung aus zwei
Sorten zusammenstellen: Walnüsse (Sorte $A$) und Mandeln (Sorte $B$).

Mischt Numa 4 kg Walnüsse mit 6 kg Mandeln, so kostet die gesamte
Mischung CHF 84.-

Werden hingegen 7 kg Walnüsse mit 3 kg Mandeln gemischt, so kostet 1
kg der zweiten Mischung CHF 9.90.

Wie viel kostet 1 kg Walnüsse? Wie viel 1 kg Mandeln?

\LoesungsBlock{
Walnüsse kosten CHF 11.40 pro kg; Mandeln 6.40 CHF/kg.
}

%%%%%%%%%%%%%%%%%%%%%%%%%%%%%%%%%%%%%%%%%%%%%%%%%%%%%%%%%
\item
Werden zwei Liter einer Sorte Spiritus mit 16 l einer anderen Sorte
gemischt und zur Mischung noch 7.4 l Wasser hinzugefügt, so erhält man
50\%igen Spiritus. Man erhält ebenfalls 50\%igen Spiritus wenn man 6 l
der ersten Sorte mit 10 l der zweiter Sorte mischt und noch 4.6 l
Wasser zugibt.

Wie viel \% Alkohol enthalten die ursprünglich verwendeten
Spiritus-Sorten?

Wir vernachlässigen hier die Volumenkontraktion, die entsteht, wenn man
Wasser und Alkohol zusammen mischt.

\LoesungsBlock{Die erste Sorte enthält ca $50\%$ und die zweit Sorte
ca. $73\%$ Alkohol.}

\end{enumerate}\newpage

%%%%%%%%%%%%%%%%%%%%%%%%%%%%%%%%%%%%%%%%%%%%%%%%%%%%%%%%%%%%%%%%%%%%%
%%%%%%%%%%%%%%%%%%%%%%%%%%%%%%%%%%%%%%%%%%%%%%%%%%%%%%%%%%%%%%%%%%%%%
\subsection{Zinsaufgaben}
\begin{enumerate}[label=\alph*)]
\item
Frau Gross hat ein Kapital in zwei Posten angelegt, einen zu 4\% und
einen zu 5\%. Nach ihrer Rechnung beträgt die Summe der Jahreszinsen
CHF 2\,560.-. Das sind aber CHF 80.- zu viel, denn sie hat die
Zinssätze verwechselt. Welche Posten hat sie zu welchem Zinssatz
angelegt?

\LoesungsBlock{32\,000.- CHF hat sie zu 4\% und 24\,000.- CHF zu 5\% angelegt.}
%%%%%%%%%%%%%%%%%%%%%%%%%%%%%%%%%%%%%%%%%%%%%%%
\item

Ein Kapital wird aufgeteilt und zu 2\% und 3\% angelegt. Zusammen
bringen die beiden Teile CHF 712.- Jahreszins. Das sind aber CHF 29.-
zu viel, da die Zinssätze vertauscht wurden. Wie groß sind die beiden
Teile des Kapitals?

\LoesungsBlock{Der 1. Teil des Kapitals betrug CHF 12\,500.- und der
zweit Teil betrug CHF 15\,400.- .}
%%%%%%%%%%%%%%%%%%%%%%%%%%%%%%%%%%%%%%%%%%%%%%%%
\item
Ein Kapital wird zu 4\% verzinst, ein anderes zu 5\%. Die Summe der
Jahreszinsen beträgt CHF 1\,410.-. Wird nach einem Jahr je der Zins zu
seinem Kapital geschlagen, so werden die beiden Kapitalien samt Zins
gleich groß.

Wie groß waren die Kapitalien am Anfang?

\LoesungsBlock{Das erste der Kapitalien war vor der Verzinsung CHF
15\,750.- wert, während das zweite Kapital den Wert CHF 15\,600.- hatte.}
%%%%%%%%%%%%%%%%%%%%%%%%%%%%%%%%%%%%%%%%%%%%%%%%%
\item
Eine Händlerin kaufte 800 Stück von Sorte I und 400 Stück von Sorte II
für insgesamt CHF 4\,200.-. Sorte I verkaufte sie mit einem Zuschlag
von 15\% und Sorte II sogar mit einem Zuschlag von 50\%. Der
Verkaufspreis betrug insgesamt CHF 5\,180.-. Für wie viel Geld hatte
sie ein Stück jeder Sorte eingekauft?

\LoesungsBlock{Die Sorte I kostete beim Einkauf CHF 4.- pro Stück und
Sorte II CHF 2.50 pro Stück.}

\end{enumerate}
\newpage
%%%%%%%%%%%%%%%%%%%%%%%%%%%%%%%%%%%%%%%%%%%%%%%%%%%%%%%%%%%%%%%%%%%
%%%%%%%%%%%%%%%%%%%%%%%%%%%%%%%%%%%%%%%%%%%%%%%%%%%%%%%%%%%%%%%%%%

\subsection{Leistung}
\begin{enumerate}[label=\alph*)]
\item
Einer Druckerei stehen für den Druck einer Zeitschrift zwei
Rotationspressen zur Verfügung. Sind beide Maschinen gleichzeitig im
Einsatz, so werden 24 Stunden für den Druck der Auflage benötigt.

Nachdem beide Maschinen drei Stunden gemeinsam gearbeitet haben, fällt
die erste Maschine aus. Der Schaden kann erst nach fünf Ausfallstunden
behoben werden. Nach erneutem zweistündigem Einsatz beider Maschinen
ist ein Viertel des Druckauftrags erledigt. In wie vielen Stunden
würde jede Maschine alleine die Zeitschrift drucken?

\LoesungsBlock{Die erste Maschine bräuchte 30h, die zweite Maschine
120h um den Auftrag alleine zu bewältigen.}
%%%%%%%%%%%%%%%%%%%%%%%%%%%%%%%%%%%%%%%%%%%%%%%%%%%%%%%%%%
\item
Um 1000 Schrauben herzustellen, braucht die alte Maschine neun Minuten
länger als die neue. Zusammen benötigen sie 20 Minuten. Wie lange
dauert es, wenn wegen eines Defekts der neuen Maschine nur die alte
eingesetzt werden kann?

\LoesungsBlock{Die alte Maschine braucht 45' für 1000 Schrauben.}
%%%%%%%%%%%%%%%%%%%%%%%%%%%%%%%%%%%%%%%%%%%%%%%%%%%%%%%%%%%%%
\item
Um einen Tank zu füllen, braucht Pumpe $A$ eine Stunde länger als
Pumpe $B$. Sind beide Pumpen gleichzeitig in Betrieb, so dauert die
Füllung sechs Stunden.

Wie lange dauert es, wenn nur Pumpe $B$ zur Verfügung steht?

\LoesungsBlock{Pumpe $B$ braucht alleine 11.52 (= 11:31) Stunden, um
den Tank alleine zu füllen.}

\end{enumerate}
\newpage
%%%%%%%%%%%%%%%%%%%%%%%%%%%%%%%%%%%%%%%%%%%%%%%%%%%%%%%%%%%%%%%
\subsection{Ziffern}
\begin{enumerate}[label=\alph*)]
\item Die Quersumme einer natürlichen zweistelligen Zahl beträgt
elf. Vertauscht man die Ziffern, so beträgt die Differenz zwischen der
neuen und der alten Zahl 63.

Wie lautet die Zahl?

\LoesungsBlock{Die Zahl lautet 29 bzw. 92.}
%%%%%%%%%%%%%%%%%%%%%%%%%%%%%%%%%%
\item
Die Quersumme einer zweiziffrigen natürlichen Zahl ist neun.
Vertauscht man die Ziffern, so erhält man das 0.375-fache der Zahl.

Wie lautet die Zahl?

\LoesungsBlock{Die Zahl lautet 72}

%%%%%%%%%%%%%%%%%%%%%%%%%%%%%
\item
Fügt man auf beiden Seiten einer zweistelligen natürlichen Zahl
die Ziffer Acht hinzu, so ist die neue Zahl vierstellig. Diese
vierstellige Zahl ist um 51 größer als das 119-fache der ursprünglichen
Zahl.

Was ist die ursprüngliche Zahl?

\LoesungsBlock{Die ursprüngliche Zahl ist 73.}
\end{enumerate}
\newpage
%%%%%%%%%%%%%%%%%%%%%%%%%%%%%%%%%%%%%%%%%%%%%%%%%%%%%%%%%%%%%%%%%%%
%%%%%%%%%%%%%%%%%%%%%%%%%%%%%%%%%%%%%%%%%%%%%%%%%%%%%%%%%%%%%%%%
\subsection{Teilen mit Rest}

\begin{enumerate}[label=\alph*)]
\item
Eine natürliche Zahl ergibt durch 13 dividiert den Rest zwei und durch sechs
dividiert den Rest eins. Die Differenz der Quotienten (ohne
Berücksichtigung der Reste) ist 20.

Wie lautet die Zahl?

\LoesungsBlock{Die Zahl lautet 223.}
%%%%%%%%%%%%%%%%%%%%%%%%%%%%%%%%%%%%%%%%%%%%%%%%%%%%%%%%%%5
\item
Eine natürliche Zahl mit zwei Ziffern wird um 18 größer, wenn man ihre
Ziffern vertauscht. Dividiert man die ursprüngliche Zahl durch ihre
Quersumme, so erhält man vier mit Rest sechs.

Wie heißt die Zahl?

\LoesungsBlock{Die Zahl heißt 46.}
%%%%%%%%%%%%%%%%%%%%%%%%%%%%%%%%%%%
\item
Die Summe von drei natürlichen Zahlen beträgt 200. Dividiert man die
zweite durch die erste, ist das Ergebnis 5 mit Rest 2. Dividiert man
die dritte durch die zweite, erhält man das gleiche Ergebnis.

Wie lauten die drei Zahlen?

\LoesungsBlock{Die Zahlen lauten 6, 32 und 162.}

\end{enumerate}
\newpage
\subsection{schwierigere Textaufgaben}

\begin{enumerate}[label=\alph*)]
\item
Farben der Sorte $A$ bzw. $B$ mit Kilopreisen von CHF 12.- bzw. 10.-
werden gemischt. Ein Kilogramm der Mischung kommt auf CHF 10.90 zu
stehen.
Nähme man bei gleicher Gesamtmenge der Mischung von der Sorte $A$ drei
kg weniger, so würde das Kilogramm der Mischung nur noch CHF 10.40
kosten.

Wie viele kg nahm man anfänglich von jeder Sorte?

\LoesungsBlock{In der ursprünglichen Mischung sind 5.4 kg der
Farbsorte $A$ und 6.6 kg der Sorte $B$.}
%%%%%%%%%%%%%%%%%%%%%%%%%%%%%%%%%%%%%%%%%%%%%%%%%%%%%%%%%%%%%%%%%5
\item
Dividiert man eine zweiziffrige Zahl durch ihre Quersumme, so erhält
man vier Rest neun. Vertauscht man die Ziffern der ursprünglichen Zahl und
dividiert die neue Zahl durch die um 13 vermehrte Quersumme, so erhält
man drei.

Wie lautet die gesuchte Zahl?

\LoesungsBlock{Die gesuchte Zahl lautet 57.}

\end{enumerate}

\end{document}
