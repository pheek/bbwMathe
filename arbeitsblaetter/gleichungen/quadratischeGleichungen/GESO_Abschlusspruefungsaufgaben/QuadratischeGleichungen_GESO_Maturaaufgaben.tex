%%
%% Meta: TI nSpire Einführung
%%       Ziel: Damit die Grundoperationen damit durchgeführt werden können.
%%             Damit man sich an den Rechner gewöhnt.
%%

\input{bmsLayoutPage}

%%%%%%%%%%%%%%%%%%%%%%%%%%%%%%%%%%%%%%%%%%%%%%%%%%%%%%%%%%%%%%%%%%

\usepackage{amssymb} %% für \blacktriangleright

\renewcommand{\metaHeaderLine}{Aus alten Maturaaufgaben}
\renewcommand{\arbeitsblattTitel}{Quadratische Gleichungen}

\begin{document}%%
\arbeitsblattHeader{}
Lösen Sie nach $x$ auf:


%%%%%%%%%%%%%%%%%%%%%%%%%%%%%%%%%%%%%%%%%%%%%%%%%%%%%%%%%%%%%%%%%%%%%%%%
$1.\ (2018)$

$$\frac{6x-24}{3-x} + x - 2 = \frac{6}{x-3}$$

\TNTeop{$ \DefinitionsMenge{} = \mathbb{R}\backslash{}\{3\}, \LoesungsMenge{} = \{8\}$}

%%%%%%%%%%%%%%%%%%%%%%%%%%%%%%%%%%%%%%%%%%%%%%%%%%%%%%%%%%%%%%%%%%%%%%%%
$2.\ (2018)$

$$\frac{x^2}{x-2} + \frac{4}{2-x} = 3$$

\TNTeop{$ \DefinitionsMenge{} = \mathbb{R}\backslash{}\{2\}, \LoesungsMenge{} = \{1\}$}

%%%%%%%%%%%%%%%%%%%%%%%%%%%%%%%%%%%%%%%%%%%%%%%%%%%%%%%%%%%%%%%%%%%%%%%%
$3.\ (2018)$

$$\frac{x^2-16x}{x-3} + 1 = \frac{39}{3-x}$$

\TNTeop{$ \DefinitionsMenge{} = \mathbb{R}\backslash{}\{3\}, \LoesungsMenge{} = \{12\}$}


%%%%%%%%%%%%%%%%%%%%%%%%%%%%%%%%%%%%%%%%%%%%%%%%%%%%%%%%%%%%%%%%%%%%%%%%
$4.\ (2018)$

$$\frac{x^2-10x}{x-4} + 1 = \frac{24}{4-x}$$

\TNTeop{$ \DefinitionsMenge{} = \mathbb{R}\backslash{}\{4\}, \LoesungsMenge{} = \{5\}$}

%%%%%%%%%%%%%%%%%%%%%%%%%%%%%%%%%%%%%%%%%%%%%%%%%%%%%%%5
2017:
Bestimmen Sie den Definitionsbereich des folgenden Terms:
$$\frac{5x}{x^2-9x-36}$$


\TNTeop{$\DefinitionsMenge{} = \mathbb{R}\backslash{}\{-3; 12\}$}

%%%%%%%%%%%%%%%%%%%%%%%%%%%%%%%%%%%%%%%%%%%%%%%%%%%%%%%5
2017: 
$$\frac{x+1}{x-3} = \frac{10-2x}{x^2-3x}$$

\TNTeop{$\DefinitionsMenge{} = \mathbb{R}\backslash{}\{0, 3\}; \LoesungsMenge{}=\{-5, 2\}$}

\end{document}
