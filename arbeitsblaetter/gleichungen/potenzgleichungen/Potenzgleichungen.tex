\input{bmsLayoutPage}

%%%%%%%%%%%%%%%%%%%%%%%%%%%%%%%%%%%%%%%%%%%%%%%%%%%%%%%%%%%%%%%%%%

\usepackage{amssymb} %% für \blacktriangleright
\renewcommand{\metaHeaderLine}{Potenzgleichungen}
\renewcommand{\arbeitsblattTitel}{(BMS)}

\begin{document}%%
\arbeitsblattHeader{}


\section{Ganzzahlige Exponenten}
\small{Nach einem Arbeitsblatt von M. Rohner.}


\subsection{Mehrere Lösungen ?}
Bestimmen Sie die Lösungsmenge für die Variable $x$:

%%\newcommand{\aufgabeML}[2]{$#1$ ... $#2$}

\begin{bbwAufgabenBlock}

\item $x^4=81$ \hspace{10mm} $\lx = \LoesungsRaumLang{\{-3; 3\}}$\TRAINER{\\$x=\pm\sqrt[4]{81}$}\abplz{6}

\item
 $4x^6=256$ \hspace{10mm} $\lx=\LoesungsRaumLang{\{-2; 2\}}$
 \TRAINER{\\$|: 4:  x^6 = 64 \Longrightarrow x = \pm\sqrt[6]{64}$}\abplz{6}\noTRAINER{\newpage}

\item
 $-x^4=16$ \hspace{10mm} $\lx=\LoesungsRaumLang{\{\} \text{ (keine
 Wurzeln aus negativen Zahlen) }}$
 \TRAINER{\\$|: \cdot{}(-1) : x^4 = -16$}\abplz{6}

\item
 $x^3=64$ \hspace{10mm} $\lx=\LoesungsRaumLang{\{+4\} }$
 \TRAINER{\\$x = +\sqrt[3]{64}$}\abplz{6}

\item
 $-x^3=27$ \hspace{10mm} $\lx=\LoesungsRaumLang{\{-3\} }$
 \TRAINER{\\$\Longrightarrow x^3 = -27 \Longrightarrow x = -\sqrt[3]{27}$}\abplz{6}\noTRAINER{\newpage}

\item
 $\frac56 x^3 = \frac{50}{60}$ \hspace{10mm} $\lx=\LoesungsRaumLang{\{+1\}  }$
 \TRAINER{\\$\frac56 x^3 = \frac56 \text{ durch } \frac56 \text{ teilen: } x^3 = 1$}\abplz{6}

\end{bbwAufgabenBlock}
\noTRAINER{\newpage}



\subsection{Erst umformen, dann die Wurzel}%%$\sqrt{\,\vphantom{b}}$}
Lösen Sie die folgenden Potenzgleichungen, indem Sie diese zuerst auf eine einfache Form (wie oben) bringen.

Berechnen Sie die Lösungen von Hand. Wurzeln, die nicht aufgehen, lassen Sie stehen.


\begin{bbwAufgabenBlock}

\item
 $x^4-162 = 0$ \hspace{10mm} $\lx=\LoesungsRaumLang{\{-\sqrt[4]{162}; +\sqrt[4]{162}\}}$
 \TRAINER{\\$x^4 = 162 \Longrightarrow x
= \pm\sqrt[4]{162}$}\abplz{6}

\item
 $(x-3)^4 - 16 = 0$ \hspace{10mm} $\lx=\LoesungsRaumLang{\{1; 5\}}$
 \TRAINER{\\$\textrm{1. + 16; 2. } \pm \sqrt[4]{\,\,}; \,\, \Longrightarrow x-3 =\pm 2; x=3\pm 2$}\abplz{6}\noTRAINER{\newpage}

\item
 $5(x-2)^3 - 135 = 0$ \hspace{10mm} $\lx=\LoesungsRaumLang{\{5\}}$
 \TRAINER{\\$(x-2)^3 = 27$}\abplz{6}

\item
 $3(x+2)^3 + 24 = 0$ \hspace{10mm} $\lx=\LoesungsRaumLang{\{-4\}}$
 \TRAINER{\\$x+2 = -\sqrt[3]{8}$}\abplz{6}

\item
 $15-(2-x)^6 = 5$ \hspace{10mm} $\lx=\LoesungsRaumLang{\{2-\sqrt[6]{10}; 2+\sqrt[6]{10}\}}$
 \TRAINER{\\$10=(2-x)^6\Longrightarrow
x=2\pm \sqrt[6]{10} $}\abplz{6}

\noTRAINER{\newpage}

\item
$(x^2+x-4)^7 - 28 = 100$  \hspace{10mm} $\lx=\LoesungsRaumLang{\{ -3; 2  \}}$
 \TRAINER{\\$(x^2+x-4)^7 = 128\Longrightarrow x^2+x-4 = 2 \Longrightarrow x^2+x-6=0 \Longrightarrow (x+3)(x-2) = 0 $}\abplz{6}

\end{bbwAufgabenBlock}
\noTRAINER{\newpage}

\subsection{Längere Umformungen}

Die folgenden Aufgaben geben noch etwas mehr zum Umformen. Bestimmen
Sie die Lösungsmenge für die Variable $x$ (berechnen Sie die
Lösungsmenge wieder von Hand und lassen Sie irrationale Wurzeln stehen):

\begin{bbwAufgabenBlock}
\item
 $\frac{x^7-5}{2} + \frac34 = 1$ \hspace{10mm} $\lx=\LoesungsRaumLang{\{\sqrt[7]{\frac{11}2}\}}$
 \TRAINER{\\$|\cdot{}2\\
 x^7-5 +\frac32 = 2 |-\frac32\\
 x^7-5=0.5|+5\\
 x^7=5.5=\frac{11}2$}\abplz{6}

\item
 $(x+1)^3 = 3\cdot{}(x+1)^2 - 3x - 4x^3 + 4$ \hspace{10mm} $\lx=\LoesungsRaumLang{\{\sqrt[3]{\frac65\}}}$
 \TRAINER{\\$\text{\\binome ausmultiplizieren:}\\
 x^3+3x^2+3x+1 = 3(x^2+2x+1) - 3x -4x^3+4\\
 x^3  + 3x^2 + 3x + 1 = 3x^2 + 6x + 3 -3x -4x^3 +4 | \text{ (zusammenfassen) }\\
 5x^3 = 6 | : 5\\
 x^3 =\frac65$}\abplz{6}

\end{bbwAufgabenBlock}
\newpage

\section{Spezielle Exponenten}
\subsection{Negative Exponenten}

Berechnen Sie die Lösungsmengen von Hand und lassen Sie allfällige
Wurzeln stehen.

\begin{bbwAufgabenBlock}
\item
 $x^{-3}=7$ \hspace{10mm} $\lx=\LoesungsRaumLang{\{\sqrt[3]{\frac17}\}}$
 \TRAINER{\\
 $\frac{1}{x^3} = 7 | \text{(Kehrwert)}\\
 x^3=\frac17
 $}\abplz{6}

\item
 $x^{-6}+5 = 10$ \hspace{10mm} $\lx=\LoesungsRaumLang{\{-\sqrt[6]{\frac15}; +\sqrt[6]{\frac15}\}}$
 \TRAINER{\\$x=\pm\sqrt[6]{\frac15}$}\abplz{6}
\noTRAINER{\newpage}


\item
 $(x-8)^{-3} - 6 = 1$ \hspace{10mm} $\lx=\LoesungsRaumLang{\{8+\sqrt[3]{\frac17}\}}$
 \TRAINER{\\+6\\
 $(x-8)^{-3}=7 | (\text{ Kehrwert } )\\
  (x-8)^3 = \frac17 | \sqrt[3]{}\\
  x-8 = \sqrt[3]{\frac17} | +8
  $}\abplz{6}


\item
 $(x+4)^{-2} + 3 = 12$ \hspace{10mm} $\lx=\LoesungsRaumLang{\left\{-\frac{13}3; -\frac{11}3\right\}}$
 \TRAINER{\\$(x+4)^2 = \frac19\\
 x = -4 \pm \sqrt{\frac19}$}\abplz{6}

\end{bbwAufgabenBlock}
\noTRAINER{\newpage}



\subsection{Rationale Exponenten}

\begin{bbwAufgabenBlock}
\item
 $x^\frac14 = 6$ \hspace{10mm} $\lx=\LoesungsRaumLang{\{1296\}}$
 \TRAINER{\\$x=6^4$}\abplz{6}

\item
 $(x-3)^\frac15=3$ \hspace{10mm} $\lx=\LoesungsRaumLang{\{246\}}$  
 \TRAINER{\\$x=3+3^5$}\abplz{6}
\noTRAINER{\newpage}

\item
 $(x-5)^{\frac13} +4 = 5$ \hspace{10mm} $\lx=\LoesungsRaumLang{\{6 \}}$
 \TRAINER{\\$(x-5)^{\frac13} +4 = 5 | -4\\
   (x-5)^{\frac13} = 1 | X^3\\
   x-5 = 1^3 
$}\abplz{6}

\end{bbwAufgabenBlock}



\end{document}
