\input{bbwLayoutPage}


%%%%%%%%%%%%%%%%%%%%%%%%%%%%%%%%%%%%%%%%%%%%%%%%%%%%%%%%%%%%%%%%%%

\usepackage{amssymb} %% für \blacktriangleright
\renewcommand{\metaHeaderLine}{Potenzgleichungen}
\renewcommand{\arbeitsblattTitel}{(BMS)}

\begin{document}%%
\arbeitsblattHeader{}

\newcounter{aufgabennummer}
\setcounter{aufgabennummer}{1}

\newcommand\aufgabeML[2]{
Aufgabe \arabic{aufgabennummer}:\,\,
${#1} \Longrightarrow \noTRAINER{\lx = \{ } \TRAINER{{#2}}$

\stepcounter{aufgabennummer}
}
{\tiny Nach einem Aufgabenblatt von M. Rohner (BBW)}


\section{Ganzzahlige Exponenten}

\subsection{Mehrere Lösungen ?}
Bestimmen Sie die Lösungsmenge für die Variable $x$:

\aufgabeML{x^4=81}{x = \pm\sqrt[4]{81} \Longrightarrow \lx = \{-3; 3\}}

\aufgabeML{4x^6=256}{|: 4:  x^6 = 64 \Longrightarrow x = \pm\sqrt[6]{64} \Longrightarrow \lx=\{-2; 2\}}

\aufgabeML{-x^4=16}{|: \cdot{}(-1) : x^4 =
-16 \Longrightarrow \lx=\{\} \textrm{ keine Wurzeln aus
 negativen Zahlen}}

\aufgabeML{x^3=64}{x = +\sqrt[3]{64} \Longrightarrow \lx=\{+4\}}

\aufgabeML{-x^3=27}{|: \cdot{}(-1) \Longrightarrow x^3 =
 -27 \Longrightarrow x = -\sqrt[3]{27} \Longrightarrow \lx=\{-3\}}

\aufgabeML{\frac56 x^3 = \frac{50}{60}}{\textrm{ kürzen } \frac56 x^3
= \frac56 \textrm{ durch } \frac56 \textrm{ teilen: } x^3 =
1 \Longrightarrow \lx = \{1\}}


\platzFuerBerechnungenBisEndeSeite{}
\noTRAINER{\newpage}




\subsection{Erst umformen}
Lösen Sie die folgenden Potenzgleichungen, indem Sie diese zuerst auf eine einfache Form (wie oben) bringen.

Berechnen Sie die Lösungen von Hand. Wurzeln, die nicht aufgehen
lassen Sie stehen.

\aufgabeML{x^4-162 = 0}{x^4 = 162 \Longrightarrow x
= \pm\sqrt[4]{162}, \lx = \{-\sqrt[4]{162}; +\sqrt[4]{162}\} }
\aufgabeML{(x-3)^4 - 16 = 0}{\textrm{1. + 16; 2. } \pm \sqrt[4]{16} : x-3=\pm 2; x=3\pm 2 \Longrightarrow  \lx=\{1; 5\}}
\aufgabeML{5(x-2)^3 - 135 = 0}{(x-2)^3 = 27 \Longrightarrow \lx = \{5\}}
\aufgabeML{3(x+2)^3 + 24 = 0}{x+2 = -\sqrt[3]{8} \Longrightarrow \lx = \{-4\}}
\aufgabeML{15-(2-x)^6 = 5}{10=(2-x)^6\Longrightarrow
x=2\pm \sqrt[6]{10} \Longrightarrow \lx= \{2-\sqrt[6]{10}; 2+\sqrt[6]{10}\}}

\platzFuerBerechnungenBisEndeSeite{}

\noTRAINER{\newpage}

\subsection{Längere Umformungen}
Die folgenden Aufgaben geben noch etwas mehr zum Umformen. Bestimmen
Sie die Lösungsmenge für die Variable $x$ (berechnen Sie die
Lösungsmenge wieder von Hand und lassen Sie irrationale Wurzeln stehen):

\aufgabeML{\frac{x^7-5}{2} + \frac34 = 1}{x=\sqrt[7]{\frac{11}2}}
\aufgabeML{(x+1)^3 = 3\cdot{}(x+1)^2 - 3x - 4x^3 + 4}{x=\sqrt[3]{\frac65}}
\platzFuerBerechnungenBisEndeSeite{}
\newpage

\section{Spezielle Exponenten}
\subsection{Negative Exponenten}

Berechnen Sie die Lösungsmengen von Hand und lassen Sie allfällige
Wurzeln stehen.

\aufgabeML{x^{-3}=7}{x=\sqrt[3]{\frac17}}
\aufgabeML{x^{-6}+5 = 10}{x=\pm\sqrt[6]{\frac15} = \lx = \{-\sqrt[6]{\frac15}; +\sqrt[6]{\frac15}\}}
\aufgabeML{(x-8)^{-3} - 6 = 1}{x=8+\sqrt[3]{\frac17}}
\aufgabeML{(x+4)^{-2} + 3 = 12}{x = -4 \pm \sqrt{\frac19} \Longrightarrow \lx=\left\{-\frac{13}3; -\frac{11}3\right\}}
\platzFuerBerechnungenBisEndeSeite{}

\noTRAINER{\newpage}
\subsection{Rationale Exponenten}
\aufgabeML{x^\frac14 = 6}{x=6^4 = 1296}
\aufgabeML{(x-3)^\frac15=3}{x=3+3^5 = 246}
\aufgabeML{(x-5)^\frac13 +4 = 5}{x=6}
\platzFuerBerechnungenBisEndeSeite{}


\end{document}
