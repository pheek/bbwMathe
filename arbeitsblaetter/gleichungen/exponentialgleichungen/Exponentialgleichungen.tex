%%
%% Meta: TI nSpire Einführung
%%       Ziel: Damit die Grundoperationen damit durchgeführt werden können.
%%             Damit man sich an den Rechner gewöhnt.
%%

\input{bbwLayoutPage}
\renewcommand{\bbwAufgabenBlockID}{GL\_Ex}

%%%%%%%%%%%%%%%%%%%%%%%%%%%%%%%%%%%%%%%%%%%%%%%%%%%%%%%%%%%%%%%%%%

\usepackage{amssymb} %% für \blacktriangleright

\renewcommand{\metaHeaderLine}{Gleichungen Arbeitsblatt}
\renewcommand{\arbeitsblattTitel}{Exponentialgleichungen}

\begin{document}%%
\arbeitsblattHeader{}

\section{Exponentenvergleich}

Lösen Sie durch Exponentenvergleich:

\begin{bbwAufgabenBlock}
\item $5^{2x+1} = 5^{x-2}$                        $\Longrightarrow$ $\lx=\LoesungsRaumLang{\{-3\}}$
\item $7^x \cdot 7^{12} = 7^{21}$                 $\Longrightarrow$ $\lx=\LoesungsRaumLang{\{9\}}$
\item $(r^x)^3 \cdot{} r^2 = r^{x+5} \cdot{} r^x$ $\Longrightarrow$ $\lx=\LoesungsRaumLang{\{3\}}$
\item $a^4 = \frac{a^{3x}}{a^{-6}}$ $\Longrightarrow$ $\lx=\LoesungsRaumLang{\{-\frac23\}}$
\item $(f^{-7})^x \cdot{} f^3 = \frac{1}{f^{2x}} : \frac{f^{-1}}{f^{-x}}$ $\Longrightarrow$ $\lx=\LoesungsRaumLang{\{\frac12\}}$
\item $2^x=32$ $\Longrightarrow$ $\lx=\LoesungsRaumLang{\{5\}}$
\item $10^x=0.000\,01$ $\Longrightarrow$ $\lx=\LoesungsRaumLang{\{-5\}}$
\item $10^{3x}=\sqrt{10}$ $\Longrightarrow$ $\lx=\LoesungsRaumLang{\{\frac16\}}$
\item $10^{3x-8}=1$ $\Longrightarrow$ $\lx=\LoesungsRaumLang{\{\frac83\}}$

\item $3^{\frac{x-1}{3}} = \frac1{9^{\frac{x+1}{4}}}$ $\Longrightarrow$ $\lx=\LoesungsRaumLang{\{-\frac15\}}$
\end{bbwAufgabenBlock}

\platzFuerBerechnungenBisEndeSeite{}
%%%%%%%%%%%%%%%%%%%%%%%%%%%%%%%%%%%%%%%%%%%%%%%%%%%%%%%%%%%%%%%%%%%%%%%%%%%%%%%%%%%%%%%%%%%%%%

\section{Logarithmieren mit dem Taschenrechner}
Lösen Sie folgende Gleichung durch Logarithmieren und geben Sie die
Resultate einerseits mit Hilfe von Zehnerlogarithmen, andererseits mit
Hilfe des Taschenrechners auf drei signifikante Stellen an.

\begin{bbwAufgabenBlock}
\item $10^{x-4}=1000$ $\Longrightarrow$ $\lx=\LoesungsRaumLang{\{7\}}$
\item $2^x=30$ $\Longrightarrow$ $\lx=\LoesungsRaumLang{\{\frac{\lg(30)}{\lg(2)}\approx 4.91}\}$
\item $13^x=1000$ $\Longrightarrow$ $\lx=\LoesungsRaumLang{\{\frac{\lg(1000)}{\lg(13)}\approx 2.69\}}$
\item $5\cdot{}7^x = 19$ $\Longrightarrow$ $\lx=\LoesungsRaumLang{\{\frac{\lg(19)-\lg(5)}{\lg(7)}\approx 0.686\}}$
\item $7\cdot{}3^x = 2 + 4\cdot{}3^x$ $\Longrightarrow$ $\lx=\LoesungsRaumLang{\{\frac{\lg(2)-\lg(3)}{\lg(3)}\approx  \}}$
\item $3\cdot{}6^x - 36 = 6^x$ $\Longrightarrow$ $\lx=\LoesungsRaumLang{\{\frac{\lg(18)}{\lg(6)} \approx  1.61 \}}$
\item $2^{5x} = 5^{2x}$ $\Longrightarrow$ $\lx=\LoesungsRaumLang{\{0\}}$
\end{bbwAufgabenBlock}

\platzFuerBerechnungenBisEndeSeite{}
%%%%%%%%%%%%%%%%%%%%%%%%%%%%%%%%%%%%%%%%%%%%%%%%%%%%%%%%%%%%%%%%%%%%%%%%%%%%%%%%%%%%%%%%%%%%%%

\section{Wenden Sie Logarithmengesetze an}

$$\log(a^n) = n\cdot{}\log(a)$$
$$\log(A\cdot{}B) = \log(A) + \log(B)$$

Geben Sie auch in folgenden Aufgaben die Lösungen mit Hilfe von
Zehnerlogarithmen, aber auch mit Hilfe des Taschenrechners auf drei
signifikante Stellen an.

\begin{bbwAufgabenBlock}
\item $3^x=2^{x+2}$ $\Longrightarrow$ $\lx=\LoesungsRaumLang{\{\frac{2\cdot{}\lg(2)}{\lg(3)-\lg(2)} \approx  3.42  \}}$
\item $7^x=8\cdot{}6^{x+1}$ $\Longrightarrow$ $\lx=\LoesungsRaumLang{\{\frac{\lg(48)}{\lg(7)-\lg(6)} \approx 25.1   \}}$
\item $2\cdot{}3^{x+7}=4\cdot{}6^{x+5}$ $\Longrightarrow$ $\lx=\LoesungsRaumLang{\{\frac{\lg(\frac{4\cdot{}6^5}{2\cdot{}3^7})}{\lg(\frac12)} \approx  -2.83  \}}$
\item $5^{x-1}\cdot{}2^{1-x}=6^{2x}$ $\Longrightarrow$ $\lx=\LoesungsRaumLang{\{\frac{\lg(5)-\lg(2)}{\lg(5)-\lg(2)-2\lg(6)} \approx -0.344   \}}$
\item $10\cdot{}3^{4x} = 6^{1-x}\cdot{}4^{3x}$ $\Longrightarrow$ $\lx=\LoesungsRaumLang{\{\frac{\lg(6)-\lg(10)}{\lg(6\cdot{}3^4 - \lg(4^3))} \approx -0.252   \}}$

\end{bbwAufgabenBlock}

\platzFuerBerechnungenBisEndeSeite{}
%%%%%%%%%%%%%%%%%%%%%%%%%%%%%%%%%%%%%%%%%%%%%%%%%%%%%%%%%%%%%%%%%%%%%%%%%%%%%%%%%%%%%%%%%%%%%%

\section{Logarithmus Naturalis}
Geben Sie die Lösung einerseits mit dem «Logarithmus Naturalis»
($=\ln() = \log_{\e}()$) an, berechnen Sie zur Sicherheit aber auch das
Resultat auf drei signifikante Stellen.

\begin{bbwAufgabenBlock}
\item $4^x=7^{x+2}$ $\Longrightarrow$ $\lx=\LoesungsRaumLang{\{\frac{2\cdot{}\ln(7)}{\ln(4)-\ln(7)} \approx -6.95   \}}$
\item $\left(\frac14\right)^{3x-1} = \left(\frac15\right)^{2-x}$ $\Longrightarrow$ $\lx=\LoesungsRaumLang{\{\frac{-2\ln(5)-\ln(4)}{-3\ln(4)-\ln(5)} \approx 0.798   \}}$
\item $12^x+4\cdot{}3^{4-x}=0$ $\Longrightarrow$
$\lx=\LoesungsRaumLang{\{   \}}$ 
\item $2^{x+1}=5^x + 5^{x-1}$ $\Longrightarrow$ $\lx=\LoesungsRaumLang{\{\frac{\ln(3)-\ln(5)}{\ln(2)-\ln(5)} \approx 0.554   \}}$
\item $2^r=3\cdot{}2^{r-3} + 12\cdot{}5^{r-3} - 2\cdot{}5^{r-2}$ $\Longrightarrow$ $\lx=\LoesungsRaumLang{\{ 4 \}}$
\item $5^x + 5^{x+2} = 2600$ $\Longrightarrow$ $\lx=\LoesungsRaumLang{\{\frac{\ln(100)}{\ln(5)} \approx 2.86   \}}$
\item $7^{2x+1} = 100 + 7^{2x-1}$ $\Longrightarrow$ $\lx=\LoesungsRaumLang{\{\frac{\ln(100) - \ln(7-\frac17)}{2\cdot{}\ln(7)} \approx  0.689  \}}$
\item $e^x + e^{x+1} + e^{x+2} = 1$ $\Longrightarrow$ $\lx=\LoesungsRaumLang{\{\left(\frac{1}{1+e+e^2}\right) \approx -2.41  \}}$

\end{bbwAufgabenBlock}

\platzFuerBerechnungenBisEndeSeite{}
%%%%%%%%%%%%%%%%%%%%%%%%%%%%%%%%%%%%%%%%%%%%%%%%%%%%%%%%%%%%%%%%%%%%%%%%%%%%%%%%%%%%%%%%%%%%%%

\section{Substitution}
Die folgenden Exponentialgleichungen können mit Hilfe einer geeigneten
Substitution auf quadratische Gleichungen zurückgeführt werden.

\begin{bbwAufgabenBlock}
\item $4^x + 4^{2x} = 20$ $\Longrightarrow$ $\lx=\LoesungsRaumLang{\{  1   \}}$
\item $2\cdot{}5^{2r} = 3.6\cdot{}5^r - 1$ $\Longrightarrow$
$\lx=\LoesungsRaumLang{\left\{   \frac{\ln\left(\frac{9\pm\sqrt{31}}{10}\right)}{\ln(5)} \approx
0.234; -0.664 \right\}
}$
\item $10^x - 10^{-x} = 1$ $\Longrightarrow$ $\lx=\LoesungsRaumLang{\left\{  \frac{\ln\left(\frac{1+\sqrt{5}}{2}\right)}{\ln(10)} \approx  0.209   \right\}}$

\end{bbwAufgabenBlock}

\platzFuerBerechnungenBisEndeSeite{}
%%%%%%%%%%%%%%%%%%%%%%%%%%%%%%%%%%%%%%%%%%%%%%%%%%%%%%%%%%%%%%%%%%%%%%%%%%%%%%%%%%%%%%%%%%%%%%
\section{Textaufgaben}

Ein Kapital von CHF $22\,000.-$ wird zu $3.5\%$ Jahreszins angelegt.
\begin{bbwAufgabenBlock}
\item Auf welchen Betrag ist das Kapital nach 7 Jahren
angewachsen?\TRAINER{ 27 990.14 CHF}
\item Auf welchen Betrag ist das Kapital nach 50 Jahren
angewachsen? \TRAINER{ 122868.39 CHF}
\item Auf welchen Betrag ist das Kapital nach $n$ Jahren
angewachsen? \TRAINER{  $22\,000\cdot{} 1.035^n$}
\item Nach wie vielen Jahren wird das Kapital auf $31\,000.-$ CHF
angewachsen sein? \TRAINER{ 9.97 Jahre}
\end{bbwAufgabenBlock}
\platzFuerBerechnungenBisEndeSeite{}
%%%%%%%%%%%%%%%%%%%%%%%%%%%%%%%%%%%%%%%%%%%%%%%%%%%%%%%%%%%%%%%%%%%%%%%%%%%%%%%%%%%%%%%%%%%%%%

Ein Kapital von CHF $13\,000.-$ wird zu $0.3\%$ jährlichem Zins angelegt.
\begin{bbwAufgabenBlock}
\item Auf welchen Betrag ist das Kapital nach 4 Jahren
angewachsen? \TRAINER{ 13\,156.70 CHF}
\item Auf welchen Betrag ist das Kapital nach 10 Jahren
angewachsen? \TRAINER{ 13\,395.31 CHF}
\item Auf welchen Betrag ist das Kapital nach $n$ Jahren
angewachsen? \TRAINER{ $13\,000\cdot{} 1.003^n$}
\item Nach wie vielen Jahren wird das Kapital auf $15\,000.-$ CHF
angewachsen sein? \TRAINER{ Nach 47.8 Jahren.}
\end{bbwAufgabenBlock}
\platzFuerBerechnungenBisEndeSeite{}
%%%%%%%%%%%%%%%%%%%%%%%%%%%%%%%%%%%%%%%%%%%%%%%%%%%%%%%%%%%%%%%%%%%%%%%%%%%%%%%%%%%%%%%%%%%%%%
Ein Computer verliert jedes Jahr um 21\% an Wert. Anfänglich wurde ein Modell «X»
zum Neupreis von 9 870.- eingekauft.

\begin{bbwAufgabenBlock}
\item  Wie viel Wert hat der Computer (Modell «X») nach 3 Jahren
noch? \TRAINER{ 4866.29 CHF }
\item Wie viel Wert hat der Computer nach 15 Jahren noch? \TRAINER{
287.56 CHF}
\item Wie viel Wert hat der Computer nach n Jahren? \TRAINER{ $ 0.79^n \cdot 9870.-$ }
\item Wann ist der Wert des Computers auf einen Wert von 10\% des
Neuwertes gesunken? \TRAINER{ 9.77 Jahre}
\item Wann hat der Wert des Computer 60\% des Neuwertes
verloren? \TRAINER{ 3.89 Jahre}
\end{bbwAufgabenBlock}

\platzFuerBerechnungenBisEndeSeite{}
%%%%%%%%%%%%%%%%%%%%%%%%%%%%%%%%%%%%%%%%%%%%%%%%%%%%%%%%%%%%%%%%%%%%%%%%%%%%%%%%%%%%%%%%%%%%%%
Ein Wäldchen wächst jedes Jahr um 15\% Waldfläche. (Die anfängliche Fläche werde mit
100\% bezeichnet.)
\begin{bbwAufgabenBlock}
\item Auf wie viel \% ist der Wald nach 5 Jahren angewachsen? \TRAINER{ 201\%}
\item Um wie viel \% ist der Wald nach 5 Jahren angewachsen? \TRAINER{ 101\%}
\item Auf wie viel \% ist der Wald nach n Jahren
angewachsen? \TRAINER{ $1.15^n [\cdot{} 100\%]$}
\item Nach wie vielen Jahren hat sich die Waldfläche
verdreifacht? \TRAINER{ 7.86 Jahre}
\item Nach wie vielen Jahren sind 100\% Waldfläche
dazugekommen? \TRAINER{nach 4.69 Jahren }
\end{bbwAufgabenBlock}

\platzFuerBerechnungenBisEndeSeite{}
\end{document}
