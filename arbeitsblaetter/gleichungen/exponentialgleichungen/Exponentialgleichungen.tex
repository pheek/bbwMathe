%%
%% Meta: TI nSpire Einführung
%%       Ziel: Damit die Grundoperationen damit durchgeführt werden können.
%%             Damit man sich an den Rechner gewöhnt.
%%

\input{bmsLayoutPage}
\renewcommand{\bbwAufgabenBlockID}{GL\_Ex}

%%%%%%%%%%%%%%%%%%%%%%%%%%%%%%%%%%%%%%%%%%%%%%%%%%%%%%%%%%%%%%%%%%

\usepackage{amssymb} %% für \blacktriangleright

\renewcommand{\metaHeaderLine}{Arbeitsblatt}
\renewcommand{\arbeitsblattTitel}{Exponentialgleichungen}


\ifisNURAUFGABEN
\renewcommand{\abplz}[1]{\vspace{1mm}}
\fi


\newcommand{\seitenUmbruchImAufgabenteil}{
\ifisNURAUFGABEN
\else
\noTRAINER{\newpage}
\fi
}%%



\begin{document}%%
\arbeitsblattHeader{} (V 1.7 2025-10-18)

\section{Exponentenvergleich}

% \newcommand{\plz}{\noTRAINER{\\ \mmPapier{5.2}}}

Lösen Sie durch Exponentenvergleich:

\begin{bbwAufgabenBlock}
\item $5^{2x+1} = 5^{x-2}$                        $\Longrightarrow$ $\lx=\LoesungsRaumLang{\{-3\}}$ \abplz{7.2}
\item $7^x \cdot 7^{12} = 7^{21}$                 $\Longrightarrow$ $\lx=\LoesungsRaumLang{\{9\}}$ \abplz{7.2}\noTRAINER{\seitenUmbruchImAufgabenteil{}}
\item $(r^x)^3 \cdot{} r^2 = r^{x+5} \cdot{} r^x$ $\Longrightarrow$ $\lx=\LoesungsRaumLang{\{3\}}$ \abplz{7.2}
\item $a^4 = \frac{a^{3x}}{a^{-6}}$ $\Longrightarrow$ $\lx=\LoesungsRaumLang{\{-\frac23\}}$ \abplz{7.2}\noTRAINER{\seitenUmbruchImAufgabenteil{}}
\item $(f^{-7})^x \cdot{} f^3 = \frac{1}{f^{2x}} : \frac{f^{-1}}{f^{-x}}$ $\Longrightarrow$ $\lx=\LoesungsRaumLang{\{\frac12\}}$ \abplz{7.2}
\item $2^x=32$ $\Longrightarrow$ $\lx=\LoesungsRaumLang{\{5\}}$ \abplz{7.2}\noTRAINER{\seitenUmbruchImAufgabenteil{}}
\item $10^x=0.000\,01$ $\Longrightarrow$ $\lx=\LoesungsRaumLang{\{-5\}}$ \abplz{7.2}
\item $10^{3x}=\sqrt{10}$ $\Longrightarrow$ $\lx=\LoesungsRaumLang{\{\frac16\}}$ \abplz{7.2}\noTRAINER{\seitenUmbruchImAufgabenteil{}}
\item $10^{3x-8}=1$ $\Longrightarrow$ $\lx=\LoesungsRaumLang{\{\frac83\}}$ \abplz{7.2}
\item $10^{x-2}=1\,000$ $\Longrightarrow$ $\lx=\LoesungsRaumLang{\{5\}}$ \abplz{7.2}\noTRAINER{\seitenUmbruchImAufgabenteil{}}

\item $3^{\frac{x-1}{3}} = \frac1{9^{\frac{x+1}{4}}}$ $\Longrightarrow$ $\lx=\LoesungsRaumLang{\{-\frac15\}}$ \abplz{7.2}
\item $9^x = 8\cdot{}3^x + 3^x$ $\Longrightarrow$ $\lx=\LoesungsRaumLang{\{2\}}$ \abplz{7.2}

\item $10^{x-4}=1000$ $\Longrightarrow$ $\lx=\LoesungsRaumLang{\{7\}}$ \abplz{7.2}

\end{bbwAufgabenBlock}
\newpage
%
%%%%%%%%%%%%%%%%%%%%%%%%%%%%%%%%%%%%%%%%%%%%%%%%%%%%%%%%%%%%%%%%%%%%%%%%%%%%%%%%%%%%%%%%%%%%%%
%
\section{Logarithmieren mit dem Taschenrechner}
Lösen Sie folgende Gleichung durch Logarithmieren und geben Sie die
Resultate einerseits exakt (Brüche, Wurzeln, Logarithmen/Zehnerlogarithmen), andererseits mit
Hilfe des Taschenrechners auf drei oder vier signifikante Stellen an.

Formel
$$\log_b(a) = \frac{\lg(a)}{\lg(b)}$$

\begin{bbwAufgabenBlock}

\item $10^x=1001$ $\Longrightarrow$
$\lx=\LoesungsRaumLang{\{\frac{\lg(1001)}{\lg(10)} = \frac{\lg(1001)}{1} = \lg(1001) \approx 3.00\}}$ \abplz{7.2}\noTRAINER{\seitenUmbruchImAufgabenteil{}}

\item $2^x=30$ $\Longrightarrow$ $\lx=\LoesungsRaumLang{\{\frac{\lg(30)}{\lg(2)}\approx 4.91\}}$ \abplz{7.2}\noTRAINER{\seitenUmbruchImAufgabenteil{}}
\item $13^x=1000$ $\Longrightarrow$ $\lx=\LoesungsRaumLang{\{\frac{\lg(1000)}{\lg(13)}\approx 2.69\}}$ \abplz{7.2}
\item $5\cdot{}7^x = 19$ $\Longrightarrow$ $\lx=\LoesungsRaumLang{\{\frac{\lg(19)-\lg(5)}{\lg(7)}\approx 0.686\}}$ \abplz{7.2}\noTRAINER{\seitenUmbruchImAufgabenteil{}}
\item $7\cdot{}3^x = 2 + 4\cdot{}3^x$ $\Longrightarrow$ $\lx=\LoesungsRaumLang{\{\frac{\lg(2)-\lg(3)}{\lg(3)}\approx  \}}$ \abplz{7.2}
\item $3\cdot{}6^x - 36 = 6^x$ $\Longrightarrow$ $\lx=\LoesungsRaumLang{\{\frac{\lg(18)}{\lg(6)} \approx  1.61 \}}$ \abplz{7.2}\noTRAINER{\seitenUmbruchImAufgabenteil{}}
\item $2^{5x} = 5^{2x}$ $\Longrightarrow$ $\lx=\LoesungsRaumLang{\{0\}}$ \abplz{7.2}
\end{bbwAufgabenBlock}

\newpage

\section{Form $a^{T} = b$}
Bringen Sie die folgenden Gleichungen erst in die Form $a^T=b$, wobei
$T=T(x)$ ein Term in $x$ ist.

\begin{bbwAufgabenBlock}
\item $8\cdot{} 5^{2x+1} = 200$  $\Longrightarrow$ $\lx=\LoesungsRaumLang{\{ 0.5 \}}$ \abplz{7.2}


\item $\frac12 \cdot{} 4^{7x-32} = 32$  $\Longrightarrow$ $\lx=\LoesungsRaumLang{\{ 5 \}}$ \abplz{7.2}\noTRAINER{\seitenUmbruchImAufgabenteil{}}

Lösen Sie exakt von Hand und runden Sie das Resultat anschließend mit dem Taschenrechner auf vier signifikante Stellen.

\item $3.4^{\frac{x}2} = 16$  $\Longrightarrow$
$\lx=\LoesungsRaumLang{\{2\cdot{} \frac{\lg(16)}{\lg(3.4)} \approx 4.531 \}}$ \abplz{7.2}
\noTRAINER{\seitenUmbruchImAufgabenteil{}}
\item $6^{\frac{x}3} = 14$  $\Longrightarrow$
$\lx=\LoesungsRaumLang{\{3\cdot{} \frac{\lg(14)}{\lg(6)} \approx 4.419 \}}$ \abplz{7.2}

\item $1.8^{\frac{x}5+1} = 100$  $\Longrightarrow$
$\lx=\LoesungsRaumLang{\{5\cdot{}\left(\frac{2}{\lg(1.8)} - 1\right) \approx 34.17 \}}$ \abplz{7.2}\noTRAINER{\seitenUmbruchImAufgabenteil{}}


\item $3 = 4 - 5\cdot{}6^\frac{x}7$  $\Longrightarrow$ $\lx=\LoesungsRaumLang{\{ 7\cdot{} \log_{6}\left(\frac{3-4}{-5}\right)  \approx  -6.288 \}}$ \abplz{7.2}

\item $8 = 7 + 6\cdot{}5^\frac{x}4$  $\Longrightarrow$ $\lx=\LoesungsRaumLang{\{ 4\cdot{} \log_{5}\left(\frac{8-7}{6}\right)  \approx -4.453 \}}$ \abplz{7.2}\noTRAINER{\seitenUmbruchImAufgabenteil{}}

\item $40 = 50 - 40\cdot{}0.8^\frac{t}7$  $\Longrightarrow$ $\mathbb{L}_t=\LoesungsRaumLang{\{ 7\cdot{} \log_{0.8}\left(\frac{40-50}{-30}\right)  \approx 43.49 \}}$ \abplz{7.2}
\\




\item $1200=3\cdot{} 5^{\frac{t}{4}} \Longrightarrow$ $L_t = \LoesungsRaumLang{\{ 4\cdot{}\log_5(\frac{1200}3)\approx 14.89   \}}$ \abplz{7.2}\noTRAINER{\seitenUmbruchImAufgabenteil{}}

\item $0.93^{\frac{t}4}=\frac12 \Longrightarrow$ $L_t
= \LoesungsRaumLang{\{
4\cdot{}\log_{0.93}\left(\frac12\right) \approx 38.205   \}}$ \abplz{7.2}

\noTRAINER{\seitenUmbruchImAufgabenteil{}}

\item $28 = 4 + 3\cdot{} 2^\frac{t}{4} \Longrightarrow$ $L_t
= \LoesungsRaumLang{\{ 4\cdot{} \log_2\left(\frac{28-4}3\right) = 12 \}}$ \abplz{7.2}


\item $4.7 = 8 - 0.6\cdot{} 1.25^{\frac{t}{3}} \Longrightarrow$ $L_t
= \LoesungsRaumLang{\{
3\cdot{} \log_{1.25}\left(\frac{4.7-8}{-0.6}\right) \approx 22.9191  \}}$ \abplz{7.2}

\noTRAINER{\seitenUmbruchImAufgabenteil{}}


Lösen Sie von Hand:

\item $5 \cdot{} 3^{\frac{7-x}{2}} = 45$  $\Longrightarrow$ $\lx=\LoesungsRaumLang{\{  3   \}}$ \abplz{7.2}

\item $39 =  6\cdot{} 2 ^{5\cdot{} \lg(x) - 7} - 9$  $\Longrightarrow$ $\lx = \LoesungsRaumLang{\{ 100 \}}$ \abplz{7.2}
\noTRAINER{\seitenUmbruchImAufgabenteil{}}

Lassen Sie das exakte Resultat stehen:

\item $5^{\lg(x)+2} = 100$  $\Longrightarrow$ $\lx=\LoesungsRaumLang{\{  10^{\log_5(100)-2}   \}}$ \abplz{7.2}



\end{bbwAufgabenBlock}

\platzFuerBerechnungenBisEndeSeite{}
\newpage
%%%%%%%%%%%%%%%%%%%%%%%%%%%%%%%%%%%%%%%%%%%%%%%%%%%%%%%%%%%%%%%%%%%%%%%%%%%%%%%%%%%%%%%%%%%%%%

\section{Verschiedene Basen}

$$\log(a^n) = n\cdot{}\log(a)$$

Geben Sie auch in folgenden Aufgaben die Lösungen exakt (Wurzeln,
Brüche, Logarithmen) an; aber auch mit Hilfe des Taschenrechners auf drei
signifikante Stellen.

\begin{bbwAufgabenBlock}
\item $3^x=2^{x+2}$ $\Longrightarrow$ $\lx=\LoesungsRaumLang{\{\frac{2\cdot{}\lg(2)}{\lg(3)-\lg(2)} \approx  3.42  \}}$ \abplz{7.2}
\item $7^x=8\cdot{}6^{x+1}$ $\Longrightarrow$ $\lx=\LoesungsRaumLang{\{\frac{\lg(48)}{\lg(7)-\lg(6)} \approx 25.1   \}}$ \abplz{7.2}\noTRAINER{\seitenUmbruchImAufgabenteil{}}
\item $2\cdot{}3^{x+7}=4\cdot{}6^{x+5}$ $\Longrightarrow$ $\lx=\LoesungsRaumLang{\{\frac{\lg(\frac{4\cdot{}6^5}{2\cdot{}3^7})}{\lg(\frac12)} \approx  -2.83  \}}$ \abplz{7.2}
\item $5^{x-1}\cdot{}2^{1-x}=6^{2x}$ $\Longrightarrow$ $\lx=\LoesungsRaumLang{\{\frac{\lg(5)-\lg(2)}{\lg(5)-\lg(2)-2\lg(6)} \approx -0.344   \}}$ \abplz{7.2}\noTRAINER{\seitenUmbruchImAufgabenteil{}}
\item $10\cdot{}3^{4x} = 6^{1-x}\cdot{}4^{3x}$ $\Longrightarrow$ $\lx=\LoesungsRaumLang{\{\frac{\lg(6)-\lg(10)}{\lg(6\cdot{}3^4 - \lg(4^3))} \approx -0.252   \}}$ \abplz{7.2}
\end{bbwAufgabenBlock}

%\platzFuerBerechnungenBisEndeSeite{}
\noTRAINER{\seitenUmbruchImAufgabenteil{}}
%%%%%%%%%%%%%%%%%%%%%%%%%%%%%%%%%%%%%%%%%%%%%%%%%%%%%%%%%%%%%%%%%%%%%%%%%%%%%%%%%%%%%%%%%%%%%%

\section{Logarithmus Naturalis}
Geben Sie die Lösung einerseits mit dem «Logarithmus Naturalis»
($=\ln() = \log_{\e}()$) an, berechnen Sie zur Sicherheit aber auch das
Resultat auf drei signifikante Stellen.

\begin{bbwAufgabenBlock}
\item $4^x=7^{x+2}$ $\Longrightarrow$ $\lx=\LoesungsRaumLang{\{\frac{2\cdot{}\ln(7)}{\ln(4)-\ln(7)} \approx -6.95   \}}$ \abplz{7.2}
\item $\left(\frac14\right)^{3x-1} = \left(\frac15\right)^{2-x}$ $\Longrightarrow$ $\lx=\LoesungsRaumLang{\{\frac{-2\ln(5)-\ln(4)}{-3\ln(4)-\ln(5)} \approx 0.798   \}}$ \abplz{7.2}\noTRAINER{\seitenUmbruchImAufgabenteil{}}
\item $12^x+4\cdot{}3^{4-x}=0$ $\Longrightarrow$
$\lx=\LoesungsRaumLang{\{   \}}$  \abplz{7.2} 
\item $2^{x+1}=5^x + 5^{x-1}$ $\Longrightarrow$ $\lx=\LoesungsRaumLang{\{\frac{\ln(3)-\ln(5)}{\ln(2)-\ln(5)} \approx 0.554   \}}$ \abplz{7.2} \noTRAINER{\seitenUmbruchImAufgabenteil{}}
\item $2^r=3\cdot{}2^{r-3} + 12\cdot{}5^{r-3} - 2\cdot{}5^{r-2}$ $\Longrightarrow$ $\lx=\LoesungsRaumLang{\{ 4 \}}$ \abplz{7.2}
\item $5^x + 5^{x+2} = 2600$ $\Longrightarrow$ $\lx=\LoesungsRaumLang{\{\frac{\ln(100)}{\ln(5)} \approx 2.86   \}}$ \abplz{7.2} \noTRAINER{\seitenUmbruchImAufgabenteil{}}
\item $7^{2x+1} = 100 + 7^{2x-1}$ $\Longrightarrow$ $\lx=\LoesungsRaumLang{\{\frac{\ln(100) - \ln(7-\frac17)}{2\cdot{}\ln(7)} \approx  0.689  \}}$ \abplz{7.2}
\item $\e^x + \e^{x+1} + \e^{x+2} = 1$ $\Longrightarrow$ $\lx=\LoesungsRaumLang{\{\left(\frac{1}{1+\e+\e^2}\right) \approx -2.41  \}}$ \abplz{7.2}

\end{bbwAufgabenBlock}

\newpage{}%%%\platzFuerBerechnungenBisEndeSeite{}
%%%%%%%%%%%%%%%%%%%%%%%%%%%%%%%%%%%%%%%%%%%%%%%%%%%%%%%%%%%%%%%%%%%%%%%%%%%%%%%%%%%%%%%%%%%%%%

\section{Substitution}
Die folgenden Exponentialgleichungen können mit Hilfe einer geeigneten
Substitution auf quadratische Gleichungen zurückgeführt werden.

\begin{bbwAufgabenBlock}
\item $4^x + 4^{2x} = 20$ $\Longrightarrow$ $\lx=\LoesungsRaumLang{\{  1   \}}$ \abplz{7.2}
\item $2\cdot{}5^{2r} = 3.6\cdot{}5^r - 1$ $\Longrightarrow$
$\lx=\LoesungsRaumLang{\left\{   \frac{\ln\left(\frac{9\pm\sqrt{31}}{10}\right)}{\ln(5)} \approx
0.234; -0.664 \right\}
}$\abplz{7.2}

\noTRAINER{\seitenUmbruchImAufgabenteil{}}
%%\noTRAINER{\mmPapier{10}\seitenUmbruchImAufgabenteil{}}

\item $10^x - 10^{-x} = 1$ $\Longrightarrow$ $\lx=\LoesungsRaumLang{\left\{  \frac{\ln\left(\frac{1+\sqrt{5}}{2}\right)}{\ln(10)} \approx  0.209   \right\}}$ 

\end{bbwAufgabenBlock}
\platzFuerBerechnungenBisEndeSeite{}

%%\platzFuerBerechnungenBisEndeSeite{}
%%%%%%%%%%%%%%%%%%%%%%%%%%%%%%%%%%%%%%%%%%%%%%%%%%%%%%%%%%%%%%%%%%%%%%%%%%%%%%%%%%%%%%%%%%%%%%
\seitenUmbruchImAufgabenteil{}
\section{In Textform}
\subsection{Kapital I}
Ein Kapital von CHF $22\,000.-$ wird zu $3.5\%$ Jahreszins angelegt.
\begin{bbwAufgabenBlock}
\item Auf welchen Betrag ist das Kapital nach 7 Jahren
angewachsen?\TRAINER{ 27 990.14 CHF} \abplz{7.2}
\item Auf welchen Betrag ist das Kapital nach 50 Jahren
angewachsen? \TRAINER{ 122868.39 CHF} \abplz{7.2}
\item Auf welchen Betrag ist das Kapital nach $n$ Jahren
angewachsen? \TRAINER{  $22\,000\cdot{} 1.035^n$} \abplz{7.2} \noTRAINER{\seitenUmbruchImAufgabenteil{}}
\item Nach wie vielen Jahren wird das Kapital auf $31\,000.-$ CHF
angewachsen sein? \TRAINER{ 9.97 Jahre} \abplz{7.2}
\end{bbwAufgabenBlock}
\platzFuerBerechnungenBisEndeSeite{}
%%%%%%%%%%%%%%%%%%%%%%%%%%%%%%%%%%%%%%%%%%%%%%%%%%%%%%%%%%%%%%%%%%%%%%%%%%%%%%%%%%%%%%%%%%%%%%
\subsection{Kapital II}
Ein Kapital von CHF $13\,000.-$ wird zu $0.3\%$ jährlichem Zins angelegt.
\begin{bbwAufgabenBlock}
\item Auf welchen Betrag ist das Kapital nach 4 Jahren
angewachsen? \TRAINER{ 13\,156.70 CHF} \abplz{7.2}
\item Auf welchen Betrag ist das Kapital nach 10 Jahren
angewachsen? \TRAINER{ 13\,395.31 CHF} \abplz{7.2}
\item Auf welchen Betrag ist das Kapital nach $n$ Jahren
angewachsen? \TRAINER{ $13\,000\cdot{} 1.003^n$} \abplz{7.2} \noTRAINER{\seitenUmbruchImAufgabenteil{}}
\item Nach wie vielen Jahren wird das Kapital auf $15\,000.-$ CHF
angewachsen sein? \TRAINER{ Nach 47.8 Jahren.} \abplz{7.2}
\end{bbwAufgabenBlock}
\platzFuerBerechnungenBisEndeSeite{}
%%%%%%%%%%%%%%%%%%%%%%%%%%%%%%%%%%%%%%%%%%%%%%%%%%%%%%%%%%%%%%%%%%%%%%%%%%%%%%%%%%%%%%%%%%%%%%
\subsection{Abschreibung}
Ein Computer verliert jedes Jahr um 21\% an Wert. Anfänglich wurde ein Modell «X»
zum Neupreis von 9 870.- eingekauft.

\begin{bbwAufgabenBlock}
\item  Wie viel Wert hat der Computer (Modell «X») nach 3 Jahren
noch? \TRAINER{ 4866.29 CHF } \abplz{7.2}
\item Wie viel Wert hat der Computer nach 15 Jahren noch? \TRAINER{
287.56 CHF} \abplz{7.2}
\item Wie viel Wert hat der Computer nach n Jahren? \TRAINER{ $ 0.79^n \cdot 9870.-$ } \abplz{7.2} \noTRAINER{\seitenUmbruchImAufgabenteil{}}
\item Wann ist der Wert des Computers auf einen Wert von 10\% des
Neuwertes gesunken? \TRAINER{ 9.77 Jahre} \abplz{7.2}
\item Wann hat der Wert des Computer 60\% des Neuwertes
verloren? \TRAINER{ 3.89 Jahre} \abplz{7.2}
\end{bbwAufgabenBlock}

\platzFuerBerechnungenBisEndeSeite{}
%%%%%%%%%%%%%%%%%%%%%%%%%%%%%%%%%%%%%%%%%%%%%%%%%%%%%%%%%%%%%%%%%%%%%%%%%%%%%%%%%%%%%%%%%%%%%%
\subsection{Mille Feuilles}
Blätterteig heißt auf französisch «mille feuilles» (wörtlich also
«tausend Blätter»).

Um Blätterteig herzustellen wird ein Stück Teig halbiert, eingebuttert
und die beiden Hälften werden aufeinander gelegt. Das entstandene
Teigstück wird mit dem Wallholz (Nudelholz) flach ausgewallt und der
Prozess beginnt nach dem Auskühlen von vorn.

Nach dem zweiten Durchgang sind also bereits 4 Schichten übereinander.

\begin{bbwAufgabenBlock}
\item Wie viele Schichten («Blätter») sind nach fünf Durchgängen
vorhanden? \TRAINER{32 «Blätter» sind nach fünf Durchgängen
vorhanden.} \abplz{7.2}

\item Wie viele «Blätter» sind nach $n$ Durchgängen
vorhanden? \TRAINER{Nach $n$ Durchgängen sind es $2^n$ Schichten
vorhanden.} \abplz{7.2}

\item Nach wie vielen Durchgängen sind es tatsächlich tausend
Schichten, sodass «mille feuilles» seinem Namen gerecht
wird? \TRAINER{Nach 10 Durchgängen sind es 1000 (exakt 1024)
Schichten. Rechnung: $$2^n=1000 \Longrightarrow   n
= \log_2(1000) \approx 9.966 $$}

\end{bbwAufgabenBlock}

\platzFuerBerechnungenBisEndeSeite{}
%%%%%%%%%%%%%%%%%%%%%%%%%%%%%%%%%%%%%%%%%%%%%%%%%%%%%%%%%%%%%%%%%%%%%%%%%%%%%%%%%%%%%%%%%%%%%%
\TRAINER{\newpage}
\subsection{Wäldchen}
Ein Wäldchen wächst jedes Jahr um 15\% Waldfläche. (Die anfängliche Fläche werde mit
100\% bezeichnet.)
\begin{bbwAufgabenBlock}
\item Auf wie viel \% ist der Wald nach 5 Jahren angewachsen? \TRAINER{ 201\%} \abplz{7.2}
\item Um wie viel \% ist der Wald nach 5 Jahren angewachsen? \TRAINER{ 101\%} \abplz{7.2}
\item Auf wie viel \% ist der Wald nach n Jahren
angewachsen? \TRAINER{ $1.15^n [\cdot{} 100\%]$} \abplz{7.2} \noTRAINER{\seitenUmbruchImAufgabenteil{}}
\item Nach wie vielen Jahren hat sich die Waldfläche
verdreifacht? \TRAINER{ 7.86 Jahre} \abplz{7.2}
\item Nach wie vielen Jahren sind 100\% Waldfläche
dazugekommen? \TRAINER{nach 4.96 Jahren } \abplz{7.2}
\end{bbwAufgabenBlock}

\platzFuerBerechnungenBisEndeSeite{}
\end{document}
