%%
%% Meta: TI nSpire Einführung
%%       Ziel: Damit die Grundoperationen damit durchgeführt werden können.
%%             Damit man sich an den Rechner gewöhnt.
%%

\input{bmsLayoutPage}

%%%%%%%%%%%%%%%%%%%%%%%%%%%%%%%%%%%%%%%%%%%%%%%%%%%%%%%%%%%%%%%%%%

\renewcommand{\metaHeaderLine}{Quadratische Gleichungen}
\renewcommand{\arbeitsblattTitel}{Finde $a$, $b$ und $c$ für die $abc$-Formel}

\begin{document}%%
\arbeitsblattHeader{}
Finden Sie jeweils $a$, $b$ und $c$ für die quadratische
Gleichungsformel ($abc$-Formel od. Mitternachtsformel). Die
Lösungsvariable ist stets $x$.

\newcommand{\sssppp}{\noTRAINER{..........}}
\newcommand{\findeabc}[4]{$#1$ \,\,\,$\Longrightarrow$\,\,\, $a$=\sssppp\TRAINER{$#2$},
$b$=\sssppp\TRAINER{$#3$}, $c$=\sssppp\TRAINER{$#4$}

\noTRAINER{\mmPapier{2.8}}

}

\newcommand{\findeabcPlz}[4]{$#1$ \,\,\,$\Longrightarrow$\,\,\, $a$=\sssppp\TRAINER{$#2$},
$b$=\sssppp\TRAINER{$#3$}, $c$=\sssppp\TRAINER{$#4$}

\noTRAINER{\mmPapier{4}}

}

\newcommand{\findeABC}[4]{$#1$ \,\,\,$\Longrightarrow$\,\,\, $A$=\sssppp\TRAINER{$#2$},
$B$=\sssppp\TRAINER{$#3$}, $C$=\sssppp\TRAINER{$#4$}

\noTRAINER{\mmPapier{2.8}}

}

\findeabc{5x^2 + 3x + 7            = 0             }{5}{  3}{  7}
\findeabc{5x^2 - 3x + 7            = 0             }{5}{ -3}{  7}
\findeabc{5x^2 + 3x                = 0             }{5}{  3}{  0}
\findeabc{x^2 + 3x -4              = 0             }{1}{  3}{ -4}
\findeabc{3x^2                     = 16            }{3}{  0}{-16}

\noTRAINER{\newpage}

\findeabc{3\mu x            = 5z -rx^2       }{r}{3\mu}{-5z}
\findeabc{x^2 +x + 2x       = 0             }{1}{  3}{  0}
\findeABC{ax^2 + c          = bx            }{a}{ -b}{  c}
\findeABC{bx^2 -acx         = -b^2          }{b}{-ac}{b^2}
\findeabc{ux^2 - x + w      = vx^2          }{u-v}{-1}{w}

\noTRAINER{\newpage}

\findeABC{\frac{1}{2}x^2 - 2x + 3 = \frac{1}{2}x + b}{\frac{1}{2}}{-2.5}{3-b}
\findeabcPlz{rx^2 - x + s    = -t^2x + r     }{r}{t^2-1}{s-r}
\findeabcPlz{3-7x + (2-5x^2) = 3(5-x)(2x+3k)}{1}{9k-37}{5-45k}
\findeabcPlz{\sqrt{2}x^2 -\frac13\cdot{} x + \pi= \mu x^2 - \log(5)\cdot{}x - 0.111}{\sqrt{2}-\mu}{\log(5)-\frac13}{\pi+0.111}

\end{document}
