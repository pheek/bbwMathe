\input{bmsLayoutPage}
\renewcommand{\bbwAufgabenBlockID}{Gl1LinTxt}

\renewcommand{\metaHeaderLine}{Textaufgaben}
\renewcommand{\arbeitsblattTitel}{Textaufgaben zu linearen Gleichungen}

\begin{document}
\arbeitsblattHeader{}

\begin{center}V 1.0 (2025\_09\_21) \end{center}

{\tiny{Frei nach einem Arbeitsblatt von Ch. Benz. (\texttt{bms-w.ch})}}

\section{Terme und Variable}


\subsection{Terme}

\textbf{\bbwAufgabenNummer{}.}
Beschreiben Sie Sachverhalte mit Termen

\begin{bbwAufgabenBlock}
\item Die Summe zweier Kapitalien ist $79$ Franken. Eins der
Kapitalien sei $x$. Welcher Term
beschreibt das andere Kapital?

\LoesungsRaumLen{160mm}{$79-x$}\abplz{2}

\item Die Differenz zweier Zahlen ist 16. Die kleinere der Zahlen sei
$x$. Welcher Term beschreibt die andere Zahl?

\LoesungsRaumLen{160mm}{$x+16$}\abplz{2}


\item Das Produkt zweier Zahlen sei 48. Die eine Zahl sei $x$. Welcher
Term beschreibt die andere Zahl?

\LoesungsRaumLen{160mm}{$\frac{48}{x}$}\abplz{2}

\item Drei aufeinanderfolgende Zahlen sind gegeben. Die kleinste
dieser Zahlen sei $x$. Welche Terme beschreiben die anderen beiden
Zahlen?

\LoesungsRaumLen{160mm}{$x+1$ und $x+2$}\abplz{2}

\item Der Quotient zweier Zahlen sei 18. Die eine Zahl sei $x$. Welche
beiden möglichen Terme beschreiben die andere Zahl?

\LoesungsRaumLen{160mm}{$\frac{x}{18}$ bzw. $18x$}\abplz{2}

\end{bbwAufgabenBlock}

%%\platzFuerBerechnungenBisEndeSeite{}
\newpage



\subsection{Variable}
\textbf{\bbwAufgabenNummer{}.}
\textit{Erfinden} von Variablen. Eine Variable besteht aus 1. einem
Namen, 2. einer Bedeutung und 3. einer Maßeinheit.

Beispiel 1:

«Ein Auto hat einen um $4\,000.-$ Franken höheren Preis als ...»

Variable $x$ = Preis des Autos in Franken

($x$ = Name, Preis des Autos = Bedeutung, in Franken = Maßeinheit)

Beispiel 2:

«Die Summe zweier Gewichte ist gleich $280$ kg ....

Variable $x$ = erstes Gewicht in kg

($x$ = Name, erstes Gewicht = Bedeutung, in kg = Maßeinheit)


\textit{Erfinden} Sie zu den folgenden Textstellen sinnvolle Variable:

\begin{bbwAufgabenBlock}


\item Er ist drei Jahre älter als ...

\LoesungsRaumLen{160mm}{$x$ = Alter von ihm \textbf{in Jahren}}\abplz{2}

\item Die Dampfbahn kommt mit Kohle angetrieben um $1.8$ km weiter als
mit ...

\LoesungsRaumLen{160mm}{$x$ = Strecke der Dampfbahn mit Kohle
angetrieben \textbf{in km}}\abplz{2}

\item Kim kann $0.5$ m/s schneller laufen als ...

\LoesungsRaumLen{160mm}{$x$ = Laufgeschwindigkeit von Kim \textbf{in m/s}}\abplz{2}

\item Ein Haus hat einen um $4$\% höheren Verkaufspreis als ...

\LoesungsRaumLen{160mm}{$x$ = Verkaufspreis des Hauses in CHF}\abplz{2}

\noTRAINER{\newpage}

\item Maja und Marla haben zusammen 530 Fußballbildchen. Die Zahl der
Bildchen von Marla übersteigt die Anzahl der Bildchen von Maja um ....

\LoesungsRaumLen{160mm}{$x$ = Anzahl Bildchen von Maja}\abplz{2}

\item Pascall schläft am Morgen jeweils 90 Minuten länger als ...

\LoesungsRaumLen{160mm}{$x$ = Uhrzeit wann Pascall aufsteht}\abplz{2}

\end{bbwAufgabenBlock}

%%\platzFuerBerechnungenBisEndeSeite{}
\newpage

\subsection{Textverständnis}
\textbf{\bbwAufgabenNummer{}.}
Notieren Sie die zugehörigen Terme ($x$ ist jeweils die gesuchte Zahl):


\begin{bbwAufgabenBlock}
\item Die um $8$ vergrößerte Zahl ... \LoesungsRaumLen{30mm}{$x+8$}\abplz{0.8}
\item Die um $1.5$ verkleinerte Zahl ... \LoesungsRaumLen{30mm}{$x-1.5$}\abplz{0.8}
\item Das Vierfache der Zahl ... \LoesungsRaumLen{30mm}{$4x = 4\cdot{}x$}\abplz{0.8}
\item Der dritte Teil der Zahl ... \LoesungsRaumLen{30mm}{$\frac13\cdot{}x = \frac{x}3$}\abplz{0.8}
\item Der Kehrwert der Zahl ... \LoesungsRaumLen{30mm}{$\frac1x$}\abplz{0.8}
\item Das Produkt der Zahl mit $a$ ... \LoesungsRaumLen{30mm}{$ax=a\cdot{}x$}\abplz{0.8}
\item Die dritte Potenz der Zahl ... \LoesungsRaumLen{30mm}{$x^3$}\abplz{0.8}
\item Das dreifache Quadrat der Zahl ... \LoesungsRaumLen{30mm}{$3x^2$}\abplz{0.8}
\item Die Zahl wird von 70 subtrahiert ... \LoesungsRaumLen{30mm}{$70-x$}\abplz{0.8}
\item Die um $4$ verminderte Zahl ... \LoesungsRaumLen{30mm}{Absolute Verminderung $x-4$}\abplz{0.8}
\item Die $\frac14$ verminderte Zahl ... \LoesungsRaumLen{30mm}{Bei
Brüchen und Prozentzahlen wird üblicherweise die relative Verminderung
gemeint $x-\frac14\cdot{}x = 0.75\cdot{}x$}\abplz{0.8}
\item Die Gegenzahl ... \LoesungsRaumLen{30mm}{$-x$}\abplz{0.8}
\item Der Quotient der Zahl mit $a$ ... \LoesungsRaumLen{30mm}{$\frac{x}{a}$}\abplz{0.8}
\item Das doppelte der um fünf verminderten Zahl
... \LoesungsRaumLen{30mm}{$2\cdot{}(x-5)$}\abplz{0.8}
\item Der Zahl wird 15 dazugeschlagen ... \LoesungsRaumLen{30mm}{$x+15$}\abplz{0.8}
\end{bbwAufgabenBlock}

\platzFuerBerechnungenBisEndeSeite{}
\newpage

%%%%%%%%%%%%%%%%%%%%%%%%%%%%%%%%%%%%%%%%%%%%%%%%%5
\section{Textaufgaben}

\textbf{\bbwAufgabenNummer{}.}
Einfache Textaufgaben (einige sind im Kopf lösbar.)

\begin{bbwAufgabenBlock}
\item Zwei Gewichte ergeben zusammen $64$ kg. Eins der Gewichte ist
$25$ kg. Wie groß ist das andere Gewicht?

\LoesungsRaumLen{160mm}{Das andere Gewicht misst $39$ kg.}\abplz{2}

\item Die Differenz zweier Zeitdauern ist 18 Minuten. Die kleiner
Zeitdauer misst fünf Minuten. Wie lange ist die größere Zeitdauer?

\LoesungsRaumLen{160mm}{Die größere Zeitdauer misst 23 Minuten.}\abplz{2}

\item Das Produkt der Länge und Breite eines Rechtecks ergibt eine
Fläche von $36 \text{cm}^2$. Die Länge misst $12$ cm. Wie breit ist
das Rechteck?

\LoesungsRaumLen{160mm}{Das Rechteck ist $3$ cm breit.}\abplz{2}


\item Teilt man ein Kapital in drei gleich große Teile, so erhält man
CHF $220.-$ pro Anteil. Wie groß war das ursprüngliche Kapital.

\LoesungsRaumLen{160mm}{Das ursprüngliche Kapital war CHF $660.-$.}\abplz{2}

\item Das Vordach ist um $40 \text{m}^2$ kleiner als das
Hauptdach. Zusammen ist die Fläche beider Dächer $122 \text{m}^2$. Wie
groß ist das Hauptdach?

\LoesungsRaumLen{160mm}{Das Hauptdach ist $81 \text{m}^2$ groß.}\abplz{2}
\noTRAINER{\newpage}

\item Erwin ist heute doppelt so weit spaziert wie Erna. Zusammen sind
sie $18 \text{ km}$ spaziert. Wie weit ist Erna spaziert?

\LoesungsRaumLen{160mm}{Erna ist $6 \text{km}$ spaziert.}\abplz{2}

\item Karl und Karla vergleichen ihr Erspartes. Wenn man vom doppelten
von Karls Geld CHF $15.-$ abzieht, so erhält man das ersparte von
Karla. Das Ersparte von Karla erhält man auch, wenn man zum dritten
Teil von Karls erspartem noch CHF $18.-$ dazuschlägt. Wie groß ist
Karls Erspartes?

\LoesungsRaumLen{160mm}{1. Verstehen: Karl und Karla haben Erspartes
in CHF.

2. Variable $k$ = Karls erspartes.

3. Terme:

$2k$ = das doppelte Ersparte von Karl.

$2k-15$ = das um 15.- CHF verminderte doppelte ersparte von Karl.

Laut Text ist das auch Karlas Ersparte.

$\frac{k}3$ ist der dritte Teil von Karls erspartem. Somit:

$\frac{k}3 + 18$ = das um 18.- CHF dazugeschlagene zu Karls Drittel.

Doch laut Text ist auch das gleich dem Ersparten von Karla.

4. Gleichungen aufstellen:

$$2k-15 = \frac{k}3 + 18$$

5. Lösen: $\frac{5k}3 = 33$, somit $k=\frac{99}5 = 19.80$. Das
Ersparte von Karla ist somit entweder $2k-15$ oder $\frac{k}3+18$, was
beides $24.60$.

Karls Erspartes sind  $19.80$ CHF. (Karl hat somit
$24.60$ gespart.)}\abplz{2}



\item Addiert man zum Doppelten einer Strecke $50$ Meter, so erhält man
$133$ Meter. Wie lang ist die Strecke?

\LoesungsRaumLen{160mm}{Die Strecke misst $41.5$ Meter.}\abplz{2}


\item Die Summe zweier Kräfte sei 49 Newton. Die Differenz der Kräfte
ist 13 Newton. Wie groß ist die kleinere der beiden Kräfte?

\LoesungsRaumLen{160mm}{Die kleinere Kraft ist 18 Newton (die größere
somit 31 Newton).}\abplz{2}

\item Subtrahiert man vom Dreifachen einer Zeitspanne $17$ Sekunden, so
erhält man die Hälfte der um $4$ Sekunden verminderten Zeitspanne. Wie
viele Sekunden dauerte die Zeitspanne?

\LoesungsRaumLen{160mm}{Sei $s$ die Zeitspanne in Sekunden.
Mit $3s-17$ und $\frac{s-4}2$ ergibt sich die Gleichung («... so
erhält man...»):
$$3s - 17 = \frac{s-4}2$$
(lösen. \zB TR liefert)

Die Zeitspanne misst $6$ Sekunden.}\abplz{2}

\noTRAINER{\newpage}

\item Wie viele Fußballbildchen erhält Max, wenn 960 Bildchen so
zwischen ihm und Moriz aufgeteilt werden, dass Moriz 72 Bildchen
weniger erhält als Max?

\LoesungsRaumLen{160mm}{Max erhält 516 Bildchen (Moriz erhält 444).}\abplz{2}


\item Wie viel Geld von CHF $480.-$ erhält Lisa, wenn das Geld
zwischen ihr und Lion so aufgeteilt wird, dass Lisa die Hälfte von
Lions Anteil erhält?

\LoesungsRaumLen{160mm}{Lisa erhält CHF $160.-$}\abplz{2}

\item Dagobert Duck hat vier mal so viele Tannenzapfen wie Gitta
Gans. Gäbe er ihr $19$ seiner Tannenzapfen, so hätte er immer noch einen
mehr als sie. Wie viele Tannenzapfen hat sie?

\LoesungsRaumLen{160mm}{Gitta hat $13$ Tannenzapfen.}\abplz{2}

\item Bei einer Wahl erhielten die vier Kandidaten zusammen $5219$
Stimmen. Der Gewinner übertraf den ersten Gegenkandidaten um $22$, den
zweiten um $30$ und den dritten um $73$ Stimmen. Wie viele Stimmer erhielt
der Gewinner?

\LoesungsRaumLen{160mm}{Der Gewinner erhielt $1336$ Stimmen.}\abplz{2}

\noTRAINER{\newpage}

\item Toni und Jaymee vergleichen ihre Handy Abos.
Würden beide CHF $0.10$ mehr bezahlen, so wäre das Abo von Toni 60 \% günstiger als dasjenige von Jaymee.
Würde hingegen Toni CHF $7.50$ mehr und gleichzeitig Jaymee CHF $7.50$ weniger bezahlen, so wären die beiden Abos gleich teuer.

Was kostet Tonis Handy Abo?


{\tiny{(Alle Preise verstehen sich pro Monat inklusive SMS und Anrufe in der Schweiz inklusive Datenepaket von 5 GiB pro Monat; nicht kumulierbar.)}}

\TRAINER{$x$ = Tonis Abo in CHF pro Monat. $x + 2\cdot{} 7.50 = x+15.00$: Jaymees Abokosten.

Gleichung: $x+0.10 = 40\% \text{ von } (x+15.00 + 0.10) = 0.4 \cdot{} (x+15.10)$. Ergo $x=9.9$}

\LoesungsRaumLen{160mm}{Tonis Abo kostet CHF $9.90$ pro Monat.}\abplz{2}


\end{bbwAufgabenBlock}


\platzFuerBerechnungenBisEndeSeite{}%
\end{document}%
