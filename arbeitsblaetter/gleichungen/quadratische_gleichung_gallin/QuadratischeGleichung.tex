%%
%% Meta: TI nSpire Einführung
%%       Ziel: Damit die Grundoperationen damit durchgeführt werden können.
%%             Damit man sich an den Rechner gewöhnt.
%%

\input{bbwSeite}

%%%%%%%%%%%%%%%%%%%%%%%%%%%%%%%%%%%%%%%%%%%%%%%%%%%%%%%%%%%%%%%%%%

\usepackage{amssymb} 
\renewcommand{\metaHeaderLine}{Arbeitsblatt}
\renewcommand{\arbeitsblattTitel}{Lösungsformel zur quadratischen Gleichung}

\begin{document}%%
\arbeitsblattHeader{}
\textbf{Zur Auflösungsformel für die quadratische Gleichung $ax^2+bx+c=0$}

Bestimmen Sie die Lösungsmenge jeder Gleichung. In der Regel hat jede
Gleichung \textbf{zwei} Lösungen. Schreiben Sie Ihre Gedankengänge
auf. Insbesondere: Was ist neu gegenüber der vorangehenden
Aufgabe. Vereinfachen Sie die Wurzeln so, dass

\begin{enumerate}[label=\alph*)]
\item \textbf{keine Wurzeln im Nenner} vorkommen (Erweitern)
\item jede Wurzel \textbf{maximal ausgezogen} ist (Bsp. :
$\sqrt{50}=\sqrt{2 \cdot 25} =5 \sqrt{2}$)
\end{enumerate}

\newcommand{\noteSpace}{\noTRAINER{\rule{0pt}{6.5ex}}}

\begin{tabular}{p{5cm}|p{12cm}}

1. $x^2 = 4$     & \noteSpace{}\TRAINER{$x_{1,2}=\pm 2$ Wurzel ziehen}\\
\hline

2. $x^2 - 3 = 0$ & \noteSpace{}\TRAINER{$x_{1,2}=\pm \sqrt{3}$ Drei
auf die andere Seite nehmen.}\\
\hline

3. $2x^2 - 1 = 0$ & \noteSpace{}\TRAINER{$x_{1,2}=\pm \frac{\sqrt{2}}{2}$}\\
\hline

4. $x^2 = 6$ & \noteSpace{}\TRAINER{$x_{1,2}= \pm\sqrt{6}$ }\\
\hline

5. $(x+2)^2 = 6$ & \noteSpace{}\TRAINER{$x_{1,2}=-2\pm\sqrt{6}$ }\\
\hline

6. $x^2 -6x+9 = \frac{25}{4}$ & \noteSpace{}\TRAINER{$x_{1,2}=3\pm\frac{5}{2}$ }\\
\hline

7. $x^2-6x=31$ & \noteSpace{}\TRAINER{$x_{1,2}=3\pm\sqrt{40}$ }\\
\hline

8. $x^2 +4x = - \frac{7}{4}$ & \noteSpace{}\TRAINER{$x_{1,2}=-2\pm\frac{3}{2}$ }\\
\hline

9. $x^2 - \frac{2}{3}x = -\frac{1}{9}$
& \noteSpace{}\TRAINER{$x_{1,2}=\frac{1}{3}$  Mit $\frac{1}{9}$
quadratisch ergänzen.}\\
\hline

10. $x^2 - 3x =  - \frac{25}{4}$ & \noteSpace{}\TRAINER{$\mathbb{L}_x=\{\}$ }\\
\hline

11. $2x^2 +4x-7 = 0$ & \noteSpace{}\TRAINER{$x_{1,2}=-1\pm\frac{3\sqrt{2}}{2}$ }\\
\hline

12. $\frac{1}{6}x^2 - \frac{1}{4}x - \frac{1}{6}  = 0$
& \noteSpace{}\TRAINER{$x_{1,2}=-\frac{1}{2}, 2$
Quadratisch ergänzen mit $\frac{9}{16}$}\\
\hline

13. $x^2 +2px + q = 0$ & \noteSpace{}\TRAINER{$x_{1,2}=-p\pm\sqrt{p^2-q}$ }\\
\hline

14. $ax^2 +bx + c = 0$ & \noteSpace{}\TRAINER{$x_{1,2}=\frac{-b\pm\sqrt{b^2-4ac}}{2a}$ }\\
\hline
\end{tabular}

\end{document}
