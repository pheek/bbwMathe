%%
%% Meta: Formelsammlung GESO BBW
%%

\input{bbwSeite}

%%%%%%%%%%%%%%%%%%%%%%%%%%%%%%%%%%%%%%%%%%%%%%%%%%%%%%%%%%%%%%%%%%

%%\usepackage{amssymb} %% für \blacktriangleright
\usepackage{makecell}
\renewcommand{\metaHeaderLine}{Formelsammlung GESO}
\renewcommand{\arbeitsblattTitel}{Version 0.0.2 2021-06-12 fp}

\titlespacing*{\section}{0pt}{0.1\baselineskip}{0.1\baselineskip}
\titlespacing*{\subsection}{0pt}{0.1\baselineskip}{0.1\baselineskip}

\begin{document}%%
\arbeitsblattHeader{}
 \begin{multicols}{2}

\section{Zahlmengen}
%% Benutze
%%\columnbreak
%% für harten Break
\begin{definition}{}{}
$\mathbb{N} = \{1,2, ...\}  $ = Natürliche Zahlen\\
$\mathbb{Z} = \{..., -2, -1, 0, 1,2, ...\}  $ = Ganze Zahlen\\
$\mathbb{Q} = \{\frac{a}b | a\in \mathbb{Z}, b\in\mathbb{N}\}$ = Rationale Zahlen (Brüche)\\
$\mathbb{R} = \{..., -1, ..., 0, ... \frac12, ..., \sqrt{2}, ..., e,\\
..., 3, ..., \pi, ...\}  $ = \textbf{reelle} Zahlen = alle Zahlen auf der Zahlengerade\\

Eine natürliche Zahl heißt Primzahl, wenn sie genau zwei Teiler hat (Eins und sich selbst). $\mathbb{P} = \{2, 3, 5, 7, 11, 13, ...\}$
\end{definition}

\hrulefill

\section{Algebra}
%%\subsection{Binomische Formeln}

\begin{gesetz}{Binomische Formeln}{}
$$(a+b)^2 = a^2+2ab + b^2$$
$$(a-b)^2=a^2-2ab+b^2$$
$$(a+b)\cdot(a-b) = a^2 - b^2$$
\end{gesetz}

\subsection{Vertauschte Differenz}
(-1 ausklammern)

$(b-a)=(-1)\cdot{}(h-b)$

Beispiel:

$$\frac{x}{a-b} +  \frac{y}{b-a} = \frac{x}{a-b} + \frac{-y}{a-b} = \frac{x-y}{a-b}$$

\subsection{Summenzeichen}
$$\sum_{k=1}^n{T(k)} = T(1) + T(2) + ... + T(n)$$
Beispiel

$$\sum_{i=3}^7{i^2} = 3^2 + 4^2 + 5^2 + 6^2 + 7^2$$

Taschenrechner: \tiprobutton{math} 4


\subsection{(Absolut)betrag einer Zahl}

$|a| = a$ falls $a \ge 0$

$|a| = (-1)\cdot{}a$ falls $a < 0$

\hrulefill

\section{Dezimalzahlen}
\subsection{Runden}
Dezimalen sind die Stellen nach dem Komma.
Runden auf $n$ Dezimalen: betrachte die nächste Ziffer und runde auf, wenn diese $\ge 5$.

Beispiel

3.4729\textbf{6}44 auf vier Dezimalen, so ist die 5. Dezimale (hier die Ziffer 6) zu beachten:

$$3.4729\textbf{6}44 \approx 3.4730$$
(Alle vier Dezimalen sind anzugeben.)

\subsection{Signifikante Stellen}
Die Zählung der signifikanten Stellen einer Zahl beginnt (von links) mit der ersten von Null verschiedenen Ziffer. Ab hier zählen alle Ziffern, auch die 0, bis zur verlangten Rundungsstelle.

Bsp.: 30.896 auf \textbf{vier} sig. Stellen = 30.90\\
(4. Ziffer stehen lassen, auch wenn = 0)

\subsection{Wissenschaftliche Schreibweise}
Für große, aber auch für Zahlen sehr nahe an 0, wird die wissenschaftliche Schreibweise vorgezogen. Dabei steht vor dem Dezimalpunkt immer genau eine Ziffer:

\begin{tabular}{lcccr}
Zahl  & & wissenschaftl. & & TR: \tiprobutton{EE} \\
1400  &=& $1.4\cdot{}10^3$ &=& 1.4E3\\
0.004 &=& $4\cdot{}10^{-3}$ &=&4E-3\\
\end{tabular}

\hrulefill

\section{Potenzen}

\subsection{gleiche Basis}
\begin{tabular}{cc}
$a^m\cdot{}a^n = a^{m+n},$ & $a^m:a^n=a^{m-n},$ \\
$\left(a^m\right)^n = a^{m\cdot{}n}$ &\\
 \end{tabular} 

\subsection{gleicher Exponent}

\begin{tabular}{cc}
$a^n\cdot{}b^n = (ab)^n,$ & $\frac{a^n}{b^n} =\left(\frac{a}b\right)^n $\\
 \end{tabular}
 

\subsection{Vorzeichen}
Fall: $n$ \textbf{gerade}:

\begin{tabular}{cc}
 $-(a^n) = -a^n,$ & aber: $(-a)^n = +a^n$\\
 \end{tabular} 

Fall: $n$ \textbf{un}gerade:

\begin{tabular}{cc}
 $-(a^n) = -a^n,$ & aber: $(-a)^n = -a^n$\\
 \end{tabular} 


\begin{gesetz}{Potenzen}{}

Für Basis $a\ne 0, b\ne 0$ und $r, s\in\mathbb{Q}$ gilt:

\begin{tabular}{cc}
$a^0=1,$ & $a^{-r} = \frac1{a^r},$  \\
$\left(\frac{a}b\right)^{-r} = \left(\frac{b}a\right)^r,$ & $a^{\frac{r}s} = \sqrt[s]{a^r}$\\
 \end{tabular}
\end{gesetz}

\begin{rezept}{«Kielholen»}{}{}
Exponenten vertauschen ihr Vorzeichen beim Übertreten des Bruchstrichs:
$$\frac{a^{-3}b^2}{c^5d^{-6}} = \frac{b^2d^6}{a^3c^5}$$
\end{rezept}


\subsection{$n$-te Wurzel}
$\sqrt[n]{a} := \left(a\right)^\frac1n$


\hrulefill
\section{Logarithmen}

\begin{definition}{$\log$}{}

Für $a>0, a\ne 1$ ist

$$\log_a{}(z)=x \Longleftrightarrow{} a^x = z$$
\end{definition}
$a$ = Basis, $z$ = Numerus (stets > 0), $x$ = Logarithmus (= Exponent in der Potenzschreibweise)

Logarithmen sind Exponenten zu einer fest gewählten Basis.

\begin{gesetz}{}{}
$$\log_a(b^x) = x\cdot{}\log_a(b)$$
\end{gesetz}

\subsection{Spezielle Werte}
\begin{tabular}{ccc}
$\log(1)=0$     &               &\\
$\log_a(a) = 1$ & $\lg(10) = 1$ & $\ln(e) = 1$\\
\end{tabular} 

\hrulefill
\section{Textaufgaben}

\subsection{Geschwindigkeitsaufgaben}
\fbox{$\frac{s}{v \cdot{}t}$} (Einheiten kompatibel wählen!)

\subsection{Leistungsaufgaben}
Bei $x$ Stunden für die ganze Arbeit:

Bruchteil Arbeit pro h = Leistung

Leistung = $\frac{1 \textrm{ [Arbeit]}}{x \textrm{ [Stunde]}}$

\subsection{Restaufgaben}
$$A:B = C \textrm{ Rest } R \Longleftrightarrow{}  C\cdot{}B+R = A$$

\subsection{Zinseszins}
\begin{itemize}
\item $K_0$ = Startkapital; $K_n$ = Endkapital
\item $p$ = Zinssatz (in \%); $f = \frac{p}{100}$ = Zinsfaktor
(Wachstumsfaktor); $p<0$, so ist $f$ der Zerfallsfaktor
\end{itemize}
$$K_n = K_0 \cdot{} \left( 1+\frac{p}{100} \right)^n = K_0\cdot{}f^n$$

\subsubsection{Beispiele}
$$p = 2.5\% \Longleftrightarrow{} f = 1.025$$
$$p = -3\% \Longleftrightarrow{} f = 0.97$$


\hrulefill
\section{Gleichungen}

\subsection{Bruchgleichungen}
Immer Definitionsbereich $\mathbb{D}$ überlegen: Nenner (unten) darf
nicht 0 werden.

Bsp.: $$\frac5{x+2}=\frac{7+x}{x-3} \Leftrightarrow{} \mathbb{D}_x=\mathbb{R}\backslash{}\{-3, 2\}$$

Schlusskontrolle!



\subsection{Lineare Gleichungen}
$$5x-3a = 2cx-7b$$
1. Alle $x$ nach links, übriges rechts:
$$5x - 2cx = 3a-7b$$
2. $x$ ausklammern
$$x\cdot{}(5-2c) = 3a-7b$$
3. Durch Klammer teilen
$$x = \frac{3a-7b}{5-2c}$$


\subsection{num-solv}
Der Taschenrechner löst Gleichungen mit Zahlen nach $x$ auf:

$5^x = 100 -x$; verwende \tiprobutton{2nd}\tiprobutton{sin_num-solv} $x\approx{}2.843$


\end{multicols}


\subsection{quadratische Gleichungen}
\begin{gesetz}{Lösungsformel}{}
\textbf{Grundform}: $ax^2 + bx+c = 0$

$$x_{1,2}\frac{-b \pm \sqrt{b^2-4ac}}{2a}$$
\end{gesetz}

\textbf{Diskriminante} $D = b^2-4ac$:
$D>0$: Zwei Lösungen;
$D=0$: Eine Lösung;
$D<0$: keine Lösung

Wie vorgehen?

\begin{tabular}{|p{44mm}|p{53mm}|p{64mm}|}
	\hline
	Spezialfall $b=0$               & $5x^2 = 3$                   & Durch 5 dividieren, dann positive und negative Wurzel ziehen.\\
	\hline
	Spezialfall $c=0$               & $3x^2 = 5x$                   & Fallunterscheidung $x$ = 0 und $x \ne 0$\\
	\hline
	Bereits faktorisiert       & $(x-4.6 + b)\cdot{}(x+a) = 0$ & Lösungen hier $x_1=4.6-b$ und $x_2 = -a$\\
	\hline
	Einfache Zahlen            & $x^2 -2x + 1= 0$           & $a$-$b$-$c$--Formel (Mitternachtsformel) oder faktorisieren $x^2-2x+1=(x-1)^2$ und danach jeden einzelnen Faktor $=0$ setzen.\\
	\hline
	Komplizierte Zahlen        & $7.3x^2 - 8x - 3.4 = 0$       & Taschenrechner \tiprobutton{2nd}\tiprobutton{cos_poly-solv}             \\
	\hline
	Variable (Parameter)       & $7.3x^2 - cx + 2.6=0$         & $a$-$b$-$c$-Formel (Mitternachtsformel) \\
	\hline
	Komplexe Terme im Quadrat  & $3(x-6)^2 - 17(x-6)  = 0$     & Substitution                            \\
	\hline
\end{tabular}

\mmPapier{4}

\newpage


\begin{multicols}{2}


\subsection{Gleichungen mit Exponenten}
\subsubsection{Potenzgleichungen}

$$x^a=c \Leftrightarrow x=\sqrt[a]{c}$$

Allgemein

$$x^{\frac{a}b} = c \Leftrightarrow{}
x=c^{\frac{b}a} = \sqrt[a]{c^b} = \left(\sqrt[a]c\right)^b$$

\subsection{Exponentialgleichungen}
\textbf{Fall 1} Exponentenvergleich. Bsp.:

$$5^4=125^x \Rightarrow{} 5^4=(5^3)^x=5^{3x} \Rightarrow{}$$
$$ 4=3x \Rightarrow x=\frac43$$

\textbf{Fall 2} Logarithmen. Bsp.:

$$7^x=105 \Rightarrow x=\log_7(105)$$
oder so:
$$7^x=105 \Rightarrow$$
$$ \log(7^x)=\log(105) \Rightarrow$$
$$x\cdot{}\log(7)=\log(105) \Rightarrow$$
$$ x=\frac{\log(105)}{\log(7)} = \frac{\lg(105)}{\lg(7)}$$

\begin{gesetz}{Exponentialgleichung}{}
$$a^x=b \Rightarrow{} x=\log_a(b) = \frac{\lg(b)}{\lg(a)}$$
\end{gesetz}

\subsection{Lineare Gleichungssysteme}
Erst in \textbf{Grundform} bringen und dann
\gleichungZZ{5x+7y}{19}{16x -13y}{-4}
mit Taschenrechner \tiprobutton{2nd}\tiprobutton{tan_sys-solv} lösen.

Verfahren
\begin{itemize}
\item Einsetzungsverfahren
\item Additionsverfahren
\item Gleichsetzungsverfahren
\item Graphische Lösung mit zwei Geradengleichungen
\item Substitution
\end{itemize}

\hrulefill
\section{Funktionen}

Eine Funktion ordnet jeder Zahl aus einem Definitionsbereich
$\mathbb{D}$ genau eine Zahl aus einem Wertevorrat (=Zielmenge) (meist
$\mathbb{R}$)
zu. Die \textit{getroffenen} Funktionswerte bilden den Wertebereich
(= Bildmenge) $\mathbb{W}$.

\hrulefill
\section{Lineare Funktionen}
Explizite Form:
$$g: y = f(x) = a\cdot{}x + b$$

\bbwCenterGraphic{8cm}{img/lineareFunktion.png}
$a$ = Steigung = $\frac{V}{H}=\frac{\Delta y}{\Delta x} = \frac{y_2-y_1}{x_2-x_1}$.\\
Zwei Geraden sind \textbf{parallel}, wenn ihre Steigungen gleich sind.

$b$ = $y$-Achsenabschnitt (=Ordinatenabschnitt)


\subsection{Nullstelle...} ...einer Funktion: $y=0$ setzen und nach $x$
auf\/lösen.

Bsp.: $y=7x+4$ Nullstelle: $0 = 7x+4$, d.\,h. $x_0=\frac{-4}{7}$

\subsection{Ordinatenabschnitt}
\textbf{Schnittpunkt mit der $y$-Achse}: $x=0$ setzen.

Bsp.: $y=7x+4$ Achsenabschnitt = Schnittpunkt mit der $y$-Achse:
$x=0$.\\
Also $y=7\cdot{}0 + 4 = 4$. Somit liegt der Schnittpunkt bei $(0|4)$.


\subsection{Schnittpunkt zweier Geraden}
Gegeben: $g: y=ax+b$ und $h: y=cx+d$.

Gleichsetzen:

\gleichungZZ{y}{ax+b}{y}{cx+d}
Finde $x_S$ des Schnittpunktes $S=(x_S|y_S)$
$$ax+b = cx+d$$
$$x_S = \frac{d-b}{a-c}$$
\subsection{Mittelpunkt $M$ einer Strecke $AB$}
Gegeben $A=(x_A|y_A)$ und $B=(x_B|y_B)$

Gesucht Mittelpunkt $M=(x_M|y_M)$

Lösung Mittelwert $x$ und $y$ separat:

$$x_M = \frac{x_A+x_B}2; y_M=\frac{y_A+y_B}2$$

\subsection{Horizontale Gerade}

$g: y=ax+b$ ist horizontal, wenn die Steigung $a$ wegfällt ($a=0$).

$$g:  y=0\cdot{}x+b \Rightarrow y=b$$

\subsection{Zweipunkte Aufgaben}
Gesucht Gerade $g: y=ax+b$ durch zwei gegebene Punkte $A=(x_A|y_A)$
und $B=(x_B|y_B)$

\textbf{Variante 1}: Steigung $a = \frac{V}H
= \frac{y_B-y_A}{x_B-x_A}$ berechnen und einen der beiden Punkte in
$y=ax+b$ einsetzen, um $b$ zu finden.

\textbf{Variante 2}: Gleichungssystem nach $a$ und $b$ auf\/lösen:
\gleichungZZ{y_A}{a\cdot{}x_A+b}{y_B}{a\cdot{}x_B + b}

\hrulefill

\section{Potenzfunktionen}
Grundform:
$$y=ax^z$$
$z\in\mathbb{Z}\backslash\{0,1\} = \{...-2, -1, 2, 3, 4, ...\}$

$z>0$: Parabeln

$z<0$: Hyperbeln

$z$ gerade: Gespiegelt an der $y$-Achse

$z$ ungerade: Gespiegelt am Ursprung $O=(0|0)$

$a>0$: gerade Parabel nach oben geöffnet bzw. Hyperbel oder ungerade
Parabel im Quadranten I und III

$a<0$: gerade Parabel nach unten geöffnet bzw. Hyperbel oder ungerade
Parabel im Quadranten II und IV

\bbwCenterGraphic{8cm}{img/potenzFunktionen.png}

\subsection{Quadratische Funktion}
$$y=ax^2$$
$a$ heißt Öffnung ($a<0$: Die Parabel ist nach unten geöffnet)
$a=1$ oder $a=-1$: Normalparabel

\hrulefill
\section{Exponentialfunktionen}
\begin{gesetz}{Exponentialfunktion}{}
$$f(t) = b\cdot{}a^t$$
$b$ = Anfangswert bei $t_0=0$; $b=f(0)$\\
\phantom{$b$} = $y$-Achsenabschnitt\\
$a$ = Vervielfältigungsfaktor ($a\in\mathbb{R}^{+}$)\\\
$a>0$: Wachstumsrate\\
$a<0$: Zerfallsrate
\end{gesetz}


\begin{rezept}{Intervall-Skalierung}{}
$$f(t) = b\cdot{}a^{\frac{t}{\tau}}$$
$\tau$ = Beobachtungsintervall zu $a$\\
$a$ = Wachstums(/Zerfalls)rate im Zeitraum $\tau$
\end{rezept}

Beispiel: Die Algen mit Startwert von $3\textrm{m}^2$ vervierfachen
sich alle fünf Tage:\\
$b=3$, $a=4$, $\tau=5$
$$f(t)= 3\cdot{}4^\frac{t}{5}$$

\subsubsection{Werte Rechnen}
\begin{itemize}
\item
Wert zum Zeitpunkt 25 Tage: Wert einsetzen:\\
$f(25) = 3\cdot{}4^\frac{25}{5}$

\item
Tage rechnen bei vorgegebenem Wert (z. B. $30m^2$):\\
Löse die Exponentialgleichung nach $t$ auf:\\
$$30 = 3\cdot{}4^{\frac{t}{5}}$$
\end{itemize}

\subsection{Halbwertszeit}
$\frac12 \cdot{} b = f(t) = b\cdot{}a^{\frac{t}{\tau}}$
$$t_{\frac12} = \tau\cdot{}\log_a(\frac12)$$

\subsection{Verdopplungszeit}
$2\cdot{}b = f(t) = b\cdot{}a^{\frac{t}{\tau}}$
$$t_{2} = \tau\cdot{}\log_a(2)$$

\subsection{Sättigungsfunktionen}
\begin{gesetz}{Beschränkter Zerfall}{}
$$f(t) = c + b\cdot{}a^\frac{t}\tau$$
$c$ = Sättigungsgrenze\\
$a<1$ = Zerfallsrate\\
$b$ = Sättigungsdefizit bei $t=0$
\end{gesetz}

\begin{gesetz}{Beschränktes Wachstum}{}
$$f(t) = c - b\cdot{}a^\frac{t}\tau$$
$c$ = Sättigungsgrenze\\
$a<0$ = Zerfallsrate\\
$b$ = Sättigungsdefizit bei $t=0$
\end{gesetz}

\begin{bemerkung}{Sättigungsdefizit}{}

$b$ = Sättigungsdefizit bei $t=0$.\\
$b\cdot{}a^\frac{t}\tau$ = Sättigungsdefizit zur Zeit $t$
\end{bemerkung}

\end{multicols}

\hrulefill
\section{Datenanalyse}

\subsection{Skalen}
\bbwCenterGraphic{16cm}{allg/daan/img/Datentypen.png}





\begin{multicols}{2}

\subsection{Lagemaße}
\subsubsection{Durchschnitt}
= Mittelwert = arithmetisches Mittel

\begin{definition}{Mittelwert}{}
$$\overline{x} = \frac{x_1 + x_2 + x_3 + ... + x_n}{n}$$
\end{definition}

\subsubsection{Median}
Der Median ist der Wert in der Mitte der geordneten Datenreihe. Ist
die Anzahl Werte gerade, so wird der Mittelwert der beiden in der
Mitte stehenden Werte genommen.
\begin{definition}{Median}{}

Der Median $\mediantilde{x} = x_{\textrm{MED}}$ heißt auch Zentralwert
\end{definition}

\subsubsection{Quartil}

\begin{rezept}{Quartile}{}

Die Quartilsgrenzen $Q_1$ bzw. $Q_3$ sind die Mediane der linken,
bzw. rechten Datenhälfte (nachdem der eine Zentralwert entfernt
wurde).

Am einfachsten mit dem Taschenrechner:

Dateneingabe: \tiprobutton{data}

Auswertung: \tiprobutton{data_stat-reg-distr}
\end{rezept}

\subsubsection{Modus}
Der am häufigsten auftretende Wert wird als Modus bezeichnet.

Manchmal ist dies nicht eindeutig, dann sprechen wir von einer
multimodalen Verteilung.

\subsubsection{Maximum, Minimum}
Maximum bzw. Minimum sind höchste bzw. kleinste Datenwerte: Die Ausreißer
zählen hier mit!

\subsection{Streumaße}
\subsubsection{Spannweite}
Die Spannweite $R$ (Range) ist nichts anderes als das Maximum minus
das Minimum.

\subsubsection{Interquartilsabstand}
Der Abstand zwischen $Q_1$ und $Q_3$ wird als Quartilsdifferenz oder
englisch \textit{Interquartil-range} = \textbf{IQR} bezeichnet.

\subsubsection{Standardabweichung}
Mit $\sigma$ bzw. $S$ wird die Standardabweichung bezeichnet. Sie
zeigt die Wendepunkte der Normalverteilung an.

$\sigma$ = Standardabweichung der Grundgesamtheit

$S$ = Standardabweichung einer gemessenen Stichprobe

Die Standardabweichung wird auch mit dem Taschenrechner
unter \tiprobutton{data_stat-reg-distr} gefunden.

\subsubsection{Robustheit}

\begin{tabular}{|c|c|c|}\hline
   & Lage- & Streumaß\\\hline
 \multirow{4}{*}{\rotatebox{90}{robust}}  & Median $\mediantilde{x}$ & IQR \\
    & Modus $x_{mod}$ & \\
    & Quartile ($Q_1$,$Q_3$) & \\
    & (Ausreißerschwellen) & \\\hline
 \multirow{3}{*}{\rotatebox{90}{«fragil»}}  & Mittelwert
 $\overline{x}$ & Standardabweichung\\
    & Minimum & Spannweite\\
    & Maximum & \\\hline
 \end{tabular}

\subsection{Diagramme}
\subsubsection{Balken / Kuchen}
\begin{definition}{Relative Häufigkeit}{}

$n$ = Anzahl Werte

$h_i$ = Absolute Häufigkeit des Merkmals $i$

$f_i = \frac{h_i}n$ = relative Häufigkeit

$p_i = f_i\cdot{}100\%$ = prozentuale Häufigkeit

$\varphi_i = f_i\cdot{}360\degre$ = Zentriwinkel im Kuchendiagramm
\end{definition}



\subsubsection{Histogramm}
\bbwCenterGraphic{8cm}{img/histogramm.png}
\begin{itemize}
\item Alle Balken sind gleich breit
\item Alle Balken berühren sich
\item Die Höhe der Balken wird durch die Häufigkeit der entsprechenden
Balken festgelegt
\item Werte auf Balkengrenzen werden immer rechts mitgezählt
\end{itemize}

\subsubsection{Streudiagramm}
\bbwCenterGraphic{8cm}{img/streudiagramm.png}


\subsubsection{Boxplot}
Vorgehen
\begin{enumerate}
\item Daten in TR eingeben \tiprobutton{data}...
\item ... und mit \tiprobutton{data_stat-reg-distr} auslesen.
\item Median, und Quartile einzeichnen
\item IQR (=QD in der Grafik) := Q3-Q1 rechnen
\item IQR mit 1.5 multiplizieren
\item $f_u = Q_3 + 1.5\cdot{}\textrm{IQR}$ und $f_o=Q_1 - 1.5\cdot{}\textrm{IQR}$ (ausradierbar) als
Ausreißerschwellen fein markieren (gehören nicht zum Boxplot)
\item Alle Ausreißer mit Ring oder Stern einzeichnen.
\item Whisker (entfernteste Werte innerhalb der Ausreißerschwellen)
einzeichnen.
\item Boxdekoration horizontale Linien einzeichnen.
\end{enumerate}
\bbwCenterGraphic{8cm}{img/boxplot.png}


\hrulefill
\section{Stochastik}

\subsection{Kombinatorik}

\subsubsection{Produktregel}
Hat eine zweistufige Situation zunächst $k_1$ Möglichkeiten, gefolgt
von $k_2$ Möglichkeiten, so hat man insgesamt $k_1\cdot{}k_2$
Möglichkeiten. Dies ist auf beliebig viele Stufen verallgemeinerbar.

\subsubsection{Permutationen (Sitzordnungen)}

Bei $n$ Elementen gibt es $n!$ Permutationen:

$$n! = n\cdot{}(n-1)\cdot{}(n-2)\cdot{}(n-3)\cdot{} ... \cdot{}2\cdot{}1$$

Bei Null Elementen gibt es nur eine Variation: $0! = 1$


\subsubsection{Variation mit Wiederholung}
Variation = Reihenfolge wesentlich

Mit Wiederholung = mit Zurücklegen

Gegeben: $n$ Elemente und $k$-malige Auswahl
(\zB Zahlenschloss)

Total $n^k$ Möglichkeiten

\subsubsection{Variation ohne Wiederholung}
$k$ Kugeln aus einer Urne mit $n$ Kugeln ziehen (ohne Zurücklegen)

Total $\frac{n!}{(n-k)!}$ Möglichkeiten. Taschenrechner: \texttt{nPr}. 

\subsubsection{Kombination ohne Wiederholung}
Kombination = Reihenfolgen \textbf{nicht} wesentlich

Beispiele:

\begin{itemize}
\item Aus $n$ Kästchen $k$ Kästchen ankreuzen
\item Eine Delegation von $k$ Personen aus einer Klasse von $n$
Personen wählen.
\item $k$ Elemente aus $n$ unterscheidbaren Elementen wählen und diese
auf $k$ ununterscheidbare Plätze verteilen.
\item $k$ Kugeln aus einer Urne mit $n$ unterscheidbaren Kugeln
auswählen; die Reihenfolge der Auswahl spielt keine Rolle.
\end{itemize}

Wähle $k$ Elemente aus $n$ Elementen ohne Zurücklegen und ohne
Beachten der Reihenfolge:

 ${n\choose k} = \frac{n!}{k!\cdot{}(n-k)!}$ Möglichkeiten

Taschenrechner: \textbf{nCr}



\end{multicols}




\subsection*{Überblick}
\bbwCenterGraphic{15cm}{img/Kombinatorik.png}
\hrulefill


\begin{multicols}{2}


\subsection{Wahrscheinlichkeit}
\subsubsection{Grundbegriffe}

\textbf{Ergebnismenge} $\Omega$: Menge der möglichen Ergebnisse
(Ausgänge) eines Zufallsversuchs. Beispiel die sechs Würfelergebnise
eines Spielwürfels: $\Omega = \{\epsdice{1}, \epsdice{2}, \epsdice{3},
..., \epsdice{6}\}$

\textbf{Ereignis} $A$: Eine Teilmenge von $\Omega$. Beispiel: Eine
ungerade Zahl zu werfen: $A  = \{\epsdice{1}, \epsdice{3}, \epsdice{5}\}$

\textbf{Gegenereignis} $\overline{A} = \Omega \backslash A$ =
Gegenmenge von $A$ = «nicht» $A$. Es gilt $A \cup \overline{A} = \Omega$

\textbf{Elementarereignis}: Ereignis bestehend aus einem einzigen
Ergebnis aus $\Omega$. Beispiel: $E_1$ = Wirf eine \textbf{Eins}: $E_1
= \{\epsdice{1}\}$.


\subsection{Elementare Wahrscheinlichkeit}
Die Wahrscheinlichkeit des \textbf{Gegenereignisses} von $E$ ist
$$P(\overline{E}) = 1- P(E)$$
Bei Aufgaben wie «... mindestens eins ...» ist das Arbeiten mit der
Gegenwahrscheinlichkeit häufig einfacher.

Ereignisse sind voneinander \textbf{unabhängig}, wenn sie keine
gleichen Ergebnisse aufweisen.
$$A \textrm{ unabhängig von } B \Leftrightarrow A\cap B=\{\}$$

Für voneinander \textbf{unabhängige} Ergebnisse 
gilt:

$$P(A\textrm{ «oder» }B) = P(A\cup B) = P(A) + P(B)$$

\subsubsection{Baumdiagramme}

\textbf{Pfadregel}:
Entlang eines Pfades (Äste hintereinander) multiplizieren sich die
Wahrscheinlichkeiten.

\textbf{Summenregeln}:
\begin{itemize}
\item Alle von einem Knoten ausgehenden Wahrscheinlichkeiten haben in
der Summe 100\%, also $P=1.0$.
\item Die Wahrscheinlichkeit des gesuchten Ereignisses ist gleich der
Summe der zugehörigen Pfadwahrscheinlichkeiten (aller gewünschten Endergebnisse).
\end{itemize}




\subsubsection{Laplace-Wahrscheinlichkeit}
Sind alle möglichen Ergebnisse eines Zufallsversuchs gleich
wahrscheinlich, so ist die Wahrscheinlichkeit $P(E)$ für das Ereignis
$E$ wie folgt zu berechnen:

$$P(E) = \frac{|E|}{|\Omega|}$$
In Worten:
$$P(E) = \frac{\textrm{Anzahl aller Ergebnisse
im Ereignis
}E}{\textrm{Anzahl aller Ergebnisse in }\Omega}$$


\subsubsection{Bernoulli-Modell}
\begin{gesetz}{Bernoulli}{}
$$P(X=k) = {n \choose k}\cdot{}p^k\cdot{}(1-p)^{n-k}$$
Taschenrechner
unter \tiprobutton{data_stat-reg-distr} \textbf{\texttt{binomialpdf}} wählen.

$n$ = Anzahl Züge\\
$p$ = Wahrscheinlichkeit des Treffers\\
$k$ = \textbf{genaue} gewünschte Anzahl Treffer

An jeder Verzweigung im Baum sind Treffer und Niete gleich
wahrscheinlich.

Im Urnenmodell: Mit Zurücklegen.
\end{gesetz}

\subsubsection{Kumulierte Wahrscheinlichkeit}

\begin{gesetz}{Kumulierter Bernoulli}{}
$$P(X\le k) = \sum_{i=0}^{k}{n \choose i}\cdot{}p^i\cdot{}(1-p)^{n-i}$$
Taschenrechner
unter \tiprobutton{data_stat-reg-distr} \textbf{\texttt{binomialcdf}} wählen.

$n$ = Anzahl Züge\\
$p$ = Wahrscheinlichkeit des Treffers\\
$k$ = \textbf{maximal} gewünschte Anzahl Treffer
\end{gesetz}

\subsubsection{Hypergeometrische Verteilung}
Bei
\begin{itemize}
\item $N$ Elementen (Kugeln),
\item $M$ möglichen Treffern,
\item $n$ gezogenen Elementen (Kugeln) und 
\item $k$ gewünschten Treffern
\end{itemize}
ergibt sich eine Wahrscheinlichkeit (genau $k$ Treffer zu ziehen) von:

$$P(X=k) = \frac{ {M \choose k} \cdot {{N-M}  \choose {n-k}} }{{N \choose n}}$$



\subsubsection{Kontingenztafel}
Vierfeld-Tafeln=Kontingenztafeln mit 4 Feldern.

Man hat zwei Merkmale, A und B. Jedes Merkmal kann nur zwei Werte
annehmen (+ / -). Dies ergibt vier Möglichkeiten. Die Vierfeldtafel
enthält die absoluten oder die relativen Häufigkeiten der vier
Kombinationen der Merkmalswerte.

Zudem werden noch die Randsummen notiert.

\bbwCenterGraphic{8cm}{img/kontingenztafel.png}

\subsubsection{Bedingte Wahrscheinlichkeit}

\begin{definition}{Bedingte Wahrscheinlichkeit}{}
Mit
$$P(A|B)$$
Wird die Wahrscheinlichkeit von $A$ angegeben, unter der Bedingung,
dass das Ereignis $B$ bereits eingetroffen ist.
\end{definition}

Beispiel:

\begin{tabular}{c|c|c|c}
             & gesund (G)& krank (K)& $\Sigma$ \\\hline
weiblich (W) &        30 &       40 &       70 \\\hline
männlich (M) &        25 &       35 &       60 \\\hline
$\Sigma$     &        55 &       75 &      130 \\\hline
 \end{tabular}

\textbf{Normale Wahrscheinlichkeit}\\
«Wie groß ist die Wahrscheinlichkeit,gesund (G) zu sein?»
$$P(G) = \frac{|G|}{|\Omega|} = \frac{55}{130}$$

\textbf{Wahrscheinlichkeit der Schnittmenge}\\
«Wie groß ist die Wahrscheinlichkeit, eine weibliche gesunde Person zu
treffen?»
$$P(G\cap W) = \frac{|G\cap W|}{|\Omega|} = \frac{30}{130}$$

\textbf{Bedingte Wahrscheinlichkeit}\\
«Wie groß ist die Wahrscheinlichkeit, unter den Gesunden, eine weibliche Person zu
treffen?»
$$P(W | G) = \frac{|W\cap G|}{|G|} = \frac{30}{55}$$
«Wie groß ist die wahrscheinlichkeit, unter den weiblichen Personen, 
eine gesunde Person zu treffen?»
$$P(G | W) = \frac{|G \cap W|}{|W|} = \frac{30}{70}$$



\end{multicols}

\end{document}
