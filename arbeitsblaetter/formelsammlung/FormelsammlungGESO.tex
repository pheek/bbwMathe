%%
%% Meta: Formelsammlung GESO BBW
%%

\newcommand{\versionsnummerFoSa}{2021-06-29 0.2.3}

\input{bbwSeite}

%%%%%%%%%%%%%%%%%%%%%%%%%%%%%%%%%%%%%%%%%%%%%%%%%%%%%%%%%%%%%%%%%%


%% Hintergrundfarbe in tabular
%%\usepackage[table]{xcolor}% http://ctan.org/pkg/xcolor

%%\usepackage{amssymb} %% für \blacktriangleright
\usepackage{makecell}

\renewcommand{\shortAuthor}{fp he @ bbw.ch [\versionsnummerFoSa{}]}

\renewcommand{\metaHeaderLine}{Formelsammlung (GESO \noTRAINER{2021}\TRAINER{\versionsnummerFoSa{}} BBW)}

\renewcommand{\arbeitsblattTitel}{Formelsammlung für die gesundheitlich-soziale BMS (working Draft)}

\titlespacing*{\section}{0pt}{0.1\baselineskip}{0.1\baselineskip}
\titlespacing*{\subsection}{0pt}{0.1\baselineskip}{0.1\baselineskip}

%% Millimeterpapier (2.8mm) für Notizen \mmPap
\newcommand{\mmPap}[1]{\mmPapierZwei{#1}{8.4}}


%% Listen nicht einrücken:
\setlist[itemize]{leftmargin=*}


\definecolor{FarnFarbe}{HTML}{4f8f00}

%%% MODERN:
\definecolor{none}{HTML}{eeeeee}
\renewcommand{\rmdefault}{phv} % Arial
\renewcommand{\sfdefault}{phv} % Arial
\definecolor{fsgray}{HTML}{eeeeee}
\renewcommand{\rezeptFarbe}{none}
%%\renewcommand{\definitionFarbe}{fsgray}
\renewcommand{\gesetzFarbe}{none}
\renewcommand{\beispielFarbe}{none}
\renewcommand{\bemerkungFarbe}{none}
%%% END MODERN


%% Benutze
%%\columnbreak
%% für harten Break:
%% Force column-break
\newcommand{\forceCB}{\vfill\null\columnbreak}

\begin{document}%%

\TRAINER{Alle Definitions-Boxen weg. Höchstens Trauerrand.}

\arbeitsblattHeader{}

\begin{multicols}{2}%%

\section*{Zahlmengen}
\begin{tcolorbox}[colback=white]
\textbf{Definition}: Die Zahlmengen heißen

$\mathbb{N} = \{1,2, ...\}$                                   \textbf{natürlich}\\
$\mathbb{Z} = \{..., -2, -1, 0, 1,2, ...\}$                   \textbf{ganz}\\
$\mathbb{Q} = \{\frac{a}b|a\in \mathbb{Z},b\in\mathbb{N}\}$   \textbf{rational} (Brüche)\\
$\mathbb{R} = \{..., -\frac12, ..., 0, ..., \sqrt{2}, ..., e,
... \pi, ...\}$\,\textbf{reell}\\
%%\phantom{$\mathbb{R}$} $=$ \textbf{reell}
\end{tcolorbox}

$$\mathbb{N} \subset \mathbb{Z} \subset \mathbb{Q} \subset \mathbb{R}$$

\hrulefill%% hrulefill vor allen neuen "sections" auf derselben Seite

\section*{Dezimalzahlen}
\subsection*{Runden}
Bsp.: Runden auf \textbf{\color{FarnFarbe}vier}  Dezimalen:
Betrachte die nächste Ziffer und runde auf, falls diese
{\color{red}Ziffer} $\ge 5$. Bsp.:

$3.\mathbf{\color{FarnFarbe}4729}\mathbf{\color{red}6}44 \approx
3.\mathbf{\color{FarnFarbe}4730}$; \hfill{ }
$70.\mathbf{\color{FarnFarbe}0031}\mathbf{\color{red}4}998\approx
70.\mathbf{\color{FarnFarbe}0031}$

(Alle vier Dezimalen sind anzugeben.)

\subsection*{Signifikante Stellen}
Führende Nullen zählen nicht, nachfolgende Nullen
zählen. Bsp. \textbf{\color{FarnFarbe}vier} signifikante Stellen:

$$\mathbf{\color{FarnFarbe}208.0} \textrm{ cm} = \mathbf{\color{FarnFarbe}2.080} \textrm{ m} = 0.00\mathbf{\color{FarnFarbe}2080} \textrm{ km}$$

%%\forceCB{}%%
\subsection*{Wissenschaftliche Notation}
Für große, aber auch für Zahlen sehr nahe an 0, wird die wissenschaftliche Schreibweise vorgezogen. Dabei steht vor dem Dezimalpunkt immer \textbf{\color{blue}genau eine Ziffer}:

\begin{tabular}{lcccr}
Zahl    & & wissenschaftl. & & TR: \tiprobutton{EE} \\
$3400$  &=& $\mathbf{\color{blue}3}.4\cdot{}10^3$ &=& \textbf{\color{blue}3}.4E3\\
$0.004$ &=& $\mathbf{\color{blue}4}\cdot{}10^{-3}$ &=& \textbf{\color{blue}4}E-3
\end{tabular}



\forceCB
\section*{Algebra}
\subsection*{Hierarchie der Rechenoperationen}
«Klapopustri»
(1.) Klammern, (2.) Wurzeln/Potenzen, (3.) Punktoperationen, (4.) Strichoperationen

\textbf{Termarten}

\begin{tabular}{rlrl}
$a+3x$  &Summe  & $5b^2-b$ & Differenz\\
$x(3-a)$&Produkt& $7:(b+1)$& Quotient\\
$(3x)^4$&Potenz &          &
\end{tabular}

\textbf{Achtung}

Differenzterm:  $3-(-7) = +10$

Produktterm:\, $-3(-7) = +21$


\subsection*{Grundrechenarten}

\textbf{Assoziativgesetz}:\\ $(a+b)+c = a+(b+c)$,\\ $(a\cdot{}b)\cdot{}c = a\cdot{}(b\cdot{}c)$\\
\textbf{Kommutativgesetz}:\\ $a+b = b+a$,\\ $a\cdot{}b= b\cdot{}a$\\
\textbf{Distributivgesetz}:

$a\cdot{}(x+ y) = a\cdot{}x + a\cdot{}y$,\phantom{ and } $a\cdot{}(x - y) = a\cdot{}x - a\cdot{}y$

\textbf{Vorzeichen}

$(+2)\cdot(+3)=+6$\\
$(-2)\cdot(-3)=+6$\\
$(+2)\cdot(-3)=-6$\\
$(-2)\cdot(+3)=-6$\\

%%$$\fbox{+}\fbox{+} = \fbox{+}$$
%%$$\fbox{-}\fbox{-} = \fbox{+}$$
%%$$\fbox{+}\fbox{-} = \fbox{-}$$
%%$$\fbox{-}\fbox{+} = \fbox{-}$$


\subsection*{Binomische Formeln}
\begin{tabular}{lcl}
 $(a+b)^2$  & =  & $a^2 + 2ab + b^2$ \\
 $(a-b)^2$  & =  & $a^2 - 2ab + b^2$ \\
 $(a+b)(a-b)$  & =  & $a^2 - b^2$
  \end{tabular} 

\subsection*{(Absolut)betrag einer Zahl}%%
%%
\hfill\, $|+7| = 7$ \hfill\, $|-7| = 7$ \hfill\, %%

\forceCB%%
\section*{Bruchterme}
    $\frac{a}b = a : b$,\phantom{ and }  $\frac{a}b : \frac{x}y = \frac{a}b\cdot{}\frac{y}x = \frac{ay}{bx}$
    
\subsection*{Faktorisieren}

\begin{enumerate}
\item gemeinsame Faktoren ausklammern:
\begin{itemize}
\item Zahlen und Variable:
$$ax + ay = a(x+y)$$
\item
Teilsummen ausklammern
$$3(7r+6) - b(7r+6) = (3-b)(7r+6)$$
\item $(-1)$ ausklammern
  $$(b-a)=(-1)\cdot{}(a-b)$$
  $$(b-a)^2 = \left((-1)\cdot{}(a-b)\right)^2 = (a-b)^2$$
\end{itemize}

\item Binomische Formeln
$$x^2-1 = (x-1)(x+1)$$
$$4b^2 -12bc + 9c^2=(2b-3c)^2$$
\item Zweiklammeransatz
$$x^2+2x-15 = (x+5)(x-3)$$
Faktoren mit Taschenrechner:\\ a) \texttt{poly-solv} \tiprobutton{cos_poly-solv}
oder\\ b) \texttt{PFactor} \tiprobutton{math}\tiprobutton{4}

\textbf{\color{red}Achte auf Vorzeichen!}
\end{enumerate}

%% Forced column break
%%\forceCB


\mmPap{5.2}

\forceCB
\section*{Potenzen}
\begin{tcolorbox}[colback=white]
$$a^n := \underbrace{a\cdot{}a\cdot{}a\cdot{}...\cdot{}a}_{n\textrm{ Faktoren}}$$
$$a^{-n} := \frac1{a^n}$$
  $$a^0 := 1$$
\end{tcolorbox}

\subsection*{gleiche Basis}
\begin{tabular}{cc}
$a^m\cdot{}a^n = a^{m+n},$ & $a^m:a^n=a^{m-n},$ \\
$\left(a^m\right)^n = a^{m\cdot{}n}$ &
 \end{tabular} 

\subsection*{gleicher\,Exponent}
\begin{tabular}{cc}
$a^n\cdot{}b^n = (a\cdot{}b)^n,$ & $\frac{a^n}{b^n} =\left(\frac{a}b\right)^n $
 \end{tabular}

\subsection*{Vorzeichen}
Fall: $n$ \textbf{gerade}:

\begin{tabular}{cc}
 $-(a^n) = -a^n,$ & aber: $(-a)^n = +a^n$
 \end{tabular} 

Fall: $n$ \textbf{un}gerade:

\begin{tabular}{cc}
 $-(a^n) = -a^n,$ & aber: $(-a)^n = -a^n$
 \end{tabular} 


\subsection*{negative Exponenten}

Für Basis $a\ne 0, b\ne 0$ und $m \in\mathbb{Q}$ gilt:

\begin{tabular}{cc}
$a^{-m} = \frac1{a^m},$ & $\left(\frac{a}b\right)^{-m} = \left(\frac{b}a\right)^m$
 \end{tabular}

Exponenten vertauschen ihr Vorzeichen beim Übertreten des Bruchstrichs:
$$\frac{a^{-3}b^2}{c^5d^{-6}} = \frac{b^2d^6}{a^3c^5}$$


\subsection*{$n$te Wurzel}
$\sqrt[n]{a} := \left(a\right)^\frac1n$\hfill{}Bsp.: $8^{\frac13}=\sqrt[3]{8}=2$

$$a^{\frac{m}n} = \sqrt[n]{a^m} = \left(\sqrt[n]a\right)^m$$
$$\sqrt[n]{\sqrt[m]{a}}   = \left((a)^\frac1m \right)^\frac1n = a^\frac1{nm} = \sqrt[m]{\sqrt[n]{a}}  $$

$$\sqrt[n]{a\cdot{}b} = \sqrt[n]{\vphantom{b}a} \cdot \sqrt[n]b$$

%%$$\sqrt[n]{a\cdot{}\sqrt[m]b} =   \sqrt[n]{\vphantom{b}a} \cdot \sqrt[n m]b$$
%%
 
%%\forceCB%%
\forceCB
\section*{Logarithmen}
\begin{tcolorbox}[colback=white]
  \textbf{Definition Logarithmus}\\
  Für $a>0, a\ne 1, z>0$ ist

$$\log_a{}(z)=x \Longleftrightarrow{} a^x = z$$
\end{tcolorbox}

Logarithmen sind Exponenten zu einer fest gewählten Basis.

\begin{tcolorbox}[colback=white]
\textbf{Gesetze}
$$\log_a(u\cdot v) = \log_a(u) + \log_a(v)$$
$$\log_a(u : v) = \log_a(u) - \log_a(v)$$
$$\log_a(b^x) = x\cdot{}\log_a(b)$$
$$\log_{\color{FarnFarbe}a}({\color{red}b}) = \frac{\lg({\color{red}b})}{\lg({\color{FarnFarbe}a})} = \frac{\ln({\color{red}b})}{\ln({\color{FarnFarbe}a})}$$
\end{tcolorbox}



\hrulefill
\section*{Gleichungen}

\subsection*{Lineare Gleichungen}

\textbf{Beispiele} ($x$ ist die Unbekannte)
$$2x+1= 2x+5$$
$$a^2x-b=b^2x+a$$


\subsection*{num-solv}
Der Taschenrechner löst Gleichungen mit Zahlen nach $x$ auf:

$5^x = 100 -x$; verwende \tiprobutton{2nd}\tiprobutton{sin_num-solv} $x\approx{}2.843$

\mmPap{5.6}
\forceCB
\subsection*{Quadratische Gleichungen}

\textbf{Grundform}: $ax^2 + bx+c = 0$

\begin{tcolorbox}[colback=white]
  \textbf{Lösungsformel}
  $$x_{1,2} = \frac{-b \pm \sqrt{b^2-4ac}}{2a}$$
\end{tcolorbox}
\textbf{Diskriminante} $D = b^2-4ac$\\
$D>0\Longrightarrow$ Zwei Lösungen;\\
$D=0\Longrightarrow$ Eine Lösung;\\
$D<0\Longrightarrow$ Keine Lösung


\subsection*{Bruchgleichungen}

\textbf{Definitionsbereich} $\mathbb{D}$: Welche Werte dürfen in einen Term eingesetzt werden

Bsp.: Nenner darf nicht Null werden:

$$\frac5{x+2}=\frac{7+x}{x-3} \Leftrightarrow{} \mathbb{D}_x=\mathbb{R}\backslash{}\{-2, 3\}$$

\subsection*{Potenzgleichungen}

  \TRAINER{ev. nur letztes Gesetz: $x^a = c$ und $x^\frac1b =c$ sind darin enthalten.  }

\begin{tabular}{rcl}
$x^a=c$ & $\Leftrightarrow$ & $x=\sqrt[a]{c}$\\
%%$x^\frac1b=c$&$\Leftrightarrow$&$x=c^b$\\
$x^{\frac{a}b} = c$&$\Leftrightarrow{}$&$x=c^{\frac{b}a} = \sqrt[a]{c^b}$
\end{tabular}

\subsection*{Exponentialgleichungen}
\subsubsection*{mit Exponentenvergleich}

$$5^4 = 5^{2x+1} \Longrightarrow  4=2x+1$$

\subsubsection*{durch Logarithmieren}
$$a^x=b \Rightarrow{} x=\log_a(b) = \frac{\lg(b)}{\lg(a)}$$

\noTRAINER{\mmPap{5.6}}


\forceCB


\subsection*{Lineare Gleichungssysteme}
Erst in \textbf{Grundform} bringen und dann
\gleichungZZ{5x+7y}{19}{16x -13y}{-4}
mit Taschenrechner lösen.

\subsection*{Substitution}
Beispiel Bruchgleichungssysteme:

\gleichungZZ{{\color{FarnFarbe}\frac1{x+1}} + {\color{red}\frac3{y-2}}}{66}{{\color{FarnFarbe}\frac2{x+1}} + {\color{red}\frac5{y-2}}}{16}
Substituiere
${\color{FarnFarbe}a}:={\color{FarnFarbe}\frac1{x+1}}$ und
${\color{red}b} : ={\color{red}\frac1{y-2}}$:

\gleichungZZ{{\color{FarnFarbe}a} + 3{\color{red}b}}{66}{2{\color{FarnFarbe}a} + 5{\color{red}b}}{16}

Dann: Rücksubstitution

\mmPap{14}

\forceCB

\section*{Textaufgaben}
\subsection*{Geschwindigkeitsaufgaben}
\fbox{$s = v\cdot{} t$} (Einheiten kompatibel wählen!)

\subsection*{Leistungsaufgaben}
Bei $x$ Zeiteinheiten für die ganze Arbeit:

Leistung = Arbeit pro Zeiteinheit

Leistung = $\frac{1 \textrm{ [Arbeit]}}{x \textrm{ [Zeiteinheiten]}}$

%%hysik: Leistung = $\frac{\textrm{[Joule]}}{\textrm{[Sekunde]}} = [\textrm{Watt}]$ \TRAINER{Watt ev weglassen?}

\subsection*{Restaufgaben}
$$A:B = C \textrm{ Rest } R \Leftrightarrow{}\frac{A-R}B=C$$


\subsection*{Zinseszins}
\begin{itemize}
\item $K_0$ = Startkapital; $K_n$ = Endkapital
\item $p$ = Zinssatz (in \%);\\ $a = 1+\frac{p}{100}$ = Zinsfaktor
(Wachstumsfaktor);\\ $p<0$, so ist $a$ der Zerfallsfaktor
\end{itemize}
$$K_n = K_0 \cdot{} \left( 1+\frac{p}{100} \right)^n = K_0\cdot{}a^n$$

Beispiele:
$$p = 2.5\% \Longleftrightarrow{} a = 1.025$$
$$p = -3\% \Longleftrightarrow{}  a = 0.97 $$

\mmPap{8.4}

\forceCB
\section*{Funktionen}

Eine Funktion ist eine {\color{red} eindeutige Zuordnung}. Dabei wird jedem {\color{FarnFarbe}x} aus der {\color{FarnFarbe}Definitionsmenge $\mathbb{D}$} eindeutig ein {\color{blue}y} aus der {\color{blue}Wertemenge $\mathbb{W}$} zugeordnet.

\bbwCenterGraphic{8cm}{FunktionDefinition.pdf}


\hrulefill
\section*{Lineare Funktionen}

$$g: y = f(x) = a\cdot{}x + b$$

\bbwCenterGraphic{8cm}{img/FunktionenLineareEinfuehrung.pdf}
Steigung \fbox{$a = \frac{\Delta y}{\Delta x} = \frac{y_2-y_1}{x_2-x_1}$}.\\
Zwei Geraden sind \textbf{parallel}, wenn ihre Steigungen gleich sind.

$b$ = $y$-Achsenabschnitt (=Ordinatenabschnitt)

\subsection*{Achsenschnittpunkte}

\textbf{Nullstelle}: Schnittstelle mit der $x$-Achse:\\ $y$ null setzen und auf\/lösen.

\textbf{Schnittpunkt mit der $y$-Achse}: 
$x=0$ setzen und $y$ ausrechnen.


\subsection*{Schnittpunkt zweier Geraden}
Gegeben: $g: y=ax+b$ und $h: y=cx+d$.

Gleichsetzen: $ax+b = cx+d$


\subsection*{Mittelpunkt $M$ einer Strecke $\overline{AB}$}
Gegeben: $A=(x_A|y_A)$ und $B=(x_B|y_B)$

Gesucht: Mittelpunkt $M=(x_M|y_M)$

Lösung Mittelwert $x$ und $y$ separat:

$$x_M = \frac{x_A+x_B}2; y_M=\frac{y_A+y_B}2$$

\subsection*{Horizontale Gerade}

$g: y=ax+b$ ist horizontal, wenn die Steigung $a$ Null ist.

$$g:  y=0\cdot{}x+b \Rightarrow y=b$$

\subsection*{Zweipunkte Aufgaben}
Gesucht Gerade $g: y=ax+b$ durch zwei gegebene Punkte $A=(x_A|y_A)$
und $B=(x_B|y_B)$

\textbf{Variante 1}: Steigung $a$ berechnen und einen der beiden Punkte in
$y=ax+b$ einsetzen, um $b$ zu finden.

\textbf{Variante 2}: Gleichungssystem nach $a$ und $b$ auf\/lösen:
\gleichungZZ{y_A}{a\cdot{}x_A+b}{y_B}{a\cdot{}x_B + b}

\mmPap{10}

%%%%%%%%%%%%%%%%%%%%%%%%%%%%%%%%%%%%%%%%%%%%%%%%%%%%%%%%%%%%%%%%%%%%%%%%%%%%%%%%%%%%%%%%%%%%%%%%%%%%%%%%%%%%%%%%%%
\forceCB
\section*{Potenzfunktionen}

\subsection*{Quadratische Funktion}
$$y=ax^2$$
$a$ heißt Öffnung\\
($a<0$: Die Parabel ist nach unten geöffnet)

$a=1$ oder $a=-1$: Normalparabel
%% Fülle, damit nicht vertikal auseinandergezogen
\vspace*{\fill}

\begin{tcolorbox}[colback=white]
  \textbf{Grundform der Potenzfunktion}
$$y=ax^m$$
$m\in\mathbb{Z}\backslash\{0,1\} = \{...-2, -1, 2, 3, 4, ...\}$
\end{tcolorbox}

$m$ gerade: Achsensymmetrisch ($y$-Achse)

$m$ ungerade: Punktsymmetrisch (Ursprung $O=(0|0)$)

%%Ist ${\color{green}a>0}$, so sind Graphen mit
%%geraden Exponenten nach oben geöffnet bzw. mit ungeraden Exponenten im
%%Quadranten I und III.

Ist ${\color{blue}a<0}$, so werden Graphen der folgenden Funktionen an der $x$-Achse gespiegelt.

\end{multicols}

\begin{tabular}{cc}
$m$ > 0: Parabel ($m$ gerade) & $m$ > 0: Parabel ($m$ ungerade)\\   %%\bbwCenterGraphic{8cm}{PotenzFkt.pdf}
  \includegraphics[width=8cm]{PotenzFktx2x4.pdf} &
  \includegraphics[width=8cm]{PotenzFktx3x5.pdf}\\

$m<0$: Hyperbel  ($m$ gerade) & $m<0$: Hyperbel ($m$ ungerade)\\
  \includegraphics[width=8cm]{PotenzFkt1overx2x4.pdf}&
  \includegraphics[width=8cm]{PotenzFkt1overx3x5.pdf}
  \end{tabular}

%% \includegraphics[width=4cm]{allg/funktionen/img/potenzfct/potenzFunktionenGerade.png}\hfill{}\includegraphics[width=4cm]{allg/funktionen/img/potenzfct/potenzFunktionenUngerade.png}


%%\hrulefill  
\newpage

\begin{multicols}2

  \subsection*{Transformationen}
  Der Faktor $a$ in $y=a\cdot{}x^z$ streckt die Funktion in $x$-Richtung, bzw. spiegelt die Funktion an der $x$-Achse.
  \bbwCenterGraphic{8cm}{PotenzFktTransformation.pdf}

  \hrulefill
  
\section*{Wachstumsmodelle}

\textbf{Lineares Wachstum}
$$f(t) = a\cdot t + b$$
Bei linearem Wachstum wird pro Zeiteinheit der selbe Wert $a$ \textbf{addiert}.

\textbf{Exponentielles Wachstum}

$$f(t) = b\cdot{}a^t$$
Bei exponentiellem Wachstum wird pro Zeiteinheit mit dem selben Wert $a$ \textbf{multipliziert}.

  \begin{tabular}{rl}
   $b$  & Startwert bei $t=0$; $b=f(0)$\\
        & \phantom{$b$} = $y$-Achsenabschn.\\
   $a$  & beschreibt das Wachstum
  \end{tabular}

  \bbwCenterGraphic{8cm}{ExpFktVergleichMitLinearemWachstum.pdf}
  \TRAINER{f(x) noch durch f(t) ersetzen.}

  \subsection*{Exponentialfunktionen}
%%\bbwCenterGraphic{8.5cm}{allg/funktionen/img/exp/exponentielles_wachstum.png}
\begin{tcolorbox}[colback=white]$$y=f(t) = b\cdot{}a^t$$\end{tcolorbox}

Wachstum: $a>1$, Zerfall: $0<a<1$


\begin{tcolorbox}[colback=white]
  \textbf{Beobachtungsintervall}
  
  $$f(t) = {\color{blue}b}\cdot{}{\color{red}a}^{\frac{t}{\color{FarnFarbe}\tau}}$$
  \begin{tabular}{ccp{60mm}}
$\color{FarnFarbe}\tau$ &=& Beobachtungsintervall zu $\color{red}a$\\
    $\color{red}a$ &=& Wachstums(bzw. Zerfalls)rate im Zeit\-raum $\color{FarnFarbe}\tau$
    \end{tabular}

Beispiel: Die Algen mit Startwert von ${\color{blue}3} \textrm{ m}^2$ ver{\color{red}vier}fachen
sich alle {\color{FarnFarbe}fünf} Tage:\\
${\color{blue}b}={\color{blue}3}$, ${\color{red}a}={\color{red}4}$, ${\color{FarnFarbe}\tau}={\color{FarnFarbe}5}$
$$f(t)= {\color{blue}3}\cdot{}{\color{red}4}^\frac{t}{\color{FarnFarbe}5}$$
\end{tcolorbox}


\subsection*{Basis $e$ Zeitvariable oft $x$}

$$y=c+b\cdot{}a^x = c + b\cdot{}e^{mx}$$ $\textrm{ mit } m = \ln(a)
\,\,(\textrm{bzw. } a = e^m)$

bzw.

$$y=c+b\cdot{}a^\frac{x}\tau = c+b\cdot{}e^{mx}$$ $\textrm{ mit }
m=\ln\left(a^\frac1\tau\right) = \frac1\tau \cdot{}\ln(a)
\,\,(\textrm{bzw. } a = e^{m\tau})$

\subsection*{Halbwertszeit}
$\frac12 \cdot{} b = f(t) = b\cdot{}a^{\frac{t}{\tau}}\phantom{xxx}\Longrightarrow\phantom{xxx}t= \tau\cdot{}\log_a\left(\frac12\right)$

\subsection*{Verdopplungszeit}
$2\cdot{}b = f(t) = b\cdot{}a^{\frac{t}{\tau}}\phantom{xxx}\Longrightarrow\phantom{xxx}t = \tau\cdot{}\log_a(2)$
\forceCB
\subsection*{Sättigungsfunktionen}


\bbwCenterGraphic{7.5cm}{img/ExpFunktionBeschraenktesWachstum.pdf}

\subsubsection*{Beschränktes Wachstum (Zerfall)}


\textbf{Zerfallsrate} immer: $0<a<1$

\textbf{Sättigungsgrenze}: $c$

\textbf{Wachstum}:
$$f(t) = c-ba^\frac{t}\tau$$
$$f(0) = c-b$$
\textbf{Zerfall}:
$$f(t) = c+ba^\frac{t}\tau$$
$$f(0) = c+b$$


%%\textbf{Sättigungsdefizit}
%%$b$ = Differenz zu $c$ bei $t=0$.\\
%%$b\cdot{}a^\frac{t}\tau = m(t)$ = Sättigungsdefizit zur Zeit $t$

\end{multicols}

\mmPapier{8}

\newpage
\section*{Datenanalyse}

\subsection*{Skalen}
\bbwCenterGraphic{17cm}{img/Skalentyp.pdf}
%%\bbwCenterGraphic{15cm}{img/DatenanalyseSkalentypen.eps}
\TRAINER{Stetig/Diskret?}
\TRAINER{Mögliche Diagramm-Arten?}
\vspace{5mm}


\hrulefill
\begin{multicols}{2}

\subsection*{Lagemaße}
\subsubsection*{Mittelwert}
(= Durchschnitt = arithmetisches Mittel)


$$\overline{x} = \frac{x_1 + x_2 + x_3 + ... + x_n}{n}=  \frac{\sum\limits_{k=1}^nx_n}n$$


\subsubsection*{Median}


Der \textbf{Median} (Zentralwert) ist der Wert in der Mitte der geordneten Datenreihe. Ist
die Anzahl Werte gerade, so wird der Mittelwert der beiden in der
«mittleren» Werte genommen.

Symbol: $\mediantilde{x} = x_{\textrm{MED}}$


\subsubsection*{Quartil}

Die Quartilsgrenzen $Q_1$ bzw. $Q_3$ sind die Mediane der linken,
bzw. rechten Datenhälfte (nachdem der eine Zentralwert entfernt
wurde).

\subsubsection*{Modus}
Der am häufigsten auftretende Wert wird als Modus bezeichnet.

Manchmal ist dies nicht eindeutig, dann sprechen wir von einer
\textbf{multimodalen} Verteilung. Bei zwei Modi von \textbf{bimodal}.

\subsubsection*{Maximum, Minimum}
Maximum bzw. Minimum sind höchste bzw. kleinste Datenwerte: Die Ausreißer
zählen mit!

\subsection*{Streumaße}
\subsubsection*{Spannweite}
Die Spannweite $R$ (Range) ist das Maximum minus
das Minimum.

\subsubsection*{Interquartilsabstand}
Der Abstand zwischen $Q_1$ und $Q_3$ wird als
Quartilsdifferenz \textbf{QD} oder
englisch \textit{Inter Quartile Range} = \textbf{IQR} bezeichnet.

\subsubsection*{Standardabweichung}
Mit $\sigma$ bzw. $s$ wird die Standardabweichung bezeichnet.

$\sigma$ = Standardabweichung der Grundgesamtheit

$s$ = Standardabweichung einer gemessenen Stichprobe \TRAINER{Taschenrechner großes S}


\subsubsection*{Robustheit}
Verändert sich der Wert eines statistischen Maßes nicht, wenn sich ein Ausreißer
weiter ins Extreme bewegt, so sprechen wir von einem \textbf{robusten} Maß.

%\begin{tabular}{|c|c|c|}\hline
%   & Lage- & Streumaß\\\hline
% \multirow{4}{*}{\rotatebox{90}{robust}}  & Median $\mediantilde{x}$ & IQR \\
%    & Modus $x_{mod}$ & \\
%    & Quartile ($Q_1$,$Q_3$) & \\
%    & (Ausreißerschwellen) & \\\hline
% \multirow{3}{*}{\rotatebox{90}{«fragil»}}  & Mittelwert
% $\overline{x}$ & Standardabw.: $\sigma$\\
%    & Minimum & Spannweite\\
%
% & Maximum & \\\hline
% \end{tabular}

\subsection*{Summenzeichen}
%%$$\color{orange}\sum_{\color{blue}k=1}^{\color{blue}n} {{\color{FarnFarbe}T}(k)} = {\color{FarnFarbe}T}({\color{blue}1}) {\color{orange}+} {\color{FarnFarbe}T}({\color{blue}2}) {\color{orange}+} ... {\color{orange}+} {\color{FarnFarbe}T}({\color{blue}n})$$

  $${\color{FarnFarbe}\sum_{{\color{red}k}{{\color{FarnFarbe}{\color{black}=}\color{blue}1}}}^{\color{blue}n\color{FarnFarbe}}}
  {\color{orange}T(}{{\color{red}k}}{\color{orange})} = {\color{orange}T(}{\color{blue}1}{\color{orange})} {\color{FarnFarbe}+} {\color{orange}T(}{\color{blue}2}{\color{orange})} {\color{FarnFarbe}+}
  ... {\color{FarnFarbe}+} {\color{orange}T(}{\color{blue}n}{\color{orange})}$$

Beispiel

  $${\color{FarnFarbe}\sum_{{\color{red}n}{{\color{FarnFarbe}{\color{black}=}\color{blue}3}}}^{\color{blue}5\color{FarnFarbe}}}  {\color{red}n}^{{\color{orange}7}} = {\color{blue}3}^{\color{orange}7} {\color{FarnFarbe}+} {\color{blue}4}^{\color{orange}7} {\color{FarnFarbe}+} {\color{blue}5}^{\color{orange}7}$$


Taschenrechner (\texttt{sum}): \tiprobutton{math}\tiprobutton{5}

\mmPap{8}

\forceCB
\subsection*{Diagramme}
\subsubsection*{Balken / Kuchen}
\begin{tcolorbox}[colback=white]
  \textbf{Relative Häufigkeit}\\
\begin{tabular}{lcl}
$n$   &$=$& Anzahl Werte\\
$h_i$ &$=$& Absolute Häufigkeit\\
      & & des Merkmals $i$\\
$f_i = \frac{h_i}n$ &$=$& relative Häufigkeit\\
$p_i = f_i\cdot{}100\%$ &$=$& prozentuale Häufigkeit\\
$\varphi_i = f_i\cdot{}360\degre$ &$=$& Zentriwinkel im\\
      & &Kuchendiagramm
  \end{tabular}
\end{tcolorbox}



\subsubsection*{Histogramm}
\bbwCenterGraphic{8cm}{img/Histogramm.pdf}
\TRAINER{Entferne «Sportler*innen»}

\begin{itemize}
\item Alle Balken sind gleich breit und berühren sich.
\item Die Höhe der Balken wird durch die (absolute / relative) Häufigkeit der entsprechenden
Werte festgelegt.
\item Werte auf Balkengrenzen werden immer rechts mitgezählt.
\item Das obige Histogramm ist \textbf{linksschief} und rechtssteil.
\end{itemize}


%Vorgehen
%\begin{enumerate}
%\item Daten in TR eingeben \tiprobutton{data}...
%\item ... und mit \tiprobutton{2nd} \tiprobutton{data_stat-reg-distr} auslesen.
%\item {\color{orange} Median ($\mediantilde{x} = x_{\textrm{MED}}$)},
%und Quartile ({\color{blue}$Q_1$}, {\color{FarnFarbe}$Q_3$})einzeichnen
%\item IQR := {\color{FarnFarbe}Q3}-{\color{blue}Q1} rechnen
%\item IQR mit 1.5 multiplizieren
%\item {\color{red} untere Ausreißerschwelle\\ uAs} $ = Q_1 -
%1.5\cdot{}\textrm{IQR}$\\ und {\color{red} obere Ausreißerschwelle\\ oAs} $=Q_3 + 1.5\cdot{}\textrm{IQR}$\\ (ausradierbar) fein markieren (gehören nicht zum Boxplot)
%\item Alle Ausreißer mit Ring $\circ$ oder Stern $\star$ einzeichnen.
%\item Whisker (entfernteste Werte innerhalb der Ausreißerschwellen)
%einzeichnen.
%\item Boxdekoration horizontale Linien einzeichnen.
%\end{enumerate}
%\bbwCenterGraphic{8cm}{img/Boxplot.pdf}

\end{multicols}


\newpage
\subsubsection*{Boxplot}
%%\TRAINER{«Schritte zum Boxplot als Fließtext
%%  daneben. Ausreißerschwellen {\color{blue}blau} im
%%  Diagramm. Dann im Text auch. Ausreißerschwellen ist ein Begriff von
%%  Frommenwiler, nicht von Marthaler.}%% END TRAINER
Schritte zum Boxplot:\\
Fühler (Whisker) werden auf $1.5\cdot\textrm{IQR}$ begrenzt ({\color{blue} Ausreißerschwellen blau}),
danach nach \textit{\color{FarnFarbe}innen} gehen: Fühler sind die von der Box
entferntesten Nicht-Ausreißer ({\color{FarnFarbe}grün}). Ausreißer
({\color{red} rot}) werden markiert.
\bbwCenterGraphic{18cm}{BoxplotMitStreudiagramm3.pdf}
%%\TRAINER{Entferne «Sportler*innen»}

\hrulefill

\begin{multicols}{2}
\section*{Stochastik}

\subsection*{Kombinatorik}

\subsubsection*{Produktregel}
Hat eine zweistufige Situation zunächst $k_1$ Möglichkeiten, gefolgt
von $k_2$ Möglichkeiten, so hat man insgesamt $k_1\cdot{}k_2$
Möglichkeiten. Dies ist auf beliebig viele Stufen verallgemeinerbar.

\begin{tabular}{p{5cm}c}
Andrea hat drei verschiedene Hosen und vier verschiedene Pullover.
Es gibt somit $3\cdot{}4$ mögliche Kombinationen.&\raisebox{-3.5cm}{\includegraphics[width=3cm]{img/KombinatorikMultiplikation.pdf}}\end{tabular}

%%\bbwCenterGraphic{3cm}{img/KombinatorikMultiplikation.pdf}
%%\TRAINER{Ersetze «Ein:e Schüler:in» durch «Andrea»}
\forceCB
\subsubsection*{Permutationen (Sitzordnungen)}

Bei $n$ Elementen gibt es $n!$ Permutationen (=Vertauschungen):

\begin{tcolorbox}[colback=white]
  \textbf{Fakultät}\\
  $$n! = n\cdot{}(n-1)\cdot{}(n-2)\cdot{}(n-3)\cdot{} ... \cdot{}2\cdot{}1$$
  $$1! = 1\phantom{xxx}0! = 1$$
\end{tcolorbox}

\def\missifarbig{M{\color{FarnFarbe}i}{\color{red}ss}{\color{FarnFarbe}i}{\color{red}ss}{\color{FarnFarbe}i}{\color{cyan}pp}{\color{FarnFarbe}i}}
«\missifarbig»-Formel: Sind einige Buchstaben gleich, so reduziert
sich die Anzahl der Möglichkeiten. Wie viele verschiedene Wörter sind mit den Buchstaben des Wortes «\missifarbig» bildbar?
$$N = \frac{11!}{{\color{FarnFarbe}4}!\cdot{}{\color{red}4}!\cdot{}{\color{cyan}2}!}$$

\end{multicols}
\newpage


\subsection*{Überblick Auswahlprobleme}
$$\textrm{Binomialkoeffizient:} {{\color{blue}n}\choose {\color{red}k}} :=\frac{{\color{blue}n}!}{({\color{blue}n}-{\color{red}k})!\cdot {\color{blue}n}!}\phantom{etwas Platz lassen} {n\choose 0}= {n\choose n}= 1$$
\bbwCenterGraphic{17cm}{img/Kombinatorik.pdf}

\hrulefill



\begin{multicols}{2}


\subsection*{Wahrscheinlichkeit}
\subsubsection*{Grundbegriffe}

\textbf{Ergebnismenge} $\Omega$: Menge der möglichen Ergebnisse
(Ausgänge) eines Zufallsversuchs.\\
Beispiel: Die sechs Würfelergebnise
eines Spielwürfels: $\Omega = \left\{\epsdice{1}, \epsdice{2}, \epsdice{3}, \epsdice{4}, \epsdice{5}, \epsdice{6}\right\}$

\textbf{Ereignis} $A$: Eine Teilmenge von $\Omega$. Beispiel: Eine
ungerade Zahl zu werfen: $A  = \left\{\epsdice{1}, \epsdice{3}, \epsdice{5}\right\}$

\textbf{Gegenereignis} $\overline{A} = \Omega \backslash A$ =
\textit{Gegenmenge}

$\overline{A}$ sprich «nicht» $A$.

Es gilt $A \cup \overline{A} = \Omega$

\textbf{Elementarereignis}: Ereignis bestehend aus einem einzigen
Ergebnis aus $\Omega$. Beispiel: $E_1$ = Wirf eine \textbf{Eins}: $E_1
= \left\{\epsdice{1}\right\}$.

\forceCB
\subsection*{Elementare Wahrscheinlichkeit}
Die Wahrscheinlichkeit des \textbf{Gegenereignisses} von $E$ ist
$$P(\overline{E}) = 1- P(E)$$
Bei Aufgaben wie «... mindestens eins ...» ist das Arbeiten mit der
Gegenwahrscheinlichkeit häufig einfacher.

Ereignisse sind voneinander \textbf{unabhängig}, wenn sie keine
gleichen Ergebnisse aufweisen.
$$A \textrm{ unabhängig von } B \Leftrightarrow A\cap B=\{\}$$

Für voneinander \textbf{unabhängige} Ergebnisse 
gilt:

$$P(A\textrm{ «oder» }B) = P(A\cup B) = P(A) + P(B)$$

\forceCB
\subsubsection*{Baumdiagramme}

\bbwCenterGraphic{8cm}{img/WahrscheinlichkeitBaumdiagramm.pdf}

Wahrscheinlichkeiten entlang eines Pfades werden multipliziert (grün)
$$P(KK) = 0.6\cdot0.6=0.36$$
Wahrscheinlichkeiten verschiedener (rot) Pfade werden addiert.
$$P(KK oder ZZ) = P(KK) + P(ZZ) =0.6\cdot0.6 + 0.4\cdot0.4$$

\subsection*{Laplace-Wahrscheinlichkeit}
Sind alle möglichen Ergebnisse eines Zufallsversuchs gleich
wahrscheinlich (fairer Würfel), so ist die Wahrscheinlichkeit $P(E)$ für das Ereignis
$E$ wie folgt zu berechnen:

$$P(E) = \frac{\textrm{Anzahl Ergebnisse in
}E}{\textrm{Anzahl Ergebnisse in }\Omega}$$


\subsection*{Bernoulli-Modell}

\begin{tcolorbox}[colback=white]
\textbf{Bernoulli}\\
$k$ = \textbf{genaue} gewünschte Anzahl Treffer

$$P(X=k) = {n \choose k}\cdot{}p^k\cdot{}(1-p)^{n-k}$$
Taschenrechner:  \textbf{\texttt{binomialpdf}}

$n$ = Anzahl Züge\\
$p$ = Wahrscheinlichkeit des Treffers\\

An jeder Verzweigung im Baum sind Treffer und Niete gleich
wahrscheinlich.

Im Urnenmodell: Mit Zurücklegen.
\end{tcolorbox}

\subsubsection*{Kumulierte Wahrscheinlichkeit}

\begin{tcolorbox}[colback=white]

$k$ = \textbf{maximal} gewünschte Anzahl Treffer


$$P(X\le k) = \sum_{i=0}^{k}{n \choose i}\cdot{}p^i\cdot{}(1-p)^{n-i}$$
Taschenrechner: \textbf{\texttt{binomialcdf}}

$n$ = Anzahl Züge\\
$p$ = Wahrscheinlichkeit des Treffers\\
\end{tcolorbox}


Bei \textbf{minimal} arbeite mit \textbf{Gegenereignis}.

\textbf{Achtung}: Gegenereignis von «minimal 4 Mal» = «maximal 3 Mal».
$$P(X \ge 4) = 1 - P(X < 4) = 1-P(X\le 3)$$

\forceCB
\subsection*{Hypergeometrische Verteilung}
Gegeben $N+T$ Kugeln in einer Urne. Herausgezogen werden $n+t$:
\begin{itemize}
\item $T$ Anzahl aller Treffer
\item $N$ Anzahl aller Nieten
\item $t$ gewünschte Treffer
\item $n$ «gewünschte» Nieten
\end{itemize}
Wahrscheinlichkeit \textbf{genau} $t$ Treffer zu ziehen:

$$P(X=t) = \frac{ {T \choose t} \cdot {N  \choose n} }{{T+N \choose t+n}}$$


\subsection*{Vierfeldtafeln}
Vierfeld-Tafeln=Kontingenztafeln mit 4 Feldern.

Man hat zwei Merkmale, A und B. Jedes Merkmal kann nur zwei Werte
annehmen (+ / -). Dies ergibt vier Möglichkeiten. Die Vierfeldtafel
enthält die absoluten oder die relativen Häufigkeiten der vier
Kombinationen der Merkmalswerte.

Zudem werden noch die Randsummen notiert.

\bbwCenterGraphic{8cm}{img/kontingenztafel.png}


\subsection*{Bedingte Wahrscheinlichkeit}

\begin{tcolorbox}[colback=white]
  \textbf{Bedingte Wahrscheinlichkeit}\\
Mit
$$P(A|B)$$
Wird die Wahrscheinlichkeit von $A$ angegeben, unter der Bedingung,
dass das Ereignis $B$ bereits eingetroffen ist.
\end{tcolorbox}

Beispiel:

\begin{tabular}{c|c|c|c}
           & gesund (G)& krank (K)& $\Sigma$ \\\hline
Frauen (F) &        30 &       40 &       70 \\\hline
Männer (M) &        25 &       35 &       60 \\\hline
$\Sigma$   &        55 &       75 &      130 \\\hline
 \end{tabular}

\textbf{Normale Wahrscheinlichkeit}\\
«Wie groß ist hier die Wahrscheinlichkeit, gesund (G) zu sein?»
$$P(G) = \frac{\textrm{Anzahl Gesunde}}{\textrm{Alle}} =  \frac{55}{130}$$

\textbf{Wahrscheinlichkeit der Schnittmenge}\\
«Wie groß ist hier die Wahrscheinlichkeit, eine gesunde Frau anzutreffen treffen?»
$$P(G\cap F) = \frac{\textrm{\ Anzahl gesunde Frauen}}{\textrm{Alle}}= \frac{30}{130}$$

\textbf{Bedingte Wahrscheinlichkeit}\\
«Wie groß ist die Wahrscheinlichkeit, unter den Gesunden eine Frau anzutreffen?»
$$P(F | G) = \frac{\textrm{Anzahl gesunde Frauen}}{\textrm{Anzahl Gesunde}} =  \frac{30}{55}$$
«Wie groß ist die Wahrscheinlichkeit, unter den Frauen eine gesunde Person zu treffen?»
$$P(G | F) = \frac{\textrm{Anzahl gesunde Frauen}}{\textrm{Anzahl Frauen}}=  \frac{30}{70}$$


\end{multicols}


%% Leerseiten für Notizen
\newpage

%% TASCHENRECHNER

\section*{Taschenrechner}

\begin{tabular}{cc}


\raisebox{-45mm}{\includegraphics[width=45mm]{img/tiprobuttonimages/ti30.png}}
&

\begin{tabular}{c|c|p{80mm}}\hline
\tiprobutton{math}         &math              & Summenzeichen $\Sigma$ \\\hline
\tiprobutton{data}         &data              & Daten eingeben\\
                           &\texttt{stat-reg} & 1-VAR STATS anwählen und calc drücken für Standardabweichung, Q1, Q3, Median etc.\\
                           &\texttt{distr}    & Bernoulliformel (4: Binomial\textbf{p}df)\\
                           &                  & Kumulierte: (5: Binomial\textbf{c}df)\\\hline
\tiprobutton{cos_poly-solv}&\texttt{poly-solv}&  Quadratische Gleichungen in Grundform $ax^2+bx+c=0$\\\hline
\tiprobutton{tan_sys-solv} &\texttt{sys-solv} &  Lineare Gleichungssysteme\\\hline
\tiprobutton{sin_num-solv} &\texttt{num-solv} &  Gleichungen mit Zahlen nach $x$ auf\/lösen\\\hline
\tiprobutton{table}        & table            &  Wertetabelle für lineare Gleichungen, lineare Funktionen, diverse Funktionen\\\hline
\tiprobutton{ncrnpr}       &Kombinatorik      &  Fakultät: {\color{red}\textbf{!}}\\
                           &                  &  $n \choose k$: Tippe $n$ nCr $k$\\\hline
\tiprobutton{ln_log}       &  Logarithmen     &  log = Zehnerlogarithmus ($\lg$)\\
                           &                  &  ln  = Log. zur Basis $e$\\
                           &                  &  3x drücken = log zu beliebiger Basis\\\hline
\tiprobutton{sto_recall}   & Werte speichern  &  Erst \texttt{sto} drücken, danach können die Variable gespeichert werden (\tiprobutton{xyzabcd}). \\\hline
\tiprobutton{xyzabcd}      &                  & gespeichert und wieder ausgelesen und verwendet werden.
\end{tabular}
\end{tabular}


\TRAINER{Generelle Fragen:\\
Fehlen Lernziele?\\
Sind Lernziele zu viel?\\
Sind Beispiele zu viel/zu wenig?\\
Hat es genügend Platz für die SuS (Schülerinnen und Schüler) für weitere Notizen/Beispiele/Rezepte/Vorgehen?\\
Ganz am Schluss:

* Rechtschreibeprogramm

* Zeilenumbrüche

* Einzüge (0pt bei Aufzählungen und Nummerierungen)

* Absatz-Umbrüche

* Seitenumbrüche

* Notizraster für eigene Notizen


}



\noTRAINER{\vspace{15mm}}
Eigene Notizen

\noTRAINER{\mmPapier{5.2}}

\newpage

\mmPapier{20}

\vspace{5mm}

Fehler gefunden? \texttt{philipp.freimann@bbw.ch / christian.hersberger@bbw.ch}
%%v 0.0.29 DRAFT 2021-06-22 fp)}

\TRAINER{
{\footnotesize Quelltext dieser Formelsammlung:\\
\texttt{https://github.com/pheek/bbwMathe/tree/main/arbeitsblaetter/formelsammlung}}

{\footnotesize Version \versionsnummerFoSa{}}
}%% END TRAINER


\end{document}
