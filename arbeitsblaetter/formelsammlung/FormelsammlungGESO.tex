%%
%% Meta: TI nSpire Einführung
%%       Ziel: Damit die Grundoperationen damit durchgeführt werden können.
%%             Damit man sich an den Rechner gewöhnt.
%%

\input{bbwSeite}

%%%%%%%%%%%%%%%%%%%%%%%%%%%%%%%%%%%%%%%%%%%%%%%%%%%%%%%%%%%%%%%%%%

\usepackage{amssymb} %% für \blacktriangleright
\renewcommand{\metaHeaderLine}{Formelsammlung GESO}
\renewcommand{\arbeitsblattTitel}{Abschlussprüfung V. 0.0.0 2021-06-08 fp}

\begin{document}%%
\arbeitsblattHeader{}
 \begin{multicols}{2}

\section{Zahlmengen}
%% Benutze
%%\columnbreak
%% für harten Break

$\mathbb{N} = \{1,2, ...\}  $ = Natürliche Zahlen\\
$\mathbb{Z} = \{..., -2, -1, 0, 1,2, ...\}  $ = Natürliche Zahlen\\
$\mathbb{Q} = \{\frac{a}b | a\in \mathbb{Z}, b\in\mathbb{N}\}$ Menge der Brüche (\textbf{rationale} Zahlen)\\
$\mathbb{R} = \{..., -2, ..., -1, ...., 0,
\frac12, 1,...,\sqrt{2},...,2, e,\\
..., 3, ..., \pi, ...\}  $ = \textbf{reelle} Zahlen = alle Zahlen auf der Zahlengerade\\

Eine natürliche Zahl heißt Primzahl, wenn sie genau zwei Teiler hat (Eins und sich selbst). $\mathbb{P} = \{2, 3, 5, 7, 11, 13, ...\}$

\hrulefill

\section{Algebra}
\subsection{Binomische Formeln}

$$(a+b)^2 = a^2+2ab + b^2$$
$$(a-b)^2=a^2-2ab+b^2$$
$$(a+b)\cdot(a-b) = a^2 - b^2$$

\hrulefill

\subsection{Vertauschte Differenz}
(-1 ausklammern)

$(b-a)=(-1)\cdot{}(h-b)$

Beispiel:

$$\frac{x}{a-b} +  \frac{y}{b-a} = \frac{x}{a-b} + \frac{-y}{a-b} = \frac{x-y}{a-b}$$

\hrulefill

\subsection{Summenzeichen}
$$\sum_{k=1}^n{T(k)} = T(1) + T(2) + ... + T(n)$$
Beispiel

$$\sum_{i=3}^7{i^2} = 3^2 + 4^2 + 5^2 + 6^2 + 7^2$$

Taschenrechner: \tiprobutton{math} 4


\hrulefill

\subsection{(Absolut)betrag einer Zahl}

$|a| = a$ falls $a \ge 0$

$|a| = (-1)\cdot{}a$ falls $a < 0$


\section{Dezimalzahlen}
\subsection{Runden}
Dezimalen sind die Stellen nach dem Komma.
Runden auf $n$ Dezimalen: betrachte die $(n+1)$-te Dezimale und runde auf, wenn diese $\ge 5$.

Beispiel

3.4729\textbf{6}44 auf vier Dezimalen, so ist die 5. Dezimale (hier die Ziffer 6) zu beachten:

$$3.4729\textbf{6}44 \approx 3.4730$$
(Alle vier Dezimalen sind anzugeben.)

\hrulefill

\subsection{Signifikante Stellen}
Die Zählung der signifikanten Stellen einer Zahl beginnt (von links) mit der ersten von Null verschiedenen Ziffer. Ab hier zählen alle Ziffern, auch die 0, bis zur verlangten Rundungsstelle.

\subsection{Wissenschaftlicher Schreibweise}
Für große, aber auch für Zahlen sehr nahe an 0, wird die Wissenschaftliche Schreibweise vorgezogen. Dabei steht vor dem Dezimalpunkt immer genau eine Ziffer:

\begin{tabular}{lcccr}
Zahl  & & wissenschaftl. & & TR: \tiprobutton{EE} \\
1400  &=& $1.4\cdot{}10^3$ &=& 1.4E3\\
0.004 &=& $4\cdot{}10^{-3}$ &=&4E-3\\
\end{tabular}

\section{Potenzen}
Für Basis $a\ne 0$ und $r\in\mathbb{Q}$ gilt:

\begin{tabular}{cc}
$a^0=1,$ & $a^{-n} = \frac1{a^n},$  \\
$\left(\frac{a}b\right)^{-n} = \left(\frac{b}a\right)^n,$ & $a^{\frac{m}n} = \sqrt[n]{a^m}$\\
 \end{tabular} 

\subsection{gleiche Basis}
\begin{tabular}{cc}
$a^m\cdot{}a^n = a^{m+n},$ & $a^m:a^n=a^{m-n},$ \\
$\left(a^m\right)^n = a^{m\cdot{}n}$ &\\
 \end{tabular} 

\subsection{gleicher Exponent}

\begin{tabular}{cc}
$a^n\cdot{}b^n = (ab)^n,$ & $\left(\frac{a}b\right)^n = \frac{a^n}{b^n}$\\
 \end{tabular}
 

\subsection{Vorzeichen}
Fall: Sei $n$ gerade:

\begin{tabular}{cc}
 $-(a^n) = -a^n,$ & aber: $(-a)^n = +a^n$\\
 \end{tabular} 

Fall: Sei $n$ \textbf{un}gerade:

\begin{tabular}{cc}
 $-(a^n) = -a^n,$ & aber: $(-a)^n = -a^n$\\
 \end{tabular} 


\section{Logarithmen}

\begin{definition}{$\log$}{}
Für $a>0, a\ne 1$ sei

$$\log_a{}(z)=x \Longleftrightarrow{} a^x = z$$
\end{definition}
$a$ = Basis, $z$ = Numerus (stets > 0), $x$ = Logarithmus (= Exponent in der Potenzschreibweise)

Logarithmen sind Exponente n zu einer fest gewählten Basis.

\begin{gesetz}{}{}
$$\log_a(u^n) = n\cdot{}\log_a(u)$$
\end{gesetz}









\end{multicols}

\end{document}
