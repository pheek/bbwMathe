%%
%% Meta: TI nSpire Einführung
%%       Ziel: Damit die Grundoperationen damit durchgeführt werden können.
%%             Damit man sich an den Rechner gewöhnt.
%%

\input{bbwSeite}

%%%%%%%%%%%%%%%%%%%%%%%%%%%%%%%%%%%%%%%%%%%%%%%%%%%%%%%%%%%%%%%%%%

\usepackage{amssymb} %% für \blacktriangleright
\renewcommand{\metaHeaderLine}{Formelsammlung GESO}
\renewcommand{\arbeitsblattTitel}{Abschlussprüfung V. 0.0.0 2021-06-08 fp}

\titlespacing*{\section}{0pt}{0.1\baselineskip}{0.1\baselineskip}
\titlespacing*{\subsection}{0pt}{0.1\baselineskip}{0.1\baselineskip}

\begin{document}%%
\arbeitsblattHeader{}
 \begin{multicols}{2}

\section{Zahlmengen}
%% Benutze
%%\columnbreak
%% für harten Break

$\mathbb{N} = \{1,2, ...\}  $ = Natürliche Zahlen\\
$\mathbb{Z} = \{..., -2, -1, 0, 1,2, ...\}  $ = Ganze Zahlen\\
$\mathbb{Q} = \{\frac{a}b | a\in \mathbb{Z}, b\in\mathbb{N}\}$ Menge der Brüche (\textbf{rationale} Zahlen)\\
$\mathbb{R} = \{..., -2, ..., -1, ...., 0,
\frac12, 1,...,\sqrt{2},...,2, e,\\
..., 3, ..., \pi, ...\}  $ = \textbf{reelle} Zahlen = alle Zahlen auf der Zahlengerade\\

Eine natürliche Zahl heißt Primzahl, wenn sie genau zwei Teiler hat (Eins und sich selbst). $\mathbb{P} = \{2, 3, 5, 7, 11, 13, ...\}$

\hrulefill

\section{Algebra}
\subsection{Binomische Formeln}

$$(a+b)^2 = a^2+2ab + b^2$$
$$(a-b)^2=a^2-2ab+b^2$$
$$(a+b)\cdot(a-b) = a^2 - b^2$$

\hrulefill

\subsection{Vertauschte Differenz}
(-1 ausklammern)

$(b-a)=(-1)\cdot{}(h-b)$

Beispiel:

$$\frac{x}{a-b} +  \frac{y}{b-a} = \frac{x}{a-b} + \frac{-y}{a-b} = \frac{x-y}{a-b}$$

\hrulefill

\subsection{Summenzeichen}
$$\sum_{k=1}^n{T(k)} = T(1) + T(2) + ... + T(n)$$
Beispiel

$$\sum_{i=3}^7{i^2} = 3^2 + 4^2 + 5^2 + 6^2 + 7^2$$

Taschenrechner: \tiprobutton{math} 4


\hrulefill

\subsection{(Absolut)betrag einer Zahl}

$|a| = a$ falls $a \ge 0$

$|a| = (-1)\cdot{}a$ falls $a < 0$


\section{Dezimalzahlen}
\subsection{Runden}
Dezimalen sind die Stellen nach dem Komma.
Runden auf $n$ Dezimalen: betrachte die $(n+1)$-te Dezimale und runde auf, wenn diese $\ge 5$.

Beispiel

3.4729\textbf{6}44 auf vier Dezimalen, so ist die 5. Dezimale (hier die Ziffer 6) zu beachten:

$$3.4729\textbf{6}44 \approx 3.4730$$
(Alle vier Dezimalen sind anzugeben.)

\hrulefill

\subsection{Signifikante Stellen}
Die Zählung der signifikanten Stellen einer Zahl beginnt (von links) mit der ersten von Null verschiedenen Ziffer. Ab hier zählen alle Ziffern, auch die 0, bis zur verlangten Rundungsstelle.

\subsection{Wissenschaftliche Schreibweise}
Für große, aber auch für Zahlen sehr nahe an 0, wird die Wissenschaftliche Schreibweise vorgezogen. Dabei steht vor dem Dezimalpunkt immer genau eine Ziffer:

\begin{tabular}{lcccr}
Zahl  & & wissenschaftl. & & TR: \tiprobutton{EE} \\
1400  &=& $1.4\cdot{}10^3$ &=& 1.4E3\\
0.004 &=& $4\cdot{}10^{-3}$ &=&4E-3\\
\end{tabular}

\section{Potenzen}
Für Basis $a\ne 0$ und $r\in\mathbb{Q}$ gilt:

\begin{tabular}{cc}
$a^0=1,$ & $a^{-n} = \frac1{a^n},$  \\
$\left(\frac{a}b\right)^{-n} = \left(\frac{b}a\right)^n,$ & $a^{\frac{m}n} = \sqrt[n]{a^m}$\\
 \end{tabular} 

\subsection{gleiche Basis}
\begin{tabular}{cc}
$a^m\cdot{}a^n = a^{m+n},$ & $a^m:a^n=a^{m-n},$ \\
$\left(a^m\right)^n = a^{m\cdot{}n}$ &\\
 \end{tabular} 

\subsection{gleicher Exponent}

\begin{tabular}{cc}
$a^n\cdot{}b^n = (ab)^n,$ & $\left(\frac{a}b\right)^n = \frac{a^n}{b^n}$\\
 \end{tabular}
 

\subsection{Vorzeichen}
Fall: Sei $n$ gerade:

\begin{tabular}{cc}
 $-(a^n) = -a^n,$ & aber: $(-a)^n = +a^n$\\
 \end{tabular} 

Fall: Sei $n$ \textbf{un}gerade:

\begin{tabular}{cc}
 $-(a^n) = -a^n,$ & aber: $(-a)^n = -a^n$\\
 \end{tabular} 


\section{Logarithmen}

\begin{definition}{$\log$}{}
Für $a>0, a\ne 1$ sei

$$\log_a{}(z)=x \Longleftrightarrow{} a^x = z$$
\end{definition}
$a$ = Basis, $z$ = Numerus (stets > 0), $x$ = Logarithmus (= Exponent in der Potenzschreibweise)

Logarithmen sind Exponente n zu einer fest gewählten Basis.

\begin{gesetz}{}{}
$$\log_a(u^n) = n\cdot{}\log_a(u)$$
\end{gesetz}

\subsection{Spezielle Werte}
\begin{tabular}{ccc}
$\log_a(1)=0$ & $\lg(10) = 1$ & $\ln(e) = 1$
\end{tabular} 

\section{Textaufgaben}

\subsection{Geschwindigkeitsaufgaben}
\fbox{$\frac{s}{v \cdot{}t}$} (Einheiten kompatibel wählen!)

\subsection{Leistungsaufgaben}
Bei $x$ Stunden für die ganze Arbeit:

Bruchteil Arbeit pro h = Leistung

Leistung = $\frac{1 \textrm{ [Arbeit]}}{x \textrm{ [Stunde]}}$

\subsection{Restaufgaben}
$$A:B = C \textrm{ Rest } R \Longleftrightarrow{}  C\cdot{}B+R = A$$

\subsection{Zinseszins}
\begin{itemize}
\item $K_0$ = Startkapital; $K_n$ = Endkapital
\item $p$ = Zinssatz (in \%); $f = \frac{p}{100}$ = Zinsfaktor
(Wachstumsfaktor); $p<0$, so ist $f$ der Zerfallsfaktor
\end{itemize}
$$K_n = K_0 \cdot{} \left( 1+\frac{p}{100} \right)^n = K_0\cdot{}f^n$$

\subsubsection{Beispiele}
$$p = 2.5\% \Longleftrightarrow{} f = 1.025$$
$$p = -3\% \Longleftrightarrow{} f = 0.97$$



\section{Gleichungen}

\subsection{Bruchgleichungen}
Immer Definitionsbereich $\mathbb{D}$ überlegen: Nenner (unten) darf
nicht 0 werden.

Bsp.: $$\frac5{x+2}=\frac{7+x^2}{x-3} \Leftrightarrow{} \mathbb{D}_x=\mathbb{R}\backslash{}\{-3, 2\}$$

Schlusskontrolle!



\subsection{Lineare Gleichugen}
$$5x-3a = 2cx-7b$$
1. Alle $x$ nach links, übriges rechts:
$$5x - 2cx = 3a-7b$$
2. $x$ ausklammern
$$x\cdot{}(5-2c) = 3a-7b$$
3. Durch Klammer teilen
$$x = \frac{3a-7b}{5-2c}$$


\subsection{num-solv}
Der Taschenrechner löst Gleichungen mit Zahlen nach $x$ auf:

$5^x = 100 -x$, dann \tiprobutton{2nd}\tiprobutton{sin_num-solv} $x\approx{}2.843$


\end{multicols}


\subsection{quadratische Gleichungen}
\textbf{Grundform}: $ax^2 + bx+c = 0$

\textbf{Lösungsformel} $x_{1,2}\frac{-b \pm \sqrt{b^2-4ac}}{2a}$

\textbf{Diskriminante} $D = b^2-4ac$:\\
$D>0$: Zwei Lösungen
$D=0$: Eine Lösung
$D<0$: keine Lösung

Wie vorgehen?

\begin{tabular}{|p{44mm}|p{53mm}|p{64mm}|}
	\hline
	Spezialfall $b=0$               & $5x^2 = 3$                   & Durch 5 dividieren, dann positive und negative Wurzel ziehen.\\
	\hline
	Spezialfall $c=0$               & $3x^2 = 5x$                   & Fallunterscheidung $x$ = 0 und $x \ne 0$\\
	\hline
	Bereits faktorisiert       & $(x-4.6 + b)\cdot{}(x+a) = 0$ & Lösungen hier $x_1=4.6-b$ und $x_2 = -a$\\
	\hline
	Einfache Zahlen            & $x^2 -2x + 1= 0$           & $a$-$b$-$c$--Formel (Mitternachtsformel) oder faktorisieren $x^2-2x+1=(x-1)^2$ und danach jeden einzelnen Faktor $=0$ setzen.\\
	\hline
	Komplizierte Zahlen        & $7.3x^2 - 8x - 3.4 = 0$       & Taschenrechner \tiprobutton{2nd}\tiprobutton{cos_poly-solv}             \\
	\hline
	Variable (Parameter)       & $7.3x^2 - cx + 2.6=0$         & $a$-$b$-$c$-Formel (Mitternachtsformel) \\
	\hline
	Komplexe Terme im Quadrat  & $3(x-6)^2 - 17(x-6)  = 0$     & Substitution                            \\
	\hline
\end{tabular}


\begin{multicols}{2}


\section{Gleichungen mit Exponenten}
\subsection{Potenzgleichungen}

$$x^a=c \leftrightarrow x=\sqrt[a]{c}$$

Allgemein

$$x^{\frac{a}b} = c \Longrightarrow{}  x=c^{\frac{b}a}$$

\subsection{Exponentialgleichungen}
\textbf{Fall 1} Exponentenvergleich. Bsp.:

$$5^4=125^x \Rightarrow{} 5^4=(5^3)^x=5^{3x} \Rightarrow{}$$
$$ 4=3x \Rightarrow x=\frac43$$

\textbf{Fall 2} Logarithmen. Bsp.:

$$7^x=105 \Rightarrow x=\log_7(105)$$
oder so:
$$7^x=105 \Rightarrow \log(7^x)=\log(105) \Rightarrow$$
$$x\cdot{}\log(7)=\log(105) \Rightarrow x=\frac{\log(105)}{\log(7)} = \frac{\lg(105)}{\lg(7)}$$

\begin{gesetz}{Exponentialgleichung}{}
$$a^x=b \Rightarrow{} x=\log_a(b) = \frac{\lg(b)}{\lg(a)}$$
\end{gesetz}

\subsection{Lineare Gleichungssysteme}
Erst in \textbf{Grundform} bringen und dann
\gleichungZZ{5x+7y}{19}{16x -13y}{-4}
mit Taschenrechner \tiprobutton{2nd}\tiprobutton{tan_sys-solv} lösen.

Verfahren
\begin{itemize}
\item Einsetzungsverfahren
\item Additionsverfahren
\item Gleichsetzungsverfahren
\item Graphische Lösung mit zwei Geradengleichungen
\item Substitution
\end{itemize}


\section{Funktionen}

Eine reelle Funktion ordnet jeder Zahl $x$ aus der Definitionsmenge
$\mathbb{D}$ genau eine Zahl $y$ aus einer Wertemenge
$\mathbb{W} \subset \mathbb{R}$ zu:


\section{Lineare Funktionen}
Explizite Form:
$$g: y = f(x) = a\cdot{}x + b$$

\bbwCenterGraphic{8cm}{img/lineareFunktion.png}
$a$ = Steigung = $\frac{V}{H}=\frac{\Delta y}{\Delta x} = \frac{y_2-y_1}{x_2-x_1}$
Zwei Geraden sind \textbf{parallel}, wenn ihre Steigungen gleich sind.

$b$ = $y$-Achsenabschnitt (=Ordinatenabschnitt)


\subsection{Nullstelle} einer Funktion: $y=0$ setzen und nach $x$
auf\/lösen.

Bsp.: $y=7x+4$ Nullstelle: $0 = 7x+4$, d.\,h. $x_0=\frac{-4}{7}$

\subsection{Ordinatenabschnitt}
\textbf{Schnittpunkt mit der $y$-Achse}: $x=0$ setzen.

Bsp.: $y=7x+4$ Achsenabschnitt = Schnittpunkt mit der $y$-Achse heißt
$x=0$. Also $y=7\cdot{}0 + 4 = 4$. Somit liegt der Schnittpuntk bei $(0|4)$.


\subsection{Schnittpunkt zweier Geraden}
Gegeben: $g: y=ax+b$ und $h: y=cx+d$.

Gleichsetzen:

\gleichungZZ{y}{ax+b}{y}{cx+d}
Finde $x_S$ des Schnittpunktes $S=(x_S|y_S)$
$$ax+b = cx+d$$

\subsection{Mittelpunkt $M$ einer Strerke $AB$}
Gegeben $A=(x_A|y_A)$ und $B=(x_B|y_B)$

Gesucht Mittelpunkt $M=(x_M|y_M)$

Lösung Mittelwert $x$ und $y$ separat:

$$x_M = \frac{x_A+x_B}2; y_M=\frac{y_A+y_B}2$$

\subsection{Horizontale Gerade}

$g: y=ax+b$ ist horizontal, wenn die Steigung $a$ wegfällt ($a=0$).

$$g:  y=0\cdot{}x+b \Rightarrow y=b$$

\subsection{Zweipunkte Aufgabe}
Gesucht Gerade $g: y=ax+b$ durch zwei gegebene Punkte $A=(x_A|y_A)$
und $B=(x_B|y_B)$.

\textbf{Variante 1}: Steigung $a = \frac{V}H
= \frac{y_B-y_A}{x_B-x_A}$ berechnen und einen der beiden Punkte in
$y=ax+b$ einsetzen, um $b$ zu finden.

\textbf{Variante 2}: Gleichungssystem nach $a$ und $b$ auf\/lösen:
\gleichungZZ{y_A}{a\cdot{}x_A+b}{y_B}{a\cdot{}x_B + b}


\section{Potenzfunktionen}
Grundform:
$$y=ax^z$$
$z\in\mathbb{Z}\backslash\{0,1\} = \{...-2, -1, 2, 3, 4, ...\}$

$z>0$: Parabeln

$z<0$: Hyperbeln

$z$ gerade: Gespiegelt an der $y$-Achse

$z$ ungerade: Gespiegelt am Ursprung $O=(0|0)$

$a>0$: gerade Parabel nach oben geöffnet bzw. Hyperbel oder ungerade
Parabel im Quadranten I und III

$a<0$: gerade Parabel nach unten geöffnet bzw. Hyperbel oder ungerade
Parabel im Quadranten II und IV

\bbwCenterGraphic{8cm}{img/potenzFunktionen.png}

\subsection{Quadratische Funktion}
$$y=ax^2$$
$a$ heißt Öffnung ($a<0$: Die Parabel ist nach unten geöffnet)
$a=1$ oder $a=-1$: Normalparabel


\section{Exponentialfunktionen}

Grundform:
\begin{gesetz}{Exponentialfunktion}{}
$$f(t) = b\cdot{}a^{\frac{t}{\tau}}$$
$b$ = Startwert bei $t_0=0$\\
\,\,\,\, = $y$-Achsenabschnitt\\
$a$ = Vervielfältigungsfaktor ($a\in\mathbb{R}^{+}$)\\\
$a>0$: Wachstumsrate\\
$a<0$: Zerfallsrate\\
$\tau$ = Beobachtungsintervall zu $a$
\end{gesetz}
Beispiel: Die Algen mit Startwert von $3\textrm{m}^2$ vervierfachen
sich alle fünf Tage: $b=3$, $a=4$, $\tau=5$
$$f(t)= 3\cdot{}4^\frac{t}{5}$$

\subsection{Halbwertszeit}
$\frac12 b = f(t) = b\cdot{}a^{\frac{t}{\tau}}$
$$t_{\frac12} = \tau\cdot{}\log_a(\frac12)$$

\subsection{Sättigungsfunktionen}

\section{Datenanalyse}

\subsection{Skalen}

\subsection{Lagemaße}

\subsection{Streumaße}

\subsection{Diagramme}

\subsubsection{Histogramm}

\subsubsection{Streudiagramm}

\subsubsection{Boxplot}


\section{Stochastik}

\subsection{Kombinatorik}
\end{multicols}

\subsection*{Überblick}
\bbwCenterGraphic{15cm}{img/Kombinatorik.png}


\begin{multicols}{2}


\subsection{Wahrscheinlichkeit}


\end{multicols}

\end{document}
