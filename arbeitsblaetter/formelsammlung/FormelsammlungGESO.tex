%%
%% Meta: TI nSpire Einführung
%%       Ziel: Damit die Grundoperationen damit durchgeführt werden können.
%%             Damit man sich an den Rechner gewöhnt.
%%

\input{bbwSeite}

%%%%%%%%%%%%%%%%%%%%%%%%%%%%%%%%%%%%%%%%%%%%%%%%%%%%%%%%%%%%%%%%%%

\usepackage{amssymb} %% für \blacktriangleright
\renewcommand{\metaHeaderLine}{Formelsammlung GESO}
\renewcommand{\arbeitsblattTitel}{Abschlussprüfung V. 0.0.0 2021-06-08 fp}

\begin{document}%%
\arbeitsblattHeader{}
 \begin{multicols}{2}

\section{Zahlmengen}
%% Benutze
%%\columnbreak
%% für harten Break

$\mathbb{N} = \{1,2, ...\}  $ = Natürliche Zahlen\\
$\mathbb{Z} = \{..., -2, -1, 0, 1,2, ...\}  $ = Natürliche Zahlen\\
$\mathbb{Q} = \{\frac{a}b | a\in \mathbb{Z}, b\in\mathbb{N}\}$ Menge der Brüche (\textbf{rationale} Zahlen)\\
$\mathbb{R} = \{..., -2, ..., -1, ...., 0,
\frac12, 1,...,\sqrt{2},...,2, e,\\
..., 3, ..., \pi, ...\}  $ = \textbf{reelle} Zahlen = alle Zahlen auf der Zahlengerade\\

Eine natürliche Zahl heißt Primzahl, wenn sie genau zwei Teiler hat (Eins und sich selbst). $\mathbb{P} = \{2, 3, 5, 7, 11, 13, ...\}$

\hrulefill

\section{Algebra}
\subsection{Binomische Formeln}

$$(a+b)^2 = a^2+2ab + b^2$$
$$(a-b)^2=a^2-2ab+b^2$$
$$(a+b)\cdot(a-b) = a^2 - b^2$$

\hrulefill

\subsection{Vertauschte Differenz}
(-1 ausklammern)

$(b-a)=(-1)\cdot{}(h-b)$

Beispiel:

$$\frac{x}{a-b} +  \frac{y}{b-a} = \frac{x}{a-b} + \frac{-y}{a-b} = \frac{x-y}{a-b}$$

\hrulefill

\subsection{Summenzeichen}
$$\sum_{k=1}^n{T(k)} = T(1) + T(2) + ... + T(n)$$
Beispiel

$$\sum_{i=3}^7{i^2} = 3^2 + 4^2 + 5^2 + 6^2 + 7^2$$

Taschenrechner: \tiprobutton{math} 4


\hrulefill

\subsection{(Absolut)betrag einer Zahl}

$|a| = a$ falls $a \ge 0$

$|a| = (-1)\cdot{}a$ falls $a < 0$


\section{Dezimalzahlen}
\subsection{Runden}
Dezimalen sind die Stellen nach dem Komma.
Runden auf $n$ Dezimalen: betrachte die $(n+1)$-te Dezimale und runde auf, wenn diese $\ge 5$.

Beispiel

3.4729\textbf{6}44 auf vier Dezimalen, so ist die 5. Dezimale (hier die Ziffer 6) zu beachten:

$$3.4729\textbf{6}44 \approx 3.4730$$
(Alle vier Dezimalen sind anzugeben.)

\hrulefill

\subsection{Signifikante Stellen}
...
\end{multicols}

\end{document}
