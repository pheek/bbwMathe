%%
%% Meta: Master Document
%% Wichtige Befehle für GESO mit TI 30 Pro MathPrint
%%

\input{bbwLayoutPage}
\renewcommand{\metaHeaderLine}{Arbeitsblatt}
\renewcommand{\arbeitsblattTitel}{Taschenrechner TI-30 Pro MathPrint
-- Befehle}


\begin{document}%%
\arbeitsblattHeader{}

\begin{center}
\includegraphics[width=5cm]{img/tiprobuttonimages/ti30.png}
\end{center}
\newpage


%%%%%%%%%%%%%%%%%%%%%%%%%%%%%%%%%%%%%%%%%%%%%%%%%%%%%%%%%%%%%%%%%%%%%%%%%%%%
\section{Benutzen}
\subsection{Einschalten}
Schalten Sie den Rechner ein: \tiprobutton{on} unten links. Sollte schon
was da stehen, so verwenden Sie \tiprobutton{clear}\label{clear} oder \tiprobutton{2nd}\tiprobutton{mode_quit}, um in den
Initialzustand zu gelangen.

\subsubsection{2nd}\label{mehrfachbelegung}
Die \tiprobutton{2nd}-Taste aktiviert die klein geschriebene blaue Funktion über der Taste.

\subsubsection{quit}\label{modus_verlassen}
Verlassen der aktuellen Funktionalität geschieht mit \tiprobutton{2nd}\tiprobutton{mode_quit}

\subsubsection{\texttt{delete}
und \texttt{clear}}\label{delete_and_clear}
Die Taste \tiprobutton{delete} löscht die letzte eingegebene Ziffer. Die
Taste \tiprobutton{clear}\label{clear} löscht die ganze Rechnung.


\subsubsection{Reset}
Um den Rechner zurückzusetzen entweder \tiprobutton{on_off}\tiprobutton{clear} gleichzeitig drücken oder aber \tiprobutton{2nd}\tiprobutton{0_reset}.

\subsection{Ausschalten}
 \fbox{{\color{cyan}\textbf{\textit{off}}}} = \tiprobutton{2nd}\tiprobutton{on_off}
\newpage


%%%%%%%%%%%%%%%%%%%%%%%%%%%%%%%%%%%%%%%%%%%%%%%%%%%%%%%%%%%%%%%%%
\section{Grundoperationen}
\subsection{Genau vs. «verständlich»}\label{approx}

Tippen Sie nun $\frac{9}{8}$:

\tiprobutton{9}\tiprobutton{div}\tiprobutton{8}\tiprobutton{enter}



und schalten Sie mit der
Taste \tiprobutton{approx} zwischen $1.125$ und $\frac{9}{8}$ um.


Probieren Sie dies auch mit $\sqrt{8}$:

\tiprobutton{2nd}\tiprobutton{sqr_sqrt}\tiprobutton{8}\tiprobutton{enter}:

$2\sqrt{2} \approx 2.82824$. 

dann: \tiprobutton{approx}


\subsection{Negative Zahlen}
Zur Eingabe von negativen Zahlen dient die Taste \tiprobutton{neg}.

Leider beim TI-30X Pro muss das
Zeichen \textbf{vor} der Zahl eingegeben werden; im Gegensatz zu den
meisten gängigen Taschenrechnern.


%%%%%%%%%%%%%%%%%%%%%%%%%%%%%%%%%%%%%%%%%%%%%%%%%%%%%%%%%%%%%%%%%%%%%%%%%
\subsection{Wissenschaftliche Notation}
Der TI-30x Pro MathPrint ist in der Lage, Zahlen auch in der
Wissenschaftlichen Notation anzuzeigen. Dazu drücken
Sie \tiprobutton{mode}.

Wählen Sie \textbf{\texttt{SCI}} aus der zweiten Zeile
neben \texttt{NORMAL} und bestätigen Sie mit \tiprobutton{enter}:

\texttt{NORMAL\,\,\,\textbf{SCI}\,\,\,ENG}

\newpage


%%%%%%%%%%%%%%%%%%%%%%%%%%%%%%%%%%%%%%%%%%%%%%%%%%%%%%%%%%%%%%%%%%%%%%%%%
\section{Speichern}

\subsection{answer}\label{AnsSto}

Um mit dem Resultat der vorangehenden Rechnung weiterzufahren,
existiert die Tastenkombination \tiprobutton{2nd}\tiprobutton{neg_answer}. \texttt{ans} steht für den Wert des
Resultates der zuletzt ausgeführten Operation. Wir berechnen $16.3 +
19.8$ und danach das Resultat $\cdot4$ (Stimmts? 144.4). Starten Sie
eine Rechnung mit einem Operationszeichen (\zB \tiprobutton{mult}\tiprobutton{3}), so
wird \texttt{ans} vorangestellt und gleich mit der Antwort weitergerechnet.

Noch ältere Resultate können mit der Pfeiltaste nach oben \tiprobutton{pfeiltasten} zurückgeholt werden.

\subsection{STO}\label{sto}

Daneben gibt es die Möglichkeit, Zwischenresultate in Variablen (nummerierte
Speicherplätze) mit \tiprobutton{sto} (gefolgt von einem
Variablennamen \tiprobutton{xyzabcd}) zu
speichern. Die Variablennamen $y$, $z$, $t$, $a$, $b$, $c$ und $d$
werden mit mehrmaligem Drücken dieser Taste
erreicht. Mit \tiprobutton{2nd}\tiprobutton{sto_recall} kann die Variable wieder 
ausgelesen und weiterverwendet werden.

Beispiel: Erst Zahl abspeichern mit \tiprobutton{sto} \tiprobutton{xyzabcd}. Es
erscheint

\texttt{ans$\rightarrow{}$x}

Zum Beispiel kann nun
mit  \tiprobutton{xyzabcd} \tiprobutton{plus} \tiprobutton{5} \tiprobutton{enter}
dem Resultat gleich fünf dazugezählt werden.

\subsection{Gespeicherte Konstanten}\label{constants}
Der TI-30 Pro MathPrint hat zusätzlich die wichtigsten physikalischen
Konstanten bereits integriert. Die Mol-Zahl (Avogadro-Konstante) \zB
finden wir mittels \tiprobutton{2nd} \tiprobutton{constants}. Wollen wir \zB die
Zahl 12 durch die Mol-Zahl teilen, so tippen wir

\tiprobutton{1}\tiprobutton{2}\tiprobutton{div}\tiprobutton{2nd}\tiprobutton{constants}\tiprobutton{4}\tiprobutton{enter}.
\newpage


%%%%%%%%%%%%%%%%%%%%%%%%%%%%%%%%%%%%%%%%%%%%%%%%%%%%%%%%%%%%%%%%%%%%%%%%
\section{Quadratische Gleichungen}\label{quadGL}
Quadratische Gleichungen mit Zahlen können einfach mit \tiprobutton{2nd}\tiprobutton{cos_poly-solv} gelöst werden. Dazu müssen die Gleichungen vorab in die Grundform gebracht werden:

$$7x^2 - 55x - 22 = 0$$

und danach $a$, $b$ und $c$ für die $abc$-Formel abgelesen werden:

$$a=7; b=-55; c=-22$$


%%%%%%%%%%%%%%%%%%%%%%%%%%%%%%%%%%%%%%%%%%%%%%%%%%%%%%%%%%%%%%%%%%%%%%%%
\section{Lineare Gleichungssysteme}\label{gleichungssysteme}
Lineare Gleichungssysteme können mit der \texttt{sys-solv}-Funktion gelöst werden.
Dazu muss das Gleichungssystem zunächst in die Grundform gebracht werden.

\gleichungZZ{5x-3y}{4}{7x+2y}{-3}

Nach der Eingabe von \tiprobutton{2nd}\tiprobutton{tan_sys-solv} wählen Sie \texttt{1} für ein Gleichungssystem mit zwei Gleichungen und zwei Unbekannten:

\texttt{1: 2x2 linear EQs}

Zahlen mit \tiprobutton{enter} bestätigen. Die Subtraktion links des
Gleichheitszeichens mit \tiprobutton{minus} eingeben; negative Werte rechts
des Gleichheitszeichens mit \tiprobutton{neg} eingeben.

Resultat für obiges Beispiel zur Kontrolle: $x=\frac{-1}{31}$ und
$y=\frac{-43}{31}$.

\newpage


%%%%%%%%%%%%%%%%%%%%%%%%%%%%%%%%%%%%%%%%%%%%%%%%%%%%%%%%%%%%%%%%%%%%%%%%%%%5
\section{Potenzen, Wurzeln und
Logarithmen}\label{ln_log}\label{xhoch_nwurz}

Quadratzahlen werden mit der \tiprobutton{sqr} berechnet.

Quadratwurzeln liegen auf der selben Taste
mit \tiprobutton{2nd} \tiprobutton{sqr_sqrt}.


Zehnerpotenzen ($10^x$) werden mit der Taste \tiprobutton{ex_10x} eingegeben. Dies gilt auch für Exponenten zur Basis $e$.

Große Zehnerpotenzen\label{ee} für die Wissenschaftliche Notation werden mit der \tiprobutton{EE}-Taste eingegeben: Beispiel: $3.84 \cdot{} 10^9$:\\
\tiprobutton{3}\tiprobutton{dot}\tiprobutton{8}\tiprobutton{4}\tiprobutton{EE}\tiprobutton{9}

$x^4$, $x^7$ ... werden mit der Taste \tiprobutton{xhoch} eingegeben.

Analog die $n$-te Wurzel mit \tiprobutton{2nd}\tiprobutton{xhoch_nwurz}.

Den Zehnerlogarithmus geben wir mit zweimaligem Drücken
von \tiprobutton{ln_log}\tiprobutton{ln_log}  ein. Analog Logarithmen zu anderen
Basen; dazu muss die \tiprobutton{ln_log}-Taste jedoch drei Mal gedrückt
werden.
Wird die Taste \tiprobutton{ln_log} nur einmal gedrückt, handelt es sich um
den natürlichen Logarithmus zur Basis $e$ ($e = 2.71828...$).



%%%%%%%%%%%%%%%%%%%%%%%%%%%%%%%%%%%%%%%%%%%%%%%%%%%%%%%%%%%%%%%%%%55
\section{Faktorisieren}\index{faktorisieren}\label{faktorisieren}


Will man einfach nur Zahlen Faktorisieren, \zB um den \textbf{Zweiklammeransatz} zu verwenden, geschieht das mit der Funktion \texttt{Pfactor}, welche sich hinter der \tiprobutton{math}-Taste verbirgt.

Faktorisieren von Termen der Form $x^2 +4x -32$ ist zudem mit
quadratischer Gleichung möglich, indem wir den Term als quadratische
Gleichung auffassen indem wir den Term gleich 0 setzen. Hier müssen wir jedoch aufpassen, denn der Term wird als quadratische Gleichung aufgefasst und die Lösung der quadratischen Gleichung sind die \textbf{negativen} Parameter des faktorisierten Terms! Bitte nachrechnen!
\newpage

%%%%%%%%%%%%%%%%%%%%%%%%%%%%%%%%%%%%%%%%%%%%%%%%%%%%%%%%%%%%%%%%%%%%5
\section{Funktionen / Wertetabelle}\label{wertetabelle}
Wertetabellen können mit dem Befehl \tiprobutton{table} einfach erstellt
werden.

Beispiel Wertetabelle zu $f: y=-0.5x + 2.5$:

Drücken Sie \tiprobutton{table}:

\texttt{1: Add/Edit Func.}

Mit \tiprobutton{1} können Sie die Funktion definieren:

$$f(x) = -0.5 * x + 2.5$$

Die Funktion $g()$ können Sie leer lassen (mit \tiprobutton{enter}
bestätigen).

Das System erfragt danach die Werte

\texttt{start}: Welches ist der erste $x$-Wert?

und

\texttt{step}: Schrittweite zwischen den $x$-Werten. Mit Bestätigung
(\tiprobutton{enter}) erscheint die Wertetabelle

\begin{tabular}{r|l}
$x$ & $y=f(x)$\\
\hline\\
-4 & 4.5\\
-2 & 3.5\\
... &  ... \\
\end{tabular}
\newpage
\section{Geradengleichung durch zwei gegebene Punkte}
Sind zwei Punkte (\zB $P=(3.6 | 4.9)$ und $Q=(-5.2 | 7.1)$) gegeben,
so können $a$ und $b$ der Geradengleichung $y=a\cdot{}x+b$ mittels der
linearen Regression gefunden werden:

\begin{enumerate}
\item Die beiden Punkte in die Datenreihen \textbf{\texttt{L1}}
und \textbf{\texttt{L2}} mittels \tiprobutton{data} eingeben. Dabei
entsprechen \textbf{\texttt{L1}} den $x$- bzw. \textbf{\texttt{L2}} den
$y$-Werten:
%%\bbwCenterGraphic{2.4cm}{taschenrechner/tr_ti30_geso_wichtige_befehle/img/ZweiPunkteEingeben.jpg}
\bbwCenterGraphic{8cm}{img/ZweiPunkteEingeben.jpg}

\item Unter \tiprobutton{data_stat-reg-distr} \tiprobutton{data}
unter \texttt{\textbf{STAT-REG}} die nummer \texttt{\textbf{4}} suchen:
«\texttt{\textbf{LinReg $ax+b$}}».

\item Beim nächsten Srceen alles so lassen, wie es
ist: \texttt{\textbf{xDATA: L1}}, \textbf{\texttt{yDATA: L2}}.
\item Mittels \texttt{\textbf{calc}} die Daten (in unserem Beispiel
$a=-0.25$ und $b=5.8$) ablesen.
\end{enumerate}


\newpage

%%%%%%%%%%%%%%%%%%%%%%%%%%%%%%%%%%%%%%%%%%%%%%%%%%%%%%%%%%%%%%%%%%%%5
\section{Gleichungen numerisch}\index{numGleichungen}\label{numGleichungen}
Oft verlieren wir viel Zeit, nach dem Lösen einer Gleichung, mit dem Prüfen, ob unser Resultat stimmt.

Der «num-solv» Befehl kann jedoch Gleichungen mit Zahlen (numerische Gleichungen) lösen, sodass wir rasch sehen, ob wir richtig gerechnet haben.

Beispiel: Es sei $a=4$ gegeben und die folgende Gleichung zu lösen: $$5x-a = 18x-\frac32x + 2$$.

Zuerst lösen wir die Gleichung von Hand nach $x$ auf indem wir $4$ für $a$ einsetzen:

$$5x-a = 18x -\frac32x + 2$$
$$5x-4 = 18x -\frac32x + 2$$
$$  -4 = 13x -\frac32x + 2$$
$$  -6 = 13x -\frac32x$$
$$  -6 = 11.5x$$
$$ x = \frac{-6}{11.5}\approx{-0.5217}$$

Nun kontrollieren wir mit dem Taschenrechner.

\begin{enumerate}
\item 4 in die Variable $a$ abspeichern: \tiprobutton{4} \tiprobutton{sto} $5\times$\tiprobutton{xyzabcd}
\item Num-Solver starten: \tiprobutton{2nd} \tiprobutton{sin_num-solv}
\item Danach Eingabe der Gleichung
$$5x-a = 18x -\frac32x + 2$$
\item Bei \texttt{EDIT VARIABLE IF NEEDED} kann die Variable $x$ ignoriert werden; sie wird ja berechnet. Jedoch kann $a$ überprüft werden.
\item Bei \texttt{SELECT SOLUTION VARIABLE} muss $x$ ausgewählt werden, wenn nicht schon gesetzt.
\item Bei \texttt{ENTER SOLUTION BOUNDS} könnte die Lösung in ein Intervall eingegrenzt werden. $-1E99$ und $1E99$ sind in der Regel ausreichend.
\item Mit \texttt{SOLVE} (\tiprobutton{enter}-Taste) wird das Resultat für $x$ angezeigt: $x = -0.521739...$.
\end{enumerate}




%%%%%%%%%%%%%%%%%%%%%%%%%%%%%%%%%%%%%%%%%%%%%%%%%%%%%%%%%%%%%%%%%%%%5
\section{Datenanalyse}\index{daan}\label{daan}
Hinter \tiprobutton{data_stat-reg-distr} verstecken sich statistische
Funktionen zur Berechnung wichtiger \textbf{Kennzahlen}:
Standardabweichung, Mittelwert und insb. alle wichtigen Zahlen für
den \textbf{Boxplot}: Q1, Median, Q3.

Gleich ein Beispiel:

Wir haben die Messwerte

5, 8, 4, 9, 2, 2 und 6

Drücken Sie \tiprobutton{data} und geben Sie die Daten
bei \texttt{L1} ein.
Mit \tiprobutton{delete} können Sie Werte auch wieder löschen.

Sind alle Zahlen eingegeben so tippen
Sie \tiprobutton{2nd}\tiprobutton{data_stat-reg-distr}. Es erscheint eine
Auswahl. Wählen Sie \tiprobutton{2}:

\texttt{2: 1-VAR STATS}

Danach wählen Sie:

\texttt{Data: \textbf{L1}}

und

\texttt{Freq: \textbf{ONE}}

Im Resultat bezeichnen:

\begin{tabular}{r|r l|p{9cm}}
 1: & $n$            = & 6     & Anzahl der Werte\\
 2: & $\overline{x}$ = & 5     & Mittelwert (arithmetisch) = Durchschnitt\\
 3: & Sx             = & 2.966 & Standardabweichung \\
 8: & Q1             = & 2     & Untere Quartilsgrenze \\
 9: & Med            = & 4.5   & Median \\
10: & Q3             = & 8     & Obere Quartilsgrenze \\


\end{tabular}
\newpage

%%%%%%%%%%%%%%%%%%%%%%%%%%%%%%%%%%%%%%%%%%%%%%%%%%%%%%%%%%%%%%%%%%%%%%
\section{Wahrscheinlichkeitsrechnung}\label{wahrscheinlichkeitsrechnung}
Anzahl Möglichkeiten bei Zurücklegen oder ohne Zurücklegen geschieht mit der \tiprobutton{ncrnpr}-Taste, die je nach Funktion mehrmals zu drücken ist. 

Dabei bezeichnen

\begin{itemize}
\item \texttt{\textbf{!}}: Fakultät: Anzahl Permutationen von $n$ Elementen auf $n$ Plätzen:\\
4 ! = 24

\item \texttt{\textbf{nCr}}: Anzahl Kombinationen: Wähle 2 aus 8 aus; die Reihenfolge ist nicht wesentlich:\\
\texttt{8 nCr 2} = 28

\item \texttt{\textbf{nPr}}: Anzahl Permutationen: Wähle 2 aus 8 aus, diesmal ist die Reihenfolge wesentlich:\\
\texttt{8 nPr 2} = 56
\end{itemize}

\subsubsection{Binomialverteilung}
Die Formel
$$P(E = k) = {n \choose k}\cdot{}p^k\cdot{}(1-p)^{n-k}$$
kann direkt mit \texttt{nCr} eingegeben werden.

Doch eine Abkürzung kann unter der \tiprobutton{data_stat-reg-distr}-Taste gefunden werden:

\tiprobutton{data_stat-reg-distr}: \texttt{DISTR} :
4:Binomialpdf\footnote{pdf = probability density function} :
\texttt{SINGLE} : dann $n$ bei $n$, $p$ bei $p$ und $k$ bei $x$ eintragen.

Auch kumuliert:

 $$P(E \le k) = \sum_{i=0}^{k}{n \choose i}\cdot{}p^i\cdot{}(1-p)^{n-i}$$

\tiprobutton{data_stat-reg-distr}: \texttt{DISTR} :
4:Binomialcdf\footnote{cdf = cumulated density function} :
\texttt{SINGLE} : dann $n$ bei $n$, $p$ bei $p$ und $k$ bei $x$ eintragen.

     


\newpage


\section*{Überblick über die Tasten}

\begin{tabular}{c|p{10cm}|l}
\textit{Taste}                           & \textit{Beschreibung}                       & \textit{Kapitel}              \\
\hline
\hline

\tiprobutton{2nd}                        & zweite Funktion auf den Tasten (klein blau) & \ref{mehrfachbelegung}        \\

\hline

\tiprobutton{2nd}\tiprobutton{mode_quit}   & Verlassen des aktuellen Modus               & \ref{modus_verlassen}         \\

\hline

\tiprobutton{delete}                     & Aktuelle Eingabe löschen                   & \ref{delete_and_clear}                              \\

\hline

\tiprobutton{ln_log}                     & Logarithmen                                & \ref{ln_log}                   \\

\hline

\tiprobutton{math}                     & Mathematische Funktionen wie kgV, ggT und Faktorisieren  & \ref{faktorisieren}                   \\

\hline

\tiprobutton{data_stat-reg-distr}      & Datenanalyse / Wahrscheinlichkeitsrechnung             & \ref{daan}                   \\

\hline

\tiprobutton{ex_10x}                   & Potenzen $e^x$ und $10^x$                       & \ref{ln_log}                   \\

\hline

\tiprobutton{EE}                      & Eingabe großer Zehnerpotenzen in wissenschaftlicher Notation     & \ref{ee}                   \\

\hline

\tiprobutton{ncrnpr}                  & Anzahlen für die Wahrscheinlichkeitsrechnung     & \ref{wahrscheinlichkeitsrechnung}                   \\

\hline

\tiprobutton{table}                  & Wertetabellen zu Funktionen                      & \ref{wertetabelle}                   \\

\hline

\tiprobutton{clear}                  & Ganze Rechnung löschen                           & \ref{clear}                   \\

\hline

\tiprobutton{2nd}\tiprobutton{sin_num-solv}  & Lösen von Gleichungen                         & \ref{numGleichungen}                 \\

\hline

\tiprobutton{2nd}\tiprobutton{cos_poly-solv}  & Quadratische Gleichungen lösen              & \ref{quadGL}                 \\

\hline

\tiprobutton{2nd}\tiprobutton{tan_sys-solv}  & Gleichungssysteme lösen                      & \ref{gleichungssysteme}                 \\

\hline

\tiprobutton{2nd}\tiprobutton{constants}                        &physikalische Konstanten                  & \ref{constants}                 \\

\hline

\tiprobutton{xhoch_nwurz}              & Potenzen und Wurzeln                          & \ref{xhoch_nwurz}                 \\

\hline

\tiprobutton{sqr_sqrt}              & Quadrate und Quadratwurzeln                        & \ref{xhoch_nwurz}                 \\

\hline

\tiprobutton{xyzabcd}              & Variablen (\zB für die \tiprobutton{sto}-Funktion)                      & \ref{sto}                 \\

\hline

\tiprobutton{sto}                        & Speichern und auslesen                      & \ref{AnsSto}                 \\


\hline

\tiprobutton{approx}                        & Umschalten zwischen exakter Form und Dezimalbruchdarstellung      & \ref{approx}                 \\

\end{tabular}


\end{document}
