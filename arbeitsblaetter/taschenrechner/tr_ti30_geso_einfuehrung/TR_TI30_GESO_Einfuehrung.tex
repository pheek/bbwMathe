%%
%% Meta: Master Document
%% Arithmetik und Algebra I
%%

\input{bmsLayoutPage}
\renewcommand{\metaHeaderLine}{Arbeitsblatt}
\renewcommand{\arbeitsblattTitel}{Taschenrechner TI-30 Pro MathPrint
--- Einführung}

\begin{document}%%
\arbeitsblattHeader{}

\section{Einschalten}
Schalten Sie den Rechner ein. \TRAINER{ON unten links. Sollte schon
was da stehen: clear oder 2nd 'quit'}


\section{Grundoperationen und Vorrangregeln}
Wir berechnen $7 + 8(5-2)$  (Stimmts? 31) und $7 + 8\cdot 5 - 2$ (Stimmts? 45).

\section{Fehlerkorrektur}
Mit \tiprobutton{delete} kann die letzte Eingabe, mit \tiprobutton{clear}
kann das ganze Display gelöscht werden.

\section{Genau vs. verständlich}
Berechnen Sie nun $9 \div 8$ (\tiprobutton{enter}) und schalten Sie mit der
Taste \tiprobutton{approx} zwischen $1.125$ und $\frac{9}{8}$ um.
Probieren Sie es mit \tiprobutton{2nd} \tiprobutton{sqr_sqrt} \tiprobutton{8} \tiprobutton{enter}:
$2\sqrt{2} \approx 2.82824$. 

\section{Negative Zahlen}
Zur Eingabe von negativen Zahlen Dient das
Symbol: \tiprobutton{neg}. Leider beim TI-30X Pro muss das
Zeichen \textbf{vor} der Zahl eingegeben werden; im Gegensatz zu den
meisten gängigen Taschenrechnern.

Berechnen Sie

$$(-3)^4$$
und

$$3* (-8)$$
\newpage


\section{ANS und STO}
ANS steht für den Wert des
Resultates der zuletzt ausgeführten Operation. Wir Berechnen $2 * 7.8$ ... (ergibt 15.6).

Nun können wir mit dem Resultat weiterrechnen: 

\tiprobutton{plus} \tiprobutton{2} \tiprobutton{enter} ergibt $17.6$.

Das Resultat kann auch in der Mitte einer Rechnung verwendet werden.

2.4 \tiprobutton{plus} \tiprobutton{2nd} \tiprobutton{neg_answer} \tiprobutton{enter}

(ergibt 20)


\subsection{sto: $x$, $y$, $z$, $t$, $a$, $b$, $c$ und $d$}
Daneben gibt es die Möglichkeit, Zwischenresultate in nummerierte
Speicherplätze mit \tiprobutton{sto} (gefolgt von einem
Variablennamen \tiprobutton{xyzabcd}) zu
speichern. Mit \tiprobutton{2nd} \tiprobutton{sto_recall} kann die Variable wieder
ausgelesen und weiterverwendet werden.


\section{Was ist das alles?}
Am Beispiel $nCr$ (number of combinations) erfahren wir, dass all
diese ``Buttons'' irgendeine Bedeutung haben, die uns das Leben
einfacher machen können. Wir werden jedoch nicht alle verwenden.
Bsp.: Möglichkeiten im Schweizer Zahlenlotto: 6
aus 42 und Glückszahl 1 aus 6:
Tippen Sie: (\tiprobutton{clear}) \tiprobutton{4}\tiprobutton{2} \tiprobutton{ncrnpr} \tiprobutton{ncrnpr}
\tiprobutton{6} \tiprobutton{enter} \tiprobutton{mult}
\tiprobutton{6}. (Kontrolle: 5\,245\,786 * 6 = 31\,474\,716)

\section{Ausschalten}
Schalten Sie den Rechner wieder aus, um Batterie zu sparen.

\end{document}

