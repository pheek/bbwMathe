\input{bbwSeite}


%%%%%%%%%%%%%%%%%%%%%%%%%%%%%%%%%%%%%%%%%%%%%%%%%%%%%%%%%%%%%%%%%%

\usepackage{amssymb} %% für \blacktriangleright
\renewcommand{\metaHeaderLine}{Potenzgleichungen}
\renewcommand{\arbeitsblattTitel}{(BMS)}

\begin{document}%%
\arbeitsblattHeader{}

\newcounter{aufgabennummer}
\setcounter{aufgabennummer}{1}

\newcommand\aufgabeML[2]{
Aufgabe \arabic{aufgabennummer}:\,\,
${#1} \Longrightarrow  \LoesungsRaum{{#2}}$

\stepcounter{aufgabennummer}
}
{\tiny Nach einem Aufgabenblatt von M. Rohner (BBW)}

\section{Mehrere Lösungen}
Bestimmen Sie die Lösungsmenge für die Variable $x$:

\aufgabeML{x^4=81}{x = \pm\sqrt[4]{81} \Longrightarrow \mathbb{L}_x = \{-3; 3\}}

\aufgabeML{4x^6=256}{|: 4:  x^6 = 64 \Longrightarrow x = \pm\sqrt[6]{64} \Longrightarrow \mathbb{L}_x=\{-2; 2\}}

\aufgabeML{-x^4=16}{|: \cdot{}(-1) : x^4 =
-16 \Longrightarrow \mathbb{L}_x=\{\} \textrm{ keine Wurzeln aus
 negativen Zahlen}}

\aufgabeML{x^3=64}{x = +\sqrt[3]{64} \Longrightarrow \mathbb{L}_x=\{+4\}}

\aufgabeML{-x^3=27}{|: \cdot{}(-1) \Longrightarrow x^3 =
 -27 \Longrightarrow x = -\sqrt[3]{27} \Longrightarrow \mathbb{L}_x=\{-3\}}

\aufgabeML{\frac56 x^3 = \frac{50}{60}}{\textrm{ kürzen } \frac56 x^3
= \frac56 \textrm{ durch } \frac56 \textrm{ teilen: } x^3 =
1 \Longrightarrow \mathbb{L}_x = \{1\}}


\platzFuerBerechnungen{14}
\noTRAINER{\newpage}

\subsection{...}

\aufgabeML{...}{...}

\platzFuerBerechnungen{16}



\end{document}
