%%
%% Meta: TI nSpire Einführung
%%       Ziel: Damit die Grundoperationen damit durchgeführt werden können.
%%             Damit man sich an den Rechner gewöhnt.
%%

\input{bbwSeite}

%%%%%%%%%%%%%%%%%%%%%%%%%%%%%%%%%%%%%%%%%%%%%%%%%%%%%%%%%%%%%%%%%%

\usepackage{amssymb} %% für \blacktriangleright
\renewcommand{\metaHeaderLine}{Arbeitsblatt}
\renewcommand{\arbeitsblattTitel}{Taschenrechner TI-$n$spire CX II-T CAS --- Einführung}

\begin{document}%%
\arbeitsblattHeader{}

\section{Einschalten}
Schalten Sie den Rechner ein. \TRAINER{ON unten links. Sollte schon
was da stehen: clear}

Wir schalten gleich um in den ``notes''-Modus.
Entweder mit \fbox{doc}, danach \fbox{1} \fbox{6}; oder mit Klicken auf
das gelb/orange Notizblatt-Symbol:
\raisebox{-3mm}{\includegraphics[width=8mm]{img/nspire_notes.png}}

Mit \fbox{CTRL} \fbox{M} können wir eine Math-Box starten, worin wir Variable definieren können, worin wir aber auch symbolisch/algebraisch rechnen können.


\section{Grundoperationen und Vorrangregeln}
Wir berechnen $7 + 8(5-2)$ (danach \fbox{ENTER}) [Stimmts? $\blacktriangleright 31$];\newline und weiter rechnen wir \fbox{$7 + 8\cdot 5 - 2$} [$\blacktriangleright 45$].

Berechnen Sie weiter: \fbox{$\sqrt{(2*25)}$} [$\blacktriangleright 5\sqrt{2}$] 
und für Radius = \fbox{$r$ := 17}.\newline{}Berechnen Sie nun: \fbox{$r^2\pi$} [$\blacktriangleright 289 \pi$]

\subsection{Achtung Vorrang}
Berechnen Sie einerseits \fbox{$111s \div 37s$} [$\blacktriangleright 3s^2$] und andererseits \fbox{$111s \div (37s)$} [$\blacktriangleright 3$].\footnote{Multiplikationen ohne Punkt (\zB 8m für acht Meter) binden stärker als Multiplikationen mit Multiplikationszeichen. Diese Unterscheidung kennt jedoch der «TI-$n$spire CX II-T CAS» nicht!}

Bem.: Lieber zwei Klammern mehr als nötig!

\section{Genau vs. verständlich}
Die obigen Resultate sind oft nicht die gewünschten. Eine dezimale
Annäherung wird erreicht, indem in der
Math-Box \fbox{CTRL} \fbox{ENTER} (also $\approx$) getippt wird.

$$\sqrt{2\cdot 25} \blacktriangleright 7.071069$$
und
$$r^2\pi \blacktriangleright 907.9203$$

\newpage
\section{Variable}
Am besten weisen Sie Resultate immer sofort Variablen zu.
Es seien für eine Konservendose folgende Werte gegeben ...
\begin{itemize}
  \item $r := 10$ (\zB 10 cm)
  \item $h := 2\cdot r$
  \item $g := r^2\cdot \pi$
  \item $v := g\cdot h$
\end{itemize}

... und nun verdoppeln wir $r$: \fbox{$r := 20$}. Oh Wunder: Alle
Zwischenresultate rechnen automatisch mit (dies ist der effektive
Vorteil der «notes»-Dokumente). \TRAINER{Ebenso kann der Rechner mit
Größen umgehen. Geben Sie nun \fbox{r := 10\_cm} ein. Den «Bodenstrich»
finden wir bei der \fbox{?!$\blacktriangleright$}  Taste}

Bemerkung: Anstelle von \fbox{$r := 10$} können wir auch \fbox{$\mathstrut{}10$} \fbox{CTRL} \fbox{sto $\to$} \fbox{$\mathstrut{}r$} schreiben.



\section{Was kann der alles noch?}
Tippen Sie in einer Math-Box (s. oben) folgendes:
\fbox{$ncr(42,6)$} \fbox{ENTER}.
Das Resultat 5\,245\,786 zeigt die Anzahl Kombinationen an, die es
gibt, im Lotto 6 Zahlen aus 42 auszuwählen.

\section{Ausschalten}
Schalten Sie den Rechner wieder aus, um Batterie zu
sparen.\TRAINER{ \fbox{CTRL} \fbox{ON}}

\end{document}
