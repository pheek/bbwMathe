
\input{bmsLayoutPage}


%%%%%%%%%%%%%%%%%%%%%%%%%%%%%%%%%%%%%%%%%%%%%%%%%%%%%%%%%%%%%%%%%%

\usepackage{amssymb} %% für \blacktriangleright
\renewcommand{\metaHeaderLine}{Arbeitsblatt}
\renewcommand{\arbeitsblattTitel}{Gemischte Aufgaben aus alten Maturaprüfungen}

\begin{document}%%
\arbeitsblattHeader{}
\section{Terme Vereinfachen}
\subsection{2018 Serie 1 Aufg. 1}

Vereinfachen Sie so weit wie möglich:
$$\left(1-\frac{x}{2y}\right) : \frac{4y^2-x^2}{4y^2}$$

\subsection{2018 Serie 2 Aufg. 1}
Vereinfachen Sie so weit wie möglich.

$$\left(\frac{1}{b}-\frac{1}{a}\right) : \left( \frac{b-a}{a}\right)$$


\subsection{2016 Aufg 2a)}
Fassen Sie zusammen und schreiben Sie das Resultat möglichst einfach:

$$\frac{1.5x}{12-3x} - \frac{2.5x}{x-4}$$

\subsection{2018 Serie 3 Aufg. 1}
Vereinfachen Sie so weit wie möglich.
$$\frac{6-3a}{b} : \frac{6a-12}{-a}$$

\subsection{2018 Serie 4 Aufg. 1}
Vereinfachen Sie so weit wie möglich.
$$\left(\frac{y}{x}  - \frac{x}{y} \right) : \left( \frac{1}{x} - \frac{1}{y} \right) $$


\subsection{2017 Aufg 1}
Vereinfachen Sie so weit wie möglich.
$$\frac{1}{5}\left( a-\frac{b^2}{a}\right): \frac{a+b}{a}$$

\subsection{2018 Serie 1 Aufg. 2}
Vereinfachen Sie so weit wie möglich.\\
Notieren Sie das Resultat in Wurzelschreibweise.

$$\sqrt[3]{b\cdot{}\sqrt{b^3\cdot{}\sqrt[3]{b}}}$$


\subsection{2018 Serie 2 Aufg. 2}
Vereinfachen Sie so weit wie möglich.
Notieren Sie das Resultat in Form einer einzigen Wurzel.
$$\sqrt[3]{a^2 \cdot{} b \cdot{} \sqrt{a\cdot{}b^{-1}}}$$

\subsection{2018 Serie 3 Aufg. 2}

Vereinfachen Sie so weit wie möglich. Notieren Sie das Resultat in
Wurzelschreibweise.

$$\sqrt[3]{a^2}   \cdot{}    \sqrt[4]{a^3 \cdot{} \sqrt[3]{a}}$$

\subsection{2018 Serie 4 Aufg. 2}
Vereinfachen Sie so weit wie möglich.
$$\sqrt[3]{a} \cdot{} \sqrt{a^3\cdot{} \sqrt[3]{a} }$$

\subsection{2017 Aufg. 2}
Vereinfachen Sie so weit wie möglich.
$$\sqrt{a^{\frac{1}{2}}} \cdot{} \sqrt[4]{a\cdot{}\sqrt[5]{a^{10}}} $$


\subsection{2017 Aufg. 3}
Vereinfachen Sie so weit wie möglich.
$$\left( \frac{a\cdot{}b^{-2}\cdot{}c^3}{a^{-2}\cdot{}b}\right)^{-2}
\cdot{} \left( \frac{a^3 \cdot{} c^2}{b^2}\right)^3$$

\subsection{2016 Aufg. 2b)}

Fassen Sie zusammen und schreiben Sie das Resultat möglichst einfach:
Wurzel- oder Potenzschreibweise sind möglich.

$$\sqrt{x^{10}}  \cdot{}   \sqrt[4]{x^3} \cdot{}  x^{\frac{1}{4}}$$

\subsection{2018 Serie 1 Aufg. 3}
a) Vereinfachen Sie so weit wie möglich.

$$\left( \frac{a^2\cdot{}b^{-3}\cdot{}c^3}{a\cdot{b}} \right)^{-3}
\cdot{} \left(\frac{c^5}{a\cdot{}b} \right)^2$$

b) Vereinfachen Sie so weit wie möglich.

$$-\left( \left(-a\right)^3 \right)^{10}$$


\subsection{2018 Serie 2 Aufg. 3}
  a) Vereinfachen Sie so weit wie möglich.
  $$\left( \frac{3a^{-1}b^2}{2ac^{-1}}\right)^{-2} \cdot \frac{\left(3b^2 \right)^2 c^2}{4}$$

  b)Vereinfachen Sie so weit wie möglich.

$$\left( - \left( -x^2 \right) \right)^7$$

  \subsection{2018 Serie 3 Aufg. 3}
  a) Vereinfachen Sie so weit wie möglich. Im Resultat müssen alle
  Exponenten positiv sein.
  $$\left( \frac{a^3 \cdot{} c^{-1}}{b^{-2}} \right)^{-4}  \cdot  \left( \frac{a^4\cdot{}b^3}{c^{-2}} \right)^3 $$


  b) Vereinfachen Sie so weit wie möglich.
  $$-\left(  (a^3) \cdot{} a^{-1} \right)^2$$


\subsection{2016 Aufg. 2c)}
Fassen Sie zusammen und schreiben Sie das Resultat möglichst einfach:

$$-\left(  -\left( -a\right)^2   \right)^0$$  

\subsection{2016 Aufg 2d)}
Fassen Sie zusammen und schreiben Sie das Resultat möglichst einfach:
$$\frac{6\cdot{} a^3 \cdot{} b^7 \cdot{} 3}{(ab)^4 \cdot 9 \cdot a^{-1}}$$
Im Resultat sollen nur positive Exponenten vorkommen.

  \subsection{2018 Serie 4 Aufg. 3}
  a) Vereinfachen Sie so weit wie möglich.
  $$\left( \frac{a^4\cdot{} b^{-2}\cdot{} c}{a^2 \cdot{} c^{-3}} \right)^{-2}
  \cdot{}
  \left(\frac{c^2\cdot{}b}{a^{-1}} \right)^4$$ %%


  b) Vereinfachen Sie so weit wie möglich.

  $$-\left(  -a^3 \cdot (-a)^2  \right)^{-4}$$

  \subsection{2017 Aufg. 4}
  Bestimmen Sie den Definitionsbereich des folgenden Terms bezüglich
  der Grundmenge $\mathbb{R}$.
  $$\frac{5x}{x^2 -9x - 36}$$

  \subsection{2017 Aufg. 6}
  Notieren Sie die einzelnen Summanden in möglichst einfacher Form und berechnen Sie die
  Summe.
  $$\sum_{k=0}^{3}\left(\frac{1}{2}k\right)$$




  \subsection{2016 Aufg. 1}
  Schreiben Sie jeden einzelnen Summanden auf und berechnen Sie die
  Summe:
  $$\sum_{k=0}^{5}(2k+1)$$



 %%%%%%%%%%%%%%%%%%%%%%%%%%%%%%%%%%%%%%%%%%%%%%%%%%%%%%%%%%%%%%%%%%%%%%%%%%%%%% 
  \section{Gleichungen}
  
\subsection{2018 Serie 1 Aufg. 4}

Lösen Sie die Gleichung mithilfe von Zehnerlogarithmen nach $x$ auf.
Notieren Sie die einzelnen Lösungsschritte und geben Sie das Resultat
möglichst einfach an.

$$\frac{3^x}{5} = 20$$


\subsection{2018 Serie 2 Aufg. 5}
Lösen Sie die Gleichung mithilfe von Zehnerlogarithmen nach $x$ auf.
Notieren Sie die einzelnen Lösungsschritte und geben Sie das Resultat
möglichst einfach an.

$$\frac{2^x}{8}= 12.5$$

\subsection{2018 Serie 3 Aufg. 5}
Lösen Sie die Gleichung mithilfe von Zehnerlogarithmen nach $x$ auf.
Notieren Sie die einzelnen Lösungsschritte und geben Sie das Resultat
möglichst einfach an.

$$\frac{1}{4}\cdot{} 6^x = 250$$


\subsection{2018 Serie 4 Aufg. 5}

Lösen Sie die Gleichung mithilfe von Zehnerlogarithmen nach $x$ auf.
Notieren Sie die einzelnen Lösungsschritte und geben Sie das Resultat
möglichst einfach an.
$$\frac{1}{2}\cdot{} 12^{3x}=5$$

\subsection{2017 Aufg. 5}
Berechnen Sie $x$. Schreiben Sie das Resultat exakt, d.\,h. mithilfe von Zehnerlogarithmen.
$$\frac{1}{2}\cdot{} 4^x = 9.5$$


\subsection{2016 Aufg. 3b)}
Lösen Sie die Gleichung nach x auf und schreiben Sie das Resultat
möglichst einfach:

$$32 = \left( 2^3 \right)^x  \cdot 2^{1.5x - 4}$$

Es wird ein ausführlicher Lösungsweg \textit{\textbf{ohne Anwendung der Solve-Funktion des
Taschenrechners}} verlangt .


\subsection{2018 Serie 1 Aufg. 5}
Bestimmen Sie den Definitionsbereich und die Lösungsmenge der Gleichung.
Die Gleichung ist auf Grundform $ax^2 + bx + c = 0$ zu bringen und
kann dann mit dem entsprechenden Taschenrechnermodus gelöst werden.

$$\frac{6x-24}{3-x}+x-2=\frac{6}{x-3}$$


\subsection{2018 Serie 2 Aufg. 4}

Bestimmen Sie den Definitionsbereich und die Lösungsmenge der Gleichung.
Die Gleichung soll auf die Grundform $ax^2 + bx + c = 0$ gebracht werden und
kann dann mit dem entsprechenden Taschenrechnermodus gelöst werden.

$$\frac{x^2}{x-2} + \frac{4}{2-x} = 3$$


\subsection{2018 Serie 3 Aufg. 4}
Bestimmen Sie den Definitionsbereich und die Lösungsmenge der Gleichung.
Die Gleichung ist auf Grundform $ax^2 + bx + c = 0$ zu bringen und kann dann mit
dem entsprechenden Taschenrechnermodus gelöst werden.

$$\frac{x^2-16x}{x-3} + 1= \frac{39}{3-x} $$


\subsection{2018 Serie 4 Aufg. 4}
Bestimmen Sie den Definitionsbereich und die Lösungsmenge der Gleichung.
Die Gleichung ist auf Grundform $ax^2 + bx + c = 0$ zu bringen und kann dann
mit dem entsprechenden Taschenrechnermodus gelöst werden.

$$\frac{x^2 - 10x}{x-4}+1= \frac{24}{4-x}$$

\subsection{2017 Aufg. 8}

Bestimmen Sie die Lösung(en) der Gleichung in der Grundmenge
$\mathbb{R}$.

$$\frac{1+x}{x-3} = \frac{10-2x}{x^2-3x}$$



\subsection{2018 Serie 1 Aufg 7}

Lösen Sie die Gleichung nach $x$ auf.
Notieren Sie das Ergebnis so einfach wie möglich.

$$4(ax-b^2) = 2(ax+2a^2 -bx + 4ab)$$

\subsection{2018 Serie 2 Aufg. 7}

Lösen Sie die Gleichung nach $x$ auf. Notieren Sie das Ergebnis so einfach wie möglich.

$$a\cdot{} (x-3a) = b\cdot{} (x-6a+3b)$$

\subsection{2018 Serie 3 Aufg. 7}
Lösen Sie die Gleichung nach $x$ auf.
Notieren Sie das Ergebnis so einfach wie möglich.

$$\frac{3a (x-3a+2.5b)}{b}  =  -x + 7.5a -b$$

\subsection{2018 Serie 4 Aufg. 6}
Lösen Sie die Gleichung nach x auf.
Notieren Sie das Ergebnis so einfach wie möglich.
$$ a(x - a + 10.5b) = 5b( x + 2.1a - 5b)$$


\subsection{2017 Aufg. 7}
Bestimmen Sie die Lösung. Schreiben Sie das Ergebnis so einfach wie möglich.
Die Lösungsvariable ist $x$.
$$7a - x - c = \frac{7b^2 -ac + bx}{a}$$


\subsection{2016 Aufg. 3}

Lösen Sie die Gleichung nach x auf und schreiben Sie das Resultat möglichst einfach:
a)

$$a(2a-2b)  - ax = - \frac{b(x+b)}{2}$$


\section{Textaufgaben}

\subsection{2018 Serie 1 Aufg. 8}
Ein Computerhändler kaufte 30 Stück des Computermodells „Standard“ und
20 Stück des Modells „High Speed“ für insgesamt CHF 168\,000.- ein.
Das Modell „Standard“ verkaufte er mit einem Zuschlag von 20\% und das
Modell „High Speed“ mit einem Zuschlag von 25\%. Nachdem alle Geräte
verkauft waren, resultierte ein Gewinn von CHF 37\,800.-.
Für wie viel CHF hatte er ein Stück jeder Sorte eingekauft?
Die Aufgabe ist mittels einer Gleichung oder eines Gleichungssystems zu lösen.
Es wird ein Antwortsatz verlangt.

\subsection{2018 Serie 2 Aufg. 8}
in Vermögen von CHF 145\,000.- wird auf zwei Sparkonten aufgeteilt:\\
Ein Teil des Vermögens liegt auf dem Konto A zu einem Zinssatz von 1.2\%.
Der andere Teil liegt auf Konto B zu einem Zinssatz von 2.5\%.
Der totale Ertrag beider Jahreszinsen beträgt CHF 2\,305.50.
Wie gross war der Vermögensteil, der zu 1.2\% angelegt war und wie gross
derjenige Teil, der zu 2.5\% angelegt war?
Es werden eine Gleichung oder ein Gleichungssystem sowie ein Antwortsatz
verlangt.


\subsection{2017 Aufg. 9}
Herr Graf hat ein Kapital in zwei Teil-Posten, A und B, angelegt. Posten A wird zu 4\%,
Posten B zu 5\% verzinst.
Herr Graf möchte die Summe der Jahreszinsen beider Posten berechnen. Es unterläuft
ihm jedoch ein Fehler: Er verwechselt die Zinssätze und kommt so auf eine Zinssumme
von CHF 2\,480.-. Gegenüber der korrekten Berechnung mit den richtigen Zinssätzen sind
dies jedoch CHF 80.- zu wenig.\\
Wie gross ist der Posten A, der zu 4\% verzinst wird?\\
Die Aufgabe ist mithilfe einer Gleichung oder eines Gleichungssystems zu lösen.\\
Es wird ein Antwortsatz verlangt.

\subsection{2016 Aufg. 6}
Folgende Aufgabe ist mit Hilfe einer Gleichung zu lösen:

Gegeben sind sieben aufeinanderfolgende natürliche Zahlen.
Multipliziert man die Summe dieser sieben Zahlen mit 15, so ist das Ergebnis um 715
größer als das Produkt der beiden kleinsten Zahlen.
Berechnen Sie die größte dieser sieben Zahlen. Geben Sie alle Lösungsmöglichkeiten
an.



\end{document}
