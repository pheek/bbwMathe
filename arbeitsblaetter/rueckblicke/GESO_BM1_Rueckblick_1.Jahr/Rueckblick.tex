
\input{bbwLayoutPage}

%%%%%%%%%%%%%%%%%%%%%%%%%%%%%%%%%%%%%%%%%%%%%%%%%%%%%%%%%%%%%%%%%%

\usepackage{amssymb} %% für \blacktriangleright
\renewcommand{\metaHeaderLine}{Rückblick}
\renewcommand{\arbeitsblattTitel}{1. Jahr GESO}

\begin{document}%%
\arbeitsblattHeader{}

\section{Zahlmengen}
Zu welchen Mengen $\mathbb{N}, \mathbb{Z}, \mathbb{Q}, \mathbb{R}$ gehören die Zahlen

1; -4; 8.3; $\frac{\pi}{3}$; $\sqrt{4}$

\section{Signifikante Stellen}
Runden Sie

$34.4496$

a) auf 3 Dezimalstellen \TRAINER{34.450}

b) auf 3 Signifikante Stellen \TRAINER{34.4}


\section{Termanalyse}
Was ist das für ein Term?

$3 - (a\cdot{}r)^{2-x}$ \TRAINER{Subtraktion / Differenz}

\section{Brüche und -1}
Kürzen Sie den folgenden Bruch:

$\frac{7b^2 - 7a^2}{-21(a-b)}$\TRAINER{$ = \frac{a+b}{3}$}

\section{Binomische Formeln / Faktorisieren}
Faktorisieren Sie:
$r^2 + s^2 + 2rs$    \TRAINER{$(r+s)^2$}

Faktorisieren Sie:
$x^2 + 5x + 6$ \TRAINER{$(x+2)(x+3)$}

\section{Datenanalyse}
\subsection{Kuchendiagramm}
Die folgenden Prozentzahlen sollten als Winkel in einem Kuchendiagramm umgerechnet werden:

\begin{tabular}{ccc}
rot   &  20\% & .....\TRAINER{72 deg}\\
blau  &  40\% & .....\TRAINER{144 deg}\\
grün  &  15\% & .....\TRAINER{54 deg}\\
gelb  &  ??\%\TRAINER{25\%} & .....\TRAINER{90 deg}\\
\end{tabular}

\subsection{Histogramm}
Nennen Sie einige Merkmale, die ein sauberes Histogramm aufweisen sollten:

\noTRAINER{
............................................

............................................

............................................

............................................
}
\TRAINER{Alle Säulenbreiten gleich; Achsen beschriften; y-Achse bei 0 beginnen; Säulenenden (links/rechts) einfache Zahlen und einfach ablesbar; Werten auf Säulenenden konsequent nach links bzw. rechts.}


\subsection{Boxplot}
Zeichnen Sie einen Boxplot zu den folgenden Messwerten:

30, 35, 35, 39, 39, 42, 42, 45, 45, 45, 46, 48, 51, 53, 54, 54, 102

\noTRAINER{\vspace{3cm}}
\TRAINER{Box: Q1 = 39, Median = 45, Q3 = 52;

Whisker und Ausreißer: IQR: 52-39 = 13; 1.5*IQR = 19.5; Untere Ausreißer Schwelle: 39-19.5 = 19.5; obere Ausreißer Schwelle: 71.5; keine unteren Ausreißer, ergo unterer Whisker bei 30. Oberer Ausreißer bei 102 ergo oberer Whisker bei 54}


\subsection{Manipulation}
Welche Arten den Datenmanipulation kennen Sie?

\noTRAINER{
............................................

............................................

............................................

............................................
}
Antworten S. Skript und Buch Auf 34 S. 408.


\section{lineare Gleichungen}
Lösen Sie nach $x$ auf: $7x+3=0$ \TRAINER{$x = \frac{-3}{7}$}

Lösen Sie nach $x$ auf: $ax + b = 0$ \TRAINER{$\lx = \frac{-b}{a}$}



\subsection{quadratische Gleichung}
Lösen Sie von Hand nach $x$ auf:

$$4x^2 +4x = 8$$\TRAINER{$\lx = \{-2; 1\}$}

Lösen Sie mit dem Taschenrechner:

$$4x^2 + 4x = 9$$\TRAINER{Verwende poly-solve}

\section{Bruchgleichung}
Bestimmen Sie die Definitions- und die Lösungsmenge der folgenden Bruchgleichung:

$$\frac{1}{x-3} = 5$$\TRAINER{D=R ohne 3; L = 3.2}

\end{document}
