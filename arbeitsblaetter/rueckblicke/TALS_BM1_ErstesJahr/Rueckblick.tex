\input{bbwLayoutPage}

%%%%%%%%%%%%%%%%%%%%%%%%%%%%%%%%%%%%%%%%%%%%%%%%%%%%%%%%%%%%%%%%%%

\usepackage{amssymb} %% für \blacktriangleright
\renewcommand{\metaHeaderLine}{Rückblick}
\renewcommand{\arbeitsblattTitel}{1. Jahr TALS}

\begin{document}%%
\arbeitsblattHeader{}

\section{Zahlmengen}
Zu welchen Mengen $\mathbb{N}, \mathbb{Z}, \mathbb{Q}, \mathbb{R}$ gehören die Zahlen

1; -4; 8.3; $\frac{\pi}{3}$; $\sqrt{4}$; $\frac{3\pi}{2\sqrt{\pi^2}}$; $\sqrt{18-7\pi}$

\section{Signifikante Stellen}
Runden Sie

$34.4496$

a) auf 3 Dezimalstellen \TRAINER{34.450}

b) auf 3 Signifikante Stellen \TRAINER{34.4}


\section{Betrag}
Für welche $x$ gilt:

$|x+1| < 4$

\TRAINER{x < 3 UND x > -5}

\section{Termanalyse}
Was ist das für ein Term?

$3 - (a\cdot{}r)^{2-x}$ \TRAINER{Subtraktion / Differenz}

\section{Binomische Formeln / Faktorisieren}
Faktorisieren Sie:

$r^2 + s^2 + 2rs$    \TRAINER{$(r+s)^2$}

$x^2 + 5x + 6$ \TRAINER{$(x+2)(x+3)$}


\section{Potenzen und Wurzeln}
Vereinfachen Sie $a^3 \cdot{} a^4$. \TRAINER{$a^7$}

Fassen Sie zusammen $a^4 \cdot{} b^4$. \TRAINER{$(ab)^4$}

Vereinfachen Sie $\sqrt{4a^2b}$. \TRAINER{$2a\sqrt{b}$}

\section{lineare Gleichungen}
Lösen Sie nach $x$ auf: $7x+3=0$ \TRAINER{$x = \frac{-3}{7}$}

Lösen Sie nach $x$ auf: $ax + b = 0$ \TRAINER{$\lx = \frac{-b}{a}$}

\section{Trigonometrie}
Zeichnen Sie ein rechtwinkliges Dreieck mit den Seiten $a=3$ und $b=4$.
Wie groß ist die Hypotenuse $c$? \TRAINER{5 (Pythagoras)}

Welches Verhältnis beschreibt der Sinus des Winkels $\beta$ (gegenüber von $b$)?\TRAINER{$\sin(\beta)$ = Gegenkathete dividiert durch Hypotenuse = $\frac{b}{c}$}

Berechnen Sie $\beta$ aus obigem Beispiel: \TRAINER{53.13 Grad}


\subsection{Sinussatz / Kosinussatz}
In einem Dreieck ABC sind $b$, $c$ und der dazwischen liegende Winkel $\alpha$ gegeben. Wie lange ist die Seite $a$?\TRAINER{$a=\sqrt{b^2 + c^2 -2bc\cos(\alpha)}$}

Lösen Sie $a$ mit dem Taschenrechner für $b=5$cm , $c=7.3$cm und $\alpha=40^{\circ{}}$. \TRAINER{$a \approx{} 4.72956$cm}

\subsection{Sinussatz}
Wie lautet der Sinussatz im allg. Dreieck?\TRAINER{$\frac{a}{\sin(\alpha)} = \frac{b}{\sin(\beta)} = \frac{c}{\sin(\gamma)}$}

\subsection{Fläche}
Berechnen Sie mit der (sin-)Flächenformel die Fläche des Dreiecks $b = 5$cm, $c=7.3$cm mit Zwischenwinkel $\alpha{} = 40^{\circ{}}$. \TRAINER{$A=bc*\sin(\alpha)/2$. Hier $A\approx{} 11.73cm^2$}

\section{lineare Funktionen}
Zeichnen Sie $f: y=\frac{1}{3}x+1.5$.

Wie lauten die Schnittpunkte mit den Achsen der obigen Gerade?\TRAINER{$(-4.5|0)$ und $(0| 1.5)$}

Wie lauten die Schnittpunkte mit den Achsen der allgemeinen Form $f: y=ax+b$? \TRAINER{$(\frac{-b}{a}|0)$ und $(0|b)$}.

Woher kennen Sie den Term $\frac{-b}{a}$?\TRAINER{Nullstelle der linearen Funktion = Lösung der entsprechenden linearen Gleichung.}

\section{quadratische Gleichung}
Lösen Sie von Hand nach $x$ auf:

$$4x^2 +4x = 8$$\TRAINER{$\lx = \{-2; 1\}$}

Lösen Sie mit dem Taschenrechner:

$$4x^2 + 4x = 9$$\TRAINER{Verwende SOLVE}


\section{Planimetrie}
\subsection{spezielle Dreiecke}
In einem rechtwinkligen Dreieck mit einem 30 Grad Winkel, misst die Hypotenuse 4.6cm. Wie lange ist die kürzere Kathete? \TRAINER{30/60 Dreieck: 4.6/2 = 2.3cm}

\subsection{Kreisberührung}
Was ist die generelle Strategie in Kreisberührungsaufgaben?\TRAINER{Verbinde die Mittelpunkte zweier sich berührender Kreise (Drei-Punkte-Strecke).}

\bbwCenterGraphic{5cm}{img/Kreisberuehrung.png}
Drücken Sie in obiger Graphik $r$ durch $a$ aus.
\TRAINER{$$r = \frac{a}{6}$$ Herleitung: $(a-r)^2
= \left(\frac{a}2\right)^2 + \left(\frac{a}2 + r\right)^2$}
\subsection{Ähnlichkeit und Strahlensätze}
Zwei Figuren $A$ und $A'$ seien ähnlich:

Wie wird der Faktor $k$ bezeichnet, mit dem sich alle Strecken der Figur $A$ in die entsprechenden Strecken der Figur $A'$ überführen lassen?\TRAINER{$k$ = Streckungsfaktor}

Wie verhalten sich die Flächen ähnlicher Figuren?\TRAINER{Fläche($A'$)
/ Fläche ($A$) = $k^2$}

Zwei Kreise haben ein Flächenverhältnis von 7:9. Der Radius des
kleinen Kreises misst 1.5cm. Wie lang ist der Durchmesser des großen
Kreises? (Genau lesen: Radius ist gegeben, Durchmesser ist gesucht!)
\TRAINER{$$k^2 = \frac97 \Longrightarrow k=\frac{3\cdot{}\sqrt{7}}7$$
Daher: Durchmesser Großer Kreis = $2\cdot{}$ Radius großer Kreis:
$$D = 2\cdot{}1.5\cdot{}\frac{3\cdot{}\sqrt{7}}7
= \frac{9\cdot{}\sqrt{7}}7 \approx 3.40168$$}


\end{document}
