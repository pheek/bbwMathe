%%
%% Meta: Boxplot erstellen. Viele Beispiele
%%
  
\input{bbwLayoutSlide}


%%%%%%%%%%%%%%%%%%%%%%%%%%%%%%%%%%%%%%%%%%%%%%%%%%%%%%%%%%%%%%%%%%


\usepackage{cancel}

\begin{document}%%
\thispagestyle{fancy}
\section*{Wie lernen?}
{\small{Nach einer Idee von M. Rohner @ bbw}}
\newpage
Welche Kantone bezeichnen die folgenden Wappen?

\def\wapp#1{\includegraphics[width=25mm]{img/#1.png}}

\begin{tabular}{cccc}
 1         & 2         & 3         & 4           \\
 \wapp{zh} & \wapp{ur} & \wapp{ge} & \wapp{sg}   \\
 5         & 6         & 7         & 8           \\
 \wapp{ag} & \wapp{gr} & \wapp{gl} & \wapp{ju}%%
 \end{tabular}%%
%%
%% implicit newpage because of section
\section*{Auf\/lösung}
\begin{tabular}{cccc}
 1. Zürich  & 2. Uri        & 3. Genf   & 4. St. Gallen \\
 \wapp{zh}  & \wapp{ur}     & \wapp{ge} & \wapp{sg}     \\
 5. Aargau  & 6. Graubünden & 7. Glarus & 8. Jura       \\
 \wapp{ag}  & \wapp{gr}     & \wapp{gl} & \wapp{ju}%%
 \end{tabular}%%
\newpage
\section*{Umkehrung}
Zeichnen Sie die Wappen von
\begin{itemize}
\item Zürich
\item Aargau
\item Tessin
\item Freiburg
\item Schaffhausen

\end{itemize}
\newpage

\section*{Auf\/lösung}
\begin{tabular}{ccccc}
 Zürich    & Aargau    & Tessin     & Freiburg  & Schaffhausen \\
 \wapp{zh} & \wapp{ag} & \wapp{ti}  & \wapp{fr} & \wapp{sh}
 \end{tabular}%%
\newpage

\section*{Lerntipps}
\begin{itemize}
\item Möglichst aktiv lernen (tun); nicht bloß anschauen/zuhören
\item Kontinuierlich: Besser 5 Mal 15 Minuten als zwei Stunden am Stück
\item Reflektieren, Eigene Fehler verstehen
\end{itemize}

\end{document}
