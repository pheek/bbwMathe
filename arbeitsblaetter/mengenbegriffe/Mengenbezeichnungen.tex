%%
%% 2019 11 25 Ph. Freimann
%%

\input{bbwLayoutPage}

%%%%%%%%%%%%%%%%%%%%%%%%%%%%%%%%%%%%%%%%%%%%%%%%%%%%%%%%%%%%%%%%%%

\usepackage{amssymb} %% für \blacktriangleright
\renewcommand{\metaHeaderLine}{Theorieblatt}
\renewcommand{\arbeitsblattTitel}{Mengenbezeichnungen}

\begin{document}%%
\arbeitsblattHeader{}

\section{Zahlmengen}
Wir bezeichnen als

\begin{tabular}{lcl}
  $\mathbb{N}$ &=& Natürliche Zahlen: $\{1, 2, 3, 4, ...\}$ \\
  $\mathbb{Z}$ &=& Ganze Zahlen: $\{..., -4, -3, -2, -1, 0, 1, 2, 3, 4, ...\}$ \\
  $\mathbb{Q}$ &=& Rationale Zahlen: Alle Brüche = $\{\frac{a}{b} |
  a\in\mathbb{Z}\textrm{\, und\, } b\in\mathbb{N}\}$ \\
  $\mathbb{R}$ &=& Alle reellen Zahlen\\
\end{tabular}

Einschränkungen: Ein hochgestelltes Minus- bzw Pluszeichen ($-$, $+$)
schränkt die Menge auf negative bzw positive Zahlen ein. Ebenso eine
tiefgestellte Null ($0$) sagt, dass die Zahl Null explizit mit
eingeschlossen ist. Beispiele:

\begin{tabular}{lcl}
  $\mathbb{Z}^{-}$ &=& $\{-1, -2, -3, -4, ...\}$ \\
  $\mathbb{Q}^{+}$ &=& $\{x\in \mathbb{Q} | x > 0\}$ \\
  $\mathbb{Q}^{+}_0$ &=& $\{x\in \mathbb{Q} | x \ge 0\}$ \\
  $\mathbb{R}\backslash{}\{0\}$ &=& $\{x\in \mathbb{R} | x \ne 0\}$ \\
\end{tabular}

Dies gilt analog für $\mathbb{R}^{-}$ etc.

\section{Definitionsmenge für Terme}

\begin{tabular}{lclcp{10cm}}
  $\DefinitionsMenge{}$ &=& Definitionsmenge &=& Menge aller Zahlen, für die ein Term definiert ist.\\
\end{tabular}

Beispiel $\DefinitionsMenge{}(\frac{1}{\sqrt{x}}) = \mathbb{R}^{+}\backslash \{0\}$.

\section{Mengenbezeichnungen bei Gleichungen}
\begin{tabular}{lclcp{10cm}}
  $\mathbb{G}$ &=& Grundmenge &=& Menge aller Zahlen, welche für die Lösung zugelassen sind (typischerweise gilt $\mathbb{G}=\mathbb{R}$). \\
  $\DefinitionsMenge{}$ &=& Definitionsmenge &=& Diejenige Teilmenge von
                                        $\mathbb{G}$ für welche die Terme der Gleichung definiert sind\\
   & & & &  ($\DefinitionsMenge{}=\DefinitionsMenge{}(\textrm{Term links})\cap\DefinitionsMenge{}(\textrm{Term rechts})$).\\

   $\LoesungsMenge{}$ bzw. $\lx$ &=& Lösungsmenge &=& Alle Zahlen
  aus $\DefinitionsMenge{}$, welche die Gleichung lösen.\\

  $\lx$ &=&$\{\}$ &:& Leere Menge: Es gibt keine Zahlen
  (aus $\DefinitionsMenge{}$), welche die Gleichung lösen (\zB{} $x+2=x$).\\

  $\lx$ &=&$\{0\}$ &:& Einzelne Zahl: Nur eine Zahl löst die
  Gleichung (hier \zB{} $6+5x-10=3x+6x-4$).\\

 $\lx$ &=&$\{3, -2\}$ &:& Einzelne Zahlen: Nur einzelne
  Zahlen lösen die Gleichung (hier \zB{} $x^2-x=6$).\\

  $\lx$ &=& $\mathbb{R}$ &:& Alle Zahlen lösen die Gleichung
  (\zB{} $2x+x=4x-x$).\\

\end{tabular}


\end{document}
