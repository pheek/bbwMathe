%%
%% Meta: TI nSpire Einführung
%%       Ziel: Damit die Grundoperationen damit durchgeführt werden können.
%%             Damit man sich an den Rechner gewöhnt.
%%

\input{bbwLayoutPage}
\renewcommand{\bbwAufgabenBlockID}{F1Lin}
%%%%%%%%%%%%%%%%%%%%%%%%%%%%%%%%%%%%%%%%%%%%%%%%%%%%%%%%%%%%%%%%%%

\usepackage{amssymb} %% für \blacktriangleright
\renewcommand{\metaHeaderLine}{Arbeitsblatt}
\renewcommand{\arbeitsblattTitel}{Lineare Funktionen (V 3.1)}

\begin{document}%%
\arbeitsblattHeader{}

\title{Aufgaben Funktionsbegriff}

\section{Darstellung}
\subsection{Wertetabelle und Graph}
Gegeben ist die Funktion $f$:
$$y=\frac32x-5$$


a) Erstellen Sie eine Wertetabelle für $f$:

\TRAINER{%%
\begin{tabular}{|c|c|c|c|c|c|c|c|c|c|c|}\hline
$-4$ & $-3$ & $-2$ & $-1$ & $0$ & $1$ & $2$ & $3$ & $4$ & $5$ & $6$\\\hline%%
$-11$ & $-9.5$ & $-8$ & $-6.5$ & $-5$ & $-3.5$ & $-2$ & $-0.5$ & $1$ & $2.5$ & $4$\\\hline%%
\end{tabular} 
}%% end TRAINER
\noTRAINER{%%
\begin{tabular}{|c|c|c|c|c|c|c|c|c|c|c|}
$-4$ & $-3$ & $-2$ & $-1$ & $0$ & $1$ & $2$ & $3$ & $4$ & $5$ & $6$\\\hline
 \hspace{7mm} & \hspace{7mm} & \hspace{7mm} & \hspace{7mm} &\hspace{7mm} &\hspace{7mm} &\hspace{7mm} &\hspace{7mm} &\hspace{7mm} &\hspace{7mm} &\hspace{7mm}\,\,\,\,%%
\end{tabular} 
}%% end noTRANIER

b) Zeichnen Sie den Graphen von $f$ ins folgende
Koordinatensystem; natürlich nur im passenden Bereich (Tipps:
Taschenrechner / Grundform):

\bbwGraph{-2}{6}{-7}{2}{%%
\TRAINER{\bbwFunc{\x / 2 * 3 - 5}{-1:4}}
}%% end bbwGraph

c) Geben Sie die beiden Schnittpunkte mit den Achsen an:
$$S_{y\text{-Achse}} = \LoesungsRaum{(0|-5)}
\text{ und }
S_{x\text{-Achse}} = \LoesungsRaum{\left(\frac{10}3 \middle| 0\right)}$$
\newpage

\subsection{Skizzieren}
Skizzieren Sie die folgenden Funktionen $f_1$ bis $f_6$:

$f_1$: $y=2x+3$

\bbwGraph{-4}{5}{-3}{5}{
\TRAINER{\bbwFunc{2*\x + 3}{-3:1}}
}


$f_2$: $y=\frac12x-1$

\bbwGraph{-4}{5}{-3}{1}{
\TRAINER{\bbwFunc{0.5*\x  - 1}{-3:4}}
}
\newpage


$f_3$: $y=-x+1.5$

\bbwGraph{-4}{5}{-3}{5}{
\TRAINER{\bbwFunc{-\x  +1.5}{-3:4}}
}



$f_4$: $y=\frac23x-3$

\bbwGraph{-4}{6}{-5}{2}{
\TRAINER{\bbwFunc{\x*2/3 -3}{-3:5}}
}
\newpage


$f_5$: $2x=2y-4$ \TRAINER{$\Longrightarrow  y=x+2$}

\bbwGraph{-5}{5}{-2}{5}{
\TRAINER{\bbwFunc{\x+2}{-4:3}}
}



$f_6$: $\frac{2-x}3 = \frac{y-x}6$ \TRAINER{$\Longrightarrow  y=-x+4$}

\bbwGraph{-3}{5}{-2}{6.5}{
\TRAINER{\bbwFunc{-\x  +4}{-2:5}}
}
\newpage

\section{$y$-Achsenabschnitt (= Ordinatenabschnitt)}

\subsection{$y$-Achsenabschnitt rechnen}

Geben Sie jeweils den $y$-Achsenabschnitt an:

a) $y=3x-2$ $\Longrightarrow$ $y$-Achsenabschnitt =
$\LoesungsRaum{-2}$

\noTRAINER{\mmPapier{2.4}}

b) $y=4x+\frac12$ $\Longrightarrow$ $y$-Achsenabschnitt =
$\LoesungsRaum{\frac12}$

\noTRAINER{\mmPapier{3.2}}

c) $2(x+y+1) = y-4x+3$ $\Longrightarrow$ $y$-Achsenabschnitt =
$\LoesungsRaum{1}$

\TNTeop{
Umformen in Grundform:
$$2(x+y+1) = y-4x+3$$
$$2x+2y+2 = y-4x+3$$
$$2x+2y-1 = y-4x$$
$$6x+2y-1 = y$$
$$6x+y-1 = 0$$
$$6x+y= 1$$
$$y= -6x + 1$$
}
%% implicit: \newpage

\subsection{Funktionsgleichung finden}
Die Funktion $y=3x+b$ geht durch den Punkt $(0|7)$.
Berechnen Sie den Parameter $b$ oder mit anderen Worten: Geben Sie die
Funktionsgleichung an:

\TNT{4.8}{$$y=3x+7$$ denn $7$ ist als $y$-Achsenabschnitt gegeben.}
\newpage

\section{Steigung}

\subsection{Steigung von Strecken}
Geben Sie die Steigungen der folgenden Strecken an:

\bbwGraph{-8}{7}{-6}{6}{%
\bbwStrecke{ 2, 3}{ 6, 5}{ 4  , 4.5}{a)}{blue}
\bbwStrecke{-6, 4}{-3, 3}{-4.5,4}{b)}{orange}
\bbwStrecke{-7,-4}{-4,-4}{-5.5,-4.5}{c)}{green}
\bbwStrecke{-3,-2}{ 2,-5}{ 1  ,-4  }{d)}{black}
\bbwStrecke{ 0,-1}{ 4, 2}{ 2  , 1  }{e)}{pink}
\bbwStrecke{ 4,-3}{ 4,-5}{ 4.5,-4  }{f)}{ForestGreen}
}%% end BBW Graph

Lösungen:

\TNTeop{
a) $\frac12$; b) $-\frac13$; c) $0$; d) $-\frac35$; e) $\frac34$

Bei f) wäre die Steigung unendlich hoch. Da die Gerade, welche die
Strecke $f$ verlängert aber sowieso keine Funktion ist, entfällt eine
weitere Betrachtung; f) hat keine Lösung.
}
\newpage

\subsection{Geradensteigung und $y$-Achsenabschnitt bestimmen}

%%\bbwGraph{-8}{8}{-6}{5}{}

\newcommand{\bbwDotGerade}[7]{%%
\bbwLine{#1}{#2}{#5}
\bbwDot{#3}{#5}{west}{}
\bbwDot{#4}{#5}{west}{}%
\draw (#7) node{{\color{#5}#6}};
}%%

\bbwGraph{-7}{8}{-5}{5}{%
% a (blau)
\bbwDotGerade{-7,-1.5}{4,4}{-6,-1}{2,3}{blue}{a)}{2.5, 4}
% b (rot)
\bbwDotGerade{-6,2}{3,-4}{-4.5, 1}{1.5,-3}{red}{b)}{2.5,-4.5}
% c (ForestGreen)
\bbwDotGerade{-4,5}{4,-3}{-3, 4}{2.5,-1.5}{ForestGreen}{c)}{3.5,-2}
% d (black)
\bbwDotGerade{-7,2.5}{6,2.5}{-4, 2.5}{3,2.5}{black}{d)}{6,3}
% e (orange)
\bbwDotGerade{-7,-3}{8,2}{-4, -2}{5,1}{orange}{e)}{-5,-3}

}%% end BBW Graph



\TNTeop{
a) $y = \frac12 x + 2$

b) $y = \frac{-2}{3}x - 2$

c) $y = -x + 1$

d) $y = 0x + 2.5 = 2.5$

e) $y = \frac13 x  - \frac23$
}
%%%%%%%%%%%%%%%%%%%%%%%%%%%%%%%%%%%%%%%%%%%%%%%%%%%%%%%%%%%5

\section{Nullstelle}

\subsection{Steigung, $y$-Achsenabschnitt und Nullstelle bestimmen}
Bestimmen Sie den $y$-Achsenabschnitt $b$, die Nullstelle ($y=0$, d.\,h. $x=\frac{-b}{a}$) und die Steigung $a$ der folgenden Funktionen:

$$y = a\cdot{}x + b$$

\begin{bbwFillInTabular}{l|c|c|c}
 Funktionsgleichung  & $y$-Achsenabschnitt $b$ & Nullstelle $\frac{-b}{a}$ & Steigung $a$\\\hline
 
$y=3x + 3$ & \LoesungsRaum{3} & \LoesungsRaum{$-1$} & \LoesungsRaum{3} \\\hline

$y=4x-2$ & \LoesungsRaum{-2} & \LoesungsRaum{$\frac{1}{2}$} & \LoesungsRaum{4} \\\hline

$y=-\frac{1}{2}x + 1.5$ & \LoesungsRaum{1.5} & \LoesungsRaum{$\frac{1.5}{\frac{1}{2}}=3$} & \LoesungsRaum{$-\frac{1}{2}$} \\\hline

$y=-\frac{2}{5}x - 3.5$ & \LoesungsRaum{-3.5} & \LoesungsRaum{-8.75} & \LoesungsRaum{$-\frac{2}{5}$} \\\hline

$y=1.8x$ & \LoesungsRaum{0} & \LoesungsRaum{0} & \LoesungsRaum{1.8} \\\hline

$y=-3x$ & \LoesungsRaum{0} & \LoesungsRaum{0} & \LoesungsRaum{-3} \\\hline

$y=3.7$ & \LoesungsRaum{3.7} & \LoesungsRaum{keine Nullstelle} & \LoesungsRaum{0} \\\hline

$y=0$ & \LoesungsRaum{0} & \LoesungsRaum{alle $x$ sind Nullstelle} & \LoesungsRaum{0} \\\hline

$y=-x$ & \LoesungsRaum{0} & \LoesungsRaum{0} & \LoesungsRaum{-1} \\\hline

$5y-3 = 2x+7$ & \LoesungsRaum{2} & \LoesungsRaum{$-5$} & \LoesungsRaum{$\frac25$} \\\hline

\end{bbwFillInTabular}
\newpage


\noTRAINER{

Hilfsblatt

\bbwGraph{-7}{7}{-4}{5}{}

\bbwGraph{-7}{7}{-4}{5}{}
\newpage
}%% END noTRAINER

\subsection{Funktionsgleichung finden}

Bestimmen Sie die Funktionsgleichung $y=ax+b$ (und die anderen
fehlenden Werte):

\begin{bbwFillInTabular}{c|c|c|l}
 $y$-Achsenabschnitt $b$ & Nullstelle $\frac{-b}{a}$& Steigung $a$& Funktionsgleichung $y=...$\\
\hline

-2 & \LoesungsRaum{$\frac{2}{3}$} & 3 & \LoesungsRaum{$y=3x-2$}\\
\hline

1.5 & \LoesungsRaum{$\frac{15}{17}$} & -1.7 & \LoesungsRaum{$y=-1.7x + 1.5$}\\
\hline

0 & \LoesungsRaum{$0$} & 1.4 & \LoesungsRaum{$y=1.4x$}\\
\hline

\LoesungsRaum{4} & -4 & 1 & \LoesungsRaum{$y=x+4$}\\
\hline

\LoesungsRaum{-6} & -3 & -2 & \LoesungsRaum{$y=-2x-6$}\\
\hline

2 & $\frac{1}{5}$ & \LoesungsRaum{-10} & \LoesungsRaum{$y=-10x + 2$}\\
\hline

5 & \LoesungsRaum{keine Nullstelle}& 0 & \LoesungsRaum{$y=5$}\\
\hline

\LoesungsRaum{0} & 0 & $-\frac{3}{2}$ & \LoesungsRaum{$y=-\frac{3}{2}x$}\\
\hline

0 & -1.5 & \LoesungsRaum{0} & \LoesungsRaum{$y=0$}\\
\hline


0 & \LoesungsRaum{Alle $x\in\mathbb{R}$ sind Nullstelle} & 0 & \LoesungsRaum{$y=0$}\\  %
\hline

0 & 0 & \LoesungsRaum{$\mathbb{R}$} & \LoesungsRaum{$y=0$ oder $y=7x$
oder $y=-22.6x$ }\\  %
\hline

\end{bbwFillInTabular}
\newpage


\noTRAINER{
Hilfsblatt

\bbwGraph{-7}{7}{-4}{5}{}

\bbwGraph{-7}{7}{-4}{5}{}
\newpage
}%% END noTRAINER
\newpage
\subsection{Schnittpunkt mit Achsen I}
Geben Sie von der Geraden $x\mapsto 3x-\frac45$ die Schnittpunkte mit
den Achsen an:

\vspace{15mm}
Schnittpunkt mit der $x$-Achse: \LoesungsRaumLang{$\left(\frac{4}{15}\middle|0\right)$}

\vspace{5mm}

Schnittpunkt mit der $y$-Achse: \LoesungsRaumLang{$\left(0\middle|\frac{-4}5\right)$}

\TNTeop{%
$x$-Achse: $y=0$ setzen:

$$0=3x-\frac45$$
$$0 = 15x - 4$$
$$4 = 15x$$

$$x = \frac{4}{15}$$

$y$-Achse: Achsenabschnitt ist bereits in der Funktionsgleichung gegeben.
}% end TNT EOP
\newpage
%%%%%%%%%%%%%%%%%%%%%%%%%%%%%%%%%%%%%%%%%%%%%%%%%%%%%%%%%%%%%%%%%%%%%%%%%
\subsection{Schnittpunkt mit Achsen II}
Geben Sie die Schnittpunkte mit den Koordinatenachsen der folgenden
linearen Funktion an:

$$\frac{y+45}9 = 8x$$

\vspace{15mm}
Schnittpunkt mit der $x$-Achse: \LoesungsRaumLang{$\left(\frac58\middle|0\right)$}

\vspace{5mm}

Schnittpunkt mit der $y$-Achse: \LoesungsRaumLang{$\left(0\middle|-45\right)$}

\TNTeop{%
$x$-Achse: $y=0$ setzen:
 
$$\frac{y+45}9 = 8x$$
$$\frac{45}9 = 8x$$
$$5 = 8x$$
$$\frac58 = x$$

$y$-Achse: $x=0$ setzen:
$$\frac{y+45}9 = 8x$$
$$\frac{y+45}9 = 0$$
$$y+45 = 0$$
$$y = -45$$
}%% END TNT eop

\newpage
%%%%%%%%%%%%%%%%%%%%%%%%%%%%%%%%%%%%%%%%%%%%%%%%%%%%%%%%%%%%%%%%%%%%%%%%%

\subsection{Schnittpunkt mit Achsen III}
Bestimmen Sie die Schnittpunkte der Geraden
$$y-2x=3\left(x+\frac23\right)$$
mit den Koordinatenachsen!


Schnittpunkt mit der $x$-Achse: \LoesungsRaum{$(-\frac25|0)$}

\vspace{5mm}
Schnittpunkt mit der $y$-Achse: \LoesungsRaum{$(0|2)$}

\TNTeop{
Tipp: Erst in Form bringen: In Grundform:

$$y-2x=3x+2$$
$$y=5x+2$$
Hier erhalten wir sofort den Schnittpunkt mit der $y$-Achse
(Achsenabschnitt = 2)

Nun nach $y=0$ setzen und nach $x$ auflösen:

$$0=5x+2$$
$$-2=5x$$
}
%%%%%%%%%%%%%%%%%%%%%%%%%%%%%%%%%%%%%%%%%%%%%%%%%%%%%%%%%%%%%%%%

%%%%%%%%%%%%%%%%%%%%%%%%%%%%%%%%%%%%%%%%%%%%%%%%%%%%%%%%%%%%%%%%%%
\subsection{Funktionsgleichung finden}
Gegeben ist die folgende Wertetabelle:

\begin{tabular}{l|c|c|c|c|c|c|}
$x$    & $-1$ & $0$   & $1$ & $2$   & $3$ & $4$ \\\hline
$f(x)$ & $2$  & $1.5$ & $1$ & $0.5$ & $0$ & $-0.5$ \\
\end{tabular}

a) Skizzieren Sie den Graphen ins folgende Koordinatensystem:

\bbwGraph{-2}{5}{-2}{3}{%
\TRAINER{%
\bbwLine{-1,2}{4,-1}{red}
}%% end TRAINER
}%% end BBW Graph

b) Geben Sie die Funktionsgleichung an:

$$f(x) = \LoesungsRaum{-\frac12x+1.5}$$
\TNTeop{}

%%%%%%%%%%%%%%%%%%%%%%%%%%%%%%%%%%%%%%%%%%%%%%%%%%%%%%%%%%%%%%%%%%%%%%
\subsection{Kerze}
Eine Kerze von anfänglich 30 cm Höhe brennt langsam ab.

Nach 25 Minuten sind 4.3 cm abgebrannt.

a) Wie lautet die Funktionsgleichung $y=f(x)$ mit $x$=Zeit in Minuten
nach dem Entzünden und $y$ = Höhe der Kerze in cm.

b) Wann wird die Kerze abgebrannt sein?

\TNTeop{
a) $f(x) = 30 - 4.3\cdot{}\frac{x}{25} = \frac{- 4.3}{25}\cdot{}x + 30
= -0.172x + 30$

b) $x_0 = \frac{-b}a = -b : a = 30 : \frac{4.3}{25} = 30 \cdot{} 25 : 4.3 \approx 174.4 \text{
min} = 2 \text{ h } 54.4 \text{ min}$
}
%%%%%%%%%%%%%%%%%%%%%%%%%%%%%%%%%%%%%%%%%%%%%%%%%%%%%%%%%%%%%%%%%%%%%
\subsection{Alkohol (Promille)}
Alkohol baut sich im Körper im Durchschnitt mit
$0.15$\textperthousand{} pro Stunde ab (wir gehen von einem annähernd
linearen Abbau aus).

Andrea hat um 22:00 Uhr nach einer unachtsamen Feier einen
Alkoholgehalt von $1.8$\textperthousand{} im Blut.

a) Wenn Andrea nun (ab 22:00 Uhr) keinen Alkohol mehr zu sich nimmt:
Wann ist der Alkoholgehalt in Andreas Blut auf $1.2$\textperthousand{}
gesunken?

\TNT{4}{Differenz von 1.2 zu 1.8 sind 0.6
[\textperthousand{}]. Somit teilen wir die 0.6 durch 0.15 und erhalten
4 [Stunden].

Nach vier Stunden ist der Alkoholgehalt auf $1.2$\textperthousand{}
gesunken.}%% end TNT

b) Geben Sie die (lineare) Funktionsgleichung $y=f(x)$ an mit
$$y = \text{ Alkoholgehalt in \textperthousand{} im Blut}$$
$$x = \text{ Anzahl Stunden nach der Feier (also nach 22:00 Uhr)}$$

Beispiel: $x=3$ bedeutet also 1 Uhr morgens.

$$y = f(x) = \LoesungsRaumLang{-0.15\cdot{}x + 1.8 [\text{\textperthousand{}}]}$$
\TNT{4}{}

c) Geben Sie den Alkoholgehalt nach $6.45$ Stunden an:

$$f(6.45) = \LoesungsRaumLang{1.8-0.15\cdot{}6.45 = 0.8325
[\text{\textperthousand}]}$$
\TNTeop{}
%%%%%%%%%%%%%%%%%%%%%%%%%%%%%%%%%%%%%%%%%%%%%%%%%%%%%%%%%%%%
d) Geben Sie den Alkoholgehalt um 5:15 am nächsten Morgen an:

$$y = f(\LoesungsRaum{7.25}) = \LoesungsRaum{0.7125}\textperthousand$$

\TNT{4}{}

e) Um welche Uhrzeit wir der Wert auf $0.5$\textperthousand{} gesunken
sein?

\vspace{5mm}
Dies wird um \LoesungsRaum{6:40} Uhr eintreten.
\TNTeop{
$$0.5 = 1.8-0.15x$$
$$0.15x = 1.8 - 0.5$$

ergo
$$x = \frac{1.8-0.5}{0.15} = 8.666... \text{h} = 8\text{h} 40\text{ min}$$
}%% end TNTeop

\section{Schnittpunkt(e)}
\subsection{Wo schneiden sich $f$ und $g$?}
Finden Sie den Schnittpunkt $S$ von $f$ und $g$:

$$f: y= 2x-3$$
$$g: y=-x+2$$


\TRAINER{$$S = \left(\frac53\middle|\frac13\right)$$}
\noTRAINER{
\vspace{8mm}
$$S = $$}%% end noTRAINER

\TNTeop{
Gleichsetzen:
$$2x-3=-x+2$$
$$3x=5$$
$$x=\frac53$$

$x$ in eine der beiden Gleichungen einsetzen:

$$y=2\cdot{}\frac52 -3 = \frac{10}3 - 3 = \frac13$$
}
\newpage

\subsection{Schnittpunkt}
Finden Sie den Schnittpunkt $S$ von $f$ und $g$:

$$f: y=-\frac23x + 6$$

$$g: 2x-4y = 6$$

\vspace{5mm}
$$S=\LoesungsRaumLang{\left(\frac{45}7\middle|\frac{12}7\right)}$$
\TNTeop{
$g:$
$$-4y=6-2x$$
$$y=\frac{-2x+6}{-4} = \frac{-2x}{-4} + \frac{6}{-4} = \frac12x
-\frac32$$
$f$ und $g$ gleichsetzen:
$$-\frac23x+6=\frac12x-\frac32$$
$\cdot{}6:$
$$-4x+36=3x-9$$
$$45=7x$$
$$S_x = \frac{45}7$$
$$S_y = -\frac23S_x+6 = -\frac23\cdot{}\frac{45}7 + 6 =
... =\frac{12}7$$

$$S = (S_x | S_y) = \left(\frac{45}7\middle|\frac{12}7\right)$$

}%% end TNT eop
%%%%%%%%%%%%%%%%%%%%%%%%%%%%%%%%%%%%%%%%%%%%%%%%%

\subsection{Schnittpunkt ?}
Wo schneidet sich

$$f(x) = 3.2x - 8$$
mit

$$g: x\mapsto \frac{16x+3.68}5 \text{ ?}$$

\vspace{5mm}
$$\LoesungsMenge{}_S=\LoesungsRaumLang{\{\}}$$
(Warum versagt hier der numerische \textit{Solver} des Taschenrechners?
Tipp: Resultate immer auch noch nachprüfen!)

\TNTeop{
Beide Funktionen haben die selbe Steigung (3.2) aber einen verschiedenen
$y$-Achsenabschnitt.
Die beiden sind parallel und haben somit keinen gemeinsamen
Schnittpunkt in $\mathbb{R}\times\mathbb{R}$

Der numerische Taschenrechner versagt, wenn von einem Startwert >0
ausgegangen wird. Startet der Rechner \zB{} mit $x=1$, so kann das
Resultat gut und gerne $x=5.96 \cdot{}10^{16}$ werden. Für den Rechner
ist dieses Resultat so OK, denn in den Größenordnungen kann der
Rechner die hinterste Ziffer nicht mehr genau annähern und bekommt für
beide Seiten der Gleichung den selben $y$-Wert.
}%% end TNT eop

\subsection{Schnittpunkt(e)?}
Wo schneiden sich die Geraden $f$ und $g$?

Gegeben:
$$f(x) = 3x - 8$$
$$g: 2x=\frac23y+\frac{16}3$$

\vspace{5mm}
$$\LoesungsMenge{}_S=\LoesungsRaumLang{\{ (S_x|S_y) | S_y = 3x-8  \} \text{ oder } \{ (S_x|S_y) | S_x = \frac{y+8}3  \}}$$
\TNTeop{

Die beiden Geraden haben äquivalente Funktionsgleichungen und sind
somit \textbf{zusammenfallend}.
}%% end TNT
%%%%%%%%%%%%%%%%%%%%%%%%%%%%%%%%%%%%%%%%%%%%%%%%%%%%%%%%%%%%%%%%%%
\subsection{Tropfsteinhöhlen}

Stalaktiten bilden sich durch herabtropfendes Wasser in Tropfsteinhöhlen.
Wir gehen hier davon aus, dass es $3\,000$ Jahre benötigt, bis ein
Stalaktit $50$ cm von der Decke heruntergewachsen ist. Wir gehen von
linearem Wachstum aus.

Stalagmiten sind die Gegenstücke, welche vom Boden
«heraufwachsen». Gehen wir hier von der Annahme aus, dass unser
Stalagmit dreimal länger braucht, um die selbe Höhe wie der Stalaktit zu erreichen.

In unserer fiktiven Höhle ist der Abstand vom Boden zur Decke $2.3$ m
(= $230$ cm). und der Stalagmit (von unten) wird auf $34$ cm gemessen.

a) Zeichnen Sie das Ende des Stalagmiten als
lineare Funktion in ein kartesisches Koordinatensystem. Verwenden Sie
eine $y$-Achse von $0 - 200$ cm und eine $x$-Achse von $0-12\,000$
Jahren. Nehmen Sie selbständig eine sinnvolle Skalierung vor.

\noTRAINER{\mmPapier{14}}

b) Zeichnen Sie das Ende des Stalaktiten (nicht dessen Höhe) in
dasselbe Koordinatensystem. Die Gerade beginnt für $x=0$ Jahre
natürlich bei $200$ cm.
\newpage

c) Geben Sie sowohl das Ende des Stalagmiten ($f$), wie auch das des
Stalaktiten ($g$) als Funktionsterm an mit $x$ = Alter in Jahren und
$y$ = Spitze über Boden.

\TNT{6}{%
 $$f(x) = 50\cdot{}\frac{x}{9000} = \frac{x}{180} [\text{ cm}]$$
 $$g(x) = 200 - \frac{x}{60}  [\text{ cm}]$$
}%% end TNT


d) Wo schneiden sich die Geraden $f$ und $g$? Mit anderen Worten: Wann
(nach deren Entstehung) wachsen die beiden zusammen?
\TNTeop{%
$$\frac{x}{180} = 200 - \frac{x}{60}$$
$$x = 36\,000 - 3x$$
$$4x = 36\,000 $$
$$x = 9\,000 \text{[Jahre]}$$
}%% end TNT
%%%%%%%%%%%%%%%%%%%%%%%%%%%%%%%%%%%%%%%%%%%%%%%%%%%%%%%%%%%%%%%%%%%%%%%%%


%%%%%%%%%%%%%%%%%%%%%%%%%%%%%%%%%%%%%%%%%%%%%%%%%%%%%%%%%%%%%%%%%%
\section{Parallele}
\subsection{Zeichnen}
a) Zeichnen Sie $y=\frac23x+b$ für die Parameter

1. $b=1.5$

2. $b=-1$

3. $b= -4$

\bbwGraph{-4}{4}{-5}{3}{
\TRAINER{\bbwFunc{\x*2/3 + 1.5}{-3:3}}
\TRAINER{\bbwFunc{\x*2/3 - 1  }{-3:3}}
\TRAINER{\bbwFunc{\x*2/3 - 4  }{-3:3}}
}%% end bbwGraph

b) Was fällt auf?
\TNT{4}{
Die drei Geraden (Graphen der Funktionen) sind parallel. Sie alle
haben die selbe Steigung.
}
\newpage
\subsection{Parallele Graphen}
Je zwei der folgenden sechs linearen Funktionen besitzen zu einander parallele
Graphen.

Ordnen Sie zu:

$$x\mapsto -\frac13 x + 2.5$$

$$42 = 12y + 1.5x$$

$$f(x) = 3x-6$$

$$x=-48-8y$$

$$2y-6x = 16$$

$$y= - \frac13 x + \frac74$$
\TNTeop{

$$x\mapsto -\frac13 x + 2.5   \Longleftrightarrow  y= - \frac13 x + \frac74$$

$$42 = 12y + 1.5x \Longleftrightarrow  x=-48-8y$$

$$f(x) = 3x-6 \Longleftrightarrow  2y-6x = 16$$

}%% end TNTeop
%%%%%%%%%%%%%%%%%%%%%%%%%%%%%%%%%%%%%%%%%%%%%%%%%%%%%%%%%%%%%%%%%%%%%%%%%%%%%
\section{Punkte-Aufgaben}

\subsection{Ordinatenabschnitt finden}
Gegeben ist die Steigung $a=3.7$ der Funktion
$$y=3.7x + b.$$

a) Berechnen Sie $b$ so, dass der Graph von $f$ durch den Punkt $(0|6)$
verläuft:

$$b = \LoesungsRaum{6}$$

\TNT{2.4}{$b$ ist der $y$-Achsenabschnitt. Hier gibt es nichts zum Rechnen.}


b) Berechnen Sie $b$ so, dass der Graph von $f$ durch den Punkt
$(2|2)$ verläuft:

$$b= \LoesungsRaum{-5.4}$$
\TNTeop{$x=y=2$ in die Funktionsgleichung einsetzen:
$$2 = 3.7\cdot{}2 + b$$
nun nach $b$ auflösen
}%% end TNTeop


\subsection{Steigung finden}

Von einer linearen Funktion ist der Ordinatenabschnitt (=
$y$-Achsenabschnitt) $b$ gegeben. Ebenso ist bekannt, dass die
Funktion durch den Punkt $P$ verläuft.

Geben Sie jeweils die Funktionsgleichung an:

a) $b=3$ und $P=(3|4)$

\noTRAINER{\mmPapier{4}}\TRAINER{$y=\frac13 x + 3$}

b) $b=2.5$ und $P=(6|2.5)$

\noTRAINER{\mmPapier{4}}\TRAINER{$y=2.5$}

c) $b=-1.4$ und $P=(\frac74|0)$

\TNTeop{$y=0.8x - 1.4$}
%%%%%%%%%%%%%%%%%%%%%%%%%%%%%%%%%%%%%%%%%%%%%%%%%%%%%
\subsection{Ablesen? Rechnen!}

\bbwGraph{-5}{8}{-1}{6}{%

\bbwDotGerade{-5,5.1818}{8,2.8182}{-4, 5}{7,3}{ForestGreen}{g}{5,4}

}%% end BBW Graph

a) Wie lautet die Funktionsgleichung der abgebildeten Funktion $g$?

(Die Gerade verläuft durch die beiden angegebenen Gitterpunkte.)

\TNT{4}{$g:   x\mapsto \frac{-2}{11}x+\frac{47}{11}$}


b) Gegeben ist der Punkt $P=(-3|2)$. Wie lautet die Funktionsgleichung der Funktion $h$, für die gilt
$h||g$ und $P\in h$?

Dabei steht das Symbol $||$ für «ist parallel zu».

\TNTeop{
Setze $P=(-3|2)$ in $f(x) = \frac{-2}{11}x+\frac{47}{11}$ ein:

$$2 = \frac{-2}{11}\cdot{}(-3) + b$$
Mal 11:
$$22 = 6 + 11b$$
-6:
$$16=11b$$
Durch 11:
$$b = \frac{16}{11}$$
$$\Longrightarrow  y = \frac{-2}{11} x + \frac{16}{11}$$
}%% end TNTeop

%%%%%%%%%%%%%%%%%%%%%%%%%%%%%%%%%%%%%%%%%%%%%%%%%%%%%5
\subsection{Parallele (Funktionsgleichung finden)}

Gegeben ist die Gerade $f: y=\frac{-2}3x+9.6$.

Legen Sie eine parallele Gerade zu $f$ so, dass diese Parallele durch
den Punkt $(9|7)$ verläuft und geben Sie die Funktionsgleichung der
gefundenen Geraden an.

\vspace{15mm}
\LoesungsRaumLang{$y=\frac{-2}3x + 13$}
\TNTeop{
 Die neue Gerade hat die selbe Steigung $\frac{-2}3$; somit lautet die
 Gerade

$$y = \frac{-2}3x + b$$

Nun setzen wir die Koordinaten des Punktes in die Funktionsgleichung
ein:

$$7 = \frac{-2}3\cdot{}9 + b$$
und lösen nach $b$ auf:
$$7 = -6 + b$$
$$13 = b$$
}



%%%%%%%%%%%%%%%%%%%%%%%%%%%%%%%%%%%%%%%%%%%%%%%%%%%%%%%%%%%%%%55
\subsection{Eine «Zwei Punkte Aufgabe»}
Die Funktion $f$ schneide die Achsen in $\left(0\middle|\frac87\right)$ und
$(-2|0)$. Geben Sie die Funktionsgleichung von $f$ an:

\vspace{20mm}

$$f: \LoesungsRaum{\frac47x + \frac87}$$

\TNTeop{
$y$-Achsenabschnitt (= Ordinatenabschnitt):

$$y=ax+\frac87$$
Nun Punkt $(-2|0)$ (= Nullstelle) einsetzen:

$$0 = a\cdot{}(-2) + \frac87$$
$$0 = a\cdot{}(-14) + 8$$
$$-8 = a\cdot{}(-14)$$
$$8 = a\cdot{}(14)$$
$$4 = a\cdot{}(7)$$
$$\frac47 = a$$
}%% end TNT eop

%%%%%%%%%%%%%%%%%%%%%%%%%%%%%%%%%%%%%%%%%%%%%%%%%%%%%%%%%%%%%%%%%%%%%%%%%%%%%%%
\subsection{Algebraische Prüfung}
Prüfen Sie für jeden der Punkte $A$, $B$ und $C$ algebraisch, ob sie auf dem
Graphen der Funktion $f$ liegen:

$$f = \frac34x-\frac13$$

$$A = \left(0\middle|-\frac13\right), \hspace{5mm} B =
(4|2), \hspace{5mm} C = \left(\frac49\middle|0\right)$$
\TNTeop{$A\in f, B\not \in f, C \in f$}


%%%%%%%%%%%%%%%%%%%%%%%%%%%%%%%%%%%%%%%%%%%%%%%%%%%%%%%%%%%%%%%%%%%%%%%%%%%%%%%
\subsection{Fehlende Abszisse}

Bestimmen Sie die fehlende Koordinate $P_x$ sodass $P\in AB$ (also
sodass $P$ auf der Geraden $AB$ liegt):

$$A=(7|5), B=(-2|-1), P=(P_x|4)$$

\TNTeop{1. Funktionsgleichung bestimmen:
$$y = \frac23x+\frac13$$
2. $y$ einsetzen:

$$4 = \frac23x+\frac13$$

3. nach $x$ auflösen:

$$P_x = \frac{11}2$$
}%% end TNTeop

%%%%%%%%%%%%%%%%%%%%%%%%%%%%%%%%%%%%%%%%%%%%%%%%%%%%%%%%%%%%%%%%%%%%%%%%%%%%%%%
\subsection{Mathematikprüfung}
Eine Mathematikprüfung wurde mit einer linearen Notenskala bewertet.
Das erforderliche Punktemaximum ergibt die Note 6 während Null Punkte
die Note eins ergeben.

Kim hat mit 28 Punkten die Note 5 erzielt.

a) Wie lautet die lineare Funktionsgleichung $f$, welche aus einer
Punktzahl $x$ die Note $y$ errechnet?

\TNT{4}{$y = \frac17 x + 1$}

b) Was war das erforderliche Punktemaximum für die Note sechs?

\TNTeop{ $y$-Koordinate einsetzen:
$$6 = \frac17 x + 1$$
$$\Longrightarrow x = 35$$
Das erforderliche Punktemaximum war bei 35 Punkten.
}




%%%%%%%%%%%%%%%%%%%%%%%%%%%%%%%%%%%%%%%%%%%%%%%%%%%%%%%%%%%%%%%%%%%%%%%%%%%%%%%
\subsection{Nullstelle und Ordinatenabschnitt}
Die Funktion $f$ habe die Nullstelle $\frac83$ und den
Ordinatenabschnitt 2. Geben Sie die Funktionsgleichung in der
Grundform an:

\vspace{20mm}
$$f: \LoesungsRaumLang{y = \frac{-3}4 \cdot{} x + 2}$$

\TNTeop{
Ordinatenabschnitt führt direkt zu Parameter $b$ in $y=ax+b$:
$$y = ax + 2$$
Nun die Nullstelle $\left(\frac83\middle|0\right)$ einsetzen:
$$0 = a\cdot{}\frac83 + 2$$
$$-2 = a\cdot{}\frac83 $$
$$-6 = a\cdot{}8 $$
$$-\frac68 = a$$
}%% end TNT eop
%%%%%%%%%%%%%%%%%%%%%%%%%%%%%%%%%%%%%%%%%%%%%%%%%%%%%%%%%%%%%%%%%%%%%%%%%%%%%%%%%%
\subsection{Funktionsgleichung durch zwei Punkte}
Geben Sie die Funktionsgleichung $y=ax+b$ der Geraden $g$ an, welche
durch die Punkte $A$ und $B$ verläuft:

a) $A = (3|9)$, $B=(4|10)$

\vspace{15mm}

$$g: y= \LoesungsRaum{x + 6}$$
\vspace{5mm}

b) $A = \left(-\frac13\middle|\frac45\right)$, $B=\left(\frac29\middle|\frac{-7}3\right)$

\vspace{15mm}

%%\TRAINER{$$g: L_{(a;b)} \LoesungsRaum{\left( \frac{-141}{25} ; \frac{27}{25}\right)}$$}
$$g: y= \LoesungsRaum{ \frac{-141}{25}} \cdot{} x -  \LoesungsRaum{\frac{27}{25}}$$
\vspace{5mm}
\TNTeop{%
a) und b) mit TR (lin-reg)
}%% end TNT eop
\newpage
(challenge:)

\vspace{3mm}

c) $A = \left( t \middle| \sqrt{3} \right)$,
$B=\left( \sqrt{2} \middle| s \right)$

(Resultat exakt angeben)

\vspace{15mm}

$$g: y= \LoesungsRaum{\left( \frac{s-\sqrt{3}}{\sqrt{2}-t} \middle| \frac{st-\sqrt{6}}{t-\sqrt{2}}\right)}$$
\vspace{5mm}


\TNTeop{%

c)
\gleichungZZ{\sqrt{3}}{at+b}{s}{a\sqrt{2}+b}
II - I: $s-\sqrt{3} = a(\sqrt{2}-t) \Longrightarrow a
= \frac{s-\sqrt{3}}{\sqrt{2}-t}$.

Dies nun in eine der Gleichungen einsetzen (z. B. in A):

$$\sqrt{3} = at+b$$
$$b = \sqrt{3} - at$$
$$b = \sqrt{3} - \frac{s-\sqrt{3}}{\sqrt{2}-t} \cdot{}t$$
Linken Bruch erweitern mit $\sqrt{2}-1$:
$$b = \frac{\sqrt{3}(\sqrt{2}-t)}{\sqrt{2}-t}
- \frac{s-\sqrt{3}}{\sqrt{2}-t} \cdot{}t$$
alles auf einen Bruchstrich:
$$\frac{\sqrt{3}\sqrt{2} - \sqrt{3}t+\sqrt{3}t-st}{\sqrt{2}-t}$$
$$\frac{\sqrt{6}-st}{\sqrt{2}-t}$$

}% END TNTeop
%%%%%%%%%%%%%%%%%%%%%%%%%%%%%%%%%%%%%%%%%%%%%%%%%%%%%%%%%%%%%%%%%%%%
\subsection{Vier Punkte / Zwei Geraden}
Die Gerade $f$ verläuft durch die Punkte $(7|2)$ und $(-2|8)$ während
die Gerade $g$ durch $(-2|6)$ und $\left(2\middle|\frac14\right)$
verläuft.

Geben Sie den Schnittpunkt von $f$ und $g$ an:

\vspace{15mm}

$$\LoesungsRaumLang{\left( \frac{-170}{37} \middle| \frac{360}{37} \right)}$$

\TNTeop{%%
1. $f$:
\gleichungZZ{2}{7a + b}{8}{-2a+b}
$$\Longrightarrow  a = -\frac23 \text{ und } b = \frac{20}3$$

2. $g$:
\gleichungZZ{6}{-2a + b}{\frac14}{2a+b}
$$\Longrightarrow  a = -\frac{23}{16} \text{ und } b = \frac{25}8$$

3. $f \cap g$:

$$-\frac23 x + \frac{20}3 = \frac{23}{16} x + \frac{25}8$$
solve:
$$x = \frac{-170}{37} = -4.\overline{594}$$
und
$$y = \frac{360}{37} = 9.\overline{729}$$

}%% end TNT eop
%%%%%%%%%%%%%%%%%%%%%%%%%%%%%%%%%%%%%%%%%%%%%%%%%%%%%%%%%%%%%%%%%%%%%
\subsection{Taxi}
Taxiunternehmen \textbf{A} kostet pro gefahrenen Kilometer CHF
0.60. Daneben verlangt das Unternehmen eine Grundgebühr von CHF 5.-.

Ein neues Taxiunternehmen (nennen wir es \textbf{B}) geht davon aus, dass viele Fahrten über 4 km
sind. Das Unternehmen \textbf{B} will nun erreichen, dass für exakt 4
km Fahrten beide Unternehmen gleich teuer sind um wenigstens für die
kurzen Fahrten Unternehmen b nicht zu konkurrenzieren. Jedoch will das
Unternehmen \textbf{B} keine Grundgebühr sondern einfach einen
km-Preis erheben.

Was sind die zu erwartenden Kosten pro km bei Unternehmen \textbf{B}?

\TNTeop{CHF 1.85  $= 7.40 / 4 = (5+4\cdot{}0.6) / 4$}
%%%%%%%%%%%%%%%%%%%%%%%%%%%%%%%%%%%%%%%%%%%%%%%%%%%%%%%%%%%%%%%%%%%%%%
\subsection{Taxi zwei}
Ein Taxiunternehmen hat den Markt analysiert und will nun für 3
km-Fahrten  eine Gebühr von CHF 8.- und für 9 km Fahrten eine Gebühr
von CHF 20.- erheben.

Um die Berechnung der Fahrtkosten für die Kunden verständlich zu
machen, will das Unternehmen eine Grundgebühr und eine km-Gebühr
erheben.

Wie hoch wird diese Grundgebühr und wie hoch die km-Gebühr ausfallen?

\TNTeop{ Die Grundgebühr wird CHF 2.- sein und die km-Gebühr ebenfalls
CHF 2.-

Lösung am Einfachsten mit Lin-Reg-Funktion des Taschenrechners. Ohne
TR: Gleichungssystem: 2-Punkte-Aufgabe.
}

\end{document}
