
\subsection{Zerfall}
\textit{Nicht immer ist die Wachstumsrate größer als 1}

\bbwActAufgabenNr{} \textbf{Radioaktiver Stoff}

Ein radioaktiver Stoff zerfällt jedes Jahr um elf Prozent. Anfänglich ist ein kg des Stoffes vorhanden.

\begin{bbwAufgabenBlock}
\item Wie viel vom Stoff ist nach vier Jahren noch übrig?
\TRAINER{$0.89^4 \textrm{kg} \approx 62.74\%$}
\item Wie viel vom Stoff ist nach $n$ Jahren noch übrig?
\TRAINER{$0,89^n \textrm{kg}$}
\item Nach wie vielen Jahren wird noch 50\% des Stoffes übrig bleiben? (Diese Zeitspanne nennt man die Halbwertszeit $T_2$.)
\TRAINER{$T_2 = n = \log_{0.89}(0.5)\approx 5.948$ Jahre.}
\end{bbwAufgabenBlock}
\platzFuerBerechnungenBisEndeSeite{}



%%%%%%%%%%%%%%%%%%%%%%%%%%%%%%%%%%%%%%%%%%%%%%%%%%



\bbwActAufgabenNr{} \textbf{Wert des Autos}

Um wie viel Prozent hat der Wert eines Autos jährlich verloren, wenn
der Wert nach 4 Jahren um 60 \% gesunken ist?
\nextBbwAufgabenNummer%% Dies geschieht im bbwAufgabenBlock implizit
\platzFuerBerechnungenBisEndeSeite{}

\TRAINER{
$$100 - 60 = 0.4$$
somit
$$0.4 = a^4$$
$$a = \sqrt[4]{0.4} \approx 79.527\%$$
Dies entspricht einer jährlichen Abnahme von.
$$ 1 - 79.527\%  \approx 0.2047 $$
Also eine Prozentuale Abnahme von 20.47\%.
}%% END TRAINER


%%%%%%%%%%%%%%%%%%%%%%%%%%%%%%%%%%%%%%%%%%%%%%%%%%
\bbwActAufgabenNr{} \textbf{Wohnwagen}
Ein Wohnwagen hat innerhalb von fünf Jahren 75\% an Wert verloren. 

\begin{bbwAufgabenBlock}

\item Geben Sie eine Formel für den Wert des Wohnwagens in
Abhängigkeit der Zeit an.

\item Wie viel Wert (in \%) verlor der Wohnwagen jährlich?
      \TRAINER{... Lösung noch ausstehend ...}

\item Nach wie vielen Jahren (ab Beginn) wird der Wohnwagen noch 10\%
seines ursprünglichen Wertes haben?

\TRAINER{
  Der Wert nach 5 Jahren beträgt noch 25\%.
  Wenn die Formel $y=a^t$ verwendet wird mit

  $y$=Wert nach $t$ Jahren.
  
  $t$=Anzahl Jahre

  $a$=Abnahmefaktor pro Jahr

  so können wir alles bekannte einsetzen:

  $$y=a^t$$
  $$25\% = a^5$$
  $$0.25=a^5$$
  $$\sqrt[5]{0.25} = a$$

  Daraus ergibt sich der Jährliche Faktor $a \approx 0.7578582$.
  Somit verbleiben Jedes Jahr ca. 75.79\% des Wertes vorhanden, was
  einer Abnahme von 1-0.7578... , also 24.21\% entspricht.
}%% END TRAINER

\end{bbwAufgabenBlock}
\platzFuerBerechnungenBisEndeSeite{}

%%%%%%%%%%%%%%%%%%%%%%%%%%%%%%%%%%%%%%%%%%%%%%%%%%
\newpage
