%%%%%%%%%%%%%%%%%%%%%%%%%%%%%%%%%%%%%%%%%%%%%%%%%5

\subsection{Begrenztes Wachstum}


\bbwActAufgabenNr{} \textbf{Cola}

Eine Cola wird bei $5^\circ \textrm{ C}$ aus der Kühlbox genommen. Die
Umgebungstemperatur ist $32^\circ \textrm{ C}$. Nach 3 Minuten messen
wir bereits eine Temperatur von $9^\circ \textrm{ C}$ 

Wir gehen davon aus, dass die Temperaturdifferenz der Cola zur
Umgebungstemperatur exponentiell abnimmt.

\begin{bbwAufgabenBlock}
\item Machen Sie eine Skizze im Koordiantensystem, welche die
Abhängigkeit von der Cola-Temperatur von der Zeit ($t$) aufzeichnet.

\TRAINER{Graph}

\item Geben Sie eine Funktionsgleichung $f(t)$ an, welche die
Temperatur in Minuten ($t$) nach dem herausnehmen der Cola aus der Kühlbox
angibt.

\TRAINER{$$f(t) = c - b\cdot{} a^{\frac{t}{\tau}}$$

$$f(t) = 32 - 27 \cdot \left( \frac{23}{27}\right)^{\frac{t}{3}}$$
}

\item Wie «warm» ist die Cola nach 10 Minuten?

\TRAINER{
$$f(10) = 32 -
27 \cdot \left( \frac{23}{27}\right)^{\frac{10}{3}} \approx
16.18^\circ \textrm{ C}$$
}%% end TRAINER

\item Nach wie vielen Minuten ist die Cola $18^\circ \textrm{ C}$
«warm»?

\TRAINER{
$$f(t) = 18 = 32 - 27\cdot{} \left(\frac{23}{27}\right)^\frac{t}{3}$$
$$-14 =  - 27\cdot{} \left(\frac{23}{27}\right)^\frac{t}{3}$$
$$\frac{14}{27} =  - \left(\frac{23}{27}\right)^\frac{t}{3}$$
$$t =
3\cdot{} \log_{\left(\frac{23}{27}\right)}\left(\frac{14}{27}\right) \approx
12.29 \textrm{ min.}$$
}%% END TRAINER

\end{bbwAufgabenBlock}


\platzFuerBerechnungenBisEndeSeite{}


%%%%%%%%%%%%%%%%%%%%%%%%%%%%%%%%%%%%%%%%%%%%%%%%%%%%%%%%%%%%%%%%%%%%%%%%%%%%%%%%%%%%%%%%%%%%%%%%%%


\bbwActAufgabenNr{} \textbf{«Silly Blubb»}

Das neue Waschmittel «Silly Blubb» will sich im Markt etablieren.
Dank einer tollen Fernseh und Internetwerbung nehmen die
Verkaufszahlen rasant zu.

Es ist jedoch zu erwarten, dass nicht mehr als 20\% aller Käufer auf
«Silly Blubb» umschwenken werden.

Nach dem ersten Monat sind bereits 5\% der Waschmittelkäufer auf
«Silly Blubb» umgelenkt worden. Nach zwei weiteren Monaten sind wir
bei 8\% gelandet.


\begin{bbwAufgabenBlock}
\item Machen Sie eine Skizze im Koordiantensystem, welche die
Abhängigkeit von Monat ($x$-Achse) zu Käuferzahl in Prozent darstellt.

\TRAINER{Graph}

\item Geben Sie einen mögliche Funktionsterm $f(t)$ an, welcher die
Prozentzahl der Käufer nach Monaten angibt.

\TRAINER{$$f(t) = 20 - 15 \cdot{} (\frac{12}{15})^{\frac{t-1}{2}}$$
}

\item Wie viele Prozente der Käufer sind in Monat 6 nach Verkaufsstart
  bereits «Silly Blubb» Käufer?
  
\TRAINER{
$$y = 20 - 15\cdot{} \left(\frac{12}{15}\right)^{\frac{6-1}{2}} \approx 11.41\% $$%%
}%% end TRAINER

\item Nach wie vielen Monaten sind 16\% der Käufer von «Silly Blubb»
  überzeugt worden?

\TRAINER{
$$f(t) = 16 [\%] = 20 - 15\cdot{} \left(\right)^{\frac{t-1}2}$$
$$ \frac4{15} = \left(\right)^{\frac{t-1}2}$$
$$ \frac4{15} = \left(\right)^{\frac{t-1}2}$$
$$\frac{t-1}2  = \log_{\left(\frac{12}{15}\right)}\left(\frac4{15}\right)$$
$$t = 1 + 2\cdot{}
  \log_{\left(\frac{12}{15}\right)}\left(\frac4{15}\right) \approx
  12.85 \textrm{ Monate}$$
}%% END TRAINER

\end{bbwAufgabenBlock}


\platzFuerBerechnungenBisEndeSeite{}


\newpage
