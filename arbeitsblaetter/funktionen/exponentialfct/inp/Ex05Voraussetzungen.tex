\subsection{Voraussetzungen}


\bbwActAufgabenNr{} \textbf{Welche Art von Wachstum?}

Entscheiden Sie bei den folgenden Datenreihen, um welche Art von
Wachstum es sich handelt (linear, exponentiell, quadratisch)?

Tipp: Skizzieren Sie dazu entsprechende Punkte im Koordinatensystem!

\begin{bbwAufgabenBlock}
\item 5, 10, 15, 20, 25, ... \TRAINER{linear}
\item 2, 4, 6, 8, 10,  \TRAINER{linear}
\item 3, 6, 12, 24, 48, ...  \TRAINER{exponentiell}
\item 4, 9, 16, 25, 36, 49, ...  \TRAINER{quadratisch (Parabel $x^2)$}
\item 1.8, 5.4, 9, 12.6, 16.2, 19.8, 23.4, 27, ...  \TRAINER{linear}
\item 2.3, 3.45, 5.175, 7.763, 11.64, 17.46, 26.20, ...  \TRAINER{exponentiell Faktor 1.5}
\end{bbwAufgabenBlock}

\platzFuerBerechnungenBisEndeSeite{}


%%%%%%%%%%%%%%%%%%%%%%%%%%%%%%%%%%%%%%%%%%%%%%%%%%%%%%%%%%%%%%%%%%%%%%%%%%%
\bbwActAufgabenNr{} \textbf{Rate vs. Faktor}

\nextBbwAufgabenNummer{}%% Dies geschieht im aufgabenBlock implizit.

Berechnen Sie den Wachstumsfaktor aus der Wachstumsrate
bzw. umgekehrt die Wachstumsrate aus dem Wachstumsfaktor.

\begin{tabular}{c|c}\hline
  Wachstums\textbf{rate}     & Wachstums\textbf{faktor} \\
  +4\%                       & \LoesungsRaumLang{1.04}  \\\hline
  \LoesungsRaumLang{-9\%}    & 0.91                     \\\hline
  \LoesungsRaumLang{-25\%}   & 0.75                     \\\hline
  +100\%                     & \LoesungsRaumLang{2}     \\\hline
  \LoesungsRaumLang{+200\%}  &  3                       \\\hline
  -100\%                     & \LoesungsRaumLang{0}     \\\hline
\end{tabular} 

\TNTeop{}


%%%%%%%%%%%%%%%%%%%%%%%%%%%%%%%%%%%%%%%%%%%%%%%%%%%%%%%%%%%%%%%%%%%%%%%%%%%
\newpage
\bbwActAufgabenNr{} \textbf{Skizzieren}

Skizzieren Sie die Graphen zu den folgenden Funktionen. Die $y$-Achse
zeigt wie immer den Funktionswert $f(x)$ bzw. $f(t)$ an. Die $x$-Achse
wird bei Funktionen, welche von der Zeit abhängen mit $t$ bezeichnet.

(Mit $\e$ wird die Eulersche Konstante $2.71828182846...$ bezeichnet.)

\begin{bbwAufgabenBlock}
\item $f: y=1.2^x$
\item $f(x) = 2^{0.263\cdot{}x}$
\item $t\mapsto{} 0.8^t$
\item $y=3^{\frac{-t}{4.923}}$
\item $x\mapsto \e^{0.1823x}$
\end{bbwAufgabenBlock}

Was fällt Ihnen dabei auf? Welche der obigen Funktionen gehören zur
Wachstumsrate von $-20\%$? Welche der obigen Funktionen gehören zum
Wachstumsfaktor 1.2?

\TNTeop{
$1.2 \approx 2^{0.263} \approx{} e^{0.1823} $, somit bezeichnen  a) b)
  und e) annähernd die selben Prozesse. Ebenso: $0.8 \approx{}
  3^{\frac{1}{4.923}}$ Somit sind c) und d) annähernd gleich.
}


%%%%%%%%%%%%%%%%%%%%%%%%%%%%%%%%%%%%%%%%%%%%%%%%%%%%%%%%%%%%%%%%%%%%%%%%%%%

\newpage
