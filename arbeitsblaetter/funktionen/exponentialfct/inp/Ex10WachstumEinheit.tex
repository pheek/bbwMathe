\subsection{Einfache Wachstumsprozesse}
\textit{Startwert ist hier eine Einheit.}


\bbwActAufgabenNr{} \textbf{Pilz}

An einer Kellerwand wächst eine Schimmelpilzart, deren Fläche
exponentiell zunehme.
Am Anfang ist $1 \textrm{ m}^2$ der Wand mit dem Pilz bedeckt. Jeden Tag nimmt die Fläche um Faktor 1.25 zu.

\begin{bbwAufgabenBlock}
\item Welche Fläche ist nach 3 Tagen bedeckt?
\TRAINER{$1.25^3 \approx 1.9531 \textrm{m}^2$}
\item Welche Fläche ist nach $n$ Tagen bedeckt?
\TRAINER{$1.25^n \textrm{m}^2$}
\item Nach wie vielen Tagen sind $5 \textrm{ m}^2$ der Wand mit dem Pilz bedeckt?
\TRAINER{$n=\log_{1.25}(5)\approx 7.213$ Tage.}
\end{bbwAufgabenBlock}
\platzFuerBerechnungenBisEndeSeite{}

%%%%%%%%%%%%%%%%%%%%%%%%%%%%%%%%%%%%%%%%%%%%%%%%%%

\bbwActAufgabenNr{} \textbf{Karton}

Ein Karton von einem Millimeter (mm) Dicke wird halbiert
und aufeinandergelegt, sodass ein Karton der Dicke 2 mm entsteht.

Dieses Verfahren wird nun wiederholt und nach einer zweiten
Verdopplung hat der Karton eine Dicke von 4 mm.

\begin{bbwAufgabenBlock}
  \item Wie dick ist der Karton nach einer weiteren Verdopplung, also
    nach der 3. Verdopplung der Dicke?
    \TRAINER{Der Karton wird nun 8mm dick. 8 mm = 4 mm + 4 mm =
      2$\cdot{}$4 mm}
    
  \item Wie dick ist der Karton nach fünf Verdopplungen?
    \TRAINER{Der Karton wird $2^5$ mm = $32$ mm dick.}

  \item Wie dick ist der Karton nach 7 Verdopplungen?
    \TRAINER{Der Karton wird nun $2^7$ = $128$ mm dick.}

  \item Wie dick ist der Karton nach $n$ Verdopplungen ($n \in \mathbb{N}$)?
      \TRAINER{Der Karton wird nun $2^n$ mm dick, wobei wir das $n$
        natürlich beliebieg wählen können.}
    \item Nach wie vielen Verdopplungen ist der Karton 1 m dick?
      \TRAINER{Hier kann man a) auspropieren und kommt bei 10
        Verdopplungen auf 1.024 m. Natürlich geht das auch mit einer
        Formel:
        $$2^n = 1000$$
        Logarithmieren
        $$n = \log_2(1000) \approx 9.966 $$
        Das heißt, so gut wie exakt ein Meter dick wäre es bei 9.966
        Verdopplungen. Dies ist aber technisch nicht möglich, somit
        müssen wir auf die nächste Verdopplung aufrunden und erhalten
        auch 10 Verdopplungen.
      }

    \item Nach wie vielen Verdopplungen wäre der Karton bereits über 100 m
      hoch, also mind. so hoch wie das Sulzer Hochhaus? Wie wiel höher als
      des Gebäude von 100 m würde der Karton dadurch?
      \TRAINER{$$2^n = 100\,000$$ $$n = \log_2(100\,000)\approx
        16.61$$ Somit brauche ich 17 Verdopplungen. Da $2^{17} =
        131072$ ergibt, wäre der Kartonstapel bereits 31.072 m höher
        als das Gebäude.}
    
\end{bbwAufgabenBlock}


\platzFuerBerechnungenBisEndeSeite{}


%%%%%%%%%%%%%%%%%%%%%%%%%%%%%%%%%%%%%%%%%%%%%%%%%%

\bbwActAufgabenNr{} \textbf{Sparen}

Max spart auf eine neue Spielekonsole. Die Mutter will Max dabei unterstützen und vor allem will sie, das Max lernt zu sparen.
Sie schlägt ihm daher zwei Spar-Varianten vor.

Bei Variante $A$ erhält Max Anfangs der ersten Woche  $5.-$ \euro{} und danach jede Folgewoche $5.-$ \euro{} mehr als in der Vorangehenden Woche. Also 1. Woche $5.-$; 2. Woche $10.-$; 3. Woche $15.-$ \euro{} etc.

Bei Variante $B$ erhält er Anfangs ersten Woche ebenfalls $5.-$
\euro{}, jedoch in jeder Folgewoche 40\% mehr als in der vorangehenden
Woche. Also in der 1. Woche auch $5.-$, in der 2. Woche bereits 7.-
\euro{} und in der 3. Woche $9.80$ \euro{} etc.

\begin{bbwAufgabenBlock}
\item In der wie vielten  Woche erhielte Max zum ersten Mal mehr mit Variante $B$ als mit Variante $A$? (Hier ist nicht das kumulierte Vermögen, sondern das «Einkommen» gefragt.)
   \TRAINER{ In Woche 7 erhält er mit Variante $A$ 35.- \euro{},
     während er mit Variante $B$ in der 7. Woche bereits 37.65 \euro{} erhält.}

\item In welchem Woche überholt das angesparte Vermögen (= Summe aller
  Spargelder) durch exponentielles Einkommenswachstum (Varianten $B$) das Vermögen, das durch die Variante $A$ angespart wurde?
   \TRAINER{ In Woche 9 hat er mit Variante $A$ total 225.- \euro{}
     angespart; wohingegen mit Variante $B$ in Woche 9 bereits 245.76 \euro angespart wurden.}

\item Wie sieht das angesparte Vermögen bei Variante $A$ nach 12 Wochen aus?
      \TRAINER{ In der 12. Woche erhält er 60.- \euro, was sich auf 390.- \euro{} kumuliert.}

\item Wie sieht das angesparte Vermögen bei Variante $B$ nach 12 Wochen aus?
      \TRAINER{ In der 12. Woche erhält er 202.48 \euro, was sich auf 696.17 \euro{} kumuliert.}
\end{bbwAufgabenBlock}
\platzFuerBerechnungenBisEndeSeite{}


%%%%%%%%%%%%%%%%%%%%%%%%%%%%%%%%%%%%%%%%%%%%%%%%%%

\newpage
