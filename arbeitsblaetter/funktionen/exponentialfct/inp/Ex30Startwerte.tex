\subsection{Startwerte}
\textit{Andere Startwerte als eine Einheit.}


\bbwActAufgabenNr{} \textbf{Gummiball}

Ein Gummiball wird fallen gelassen. Der Ball springt jeweils 72\% der Fallhöhe wieder zurück.

Anfänglich wird der Ball aus $1.7 \textrm{m}$ Höhe fallen gelassen

\begin{bbwAufgabenBlock}

\item Wie hoch springt der Ball nach dem 3. Aufprall wieder zurück?
      \TRAINER{$1.7 \cdot{} 0.72^3 \approx 63.45 \textrm{cm}$}
\item Wie hoch springt der Ball nach dem $n$. Aufprall wieder hoch?
      \TRAINER{$1.7 \cdot{} 0.72^n$}
\item Nach dem wievielten Aufprall springt der Ball noch 10 cm hoch?
      \TRAINER{$1.7 \cdot{} 0.72^n = 0.1 \Longrightarrow n = \log_{0.72}\left(\frac{0.1}{1.7}\right) \approx 8.6$. Das heißt: Nach 8 Sprüngen ist der Ball noch höher als 10 cm; nach dem 9. Sprung hingegen weniger als 10 cm.}
\end{bbwAufgabenBlock}
\platzFuerBerechnungenBisEndeSeite{}

%%%%%%%%%%%%%%%%%%%%%%%%%%%%%%%%%%%%%%%%%%%%%%%%%%


\bbwActAufgabenNr{} \textbf{Neophytenplage}

Eine neu eingeschleppte Brombeerenart vermehrt sich im Wald exponentiell.

Anfänglich werden 82 Pflanzen gezählt. Ein Jahr später sind es bereits 102 Pflanzen.

\begin{bbwAufgabenBlock}

\item Wie groß ist der jährliche Zunahmefaktor?
      \TRAINER{$a = \frac{102}{82} = 1.\overline{24390}$}
\item Wie groß ist die jährliche Zuwachsrate?
      \TRAINER{$p = \frac{102}{82} - 1 \approx 24.39\%$}
      
\item Wie viele Brombeerpflanzen sind nach fünf Jahren zu erwarten?
      \TRAINER{$82\cdot{}a^5 \approx 244$ Pflanzen}

\item Wie viele Brombeerpflanzen sind nach $n$ Jahren zu erwarten?
      \TRAINER{$82\cdot{}a^n \approx{} 82\cdot{} 1.2439^n$ Pflanzen}
\item Nach wie vielen Jahren werden 1000 Pflanzen erwartet, wenn sich die Pflanzenart weiterhin ungehindert ausbreiten kann?
\TRAINER{$82\cdot{} a^n = 1000 \Longrightarrow n=\log_a(\frac{1000}{82}) \approx 11.46$}
\end{bbwAufgabenBlock}
\platzFuerBerechnungenBisEndeSeite{}



%%%%%%%%%%%%%%%%%%%%%%%%%%%%%%%%%%%%%%%%%%%%%%%%%%


\bbwActAufgabenNr{} \textbf{Tierpopulation}

Eine Tierpopulation von fünf tausend Stück nimmt wegen verändernder Klima-
und Umweltbedingungen jährlich um sechs Prozent ab.


\begin{bbwAufgabenBlock}

\item Skizzieren Sie die Population über die nächsten 30 Jahre.
         \TRAINER{Graph}
\item Wie lautet eine Funktionsgleichung, welche die Tierpopulation in
  Abhängigkeit vom Jahr angibt?
\TRAINER{$y = 5\,000 \cdot{} 0.94^{t}$}
\item Um wie viele Tiere hat die Population nach 15 Jahren abgenommen?

  \TRAINER{$$y = 5\,000 \cdot{} 0.94^{15} = 1976 \textrm{ Tiere sind
      noch übrig}$$
d.\,h.:
  $$5000 - 1976  = 3024 \textrm{ Tiere sind «verschwunden»}$$}%% End TRAINER 

  

\item Wann wird die Population auf eine kritische «Größe» von 100
  Individuen geschrumpft sein?
  \TRAINER{$y = 100 = 5\,000 \cdot{} 0.94^{t}$
    $$\frac{100}{5\,000} = 0.94^t$$
$$t = \log_{0.94}\left(\frac{100}{5\,000}\right) \approx{} 63
    \textrm{ Jahre} $$
  }%% end TRAINER
\end{bbwAufgabenBlock}
\platzFuerBerechnungenBisEndeSeite{}

%%%%%%%%%%%%%%%%%%%%%%%%%%%%%%%%%%%%%%%%%%%%%%%%%%


\bbwActAufgabenNr{} \textbf{Licht im Wasser}

In einem Wasserbecken nimmt die Lichtintensität pro Meter auf 24\% ab.
Dies spielt keine Rolle, ob horizontal oder vertikal gemessen wird.


\begin{bbwAufgabenBlock}

 \item Um wie viel nimmt die Intensität pro Meter ab? \TRAINER{Wenn
   etwas \textbf{auf} 24\% abnimmt, so nimmt es \textbf{um} 76\% ab.}
  
\item Skizzieren Sie die Intensität im Koordinatensystem wie folgt:
   $x$-Achse = Eindringtiefe in Metern. Die Maßeinheit der
   Lichtintensität wird in Watt pro Quadratmetern gemessen ($\textrm{W}/\textrm{m}^2$); dies
   hat für die Aufgabe hier jedoch keine Relevanz.
   $y$-Achse = Lichtintensität in \%. (Gestartet bei 100\%)
         \TRAINER{Graph}
\item Wie lautet eine Funktionsvorschrift von $\textrm{m}\mapsto \%$?
\TRAINER{$x\mapsto 100\% \cdot{} 0.24^x$}
\item Wie viel Prozent der ursprünglichen Lichtintensität sind nach 5
   m noch übrig? \TRAINER{$0.24^5 \approx 0.07963 \%$}

\item In wie vielen Metern ist noch 1\% der Lichtintensität übrig?
\TRAINER{$x = \log_{0.24}(0.01) \approx  3.22 [\textrm{m}]$}
\end{bbwAufgabenBlock}
\platzFuerBerechnungenBisEndeSeite{}


%%%%%%%%%%%%%%%%%%%%%%%%%%%%%%%%%%%%%%%%%%%%%%%%%%%%%%%%%%%%%%%%%%%%%


\bbwActAufgabenNr{} \textbf{Federpendel}

Ein Federpendel wird in Schwingung gebracht. Wegen diversen
physikalischen Einflüssen (Reibung, ...) nimmt die Amplitude
exponentiell ab.

Nach einer Minute beträgt die Amplitude 5.3 cm und nach 4 Minuten nur noch
2.6 cm.

\begin{bbwAufgabenBlock}

\item Machen Sie eine Skizze.
      \TRAINER{Graph}
\item Geben Sie eine mögliche Funktionsgleichung der Amplitude in Abhängigkeit
der Zeit an.
      \TRAINER{... Lösung noch ausstehend ...}
      
\item Wie war die Amplitude zu Beginn ($t=0$)?
      \TRAINER{... Lösung noch ausstehend ...}

\item Wie groß wird die Amplitude nach 10 Minuten noch sein?
      \TRAINER{... Lösung noch ausstehend ...}
\end{bbwAufgabenBlock}
\platzFuerBerechnungenBisEndeSeite{}



%%%%%%%%%%%%%%%%%%%%%%%%%%%%%%%%%%%%%%%%%%%%%%%%%%
\bbwActAufgabenNr{} \textbf{Luftdruck}

Der Atmosphärische Luftdruck nimmt ab, je weiter wir uns von der
Meereshöhe entfernen. Der Druck wird in $\textrm{hPa}$ = Hectopascal
gemessen. Ein $\textrm{Pa}$ ist gleich $1 \textrm{N}/\textrm{m}^2$, also ein
Pascal ist die Kraft von einem Newton pro Quadratmeter.

Auf Meereshöhe ist der Luftdruck ca. $ 1013 \textrm{ hPa}$ und nimmt
ca. 13\% pro km Höhenunterschied ab.

\begin{bbwAufgabenBlock}

\item Zeichnen Sie den Funktionsgraphen, der den Luftdruck ($y$-Achse)
  in Abhängigkeit der Höhe über Meer ($x$-Achse) angibt.

  \TRAINER{Graph}

\item Wie lautet ein möglicher Funktionsterm, der die Abhängigkeit
  Luftdruck ($\textrm{hPa}$) von Meereshöhe (in Metern) angibt?

  \TRAINER{$y = 1013 \cdot{} 0.87^{\frac{x}{1000}}$}


\item Wie hoch ist der Luftdruck auf dem Matterhorn (4478 Meter über
  Meeresspiegel)?

  \TRAINER{$y = 1013 \cdot{} 0.87^{\frac{4478}{1000}} \approx 543
    \textrm{ hPa}$}

\item In welcher Höhe über Meeresspiegel ist der Luftdruck nur noch
  $500 \textrm{ hPa}$?

  \TRAINER{
    $$500 = 1013 \cdot{} 0.87^{\frac{x}{1000}}$$
    $$\frac{500}{1013} = 0.87^{\frac{x}{1000}}$$
    $$x = 1000 \cdot{} \log_{0.87}\left(\frac{500}{1013}\right)
    \approx 5070 \textrm{ m}$$
    }%% END TRAINER
  
\end{bbwAufgabenBlock}
\platzFuerBerechnungenBisEndeSeite{}

%%%%%%%%%%%%%%%%%%%%%%%%%%%%%%%%%%%%%%%%%%%%%%%%%%%%%%%%%%%


\bbwActAufgabenNr{} \textbf{Temperierte Stimmung}

\bbwCenterGraphic{15cm}{img/tonleiter.png}
\begin{center}Legende: c-Dur Tonleiter\end{center}


In der Musik wird heute fast ausschließlich die \textit{temperierte
  Stimmung} verwendet. Dabei wird eine Oktave in zwölf gleiche
Halbtonschritte eingeteilt. Gleich bedeutet hier, dass das Verhältnis
der Frequenzen von zwei aufeinanderfolgenden Halbtönen jeweils immer
gleich ist.

Das Verhältnis der Frequenzen einer Oktave ist 1:2. Hat ein Ton \zB
die Frequenz 220 Hz, so hat seine Oktave die Frequenz 440 Hz.

Eine Oktave wird in die Töne c, cis, d, dis, e, f, fis, g, gis, a, b und h
unterteilt; danach folgt das c der nächst höheren Oktave.

Das Verhältnis der Frequenzen von d zu dis ist also genau gleich, wie
das Verhältnis der Frequenzen von f zu fis.

Bemerkung:  In der oben
abgebildeten c-Dur Tonleiter sind nicht alle Halbtonschritte
vorhanden. Eine Tonleiter mit nur Halbtonschritten wird chromatisch
genannt.

\begin{bbwAufgabenBlock}

\item Gesucht ist das Verhältnis der Frequenzen zwischen zwei aufeinanderfolgenden Halbtönen. Tipp: Wird
  das Verhältnis zwölf mal angewandt, so haben wir eine Oktave und das
  totale Verhältnis von 1:2 erreicht.

  \TRAINER{Verhältnis sei $a$. Somit gilt $1 : a^{12} = 1:2$ oder
    $a^{12} = 2$. Lösen wir das nach $a$ auf, so erhalten wir das
    Verhältnis von $a= \sqrt[12]{2} \approx 1.059463$}

\item In der \textit{reinen Stimmung} wäre das Verhältnis von Grundton
  (c) zur Quinte (g) 3:2. Dazwischen liegen sieben
  Halbtonschritte. Berechnen Sie das Verhältnis der Frequenzen von c zu
  g in der \textit{temperierten Stimmung} und vergleichen Sie das
  Resultat mit der \textit{reinen Stimmung}.

  

  \TRAINER{$a^7 \approx 1.4983$, was etwas weniger als 1.5 ist. Somit
    ist das \textit{reine} g leicht höher als das \textit{temperierte}
    g.}%% ENd Trainer

\item Der Kammerton a hat oft die Frequenz 440 Hz. Berechnen Sie die
  Frequenz des darüber liegenden Tones c, wenn Sie wissen, dass
  zwischen a und c drei Halbtonschritte liegen.

  \TRAINER{Frequenz von c $= 440\cdot{} a^3 \approx 523.3 \textrm{ Hz}$}%% end Trainer

    
\end{bbwAufgabenBlock}
\platzFuerBerechnungenBisEndeSeite{}

%%%%%%%%%%%%%%%%%%%%%%%%%%%%%%%%%%%%%%%%%%%%%%%%%%%%%%%%%%%


\bbwActAufgabenNr{} \textbf{Feinstaub}

Nachdem die letzten Handwerker das Haus verlassen hatten; als endlich der
Elektriker die letzten Steckdosen montiert und der Maler die letzten
Abdeckungen zusammengesucht und in seinen Wagen geladen hatte, kehrte
endlich Ruhe in das alte Gemäuer ein. Doch eines blieb. Der Staub!
In allen Ritzen hatte er sich versteckt, erpicht darauf,
jederzeit bei kleinstem Windstoß hervorzutreten und den zukünftigen
Mietern das Husten zu erleichtern und die Ruhe zu stehlen!

Der Hausherr machte sich selbst auf in den Kampf gegen den Baustellenstaub und
begann mit Besen, Tüchern und einem Staubsauger dem beinahe
unberechenbaren Störenfried den Garaus zu machen. Er begann ganz oben
im Dachstock und die Putzerei endete nach einigen Stunden im Keller.

Doch der Staub blieb. Er ließ sich nicht so einfach vertreiben. Etwas
weniger war da, doch eine gewisse Erinnerung an Sisyphus der alten
griechischen Mythologie begann im Hausherrn aufzusteigen. Der
vorhandene Staub verteilte sich gemütlich wieder gleichmäßig in Haus
und Ritzen und wartet auf ein nächstes Opfer.

Das Putzen wurde wiederholt, doch immer noch waren Ritzen,
Staubsaugersack, Kleider, Schuhe; kurz alles noch voller Staub.

Einige weitere Male wurde die Saug- und Wisch-Aktion wiederholt. Und
erst nachdem zum ersten Mal weniger als 0.1\% (also ein Promille) der ursprünglichen
Staubmenge vorhanden waren, gab sich der Hausherr mit seiner Arbeit
zufrieden.

Wie oft hatte der Hausherr die Putz-Aktion durchgeführt?

Gehen Sie davon aus, dass nach jedem Putzen noch 37\% der vorhandenen
Staubmenge im Haus verbleibt.

\begin{bbwAufgabenBlock}

\item Machen Sie eine Skizze bei der die $x$-Achse die Anzahl der
  Putz-Aktionen und die $y$-Achse die Staubmenge, begonnen bei 100\%,
  aufzeigt.
  
\item Berechnen Sie die Anzahl der Putz-Aktionen, die nötig sind,
    sodass maximal ein Promille der ursprünglichen Staubmenge
    vorhanden bleibt.
\TRAINER{
    $$0.37^n = 0.001$$
    Logarithmieren
    $$n = \log_{0.37}(0.001) \approx 6.94$$
    Nach «6.94» Putz-Aktionen wäre noch exakt 0.1\% der
    ursprünglichen Staubmenge vorhanden. Somit kann der Hausherr nach
    dem siebten Mal getrost aufhören und sich mit einem Promille
    Reststaub zufrieden geben.
}%% END Trainer

\end{bbwAufgabenBlock}
\platzFuerBerechnungenBisEndeSeite{}

%%%%%%%%%%%%%%%%%%%%%%%%%%%%%%%%%%%%%%%%%%%%%%%%%%%%%%%%%%%


\newpage
