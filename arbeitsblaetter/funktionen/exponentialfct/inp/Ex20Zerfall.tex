\subsection{Zerfall}
\textit{Nicht immer ist die Wachstumsrate größer als 1}



\bbwActAufgabenNr{} \textbf{Wert des Autos}

Um wie viel Prozent hat der Wert eines Autos jährlich verloren, wenn
der Wert nach 4 Jahren um 60 \% gesunken ist?
\nextBbwAufgabenNummer%% Dies geschieht im bbwAufgabenBlock implizit
\platzFuerBerechnungenBisEndeSeite{}

\TRAINER{
$$100 - 60 = 0.4$$
somit
$$0.4 = a^4$$
$$a = \sqrt[4]{0.4} \approx 79.527\%$$
Dies entspricht einer jährlichen Abnahme von.
$$ 1 - 79.527\%  \approx 0.2047 $$
Also eine Prozentuale Abnahme von 20.47\%.
}%% END TRAINER


%%%%%%%%%%%%%%%%%%%%%%%%%%%%%%%%%%%%%%%%%%%%%%%%%%
\bbwActAufgabenNr{} \textbf{Grogg}

In der Küche der Berghütte steht eine Flasche Grogg, die genau einen Liter des
heiß begehrten Getränks enthält. Abgemacht war unter den Bergleuten,
dass der Grogg nach der harten Bergtour gerecht aufgeteilt werde.

Nun ist es aber so, dass die Personen alle nacheinander in die Hütte
kommen und alle denken sich, ich nehme nun nur ein Schlückchen vorab; 
nur 8\%, nur so wenig also, dass es nicht auffällt.

\begin{bbwAufgabenBlock}

\item Die ersten vier Personen sind Andrea, gefolgt von Bert, dann
  Clau und schließlich Dani.
  Wie viel Grog ist nach der 4. Person noch in der Flasche?
  \TRAINER{Der Abnahmefaktor beträgt $1-0.08 = 0.92 = 92\%$. Nach der 4. Flasche ist $0.92^4$ Liter also noch 7.16 dl
    Grog in der Flasche.}

\item Geben Sie an, wie viel Grog sich nach der 8. Person noch in der
  Flasche befindet.
  \TRAINER{$0.92^8 \approx 0.513$, also ca. 5.13 dl.}

\item Geben Sie eine Formel an, die angibt wie viel Grog nach der
  $n$-ten Person in der Flasche ist.
  \TRAINER{$f(n) = 0.92^n$}
  
\item Weitere Personen erreichen die Hütte. Nach total wie vielen Personen
  ist nur ein ein Rest von 2dl in der Flasche?
  \TRAINER{$$0.2 = 0.92^n$$ Somit erhalten wir durch logarithmieren:
    $$n = log_{0.92}(0.2) \approx{} 19.30$$
  Somit ist nach der 19. Person noch etwas mehr und nach der
  20. Person schon etwas weniger als 2 dl in der Flasche.}
\end{bbwAufgabenBlock}
\platzFuerBerechnungenBisEndeSeite{}
%%%%%%%%%%%%%%%%%%%%%%%%%%%%%%%%%%%%%%%%%%%%%%%%%%
\bbwActAufgabenNr{} \textbf{Wohnwagen}
Ein Wohnwagen hat innerhalb von fünf Jahren 75\% an Wert verloren. 

\begin{bbwAufgabenBlock}

\item Geben Sie eine Formel für den Wert des Wohnwagens in
Abhängigkeit der Zeit an.

\item Wie viel Wert (in \%) verlor der Wohnwagen jährlich?
  \TRAINER{$$f(t) = a^t$$ mit
    $$0.25 = a^5$$
    Somit ist 
    $$a = \sqrt[5]{0.25} \approx  0.757858$$
    
  $$f(t) = \left(\sqrt[5]{0.25}\right)^t \approx 0.757858^t $$}

\item Nach wie vielen Jahren (ab Beginn) wird der Wohnwagen noch 10\%
seines ursprünglichen Wertes haben?

\TRAINER{
  Der Wert nach 5 Jahren beträgt noch 25\%.
  Wenn die Formel $y=a^t$ verwendet wird mit

  $y$=Wert nach $t$ Jahren.
  
  $t$=Anzahl Jahre

  $a$=Abnahmefaktor pro Jahr

  so können wir alles bekannte einsetzen:

  $$y=a^t$$
  $$25\% = a^5$$
  $$0.25=a^5$$
  $$\sqrt[5]{0.25} = a$$

  Daraus ergibt sich der Jährliche Faktor $a \approx 0.757858$.
  Somit verbleiben Jedes Jahr ca. 75.79\% des Wertes vorhanden, was
  einer Abnahme von 1-0.7578... , also 24.21\% entspricht.
}%% END TRAINER

\end{bbwAufgabenBlock}
\platzFuerBerechnungenBisEndeSeite{}

%%%%%%%%%%%%%%%%%%%%%%%%%%%%%%%%%%%%%%%%%%%%%%%%%%
\newpage
