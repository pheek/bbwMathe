
\subsection{Begrenzter Zerfall}

\bbwActAufgabenNr{} \textbf{Asymptote}
Gegeben ist die Funktion $f(t) = 2.87^{-t} + 1.34$.
\begin{bbwAufgabenBlock}
\item
  Skizzieren Sie die Funktion $f(t)$.
  
    \bbwGraph{-1}{9}{-1}{4}{
\TRAINER{      \bbwFunc{exp(-1.0543* \x)+ 1.34}{-0.5:8.5}
      \bbwFuncC{1.34}{-1:9}{ForestGreen}}
    }%% end bbw Graph

  
\item
\noTRAINER{Finden Sie die zugehörige \textbf{Asymptote}. Eine Asymptote ist diejenige Gerade, an den
sich ein Funktionsgraph \textit{anschmiegt}.}%% end noTRAINER
\TRAINER{$g(t) = 1.34$}

  \item Was hat die Zahl $2.87$ mit dem Startwert und der Asymptote zu
    $f$ zu tun? \TRAINER{Die
    $2.87$ gibt an, wie rasch sich der Graph von $f$ an $g$
    anschmiegt. Auf den Startwert (1+1.34) und auf die Asymptote $g: t\mapsto 1.34$ hat die Zahl $2.87$ keinen Einfluss.}
\end{bbwAufgabenBlock}

\noTRAINER{}

\platzFuerBerechnungenBisEndeSeite{}

%%%%%%%%%%%%%%%%%%%%%%%%%%%%%%%%%%%%%%%%%%%%%%%%%%%%%%%


\bbwActAufgabenNr{} \textbf{Wäsche Trocknen}

Wäsche trocknet schneller, umso trockener die Umgebungsluft ist. Wir
vernachlässigen hier die Sonne, die Temperatur und den Wind und gehen
davon aus, dass die Differenz der Umgebungs-Feuchte und der
Wäsche-Feuchte exponentiell abnimmt.

Frisch gewaschene Wäsche werde aufgehängt und habe im Geflecht eine «Luftfeuchtigkeit» von 95\%
(= relative Luftfeuchtigkeit).

Die Außenluft habe eine relative Luftfeuchtigkeit von 35\%. 


Nach einer
Stunde messen wir eine relative «Luftfeuchtigkeit im Geflecht von
54\%.


\begin{bbwAufgabenBlock}
\item Machen Sie eine Skizze im Koordinatensystem, welche die
Abhängigkeit von der Luftfeuchtigkeit von der Zeit ($t$) aufzeichnet.

\TRAINER{Graph}

\item Geben Sie eine mögliche Funktionsgleichung $f(t)$ an, welche die relative
«Luftfeuchtigkeit im Geflecht» in Stunden ($t$) nach dem Aufhängen
angibt.

\TRAINER{$$f(t) = 35 +
(95-35) \cdot \left( \frac{19}{95-35}\right)^t$$

$$f(t) = 35 + 60 \cdot \left( \frac{19}{60}\right)^t$$
}

\item Wie «trocken» ist die Wäsche nach 1.5 Stunden?

\TRAINER{
$$f(1.5) = 35 + 60 \cdot \left( \frac{19}{60}\right)^{1.5} \approx 45.69\%$$

}

\item Nach wie vielen Stunden hat die Wäsche eine relative
«Wäsche-Feuchtigkeit» von 40\%.


\TRAINER{

$$f(t) = 35 + 60 \cdot \left( \frac{19}{60}\right)^t$$
$f(t) = 40[\%]$ einsetzen:

$$40 [\%| = 35 + 60 \cdot \left( \frac{19}{60}\right)^t$$
$$5 [\%| = 60 \cdot \left( \frac{19}{60}\right)^t$$
$$\frac5{60} [\%| =\left( \frac{19}{60}\right)^t$$
$$t = \log_{\frac{19}{60}}(\frac5{60}) \approx 2.16 [\textrm{h}]$$
}%% END TRAINER

\end{bbwAufgabenBlock}


\platzFuerBerechnungenBisEndeSeite{}

%%%%%%%%%%%%%%%%%%%%%%%%%%%%%%%%%%%%%%%%%%%%%%%%%%%%%%%%%%%%%%%%%%%%%%%%%%%%5


\bbwActAufgabenNr{} \textbf{Kara Ben Nemsi}
\nextBbwAufgabenNummer{}

Kara Ben Nemsi ist gut im Spurenlesen. Er weiß, dass Steine rund um ein Feuer eine Temperatur von rund $400^\circ$ C annehmen.
Ebenfalls ist ihm der gesättigte Zerfall bei der Temperaturkurve nach dem Löschen des Feuers bekannt.

Es gilt:

$$f(t) = U + \left(f(0) - U\right) \cdot{} a^t$$
Dabei sind

$f(t)$ die Temperatur nach $t$ Minuten

$U$ ist die Umgebungstemperatur (hier die Sättigungsgrenze)

$f(0)$ ist die Anfangstemperatur nach Löschen der Steine (also $400^\circ$ C)

$a$ ist ein spezifischer Abnahmefaktor für Wüstensteine ($a\approx 0.9753$).


Kara Ben Nemsi verfolgt einige Ganoven und erreicht eine Feuerstelle, welche offensichtlich seine Verfolgten benutzt hatten. Er misst eine Umgebungstemperatur von $28^\circ$ C und ermittelt die Temperatur der Steine auf $40^\circ$ C.

Welchen Vorsprung (in Minuten) haben die Ganoven? (Oder: Vor wie vielen Minuten wurde das Feuer gelöscht?)


\platzFuerBerechnungenBisEndeSeite{}
\TRAINER{
$$f(t) = 28 + (400-28)\cdot{} 0.9753^t$$
$$40 = 28 + (400-28)\cdot{} 0.9753^t$$
$$12 =  (400-28)\cdot{} 0.9753^t$$
$$12 =  (372)\cdot{} 0.9753^t$$
$$\frac{12}{372} = 0.9753^t$$
$t=log_a(\frac{12}{372}) \approx{} 137.3 \textrm{min}$}


%%%%%%%%%%%%%%%%%%%%%%%%%%%%%%%%%%%%%%%%%%%%%%%%%%%%%%%%%%%%%%%%%%%%%%%%%%%%%%%%%%%%%%%%% 


\bbwActAufgabenNr{} \textbf{Impfung}
\nextBbwAufgabenNummer{}

Nach einer rigorosen Durchimpfung kann eine Krankheit (die hier nicht genannt werden will) von anfänglich 30\% infizierten Personen drastisch reduziert werden. Es ist davon auszugehen, dass jedoch im Endeffekt immer noch 2\% der Bevölkerung die Krankheit bekommen kann bzw. ansteckend bleiben wird.

Die Funktionsgleichung der Prozentzahl $p()$ der angesteckten Bevölkerung nach $t$ Wochen lautet somit:

$$p(t) = 2 + (30-2)\cdot \e^{qt}$$

Ermitteln Sie $q$, wenn Sie wissen, dass die Krankheit nach 8 Wochen von 30\% bereits auf 12\% gesunken ist.

$$p(t) = 2 + \LoesungsRaumLang{28\cdot{}e^{-0.1287\cdot{}t}}$$


\platzFuerBerechnungenBisEndeSeite{}
\TRAINER{
$$p(8) = 12$$
$$2 + (30-2) \cdot{} \e^{8\cdot{}q} = 12$$
$$(30-2) \cdot{} \e^{8\cdot{}q} = 10$$
$$28 \cdot{} \e^{8\cdot{}q} = 10$$
$$\e^{8\cdot{}q} = \frac{10}{28}$$
$$8\cdot{}q = \ln(\frac{10}{28})$$
$$q = \ln(\frac{10}{28}) / 8\approx{} -0.1287$$
}



%%%%%%%%%%%%%%%%%%%%%%%%%%%%%%%%%%%%%%%%%%%%%%%%%%%%%%%%%%%%%%%%%%%%%%%%%%%%%%%%%%%%%%%%%%%%%%%%%%%%%%%%%%%%%%%
\bbwActAufgabenNr{} \textbf{Brot}

Frisch gebackenes Brot wird kühl gestellt. Die Umgebungstemperatur
misst zwölf Grad. Nach zehn Minuten beträgt die Temperatur noch achtzig Grad und
nach weiteren zehn Minuten noch 48 Grad.


\begin{bbwAufgabenBlock}
\item Machen Sie eine Skizze zum Temperaturverlauf.
  \TRAINER{Graph}
  
\item Geben Sie eine allgemeine Funktionsgleichung für diese
  Temperaturabnahme an
  \TRAINER{$$f(t) = 12+ b\cdot{} a^t$$
    Punkte einsetzen:
    $$f(10):  80 = 12   + b\cdot{} a^{10}$$
    $$f(20):  48 = 12   + b\cdot{} a^{20}$$
    Zwölf abzählen
    $$f(10):  68 = b\cdot{} a^{10}$$
    $$f(20):  36 = b\cdot{} a^{20}$$
    Die zweite Gleichung durch die erste teilen:
    $$\frac{36}{68} = a^{10}$$
    $$a = \sqrt[10]{\frac{36}{68}} \approx 0.938381$$
    $b$ durch Einsetzen:
    
    $$b = \frac{68}{a^{10}} = \frac{1156}{9}$$
    
  }

\item Wie warm war das Brot anfänglich?

  \TRAINER{$$f(0) = 12+b\cdot{} a^0 = 12 + b = \frac{1264}9 \approx
    140.44 \textrm{ Grad}$$}%% END Trainer
\end{bbwAufgabenBlock}  

\platzFuerBerechnungenBisEndeSeite{}


%%%%%%%%%%%%%%%%%%%%%%%%%%%%%%%%%%%%%%%%%%%%%%%%%%%%%%%%%%%%%%%%%%%%%%%%%%%%%%%%%%%%%%%%%%%%%%%%%%%%%%%%%%%%%%%%
\TALS{%% TALS, da nur CAS Lösung
\bbwActAufgabenNr{} \textbf{Dreipunkte Aufgabe}

Tee kühlt ab und die Temperatur nähert sich der Zimmertemperatur. Die folgenden Temperaturen
wurden gemessen:

\begin{itemize}
\item 52.4 Grad nach 6 Minuten
\item 38.7 Grad nach 10 Minuten
\item 31 Grad nach 14 Minuten
  \end{itemize} 

\begin{bbwAufgabenBlock}
  \item Geben Sie eine allgemeine Funktionsgleichung für diese
    Temperaturabnahme an (Da es verschiedene Zeitabstände sind, ist
    die Wahl von $\tau = 1$ sinnvoll.
    \TRAINER{$f(t) = c + b\cdot{}a^t$}

  \item Wie hoch ist die Umgebungstemperatur?

    \TRAINER{Taschenrechner Ansatz:
      $$f(t) := c + b\cdot{} a^t$$

      Gleichung:

      $$gls := \{f(6)=52.4, f(10)=38.7, f(14)=31\}$$

      Lösen:
      $$solve(gls, c) \Longrightarrow  c\approx 21.12\degre$$
    }
\end{bbwAufgabenBlock}  
}%% END TALS

\newpage
%%%%%%%%%%%%%%%%%%%%%%%%%%%%%%%%%%%%%%%%%%%%%%%%%%%%%%%%%%%%%%%%%%%%%%%%%%%%%%%%%%%%%

\bbwActAufgabenNr{} \textbf{Tonic: Die Differenz zur Sättigung als Einstiegsaufgabe.}

Ein Tonic-Getränk wird bei 5 Grad Celsius aus der Kühlbox genommen.
Nach einer Minute ist das Getränk bereits auf 15 Grad Celsius
aufgewärmt.
Kein Wunder die Außentemperatur $c$ beträgt 35 Grad ($c$ stehe hier
für die Kapazitätsgrenze, \textbf{c}apacity). 

\begin{bbwAufgabenBlock}

\item Geben Sie zum Zeitpunkt $t_0$ (Startzeit) und zum Zeitpunkt
  $t_1$ (Eine Minute später) jeweils die Temperaturdifferenz zur
  Außentemperatur $c$ an.

\TRAINER{Differenz bei $t_0$ = 30 Grad; Differenz bei $t_1$ = 20 Grad}

\item Was geschieht im Laufe der Zeit mit dieser Differenz?

  \TRAINER{Diese sog. Sättigungsdifferenz nimmt ab und geht gegen
    0. Es ist anzunehmen, dass die Temperaturdifferenz exponentiell zerfällt.}

\item Die Temperaturdifferenz zerfällt exponentiell und geht gegen 0.
  Wie lautet die Formel ($d(t)=...$) für die Temperatur\textbf{differenz}? Tipp:
  Zeichnen Sie in einem Koordinatensystem die Funktion $d(t)$ der
  Temperaturdifferenz auf.

  \TRAINER{$$d(t) = 30 \cdot{} \left(\frac{20}{30}\right)^t$$
  Probe: $d(0) = 30$ und $d(1) = 20$.}

\item Nachdem Sie nun die Formel für die Temperaturdifferenz ($d(t)$) kennen:
  Sind Sie in der Lage eine Formel für die Temperatur anzugeben?

  
\TRAINER{
  $$f(t) = 35 - d(t)$$
  somit
  $$f(t) = 35 - 30\cdot{}\left(\frac{20}{30}\right)^t$$
  Probe:
  $$f(0) = 35 - 30\cdot{}1 = 5$$
  $$f(1) = 35 - 30\cdot{}\left(\frac{20}{30}\right)^1 = 35 - 20 = 15$$
}

\end{bbwAufgabenBlock}


\platzFuerBerechnungenBisEndeSeite{}

%%%%%%%%%%%%%%%%%%%%%%%%%%%%%%%%%%%%%%%%%%%%%%%%%%%%%%%%%%%%%%%%%%%%%%%%%%%%%%%%%
