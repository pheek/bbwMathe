%% Leere Koordinatensysteme, um Zeit zu sparen

\input{bbwLayoutPage}
\renewcommand{\bbwAufgabenBlockID}{Finv}

\renewcommand{\metaHeaderLine}{Funktionen}
\renewcommand{\arbeitsblattTitel}{Umkehrfunktionen}

\begin{document}

\arbeitsblattHeader{}

Kehren Sie die folgende Funktion $f$ auf Ihrem Definitionsbreich
$\DefinitionsMenge{}$ um, und geben Sie auch den Definitionsbereich der
Umkehrfunktion $f^{-1}$ an:


\begin{bbwAufgabenBlock}
\item $f(x) = 7x-1; \DefinitionsMenge{} = [1;5[$
$\Longrightarrow \LoesungsRaumLang{
f^{-1}: y=\frac17 x + \frac17; \DefinitionsMenge{}_{f^{-1}} = [6;34[
}$\abplz{6}

\item $f:  y = -\frac17x + 2; \DefinitionsMenge{} = [-1;\pi[$
$\Longrightarrow \LoesungsRaumLang{
f^{-1}: y=-7x+14; \DefinitionsMenge{}_{f^{-1}} = [2-\frac{\pi}{7};\frac{13}{7}[
}$\abplz{6}

\item $f:  y = x^3  + 1; \DefinitionsMenge{} = \mathbb{R}$
$\Longrightarrow \LoesungsRaumLang{
f^{-1}: y=\pm\sqrt[3]{\pm(x-1)}; \DefinitionsMenge{}_{f^{-1}} = \mathbb{R}
}$\abplz{6}\noTRAINER{\newpage}

\item $f:  y = \tan(x); \DefinitionsMenge{} = ]-90^{\circ};90^{\circ}[$ ($x$ im Gradmaß)
$\Longrightarrow \LoesungsRaumLang{
f^{-1}: y=\arctan(x); \DefinitionsMenge{}_{f^{-1}} = \mathbb{R}
}$\abplz{6}

\item $f:  y = -(x-3)^2 + 2; \DefinitionsMenge{} = ]-3;2\pi]$
$\Longrightarrow \LoesungsRaumLang{
f^{-1}: y=3+\sqrt{2-x}; \DefinitionsMenge{}_{f^{-1}} = [-4\pi^2+12\pi-7;2[
\approx{} [-8.78;2[
}$\abplz{6}

\end{bbwAufgabenBlock}
\noTRAINER{\newpage}
%%\platzFuerBerechnungenBisEndeSeite{}


%%%%%%%%%%%%%%%%%%%%%%%%%%%%%%%%%%%%%%%%%%%%%%%%%5

Weitere Aufgaben zu Umkehrung von Funktionen

\begin{bbwAufgabenBlock}
\item Was ist für das Fragezeichen einzusetzen? $x: x \mapsto x+\frac1x$ mit $\DefinitionsMenge{}
= \mathbb{R}\backslash \{0\}$
$$? \mapsto 2.9$$
$\Longrightarrow \LoesungsRaumLang{\LoesungsMenge{} = \{0.4, 2.5\}}$

\item Kehren Sie die Funktion $f$ auf ihrem maximal möglichen
Definitionsbereich um:

$f: y=2-\sqrt{x+3}$

$f^{-1} = ?$


$\Longrightarrow \DefinitionsMenge{}_f = \LoesungsRaumLang{[3;\infty[}$

$f^{-1}: \LoesungsRaumLang{y = (2-y)^2 - 3}$

$\DefinitionsMenge{}_f^{-1} = \LoesungsRaumLang{[-\infty; 2[}$


\item Gegeben: $f: x\mapsto 2-\sqrt{3-x}$

\begin{itemize}
\item Skizzieren Sie $f$!
\item Geben Sie den maximalen Definitions- und Wertebereich von $f$ an!
  $$\Longrightarrow \DefinitionsMenge{}_f = \LoesungsRaumLang{-infty;2}$$
  $$\Longrightarrow \Wertebereich{}_f = \LoesungsRaumLang{-infty;3}$$
\item Geben Sie $f^{-1}$ an!
  $$\Longrightarrow f^{-1}: \LoesungsRaumLang{y=3-(2-x)^2}$$
\item Berechnen Sie den Schnittpunkt von der Graphen von $f$ mit
$f^{-1}$!
  $$\Longrightarrow \textrm{ Schnittpunkt } P=(\LoesungsRaumLang{\frac{3-\sqrt{5}}2}|\LoesungsRaumLang{\frac{3-\sqrt{5}}2})$$

\end{itemize}
\end{bbwAufgabenBlock}

\platzFuerBerechnungenBisEndeSeite{}


%%%%%%%%%%%%%%%%%%%%%%%%%%%%%%%%%%%%%%%%%%%%%%%%%5
%% Freie Seite
\platzFuerBerechnungenBisEndeSeite{}


\end{document}
