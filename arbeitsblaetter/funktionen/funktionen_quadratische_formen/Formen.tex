%% Leere Koordinatensysteme, um Zeit zu sparen

\input{bmsLayoutPage}

\renewcommand{\metaHeaderLine}{quadratische Funktionen: Formen}

\begin{document}

\section{Umwandeln der Formen der quadratischen Funktionen}

Zu den folgenden quadratischen Funktionen ist jeweils immer genau eine Form gegeben: \textbf{Grundform}, \textbf{Nullstellenform}\footnote{Die Nullstellenform ist auch als faktorisierte Form oder als Produktform bekannt.} oder \textbf{Scheitelform}

Ihre Aufgabe ist es, die Parabeln jeweils in den beiden fehlenden Formen auch noch zu notieren.

%%\renewcommand{\arraystretch}{2}

\begin{bbwFillInTabular}{r|p{5cm}|p{5cm}|p{5cm}}

ID  & Grundform & Scheitelform & Nullstellenform\\ \hline

 1  & $y=3x^2+6x-105$                 & \TRAINER{$y=3(x+1)^2-108$}       & \TRAINER{$y=3(x+7)(x-5)$}          \\ \hline
 2  & \TRAINER{$y = -4x^2 +16x + 84$} & \TRAINER{$y=-4(x-2)^2+100$}      & $y=-4(x-7)(x+3)$                   \\ \hline
 3  & \TRAINER{$y =  x^2 + 4x + 3$}   & $y=(x+2)^2 - 1$                  & \TRAINER{$y=(x+1)(x+3)$}           \\ \hline
 4  & $y = -4x^2 -8x -4$              & \TRAINER{$y=-4(x+1)^2$}          & \TRAINER{$y=-4(x+1)(x+1)$}         \\ \hline
 5  & \TRAINER{$y = -x^2-6x-8$}       & $y=-(x+3)^2+1$                   & \TRAINER{$y=-(x+4)(x+2)$}          \\ \hline
 6  & $y = \frac14x^2 -4$             & \TRAINER{$y=\frac14 (x)^2 - 4$}  & \TRAINER{$y=\frac14(x-4)(x+4)$}    \\ \hline
 7  & \TRAINER{$y = -2x^2-2x+24$}     & \TRAINER{$y=-2(x+0.5) + 24.5  $} & $-2(x-3)(x+4)$                     \\ \hline
 8  & \TRAINER{$y = 7x^2-70x+191$}    & $y=7(x-5)^2+16$                  & \TRAINER{Keine Lösung in R.}       \\ \hline
\end{bbwFillInTabular}

\end{document}
