%%%%%%%%%%%%%%%%%%%%%%%%%%%%%%%%%%%%%%%%%%%%%%5

\subsection{Begrenztes Wachstum}






\bbwActAufgabenNr{} \textbf{Sauerstoff}

Ein Patient hat im Blut eine Sauerstoffkonzentration von 70\%. Ziel
ist es, seinen Sauerstoff wieder auf 100\% zu bringen und dazu wird
sein Sauerstoff erhöht.

Zwei Minuten nach der Behandlung hat der Patient bereits 80\%
Sauerstoff seiner möglichen Sättigung erreicht. Gehen Sie von einem
(exponentiell) beschränkten Prozess aus.

\begin{bbwAufgabenBlock}


\item Machen Sie eine Skizze des Sauerstoffgehaltes.

\TRAINER{Graph}
  
\item Geben Sie die Formel $y = f(t)$ an mit $y$ = Sauerstoffgehalt und
$t$ = Zeit in Minuten.

\TRAINER{$$f(t) = 100 - 30\cdot{}\left(\frac{20}{30}\right)^{\frac{t}{2}}$$}
  
\item Wie groß ist sein Sauerstoffgehalt nach 8 Minuten?

\TRAINER{$$f(8) =
  100-30\cdot{}\left(\frac{20}{30}\right)^{\frac{8}{2}}\approx 94.07 \%$$}
  
\item
  Bei 97\% der Sauerstoffsättigung können wir die Behandlung
abbrechen. Wann wird das erreicht sein?

\TRAINER{

  $$97 = 100 - 30 \cdot{} \left( \frac{20}{30} \right)^{\frac{t}{2}}$$
  $$-3  = - 30 \cdot{} \left(\frac{20}{30}\right)^{\frac{t}{2}}$$
  $$3  =  30 \cdot{} \left(\frac{20}{30}\right)^{\frac{t}{2}}$$
  $$\frac{1}{10}  =  \left(\frac{20}{30}\right)^{\frac{t}{2}}$$
  $$\log_{\left(\frac{20}{30}\right)}(\frac{1}{10}) =   \frac{t}{2}$$
  $$2\cdot{} \log_{\left(\frac{20}{30}\right)}(\frac{1}{10}) = t
      \approx  11.36 \text{ min.}$$
}

\end{bbwAufgabenBlock}


\platzFuerBerechnungenBisEndeSeite{}


%%%%%%%%%%%%%%%%%%%%%%%%%%%%%%%%%%%%%%%%%%%%%%%%%%%%%%%%%%%%%%%%%%%%%%%%%%%%%%%%%%%%%%%%%%%%%%%%%%


\bbwActAufgabenNr{} \textbf{Cola}

Eine Cola wird bei $5^\circ \textrm{ C}$ aus der Kühlbox genommen. Die
Umgebungstemperatur ist $32^\circ \textrm{ C}$. Nach 3 Minuten messen
wir bereits eine Temperatur von $9^\circ \textrm{ C}$ 

Wir gehen davon aus, dass die Temperaturdifferenz der Cola zur
Umgebungstemperatur exponentiell abnimmt.

\begin{bbwAufgabenBlock}
\item Machen Sie eine Skizze im Koordinatensystem, welche die
Abhängigkeit von der Cola-Temperatur von der Zeit ($t$) aufzeichnet.

\TRAINER{Graph}

\item Geben Sie eine Funktionsgleichung $f(t)$ an, welche die
Temperatur in Minuten ($t$) nach dem herausnehmen der Cola aus der Kühlbox
angibt.

\TRAINER{$$f(t) = c - b\cdot{} a^{\frac{t}{\tau}}$$

$$f(t) = 32 - 27 \cdot \left( \frac{23}{27}\right)^{\frac{t}{3}}$$
}

\item Wie «warm» ist die Cola nach 10 Minuten?

\TRAINER{
$$f(10) = 32 -
27 \cdot \left( \frac{23}{27}\right)^{\frac{10}{3}} \approx
16.18^\circ \textrm{ C}$$
}%% end TRAINER

\item Nach wie vielen Minuten ist die Cola $18^\circ \textrm{ C}$
«warm»?

\TRAINER{
$$f(t) = 18 = 32 - 27\cdot{} \left(\frac{23}{27}\right)^\frac{t}{3}$$
$$-14 =  - 27\cdot{} \left(\frac{23}{27}\right)^\frac{t}{3}$$
$$\frac{14}{27} =  - \left(\frac{23}{27}\right)^\frac{t}{3}$$
$$t =
3\cdot{} \log_{\left(\frac{23}{27}\right)}\left(\frac{14}{27}\right) \approx
12.29 \textrm{ min.}$$
}%% END TRAINER

\end{bbwAufgabenBlock}


\platzFuerBerechnungenBisEndeSeite{}


%%%%%%%%%%%%%%%%%%%%%%%%%%%%%%%%%%%%%%%%%%%%%%%%%%%%%%%%%%%%%%%%%%%%%%%%%%%%%%%%%%%%%%%%%%%%%%%%%%


\bbwActAufgabenNr{} \textbf{«Silly Blubb»}

Das neue Waschmittel «Silly Blubb» will sich im Markt etablieren.
Dank einer tollen Fernseh- und Internetwerbung nehmen die
Verkaufszahlen rasant zu.

Es ist jedoch zu erwarten, dass nicht mehr als 20\% aller Käufer auf
«Silly Blubb» umschwenken werden.

Nach dem ersten Monat sind bereits 5\% der Waschmittelkäufer auf
«Silly Blubb» umgelenkt worden. Nach zwei weiteren Monaten sind wir
bei 8\% gelandet.


\begin{bbwAufgabenBlock}
\item Machen Sie eine Skizze im Koordinatensystem, welche die
Abhängigkeit von Monat ($x$-Achse) zu Käuferzahl in Prozent darstellt.

\TRAINER{Graph}

\item Geben Sie einen mögliche Funktionsterm $f(t)$ an, welcher die
Prozentzahl der Käufer nach Monaten angibt. Bedenken Sie, dass die
«Sättigungsgrenze» bei 20\% liegt.

\TRAINER{$$f(t) = 20 - 15 \cdot{} (\frac{12}{15})^{\frac{t-1}{2}}$$
}

\item Wie viele Prozente der Käufer sind in Monat 6 nach Verkaufsstart
  bereits «Silly Blubb» Käufer?
  
\TRAINER{
$$y = 20 - 15\cdot{} \left(\frac{12}{15}\right)^{\frac{6-1}{2}} \approx 11.41\% $$%%
}%% end TRAINER

\item Nach wie vielen Monaten sind 16\% der Käufer von «Silly Blubb»
  überzeugt worden?

\TRAINER{
$$f(t) = 16 [\%] = 20 - 15\cdot{} \left(\right)^{\frac{t-1}2}$$
$$ \frac4{15} = \left(\right)^{\frac{t-1}2}$$
$$ \frac4{15} = \left(\right)^{\frac{t-1}2}$$
$$\frac{t-1}2  = \log_{\left(\frac{12}{15}\right)}\left(\frac4{15}\right)$$
$$t = 1 + 2\cdot{}
  \log_{\left(\frac{12}{15}\right)}\left(\frac4{15}\right) \approx
  12.85 \textrm{ Monate}$$
}%% END TRAINER

\end{bbwAufgabenBlock}


\platzFuerBerechnungenBisEndeSeite{}


%%%%%%%%%%%%%%%%%%%%%%%%%%%%%%%%%%%%%%%%%%%%%%%%%%%%%%%%%%%%%%%%%%%%%%%%%%%%%%%%%%%%%%%%%%%%%%%%%%%%%%%%

\bbwActAufgabenNr{} \textbf{Ritter Nimmersatt}

\nextBbwAufgabenNummer{}

Ritter Nimmersatt ist nimmer satt. Erst bei einer Magenfülle von
5 Litern stößt ihm alles wieder auf. Richtig wohlgenährt ist er
normalerweise erst bei einem Magen, der zu 4.5 Liter gefüllt ist.

Er beginnt das Festmahl bei Kunigundes Hochzeit mit einer Magenfülle
von 2 Litern\footnote{Darunter würde er echte Hungerschübe
  leiden!}. In der ersten Minute schafft er es, 3 dl Flüssigkeit oder Nahrung
in sich regelrecht hineinzustopfen. Danach nimmt sein Futtern
«exponentiell gesättigt» ab (bis zur Sättigungsgrenze von 5 Litern, die er
hoffentlich nicht erreicht)\footnote{Beim «normalen» Menschen stellt
  sich das Sättigungsgefühl schlagartiger ein; eine «exponentielle»
  Sättigung wurde hier nur für diese Aufgabe erfunden.}%% end footnote

Nach wie vielen Minuten ist er so richtig gesättigt; m.\,a.\,W. wann
hat er besagte 4.5 Liter im Magen\footnote{Der Moment ist gekommen, wo
  die Wachen den Ritter Nimmersatt vorsorglich unter ominösem Vorwand aus der Burg schaffen sollten.}?

\TNT{16}{
  $c = 5$, $b = m_0 = 5-2=3$, $m=m_1= 5-2.3=2.7$, $\tau = 1$, $a=\frac{2.7}{3} = 0.9$ und somit:
  $$f(t) = 4.5 = 5 - 3\cdot{}\left(\frac{2.7}{3}\right)^t$$

  $$t = log_{\frac{4.5-5}{-3}}(\frac{2.7}3) \approx 17.006$$
Und so ist er nach 17 Minuten so richtig satt.\vspace{52mm}}%% END TNT
\newpage


%%%%%%%%%%%%%%%%%%%%%%%%%%%%%%%%%%%%%%%%%%%%%%%%%%%%%%%%%%%%%%%%%%%%%%%%%%%%%%%%%%%%%%%%%%%%%%%%%%%%%%%%

\bbwActAufgabenNr{} \textbf{WC-Spülung}\par
Ältere Spülkästen für Toiletten funktionieren nach folgendem Prinzip.
Wird die Kordel gezogen, so hebt sich eine
Ansaugglocke, welche das Wasser ansaugt. Jetzt wird aufs Mal durch den
Sog der ganze Kasten entleert.

Beim Wiederauffüllen, und darum gehts bei diesem Modell, wird zunächst
Wasser im vollen Druck der Röhre wieder in den Spülkasten
gepumpt (dies ist vorest ein linearer Prozess). Doch damit es wieder mit dem Füllen aufhört,
ist ein Schwimmer angebracht. Dieser Schwimmer ist immer auf der
Wasseroberfläche und hebt sich beim Füllen des Spülkastens an.

Über eine Stange ist der Schwimmer
mit einem Ventil verbunden, das sich umso mehr schließt, je höher der
Schwimmer steht. Also: Je höher der Pegelstand, umso geschlossener ist
das Ventil. Dies bedeutet aber wiederum, dass das Wasser umso
langsamer einfließt, je höher der Schwimmer steht. Erst bei der
maximalen Füllung von ca. 8 bis 11 Litern ist das Ventil ganz
geschlossen. 

\begin{bbwAufgabenBlock}
\item Skizzieren Sie den Prozess für einen 10-Liter Spülkasten, wenn
  Sie davon ausgehen, dass der Schwimmer erst ab 5 Litern beginnt das
  Ventil zu schließen und dass die ersten 5 Liter mit 3 Litern pro
  Minute gefüllt werden.
  Gehen Sie ab den 5 Litern von einem beschränkten Prozess aus bei dem
  in der ersten Minute (nach den 5 Litern) nur noch 2 Liter (wegen
  der Ventilschließung) in den Spülkasten fließen. Gehen Sie weiter
  davon aus, dass der Kasten bei 10 Litern ganz voll ist und auch wird
  bei 10 Litern das Ventil ganz geschlossen sein (= Sättigungsgrenze).
\TRAINER{Skizze mit zwei Bereichen $f_1$ bis Eine Minute 40 Sekunden
  (= $\frac53$ Minuten) und $f_2$ ab $\frac53$ Minuten; jetzt erst
  beginnt das beschränkte Wachstum.}%% END Trainer
  
\item Geben Sie die beiden Funktionsgleichungen der ersten Teilaufgabe
  an: Erstens die
  lineare Füllung ($f_1$) bis zu den 5 Litern; zweitens das beschränkte
  Wachstum ($f_2$) nach den 5 Litern.
  
  \TRAINER{Bis $\frac53$ Minuten: $$f_1(t) = 3t$$
  Ab $\frac53$ Minuten: $$f_2(t) = 10 -
  5\cdot{}\left(\frac35\right)^{t-\frac53}$$
  Machen Sie unbedingt die Probe für 0', für 1'40'' und für 2' 40''
  um die $t-\frac53$ im Exponenten zu verstehen!}
  
\item Geben Sie für die erste Teilaufgabe an wie viele Liter im
  Spülkasten sind nach
  \begin{enumerate}
    \item einer Minute \TRAINER{ 3 Liter}
    \item zwei Minuten \TRAINER{ $\approx 5.783$ Liter}
    \item zehn Minuten \TRAINER{ $\approx 9.929$ Liter}
  \end{enumerate}

\item Wann sind 9 Liter im Kasten?
  \TRAINER{$$f_2(t) = 9$$
$$ 9 = 10-5\cdot{}\left(\frac35\right)^{t-\frac53}$$
$$ -1 = -5\cdot{}\left(\frac35\right)^{t-\frac53}$$
$$ 1 = 5\cdot{}\left(\frac35\right)^{t-\frac53}$$
$$ 0.2 = \left(\frac35\right)^{t-\frac53}$$
$$ \log_{\frac35}(0.2) = t-\frac53$$
$$ \log_{\frac35}(0.2) + \frac53= t \approx 4.871 \textrm{min.}
    \approx 4'49.04''$$

  }%% END TRAINER
  
\end{bbwAufgabenBlock}

\platzFuerBerechnungenBisEndeSeite{}


%%%%%%%%%%%%%%%%%%%%%%%%%%%%%%%%%%%%%%%%%%%%%%%%%%%%%%%%%%%%%%%%%%%%%%%%


\bbwActAufgabenNr{} \textbf{Kondensatorladung}\par
Wird ein Kondensator mit Kapazität $C$ (in Farad) an einer Spannung
$U_0$ (in Volt) über einen Vorwiderstand $R$ (in Ohm) aufgeladen, so
beschreibt die Spannung im Kondensator $U_C(t)$ ein beschränktes
Wachstum.

Es gilt:

$$U_C(t) = U_0\left(1-\e^{-\frac{t}{R\cdot{}C}}\right)$$

Dabei sind gegeben:

Die Ladungsspannung $U_0 = 9$ V.

Der Vorwiderstand $R = 470$ Ohm.

Die Kapazität $C = 0.002$ Farad.

Die Ladungszeit $t$ wird in Sekunden gemessen.

\begin{bbwAufgabenBlock}
\item Wie viel Ladung ist im Kondensator nach 2 Sekunden?

  $$U_C(2) =
  \LoesungsRaumLang{9\cdot{}\left(1-\e^{-\frac{2}{470\cdot{}0.002}}\right)\approx
  7.9280 \textrm{ V}}$$

\item Wann ist der Kondensator auf 8.9 Volt aufgeladen?

  \TRAINER{
  $$8.9 = 9\cdot\left(1-e^{-\frac{t}{470\cdot{}0.002}}\right)$$

  $$\frac{8.9}{9} = 1 - e^{-\frac{t}{470\cdot{}0.002}}$$

  $$1 - \frac{8.9}{9} = e^{-\frac{t}{470\cdot{}0.002}}$$

  $$ln(1 - \frac{8.9}{9}) = -\frac{t}{470\cdot{}0.002}$$

  $$-ln(1 - \frac{8.9}{9}) = \frac{t}{470\cdot{}0.002}$$

  $$-ln(1 - \frac{8.9}{9})\cdot{}(470\cdot{}0.002) = t $$

  $$t\approx{} 42.30 \textrm{ s}$$
  }%% END Trainer
  
\end{bbwAufgabenBlock}

\platzFuerBerechnungenBisEndeSeite{}



\newpage
