

\section{E-Funktion}
\textit{Die Basis $\e$}

\subsection{Punkte Aufgaben}



\bbwActAufgabenNr{} \textbf{Startwert 100\%}


Eine Exponentialfunktion mit Startwert \textit{eins} (= 100\%) gehe durch den Punkt $A$. Geben Sie
die Funktionsgleichung erstens in der Form $f(x)=a^x$ und
zweitens in der Form $g(x)=\e^{qx}$ an ($\e$ sei die Eulersche Zahl).


\begin{bbwAufgabenBlock}

\item $A=(2|6)$

\TRAINER{$a$: Punkt einsetzen
$$f(2)=6=a^2$$ Wurzel ziehen:
$$a=\sqrt{6} \Longrightarrow  f(x) = (\sqrt{6})^x$$

$q$: Punkt einsetzen
$$f(2)=6=\e^{2q}$$
$$\ln(6) = 2q$$
$$q=\frac{\ln(6)}{2}$$
$$g(x) = \e^{\frac{\ln(6)}{2}\cdot{}x}$$
}%% END Trainer



\item $A=(1|\pi)$

\TRAINER{$a$: Punkt einsetzen
$$\pi = a^1 \Longrightarrow a=\pi$$
$$f(x) = \pi^x$$

$q$: Punkt einsetzen
$$\pi = \e^{q\cdot{} 1}$$

$$\pi = \e^q$$
logarithmieren
$$\ln{\pi} = q$$

$q$ in die Funktionsgleichung einsetzen:

$$g(x) = \e^{\ln{\pi}\cdot{}x}$$

}%% END TRANIER



\item $A=(3|\e^6)$

\TRAINER{$a$: Punkt einsetzen
$$f(3)=\e^6=a^3$$ Wurzel ziehen:
$$a=\sqrt[3]{\e^6} \Longrightarrow  f(x) = \left(\sqrt[3]{\e^6}\right)^x$$

$q$: Punkt einsetzen
$$f(3)=\e^6=\e^{3q}$$
$$\ln(\e^6) = 3q$$
$$6=3q$$
$$q=2$$
$$g(x) = \e^{2x}$$
}%% END Trainer



\item $A=(2.5|\frac{-2}{e})$

\TRAINER{$a$: Punkt einsetzen
$$f(2.5)=\frac{-2}{\e}=a^{2.5}$$
.... in $\mathbb{R}$ nur möglich für $a<0$ und $2.5$ ungerade; doch
2.5 ist nicht ungerade! 
}%% END Trainer



\item $A=(0|1)$

\TRAINER{Dies ist undefiniert: Alle Exponentialfunktionen der Form $f(x) = a^x$
gehen durch den Punkt $(0|1)$.}


\item $A=(x_A|y_A)$

  \TRAINER{$a$: $x_A$-te Wurzel ziehen: $a= \sqrt[x_A]{y_A}$
    \par
    $q$: $$\ln(y_A) = q \cdot{} x_A$$ somit:
    $$q=\frac{\ln(y_A)}{x_A}$$
  }%% END TRANINER
  
\end{bbwAufgabenBlock}


\platzFuerBerechnungenBisEndeSeite{}


%%%%%%%%%%%%%%%%%%%%%%%%%%%%%%%%%%%%%%%%%%%%%%%%%%%%%%%%%%%%%%5

\bbwActAufgabenNr{} \textbf{Zweipunkte Aufgaben}

Eine Exponentialfunktion gehe durch die Punkte $A$ und $B$. Geben Sie
die Funktionsgleichung erstens in der Form $f(x)=b\cdot{} a^x$ und
zweitens in der Form $g(x)=b\cdot{} e^{qx}$ an.

\begin{bbwAufgabenBlock}
\item $A=(0|2)$ und $B=(2|4)$
  \TNT{3}{
    $f(x) = 2\cdot{}(\sqrt{2})^x = 2\cdot 2^\frac{x}2$ oder
    $g(x) = 2\cdot{}e^{\frac{\ln(2)}{2}\cdot{}x} $
  }
\end{bbwAufgabenBlock}


\platzFuerBerechnungenBisEndeSeite{}




%%%%%%%%%%%%%%%%%%%%%%%%%%%%%%%%%%%%%%%%%%%%%%%%%%
\subsection{Basiswechsel-Aufgaben}


\bbwActAufgabenNr{} \textbf{Moos}

Eine Moosart bedeckt in einem Moor mehr und mehr der Fläche. Die
Fläche in $\text{m}^2$ wird in Abhängigkeit der Zeit $t$ [in Jahren]
wie folgt angegeben:

$$f(t) = 30\cdot{}\e^{1.79179\cdot{}t} $$

\begin{bbwAufgabenBlock}

\item Geben Sie die Funktion in der Form $f(t) = b\cdot{}a^t$ an, mit
  geeignetem $a$ und $b$. Dabei ist $a$ der Wachstumsfaktor in Jahren
  (auch die Zeiteinheit $t$ soll in Jahren angegeben werden).

  \TRAINER{$$f(0):   30\cdot{}e^{1.79179\cdot{}0} = b\cdot{}a^0$$
    somit
    $$30 = b$$
    Und: für $t=1$:
    $$f(t) = 30\cdot{}e^{1.79179} = b\cdot{}a^1$$
    Durch $b=30$ teilen:
    $$e^{1.79179} = a$$
    und somit
    $$a\approx 6.000$$
    $$f(t) = 30\cdot{}6.000^t$$
  }
\end{bbwAufgabenBlock}
\TNTeop{}

\newpage

\bbwActAufgabenNr{} \textbf{Algen}

Eine Algenplage verdopple sich alle 7 Stunden.

\begin{bbwAufgabenBlock}

\item Um wie viel nimmt die Algenplage pro Tag (=24 Stunden) zu?

\TRAINER{$$f(24) = b\cdot{}2^{\frac{24}{7}}$$ Der Vervielfachungsfaktor pro Tag ist also $\approx 10.77$}


\item Wie lautet die Funktionsgleichung $f(t)=b\cdot{}a^t$, wenn $t$ in Tagen, und nicht in Stunden gerechnet wird?

\TRAINER{$f(t) = b\cdot{} 10.77^t$}

\item Nach wie vielen Minuten hat die Algenplage um 2\% zugenommen?

\TRAINER{Erst mal in Minuten angeben (7 Stunden = 420 Minuten) :
$$f(t) = b\cdot{} 2^{\frac{t}{420}}$$
Faktor 1.02 = 2\%
$$1.02\cdot{}b = b\cdot{} 2^{\frac{t}{420}} | \text{ durch } b \text{ teilen.}$$
$$1.02 = 2^{\frac{t}{420}}$$
$$\log_2(1.02) = \frac{t}{420}$$
$$420 \cdot{} \log_2(1.02) = t$$
$$t\approx{} 12.00 \textrm{min}$$

}

\item Finden Sie ein passendes $q$, sodass die Algenplage mit der Funktion $f(t) = b\cdot{}\e^{qt}$ mit $t$ in Stunden angegeben werden kann. $\e$ ist hier die Eulersche Zahl 2.7182818284590.

\TRAINER{
$$b\cdot{} 2^{\frac{t}{7}} = b\cdot e^{qt}$$
$$2^{\frac{t}{7}} =e^{qt}$$
$t$-te Wurzel ziehen:
$$2^{\frac{1}{7}} =e^{q}$$
Logarithmieren zur Basis $\e$ ($\ln()$):
$$q  = \ln(2^{\frac17}) \approx 0.099021$$
}


\end{bbwAufgabenBlock}


\platzFuerBerechnungenBisEndeSeite{}

%%%%%%%%%%%%%%%%%%%%%%%%%%%%%%%%%%%%%%%%%%%%%%%%%%%%%%%%%%%%%%%%%%%%%%%%%%

\newpage
