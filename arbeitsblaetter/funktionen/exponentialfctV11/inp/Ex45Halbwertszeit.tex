
\subsection{Halbwertszeit / Verdopplungszeit}


\bbwActAufgabenNr{} \textbf{Radioaktiver Stoff}

Ein radioaktiver Stoff zerfällt jedes Jahr um elf Prozent. Anfänglich ist ein kg des Stoffes vorhanden.

\begin{bbwAufgabenBlock}
\item Wie viel vom Stoff ist nach vier Jahren noch übrig?
\TRAINER{$0.89^4 \textrm{kg} \approx 62.74\%$}
\item Wie viel vom Stoff ist nach $n$ Jahren noch übrig?
\TRAINER{$0,89^n \textrm{kg}$}
\item Nach wie vielen Jahren wird noch 50\% des Stoffes übrig bleiben? (Diese Zeitspanne nennt man die Halbwertszeit $T_2$.)
\TRAINER{$T_2 = n = \log_{0.89}(0.5)\approx 5.948$ Jahre.}
\end{bbwAufgabenBlock}
\platzFuerBerechnungenBisEndeSeite{}



%%%%%%%%%%%%%%%%%%%%%%%%%%%%%%%%%%%%%%%%%%%%%%%%%%

\bbwActAufgabenNr{} \textbf{Halbwertszeit}

Tritium zerfällt innerhalb von 1.873 Jahren auf 90\%  (also um 10\%).


\begin{bbwAufgabenBlock}
\item Erstellen Sie eine Skizze, welche den Verlauf der vorhandenen
  Menge in \% angibt. Starten Sie mit 100\%.

  \TRAINER{Graph}

\item Geben Sie eine mögliche Funktionsgleichung für den Tritiumgehalt
  (in \%)
  in Abhängigkeit der Zeit (in Jahren) an.

  \TRAINER{$$f(t) = 100\% \cdot{} 0.9^{\frac{t}{1.873}}$$}
  
\item  
  Wie groß ist die Halbwertszeit von Tritium?

  \TRAINER{$$0.5 = (0.9)^\frac{t}{1.873}$$

   $$t = 1.873 \cdot{} \log_{0.9}(0.5) \approx 12.32 \textrm{ Jahre}$$
  }
\end{bbwAufgabenBlock}
\platzFuerBerechnungenBisEndeSeite{}

%%%%%%%%%%%%%%%%%%%%%%%%%%%%%%%%%%%%%%%%%%%%%%%%%%

\bbwActAufgabenNr{} \textbf{Verdopplungszeit}

E. coli-Bakterien (Benannt nach Th. Escherich) durchlaufen bei
optimalen Bedingungen pro Tag (24 h) sechs (bis sieben) Generationen. Das
bedeutet, dass sich ein einziges Bakterium sechs mal teilt und sich
somit die Anzahl sechs mal verdoppelt.

\begin{bbwAufgabenBlock}

\item Machen Sie eine aussagekräftige Skizze zur Vermehrung dieser Bakterien
  \TNT{6}{}
  
\item Gehen Sie zunächst von sechs Generationen innerhalb von 24 h aus
  und geben Sie eine Funktion an, welche die Anzahl Bakterien in
  Abhängigkeit der Zeit (in Stunden) angibt. Gehen Sie von einer
  Startpopulation $f(0) = \text{G}_0$ aus.

  
  \TNT{2}{$f(0) = \text{G}_0 \cdot{} (2^6)^\frac{1}{24} = \text{G}_0
    \cdot{} 2^\frac{1}{4}$; $b$ = $\text{G}_0$; $a=2$ und $\tau=4$;
    bzw. $a=2^6$ und $\tau=24$}%% END TNT
  
  
\item Geben Sie die Zunahme $a_h$ pro Stunde an für (1.) 6 Generationen
  und (2.) sieben Generationen pro Tag

  \TNT{3.2}{
    (1.) $a_h = 2^\frac14 \approx 1.1892$

    (2.) $a_h = 2^\frac7{24} \approx 1.2241$

  }%% end TNT
 

  
\item Wie ist die durchschnittliche Verdopplungszeit, wenn Sie von 6.5 Generationen pro
  Tag im Schnitt ausgehen?
  
  \TNT{4}{$$f(t) = \text{G}_0\cdot 2^{\frac{6.5t}{24}} \Longrightarrow$$
$$2\cdot{}\text{G}_0 = \text{G}_0 \cdot{} 2^\frac{6.5\text{T}_2}{24}
    \Longrightarrow$$
    $$\text{T}_2 = \log_{2^{\frac{6.5}{24}}}(2) \approx 3.6923 \text{h}$$
  }

\end{bbwAufgabenBlock}
%%\platzFuerBerechnungenBisEndeSeite{}
\newpage


\bbwActAufgabenNr{} \textbf{Düngemittel}

Ein Düngemittel in einem See habe eine Halbwertszeit von acht
Monaten. Das heißt, nach acht Monaten ist jeweils noch die Hälfte der
Düngemittelkonzentration im See vorhanden.

\begin{bbwAufgabenBlock}

\item Zeichnen Sie einen Graphen, der die Düngemittelkonzentration (in \%)
  in Abhängigkeit von der Zeit (in Monaten) angibt.

  \TRAINER{Graph}

  
    \item Entwerfen Sie einen Funktionsterm, der die
      Düngemittelkonzentration in Abhängigkeit der Zeit ermitteln
      kann.

      \TRAINER{$$f(t) = 100\% \cdot\left(\frac12\right)^{\frac{t}8}$$}%% END Tranien

    \item Nach welcher Zeit sind nur noch 5\% der ursprünglichen
      Düngemittelkonzentration im See?

      \TRAINER{$$5\%  = 100\% \cdot{}
        \left(\frac12\right)^{\frac{t}8}$$
$$t = 8\cdot{} \log_{0.5}(0.05) \approx 34.6 \textrm{ Monate}$$
      }
      
\end{bbwAufgabenBlock}
\platzFuerBerechnungenBisEndeSeite{}

%%%%%%%%%%%%%%%%%%%%%%%%%%%%%%%%%%%%%%%%%%%%%%%%%%%%%%%%%%%

