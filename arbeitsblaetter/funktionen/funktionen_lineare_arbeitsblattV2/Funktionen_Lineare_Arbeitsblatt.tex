%%
%% Meta: TI nSpire Einführung
%%       Ziel: Damit die Grundoperationen damit durchgeführt werden können.
%%             Damit man sich an den Rechner gewöhnt.
%%

\input{bbwLayoutPage}

%%%%%%%%%%%%%%%%%%%%%%%%%%%%%%%%%%%%%%%%%%%%%%%%%%%%%%%%%%%%%%%%%%

\usepackage{amssymb} %% für \blacktriangleright
\renewcommand{\metaHeaderLine}{Lineare Funktionen}
\renewcommand{\arbeitsblattTitel}{\metaHeaderLine{} Arbeitsblatt (V 2.0)}

\begin{document}%%
\arbeitsblattHeader{}

\title{Aufgaben Funktionsbegriff}

\section{Darstellung}
\subsection{Wertetabelle und Graph}
Gegeben ist die Funktion $f$:
$$y=\frac32x-5$$


a) Erstellen Sie eine Wertetabelle für $f$:

\TRAINER{%%
\begin{tabular}{|c|c|c|c|c|c|c|c|c|c|c|}\hline
$-4$ & $-3$ & $-2$ & $-1$ & $0$ & $1$ & $2$ & $3$ & $4$ & $5$ & $6$\\\hline%%
$-11$ & $-9.5$ & $-8$ & $-6.5$ & $-5$ & $-3.5$ & $-2$ & $-0.5$ & $1$ & $2.5$ & $4$\\\hline%%
\end{tabular} 
}%% end TRAINER
\noTRAINER{%%
\begin{tabular}{|c|c|c|c|c|c|c|c|c|c|c|}
$-4$ & $-3$ & $-2$ & $-1$ & $0$ & $1$ & $2$ & $3$ & $4$ & $5$ & $6$\\\hline
 \hspace{7mm} & \hspace{7mm} & \hspace{7mm} & \hspace{7mm} &\hspace{7mm} &\hspace{7mm} &\hspace{7mm} &\hspace{7mm} &\hspace{7mm} &\hspace{7mm} &\hspace{7mm}\,\,\,\,%%
\end{tabular} 
}%% end noTRANIER

b) Zeichnen Sie den Graphen von $f$ ins folgende
Koordinatensystem; natürlich nur im passenden Bereich (Tipps:
Taschenrechner / Grundform):

\bbwGraph{-2}{6}{-7}{2}{%%
\TRAINER{\bbwFunc{\x / 2 * 3 - 5}{-1:4}}
}%% end bbwGraph

c) Geben Sie die beiden Schnittpunkte mit den Achsen an:
$$S_{y\text{Achse}} = \LoesungsRaum{(0|-5)}
\text{ und }
S_{x\text{-Achse}} = \LoesungsRaum{\left(\frac{10}3 \middle| 0\right)}$$
\newpage

\subsection{Skizzieren}
Skizzieren Sie die folgenden Funktionen $f_1$ bis $f_6$:

$f_1$: $y=2x+3$

\bbwGraph{-4}{5}{-3}{5}{
\TRAINER{\bbwFunc{2*\x + 3}{-3:1}}
}


$f_2$: $y=\frac12x-1$

\bbwGraph{-4}{5}{-3}{1}{
\TRAINER{\bbwFunc{0.5*\x  - 1}{-3:4}}
}
\newpage


$f_3$: $y=-x+1.5$

\bbwGraph{-4}{5}{-3}{5}{
\TRAINER{\bbwFunc{-\x  +1.5}{-3:4}}
}



$f_4$: $y=\frac23x-3$

\bbwGraph{-4}{6}{-5}{2}{
\TRAINER{\bbwFunc{\x*2/3 -3}{-3:5}}
}
\newpage


$f_5$: $2x=2y-4$ \TRAINER{$\Longrightarrow  y=x+2$}

\bbwGraph{-5}{5}{-2}{5}{
\TRAINER{\bbwFunc{\x+2}{-4:3}}
}



$f_6$: $\frac{2-x}3 = \frac{y-x}6$ \TRAINER{$\Longrightarrow  y=-x+4$}

\bbwGraph{-3}{5}{-2}{6.5}{
\TRAINER{\bbwFunc{-\x  +4}{-2:5}}
}
\newpage

\section{$y$-Achsenabschnitt (= Ordinatenabschnitt)}

\subsection{$y$-Achsenabschnitt rechnen}

Geben Sie jeweils den $y$-Achsenabschnitt an:

a) $y=3x-2$ $\Longrightarrow$ $y$-Achsenabschnitt =
$\LoesungsRaum{-2}$

\noTRAINER{\mmPapier{2.4}}

b) $y=4x+\frac12$ $\Longrightarrow$ $y$-Achsenabschnitt =
$\LoesungsRaum{\frac12}$

\noTRAINER{\mmPapier{3.2}}

c) $2(x+y+1) = y-4x+3$ $\Longrightarrow$ $y$-Achsenabschnitt =
$\LoesungsRaum{1}$

\TNTeop{
Umformen in Grundform:
$$2(x+y+1) = y-4x+3$$
$$2x+2y+2 = y-4x+3$$
$$2x+2y-1 = y-4x$$
$$6x+2y-1 = y$$
$$6x+y-1 = 0$$
$$6x+y= 1$$
$$y= -6x + 1$$
}
%% implicit: \newpage

\subsection{Funktionsgleichung finden}
Die Funktion $y=3x+b$ geht durhc den Punkt $(0|7)$.
Berechnen Sie den Parameter $b$ oder mit anderen Worten: Geben Sie die
Funktionsgleichng an:

\TNT{4.8}{$$y=3x+7$$ denn $7$ ist als $y$-Achsenabschnitt gegeben.}
\newpage

\section{Steigung}

\subsection{Steigung von Strecken}
Geben Sie die Steigungen der folgenden Strecken an:

\newcommand{\bbwStrecke}[7]{%%
\bbwDot{#1,#2}{blue}{west}{}\bbwLine{#1,#2}{#3,#4}{blue}\bbwDot{#3,#4}{blue}{west}{}%
\draw (#5,#6) node{{\color{blue}#7)}};
}%%

\bbwGraph{-8}{7}{-6}{6}{%
\bbwStrecke{ 2}{ 3}{ 6}{ 5}{ 4  }{ 4.5}{a}
\bbwStrecke{-6}{ 4}{-3}{ 3}{-4.5}{ 4  }{b}
\bbwStrecke{-7}{-4}{-4}{-4}{-5.5}{-4.5}{c}
\bbwStrecke{-3}{-2}{ 2}{-5}{ 1  }{-4  }{d}
\bbwStrecke{ 0}{-1}{ 4}{ 2}{ 2  }{ 1  }{e}
\bbwStrecke{ 4}{-3}{ 4}{-5}{ 4.5}{-4  }{f}

}%% end BBW Graph

Lösungen:

\TNTeop{
a) $\frac12$; b) $-\frac13$; c) $0$; d) $-\frac35$; e) $\frac34$

Bei f) wäre die Steigung unendlich hoch. Da die Gerade, welche die
Strecke $f$ verlängert aber sowieso keine Funktion ist, entwällt eine
weitere Betrachtung; f) hat keine Lösung.
}
\newpage

\subsection{Geradensteigung und $y$-Achsenabschnitt bestimmen}

%%\bbwGraph{-8}{8}{-6}{5}{}
\noTRAINER{\bbwCenterGraphic{18cm}{img/FunktionsgleichungAblesen.png}}
  \TRAINER{\bbwCenterGraphic{9cm}{img/FunktionsgleichungAblesen.png}}

[challenge: $y$-Achsenabschnitt bei d) e) Tipp: Dreisatz]

\TNTeop{
a) $y = -x + 1$

b) $y = \frac{-2}{3}x - 2$

c) $y = 0x + 2.5 = 2.5$

d) $y = \frac56x + 4\cdot{}\frac56 - 2 = \frac56x + \frac43$

e) $y = \frac{5}{23}x + 5.5\cdot{}\frac5{23}-1 = \frac5{23}x + \frac{9}{46}$
}
%%%%%%%%%%%%%%%%%%%%%%%%%%%%%%%%%%%%%%%%%%%%%%%%%%%%%%%%%%%5

\section{Nullstelle}

\subsection{Steigung, $y$-Achsenabschnitt und Nullstelle bestimmen}
Bestimmen Sie den $y$-Achsenabschnitt $b$, die Nullstelle ($y=0$, d.\,h. $x=\frac{-b}{a}$) und die Steigung $a$ der folgenden Funktionen:

$$y = a\cdot{}x + b$$


\begin{bbwFillInTabular}{l|c|c|c}
 Funktionsgleichung  & $y$-Achsenabschnitt $b$ & Nullstelle $\frac{-b}{a}$ & Steigung $a$\\\hline
 
$y=3x + 3$ & \LoesungsRaum{3} & \LoesungsRaum{$-1$} & \LoesungsRaum{3} \\\hline

$y=4x-2$ & \LoesungsRaum{-2} & \LoesungsRaum{$\frac{1}{2}$} & \LoesungsRaum{4} \\\hline

$y=-\frac{1}{2}x + 1.5$ & \LoesungsRaum{1.5} & \LoesungsRaum{$\frac{1.5}{\frac{1}{2}}=3$} & \LoesungsRaum{$-\frac{1}{2}$} \\\hline

$y=-\frac{2}{5}x - 3.5$ & \LoesungsRaum{-3.5} & \LoesungsRaum{-8.75} & \LoesungsRaum{$-\frac{2}{5}$} \\\hline

$y=1.8x$ & \LoesungsRaum{0} & \LoesungsRaum{0} & \LoesungsRaum{1.8} \\\hline

$y=-3x$ & \LoesungsRaum{0} & \LoesungsRaum{0} & \LoesungsRaum{-3} \\\hline

$y=3.7$ & \LoesungsRaum{3.7} & \LoesungsRaum{keine Nullstelle} & \LoesungsRaum{0} \\\hline

$y=0$ & \LoesungsRaum{0} & \LoesungsRaum{alle $x$ sind Nullstelle} & \LoesungsRaum{0} \\\hline

$y=-x$ & \LoesungsRaum{0} & \LoesungsRaum{0} & \LoesungsRaum{-1} \\\hline

$5y-3 = 2x+7$ & \LoesungsRaum{2} & \LoesungsRaum{$-5$} & \LoesungsRaum{$\frac25$} \\\hline

\end{bbwFillInTabular}
\newpage


\noTRAINER{

Hilfsblatt

\bbwGraph{-7}{7}{-4}{5}{}

\bbwGraph{-7}{7}{-4}{5}{}
\newpage
}%% END noTRAINER

\subsection{Funktionsgleichung finden}

Bestimmen Sie die Funktionsgleichung $y=ax+b$ (und die anderen
fehlenden Werte):

\begin{bbwFillInTabular}{c|c|c|l}
 $y$-Achsenabschnitt $b$ & Nullstelle $\frac{-b}{a}$& Steigung $a$& Funktionsgleichung $y=...$\\
\hline

-2 & \LoesungsRaum{$\frac{2}{3}$} & 3 & \LoesungsRaum{$y=3x-2$}\\
\hline

1.5 & \LoesungsRaum{$\frac{15}{17}$} & -1.7 & \LoesungsRaum{$y=-1.7x + 1.5$}\\
\hline

0 & \LoesungsRaum{$0$} & 1.4 & \LoesungsRaum{$y=1.4x$}\\
\hline

\LoesungsRaum{4} & -4 & 1 & \LoesungsRaum{$y=x+4$}\\
\hline

\LoesungsRaum{-6} & -3 & -2 & \LoesungsRaum{$y=-2x-6$}\\
\hline

2 & $\frac{1}{5}$ & \LoesungsRaum{-10} & \LoesungsRaum{$y=-10x + 2$}\\
\hline

5 & \LoesungsRaum{keine Nullstelle}& 0 & \LoesungsRaum{$y=5$}\\
\hline

\LoesungsRaum{0} & 0 & $-\frac{3}{2}$ & \LoesungsRaum{$y=-\frac{3}{2}x$}\\
\hline

0 & -1.5 & \LoesungsRaum{0} & \LoesungsRaum{$y=0$}\\
\hline


0 & \LoesungsRaum{Alle $x\in\mathbb{R}$ sind Nullstelle} & 0 & \LoesungsRaum{$y=0$}\\  %
\hline

0 & 0 & \LoesungsRaum{$\mathbb{R}$} & \LoesungsRaum{$y=0$ oder $y=7x$
oder $y=-22.6x$ }\\  %
\hline

\end{bbwFillInTabular}
\newpage


\noTRAINER{
Hilfsblatt

\bbwGraph{-7}{7}{-4}{5}{}

\bbwGraph{-7}{7}{-4}{5}{}
\newpage
}%% END noTRAINER
\newpage


\subsection{Schnittpuknt mit Achsen}
Bestimmen Sie die Schnittpunkte der Geraden
$$y-2x=3\left(x+\frac23\right)$$
mit den Koordinatenachsen!


Schnittpunkt mit der $x$-Achse: \LoesungsRaum{$(-\frac25|0)$}

\vspace{5mm}
Schnittpunkt mit der $y$-Achse: \LoesungsRaum{$(0|2)$}

\TNTeop{
Tipp: Erst in Form bringen: In Grundform:

$$y-2x=3x+2$$
$$y=5x+2$$
Hier erhalten wir sofort den Schnittpunkt mit der $y$-Achse
(Achsenabschnitt = 2)

Nun nach $y=0$ setzen und nach $x$ auflösen:

$$0=5x+2$$
$$-2=5x$$
}
%%%%%%%%%%%%%%%%%%%%%%%%%%%%%%%%%%%%%%%%%%%%%%%%%%%%%%%%%%%%%%%%55

\subsection{Funktionsgleichung finden}



\section{Schnittpunkt(e)}

\section{Parallele}

\section{Punkte-Aufgaben}

\end{document}
