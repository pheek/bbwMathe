%%
%% Meta: TI nSpire Einführung
%%       Ziel: Damit die Grundoperationen damit durchgeführt werden können.
%%             Damit man sich an den Rechner gewöhnt.
%%

\input{bbwLayoutPage}

%%%%%%%%%%%%%%%%%%%%%%%%%%%%%%%%%%%%%%%%%%%%%%%%%%%%%%%%%%%%%%%%%%

\usepackage{amssymb} %% für \blacktriangleright
\renewcommand{\metaHeaderLine}{Koordinatensystem}
\renewcommand{\arbeitsblattTitel}{\metaHeaderLine{} Arbeitsblatt (V 1.0)}

\begin{document}%%
\arbeitsblattHeader{}

\section{Aufgaben}

\subsection{Quadranten}
In welchen Quadranten liegen die folgenden Punkte?

a) $(1|2.5)$  \LoesungsRaum{1.}

b) $\left(7|\frac18\right)$  \LoesungsRaum{1.}

c) $\left(-8|-2\right)$  \LoesungsRaum{3.}

d) $\left(2|-3\right)$  \LoesungsRaum{4.}

d) $\left(-4|0\right)$  \LoesungsRaum{in keinem bzw. im 2. und
3. gleichzeitig. Man sagt: «Der Punkt liegt auf der $x$-Achse»}

\noTRAINER{
\bbwGraph{-8}{8}{-6}{6}{}
}
\newpage
\subsection{Streckenzug}
Verbinden Sie die folgenden Punkte in alphabetischer Reihenfolge:

$A=( 1| 1)$, $B=( 4| 1)$, $C=( 4| 3)$, $D=( 7| 0)$, 
$E=( 4|-3)$, $F=( 4|-1)$, $G=( 1|-1)$, $H=( 1|-5)$, 
$I=(-1|-5)$, $J=(-1|-1)$, $K=(-4|-1)$, $L=(-4|-3)$, 
$M=(-7| 0)$, $N=(-4| 3)$, $O=(-4| 1)$, $P=(-1|-1)$, 
$Q=(-7| 4)$, $R=( 1| 4)$, $S=A$

\noTRAINER{
\bbwGraph{-8}{8}{-6}{6}{}
}
\TRAINER{\bbwCenterGraphic{72mm}{img/SBBFake.png}}
\newpage%%
%%
\subsection{Mittelpunkt, Streckenlänge}
Zeichnen Sie die folgenden drei Punkte in das gegebene kartesische
Koordinatensystem:

$$A=(6|3), B=(-1|4), C=(-2|-3)$$

\bbwGraph{-3}{7}{-3}{4}{%%
\TRAINER{
\bbwDot{6,3}{red}{west}{A}
\bbwDot{-1,4}{red}{east}{B}
\bbwDot{-2,-3}{red}{east}{C}
}%% END Trainer
}%% END bbwGraph

a) Berechnen Sie den Mittelpunkt $M$ der Strecke $\overline{AC}$: $M=(\LoesungsRaum{2}|\LoesungsRaum{0})$

\noTRAINER{\mmPapier{0.8}}

b) Berechnen Sie den Mittelpunkt $N$ der Strecke $\overline{AB}$: $N=(\LoesungsRaum{2.5}|\LoesungsRaum{3.5})$

\noTRAINER{\mmPapier{1.2}}

c) Berechnen Sie den Mittelpunkt $P$ der Strecke $\overline{DE}$ mit
$D=(D_x|D_y)$ und $E=(E_x|E_y)$:

\noTRAINER{\mmPapier{1.2}}

$$P = (\LoesungsRaumLang{\frac{D_x+E_x}2}|\LoesungsRaumLang{\frac{D_y+E_y}2})$$

d) Berechnen Sie die Länge der Strecke $\overline{AC}$: \LoesungsRaum{10}$e^2$

\noTRAINER{\mmPapier{1.2}}

e) Berechnen Sie die Länge der Strecke $\overline{CB}$: \LoesungsRaum{$\sqrt{50}\approx{}7.071$}$e^2$
%%
\newpage%%
%%
\subsection{Dreieck}
Zeichnen Sie das Dreieck $\Delta{}ABC$ ins untenstehende
Koordinatensystem:\\
$A=(7|2), B=(5|6), C=(-1|4)$

\bbwGraph{-7}{7}{-6}{8}{
\TRAINER{
\bbwDot{7,2}{red}{west}{A}
\bbwDot{5,6}{red}{west}{B}
\bbwDot{-1,4}{red}{east}{C}
\bbwLine{-2,8}{7,-1}{blue}
}%% end TRAINER
}%% end bbwGraph

a) Spiegeln Sie das Dreieck $\Delta ABC$ am Koordinatenursprung und
geben Sie die Bildkoordinaten $A'$, $B'$ und $C'$ an:

$A' = (\LoesungsRaum{-7}|\LoesungsRaum{-2})$, 
$B' = (\LoesungsRaum{-5}|\LoesungsRaum{-6})$,\\
$C' = (\LoesungsRaum{1}|\LoesungsRaum{-4})$

b) Zeichnen Sie die Gerade $g$, welche durch die Punkte $P=(5|1)$ und
$Q=(-2|8)$ geht, ins obige Koordinatensystem ein.

c) Spiegeln Sie das Dreieck $\Delta ABC$ an $g$ aus Aufgabe b). Geben
Sie die neuen Koordinaten der gespiegelten Punkte $A''$, $B''$ und
$C''$ an:

$A'' = (\LoesungsRaum{4}|\LoesungsRaum{-1})$, 
$B'' = (\LoesungsRaum{0}|\LoesungsRaum{1})$,\\
$C'' = (\LoesungsRaum{2}|\LoesungsRaum{7})$
%%
\newpage%%
%%
\subsection{Satz von Pick}
Georg Pick\footnote{Georg Alexander Pick: 1859 - 1942 (im KZ Theresienstadt)} hat einen allgemeingültigen Term
gefunden, mit dem in einer allgemeinen polygonalen Figur (Vieleck) die
Fläche einzig aus Rand- und Innenpunkte ermittelt werden kann.
Der Term lautet
$$A = A(r,i) =  \frac{r}2 + i - 1$$
mit $r$ = Anzahl der Gitterpunkte auf dem Figurenrand und $i$ = Anzahl
Gitterpunkte im innern der Figur.

Dabei dürfen die Ecken des Vielecks natürlich nur auf ganzzahligen
Gitterpunkten liegen und ebenso zählt man nur die ganzzahligen Gitterpunkte

Gleich ein Beispiel:

\begin{tabular}{cp{84mm}}
\raisebox{-80mm}{\bbwGraph{-1}{6}{-1}{6}{
\bbwLine{1,1}{6,1}{blue}
\bbwLine{6,1}{3,2}{blue}
\bbwLine{3,2}{5,4}{blue}
\bbwLine{5,4}{2,5}{blue}
\bbwLine{2,5}{1,1}{blue}
\bbwDot{1,1}{ForestGreen}{west}{}
\bbwDot{2,1}{ForestGreen}{west}{}
\bbwDot{3,1}{ForestGreen}{west}{}
\bbwDot{4,1}{ForestGreen}{west}{}
\bbwDot{5,1}{ForestGreen}{west}{}
\bbwDot{6,1}{ForestGreen}{west}{}
\bbwDot{3,2}{ForestGreen}{west}{}
\bbwDot{4,3}{ForestGreen}{west}{}
\bbwDot{5,4}{ForestGreen}{west}{}
\bbwDot{2,5}{ForestGreen}{west}{}
%% innere Punkte: 
\bbwDot{2,2}{red}{west}{}
\bbwDot{2,3}{red}{west}{}
\bbwDot{3,3}{red}{west}{}
\bbwDot{2,4}{red}{west}{}
\bbwDot{3,4}{red}{west}{}
\bbwDot{4,4}{red}{west}{}
}%% end bbwGraph%%
}%% end Raisbox
& Die abgebildete Figur besitzt zehn Gitterpunkte
auf dem Rand (grün) und sechs innere Gitterpunkte (rot). Somit ist $r=10$ und
 $i=6$. Die Fläche ist somit: $A=\frac{10}{2}+6-1=10[e^2]$
 \end{tabular}

a) Berechnen Sie mit diesem Term die Flächen der folgenden Figuren und
prüfen Sie die Fläche geometrisch nach:

\noTRAINER{\bbwCenterGraphic{15cm}{img/Pick.png}}
\TRAINER{\bbwCenterGraphic{15cm}{img/PickTrainer.png}}

\noTRAINER{\mmPapier{4.4}}
%%
\newpage%%
%%
e) Zeichenen Sie das Fünfeck $(0|4)-(-3|2)-(-2|-2)-(2|-2)-(3|2)$ und zurück zu $(0|4)$ ins
folgende Koordinatensystem und berechnen Sie mit dem Satz von Pick
dessen Fläche:

\bbwGraph{-4}{4}{-3}{5}{
\TRAINER{
\bbwDot{0,4}{red}{west}{}
\bbwDot{-3,2}{red}{west}{}
\bbwDot{-2,-2}{red}{west}{}
\bbwDot{2,-2}{red}{west}{}
\bbwDot{3,2}{red}{west}{}
}%% end TRAINER
}%% end bbwGraph

Die Fläche beträgt: $\LoesungsRaum{24}  [e^2]$

\TRAINER{Fläche = 24, denn $r=8$ und $i = 21$}

\end{document}
