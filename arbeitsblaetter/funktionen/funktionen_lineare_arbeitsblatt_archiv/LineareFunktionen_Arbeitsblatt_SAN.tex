%%
%% Meta: TI nSpire Einführung
%%       Ziel: Damit die Grundoperationen damit durchgeführt werden können.
%%             Damit man sich an den Rechner gewöhnt.
%%

\input{bmsLayoutPage}

%%%%%%%%%%%%%%%%%%%%%%%%%%%%%%%%%%%%%%%%%%%%%%%%%%%%%%%%%%%%%%%%%%

\usepackage{amssymb} %% für \blacktriangleright
\renewcommand{\metaHeaderLine}{Lineare Funktionen}
\renewcommand{\arbeitsblattTitel}{$y$-Achsenabschnitt, Nullstelle, Steigung}

\begin{document}%%
\arbeitsblattHeader{}

\section{Achsenabschnitt, Nullstelle, Steigung}

Bestimmen Sie den $y$-Achsenabschnitt $b$, die Nullstelle ($y=0$, d.\,h. $x=\frac{-b}{a}$) und die Steigung $a$ der folgenden Funktionen:

$$y = a\cdot{}x + b$$


\begin{bbwFillInTabular}{l|c|c|c}
 Funktionsgleichung  & $y$-Achsenabschnitt $b$ & Nullstelle $\frac{-b}{a}$ & Steigung $a$\\\hline
 
$y=3x + 3$ & \LoesungsRaum{3} & \LoesungsRaum{$-1$} & \LoesungsRaum{3} \\\hline

$y=4x-2$ & \LoesungsRaum{-2} & \LoesungsRaum{$\frac{1}{2}$} & \LoesungsRaum{4} \\\hline

$y=-\frac{1}{2}x + 1.5$ & \LoesungsRaum{1.5} & \LoesungsRaum{$\frac{1.5}{\frac{1}{2}}=3$} & \LoesungsRaum{$-\frac{1}{2}$} \\\hline

$y=-\frac{2}{5}x - 3.5$ & \LoesungsRaum{-3.5} & \LoesungsRaum{-8.75} & \LoesungsRaum{$-\frac{2}{5}$} \\\hline

$y=1.8x$ & \LoesungsRaum{0} & \LoesungsRaum{0} & \LoesungsRaum{1.8} \\\hline

$y=-3x$ & \LoesungsRaum{0} & \LoesungsRaum{0} & \LoesungsRaum{-3} \\\hline

$y=3.7$ & \LoesungsRaum{3.7} & \LoesungsRaum{keine Nullstelle} & \LoesungsRaum{0} \\\hline

$y=0$ & \LoesungsRaum{0} & \LoesungsRaum{alle $x$ sind Nullstelle} & \LoesungsRaum{0} \\\hline

$y=-x$ & \LoesungsRaum{0} & \LoesungsRaum{0} & \LoesungsRaum{-1} \\\hline

$5y-3 = 2x+7$ & \LoesungsRaum{2} & \LoesungsRaum{$-5$} & \LoesungsRaum{$\frac25$} \\\hline

\end{bbwFillInTabular}
\newpage


\noTRAINER{

Hilfsblatt

\bbwGraph{-7}{7}{-4}{5}{}

\bbwGraph{-7}{7}{-4}{5}{}
\newpage
}%% END noTRAINER

\section{Funktionsgleichung}
Bestimmen Sie die Funktionsgleichung $y=ax+b$ (und die anderen
fehlenden Werte):

\begin{bbwFillInTabular}{c|c|c|l}
 $y$-Achsenabschnitt $b$ & Nullstelle $\frac{-b}{a}$& Steigung $a$& Funktionsgleichung $y=...$\\
\hline

-2 & \LoesungsRaum{$\frac{2}{3}$} & 3 & \LoesungsRaum{$y=3x-2$}\\
\hline

1.5 & \LoesungsRaum{$\frac{15}{17}$} & -1.7 & \LoesungsRaum{$y=-1.7x + 1.5$}\\
\hline

0 & \LoesungsRaum{$0$} & 1.4 & \LoesungsRaum{$y=1.4x$}\\
\hline

\LoesungsRaum{4} & -4 & 1 & \LoesungsRaum{$y=x+4$}\\
\hline

\LoesungsRaum{-6} & -3 & -2 & \LoesungsRaum{$y=-2x-6$}\\
\hline

2 & $\frac{1}{5}$ & \LoesungsRaum{-10} & \LoesungsRaum{$y=-10x + 2$}\\
\hline

5 & \LoesungsRaum{keine Nullstelle}& 0 & \LoesungsRaum{$y=5$}\\
\hline

\LoesungsRaum{0} & 0 & $-\frac{3}{2}$ & \LoesungsRaum{$y=-\frac{3}{2}x$}\\
\hline

0 & -1.5 & \LoesungsRaum{0} & \LoesungsRaum{$y=0$}\\
\hline


0 & \LoesungsRaum{Alle $x\in\mathbb{R}$ sind Nullstelle} & 0 & \LoesungsRaum{$y=0$}\\  %
\hline

0 & 0 & \LoesungsRaum{$\mathbb{R}$} & \LoesungsRaum{$y=0$ oder $y=7x$
oder $y=-22.6x$ }\\  %
\hline

\end{bbwFillInTabular}
\newpage


\noTRAINER{
Hilfsblatt

\bbwGraph{-7}{7}{-4}{5}{}

\bbwGraph{-7}{7}{-4}{5}{}
\newpage
}%% END noTRAINER

\end{document}
