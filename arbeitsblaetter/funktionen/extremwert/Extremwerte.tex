%%
%% Arbeitsblatt zu Exponentialfunktionen, Wachstum, Zerfall und Sättigung.

\input{bmsLayoutPage}

\renewcommand{\metaHeaderLine}{Extremwertaufgaben}
\renewcommand{\arbeitsblattTitel}{Extremwertaufgaben}

\begin{document}
\arbeitsblattHeader{}

Weitere Aufgaben im \href{https://olat.bbw.ch/auth/RepositoryEntry/572162090/CourseNode/109695301829998}{Skript $\varphi$ Stereometrie}:

\begin{itemize}
\item Himmelslaterne
\item Blechdose
\end{itemize}

\newpage

\section{Quadratische Funktionen}
Die folgenden Aufgaben sind typischerweise mit dem Finden des
Scheitelpunktes zu lösen.

\subsection{Wurfparabel}
Eine Wurfparabel eines Diskus beschreibe den folgenden Weg ($y$ = Höhe über Boden;
$x$ = Distanz vom Abschuss).
$$y =-0.027 x^2 + 1.03x + 1.6$$

An welcher Stelle ist der Diskus am höchsten (Extremstelle) und wie
hoch ist er dort (Extremwert).

\TNTeop{Extremstelle $x=\frac{512}{27} \approx 19.074$ und Extremwert
$y \approx 12.50160 $
}
%%%%%%%%%%%%%%%%%%%%%%%%%%%%%%%%%%%%%%%%%%%%%%%%%%%%%%%%%%%%%%%%%%%%%%%
\subsection{Grenzen}
Die folgende Funktion $f$ habe den Definitionsbereich
$\mathbb{D}=[-2;3]$.

$$f(x) = \frac25\cdot{}x^2 - \frac12\cdot{}x + 3$$

Bestimmen Sie:

a) Minimalstelle und Minimalwert

b) Wertebereich

c) Maximalstelle und Maximalwert

\TNTeop{
a) Minimalstelle bei 0.625, Minimalwert bei 2.84375

b) Es werden Werte zwischen 2.84375 und 5.6 angenommen. Somit gilt
$$\mathbb{W} = [2.84375; 5.6]$$

c) $f(-2) = \frac{28}{5} = 5.6$ und $f(3) = \frac{51}{10} = 5.1$
Somit ist -2 die Maximalstelle und 5.6 der Maximalwert.

}%% end TNTeop

%%%%%%%%%%%%%%%%%%%%%%%%%%%%%%%%%%%%%%%%%%%%%%%%%
\subsection{400m Bahn}

Eine ebene 400 m-Bahn soll so angelegt werden, dass sie ein Rechteck mit zwei
Halbkreisen begrenzt, wobei die Halbkreise den gegenüberliegenden Rechtecksseiten
angesetzt sind. Wie groß muss der Kreisradius sein und wie lang ein gerades Stück
zwischen den Kurven, wenn das Rechteck maximalen Flächeninhalt haben soll?


\TNTeop{
Radius: 31.83 m und gerades Stück: 100 m.

Frommenwiler Algebra S. 192 Aufg. 720
}
%%%%%%%%%%%%%%%%%%%%%%%%%%%%%%%%%%%%%%%%%%%%%%%%%%%%%%%%%55
\subsection{rechtwinkliges Dreieck}

Einem rechtwinkligen Dreieck mit den Katheten $a$ und $b$ ist ein Rechteck von
maximalem Flächeninhalt einzubeschreiben.

\bbwCenterGraphic{70mm}{RechteckInDreieck.png}

Das Rechteck soll dabei so liegen, dass
eine Ecke mit dem rechten Winkel des Dreiecks zusammenfällt.
Berechnen Sie Länge und Breite dieses Rechtecks.

\TNTeop{
Lösung: Rechtecksseiten $\frac{a}2$ und $\frac{b}2$

Frommenwiler Algebra S. 193 Aufg. 721
}%% end TNTeop


%%%%%%%%%%%%%%%%%%%%%%%%%%%%%%%%%%%%%%%%%%%%%%%%%%%%%
\subsection{Rechteck entlang einer Wand}

Mit einem Seil der Länge s soll an einer Hauswand eine möglichst große Rechtecksfläche abgegrenzt werden, wobei eine Seite des Rechtecks durch die Wand gebildet
wird. Bestimmen Sie Länge und Breite des Rechtecks.

\TNTeop{
Lösung: Länge 0.5s und Breite 0.25s

Frommenwiler Algebra S. 193 Aufg 222
}

%%%%%%%%%%%%%%%%%%%%%%%%%%%%%%%%%%%%%%%%%%%%%%%%%%%%%%%%%
\subsection{Weidefläche in der Ecke}

Mit einem Elektrozaun soll eine möglichst große rechteckige
Weidefläche abgesteckt werden.
Die Weidefläche wird von zwei Mauern und 50 m
Elektrozaun begrenzt.
Wie
müssen Länge und Breite des Rechtecks gewählt werden?

\bbwCenterGraphic{70mm}{WeideflaecheImEck.png}

\TNTeop{
Das Quadrat hat maximale Fläche: Breite = Länge = 25 m.

Marthaler Algebra S. 282 Aufg. 57. b)
}
%%%%%%%%%%%%%%%%%%%%%%%%%%%%%%%%%%%%%%%%%%%%%%%%%%%%%%%%%%
\subsection{Weidefläche in der verkürzten Ecke}

Mit einem Elektrozaun soll eine möglichst große rechteckige
Weidefläche abgesteckt werden.

Die Weidefläche wird von zwei Mauern
mit $a=6 \text{ m}$ und $50 \text{ m}$
Elektrozaun begrenzt.

Wie müssen Länge und Breite des Rechtecks gewählt werden?

\bbwCenterGraphic{70mm}{WeideflaecheTeileck.png}

\TNTeop{
Das Rechteck maximale Fläche bei 14 m  x 28 m

Marthaler Algebra S. 282 Aufg. 57. c)
}
%%%%%%%%%%%%%%%%%%%%%%%%%%%%%%%%%%%%%%%%%%%%%%%%%%%%%%%%%%
\subsection{Rechteckige Platte}
Von einer rechteckigen Platte mit den Seiten 120 cm und 90 cm ist an
einer Ecke ein rechtwinkliges Dreieck so abgebrochen, dass die längere
Seite des Rechtecks um 30 cm und die kürzere um 20 cm verkleinert
wird. Durch zwei Schnitte, die je parallel zu den bisherigen Seiten
verlaufen, soll nun ein Rechteck mit maximalem Flächeninhalt hergestellt werden.
Wie groß sind Länge und Breite des neuen Rechtecks?

\TNTeop{
Lösung: Länge 112.5 cm und Breite = 75 cm

Frommenwiler Algebra S. 193 Aufg. 723
}% end TNTeop
%%%%%%%%%%%%%%%%%%%%%%%%%%%%%%%%%%%%%%%%%%%%%%%%%%
\subsection{Trapezförmige Tischplatte}
Aus einer trapezförmigen Tischplatte mit
$a = 0.8 \text{ m}$, $b = 1.2 \text{ m}$ und $c = 0.7 \text{ m}$ soll eine
rechteckige mit der Länge $x$ heraus gesägt
werden. Wie muss $x$ gewählt werden, dass
die Fläche der rechteckigen Platte möglichst
groß wird? Wie breit ist das Rechteck? Welcher
Anteil (in Prozent) der ursprünglichen Trapezfläche entfällt auf die größte Rechtecksfläche?

\bbwCenterGraphic{80mm}{TrapezfoermigeTischplatte.png}

\TNTeop{
$x = 0.6 \text{ m}$; $h = 0.686 \text{ m}$ dies sind dann $60.5\%$ der
ursprünglichen Trapezfläche.

Marthaler Algebra Seite 283 Aufg 61.
}%%

%%%%%%%%%%%%%%%%%%%%%%%%%%%%%%%%%%%%%%%%%%%%%%%%%%%%%%%%%%%
\subsection{Giebeldachfenster}

Im Dachgeschoss eines Hauses soll ein Atelier
mit möglichst viel Tageslicht eingerichtet
werden. Im Hausgiebel mit der Grundlinie
g = 8 m und der Höhe h = 3.5 m wird eine
rechteckige Glaswand eingebaut. Wie müssen
Breite $a$ und Höhe $b$ der Glaswand gewählt
werden, damit möglichst viel Licht einfällt?

\bbwCenterGraphic{80mm}{Giebelfenster.png}

\TNTeop{
$a=4 \text{ m}$ und $b= 1.75\text{ m}$

Marthaler Algebra Seite 283 Aufg 62.
}%%


%%%%%%%%%%%%%%%%%%%%%%%%%%%%%%%%%%%%%%%%%%%%%%%%%%%%
\subsection{Abstände}
Die Parabel $y = x - 2x + 3$ wird von der Geraden $y = tx + 5$ in den
Punkten $P$ und $Q$ geschnitten.

Bestimmen Sie den Wert von $t$, für den die Summe der Abstände der beiden Punkte
von der $x$-Achse möglichst klein ist.


\TNTeop{$t=1$

Frommenwiler S. 192 Aufg. 718
}

%%%%%%%%%%%%%%%%%%%%%%%%%%%%%%%%%%%%%%%%%%%%%%%%%%%%%%
\subsection{Eckige Dachrinne}
Ein rechteckiges Metallblech mit einer Breite
von $b = 0.5 \text{ m}$ soll zu einer Dachrinne mit einer
rechteckigen Querschnittsfläche gebogen werden.
Wie muss die Wandhöhe $x$ gewählt werden, damit
die Querschnittsfläche möglichst groß und die
Wasserabflussmenge maximal wird?
\bbwCenterGraphic{7cm}{EckigeDachrinne.png}

\TNTeop{
$x=12.5 \text{ cm}$

Marthaler Algebra S. 282 Aufg. 58
}%% end TNT eop

%%%%%%%%%%%%%%%%%%%%%%%%%%%%%%%%%%%%%%%%%%%%%%%%%%%%
\subsection{Lüftungskanal}
Ein Lüftungskanal wird aus ästhetischen
Gründen mit der abgebildeten Querschnittsform
geplant. Der Umfang des Querschnitts soll
160 cm betragen.
Für welches Maß $x$ wird die Querschnittsfläche
maximal?
\bbwCenterGraphic{40mm}{Lueftungsschacht.png}

\TNTeop{
$x$ = 44.8 cm

Aufgabe Frommenwiler Algebra S. 149 Aufg 730.
}

%%%%%%%%%%%%%%%%%%%%%%%%%%%%%%%%%%%%%%%%%%%%%%%%%%%%
\subsection{Runde Dachrinne}
Ein rechteckiges Metallblech mit einer Breite
von $b = 0.65 \text{ m}$ soll zu einer Dachrinne mit einer
halbrunden Querschnittsfläche gebogen werden.
Wie muss die Breite $x$ der Rinne gewählt werden,
damit die Querschnittsfläche möglichst groß und somit
die Wasserabflussmenge maximal wird?

\bbwCenterGraphic{7cm}{DachrinneRund.png}

\TNTeop{
$41.38 \text{ cm}$

Marthaler Algebra S. 282 Aufg. 59
}%% end TNT eop


%%%%%%%%%%%%%%%%%%%%%%%%%%%%%%%%%%%%%%%%%%%%%%%%%%%%%%%%%%%%%%%%%%%%%%%%%%%%%%
\subsection{Gewölbegang}

Ein Gewölbegang hat einen Querschnitt
von der Form eines Rechtecks mit aufgesetztem Halbkreis. Der Umfang
des Querschnitts ist mit $U=10 \text{ m}$ fest vorgegeben.
Wie muss das Gewölbe gestaltet werden, damit die Querschnitsfläche
möglichst groß wird?

\bbwCenterGraphic{40mm}{Gewoelbegang.png}

\TNTeop{

Mathematik neue Wege (J. Klötzli et al) Seite 302 Aufgabe 41.
}
%%%%%%%%%%%%%%%%%%%%%%%%%%%%%%%%%%%%%%%%%%%%%%%%%%%%%%%%%%%%%%%%%%%%%%%%%%%%
%%%%%%%%%%%%%%%%%%%%%%%%%%%%%%%%%%%%%%%%%%%%%%%%%%%%%%%%%%%%%%%%%%%%%%%%%%%%
\newpage
\section{nicht quadratische Funktionen (TR)}
Bei den folgenden Aufgaben reicht die Scheitelpunkts-Bestimmung der
quadratischen Funktionen nicht aus. Hoch- bzw. Tiefpunkte können mit
einem Computer-Algebra-System (CAS) gefunden werden.

%%%%%%%%%%%%%%%%%%%%%%%%%%%%%%%%%%%%%%%%%%%%%%%%%%%%%%%%%%%%%%%%%%%%%%%%%%%%%
\subsection{Rechteck unter Parabel}

Aus der Fläche zwischen $x$-Achse und der
Parabel mit $y=4-x^2$ wird ein Rechteck heraus-
geschnitten:

\bbwCenterGraphic{80mm}{RechteckUnterParabel.png}
a) 
Geben Sie die Funktionsgleichung an,
welche die Abhängigkeit der Rechtecksfläche A von der waagrechten
Rechtecksseite beschreibt.

b) 
Wie groß muss die waagrechte Seite des
Rechtecks gewählt werden, damit die
Rechtecksfläche möglichst groß wird?

c)
Wie groß ist die maximale Rechtecksfläche?

\TNTeop{
a) $A(x) = 2xy = -2x^3+8x$

b)
\bbwCenterGraphic{80mm}{RechteckUnterParabelLsg.png}

Hochpunkt $H=(1.155; 6.158)$ Die waagrechte Seite ist $2x \approx 2.31$.

c) $6.158$

Marthaler Algebra S. 321 Aufg. 17
}%% end TNT eop
%%%%%%%%%%%%%%%%%%%%%%%%%%%%%%%%%%%%%%%%%%%%%%%%%%%%%%%%%%%%%%%%%%%%%%%%%%%%%
\subsection{Marmorplatte}

Eine Marmorplatte ist bei der Bearbeitung zerbrochen. Die Bruchkante lässt sich durch die Funktionsgleichung $y=x^2 - 4.6x + 4.93$ einigermaßen annähern.

Der Produzent will aus diesem Bruchstück einen kleinen rechteckigen Tisch mit möglichst großer
Fläche $A$ herstellen.

\bbwCenterGraphic{80mm}{Marmorplatte.png}

a)
Geben Sie eine mögliche Funktionsgleichung an, welche die Abhängigkeit
der Rechtecksfläche $A$ von der Seite $a=x$ beschreibt.

b) Wie groß muss $a$ gewählt werden, damit die Rechtecksfläche $A$
möglichst groß wird?

c) Geben Sie die maximale Rechtecksfläche an.

\TNTeop{
a) Zielfunktion $A(x) = xy$ mit $y=x^2-4.6x+4.93$
$$A(x) = x(x^2-4.6x+4.93)$$
\newpage
b)
\bbwCenterGraphic{80mm}{MarmorplatteLsg.png}

Hochpunkt: $H=(0.692; 1.540)$

Die waagrechte Seite a ist $a=x=0.692$

c)
Die Maximale Rechtecksfläche beträgt $1.540$.

Marthaler Algebra S. 321 Aufg. 18
}
%%%%%%%%%%%%%%%%%%%%%%%%%%%%%%%%%%%%%%%%%%%%%%%%%%%%%%%%%%%%%%%%%%%%%%%%%%%%%
\subsection{Radiergummi}
Die Firma Flamingo produziert Radiergummis. Die dabei entstehenden
Kosten werden durch die Kostenfunktion
$$K(x) = 2.5x^3 = 16x^2 + 10x + 10$$
beschrieben.

$x = \text{ Stückzahl in 100}$ und $K = \text{ Kosten in CHF}$

Marktanalysen zeigen, dass 100 Stück zu einem Preis von jeweils CHF 40
abgesetzt werden können.

a) In welchem Bereich kann Gewinn erwirtschaftet werden?

b) Bei welcher Stückzahl wird der Gewinn maximal?

c) Wie groß ist der Maximale Gewinn?

\TNTeop{

Mathematik Neue Wege (J. Klötzli et al) Seite 302 Aufg. 40
}


%%%%%%%%%%%%%%%%%%%%%%%%%%%%%%%%%%%%%%%%%%%%%%%%%%%%%%%%%%%%%%%%%%%%%%%%%%%%%%
\subsection{Truthähne}
Eine Gruppe wilder Truthähne wird auf einer Insel ohne natürliche Feinde ausgesetzt. Biologen haben
herausgefunden, dass die Anzahl der Truthähne zum Zeitpunkt $t$ mit
einer Polynomfunktion angenähert werden kann mit folgender
Funktionsgleichung:

$$h(t) = -0.00001 t^3 + 0.002 t^2 + 1.5t + 100$$

Dabei ist $t$ die Zeit, seit der die Truthähne auf der Insel freigelassen wurden in Tagen.

a)
Wann wird die Population ihr Maximum erreicht haben und wie groß ist die Population zu
diesem Zeitpunkt?

b)
Wie entwickelt sich die Population ohne äußere Einflüsse weiter und was könnte die Erklärung
sein für diese Entwicklung?

\TNTeop{
a) \bbwCenterGraphic{90mm}{Truthaehne.png}
Hochpunkt: $H=(300; 460)$

Die Population wird nach 300 Tagen ihr Maximum von 460 Truthähnen
erreichen.

b) Die Population nimmt wieder ab. Mögliche Ursachen: Feinde und deren
Vermehrung, zu wenig Nahrung, Krankheiten, ...

Marthaler Algebra Seite 320 Aufg 14.
}%% end TNT eop
%%%%%%%%%%%%%%%%%%%%%%%%%%%%%%%%%%%%%%%%%%%%%%%%%%%%%%%%%%%%%%%%%%%%%%%%%%%%
\subsection{Tagestemperatur}

Die Temperatur $T$ in $\degre$C einer Stadt kann für eine Periode von 24 Stunden mit der folgenden Funktion
angenähert werden:
 $$T(t) = 0.01t(t-24)(t-18)+10$$
 Dabei wird $t=0$ um 8:00 Uhr morgens betrachtet.

a)
Zeichnen Sie den Graphen der Funktion.

b) 
Geben Sie die Definitionsmenge $\mathbb{D}$ und die Wertemenge
$\mathbb{W}$ der Funktion an.

c)
Bestimmen Sie höchste und tiefste Temperatur und ihre Uhrzeit.

d)
Wann beträgt die Temperatur 20 $\degre$C?

\TNTeop{
a)
\bbwCenterGraphic{90mm}{Tagestemperatur.png}


b) Hochpunkt: $H = 6.79; 23.1)$

Tiefpunkt: $T = (21.2; 8.1)$

$\mathbb{D} = [0;24]$

$\mathbb{W} = [8.1; 23.1]$

c)

Tiefste Temperatur 8.1 $\degre$C um 5:13 Uhr

Höchste Temperatur 23.1 $\degre$C um 14:47 Uhr

d) 
$20 = 0.01t(t-24)(t-18)+10$

Es folgt: $t_1 = 3.28$, $t_2 = 11.01$ (und $t_3=27.71$ entfällt)

Uhrzeiten: 11:17 Uhr und 19:00 Uhr

Marthaler Algebra Seite 320 Aufgabe 15. 
}

%%%%%%%%%%%%%%%%%%%%%%%%%%%%%%%%%%%%%%%%%%%%%%%%%%%%%%%%%%%%%%%%%%%%%%%%%%
\subsection{Kanal}

Die folgende Aufgabe habe ich von meinem Sekundarschullehrer Oskar
Spillmann im Jahre 1981 erhalten:

Ein Kanal soll aus identischen Betonplatten der Breite $a$ hergestellt werden.
\bbwCenterGraphic{80mm}{KanalPlatte.png}

Der Kanal wird aus je einer Grundplatte und zwei Seitenplatten
gebildet. Siehe Grafik:

\bbwCenterGraphic{80mm}{KanalSchraegbild.png}

In welchem Winkel zu einander müssen die Platten stehen bzw. was ist
die Kanalhöhe, damit der Kanal möglichst viel Wasser führt?

\TNTeop{
$h = \frac{\sqrt{3}}{2}\cdot{}a$

Winkel $\varphi$ zwischen den Platten: $\varphi = 120\degre$
}
%%%%%%%%%%%%%%%%%%%%%%%%%%%%%%%%%%%%%%%%%%%%%%%%%%%%%%%%%%%%%%%%%%%%%%%%%%%
\subsection{Balken aus Stamm}

Aus einem Baumstamm mit kreisförmigem Querschnitt vom Durchmesser
$52 \text{ cm}$ soll ein
Balken mit rechteckigem Querschnitt herausgeschnitten werden, sodass
seine Belastbarkeit auf Biegung, d.\,h. sein Widerstandsmoment $W$
maximal wird.

Es gilt:
$$W = \frac{b\cdot{}h^2}6$$

Berechnen Sie die Querschnittsabmessungen $b$ (Breite) und $h$ (Höhe) des Balkens.

\TNTeop{
$d$ = Durchmesser

$b$ = Breite

$h$ = Höhe

$$\Longrightarrow b = \frac{d\sqrt{3}}3$$
$$\Longrightarrow h = \sqrt{d^2-b^2} \approx 42.46 \text{ cm}$$
Frommenwiler Algebra S. 243 Aufg. 922.
}

%%%%%%%%%%%%%%%%%%%%%%%%%%%%%%%%%%%%%%%%%%%%%%%%%%%%%%%%%%%%%%%%%%%%%%%%%%%
\subsection{Strecke in Viertelskreis}

Bestimmen Sie $\overline{AB}$ aus $r$ so, dass $\overline{EF}$ maximal
wird.

\bbwCenterGraphic{80mm}{StreckeInViertelskreis.png}

\TNTeop{
$\overline{AB}_{\text{max}} = \frac{\sqrt{2}}2 \cdot{} r$

Frommenwiler Algebra S. 225 Aufg 932.
}%% end TNT eop

%%%%%%%%%%%%%%%%%%%%%%%%%%%%%%%%%%%%%%%%%%%%%%%%%%%%%%%%%%%%%%%%%%%%%%%%%%%
\subsection{Sektor in Kreis}

Ein Kreissektor ist einem Kreis mit Radius 1 einbeschrieben.

Wie groß müssen Radius $r$ und Zentriwinkel $\varphi$ des Sektors gewählt werden, wenn sein
Flächeninhalt möglichst groß sein soll?

\TNTeop{

$r \approx 1.5882$

$\stackrel{\frown}{{\varphi}}\,\approx 1.30654$

$\varphi \approx 74.859\degre$

Idee: Rechtwinkliges Dreieck mit Seiten $r$, $1+x$ und $\sqrt{1-x^2}$
im halben Sektor einzeichnen $\Longrightarrow V(x) =
(1+x)\arctan\left(\frac{\sqrt{1-x^2}}{1+x}\right)$.


Frommenwiler Algebra S. 245 Aufg. 933
}%% end TNT eop

%%%%%%%%%%%%%%%%%%%%%%%%%%%%%%%%%%%%%%%%%%%%%%%%%%%%%%%%%%%%%%%%%%%%%%%%%%%%
%%%%%%%%%%%%%%%%%%%%%%%%%%%%%%%%%%%%%%%%%%%%%%%%%%%%%%%%%%%%%%%%%%%%%%%%%%%%
\newpage\section{Stereometrie (TR)}

\subsection{Kartonschachtel}
Zur Herstellung einer Schachtel wird ein rechteckiges
Stück Karton von 15 cm auf 20 cm verwendet. In
den vier Ecken wird je ein Quadrat mit Seitenlänge $x$
ausgeschnitten, anschließend wird die Schachtel
gefaltet und die Kanten werden mit Klebeband
verklebt.

a)
Geben Sie die Funktionsgleichung der Abhängigkeit $x \mapsto$ Schachtelvolumen $V$ an.

b)
Wie muss $x$ gewählt werden, damit das Volumen
der Schachtel möglichst groß wird?

c)
Geben Sie die Größe dieses maximalen Volumens
an.

\bbwCenterGraphic{70mm}{Kartonschachtel.png}

\TNTeop{
a) $V(x) = 4x^3 - 70 x^2 + 300x = 2x(x-10)(2x-15)$

b) $x_{\text{max}} = 2.829 \text{ cm} = \frac{-5\cdot(\sqrt{13} - 7)}{6}$

c) $V_{\text{max}} = 379.0378 \text{ cm}^3$

Marthaler Algebra S. 320 Aufg. 12
}%% end TNTeop
%%%%%%%%%%%%%%%%%%%%%%%%%%%%%%%%%%%%%%%%%%%%%%%%%%%%%%%%%%%%%%%%%%%%%%%%%%%%%%%%%
\subsection{Streichholzschachtel}

Eine Streichholzschachtel besteht aus einem Innenteil und einer Hülle
(s. Graphik):

\bbwCenterGraphic{160mm}{Streichholzschachtel.png}

Die Länge von $l = 5 \text{ cm}$ wird durch die Streichholzlänge
bestimmt.

Ebenso soll der Inhalt mit einem Volumen $V = 45\text{ cm}^3$
vorgegeben sein.

Berechnene Sie die Höhe $h$, wenn möglichst wenig Karton zur
Herstellung verwendet werden soll.

\TNTeop{
Sei $S$ die «surface»:

$S(b, h) = 2bh + 3bl + 4lh$ mit
$V = 45 = lbh$ und mit $l=5$ und $V=45$ gilt:

$$45 = 5bh \Longrightarrow b = \frac9h$$

Damit wird $S(b, h)$ zu $S(h)$:

$$S(h) = 2\cdot{}\frac9h\cdot{}h + 3\cdot{}\frac9h\cdot{}5 +
4\cdot{}5\cdot{}h = 18 + \frac{135}h + 20\cdot{}h$$

Maximieren:

\texttt{S(h):= 18 + 135/h + 20h}

\texttt{fMin(S(h), h, 0, 3)} liefert

$h_{\text{max}} = \frac{3\cdot{}\sqrt{3}}{2} \approx 2.598 \text{ cm}$

}

%%%%%%%%%%%%%%%%%%%%%%%%%%%%%%%%%%%%%%%%%%%%%%%%%%%%%%%%%%%%%%%%%%%%%%%%%%%%%%%%%
\subsection{Toblerone?}

Von einem Blechstück der Form eines gleichseitigen Dreiecks mit der
Seitenlänge $53 \text{ cm}$
wird durch Abschneiden der drei Drachenvierecke ein oben offener Behälter hergestellt.

Berechnen Sie die Höhe $h$ für maximales Volumen des Behälters.

\bbwCenterGraphic{80mm}{TobleroneMax.png}

\TNTeop{
$h = \frac{s\sqrt{3}}{18} \approx 0.0962250 \cdot{} s$ mit $s=53cm$
erhalten wir $h \approx 5.099927 \text{ cm}$.

Frommenwiler Algebra S. 244 Aufg. 927.
}

%%%%%%%%%%%%%%%%%%%%%%%%%%%%%%%%%%%%%%%%%%%%%%%%%%%%%%%%%%%%%%%%%%%%%%%%%%%%%%%


\subsection{Tipi}

Aus vier 5 m langen Stangen soll ein
pyramidenförmiges Zeit mit quadratischer Grundfläche errichtet werden.

Wie hoch wird es, wenn sein Volumen möglichst groß sein soll?

\TNTeop{Höhe = 2.89 m

Frommenwiler Geometrie/Stereometrie Aufg 222 Seit 173
}%% end TNTeop


%%%%%%%%%%%%%%%%%%%%%%%%%%%%%%%%%%%%%%%%%%%%%%%%%%%%%%%%%%%%%%%%%%%%%%%%%%%%
\subsection{Säule}

Eine quaderförmige Säule mit quadratischem Querschnitt hat eine Oberfläche von $4 \text{ m}^2$.

Berechnen Sie die Länge der Kanten so, dass das Volumen der Säule
maximal wird.

\TNTeop{
Der Würfel hat optimale Oberfläche, somit ist der Flächeninhalt einer
Seite 4 Sechstel.

Ergo $a = \sqrt{\frac46} = \sqrt{\frac{4\cdot{}6}{36}}
= \frac{2\sqrt{6}}{6} = \frac{\sqrt{6}}{3} \approx 0.8165 \text{ m}$

Frommenwiler Geometrie Seite 173 Aufg 223
}%% end TNTeop

%%%%%%%%%%%%%%%%%%%%%%%%%%%%%%%%%%%%%%%%%%%%%%%%%%%%%%%%%%%%%%%%%%%%%%%%%%%%
\subsection{Würfel in Pyramide}
Einer regulären vierseitigen Pyramide mit der Grundkante $a = 5$ cm und der Höhe $h = 6$ cm
soll ein Quader mit möglichst großem Volumen einbeschrieben werden.
Berechnen Sie die Kantenlängen dieses Quaders.

\TNTeop{ Die Kantenlängen messen 3.33 cm und 2.00 cm.

Frommenwiler Geometrie Seite 173 Aufg. 224 }%% end TNT eop
%%%%%%%%%%%%%%%%%%%%%%%%%%%%%%%%%%%%%%%%%%%%%%%%%%%%%%%%%%%%%%%%%%

\subsection{Offene Büchse}
Für die Herstellung einer zylinderförmigen Büchse ohne Deckel benötigt
man $240 \text{ cm}^2$ Blech. Berechnen Sie den Radius der Grundfläche und die Höhe der Büchse so, dass sie möglichst viel Flüssigkeit aufnehmen kann.

\TNTeop{
Höhe und Radius messen beide gleich viel, nämlich 5.05 cm.

Frommenwiler Seite 173 Aufg. 225
}%% end TNT  eop




%%%%%%%%%%%%%%%%%%%%%%%%%%%%%%%%%%%%%%%%%%%%%%%%%%%%%%%%%%%%%%%%%%%%%%%
\subsection{Wärme im Büroraum}

Das Volumen eines Büroraumes mit quadratischem Grundriss soll $15\,000 \text{m}^3$ betragen. Die Wärmeabstrahlung durch die Wände sei dreimal so groß wie jene durch Decke und Boden. Welche Abmessungen müsste der Raum bei kleinstem Wärmeverlust haben?

\TNTeop{
Eine Höhe von 11.9 m und eine Quadratseite von 35.6 m

Frommenwiler S. 173 Aufg 226
}%% end TNTeop


%%%%%%%%%%%%%%%%%%%%%%%%%%%%%%%%%%%%%%%%%%%%%%%%%%%%%%%%%%%%%%%%%%%%%%%
\subsection{Preisoptimierung}

Ein Behälter mit $200 \text{ dm}^3$ Volumen hat die
Form eines Quaders mit quadratischer Grundfläche (Länge der Kanten: $a$, $a$, $h$). Das Material für die Grund-
und Deckfläche kostet $15 \text{ \euro} / \text{dm}^2$, für die Seitenflächen
$5 \text{ \euro} / \text{dm}^2$. Bei welchen Kantenlängen werden die Materialkosten minimal?

\TNTeop{
Die Grundkanten messen 4.05 dm und die Höhe ist 12.2 dm

Frommenwiler Geometrie S. 173 Aufg. 227
}%% end TNTeop



%%%%%%%%%%%%%%%%%%%%%%%%%%%%%%%%%%%%%%%%%%%%%%%%%%%%%%%%%%%%%%%%%%%%%%%


\subsection{Zylinder in Kegel}
Einem Kegel mit dem Grundkreisdurchmesser $d = 17 \text{ cm}$ und der Höhe $h
= 15 \text{ cm}$ ist ein Zylinder größten Volumens einzubeschreiben. Berechnen Sie das Zylindervolumen.

\TNTeop{
$504 \text{ cm}^3$

Frommenwiler S. 173 Aufg. 228
}%% end TNTeop

%%%%%%%%%%%%%%%%%%%%%%%%%%%%%%%%%%%%%%%%%%%%%%%%%%%%%%%%%%%%%%%%%%%%%%
\subsection{Kegel um Zylinder}

Einem Zylinder wird ein Kegel mit Grundkreisradius $5 \text{ cm}$ und
Höhe $7 \text{ cm}$ umbeschrieben. Wie groß ist das maximale Volumen
des Zylinders.

\TNTeop{
$V_{\text{max}} = \frac{100}{27}\cdot{}\pi \approx 81.4 \text{ cm}^3$

Mathematik neue Wege TALS (J. Klötzli et al) Seite 456 Aufg. 8
}


%%%%%%%%%%%%%%%%%%%%%%%%%%%%%%%%%%%%%%%%%%%%%%%%%%%%%%%%%%%%%%%%%%%%%%%


\subsection{Kegel in Kegel}
Einem geraden Kegel mit dem Grundkreisdurchmesser $d = 14 \text{ cm}$ und der
Höhe $h = 11 \text{ cm}$
ist ein Kegel größten Volumens einzubeschreiben.

Dabei fällt seine Spitze mit dem Mittelpunkt der Grundfläche des
gegebenen Kegels zusammen.
Berechnen Sie das Volumen des inneren Kegels.
\TNTeop{
$83.6 \text{ cm}^3$

Frommenwiler S. 173 Aufg 229
}%% end TNTeop


%%%%%%%%%%%%%%%%%%%%%%%%%%%%%%%%%%%%%%%%%%%%%%%%%%%%%%%%%%%%%%%%%%%%%%%



\subsection{Zylindermantel in Kugel}
In eine Kugel mit $r = 37 \text{ cm}$ ist ein Zylinder mit größter
Mantelfläche einzubeschreiben.
Berechnen Sie den Zylinderradius.

\TNTeop{
26.2 cm

Frommenwiler S. 173 Aufg  230.
}%% end TNTeop

%%%%%%%%%%%%%%%%%%%%%%%%%%%%%%%%%%%%%%%%%%%%%%%%%%%%%%%%%%%%%%%%%%%%%%%%
\subsection{Frappéglas}

Nino möchte wissen, wie groß der Öffnungswinkel seines kegelförmigen
Frappéglases sein muss, damit er möglichst viel Frappé hinein füllen
kann. Die Mantellinie des Glases ist fest vorgegeben.

\TNTeop{



Mathematik neue Wege TALS (J. Klötzli et al) Seite 460 Aufg. 10
}

%%%%%%%%%%%%%%%%%%%%%%%%%%%%%%%%%%%%%%%%%%%%%%%%%%%%%%%%%%%%%%%%%%%%%%%

\subsection{Kegel in Kugel}
In eine Kugel mit $r = 23 \text{ cm}$ ist ein gerader Kegel größten Volumens einzubeschreiben. Berechnen Sie die Kegelhöhe.

\TNTeop{
30.7 cm

Frommenwiler S. 173 Aufg. 231.
}%% end TNTeop

%%%%%%%%%%%%%%%%%%%%%%%%%%%%%%%%%%%%%%%%%%%%%%%%%%%%%%%%%%%%%%%%%%%%%%
\subsection{Kegel in Halbkugel}
Ein Kreiskegel soll einer Halbkugel mit Durchmesser $16 \text{ cm}$
einbeschrieben werden. Dabei soll die Spitze des Kreiskegels mit dem
Kugelmittelpunkt zusammenfallen. Wie groß ist das maximale Volumen des
Kreiskegels?


\TNTeop{

$V=\frac{1024\pi\sqrt{3}}{9} \approx 619 \text{ cm}^3$

Mit $h=\frac{8\sqrt{3}}{3} \approx 4.62 \text{ cm}$ und
$a=\frac{8\sqrt{6}}3 \approx 6.53 \text{ cm}$

Mathematik neue Wege TALS (J. Klötzli et al) Seite 456 Aufg. 9
}%% end TNTeop

%%%%%%%%%%%%%%%%%%%%%%%%%%%%%%%%%%%%%%%%%%%%%%%%%%%%%%%%%%%%%%%%%%%%%%%

\subsection{Kegel um Kugel}

Ein gerader Kegel hat eine Inkugel mit $4 \text{ cm}$
Radius. Wie groß muss man den Grundflächenradius des Kegels wählen, damit dessen Volumen minimal wird?

\TNTeop{
5.66 cm

Frommenwiler S. 174. Aufg 233. }%% end TNTeop


%%%%%%%%%%%%%%%%%%%%%%%%%%%%%%%%%%%%%%%%%%%%%%%%%%%%%%%%%%%%%%%%%%%%%%%

\subsection{Kugel in Kegel}
Einem Kegel wird eine Kugel mit Radius $r = 3 \text{ cm}$
einbeschrieben. Berechnen Sie den Radius und die Höhe des Kegels so,
dass

a) das Volumen des Kegels minimal wird.

b) die Oberfläche des Kegels minimal wird.

\TNTeop{
a) $h= 7 \text{ cm}$ und $a=3\cdot{}\sqrt{2} \approx 4.24 \text{ cm}$

b) $h \approx 6.35 \text {cm}$ und $a\approx 3.99 \text{ cm}$

Mathematik neue Wege TALS (J. Klötzli et al) Seite 456 Aufg. 7
}

%%%%%%%%%%%%%%%%%%%%%%%%%%%%%%%%%%%%%%%%%%%%%%%%%%%%%%%%%%%%%%%%%%%%%%%

\subsection{Halbkugel auf Zylinder}
Ein Körper mit einer Oberfläche von $4 \text{ m}^2$ besteht aus einem geraden Zylinder (Radius $r$, Höhe $h$) mit aufgesetzter Halbkugel (Radius $r$). Wie sind die Masse $r$ und $h$ zu wählen, damit der Körper ein möglichst großes Volumen hat?

\TNTeop{
$r = h = 50.5 \text{ cm}$

Frommenwiler S. 174. Aufg 232. }%% end TNTeop

%%%%%%%%%%%%%%%%%%%%%%%%%%%%%%%%%%%%%%%%%%%%%%%%%%%%%%%%%%%%%%%%%%%%%%%
\subsection{Tankwagen}

Ein Lastwagen hat einen Flüssigkeitsbehälter, dessen
Form sich aus einem Zylinder und zwei Halbkugeln zusammensetzt. Der Zylinder und die Halbkugeln haben den
gleichen Radius $x$. Die Länge des Behälters beträgt $4.2 \text{ m}$.

\bbwCenterGraphic{90mm}{Tankwagen.png}

a) 
Geben Sie die Funktionsgleichung der Abhängigkeit $x \mapsto \text{ Behältervolumen } V$ an.

b)
Zeichnen Sie den Graphen der Funktion.

c)
Zwischen welchen Werten muss der Radius $x$ liegen?

d)
Wie muss $x$ gewählt werden, damit der Behälter genau $25 \text{ m}^3$
fasst?

e)
Mit welchem Radius $x$ wird das Volumen $V$ am größten?

\TNTeop{

a) $x\mapsto  \frac43\pi\cdot{}x^3 + x^2\pi\cdot{}h = \frac43\pi\cdot{}x^3+x^2\pi(4.2-2x)=\pi(4.2x^2-\frac23x^3)$

b)

\bbwCenterGraphic{90mm}{TankwagenGraph.png}

c) $0\le x \le 2.1 [\text{m}]$

d) $x\approx 1.59235 \text{ m}$

e) Die Kugel hat das größte Volumen $\Longrightarrow x=\frac{4.2}2 =
2.1 \text{ m}$

Marthaler Algebra S. 320 Aufgabe 16}%% end TNT eop

%%%%%%%%%%%%%%%%%%%%%%%%%%%%%%%%%%%%%%%%%%%%%%%%%%%%%%%%%%%%%%%%%%%%%%%

\subsection{Papiersektor wird zu Kegel}

Aus einem kreisförmigen Stück Papier mit dem Radius  $10 \text{ cm}$ wird ein Sektor mit dem Zentriwinkel $\varphi$: ausgeschnitten und daraus eine kegelförmige Tüte gebildet. Für welchen Winkel $\varphi$ ist das Fassungsvermögen am größten?

\TNTeop{
$293.9\degre$

Frommenwiler Geometrie S. 174 Aufg. 234}%% end TNTeop




%%%%%%%%%%%%%%%%%%%%%%%%%%%%%%%%%%%%%%%%%%%%%%%%%%%%%%%%%%%%%%%%%%%%%%%
\subsection{Zündholzschachtel}
Eine Zündholzschachtel --- bestehend aus Hülle und Innenteil ---

soll $5 \text{ cm}$ lang sein  und ein Volumen von $25 \text{ cm}^3$ aufweisen. Bei welcher Höhe und Breite benötigt man zur Herstellung möglichst wenig Material?

\TNTeop{
Höhe: 1.94 cm ; Breite: 2.58 cm

Frommenwiler Geometrie S. 174 Aufg. 235}%% end TNTeop

%%%%%%%%%%%%%%%%%%%%%%%%%%%%%%%%%%%%%%%%%%%%%%%%%%%%%%%%%%%%%%%%%%%%

\subsection{Schmuckdose}

Eine Schmuckdose soll eine Länge von $7 \text{ cm}$ und ein Volumen
von $49 \text{ cm}^3$ aufweisen. Wie hoch und wie breit soll die
Schmuckdose sein, damit zur Herstellung möglichst wenig Material
verwendet werden muss?

\TNTeop{
Der Würfel hat minimale Oberfläche aller Quader.

$a = h = \sqrt{7} \approx 2.56 \text{ dm}$

Mathematik neue Wege (J. Klötzli et al) Seite 456 Aufg. 6
}%% end TNTeop



%%%%%%%%%%%%%%%%%%%%%%%%%%%%%%%%%%%%%%%%%%%%%%%%%%%%%%%%%%%%%%%%%%%%%%%

\subsection{Kugelsegment auf Kegel}
Ein Körper besteht aus einem Kreiskegel (Grundkreisradius $r=1$,
Mantellinie $x$) mit aufgesetztem Kugelsegment.
Die Grundkreise von Kegel und Segment sind flächengleich.
Die verlängerten Mantellinien des Kegels sind Tangenten des
Segments. Für welches $x$ hat der Körper minimale Oberfläche?

\TNTeop{

Mathematik neue Wege TALS (J. Klötzli et al) Seite 460 Aufg. 9
}%% end TNTeop


%%%%%%%%%%%%%%%%%%%%%%%%%%%%%%%%%%%%%%%%%%%%%%%%%%%%%%%%%%%%%%%%%%%%%%%


\end{document}
