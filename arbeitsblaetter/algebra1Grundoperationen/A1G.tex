\input{bbwLayoutPage}
\renewcommand{\bbwAufgabenBlockID}{A1G}

\renewcommand{\metaHeaderLine}{Aufgabenblatt}
\renewcommand{\arbeitsblattTitel}{Grundoperationen}


\begin{document}
\arbeitsblattHeader{}
Fassen Sie zusammen:
%%\begin{enumerate}[label=(\bbwAufgabenNummer. \alph*)]
\begin{bbwAufgabenBlock}
\item $x+1x+y+x = \LoesungsRaumLang{3x+y}$
\item $ab^2+ba^2 - 1ab + a^2b + ab^2 - 1\cdot{}a -b  = \LoesungsRaumLang{2ab^2 + 2ba^2 - ab - a - b}$
\end{bbwAufgabenBlock}

Schreiben Sie ohne Klammern und vereinfachen Sie so weit wie möglich:
\begin{bbwAufgabenBlock}
\item $-c+(-4c) = \LoesungsRaumLang{-5c}$
\item $a-x-(b-(-x-a)-b-a)-x = \LoesungsRaumLang{a-3x}$
\item $6x+3xy^2 -(-3x-y^2)-9x-y^2 = \LoesungsRaumLang{3xy^2}$
\item $5-((-\sqrt{2}a - \sqrt{3}b)+\sqrt{2}a)-(\sqrt{3}b+5)= \LoesungsRaumLang{0}$
\item $-(r+(-2r-(3r+4)))-(4r-(5r))=\LoesungsRaumLang{5r+4}$
\item $3s^2+(2s-5+(4s+1-((3s)^2-1)+s^2)+1-1s)+2 = \LoesungsRaumLang{5s-5s^2}$
\end{bbwAufgabenBlock}
\platzFuerBerechnungenBisEndeSeite{}


%%%%%%%%%%%%%%%%%%%%%%%%%%%%%%%%%%%%%%%%%%%%%%%%%5

Multiplizieren Sie aus:

\begin{bbwAufgabenBlock}
\item $5(x+y)=\LoesungsRaumLang{5x+5y}$
\item $-10(x-y)=\LoesungsRaumLang{-10x+10y}$
\item $-a(b\cdot{}(-4)) = \LoesungsRaumLang{4ab}$
\item $-b(-a^2+x) = \LoesungsRaumLang{+a^2b - bx}$
\item $-1(5-x)=\LoesungsRaumLang{x-5}$
\item $5x(ab-1)=\LoesungsRaumLang{5xab - 5x}$
\item $3rst + rst=\LoesungsRaumLang{4rst}$
\end{bbwAufgabenBlock}


Schreiben Sie als möglichst einfache Summe (bzw. Differenz):
\begin{bbwAufgabenBlock}
\item $3(2a+b) + 2(a+3b)=\LoesungsRaumLang{8a+9b}$
\item $9a(a^2-b)-b(a-b^2) = \LoesungsRaumLang{9a^3 - 10ab + b^3}$
\end{bbwAufgabenBlock}

Multiplizieren Sie aus, fassen Sie zusammen und schreiben Sie so
einfach wie möglich:
\begin{bbwAufgabenBlock}
\item $(t+m)(n+1\cdot{}v)=\LoesungsRaumLang{tn+tv+mn+mv}$
\item $(s-2)(-s-5) =\LoesungsRaumLang{-s^2-3s+10}$
\item $(-u+2v)(-u+3v) = \LoesungsRaumLang{u^2-5uv +6v^2}$
\item $(a-b+c)(-d-e)=\LoesungsRaumLang{-ae+be-ce-ad+bd-cd}$
\item $-\sqrt{2}(a-\sqrt{2})(\sqrt{2}b + c)
=\LoesungsRaumLang{-2ab- \sqrt{2}ac + 2\sqrt{2}b +2c}$
\item $(a+1)(1b+c)(a-2) = \LoesungsRaumLang{a^2b+a^2c+ab+ac-2ab-2ac-2b-2c}$
\item $-(3x^2+((x-3)x-x^2)x)-(x+3) = \LoesungsRaumLang{-x-3}$
\end{bbwAufgabenBlock} 

\platzFuerBerechnungenBisEndeSeite{}

%% 2. Weitere Seite
\platzFuerBerechnungenBisEndeSeite{}


\end{document}
