%% Kapitel Funktionen
%% GESO-Kompendium
%% philipp.freimann@bbw.ch
%%

\deu{\section{Funktionen}\index{Funktionen}}
\eng{\section{Functions}\index{Functions}}
\setcounter{aufgabenNummer}{1}
\renewcommand{\kAufgabenBuchstabe}{F}

\subsection{\deu{Grundlagen}\eng{Basics}}

\kTrainingAufgabe{
\eng{translate:}
\textbf{Graphen Zeichenen}
Zeichnen Sie den Graphen der Funktion im Bereich $-4 \le{} x \le{} 4$.

a) $y=\frac12\cdot{}x$ \hspace{30mm}
b) $y=\frac1x$ \hspace{30mm}
c) $y=x^2$

\includegraphics[width=150mm]{img/fct/Fct_ssA01.png}

}{%% Lösung
\raisebox{-80mm}{\includegraphics[width=150mm]{img/fct/Fct_ssL01.png}}
}{8}



\kTrainingAufgabe{
\eng{translate:}
\textbf{Funktionen auswerten}
Werten Sie den Funktionsterm $y=x^2-2x-5$ für folgende $x$-Werte aus:

a) $\frac12$ \hspace{25mm}
b) $-1$ \hspace{25mm}
c) $-\frac12$ \hspace{25mm}
d) $-\frac34$ \hspace{25mm}

}{%% Lösung
a) $-\frac{23}4$ \hspace{25mm}
b) $-2$ \hspace{25mm}
c) $-\frac{15}{4}$ \hspace{25mm}
d) $-\frac{47}{16}$ \hspace{25mm}
}{8}

\subsection{\deu{Lineare Funktionen}\eng{Linear Functions}}

\kTrainingAufgabe{
\eng{translate:}
\textbf{Steigungsbegriff, Mittelpunkt einer Strecke}

\kKommentar{Mittelpunkt ist ein geometrisches Konzept. Kommt das im
Rahmenlehrpln vor?}

a) Bestimmen Sie die Steigungen der Strecken a bis i. Zeichnen Sie
Steigungsdreiecke ein.

b) Bestimmen Sie die Koordinaten des Mittelpunkts jeder Strecke.

\includegraphics[width=130mm]{img/fct/Fct_ssA03.png}
}{%% Lösung

\renewcommand{\arraystretch}{2}
\begin{tabular}{|c|c|c|c|c|c|c|c|c|c|}\hline
   & a & b &c &d &e &f &g &h &i\\\hline
$a=m=$ & $\frac12$ & $\frac{-1}3$ & $\frac43$ & $1$ & $\frac{-1}3$
& $\frac{-4}7$ & $-5$ & $\frac12$ & $0$\\\hline
M.: & $(-5|\frac92)$ & $(\frac52|6.5)$ & $(5.5|6)$ & $(6|2)$ &
$(-5|1)$ & $(\frac12|0)$ & $(8.5|\frac32)$ & $(7|\frac{-7}2)$ & $(-2|-5)$\\\hline
 \end{tabular}
\renewcommand{\arraystretch}{2}
 
}{8}

\kTrainingAufgabe{
\eng{translate:}
\textbf{Graph zeichnen, Achsenschnittpunkte einzeichnen}:
Zeichnen Sie die Graphen folgen-
der Funktionen und heben Sie den Schnittpunkt mit der y-Achse und die Nullstelle hervor.
a) $y=2x+1$ \hspace{25mm}
b) $y=-x+3$ \hspace{25mm}
c) $y=-0.5x+4$ \hspace{25mm}

\includegraphics[width=150mm]{img/fct/Fct_ssA04.png}
}{%% Lösung

\raisebox{-50mm}{\includegraphics[width=150mm]{img/fct/Fct_ssL04.png}}

}{8}



\kTrainingAufgabe{
\eng{translate:}
\textbf{Funktionsterm auswerten:} Werten Sie die Funktion $f(x) = 2x+1$ für folgende Argumente
aus: $2$, $-2$, $0.5$, $-0.5$.
}{%% Lösung
$f(2)= 5$ \hspace{10mm}
$f(-2)=-3$ \hspace{10mm}
$f(0.5) = 2$ \hspace{10mm}
$f(-0.5)=0$
}{8}





\kTrainingAufgabe{
\eng{translate:}
\textbf{Bedeutung der Parameter; explizite Form:} Bestimmen Sie Steigung und $y$-
Achsenabschnitt in der durch die Funktionsgleichung $3x-2y=60$ gegebenen Geraden.

}{%% Lösung
$$y = 1.5x - 30$$
Steigung = 1.5 und $y$-Achsenabschnitt = -30.
}{8}





\kNiveauAufgabe{
\eng{translate:}\textbf{Achsenschnittpunkte berechnen:}
Berechnen Sie die Achsenschnittpunkte der Geraden $g$
mit der Funktionsgleichung $y=-0.25x+4$.
}{%% Lösung
Mit $x$-Achse: $(0|4)$

Mit $y$-Achse: $(16|0)$
}{8}





\kTrainingAufgabe{
\eng{translate:}\textbf{Horizontale Gerade:}
Notieren Sie die Funktionsgleichung der horizontalen Geraden durch den Punkt $P=(5|8)$.
}{%% Lösung
$f(x) = y = 8$
}{8}





\kNiveauAufgabe{
\eng{translate:}\textbf{Zweipunkteaufgabe:}
Wie lautet die Funktionsgleichung der Geraden, die durch folgende
zwei Punkte läuft (Resultate exakt angeben)?

a) $A(5|8)$ und $B(7|12)$


b) $A(-3|4)$ und $B(2|1)$


c) $A(-1|1)$ und $B(10|5)$

}{%% Lösung
a) $A=(5|8)$ und $B=(7|12)$ $\Longrightarrow$  $y=2x-2$


b) $A=(-3|4)$ und $B=(2|1)$ $\Longrightarrow$ $y=\frac{-3}5x+\frac{11}5$


c) $A=(-1|1)$ und $B=(10|5)$ $\Longrightarrow$  $y=\frac4{11}x+\frac{15}{11}$
}{8}




\kTrainingAufgabe{
\eng{translate:}\textbf{Funktionsgleichung aus Grafik bestimmen}
Wie lauten die Funktionsgleichungen der Geraden $a$ bis $f$?

\kKommentar{Niveau Aufnahmeprüfung!}

\includegraphics[width=140mm]{img/fct/Fct_ssA10.png}

}{%% Lösung
\raisebox{-80mm}{\includegraphics[width=160mm]{img/fct/Fct_ssL10.png}}
}{8}





\kNiveauAufgabe{
\eng{translate:}\textbf{Achsenschnittpunkte berechnen}
Berechnen Sie die Schnittpunkte des Graphen mit den beiden
Koordinatenachsen.

a) $y=10-3x$ \hspace{30mm}
b) $y=5-\frac12x$ \hspace{30mm}
c) $y=\frac{10x-2}5$ 
}{%% Lösung

\renewcommand{\arraystretch}{2}
\begin{tabular}{c|c|c|c}
& a) $y=10-3x$ & b) $y=5-\frac12x$ & c) $y=\frac{10x-2}5$  \\\hline
$N(x_n|0)$ & $\left(\frac{10}3\middle|0\right)$ & $(10|0)$ &
$(\frac15|0)$ \\\hline
$B(0|b)$ & $(0|10)$ & $(0|5)$ & $\left(0\middle|\frac{-5}2\right)$
 \end{tabular}
 
\renewcommand{\arraystretch}{2}

}{8}






\kTrainingAufgabe{
\eng{translate:}\textbf{Umwandeln in die explizite Form}
Wandeln Sie die Funktionsgleichung in die explizite Form $y=ax+b$
(bzw. $y=mx+q$) um.

a) $y=2-\frac{x}2$

b) $y=\frac{2x-10}3$

c) $y=\frac{1-x}{12}$

d) $y = -(2+1.25x)$

e) $4y-3x = 1.5$
}{%% Lösung

a) $y=\frac{-1}2\cdot{}x + 2$

b) $y=\frac23\cdot{}x - \frac{10}3$

c) $y=\frac{-1}{12}\cdot{}x + \frac1{12}$

d) $y=\frac{-5}4\cdot{}x - 2$

e) $y=\frac34\cdot{}x + \frac38$
}{8}





\kNiveauAufgabe{
\eng{translate:}\textbf{Liegt $P (-4.5|7)$ auf, oberhalb oder unterhalb der Geraden $g$?}

a) $g: y=-2x-2$

b) $g: y= -0.5x + 4.7$

c) $g: y = 2.5x + 18.25$
}{%% Lösung

a) Punkt liegt \textbf{auf} der Geraden.  

b) Punkt liegt \textbf{oberhalb} der Geraden.  

c) Punkt liegt \textbf{auf} der Geraden.  

}{8}




\kNiveauAufgabe{
\eng{translate:}\textbf{Geradengleichung aus Steigung und einem Punkt
$P \in g$ (Punkt-Steigungs-Aufgabe)}

Eine Gerade $g$ hat die Steigung $-0.4$ und geht durch $P (-2| -7)$. Wie lautet
die Funktionsgleichung von $g$?
}{%% Lösung
$g : y = -0.4x - 7.8$
}{8}




\kNiveauAufgabe{
\eng{translate:}\textbf{Liegen drei gegebene Punkte auf einer gemeinsamen Geraden oder nicht?}
Liegen $A(-2.5| -20.8)$, $B(21.5|50.0)$ und $C(100.0|281.575)$ auf
einer gemeinsamen Geraden $g$ oder nicht? Begründung?
}{%% Lösung
Ja, die Steigung von $AB$ ist gleich der Steigung von $BC$, nämlich $2.95$.
}{8}




\kNiveauAufgabe{
\eng{translate:}\textbf{Fehlende Punktkoordinate bestimmen: }
Der Punkt $C(-12.5|y_C)$ liegt auf der Geraden
$g$ durch $A(-2.5| - 20.8)$ und $B(21.5|50.0)$. Wie lautet die Koordinate $y_C$?
}{%% Lösung

$y_C = -50.3$
}{8}




\kNiveauAufgabe{
\eng{translate:}\textbf{Geradengleichung einer zur gegebenen Geraden parallelen Geraden durch einen
Punkt bestimmen:}
Gegeben ist $g : y = 4.8x - 2$. Die Gerade $h$ ist parallel zu $g$ und läuft
durch den Punkt $P(-8|2)$. Wie lautet die Geradengleichung von $h$?
}{%% Lösung
$h : y = 4.8x + 40.4$
}{8}




\kNiveauAufgabe{
\eng{translate:}\textbf{Schnittpunkt S zweier Geraden berechnen:}

Berechnen Sie den Schnittpunkt von $g: \frac{-2}3x+5$ und $h: \frac{-3}4x+10$
}{%% Lösung
Schnittpunkt = $(180|125)$
}{8}



\textbf{Kombinationen obiger Grundaufgaben}

\kNiveauAufgabe{
\eng{translate:}Die Gerade g hat die Steigung $-0.5$ und die Nullstelle bei $x = 5$. Wie lautet die Funktions-
gleichung von $g$?

}{%% Lösung
$g : y = -0.5x + 2.5$
}{8}




\kNiveauAufgabe{
\eng{translate:}Die Gerade $g$ hat die Steigung $-3$. Ferner liegt $A(-2|4)$ auf $g$. Der Punkt $B(x_B | - 23)$ liegt
ebenfalls auf $g$. Wie lautet die Koordinate $x_B$?

}{%% Lösung
$x_B = 7$
}{8}




\kNiveauAufgabe{
\eng{translate:}Die Gerade $g$ geht durch $A(1|5)$ und $B(8| - 2)$. Die Gerade $h$ schneidet die $x$-Achse im
selben Punkt $N$ wie die Gerade $g$. Die Gerade $h$ hat ferner die Steigung $4$. Wie lautet die
Funktionsgleichung von $h$?

}{%% Lösung
$h : y = 4x - 24$
}{8}


\textbf{Angewandte lineare Funktionen}

\kNiveauAufgabe{
\eng{translate:}Eine Taxifahrt von 10 km kostet CHF 21, eine solche von 15 km aber CHF 27. Wie lautet
die lineaer Funktionsgleichung $y = f (x)$ mit $x$ = Anzahl km Fahrstrecke und $y$ = Anzahl CHF
Gesamttaxe? Variante: Gesucht die Funktionsgleichung $K = f (s)$, $K$ = Kosten in CHF, $s$ =
Strecke in km. Es werden beide Varianten akzeptiert.

}{%% Lösung

$y = 1.2x + 9 $

oder (Variante)

$K(s) = 1.2 \text{CHF/km} · s + 9 \text{CHF}$

}{8}





\kNiveauAufgabe{
\newcommand{\tempGrad}[1]{${}^{\circ{}}$#1}
\eng{translate:}Die englische Temperaturskala (Fahrenheit-Grade, \tempGrad{F}) ist wie folgt festgelegt:
0\tempGrad{C} (Celsius) entsprechen 32 \tempGrad{F} (Fahrenheit). 100 \tempGrad{C}
entsprechen 212 \tempGrad{F}.

a) Wie lautet die lineare Funktionsgleichung zur Umrechnung
von \tempGrad{C} (= $x$) in \tempGrad{F} (= $y$)?

b) Wie lautet die lineare umgekehrte Funktion zur Umrechnung von \tempGrad{F}(= $x$) in \tempGrad{C} (= $y$)?

}{%% Lösung
a) $y = 1.8x + 32$

b) $y= \frac95 x - \frac{160}9$
}{8}





\kNiveauAufgabe{
\eng{translate:}
Eine lineare Notenskala ist wie folgt festgelegt:
Für 24 Punkte gibt es Note 6, für 0 Punkte Note 1.

a) Wie lautet die lineare Notenfunktion $y = f (x)$? $x$ = Anzahl Punkte, $y$ = Notenzahl.

b) Welche Note (auf 1 Dezimale gerundet) wird mit 18 Punkten erreicht?

c) Wie viele Punkte ergeben die Note 4?
}{%% Lösung

a) $y=\frac5{24}x+1$

b) $4.8$

c) 14.4 Punkte
}{8}





\kNiveauAufgabe{
\eng{translate:}\textbf{}
Lineare Abschreibung einer Maschine: Eine Maschine mit Neuwert CHF 5’000.- wird linear
abgeschrieben: Jedes Jahr soll sich ihr Wert um CHF 250.- verringern.

a) Wie lautet die lineare Abschreibungsfunktion $y = f (x)$ mit $x$ = Anzahl Jahre und $y$ =
Anzahl CHF Zeitwert nach $x$ Jahren?

b) Nach wie vielen Jahren ist die Maschine vollständig abgeschrieben?
}{%% Lösung

a) $y = -250x + 5000$

b) Nach 20 Jahren.
}{8}





\kNiveauAufgabe{
\eng{translate:}\textbf{}
Für HD-TV via Internet wird neben einer Grundgebühr auch die Anzahl Stunden, während
derer man fernsieht, verrechnet. Anbieter A verlangt pro Monat eine Grundgebühr von CHF
5.- und pro Stunde fernsehen eine Gebühr von 20 Rappen. Anbieter B verlangt folgendes:
Bei 20 h fernsehen CHF 10.- und bei 100 h fernsehen CHF 20.- (beides inkl. Grundgebühr).
Wie lauten die beiden Kostenfunktionen? $x$ = Anzahl Stunden Fernsehzeit, $y$ = Anzahl CHF
Kosten. Bei welcher Stundenzahl sind beide Anbieter gleich teuer?
}{%% Lösung
Gleicher Preis bei 33 Stunden und 20 Minuten.
}{8}



\subsubsection{Grafisches Lösen von Gleichungssystemen}


\kNiveauAufgabe{
\eng{translate:}
Lösen Sie das Gleichungssystem grafisch und interpretieren Sie die
Lösung:

a) \gleichungZZ{3x-4y}{8}{y}{\frac14 x + 6}

\begin{center}\includegraphics[width = 100mm]{img/fct/Fct_ssA27a.png}\end{center}


b) \gleichungZZ{10y}{7.5x}{18.75x+25y}{1500}

\begin{center}\includegraphics[width = 100mm]{img/fct/Fct_ssA27b.png}\end{center}

c) \gleichungZZ{13x+y-9.5}{4y-\frac{28}7}{-\frac45y+x}{-5+0.2x}

\begin{center}\includegraphics[width = 100mm]{img/fct/Fct_ssA27c.png}\end{center}

d) \gleichungZZ{40}{-2y-x}{3.5x-5y+82}{4\left(x-y+\frac{51}2\right)}

\begin{center}\includegraphics[width = 100mm]{img/fct/Fct_ssA27d.png}\end{center}

}{%% Lösung

a) \gleichungZZ{3x-4y}{8}{y}{\frac14 x + 6} $$S=(8|4)$$

\begin{center}\includegraphics[width = 100mm]{img/fct/Fct_ssL27a.png}\end{center}


b) \gleichungZZ{10y}{7.5x}{18.75x+25y}{1500} $$S=(40|30)$$

\begin{center}\includegraphics[width = 100mm]{img/fct/Fct_ssL27b.png}\end{center}

c) \gleichungZZ{13x+y-9.5}{4y-\frac{28}7}{-\frac45y+x}{-5+0.2x}

$$\mathbb{L}_S = \left\{\right\}$$

\begin{center}\includegraphics[width = 100mm]{img/fct/Fct_ssL27c.png}\end{center}

d) \gleichungZZ{40}{-2y-x}{3.5x-5y+82}{4\left(x-y+\frac{51}2\right)}

Die Geraden sind zusammenfallend. 

\begin{center}\includegraphics[width = 100mm]{img/fct/Fct_ssL27d.png}\end{center}

}{2}
\newpage






\subsection{\deu{Potenzfunktionen}\eng{Power Functions}}
\kKommentar{Graphen von Potenzfunktionen sind Hyperbeln oder
  Parabeln...}
  
\kKommentar{...Dies in den Titeln korrekt vermerken (nicht
  Hyperbelfunktion etc.)}

\subsubsection{$n>1$: \deu{Parabeln}\eng{Parabolas}}

\kNiveauAufgabe{
\deu{Zeichnen Sie den Graphen folgender Potenzfunktionen:}
\eng{Draw the graph of the following power functions:}
a) $x \mapsto 0.5x^3$  \hspace{30mm}  b) $x\mapsto -x^4$

\begin{tabular}{cc}
\includegraphics[width=85mm]{img/fct/Fct_ssA28a.png}
&
\includegraphics[width=85mm]{img/fct/Fct_ssA28b.png}
\\
 \end{tabular}
 

}{%% Lösung

\begin{tabular}{cc}
a)

\includegraphics[width=70mm]{img/fct/Fct_ssL28a.png}
&b)

\includegraphics[width=70mm]{img/fct/Fct_ssL28b.png}
\\
 \end{tabular}

}{4}


\kNiveauAufgabe{
\deu{Bestimmen Sie den Faktor $a$ in $x\mapsto a\cdot{}x^4$ wenn der Graph durch den Punkt $A(3|3)$ verläuft.}
\eng{Determine the factor $a$ in $x\mapsto a \cdot x^4$ when the graph
passes through the point $A(3|3)$.}
}{%% Lösung
$3=3\cdot{}a^4 \Longrightarrow a=\frac1{27}$
}{6}%% end KNiveauAufgabe




\subsubsection{$n < 0$ \deu{Hyperbeln}\eng{Hyperbolas}}\index{\deu{Hyperbeln}\eng{Hyperbolas}}



\kNiveauAufgabe{
\deu{Bestimmen Sie den Faktor $a$ der Potenzfunktion $y=a\cdot{}x^{-3}$,
wenn der Graph dieser Hyperbel durch den Punkt
$A\left( 3 \middle| \frac1{54} \right)$ verläuft.}
\eng{Determine the factor $a$ of the power function $y = a \cdot x^{-3}$ when the graph of this hyperbola passes through the point $A\left(3 \middle| \frac{1}{54}\right)$.}
}{%% Lösungs
$\frac1{54}=a\cdot{}3^{-3} \Longrightarrow   a=27\cdot{}\frac1{54} = \frac12$
}{8}









\kNiveauAufgabe{
\deu{Welche Funktion gehört zu welchem Graphen? Ordnen Sie die Zahl dem Buchstaben zu.}
\eng{Which function belongs to which graph? Match the number to the letter.}

a) $y = -x^3$  \hspace{15mm}  b) $y=x^{-2}$  \hspace{15mm} c)
$y=3\cdot{}x^{-1}$ \hspace{15mm}  d) $y=-0.5x^4$

\begin{center}\includegraphics[width=150mm]{img/fct/Fct_ssA31.png}\end{center}

}{%% Lösung
a) $y = -x^3$  \hspace{15mm}  b) $y=x^{-2}$  \hspace{15mm} c)
$y=3\cdot{}x^{-1}$ \hspace{15mm}  d) $y=-0.5x^4$

\includegraphics[width=150mm]{img/fct/Fct_ssL31.png}

}{4}

\newpage







\subsection{\eng{translate: }Exponentialfunktionen}

\subsubsection{\eng{translate: }Exponentielle Prozesse, Exponentialfunktion}

\textbf{\eng{translate: }Wachstums und Zerfallsprozesse}

\kNiveauAufgabe{
\deu{Zinseszins; Entwicklung eines Startkapitals bei konstantem Jahreszinssatz: Ein
Kapital von CHF 100.- wird über viele Jahre zu einem konstanten
\eng{Zinssatz von 2\% verzinst.}Compound interest; development of an initial capital with a constant annual interest rate: A capital of CHF 100.- is compounded over many years at a constant interest rate of 2\%.}

\begin{enumerate}[label=\alph*)]
\item
\deu{Bestimmen Sie den Zinsfaktor.}\eng{Bestimmen Sie den Zinsfaktor.}
\item
\deu{Wie lautet die Funktion $K = f (t)$, $K$ = Kapital zur Zeit $t$,
$t$ = Zeit in Jahren?}
\eng{What is the function $K = f(t)$, where $K$ is the capital at time $t$, and $t$ is the time in years?}

\item
\deu{Stellen Sie die Kapitalentwicklung grafisch dar}%
\eng{Graphically represent the capital development}:
$0 \le t \le 320$.
\end{enumerate}

\begin{center}\includegraphics[width=150mm]{img/fct/Fct_ssA32.png}\end{center}


}{%% Lösung
\begin{enumerate}[label=\alph*)]
\item
Der Zinsfaktor ist $1.02$.
\item
$K = f (t) = 100\cdot 1.02 ^ t$, $t$ = Zeit in Jahren?
\item
\deu{S. folgende Grafik}
\eng{consult the following graphic}
\end{enumerate}
\includegraphics[width=150mm]{img/fct/Fct_ssL32.png}

}{8}





\kNiveauAufgabe{
\eng{translate:} Degressive Abschreibung:
Eine Maschine mit einem Neuwert von CHF 20\,000.- wird de-
gressiv abgeschrieben. Sie verliert jährlich 10\% an Wert.
\begin{enumerate}[label=\alph*)]
\item
 Bestimmen Sie den Abnahmefaktor.
\item
 Erstellen Sie eine Wertetabelle: $t$ = Zeit in Jahren, $W=f(t)$ = Wert Maschine zur Zeit $t$.
\item
 Stellen Sie die Werte-Entwicklung grafisch dar: $0 \le t \le 10$.
\item
 Wie lautet eine möglicher Funktionsterm zu $W=f(t)$?
\end{enumerate}

\begin{center}\includegraphics[width=150mm]{img/fct/Fct_ssA33.png}\end{center}
}{%% Lösungen
\begin{enumerate}[label=\alph*)]
\item
Der Abnahmefaktor (hier Abschreibungsfaktor) ist $100\%-10\% =
(100-10)\% = 90\% = 90\cdot{}\frac1{100} = 0.9$
\item
\begin{tabular}{|c|c|c|c|c|c|}\hline
Jahr   & 0 & 1 & 2 & 3 & 4 \\\hline
Wert   & 20\,000 & 18\,000 & 16\,200 & 14\,580 & 13\,122 \\\hline
 \end{tabular}
 
\item
\deu{S. folgende Graphik}
\eng{consult the following graphic}
\end{enumerate}
\includegraphics[width=150mm]{img/fct/Fct_ssL33.png}

}{}


\kTrainingAufgabe{
\eng{translate:} Prozentuale Wachstumsrate $\leftrightarrow$
Wachstumsfaktor: Suchen Sie zu folgenden
prozentualen Wachstumsraten den zugehörigen Wachstumsfaktor:

a) Rate = $4\%$
\hspace{15mm}
b) Rate = $2.2\%$
\hspace{15mm}
c) Rate = $0.5\%$
}{%% Lösungen

a) $1.04$ 
\hspace{15mm}
b) $1.022$
\hspace{15mm}
c) $1.005$
}{4}


\kTrainingAufgabe{
\eng{translate:} Prozentuale Abnahmerate $\leftrightarrow$ Abnahmefaktor: Suchen Sie zu folgenden prozentualen
Abnahmeraten den zugehörigen Abnahmefaktor:

a) Rate = $4\%$
\hspace{15mm}
b) Rate = $20\%$
\hspace{15mm}
c) Rate = $5.5\%$
}{%% Lösungen

a) $0.96$ 
\hspace{15mm}
b) $0.8$
\hspace{15mm}
c) $0.945$
}{4}



\newpage
\kNiveauAufgabe{
\eng{tranlate:}
\textbf{Graphen von Exponentialfunktionen:}

Zeichnen Sie die Graphen von $f(x) = 2^x$ und $g(x) = 2^{-x}$ im
Bereich $-3 \le x \le 3$.

a) Vergleichen Sie die beiden Graphen. Wo schneiden sich die Graphen?

b) Bestimmen Sie Definitionsbereich und Wertebereich der beiden Funktionen.

c) Haben die Funktionen Nullstellen? Haben Sie Asymptoten?

\bbwGraph{-4}{4}{-1}{9}{}

}{%% Lösung
\begin{center}\includegraphics[width=150mm]{img/fct/Fct_ssL36.png}\end{center}

a) Die Graphen sind spiegelsymmetrisch (gespiegelt an der
$y$-Achse. Sie schneiden sich im Punkt $(0|1)$.

b) Definitonsbereich $\mathbb{D} = \mathbb{R}$ und Wertebereich
$\mathbb{W} = \mathbb{R}^{+}\backslash \{0\}$.

c) Die Funktionen haben keine Nullstellen, da der Wertebereich
$\mathbb{R}^+\backslash{}\{0\}$ ist. Der Funktions\textbf{wert} Null
kommt also nicht vor.

Für beide Funktionen ist die $x$-Achse die Asymptote.
}{4}

\kNiveauAufgabe{
Nennen Sie eine mögliche Exponentialfunktion $f(x) = a^x$ (also mit $f(0)=1$), bei der sich der Funktionswert

a) verdoppelt, wenn man das Argument um 3 erhöht (siehe Abbildung),

\begin{center}\includegraphics[width=65mm]{img/fct/Fct_ssA37.png}\end{center}

b) verdoppelt, wenn man $x$ um 12 erhöht bzw.

c) halbiert, wenn man $x$ um 12 erhöht.


}{%% Lösung
a) $f(x) = 2^\frac{x}3 =  \left(2^\frac13\right)^x
=\left(\sqrt[3]{2}\right)^x  $

b) $f(x) = 2^\frac{x}{12} =  \left(2^\frac1{12}\right)^x
=\left(\sqrt[12]{2}\right)^x  $

b) $f(x) = \left(\frac12\right)^\frac{x}{12} =  \left(\left(\frac12\right)^\frac1{12}\right)^x
=\left(\sqrt[12]{\frac12}\right)^x  $

}{8}


\kNiveauAufgabe{
\eng{translate:}

a) Die Wachstumsrate der Bevölkerung in der Schweiz betrug im Jahr 2012 ca. 1.1\%. 2012
zählte die Schweiz 8\,039\,100 Einwohner. Wie viele Personen werden bei gleich bleibender
Wachstumsrate im Jahr 2030 in der Schweiz leben (auf 100 Personen
genau)?

b) Seit 1999 wächst die Wohnbevölkerung der Stadt Zürich wieder. Im Jahr 1999 zählte die
Stadt Zürich 360\,704 Einwohner, im Jahr 2012 waren es 394\,012 Personen. Berechnen Sie
die durchschnittliche jährliche Wachstumsrate in \% (Genauigkeit: 2 Dezimalen). (Zum
Vergleich: Im Jahr 2012 betrug die Wachsumsrate 1\%.)
}{%% Lösungen

a) $8\,039\,100 \cdot{} 1.011^t$ mit $t=2030-2012=18$ ist $9\,788\,800$.

b) $\Delta\tau = 2012-1999 = 13$, somit gilt für die Wachstumsrate
$$q^{13} = \frac{394\,012}{360\,704} \Longrightarrow
q=\sqrt[13]{...}\approx 1.0068$$
Somit beträgt die jährliche Wachstumsrate ca. 0.68\%.
}{8}


\kNiveauAufgabe{
Berechnen Sie die durchschnittliche jährliche Wachstumsrate in % für die Bevölkerung einer
Stadt, die innert 10 Jahren von 70’000 Personen auf 82’000 Personen angewachsen ist. (Ge-
nauigkeit: 1 Dezimale.)
}{%% Lösung
$\Delta\tau = 10$ somit gilt für den Wachstumsfaktor
$q^{10} = \frac{82\,000}{70\,000}$. Der Faktor ist daher $q=\sqrt[10]{\frac{82}{70}} \approx 1.0159$. Die
prozentuale Zunahme (Rate) gerundet in Prozent ist also: 1.6\%.

}{8}









\subsubsection{Sättigung}
\newpage
