\renewcommand{\kAufgabenBuchstabe}{F}
\section{Neue Aufgaben}
\setcounter{aufgabenNummer}{1000}


\kKommentar{Noch einpflegen als Diskussionsgrundlage}


\kTrainingAufgabe{
$$\left(x-3\right)\left(x+1\right) - \left(x+2\right)^2$$

$$\left(x^2-3xy^4\right)^2$$

$$
\left(x-2\right) \cdot \left(x+1\right) \cdot \left(x-1\right)$$


$$ 2x^2\cdot \left(2x+5\right)^2$$

Faktorisieren:
$$3x^3+6x^2+9x$$

$$6ab-8ac-2a$$


Brüche:

$$ \frac{2x + 3}{x + 2} \cdot \frac{x - 4}{x - 1}$$
$$ \frac{2x + 3}{x + 2} + \frac{x - 4}{x - 1}$$

Potenzen Folgender Stil fehlt:

$$x^{\frac{1}{2}} \cdot x^{\frac{3}{4}}$$

$$\frac{x^{\frac{2}{3} } }{x^{\frac{1}{4} } }$$

Lineare Gleichungen mit Brüchen (Hier fehlen Aufgaben)

$$\frac{x+1}{4} + \frac{x-2}{2} = 6$$

$$\frac{x+1}{4} - \frac{2x-1}{6} = \frac{x+2}{3}$$

$$2 - 3\left(\frac{x - 1}{2}\right) = 2\left(\frac{x + 8}{4} -
\frac{1}{2}\right)$$
}{}

\kTrainingAufgabe{
Lineare Gleichngen (noch testen, ob sie aufgehen):


$$ \frac{2x + 3}{x + 2} \cdot \frac{x - 4}{x - 1}$$


$$\left(x+5\right)^2-\left(x-4\right)^2=8x+3-\left(3x+2\right) $$

}{}

\kTrainingAufgabe{
ruchgleichungen:

$$\frac{2x}{x+4} + \frac{1}{x} = 6 $$

$$\frac{2x+1}{x-1} + \frac{3x+1}{1-x} = \frac{3}{4} $$

$$\frac{3x+1}{2x+2} - \frac{3x+1}{x+1} = \frac{4x-2}{3x+3} $$

}{%% Lösungen
  \kKommentar{Lösungen ausstehend}
}


\kNiveauAufgabe{
  Auch so was bei Potenzen

  Siehe Foto (philipps Handy von Matheninja)

  Andere Zahlen nehmen.

  
}{%% Lösungen
}

%\kKommentar{Grundsätzlich $x$ als Lösungsvaraible (nicht auf einmal $a$).}

\kTrainingAufgabe{

  Bei Linearen Gleichungen auch was mit Klammern.
  
 $$7-2(3-5x(x+2))=3-(2x+3-(8+10x^2)) $$

  $$2 - 3 \cdot \frac{x - 1}{2} = 2 \cdot \left(\frac{x + 8}{3} - \frac{1}{3}\right)$$
}{%% Lösungen
}


\kKommentar{Einfachere Textaufgabe als die Zuleitung mit 9h 36' als
  Start für Leistungsaufgaben.}

$$ \frac{2x + 3}{x + 2} \cdot \frac{x - 4}{x - 1}$$



\kKommentar{Lineare Bruchgleichungssysteme:}
Hier hat es auch Bruchgleichungssysteme

Textaufgaben: 

„Ein Swimmingpool wird von zwei Pumpen gefüllt. Die grosse Pumpe braucht alleine 4 Stunden für die Füllung, die kleine Pumpe 6 Stunden. Wie lange dauert die Füllung, wenn beide Pumpen gleichzeitig arbeiten?“ 

 

„Ein Druckauftrag wird von zwei Druckern erledigt. Zusammen brauchen die beiden Drucker 5 Stunden. Arbeitet nur der langsamere Drucker, so braucht er 2 Stunden länger für den ganzen Druckauftrag, als der schnelle Drucker alleine braucht. Wie lange braucht der langsame Drucker, wenn er alleine arbeitet?“ 

=> zahlen anpassen!!



Stinknormales Gleichungssystem in Grundform (als Trainingsaufgabe)
fehlt.

Lineare gleichungssysteme: 

$$4\cdot{}x+y=5, 2\cdot{}x-3\cdot{}y=13$$
oder so:

$$X+y=5/6, x=y+1/3 $$

Bruchgleichungssysteme (nicht bei linearen Gleichungssystemen ablegen): 

$$\frac{1}{x} + \frac{2}{y} = 8, \frac{3}{x} + \frac{4}{y} = -1 $$

=> anpassen, sodass ganzzahlige lösung

Kommentar: Bei Gleichungssystemen sind noch
Bruchgleichungssysteme. Diese in der richtigen ``Schublade'' ablegen.


\textbf{Quadratische Gleichungen}

Aufbau der Schwierigkeit steigt nich normal an:

$$3x^2=75$$

$$\left(x-1\right)\left(x+3\right)=0 $$

$$\left(2x+1\right)\left(3x-4\right)\left(2x+3\right)=0 $$

Titel quadratische Gleichungen: Substitution sollte anders lauten
weil $(x-3/x)^2$ -> Polynomgleichung besser: ``Gleichungen höheren
Grades'' (Es sind keine quadratischen Gleichungen, wenn $x^6$ etc.)

Bei ``Elementare Potenzgleichungen'' als start $x^4 = 64$

$$(x-2)^2 = 144$$


Elementare Exponentialgleichungen 

Bei der jetztigen Aufgabe G132 (version Ch. Hersberger) nur Aufgab
a) d) f) g) h)    [Aufgabe i) ist noch falsch?? Parameteraufgabe kübeln]
j) kübeln, da zu einfach mit TR

Zusätzlich $3\cdot{} 4^x = 7$
und $2\cdot{}3^(2x+1)=37 $

Bei Exponentiagleichungen mit erzeugen gleicher Basis:
noch so was.

$$e^(3x+3)*e^(x-1)=e^2 $$

$$e^(3x+3)*e^(x-1)=e^2$$


$$(1/5)^x = (1/125)$$

$$3^(x+2)=1/3$$
 
$$3^(4x+1)*(3^2)^x=3^(5x+1) $$

$$2*3^x = 5*7^x$$

$$5^(x+1)=3^(x-2) $$



Bei linearen Funktionen:

Vor Aufgabe 5. (Thomas)
Wertetabelle ausfüllen

Füllen Sie die Wertetabelle aus:    y = -3/2 x + 1/4
-1 0 1 2 3 4
............


