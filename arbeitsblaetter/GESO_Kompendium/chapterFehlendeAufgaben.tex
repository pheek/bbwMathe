\renewcommand{\kAufgabenBuchstabe}{DRAFT}
\section{Neue Aufgaben}
\setcounter{aufgabenNummer}{1000}


\kKommentar{Noch einpflegen als Diskussionsgrundlage}


\kTrainingAufgabe{\eng{translate}
  Multiplizieren Sie aus:

  a) $$\left(x-3\right)\left(x+1\right) - \left(x+2\right)^2$$

b) $$\left(x^2-3xy^4\right)^2$$


c) $$ \left(x-2\right) \cdot \left(x+1\right) \cdot \left(x-1\right)$$

d) 
$$ 2x^2\cdot \left(2x+5\right)^2$$
}{%% Lösungen
  ..... coming soon ....
}


\kTrainingAufgabe{\eng{translate}
  
  Faktorisieren Sie so weit wie möglich:

  a)
$$3x^3+6x^2+9x$$
b) 
$$6ab-8ac-2a$$
}{%% Lösugen
  }

\kTrainingAufgabe{
  \eng{translate}
  Vereinfachen Sie die folgenden Bruchterme.
  Was ist jeweils die Definitionsmenge $\mathbb{D}$?
Brüche:

$$ \frac{2x + 3}{x + 2} \cdot \frac{x - 4}{x - 1}$$
$$ \frac{2x + 3}{x + 2} + \frac{x - 4}{x - 1}$$

}{% Lösungen
  ... coming soon ...
}


\kTrainingAufgabe{
  \eng{translate}
\kKommentar{Potenzen: Folgender Stil fehlt}

a)
$$x^{\frac{1}{2}} \cdot x^{\frac{3}{4}}$$
b)

$$\frac{x^{\frac{2}{3} } }{x^{\frac{1}{4} } }$$
}{%% Lösungen
  ... coming soon ....
}



\kTrainingAufgabe{\eng{translate}

  \kKommentar{Folgende Typen fehlen gänzlich}

  Lineare Gleichungen mit Brüchen

  a)
  $$\frac{x+1}{4} + \frac{x-2}{2} = 6$$

  b) 
$$\frac{x+1}{4} - \frac{2x-1}{6} = \frac{x+2}{3}$$

  c) 
$$2 - 3\left(\frac{x - 1}{2}\right) = 2\left(\frac{x + 8}{4} -
\frac{1}{2}\right)$$
}{%% lösungen
... coming soon ...}

\kTrainingAufgabe{\eng{translate}

  \kKommentar{Nicht nur Bruchgleichungen mit Brüchen}
  
Lineare Gleichngen (noch testen, ob sie aufgehen):

a)
$$ \frac{2x + 3}{x + 2} = \frac{2x - 4}{x - 1}$$

b)
$$\left(x+5\right)^2-\left(x-4\right)^2=8x+3-\left(3x+2\right) $$

}{%% Lösungen
}

\kTrainingAufgabe{\eng{translate}

\kKommentar{Folgender Stil fehlt}
  
Bruchgleichungen:

$$\frac{2x}{x+4} + \frac{1}{x} = 6 $$

$$\frac{2x+1}{x-1} + \frac{3x+1}{1-x} = \frac{3}{4} $$

$$\frac{3x+1}{2x+2} - \frac{3x+1}{x+1} = \frac{4x-2}{3x+3} $$

}{%% Lösungen
  \kKommentar{Lösungen ausstehend}
}

\kNiveauAufgabe{ \eng{translate}
  
  \kKommentar{gemischte Wurzelausdrücke fehlen}

  Vereinfachen Sie so weit wie möglich und schreiben Sie in Potenz
  oder Wurzelschreibweise:

  (Noch andere Zahlen nehmen, da teilweise 1:1 aus Matheninja.)

  a) $$\sqrt[5]{x^4\cdot{}\sqrt[2]{x^4}} \cdot{} \sqrt[4]{x^{-5}}$$

  b) $$\frac{\sqrt[4]{x^{\frac53}}}{x^{\frac16}} + \sqrt[3]{x^4}$$

 c) $$\sqrt[3]{x^4\cdot{} y^{8n}\cdot{}  \sqrt{y^{-36}}} \cdot{} x^{-\frac85}$$

 d) $$ \sqrt[3]{\frac{y^5}{x^6}}   :  \sqrt[n]{\frac{x^{3n}}{y^{2n}}}$$
}{%% Lösungen
  ... coming soon ...
}




%\kKommentar{Grundsätzlich $x$ als Lösungsvaraible (nicht auf einmal $a$).}

\kTrainingAufgabe{\eng{translate}

  \kKommentar{folgender Typ fehlt}
  
  Bei Linearen Gleichungen auch was mit Klammern.

  Lösen Sie die Gleichungen nach $x$ auf:
  
 $$7-2(3-5x(x+2))=3-(2x+3-(8+10x^2)) $$

  $$2 - 3 \cdot \frac{x - 1}{2} = 2 \cdot \left(\frac{x + 8}{3} - \frac{1}{3}\right)$$
}{%% Lösungen
  .... coming soon ...
}


\kKommentar{
  Einfachere Start Leistungsaufgabe als die Zuleitung mit 9h 36'.}


\kKommentar{Lineare Bruchgleichungssysteme:}
Hier hat es auch Bruchgleichungssysteme

Textaufgaben: 

„Ein Swimmingpool wird von zwei Pumpen gefüllt. Die grosse Pumpe braucht alleine 4 Stunden für die Füllung, die kleine Pumpe 6 Stunden. Wie lange dauert die Füllung, wenn beide Pumpen gleichzeitig arbeiten?“ 

 

„Ein Druckauftrag wird von zwei Druckern erledigt. Zusammen brauchen die beiden Drucker 5 Stunden. Arbeitet nur der langsamere Drucker, so braucht er 2 Stunden länger für den ganzen Druckauftrag, als der schnelle Drucker alleine braucht. Wie lange braucht der langsame Drucker, wenn er alleine arbeitet?“ 

=> zahlen anpassen!!



Stinknormales Gleichungssystem in Grundform (als Trainingsaufgabe)
fehlt.

Lineare gleichungssysteme: 

$$4\cdot{}x+y=5, 2\cdot{}x-3\cdot{}y=13$$
oder so:

$$X+y=5/6, x=y+1/3 $$

Bruchgleichungssysteme (nicht bei linearen Gleichungssystemen ablegen): 

$$\frac{1}{x} + \frac{2}{y} = 8, \frac{3}{x} + \frac{4}{y} = -1 $$

=> anpassen, sodass ganzzahlige lösung

Kommentar: Bei Gleichungssystemen sind noch
Bruchgleichungssysteme. Diese in der richtigen ``Schublade'' ablegen.


\textbf{Quadratische Gleichungen}

Aufbau der Schwierigkeit steigt nich normal an:

$$3x^2=75$$

$$\left(x-1\right)\left(x+3\right)=0 $$

$$\left(2x+1\right)\left(3x-4\right)\left(2x+3\right)=0 $$

Titel quadratische Gleichungen: Substitution sollte anders lauten
weil $(x-3/x)^2$ -> Polynomgleichung besser: ``Gleichungen höheren
Grades'' (Es sind keine quadratischen Gleichungen, wenn $x^6$ etc.)

Bei ``Elementare Potenzgleichungen'' als start $x^4 = 64$

$$(x-2)^2 = 144$$


Elementare Exponentialgleichungen 

Bei der jetztigen Aufgabe G132 (version Ch. Hersberger) nur Aufgab
a) d) f) g) h)    [Aufgabe i) ist noch falsch?? Parameteraufgabe kübeln]
j) kübeln, da zu einfach mit TR

Zusätzlich $3\cdot{} 4^x = 7$
und $2\cdot{}3^(2x+1)=37 $






Bei Exponentiagleichungen mit erzeugen gleicher Basis:
noch so was:

$$e^(3x+3)*e^(x-1)=e^2 $$


$$(1/5)^x = (1/125)$$

$$3^(x+2)=1/3$$
 
$$3^(4x+1)*(3^2)^x=3^(5x+1) $$

$$2*3^x = 5*7^x$$

$$5^(x+1)=3^(x-2) $$


