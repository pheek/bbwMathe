\renewcommand{\kAufgabenBuchstabe}{DRAFT}
\section{Neue Aufgaben}
\setcounter{aufgabenNummer}{1000}


\kTrainingAufgabe{\eng{translate}

\kKommentar{Nicht nur Bruchgleichungen mit Brüchen}
  
Lineare Gleichngen (noch testen, ob sie aufgehen):

a)
$$ \frac{2x + 3}{x + 2} = \frac{2x - 4}{x - 1}$$

b)
$$\left(x+5\right)^2-\left(x-4\right)^2=8x+3-\left(3x+2\right) $$

\kKommentar{neue Aufgabe}

}{%% Lösungen
}

\kTrainingAufgabe{\eng{translate}

\kKommentar{Folgender Stil fehlt}
  
Bruchgleichungen:

$$\frac{2x}{x+4} + \frac{1}{x} = 6 $$

$$\frac{2x+1}{x-1} + \frac{3x+1}{1-x} = \frac{3}{4} $$

$$\frac{3x+1}{2x+2} - \frac{3x+1}{x+1} = \frac{4x-2}{3x+3} $$

}{%% Lösungen
  \kKommentar{Lösungen ausstehend}
}

%\kKommentar{Grundsätzlich $x$ als Lösungsvaraible (nicht auf einmal $a$).}

\kTrainingAufgabe{\eng{translate}

  \kKommentar{folgender Typ fehlt}
  
  Bei Linearen Gleichungen auch was mit Klammern.

  Lösen Sie die Gleichungen nach $x$ auf:
  
 $$7-2(3-5x(x+2))=3-(2x+3-(8+10x^2)) $$

  $$2 - 3 \cdot \frac{x - 1}{2} = 2 \cdot \left(\frac{x + 8}{3} - \frac{1}{3}\right)$$
}{%% Lösungen
  .... coming soon ...
}


\kKommentar{
  Einfachere Start Leistungsaufgabe als die Zuleitung mit 9h 36'.}


\kKommentar{Lineare Bruchgleichungssysteme:}
Hier hat es auch Bruchgleichungssysteme

Textaufgaben: 

„Ein Swimmingpool wird von zwei Pumpen gefüllt. Die grosse Pumpe braucht alleine 4 Stunden für die Füllung, die kleine Pumpe 6 Stunden. Wie lange dauert die Füllung, wenn beide Pumpen gleichzeitig arbeiten?“ 

 

„Ein Druckauftrag wird von zwei Druckern erledigt. Zusammen brauchen die beiden Drucker 5 Stunden. Arbeitet nur der langsamere Drucker, so braucht er 2 Stunden länger für den ganzen Druckauftrag, als der schnelle Drucker alleine braucht. Wie lange braucht der langsame Drucker, wenn er alleine arbeitet?“ 

=> zahlen anpassen!!



Stinknormales Gleichungssystem in Grundform (als Trainingsaufgabe)
fehlt.

Lineare gleichungssysteme: 

$$4\cdot{}x+y=5, 2\cdot{}x-3\cdot{}y=13$$
oder so:

$$X+y=5/6, x=y+1/3 $$


\kKommentar{ Bei Gleichungssystemen sind noch}

\kKommentar{Bruchgleichungssysteme. Diese in der richtigen ``Schublade'' ablegen.}

