%% Layout for Kompendium
%% (c) Nov. 2023 philipp.freimann@bms-.ch

%%%%%%%%%%%%%%%  H E A D E R   &   F O O T E R %%%%%%%%%%%%%%%%%%%%
\fancyhf[HL]{Aufgabensammlung GESO} 
\fancyhf[HR]{BMS Winterthur (\texttt{https://www.bmwin.ch})}
\fancyhf[FR]{\leftmark}
\fancyhf[FL]{\thepage{}/\pageref{LastPage}}
\fancyhf[FC]{}


%% Einzug bei Paragraphen unterdrücken, damit alle Aufgabennummern
%% ganz links kommen
\setlength{\parindent}{0pt}


%%%%%%%%%%%%%%%%%%%%%%%%%%%%%%%%%%%%%%%%%%%%%%%%%%%%%%%%%%%%%%%%%%%
%% Layout erste Seite
\newcommand{\printKFirstPage}{%%
  \thispagestyle{empty}

\begin{center}{\huge{Diskussionsgrundlage}}\end{center}

  \begin{center} 
\raisebox{-30mm}{\includegraphics[width=50mm]{img/layout/loewe.pdf}}
{\huge{\textbf{Aufgabensammlung\hspace{50mm}\,}}}\

\kKommentar{WORKING DRAFT} \end{center}

  \vspace{5mm}
  
  \begin{center} {\large{GESO}} (BMS Gesundheitlich-Soziale Richtung) \end{center}

  \kKommentar{KOMMENTIERTE VERSION (NICHT für SCHüLER/ERS/INNEN)}

  \begin{center}{\textbf{%% center und fett
    \isLoesungen{\deu{Lösungsteil}\eng{Solutions}}
    \isAufgaben{\deu{Aufgabenteil}\eng{Collection of Tasks}}
    }}%% fett
  \end{center}

  \vspace{5mm}

  \begin{center} \deu{---} \eng{english version} \end{center}

\begin{center} \auchTrainingsAufgabe{\deu{Angereichert mit
Einstiegsaufgaben/Trainingsaufgaben \,}\eng{Enhanced with simple
excerises.} \colorbox{blue!30}{(T)}} . \end{center}


  \vspace{10mm}
  \vspace{15mm}

  \begin{center}Version [\kVersion{}] Kanton Zürich \end{center}
\kKommentar{A.B.C}

\kKommentar{A=Version: Pro Jahr max eine Version, Kapitelstrukturänderung}

\kKommentar{B=Sub-Version: Formulierungen, Reihenfolgen etc.}

\kKommentar{C=Revision: Fehlerkorrektur}
%\kKommentar{Nur \textbf{eine} Version im Kanton?}

Quelltext des Skripts:\\
%%{\small \texttt{https://github.com/pheek/bbwMathe/tree/main/arbeitsblaetter/GESO\_Kompendium}}
{\small \texttt{https:// eg github.com/***Kompendium} (quelltext Online)}
  
\newpage

\auchTrainingsAufgabe{%%

\begin{tabular}{c|p{18cm}}

{\colorbox{blue!30}{(T)}}  & 
 \deu{lediglich fürs Training, Einstiegsaufgaben}%% end deu
 \eng{only for training purpouse.}%% end eng
\\\hline

{\colorbox{red!30}{(N)}} & Y
  \deu{Aufgaben mit Niveau der BMP (Berufsmaturitätsprüfung).}%% end deu
  \eng{level BMP (Berufsmaturitätsprüfung).}
%}%% end eng
\\
\end{tabular}

}%% end INKL Trainingsaufgaben

\kKommentar{Aufgabennummern ab Kapitel durchnummeriert}

\kKommentar{\hspace{5mm}startet je bei modulo 100 (bei 000 + 1)}

\kKommentar{Nutzungsidee angeben:}

(Wie ist das Kompendium zu verstehen?)

Worauf können sich die Lehrpersonen verlassen zur Prüfungsvorbereitung
und Jahresplanung?
\begin{itemize}
\item \textbf{Kompendium!}
\item Rahmenlehrplan (\textbf{RLP})?
\item Alte Abschlussprüfungen?
\item Nachreiche-Blatt Susanne Wagner?
\end{itemize}

\vspace{30mm}
\begin{center}\begin{small} \textbf{\deu{Autoren} \eng{Authors}}\\
Susanne Wagner (susanne.wagner@bms-zuerich.ch)\\
Urs Vonesch\\
Thomas Fellmann (thomas.fellmann@bms-zuerich.ch)\\
Mirjam Bräm, Wolfgang Pfalzgraf, Ulrike Gruber\\
Christian Hersberger (christian.hersberger@bmwin.ch)\\
(ph. freimann)\\
\vspace{2mm}
Nutzungsrechte: CC-NC-BY (2015-2024)

\kKommentar{ CC darf rechtlich niemals entfernt werden!}
\end{small}%%
\end{center}

\newpage


\tableofcontents{}
\newpage
\pagestyle{fancy}
}%% end firstPage
