%% Layout for Kompendium
%% (c) Nov. 2023 philipp.freimann@bbw.ch

%%%%%%%%%%%%%%%  H E A D E R   &   F O O T E R %%%%%%%%%%%%%%%%%%%%
\fancyhf[HL]{Kompendium} 
\fancyhf[HR]{BMS Winterthur (\texttt{https://www.bmwin.ch})}
\fancyhf[FR]{\leftmark}
\fancyhf[FL]{\thepage{}/\pageref{LastPage}}
\fancyhf[FC]{}


%% Einzug bei Paragraphen unterdrücken, damit alle Aufgabennummern
%% ganz links kommen
\setlength{\parindent}{0pt}


%%%%%%%%%%%%%%%%%%%%%%%%%%%%%%%%%%%%%%%%%%%%%%%%%%%%%%%%%%%%%%%%%%%
%% Layout erste Seite
\newcommand{\printKFirstPage}{%%
  \thispagestyle{empty}

  \begin{center} 
\raisebox{-30mm}{\includegraphics[width=50mm]{img/layout/loewe.pdf}}
{\huge{\textbf{Kompendium\hspace{50mm}\,}}}\\\kKommentar{WORKING DRAFT} \end{center}

  \vspace{5mm}
  
  \begin{center} {\large{GESO}} \end{center}

  \kKommentar{KOMMENTIERTE VERSION (NICHT für SCHüLER/ERS/INNEN)}

  \begin{center}
    \isLoesungen{\deu{Lösungsteil}\eng{Solutions}}
    \isAufgaben{\deu{Aufgabenteil}\eng{Collection of Tasks}}
  \end{center}

  \vspace{5mm}

  \begin{center} \deu{---} \eng{english version} \end{center}

\begin{center} \auchTrainingsAufgabe{\deu{Angereichert mit
Einstiegsaufgaben.}\eng{Enhanced with simple excerises.} \colorbox{blue!30}{(E)}} \end{center}


  \vspace{10mm}

\begin{center}\begin{small} \deu{Autoren} \eng{Authors}\\
Susanne Wagner (susanne.wagner@bms-zuerich.ch)\\
Urs Vonesch\\
Thomas Fellmann (thomas.fellmann@bms-zuerich.ch)\\
Mirjam Bräm, Wolfgang Pfalzgraf, Ulrike Gruber\\
Christian Hersberger (christian.hersberger@bmwin.ch)\\
\vspace{2mm}
Nutzungsrechte: CC-NC-BY\kKommentar{ CC darf nicht entfernt werden!}\\
 \end{small}
  \end{center}


  \vspace{15mm}

  \begin{center}Kantonale Ausgabe \kVersion{} Kanton Zürich \end{center}

\kKommentar{Nur \textbf{eine} Version im Kanton?}

Quelltext des Skripts:\\
%%{\small \texttt{https://github.com/pheek/bbwMathe/tree/main/arbeitsblaetter/GESO\_Kompendium}}
{\small \texttt{https:// eg github.com/***Kompendium} (quelltext Online)}
  
\newpage

\auchTrainingsAufgabe{
%%
\deu{Mit {\colorbox{blue!30}{(E)}} gekennzeichnete Aufgaben dienen
lediglich dem Training als Einstiegsaufgaben.\\
Mit \colorbox{red!30}{(B)} gekennzeichnete Aufgaben haben das Niveau
  der BMP (Berufsmaturitätsprüfung).
}%% end deu
\eng{Questions having \colorbox{blue!30}{(E)} are only for
    training purpouse.\\
Questions marked with  \colorbox{red!30}{(B)} have the level BMP (Berufsmaturitätsprüfung).
}%% end eng

\kKommentar{Aufgabennummern ab Kapitel durchnummeriert;}

\kKommentar{\hspace{5mm}startet je bei modulo 100 (bei 000 + 1)}

\newpage
}%% end inklusive TrainingsAufgaben

\tableofcontents{}
\newpage
\pagestyle{fancy}
}%% end firstPage
