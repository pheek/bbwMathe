%% Kompendium Kapitel Gleichungen
%% (c) Nov. 2023 philipp.freimann@bbw.ch

\deu{\section{Gleichungen}\index{Gleichungen}}
\eng{\section{Equations}\index{Equations}}
\setcounter{aufgabenNummer}{100}
  \renewcommand{\kAufgabenBuchstabe}{G}

\kMatheNinjaLink{\deu{Gleichungen}\eng{Equations}}{https://matheninja.ch/linearegleichungen/}

\subsection{\deu{Lineare Gleichungen}\eng{Linear Equations}}
\subsubsection{\eng{Satisfiable and unsatisfiable linear equations}\deu{Erfüllbare und unerfüllbare lineare Gleichungen}}

%% %%%%%%%%%%%%%%%%%%%%%%% Aufgabe %%%%%%%%%%%%%%%%%%%%%%%%%%%%%%%%%%%%%%%%%%%%%%%%

\kTrainingAufgabe{
\eng{Determine the solution set with respect to the universal set}\deu{Bestimmen Sie die Lösungsmenge bezüglich der
Grundmenge}
$\grundmenge=\mathbb{R}$:
\\

\begin{enumerate}[label=\alph*)]
\item $4x=7+3x$
\item $3x=x$
\item $(x+2)^2 + (x+3)^2 - (x+5)^2 = (x-2)^2 - 12$
\item $6x + 5 = \frac{1}{3} (18x + 15)$
\item $8x -2 = 4x + 1 - (-4x - 3.5)$
\end{enumerate}
\kVerweiseAltesKompendium{10}{1. - 5.}
\kPlatzFuerBerechnungen{6}
}{ %% Lösungsteil
\begin{enumerate}[label=\alph*)]
\item $\Lx = \{7\}$
\item $\Lx = \{0\}$
\item $\Lx = \{1\}$
\item $\Lx = \mathbb{R}$
\item $\Lx = \{\}$
\end{enumerate}
}%%


%% %%%%%%%%%%%%%%%%%%%%%%% Aufgabe %%%%%%%%%%%%%%%%%%%%%%%%%%%%%%%%%%%%%%%%%%%%%%%%

\kNiveauAufgabe{\eng{Determine the solution set with respect to the universal set}\eng{Bestimmen Sie die Lösungsmenge bezüglich der
Grundmenge}
$\grundmenge=\mathbb{R}$:
\\
$\frac{3x-1}{4} + \frac{2x+1}{5} - 1 = \frac{3x-1}{5} +6 - \frac{x+2}{3}$
\kVerweiseAltesKompendium{10}{6}
\kPlatzFuerBerechnungen{6}
}{ %% Lösungsteil
$\Lx = \{7\}$
}%%




\subsubsection{\eng{Linear equations with parameters}\deu{Lineare Gleichungen mit Parametern}}


\deu{Bestimmen Sie die Lösungen. Schreiben Sie die Ergebnisse so einfach
wie möglich. Die Lösungsvariable ist jeweils $x$:}
\eng{Determine the solutions. Write the results as simply as possible. The solution variable is denoted as $x$:}

%% %%%%%%%%%%%%%%%%%%%%%%% Aufgabe %%%%%%%%%%%%%%%%%%%%%%%%%%%%%%%%%%%%%%%%%%%%%%%%

\kNiveauAufgabe{
\begin{enumerate}[label=\alph*)]
\item $3x+a = x+9a$
\item $(t+x-s)(t-x-s) = (t-x)(s+x) - st$
\item $\frac{k-bx}{v} + x = \frac{vx-b}{v}$
\item $(k-4)x = 4k - k^2$
\item $\frac{1}{x} + \frac{1}{a} = \frac{1}{b}$
%%\item $\frac{x- \frac{n^2}{4}}{\frac{1}{2} - \frac{n}{4}} = x+n$
\end{enumerate}
\kVerweiseAltesKompendium{11}{12.-15.}
\kKommentar{Aufgabe 16. entfernt, da nicht RLP}
\kPlatzFuerBerechnungen{6}
}{%% Lösungstei
\eng{translate:} Eine Fallunterscheidug wird nicht verlangt, ist aber dennoch in
den Lösungen angegeben.
\begin{enumerate}[label=\alph*)]
\item $\Lx = \left\{ 4a  \right\}$
\item $\Lx = \left\{ t-s \right\}$ für $t\ne s$. Ist $t=s$, so ist $\Lx  = \mathbb{R}$.
\item $\Lx = \left\{ \frac{k+b}b \right\}$ für $b\ne 0$ und $v\ne
0$. \deu{Ansonsten gibt es Spezialfälle.}\eng{Otherwise, there are special cases.}
\item $\Lx = \left\{ -k  \right\}$ \deu{für}\eng{for} $k\ne 4$.
\item $\Lx = \left\{ \frac{ab}{a-b} \right\}$ für $a\ne b$.
%%\item $\frac{x- \frac{n^2}{4}}{\frac{1}{2} - \frac{n}{4}} = x+n$
\end{enumerate}
%%
}


\subsubsection{\eng{Word problems leading to linear equations}\deu{Textaufgaben, die auf lineare Gleichungen führen}}

%% %%%%%%%%%%%%%%%%%%%%%%% Aufgabe %%%%%%%%%%%%%%%%%%%%%%%%%%%%%%%%%%%%%%%%%%%%%%%%

\kNiveauAufgabe{%%
\deu{Zerlegen Sie die Zahl 188 so in zwei Summanden, dass der 3. Teil des
  ersten Summanden um 12 kleiner ist als der 5. Teil des zweiten
  Summanden. Wie heissen die beiden Summanden?}
  \eng{Divide the number 188 into two summands in such a way that the 3rd part of the first summand is 12 less than the 5th part of the second summand. What are the two summands?}
\kVerweiseAltesKompendium{11}{17}
\kPlatzFuerBerechnungen{6}
}{%% Lösung
\deu{Die Summanden sind 48 und 140.}\eng{The summands are 48 and 140.}
}


%% %%%%%%%%%%%%%%%%%%%%%%% Aufgabe %%%%%%%%%%%%%%%%%%%%%%%%%%%%%%%%%%%%%%%%%%%%%%%%

\kNiveauAufgabe{%%
\deu{Urs addiert zu einer gedachte Zahl $x$ die Zahl 5, setzt recht
  an die Summe die Ziffer 7, dividiert die entstandene Zahl durch 11,
  multipliziert den Quotienten mit 4 und erhält das 14-fache der
  ursprünglichen Zahl. Um welche Zahl $x$ handelt es sich?}

\eng{Urs adds the number 5 to an imaginary number $x$, places the digit 7 to the right of the sum, divides the resulting number by 11, multiplies the quotient by 4, and obtains 14 times the original number. What is the value of the number $x$?}
\kVerweiseAltesKompendium{11}{18}
\kPlatzFuerBerechnungen{6}
}{%% Lösung
\deu{Die gedachte Zahl ist 2.}\eng{The imaginary number is 2.}
}


%% %%%%%%%%%%%%%%%%%%%%%%% Aufgabe %%%%%%%%%%%%%%%%%%%%%%%%%%%%%%%%%%%%%%%%%%%%%%%%

\kNiveauAufgabe{%%
\deu{Jemand muss zwei Kredite von zusammen CHF 180\,000.- zu 4.5\%
  und 5.5\% verzinsen. Wären die Zinssätze vertauscht, so ergäbe sich
  ein um CHF 225.- niedriger Jahreszins. Wie hoch sind die Kredite?}
  
\eng{Someone has to accrue interest on two loans totaling CHF 180\,000
  at rates of 4.5\% and 5.5\%. If the interest rates were swapped, the
  annual interest would be CHF 225 lower. What are the amounts of the
  loans?}
\kVerweiseAltesKompendium{11}{19}
\kPlatzFuerBerechnungen{6}  
}{%% Lösung
\deu{Die beiden Kredite betragen CHF 78’750.- und CHF 101’250.-.}
\eng{The two loans amount to CHF 78,750 and CHF 101,250.}
}


%% %%%%%%%%%%%%%%%%%%%%%%% Aufgabe %%%%%%%%%%%%%%%%%%%%%%%%%%%%%%%%%%%%%%%%%%%%%%%%

\kNiveauAufgabe{%%
\deu{Eine Buchhandlung verkaufte von den vorhandenen Exemplaren eines
  neu erschienen Krimis am ersten Tag ein Achtel und 10 Stück, am
  zweiten Tag vom Restbestand die Hälfte und 15 Stück. Danach blieben
  noch 50 Krimis übrig. Wie viele Exemplare waren ursprünglich
  vorhanden?}

\eng{A bookstore sold one-eighth and 10 copies of the available
  copies of a newly released crime novel on the first day, and on the
  second day, half of the remaining stock and 15 copies were
  sold. Afterward, 50 crime novels were left. How many copies were
  originally available?}
\kVerweiseAltesKompendium{12}{21}
\kPlatzFuerBerechnungen{6}
  
}{%% Lösung
\deu{Ursprünglich waren 160 Krimis vorhanden.}
\eng{Originally, there were 160 crime novels available.}
}

%% %%%%%%%%%%%%%%%%%%%%%%% Aufgabe %%%%%%%%%%%%%%%%%%%%%%%%%%%%%%%%%%%%%%%%%%%%%%%%

\kNiveauAufgabe{%%
 \deu{Ein Kaufmann mischt 6\,kg einer Warensorte mit 14\,kg einer
  zweiten Sorte. Der Preis je kg der ersten Sorte ist um CHF 2.50
  höher als derjenige der zweiten Sorte. Er erhält eine Mischung, die
  CHF 8.- pro kg kostet. Wie teuer ist jede der beiden Sorten pro kg?}

\eng{A merchant mixes 6 kg of one type of goods with 14 kg of a second type. The price per kg of the first type is CHF 2.50 higher than that of the second type. He obtains a mixture that costs CHF 8 per kg. What is the cost of each of the two types per kg?}
\kVerweiseAltesKompendium{12}{22}
\kPlatzFuerBerechnungen{6}
}{%% Lösung
\deu{Die erste Sorte kostet CHF 9.75 und diezweite Sorte CHF 7.25.}
\eng{The first type costs CHF 9.75, and the second type costs CHF 7.25.}
}


%% %%%%%%%%%%%%%%%%%%%%%%% Aufgabe %%%%%%%%%%%%%%%%%%%%%%%%%%%%%%%%%%%%%%%%%%%%%%%%

\kNiveauAufgabe{%%
\deu{Aus 60\%-igem und 65\%-igem Alkohol sollen 150 Liter 63\%-ige
  Mischung hergestellt werden. Wie viele Liter von jeder Sorte muss
  man verwenden?}
  \eng{A 63\% mixture is to be produced from 60\% alcohol and 65\% alcohol, with a total quantity of 150 liters. How many liters of each type must be used?}
\kVerweiseAltesKompendium{12}{23}
\kPlatzFuerBerechnungen{6}
}{%% Lösung
\deu{Von der ersten Sorte nimmt man 60 Liter, von der zweiten 90
  Liter.}
  \eng{60 liters of the first type and 90 liters of the second type are used.}
}



%% %%%%%%%%%%%%%%%%%%%%%%% Aufgabe %%%%%%%%%%%%%%%%%%%%%%%%%%%%%%%%%%%%%%%%%%%%%%%%

\kNiveauAufgabe{%%
\deu{Auf dem Flohmarkt wird um ein Ölbild gefeilscht. Der Händler
  verlangt CHF 590.-, während der Käufer nur CHF 410.- bezahlen
  will. Die beiden einigen sich so, dass der Händler den Preis um
  gleich viele Prozente senkt, wie der Käufer sein Angebot
  erhöht. Welches ist der Verkaufspreis und um wie viele Prozente sind
  beide von ihren Forderungen abgewichen?}
  \eng{At a flea market, there is bargaining over an oil painting. The merchant asks for CHF 590, while the buyer is only willing to pay CHF 410. They reach an agreement where the merchant reduces the price by the same percentage that the buyer increases their offer. What is the selling price, and by how many percentage points did both deviate from their initial demands?}
\kVerweiseAltesKompendium{12}{24}
\kPlatzFuerBerechnungen{6}
}{%% Lösung
\deu{Sie wichen je 18\% von ihrer Forderung ab. Der Verkaufspreis
  betrug CHF 483.80.}
  \eng{They each deviated by 18\% from their demands. The selling price was CHF 483.80.}
}


%% %%%%%%%%%%%%%%%%%%%%%%% Aufgabe %%%%%%%%%%%%%%%%%%%%%%%%%%%%%%%%%%%%%%%%%%%%%%%%

\kNiveauAufgabe{%%
\deu{Wenn zwei Zuleitungen gleichzeitig geöffnet sind, füllen sie ein
  Bassin in 9 Minuten und 36 Sekunden. Um das Bassin alleine zu
  füllen, braucht die erste Zuleitung 12 Minuten. Wie lange benötigt
  die zweite Zuleitung, wenn sie das Bassin alleine füllen soll?}
  \eng{When two supply lines are open simultaneously, they fill a basin in 9 minutes and 36 seconds. To fill the basin alone, the first supply line takes 12 minutes. How long does the second supply line need to fill the basin alone?}
\kVerweiseAltesKompendium{12}{25}
\kPlatzFuerBerechnungen{6}
}{%% Lösung
\deu{Die zweite Leitung hätte alleine 48 min.}\eng{The second supply line would have needed 48 minutes alone.}
}




%% %%%%%%%%%%%%%%%%%%%%%%% Aufgabe %%%%%%%%%%%%%%%%%%%%%%%%%%%%%%%%%%%%%%%%%%%%%%%%

\kNiveauAufgabe{%%
\deu{Eine Weinhändlerin kauft 300 Liter Rotwein und 600 Liter
  Weisswein. Der Preis pro Liter Weisswein ist um CHF 1.50 höher als der
  Literpreis Rotwein. Weil der Weinbauer und die Händlerin alte
  Freunde aus der BM-Schulzeit sind, erhält die Weinhändlerin pro
  Liter Rotwein 10\% und pro Liter Weisswein 12\% Rabatt.
  Der Weinbauer vertauscht jedoch die beiden Rabatte und schickt der
  Händlerin somit eine Rechnung, die um CHF 60.- zu hoch ist.
  Berechnen Sie die Literpreise für Rot- bzw. Weisswein vor Abzug der
  Rabattes.}
  \eng{A wine merchant buys 300 liters of red wine and 600 liters of white wine. The price per liter of white wine is CHF 1.50 higher than the price per liter of red wine. Because the winemaker and the merchant are old friends from their high school days, the wine merchant receives a 10\% discount per liter of red wine and a 12\% discount per liter of white wine. However, the winemaker mistakenly swaps the two discounts and sends the merchant an invoice that is CHF 60 too high. Calculate the per-liter prices for red and white wine before applying the discounts.}
\kVerweiseAltesKompendium{12}{26}
\kPlatzFuerBerechnungen{6}
}{%% Lösung
\deu{Der Liter Rotwein kostet CHF 7.-, der Liter Weisswein CHF 8.50.}
\eng{The price per liter of red wine is CHF 7, and the price per liter of white wine is CHF 8.50.}
}




\subsection{\eng{Systems of equations with two unknowns.}\deu{Gleichungssysteme mit zwei Unbekannten}}

\kMatheNinjaLink{\deu{Gleichungssysteme}\eng{Systems of equations}}{https://matheninja.ch/linearegleichungssysteme/}

\deu{Lösen Sie die linearen Gleichungssysteme mithilfe eines geeigneten Lösungsverfahrens
(Additions-, Einsetzungs-, Gleichsetzungsverfahren). Bestimmen Sie die Lösungsmenge bezüglich der Grundmenge}
\eng{Solve the systems of linear equations using an appropriate
solution method (addition, equating methods). Determine
the solution set with respect to the universal set}
$\grundmenge = \mathbb{R}\times\mathbb{R}\}$



\kKommentar{Substitution nicht RLP}
%%Falls Substitutionsaufgaben an der Schlussprüfung vorkommen, werden
%%die Fachschaften vorab informiert.

%% %%%%%%%%%%%%%%%%%%%%%%%%% Aufgabe %%%%%%%%%%%%%%%%%%%%%%%%%%%%%%%%%%%%%%%%%%%%%%%%

\kNiveauAufgabe{%%
\begin{multicols}{2}
  \begin{enumerate}[label=\alph*)]
  \item
    \gleichungZZ{-\frac{x}{2}}{2y-1+x}{\frac{x}{2}+3}{1-x}
  \item
    \gleichungZZ{\frac{a+b}{6}}{\frac{a}{2}-5}{a+8}{\frac{b+a}{6}}
  \item
  \gleichungZZ{5x-8y}{1}{20x}{7+32y}
\item
  $$2x+7y = -5+8y = 24x +7$$
\item
  \gleichungZZ{3x}{2+5y}{6}{\frac{24}{3x} + \frac{10y}{x}}
\item
  \gleichungZZ{\frac{4x-1}{5}-x}{y-\frac{3y-4}{2}}{\frac{x+y-3}{2x-y+7}}{\frac{3}{4}}
  \end{enumerate}
\end{multicols}
\kVerweiseAltesKompendium{13}{27. - 32}
\kPlatzFuerBerechnungen{6}
}{% Lösung
\begin{multicols}{2}
  \begin{enumerate}[label=\alph*)]
  \item
    $\LoesungsMenge=\left\{(10; 7)\right\}$
  \item
    $\LoesungsMenge=\left\{(-26; -82)\right\}$
  \item
    $\LoesungsMenge=\left\{ \right\}$
\item
    $\LoesungsMenge=\left\{\left(\frac72; 12\right)\right\}$
\item
    $\LoesungsMenge=\left\{ \right\}$
\item
    $\LoesungsMenge=\left\{\left(\frac{11}4; \frac{11}2\right)\right\}$
  \end{enumerate}
\end{multicols}
}

%% Substitution nicht im Rahmenlehrplan:
\entfernteAufgabe{
\begin{multicols}{2}
  \begin{enumerate}[label=\alph*)]
  \item
    \gleichungZZ{\frac{b}{a+b}}{1-\frac{3}{a-b}}{\frac{3b}{a+b}}{1-\frac{2}{a-b}}

  \item
    \gleichungZZ{\frac{2y}{4x-1}+9}{\frac{5}{x+2y}}{\frac{10}{x+2y}-12}{\frac{y}{4x-1}}
  \end{enumerate}
\end{multicols}
\kVerweiseAltesKompendium{14}{33. - 34.}
\kPlatzFuerBerechnungen{6}
}{}%%%



\subsubsection{\deu{Textaufgaben, die auf Gleichungssysteme mit zwei Unbekannten führen}\eng{Word problems leading to systems of equations with two unknowns}}

%% %%%%%%%%%%%%%%%%%%%%%%% Aufgabe %%%%%%%%%%%%%%%%%%%%%%%%%%%%%%%%%%%%%%%%%%%%%%%%

\kNiveauAufgabe{%%
\deu{Farben der Sorte A und der Sorte B mit Kilopreisen von CHF 12.-
    bzw. CHF 10.- werden gemischt. Ein Kilogramm Mischung kommt nun
    auf CHF 10.90 zu stehen. Nähme man bei gleicher Gesamtmenge der
    Mischen von der Sorte A 3 kg weniger, so würde das Kilogramm nur
    noch CHF 10.40 kosten. Wie viele kg nahm man anfänglich von jeder
    Sorte?}
    \eng{Colors of type A and type B with kilogram prices of CHF 12 and CHF 10, respectively, are mixed. One kilogram of the mixture now costs CHF 10.90. If, with the same total quantity of the mixture, 3 kg less of type A were taken, the cost per kilogram would only be CHF 10.40. How many kilograms were initially taken of each type?}
\kVerweiseAltesKompendium{14}{35}
\kPlatzFuerBerechnungen{6}
}{% Lösung
\deu{Anfänglich nahm man 5.4 kg von Sorte A und 6.6 kg von Sorte B.}
\eng{Initially, 5.4 kg of type A and 6.6 kg of type B were taken.}

}


%% %%%%%%%%%%%%%%%%%%%%%%% Aufgabe %%%%%%%%%%%%%%%%%%%%%%%%%%%%%%%%%%%%%%%%%%%%%%%%

\kNiveauAufgabe{%%
     \deu{Frau Gross hat ein Kapital in zwei Posten angelegt, einen zu
      4\% und einen zu 5\%. Nach ihrer Rechnung beträgt die Summe der
      Jahreszinsen CHF 2\,560. Das sind aber CHF 80.- zu viel; sie hat
      nämlich die Zinssätze verwechselt. Welche Posten hat sie zu
      welchem Zinssatz angelegt?}
      \eng{Mrs. Gross has invested capital in two items, one at 4\% and one at 5\%. According to her calculations, the sum of the annual interest amounts to CHF 2,560. However, this is CHF 80 too much; she has confused the interest rates. Which items did she invest in and at what interest rate?}
\kVerweiseAltesKompendium{14}{36}
\kPlatzFuerBerechnungen{6}
}{% Lösung
\deu{Sie hat richtigerweise CHF 32’000 zu 4\% und CHF 24’000 zu 5\%
      angelegt.}
      \eng{She correctly invested CHF 32,000 at 4\% and CHF 24,000 at 5\%.}
}


\kNiveauAufgabe{%%
\deu{Dividiert man eine zweiziffrige Zahl durch ihre Quersumme,
        so erhält man 4 Rest 9. Vertauscht man die Ziffern der
        ursprünglichen Zahl und dividiert diese durch die um 13
        vermehrte Quersumme, so erhält man 3.}
\eng{If you divide a two-digit number by its digit sum, the remainder is 9. If you swap the digits of the original number and divide it by the digit sum increased by 13, the result is 3.}
\kVerweiseAltesKompendium{14}{37}
\kPlatzFuerBerechnungen{6}
}{% Lösung
\deu{Die gesuchte Zahl heisst 57.}
\eng{The sought number is 57.}
}


%% %%%%%%%%%%%%%%%%%%%%%%% Aufgabe %%%%%%%%%%%%%%%%%%%%%%%%%%%%%%%%%%%%%%%%%%%%%%%%

\kNiveauAufgabe{%%
\deu{Eine Händlerin kaufte 800 Stück von Sorte I und 400
          Stück von Sorte II für insgesamt CHF 4\,200. Sorte I
          verkaufte sie mit einem Zuschlag von 15\% und Sorte II mit
          einem Zuschlag von 50\%. Der Verkaufspreis betrug insgesamt
          CHF 5\,180. Für wie viel Geld hatte sie ein Stück jeder
          Sorte eingekauft?}
\eng{A merchant bought 800 units of type I and 400 units of type II for a total of CHF 4,200. She sold type I with a markup of 15\% and type II with a markup of 50\%. The total selling price was CHF 5,180. How much money did she spend on one unit of each type?}      
\kVerweiseAltesKompendium{14}{38}
\kPlatzFuerBerechnungen{6}
}{% Lösung
\deu{Ein Stück der Sorte I kostet CHF 4.- und ein Stück der Sorte II
CHF 2.50.}
\eng{One unit of type I costs CHF 4, and one unit of type II costs CHF 2.50.}
}



\subsection{\deu{Quadratische Gleichungen}\eng{Quadratic
          Equations}}\deu{\index{Quadratische Gleichungen}}\eng{\index{quadratic Equations}}

\kMatheNinjaLink{Quadratsiche Gleichungen}{https://matheninja.ch/quadratischegleichungen/}

%  Lösen Sie die Gleichungen ohne Solve-Funktion des Taschenrechners,
%  indem Sie sie auf die Grundform bringe. Die Grundform kann
%  anschliessend mit dem entsprechenden Taschenrechnermodus gelöst
%  werden.

  \subsubsection{\deu{Reinquadratische Gleichungen}\eng{Pure quadratic equations}}

\deu{Bestimmen Sie jeweils $x$:}
\eng{Determine each $x$:}

%% %%%%%%%%%%%%%%%%%%%%%%% Aufgabe %%%%%%%%%%%%%%%%%%%%%%%%%%%%%%%%%%%%%%%%%%%%%%%%

\kTrainingAufgabe{
\begin{multicols}{2}
  \begin{enumerate}[label=\alph*)]
  \item $x^2 = 0.04$
  \item $x^2+4=0$
  \end{enumerate}
  \end{multicols}
\kVerweiseAltesKompendium{16}{42. - 43.}
\kPlatzFuerBerechnungen{6}
}{%% Lösung
\begin{multicols}{2}
  \begin{enumerate}[label=\alph*)]
  \item $\Lx = \{-0.2; +0.2\}$
  \item $\Lx = \{\}$
  \end{enumerate}
  \end{multicols}
}

\kNiveauAufgabe{
$$(7+x)(7-x)=(3x+2)^2-(2x+3)^2$$
\kVerweiseAltesKompendium{SS}{AA}
\kPlatzFuerBerechnungen{6}
}{%% Lösung
$\Lx = \{-3; +3\}$
}


\subsubsection{\deu{Lösungsformel}\eng{Quadratic Formula}}
  Grundform:\\
  \begin{tabular}{|c|c|}%%
    \hline%%
\deu{Grundform}\eng{Canonical form}: $ax^2 + bx +c = 0$ & \eng{quadratic formula}\deu{Lösungsformel}: $x_{1,2}
  = \frac{-b \pm \sqrt{b^2 - 4ac}}{2a}$\\%%
    \hline%%
    \end{tabular}%%

%% %%%%%%%%%%%%%%%%%%%%%%% Aufgabe %%%%%%%%%%%%%%%%%%%%%%%%%%%%%%%%%%%%%%%%%%%%%%%%

\kNiveauAufgabe{
\deu{Bestimmen Sie jeweils $x$ von Hand und geben Sie die Resultate
    exakt an:}
\eng{Determine each $x$ by hand and provide the results exactly:}
\begin{multicols}{2}
 \begin{enumerate}[label=\alph*)]
  \item $x^2-3x-10 = 0$
  \item $x^2-11x + 30 = 0$
  \item $0=12x^2-8x+\frac43$
  \item $3x^2+5x=-1$
  \item $-5x^2=20+7.5x$
  \item $\frac18 x^2 + 2.5x + 8 = 0$
 \end{enumerate}
\end{multicols}
\kVerweiseAltesKompendium{16}{46.-49}
\kPlatzFuerBerechnungen{6}
}{%% Lösungen
\begin{multicols}{2}
 \begin{enumerate}[label=\alph*)]
  \item $\Lx = \{-2; 5\}$
  \item $\Lx = \{5; 6\}$
  \item $\Lx = \left\{\frac13\right\}$
  \item $\Lx = \left\{ \frac{-5-\sqrt{13}}6;  \frac{\sqrt{13}-5}6 \right\}$
  \item $\Lx = \{\}$
  \item $\Lx = \{-4; 16\}$
 \end{enumerate}
\end{multicols}
}%% end Aufgabe



\subsubsection{Quadratische Gleichungen: Substitution (Optional, da nicht Rahmenlehrplan)}
\kKommentar{Substition nicht RLP}
Gleichungen höheren Grades, die durch Substitution auf eine
quadratische Gleichung führen. Falls diese Aufgabentypen in
Abschlusprüfungen auftauchen, werden die Fachschaften informiert.

%% %%%%%%%%%%%%%%%%%%%%%%% Aufgabe %%%%%%%%%%%%%%%%%%%%%%%%%%%%%%%%%%%%%%%%%%%%%%%%

\kNiveauAufgabe{
  \begin{enumerate}
  \item \deu{Lösen Sie die Gleichung nach $x$ auf. Resultate exakt:}\eng{Solve the equation for $x$. Results in exact form:}
    $$\left(x-\frac{3}{x}\right)^2 - 20 = x - \frac{3}{x}$$

  \item \deu{Lösen Sie die Gleichung nach $x$ auf und runden Sie das
    Resultat auf drei Dezimalstellen:}
    \eng{Solve the equation for $x$ and round the result to three decimal places:}
    $$(x+7)^6 = 8\cdot (x+7)^3 - 15$$

  \item $$(x+7)^8 = 14(x+7)^4 + 32$$
  \end{enumerate}
\kVerweiseAltesKompendium{17}{52.-54.}
\kPlatzFuerBerechnungen{6}
\kKommentar ev. entfernen, da nicht RLP
}{}%% Entfertn, da nicht Rahmenlehrplan

\subsubsection{\deu{Textaufgaben}\eng{Word
Problems}}\deu{\index{Textaufgaben}}\eng{\index{Word Problems}}

\deu{Die Gleichungen sind auf Grundform zu bringen und können anschliessend
mit dem entsprechenden Taschenrechnermodus gelöst werden.}

\eng{The equations need to be brought into standard form and can then be solved using the corresponding calculator mode.}

%% %%%%%%%%%%%%%%%%%%%%%%% Aufgabe %%%%%%%%%%%%%%%%%%%%%%%%%%%%%%%%%%%%%%%%%%%%%%%%

\kNiveauAufgabe{
\deu{
 Ein Speicher wird durch eine Zuleitung in einer bestimmten
    Zeit gefüllt. Das Entleeren durch eine andere Leitung dauert eine
    Minute länger. Aus Unachtsamkeit bleibt während des Füllens die
    andere Leitung geöffnet, deshalb ist der Speicher erst nach 756
    Minuten voll. Wie lange dauert die Leerung des Speichers bei
    geschlossener Zuleitung?}
    \eng{A tank is filled by one inlet in a certain time. Emptying it through another outlet takes one minute longer. Due to carelessness, the other outlet remains open during filling, so the tank is only full after 756 minutes. How long does it take to empty the tank with the inlet closed?}
\kVerweiseAltesKompendium{17}{55}
\kPlatzFuerBerechnungen{6}
}{%% Lösung
\deu{Die Leerung dauert 28 Minuten.}\eng{The emptying takes 28 minutes.}
}%%


%% %%%%%%%%%%%%%%%%%%%%%%% Aufgabe %%%%%%%%%%%%%%%%%%%%%%%%%%%%%%%%%%%%%%%%%%%%%%%%

\kNiveauAufgabe{
\deu{
Das Produkt der beiden kleinsten von sechs aufeinander
      folgenden natürlichen Zahlen ist dreimal so gross wie die Summe
      der vier übrigen Zahlen. Wie heisst die kleinste Zahl?}
      \eng{The product of the two smallest of six consecutive natural numbers is three times as large as the sum of the four remaining numbers. What is the smallest number?}
\kVerweiseAltesKompendium{17}{56}
\kPlatzFuerBerechnungen{6}
}{%% Lösung
\deu{Die kleinste Zahl heisst 14.}
\eng{The smallest number is 14.}
}%%

\kNiveauAufgabe{
\deu{Eine Praxiseinrichtung mit einem Anschaffungswert von CHF
        24\,000;.- wurde zweimal mit dem gleichen Prozentsatz
        abgeschrieben und hat zur Zeit einen Wert von CHF 16\,335.-.
        Wie viele Prozente beträgt der Abschreibungssatz?}

\eng{An office furnishing with an initial value of CHF 24,000.- was depreciated twice at the same percentage rate and currently has a value of CHF 16,335.-. What is the depreciation rate in percentage?}
\kVerweiseAltesKompendium{17}{57}
\kPlatzFuerBerechnungen{6}
}{%% Lösung
\deu{Der Abschreibungssatz beträgt}
\eng{The depreciation rate is} 17.5\%.
}%%

%% %%%%%%%%%%%%%%%%%%%%%%% Aufgabe %%%%%%%%%%%%%%%%%%%%%%%%%%%%%%%%%%%%%%%%%%%%%%%%

\kNiveauAufgabe{
\deu{Dividiert man die Summe zweier positiver Zahlen durch ihre
        Differenz, so erhält man 12.5\% der grösseren Zahl.
        Berechnen Sie die grössere Zahl, wenn die kleinere 42 ist.}
\kVerweiseAltesKompendium{18}{58}
\kPlatzFuerBerechnungen{6}
}{%% Lösung
\deu{Die gesuchte (größere) Zahl heisst}
\eng{The sought (greater) number is} 56.
}%%

\kNiveauAufgabe{
\deu{Die Zehnerziffer einer zweiziffrigen Zahl ist um 6
         kleiner als die Einerziffer. Die vierfache Summe der
          Quadrate der Ziffern ist gleich der fünfunzwanzigfachen
          Quersumme. Wie heisst die Zahl?}
\eng{The tens digit of a two-digit number is 6 less than the units digit. The four times the sum of the squares of the digits is equal to twenty-five times the digit sum. What is the number?}          
\kVerweiseAltesKompendium{18}{59}
\kPlatzFuerBerechnungen{6}
}{%% Lösung
\deu{Die gesuchte Zahl heisst}
\eng{The sought number is} 17.
}%%





\subsection{\eng{Fractional equations (Domain and solution set)}\deu{Bruchgleichungen (Definitions- und Lösungsmenge)}}

\kMatheNinjaLink{\deu{Bruchgleichungen}\eng{Fractional Equations}}{https://matheninja.ch/bruchgleichungen/}

%% %%%%%%%%%%%%%%%%%%%%%%% Aufgabe %%%%%%%%%%%%%%%%%%%%%%%%%%%%%%%%%%%%%%%%%%%%%%%%

\kNiveauAufgabe{
\deu{Nennen Sie die Definitionsmenge, lösen Sie anschliessend die Gleichung
und bestimmen Sie die Lösungsmenge; $\grundmenge=\mathbb{R}$. Exakte Werte
angeben:}
\eng{State the domain, then solve the equation, and determine the solution set; $\grundmenge=\mathbb{R}$. Provide exact values.}
\\
\begin{multicols}{2}
\begin{enumerate}[label=\alph*)]
%%\item $\frac{4a+6}{2a-10} + \frac{6a-43}{5-a} = -10$
\item $\frac{4a+6}{2a-10} + \frac{6a-43}{5-a} = -10$
\item $\frac{22.5}{a-3} + \frac{9}{-a^2 + 6a - 9} = 0$
\item $\frac{1}{x(x-4)} + \frac{1}{x(x+4)} = \frac{2}{x^2-16}$
\item $\frac{3.6x+8.5}{x^2-x-12} + \frac{5.2}{x-4} = \frac{1.9}{-x-3}$
%%\item $\frac{21}{-x+7} = 3 + \frac{1.5x}{\left(\frac{7-x}{2}\right)}$    
\end{enumerate}
\end{multicols}
\kVerweiseAltesKompendium{10}{7.-10.}
\kKommentar{Aufgabe 11 entfernt, weil Doppelbruch, nicht RLP}
\kPlatzFuerBerechnungen{10}
}{%% Lösungsteil
\begin{multicols}{2}
\begin{enumerate}[label=\alph*)]
%%\item $\frac{4a+6}{2a-10} + \frac{6a-43}{5-a} = -10$
\item $\definitionsmenge = \mathbb{R}\backslash \{5\} , \Lx  = \left\{\frac23\right\}$
\item $\definitionsmenge = \mathbb{R}\backslash \{3\} , \Lx  = \left\{3.4\right\}$
\item $\definitionsmenge = \mathbb{R}\backslash \{-4, 0, 4\} = \Lx$
\item $\definitionsmenge = \mathbb{R}\backslash \{-3, 4\} , \Lx  = \left\{\frac{-165}{107}\right\}$
%%\item $\frac{21}{-x+7} = 3 + \frac{1.5x}{\left(\frac{7-x}{2}\right)}$    
\end{enumerate}
\end{multicols}
}%% end NiveauAufgabe



%% %%%%%%%%%%%%%%%%%%%%%%% Aufgabe %%%%%%%%%%%%%%%%%%%%%%%%%%%%%%%%%%%%%%%%%%%%%%%%

\kNiveauAufgabe{
  $\frac{x-3}{x+3} + \frac{x+3}{x-3} = \frac{26}{x^2-9}$
\kVerweiseAltesKompendium{SS}{AA}
\kPlatzFuerBerechnungen{6}
}{% Lösung
$\Lx = \{-2; +2\}$ und $\definitionsmenge = \mathbb{R} \backslash \{-3; +3\}$
}


%% %%%%%%%%%%%%%%%%%%%%%%% Aufgabe %%%%%%%%%%%%%%%%%%%%%%%%%%%%%%%%%%%%%%%%%%%%%%%%

\kNiveauAufgabe{
\deu{Nennen Sie die Definitionsmenge, lösen Sie anschliessend die Gleichung
und bestimmen Sie die Lösungsmenge mit dem
Taschenrechner. $\grundmenge=\mathbb{R}$.}
\eng{State the domain, then solve the equation, and determine the solution set using the calculator. $\grundmenge=\mathbb{R}$.}
\\
\begin{multicols}{2}
\begin{enumerate}[label=\alph*)]
\item $\frac{x-4}{x-5x} = \frac{30-x^2}{x^2-5}$
\item $\frac{x}3 - \frac{x^2 -19}{3x+6} = 3 + \frac{4x-7}{6-3x}$
\end{enumerate}
\end{multicols}
\kVerweiseAltesKompendium{16}{50., 51.}
\kPlatzFuerBerechnungen{6}
}{%% Lösungsteil
\begin{multicols}{2}
\begin{enumerate}[label=\alph*)]
\item $\definitionsmenge = \mathbb{R}\backslash \{0; 5\},  \Lx = \{-3\}$
\item $\definitionsmenge = \mathbb{R}\backslash \{-2; +2 \} , \Lx
= \left\{\frac43; 4 \right\}$
\end{enumerate}
\end{multicols}
}%% end NiveauAufgabe




\subsection{\eng{Elementary Power Equations}\deu{Elementare
Potenzgleichungen}}

\kMatheNinjaLink{\deu{Potenzgleichungen}\eng{Power Equations}}{https://matheninja.ch/potenzgleichungen/}

\deu{Lösen Sie die Gleichung nach
$x$ auf und schreiben Sie die Resultate als exakte Werte ($\grundmenge =\mathbb{R}$).
Ein ausführlicher, schrittweiser Lösungsweg wird
verlangt. Prüfen Sie Ihr Resultat mit dem SOLVER des Taschenrechners.}
\eng{Solve the equation for $x$ without using the calculator's solve
function and express the results as exact values ($\grundmenge = \mathbb{R}$).A detailed, step-by-step solution is required.  Check your result with the calculator's SOLVER.}

%% %%%%%%%%%%%%%%%%%%%%%%% Aufgabe %%%%%%%%%%%%%%%%%%%%%%%%%%%%%%%%%%%%%%%%%%%%%%%%

\kTrainingAufgabe{
\begin{multicols}{3}
\begin{enumerate}[label=\alph*)]
\item $\frac{2}{3}x^3 = 12$
\item $x^5 = \frac{1}{32}$
\item $\left(\frac{1}{x}\right)^6 = 13$
\item $\left(\frac{x-5}{3}\right)^3=64$
\item $(x+5)^{\frac{1}{2} = 4}$
\item $\left(x-6\right)^{\frac{1}{3}} = 2$
\end{enumerate}
\end{multicols}
\kVerweiseAltesKompendium{18}{60.- 65}
\kPlatzFuerBerechnungen{6}
}{%% Lösungen
\begin{multicols}{3}
\begin{enumerate}[label=\alph*)]
\item $x = \sqrt[3]{18}$
\item $x = \frac12$
\item $\Lx = \left\{- \frac1{\sqrt[6]{13}} ; \frac1{\sqrt[6]{13}}  \right\}$
\item $x = 17$
\item $x = 11$
\item $x = 14$
\end{enumerate}
\end{multicols}
}%% end Aufgabe



\subsection{Exponential\deu{gleichungen}\eng{equations}}

\kMatheNinjaLink{Exponential\deu{gleichungen}\eng{ Equations}}{https://matheninja.ch/exponentialgleichungen/}

\textbf{\deu{Logarithmen}\eng{Logarithms}}

%% %%%%%%%%%%%%%%%%%%%%%%% Aufgabe %%%%%%%%%%%%%%%%%%%%%%%%%%%%%%%%%%%%%%%%%%%%%%%%

\kTrainingAufgabe{%%
\deu{Berechnen Sie die Exponenten. Resultate exakt angeben:}
\eng{Compute the exponents. Provide results exactly:}
\begin{multicols}{2}
\begin{enumerate}[label=\alph*)]
\item $a^x=b$
\item $4^x=12$
\end{enumerate}
\end{multicols}
\kKommentar{Neue Aufgabe?}
\kPlatzFuerBerechnungen{6}
}{%%
\begin{multicols}{2}
\begin{enumerate}[label=\alph*)]
\item $a^x= b  \Longrightarrow    x = \log_{a}(b)$
\item $4^x=12  \Longrightarrow    x = \log_{4}(12)$
\end{enumerate}
\end{multicols}
}%%

%% %%%%%%%%%%%%%%%%%%%%%%% Aufgabe %%%%%%%%%%%%%%%%%%%%%%%%%%%%%%%%%%%%%%%%%%%%%%%%

\kNiveauAufgabe{%%
\deu{Berechnen Sie die Exponenten. Resultate exakt angeben:}
\eng{Compute the exponents. Provide results exactly:}
$$4\cdot 3^{x+1}=6$$
\kKommentar{neue Aufgabe?}
\kPlatzFuerBerechnungen{2.5}
}{%%
$$4\cdot 3^{x+1}=6  \Longrightarrow    x = \log_{3}(2)$$
}%%


%% %%%%%%%%%%%%%%%%%%%%%%% Aufgabe %%%%%%%%%%%%%%%%%%%%%%%%%%%%%%%%%%%%%%%%%%%%%%%%

\kTrainingAufgabe{%%
\deu{Lösen Sie die Exponentialgleichungen durch Erzeugen gleicher
Basis:}
\eng{Solve the exponential equations by creating equal bases:}
\begin{multicols}{3}
\begin{enumerate}[label=\alph*)]
\item $2^x=\frac{1}{8}$
\item $2^{2x}=\frac{1}{8}$
\item $4^x=\frac{1}{32}$
\item $3^x=\frac{1}{81}$
  \item $9^x=\frac{1}{3}$
  \item $27^{2x}=\frac{1}{9}$
  \item $\left(3^x\right)^6 = \frac{1}{81}$
    \item $50\cdot 5^{n-1} + 3\cdot 5^{n+1} = 5^x$
    \item $\frac{25^k}{125^3} = 5^x$
      \item $2^x=\frac{8^4}{4^8}$
\end{enumerate}
\end{multicols}
\kVerweiseAltesKompendium{9}{16}
\kVerweiseAltesKompendium{19}{72}
\kPlatzFuerBerechnungen{6}
}{%%
\begin{multicols}{3}
\begin{enumerate}[label=\alph*)]
\item $\Lx = \{-3\}$
\item $\Lx = \left\{-\frac32\right\}$
\item $\Lx = \left\{-\frac52\right\}$
\item $\Lx = \{-4\}$
\item $\Lx = \left\{-\frac12\right\}$
\item $\Lx = \left\{-\frac23\right\}$
\item $\Lx = \{n+2\}$
\item $\Lx = \{2k-9\}$
\item $\Lx = \{-4\}$
\end{enumerate}
\end{multicols}
}%%




%% %%%%%%%%%%%%%%%%%%%%%%% Aufgabe %%%%%%%%%%%%%%%%%%%%%%%%%%%%%%%%%%%%%%%%%%%%%%%%



\kNiveauAufgabe{%%


\deu{Lösen Sie die Gleichung ohne Solve-Funktion des Taschenrechners nach
$x$ auf und schreiben Sie die Resultate als exakte Werte ($\grundmenge = \mathbb{R}$). \\
Tipp: Erzeugen gleicher Basis}

\eng{Solve the equation for $x$ without using the calculator's solve function and express the results as exact values ($\grundmenge = \mathbb{R}$).\\
Tip: Create equal bases.
}


\begin{multicols}{3}
\begin{enumerate}[label=\alph*)]
\item $2^x=16$
\item $\left(\frac{1}{4}\right)^x = 64$
\item $2^x\cdot 8^{x-1} = 32$
\item $\left(\frac{1}{3}\right)^x\cdot \left(\frac{3}{27}\right)^x = 3$
\item $e^{2x}=1$
\item $\frac{6^{x-2}}{36^{x+1}} = 216$

\end{enumerate}
\end{multicols}

\kVerweiseAltesKompendium{19}{66.-71.}
\kPlatzFuerBerechnungen{6}
}{%% Lösung
\begin{multicols}{3}
\begin{enumerate}[label=\alph*)]
\item $x = 4$
\item $x = -3$ 
\item $x  = 2$
\item $x = - \frac13$
\item $x = 0$
\item $x = -7$
\end{enumerate}
\end{multicols}
}%% end Aufgabe



%% %%%%%%%%%%%%%%%%%%%%%%% Aufgabe %%%%%%%%%%%%%%%%%%%%%%%%%%%%%%%%%%%%%%%%%%%%%%%%




\kNiveauAufgabe{%%


\deu{Lösen Sie die Gleichung nach
$t$ auf und schreiben Sie die Resultate als exakte Werte. Danach nähern Sie das Resultat mit der
$\log()$-Funktion des Taschenrechners auf zwei Dezimalen an.}%% end deu


\eng{Solve the equation for $t$  and express the results as exact
values. Afterward, approximate the result using the logarithm ($\log$) function on the calculator to three decimal places.
}%% end engl


\begin{multicols}{3}
\begin{enumerate}[label=\alph*)]

\item $200 = 100 \cdot{}  1.3^{\frac{t}4}$
\item $0.8 = 2.3 \cdot{}  0.98^{\frac{t}3}$

\end{enumerate}
\end{multicols}
\kKommentar{Neue Aufgabe...}
\kKommentar{... da zum Lösen von Exp-Funktionen nötig}

\kVerweiseAltesKompendium{19}{66.-71.}
\kPlatzFuerBerechnungen{6}
}{%% Lösung
\begin{multicols}{3}
\begin{enumerate}[label=\alph*)]
\item $ t = 4 \cdot{} \log_{1.3}\left(\frac{200}{100}\right) \approx 10.57$
\item $ t = 3 \cdot{} \log_{0.98}\left(\frac{0.8}{2.3}\right) \approx 156.82$
\end{enumerate}
\end{multicols}
}%% end Aufgabe
















%% %%%%%%%%%%%%%%%%%%%%%%% Aufgabe %%%%%%%%%%%%%%%%%%%%%%%%%%%%%%%%%%%%%%%%%%%%%%%%

\kNiveauAufgabe{%%


\deu{Berechnen Sie $x$. Schreiben Sie das Ergebnis möglichst einfach.}
\eng{Calculate $x$. Express the result as simply as possible.}

\begin{multicols}{2}
\begin{enumerate}[label=\alph*)]
\item $\frac14 \cdot 7^x = 25$
\item $4\cdot{} 3^{x+1} = 6$
\item $2^x + 2^{x+3} = \frac94$
\item $e^x = 3$
\item $\frac{e^{-x}}2  -3 =0$
\end{enumerate}
\end{multicols}
\kKommentar{neue Aufgaben???}
\kPlatzFuerBerechnungen{6}
}{%% Lösung
\begin{multicols}{2}
\begin{enumerate}[label=\alph*)]
\item $x  =\log_{7}(100)= \frac{\lg(100)}{\lg(7)} = \frac2{\lg(7)}$ 
\item $x  = \log_3(0.5) = \frac{\log(0.5)}{\log(3)}$
\item $x  = -2$
\item $x  = \ln(3)$
\item $x  = \ln(6)$
\end{enumerate}
\end{multicols}
}%% end Aufgabe


\newpage
