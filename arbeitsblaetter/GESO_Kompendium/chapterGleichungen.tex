%% Kompendium Kapitel Gleichungen
%% (c) Nov. 2023 philipp.freimann@bbw.ch

\deu{\section{Gleichungen}\index{Gleichungen}}
\eng{\section{Equations}\index{Equations}}
\setcounter{aufgabenNummer}{1}
  \renewcommand{\kAufgabenBuchstabe}{G}


\subsection{\deu{Lineare Gleichungen}\eng{Linear Equations}}
\subsubsection{\eng{translate:} Erfüllbare und unerfüllbare lineare Gleichungen}

\kTrainingAufgabe{
\eng{translate:}Bestimmen Sie die Lösungsmenge bezüglich der Grundmenge
$G=\mathbb{R}$:
\\
\begin{enumerate}[label=\alph*)]
\item $4x=7+3x$
\item $3x=x$
\item $(x+2)^2 + (x+3)^2 - (x+5)^2 = (x-2)^2 - 12$
\item $6x + 5 = \frac{1}{3} (18x + 15)$
\item $8x -2 = 4x + 1 - (-4x - 3.5)$
\end{enumerate}
}{ %% Lösungsteil
\begin{enumerate}[label=\alph*)]
\item $\Lx = \{7\}$
\item $\Lx = \{0\}$
\item $\Lx = \{1\}$
\item $\Lx = \mathbb{R}$
\item $\Lx = \{\}$
\end{enumerate}
}{8}%%


\kNiveauAufgabe{
\eng{translate:}Bestimmen Sie die Lösungsmenge bezüglich der Grundmenge
$G=\mathbb{R}$:
\\
$\frac{3x-1}{4} + \frac{2x+1}{5} - 1 = \frac{3x-1}{5} +6 - \frac{x+2}{3}$
}{ %% Lösungsteil
$\Lx = \{7\}$
}{8}%%




\subsubsection{Lineare Gleichungen mit Parametern}


Bestimmen Sie die Lösungen. Schreiben Sie die Ergebnisse so einfach
wie möglich. Die Lösungsvariable ist jeweils $x$:

\kNiveauAufgabe{
\begin{enumerate}[label=\alph*)]
\item $3x+a = x+9a$
\item $(t+x-s)(t-x-s) = (t-x)(s+x) - st$
\item $\frac{k-bx}{v} + x = \frac{vx-b}{v}$
\item $(k-4)x = 4k - k^2$
\item $\frac{1}{x} + \frac{1}{a} = \frac{1}{b}$
%%\item $\frac{x- \frac{n^2}{4}}{\frac{1}{2} - \frac{n}{4}} = x+n$
\end{enumerate}
}{%% Lösungsteil
\begin{enumerate}[label=\alph*)]
\item $\Lx = \left\{ 4a  \right\}$
\item $\Lx = \left\{ t-s \right\}$ für $t\ne s$. Ist $t=s$, so ist $\Lx  = \mathbb{R}$.
\item $\Lx = \left\{ \frac{k+b}b \right\}$ für $b\ne 0$ und $v\ne
0$. Ansonsten gibt es Spezialfälle.
\item $\Lx = \left\{ -k  \right\}$ für $k\ne 4$.
\item $\Lx = \left\{ \frac{ab}{a-b} \right\}$ für $a\ne b$.
%%\item $\frac{x- \frac{n^2}{4}}{\frac{1}{2} - \frac{n}{4}} = x+n$
\end{enumerate}
%%
}{8}


\subsubsection{\eng{translate:}Textaufgaben, die auf lineare Gleichungen führen}

\kNiveauAufgabe{%%
  \eng{translate:}Zerlegen Sie die Zahl 188 so in zwei Summanden, dass der 3. Teil des
  ersten Summanden um 12 kleiner ist als der 5. Teil des zweiten
  Summanden. Wie heissen die beiden Summanden?
}{%% Lösung
Die Summanden sind 48 und 140.
}{12}


\kNiveauAufgabe{%%
\eng{translate:}Urs addiert zu einer gedachte Zahl $x$ die Zahl 5, setzt recht
  an die Summe die Ziffer 7, dividiert die entstandene Zahl durch 11,
  multipliziert den Quotienten mit 4 und erhält das 14-fache der
  ursprünglichen Zahl. Um welche Zahl $x$ handelt es sich?
}{%% Lösung
Die gedachte Zahl ist 2.
}{12}


\kNiveauAufgabe{%%
\eng{translate:}Jemand muss zwei Kredite von zusammen CHF 180\,000.- zu 4.5\%
  und 5.5\% verzinsen. Wären die Zinssätze vertauscht, so ergäbe sich
  ein um CHF 225.- niedriger Jahreszins. Wie hoch sind die Kredite?
}{%% Lösung
Die beiden Kredite betragen CHF 78’750.-
und CHF 101’250.-.
}{12}


\kNiveauAufgabe{%%
\eng{translate:}Eine Buchhandlung verkaufte von den vorhandenen Exemplaren eines
  neu erschienen Krimis am ersten Tag ein Achtel und 10 Stück, am
  zweiten Tag vom Restbestand die Hälfte und 15 Stück. Danach blieben
  noch 50 Krimis übrig. Wie viele Exemplare waren ursprünglich
  vorhanden?
}{%% Lösung
Ursprünglich waren 160 Krimis vorhanden.}{12}

\kNiveauAufgabe{%%
 \eng{translate:}Ein Kaufmann mischt 6\,kg einer Warensorte mit 14\,kg einer
  zweiten Sorte. Der Preis je kg der ersten Sorte ist um CHF 2.50
  höher als derjenige der zweiten Sorte. Er erhält eine Mischung, die
  CHF 8.- pro kg kostet. Wie teuer ist jede der beiden Sorten pro kg?
}{%% Lösung
Die erste Sorte kostet CHF 9.75 und die
zweite Sorte CHF 7.25.}{12}


\kNiveauAufgabe{%%
\eng{translate:}Aus 60\%-igem und 65\%-igem Alkohol sollen 150 Liter 63\%-ige
  Mischung hergestellt werden. Wie viele Liter von jeder Sorte muss
  man verwenden?
}{%% Lösung
Von der ersten Sorte nimmt man 60 Liter,
von der zweiten 90 Liter.
}{12}



\kNiveauAufgabe{%%
\eng{translate:}Auf dem Flohmarkt wird um ein Ölbild gefeilscht. Der Händler
  verlangt CHF 590.-, während der Käufer nur CHF 410.- bezahlen
  will. Die beiden einigen sich so, dass der Händler den Preis um
  gleich viele Prozente senkt, wie der Käufer sein Angebot
  erhöht. Welches ist der Verkaufspreis und um wie viele Prozente sind
  beide von ihren Forderungen abgewichen?
}{%% Lösung
Sie wichen je 18% von ihrer Forderung ab.
Der Verkaufspreis betrug CHF 483.80.
}{12}


\kNiveauAufgabe{%%
\eng{translate:}Wenn zwei Zuleitungen gleichzeitig geöffnet sind, füllen sie ein
  Bassin in 9 Minuten und 36 Sekunden. Um das Bassin alleine zu
  füllen, braucht die erste Zuleitung 12 Minuten. Wie lange benötigt
  die zweite Zuleitung, wenn sie das Bassin alleine füllen soll?
}{%% Lösung
Die zweite Leitung hätte alleine 48 min.
}{12}




\kNiveauAufgabe{%%
\eng{translate:}Eine Weinhändlerin kauft 300 Liter Rotwein und 600 Liter
  Weisswein. Der Preis pro Liter Weisswein ist um CHF 1.50 höher als der
  Literpreis Rotwein. Weil der Weinbauer und die Händlerin alte
  Freunde aus der BM-Schulzeit sind, erhält die Weinhändlerin pro
  Liter Rotwein 10\% und pro Liter Weisswein 12\% Rabatt.
  Der Weinbauer vertauscht jedoch die beiden Rabatte und schickt der
  Händlerin somit eine Rechnung, die um CHF 60.- zu hoch ist.
  Berechnen Sie die Literpreise für Rot- bzw. Weisswein vor Abzug der
  Rabattes.
}{%% Lösung
Der Liter Rotwein kostet CHF 7.-, der Liter
Weisswein CHF 8.50.}{12}




\subsection{\eng{translate:}Gleichungssysteme mit zwei Unbekannten}

\eng{translate:}Lösen Sie die linearen Gleichungssysteme mithilfe eines geeigneten Lösungsverfahrens
(Additions-, Einsetzungs-, Gleichsetzungsverfahren). Bestimmen Sie die Lösungsmenge be-
züglich der Grundmenge $\mathbb{G} = \mathbb{R}\times\mathbb{R}\}$;


\kNiveauAufgabe{%%
\begin{multicols}{2}
  \begin{enumerate}[label=\alph*)]
  \item
    \gleichungZZ{-\frac{x}{2}}{2y-1+x}{\frac{x}{2}+3}{1-x}
  \item
    \gleichungZZ{\frac{a+b}{6}}{\frac{a}{2}-5}{a+8}{\frac{b+a}{6}}
  \item
  \gleichungZZ{5x-8y}{1}{20x}{7+32y}
\item
  $$2x+7y = -5+8y = 24x +7$$
\item
  \gleichungZZ{3x}{2+5y}{6}{\frac{24}{3x} + \frac{10y}{x}}
\item
  \gleichungZZ{\frac{4x-1}{5}-x}{y-\frac{3y-4}{2}}{\frac{x+y-3}{2x-y+7}}{\frac{3}{4}}
  \end{enumerate}
\end{multicols}
}{% Lösung
\begin{multicols}{2}
  \begin{enumerate}[label=\alph*)]
  \item
    $\LoesungsMenge=\left\{(10; 7)\right\}$
  \item
    $\LoesungsMenge=\left\{(-26; -82)\right\}$
  \item
    $\LoesungsMenge=\left\{ \right\}$
\item
    $\LoesungsMenge=\left\{\left(\frac72; 12\right)\right\}$
\item
    $\LoesungsMenge=\left\{ \right\}$
\item
    $\LoesungsMenge=\left\{\left(\frac{11}4; \frac{11}2\right)\right\}$
  \end{enumerate}
\end{multicols}
}{8}

%% Substitution nicht im Rahmenlehrplan:
\entfernteAufgabe{
\begin{multicols}{2}
  \begin{enumerate}[label=\alph*)]
  \item
    \gleichungZZ{\frac{b}{a+b}}{1-\frac{3}{a-b}}{\frac{3b}{a+b}}{1-\frac{2}{a-b}}

  \item
    \gleichungZZ{\frac{2y}{4x-1}+9}{\frac{5}{x+2y}}{\frac{10}{x+2y}-12}{\frac{y}{4x-1}}
  \end{enumerate}
\end{multicols}
}{}{}%%%



\subsection{\eng{translate: }Textaufgaben, die auf Gleichungssysteme mit zwei Unbekannten führen}

\kNiveauAufgabe{%%
    Farben der Sorte A und der Sorte B mit Kilopreisen von CHF 12.-
    bzw. CHF 10.- werden gemischt. Ein Kilogramm Mischung kommt nun
    auf CHF 10.90 zu stehen. Nähme man bei gleicher Gesamtmenge der
    Mischen von der Sorte A 3 kg weniger, so würde das Kilogramm nur
    noch CHF 10.40 kosten. Wie viele kg nahm man anfänglich von jeder
    Sorte?
}{% Lösung
Anfänglich nahm man 5.4 kg von Sorte A und
6.6 kg von Sorte B.
}{8}


\kNiveauAufgabe{%%
     Frau Gross hat ein Kapital in zwei Posten angelegt, einen zu
      4\% und einen zu 5\%. Nach ihrer Rechnung beträgt die Summe der
      Jahreszinsen CHF 2\,560. Das sind aber CHF 80.- zu viel; sie hat
      nämlich die Zinssätze verwechselt. Welche Posten hat sie zu
      welchem Zinssatz angelegt?
}{% Lösung
Sie hat richtigerweise CHF 32’000 zu 4\% und
CHF 24’000 zu 5\% angelegt.
}{8}


\kNiveauAufgabe{%%
     Dividiert man eine zweiziffrige Zahl durch ihre Quersumme,
        so erhält man 4 Rest 9. Vertauscht man die Ziffern der
        ursprünglichen Zahl und dividiert diese durch die um 13
        vermehrte Quersumme, so erhält man 3.
}{% Lösung
Die gesuchte Zahl heisst 57.
}{8}


\kNiveauAufgabe{%%
     Eine Händlerin kaufte 800 Stück von Sorte I und 400
          Stück von Sorte II für insgesamt CHF 4\,200. Sorte I
          verkaufte sie mit einem Zuschlag von 15\% und Sorte II mit
          einem Zuschlag von 50\%. Der Verkaufspreis betrug insgesamt
          CHF 5\,180. Für wie viel Geld hatte sie ein Stück jeder
          Sorte eingekauft?
}{% Lösung
Ein Stück der Sorte I kostet CHF 4.- und ein
Stück der Sorte II CHF 2.50.
}{8}






\subsection{\deu{Quadratische Gleichungen}\eng{Quadratic
          Equations}}\index{\deu{Quadratische
          Gleichungen}\eng{quadratic Equations}}

%  Lösen Sie die Gleichungen ohne Solve-Funktion des Taschenrechners,
%  indem Sie sie auf die Grundform bringe. Die Grundform kann
%  anschliessend mit dem entsprechenden Taschenrechnermodus gelöst
%  werden.

  \subsubsection{Reinquadratische Gleichungen}

Bestimmen Sie jeweils $x$:

\kTrainingAufgabe{
\begin{multicols}{2}
  \begin{enumerate}[label=\alph*)]
  \item $x^2 = 0.04$
  \item $x^2+4=0$
  \end{enumerate}
  \end{multicols}
}{%% Lösung
\begin{multicols}{2}
  \begin{enumerate}[label=\alph*)]
  \item $\Lx = \{-0.2; +0.2\}$
  \item $\Lx = \{\}$
  \end{enumerate}
  \end{multicols}
}{8}

\kNiveauAufgabe{
$$(7+x)(7-x)=(3x+2)^2-(2x+3)^2$$
}{%% Lösung
$\Lx = \{-3; +3\}$
}{8}


\subsubsection{\eng{translate: }Gemischtquadratische Gleichungen: Lösungsformel}
  Grundform:\\
  \begin{tabular}{|c|c|}%%
    \hline%%
Grundform: $ax^2 + bx +c = 0$ & Lösungsformel: $x_{1,2}
  = \frac{-b \pm \sqrt{b^2 - 4ac}}{2a}$\\%%
    \hline%%
    \end{tabular}%%

\kNiveauAufgabe{
\eng{translate: }Bestimmen Sie jeweils $x$ von Hand und geben Sie die
  Resultate exakt an:
\begin{multicols}{2}
 \begin{enumerate}[label=\alph*)]
  \item $x^2-3x-10 = 0$
  \item $x^2-11x + 30 = 0$
  \item $0=12x^2-8x+\frac43$
  \item $3x^2+5x=-1$
  \item $-5x^2=20+7.5x$
  \item $\frac18 x^2 + 2.5x + 8 = 0$
 \end{enumerate}
\end{multicols}
}{%% Lösungen
\begin{multicols}{2}
 \begin{enumerate}[label=\alph*)]
  \item $\Lx = \{-2; 5\}$
  \item $\Lx = \{5; 6\}$
  \item $\Lx = \left\{\frac13\right\}$
  \item $\Lx = \left\{ \frac{-5-\sqrt{13}}6;  \frac{\sqrt{13}-5}6 \right\}$
  \item $\Lx = \{\}$
  \item $\Lx = \{-4; 16\}$
 \end{enumerate}
\end{multicols}
}{8}%% end Aufgabe



%%  \subsubsection{Quadratische Gleichungen: Substitution (Nicht Rahmenlehrplan)}
%%  Gleichungen höheren Grades, die durch Substitution auf eine
%%  quadratische Gleichung führen:

\entfernteAufgabe{
  \begin{enumerate}
  \item Lösen Sie die Gleichung nach $x$ auf. Resultate exakt:
    $$\left(x-\frac{3}{x}\right)^2 - 20 = x - \frac{3}{x}$$

  \item Lösen Sie die Gleichung nach $x$ auf und runden Sie das
    Resultat auf drei Dezimalstellen:
    $$(x+7)^6 = 8\cdot (x+7)^3 - 15$$

  \item $$(x+7)^8 = 14(x+7)^4 + 32$$
  \end{enumerate}
}{}{8}%% Entfertn, da nicht Rahmenlehrplan

\subsection{Textaufgaben}

Die Gleichungen sind auf Grundform zu bringen und können anschliessend
mit dem entsprechenden Taschenrechnermodus gelöst werden.

\kNiveauAufgabe{
 Ein Speicher wird durch eine Zuleitung in einer bestimmten
    Zeit gefüllt. Das Entleeren durch eine andere Leitung dauert eine
    Minute länger. Aus Unachtsamkeit bleibt während des Füllens die
    andere Leitung geöffnet, deshalb ist der Speicher erst nach 756
    Minuten voll. Wie lange dauert die Leerung des Speichers bei
    geschlossener Zuleitung?
}{%% Lösung
Die Leerung dauert 28 Minuten.
}{8}%%


\kNiveauAufgabe{
Das Produkt der beiden kleinsten von sechs aufeinander
      folgenden natürlichen Zahlen ist dreimal so gross wie die Summe
      der vier übrigen Zahlen. Wie heisst die kleinste Zahl?
}{%% Lösung
Die kleinste Zahl heisst 14.
}{8}%%

\kNiveauAufgabe{
Eine Praxiseinrichtung mit einem Anschaffungswert von CHF
        24\,000;.- wurde zweimal mit dem gleichen Prozentsatz
        abgeschrieben und hat zur Zeit einen Wert von CHF 16\,335.-.
        Wie viele Prozente beträgt der Abschreibungssatz?
}{%% Lösung
Der Abschreibungssatz beträgt 17.5\%.
}{8}%%

\kNiveauAufgabe{
Dividiert man die Summe zweier positiver Zahlen durch ihre
        Differenz, so erhält man 12.5\% der grösseren Zahl.
        Berechnen Sie die grössere Zahl, wenn die kleinere 42 ist.
}{%% Lösung
Die gesuchte (größere) Zahl heisst 56.
}{8}%%

\kNiveauAufgabe{
Die Zehnerziffer einer zweiziffrigen Zahl ist um 6
         kleiner als die Einerziffer. Die vierfache Summe der
          Quadrate der Ziffern ist gleich der fünfunzwanzigfachen
          Quersumme. Wie heisst die Zahl?
}{%% Lösung
Die gesuchte Zahl heisst 17.
}{8}%%





\subsection{\eng{translate:} Bruchgleichungen (Definitions- und Lösungsmenge)}


\kNiveauAufgabe{
\eng{translate:}Nennen Sie die Definitionsmenge, lösen Sie anschliessend die Gleichung
und bestimmen Sie die Lösungsmenge; $G=\mathbb{R}$. Exakte Werte angeben:
\\
\begin{multicols}{2}
\begin{enumerate}[label=\alph*)]
%%\item $\frac{4a+6}{2a-10} + \frac{6a-43}{5-a} = -10$
\item $\frac{4a+6}{2a-10} + \frac{6a-43}{5-a} = -10$
\item $\frac{22.5}{a-3} + \frac{9}{-a^2 + 6a - 9} = 0$
\item $\frac{1}{x(x-4)} + \frac{1}{x(x+4)} = \frac{2}{x^2-16}$
\item $\frac{3.6x+8.5}{x^2-x-12} + \frac{5.2}{x-4} = \frac{1.9}{-x-3}$
%%\item $\frac{21}{-x+7} = 3 + \frac{1.5x}{\left(\frac{7-x}{2}\right)}$    
\end{enumerate}
\end{multicols}
}{%% Lösungsteil
\begin{multicols}{2}
\begin{enumerate}[label=\alph*)]
%%\item $\frac{4a+6}{2a-10} + \frac{6a-43}{5-a} = -10$
\item $\mathbb{D} = \mathbb{R}\backslash \{5\} , \Lx  = \left\{\frac23\right\}$
\item $\mathbb{D} = \mathbb{R}\backslash \{3\} , \Lx  = \left\{3.4\right\}$
\item $\mathbb{D} = \mathbb{R}\backslash \{-4, 0, 4\} = \Lx$
\item $\mathbb{D} = \mathbb{R}\backslash \{-3, 4\} , \Lx  = \left\{\frac{-165}{107}\right\}$
%%\item $\frac{21}{-x+7} = 3 + \frac{1.5x}{\left(\frac{7-x}{2}\right)}$    
\end{enumerate}
\end{multicols}
}{10}%% end NiveauAufgabe



\kNiveauAufgabe{
  $\frac{x-3}{x+3} + \frac{x+3}{x-3} = \frac{26}{x^2-9}$
}{% Lösung
$\Lx = \{-2; +2\}$ und $\mathbb{D} = \mathbb{R} \backslash \{-3; +3\}$
}{8}


\kNiveauAufgabe{
\eng{translate:}Nennen Sie die Definitionsmenge, lösen Sie anschliessend die Gleichung
und bestimmen Sie die Lösungsmenge mit dem Taschenrechner. $G=\mathbb{R}$.
\\
\begin{multicols}{2}
\begin{enumerate}[label=\alph*)]
\item $\frac{x-4}{x-5x} = \frac{30-x^2}{x^2-5}$
\item $\frac{x}3 - \frac{x^2 -19}{3x+6} = 3 + \frac{4x-7}{6-3x}$
\end{enumerate}
\end{multicols}
}{%% Lösungsteil
\begin{multicols}{2}
\begin{enumerate}[label=\alph*)]
\item $\mathbb{D} = \mathbb{R}\backslash \{0, 5\},  \Lx = \{-3\}$
\item $\mathbb{D} = \mathbb{R}\backslash \{-2; +2 \} , \Lx
= \left\{\frac43; 4 \right\}$
\end{enumerate}
\end{multicols}
}{10}%% end NiveauAufgabe










\subsection{Elementare Potenzgleichungen}
Lösen Sie die Gleichung ohne Solve-Funktion des Taschenrechners nach
$x$ auf und schreiben Sie die Resultate als exakte Werte ($G =
\mathbb{R}$):

\kNiveauAufgabe{
\begin{multicols}{3}
\begin{enumerate}[label=\alph*)]
\item $\frac{2}{3}x^3 = 12$
\item $x^5 = \frac{1}{32}$
\item $\left(\frac{1}{x}\right)^6 = 13$
\item $\left(\frac{x-5}{3}\right)^3=64$
\item $(x+5)^{\frac{1}{2} = 4}$
\item $\left(x-6\right)^{\frac{1}{3}} = 2$
\end{enumerate}
\end{multicols}
}{%% Lösungen
\begin{multicols}{3}
\begin{enumerate}[label=\alph*)]
\item $x = \sqrt[3]{18}$
\item $x = \frac12$
\item $\Lx = \left\{- \frac1{\sqrt[6]{13}} ; \frac1{\sqrt[6]{13}}  \right\}$
\item $x = 17$
\item $x = 11$
\item $x = 14$
\end{enumerate}
\end{multicols}
}{12}%% end Aufgabe



\subsection{Exponentialgleichungen}

\textbf{Logarithmen}

\kTrainingAufgabe{%%
\eng{translate: }Berechnen Sie die Exponenten. Resultate exakt angeben:
\begin{multicols}{2}
\begin{enumerate}[label=\alph*)]
\item $a^x=b$
\item $4^x=12$
\end{enumerate}
\end{multicols}
}{%%
\begin{multicols}{2}
\begin{enumerate}[label=\alph*)]
\item $a^x= b  \Longrightarrow    x = \log_{a}(b)$
\item $4^x=12  \Longrightarrow    x = \log_{4}(12)$
\end{enumerate}
\end{multicols}
}{4}%%

\kNiveauAufgabe{%%
\eng{translate: }Berechnen Sie die Exponenten. Resultate exakt angeben:
$$4\cdot 3^{x+1}=6$$
}{%%
$$4\cdot 3^{x+1}=6  \Longrightarrow    x = \log_{3}(2)$$
}{4}%%


\kNiveauAufgabe{%%
\eng{translate: }Lösen Sie die Exponentialgleichungen durch Erzeugen gleicher Basis:
\begin{multicols}{3}
\begin{enumerate}[label=\alph*)]
\item $2^x=\frac{1}{8}$
\item $2^{2x}=\frac{1}{8}$
\item $4^x=\frac{1}{32}$
\item $3^x=\frac{1}{81}$
  \item $9^x=\frac{1}{3}$
  \item $27^{2x}=\frac{1}{9}$
  \item $\left(3^x\right)^6 = \frac{1}{81}$
    \item $50\cdot 5^{n-1} + 3\cdot 5^{n+1} = 5^x$
    \item $\frac{25^k}{125^3} = 5^x$
      \item $2^x=\frac{8^4}{4^8}$
\end{enumerate}
\end{multicols}
}{%%
\begin{multicols}{3}
\begin{enumerate}[label=\alph*)]
\item $\Lx = \{-3\}$
\item $\Lx = \left\{-\frac32\right\}$
\item $\Lx = \left\{-\frac52\right\}$
\item $\Lx = \{-4\}$
\item $\Lx = \left\{-\frac12\right\}$
\item $\Lx = \left\{-\frac23\right\}$
\item $\Lx = \{n+2\}$
\item $\Lx = \{2k-9\}$
\item $\Lx = \{-4\}$
\end{enumerate}
\end{multicols}
}{4}%%




\eng{translate:} Lösen Sie die Gleichung ohne Solve-Funktion des Taschenrechners nach
$x$ auf und schreiben Sie die Resultate als exakte Werte ($G =
\mathbb{R}$).
Tipp: Erzeugen gleicher Basis

\kNiveauAufgabe{%%
\begin{multicols}{3}
\begin{enumerate}[label=\alph*)]
\item $2^x=16$
\item $\left(\frac{1}{4}\right)^x = 64$
\item $2^x\cdot 8^{x-1} = 32$
\item $\left(\frac{1}{3}\right)^x\cdot \left(\frac{3}{27}\right)^x = 3$
\item $e^{2x}=1$
\item $\frac{6^{x-2}}{36^{x+1}} = 216$
\end{enumerate}
\end{multicols}
}{%% Lösung
\begin{multicols}{3}
\begin{enumerate}[label=\alph*)]
\item $x = 4$
\item $x = -3$ 
\item $x  = 2$
\item $x = - \frac13$
\item $x = 0$
\item $x = -7$
\end{enumerate}
\end{multicols}
}{12}%% end Aufgabe





\eng{translate:} 
Berechnen Sie $x$. Schreiben Sie das Ergebnis möglichst einfach.

\kNiveauAufgabe{%%
\begin{multicols}{2}
\begin{enumerate}[label=\alph*)]
\item $\frac14 \cdot 7^x = 25$
\item $4\cdot{} 3^{x+1} = 6$
\item $2^x + 2^{x+3} = \frac94$
\item $e^x = 3$
\item $\frac{e^{-x}}2  -3 =0$
\end{enumerate}
\end{multicols}
}{%% Lösung
\begin{multicols}{2}
\begin{enumerate}[label=\alph*)]
\item $x  = \frac{\log(100)}{\log(7)} = \frac{\lg(100)}{\lg(7)} = \frac2{\lg(7)}$ 
\item $x  = \log_3(0.5) = \frac{\log(0.5)}{\log(3)}$
\item $x  = -2$
\item $x  = \ln(3)$
\item $x  = \ln(6)$
\end{enumerate}
\end{multicols}
}{12}%% end Aufgabe


\newpage
