%% Makros for Kompendium


%% Text nur bei deutscher Sprache ausgeben:
\NewDocumentCommand{\deu}{m}{%
\ifthenelse{\equal{\kLanguage}{deu}}{%%
{#1}}{%%
}%% end ifthenelse
}% end deu

%% Text nur bei englischer Sprache ausgeben:
\NewDocumentCommand{\eng}{m}{%
\ifthenelse{\equal{\kLanguage}{eng}}{%%
{#1}}{%%
}%% end ifthenelse
}% end eng


%%%%%%%%%%%%%%%%%%%%%%%%%%%%%%%%%%%%%%%%%%%%%%%%%%%%%%%%%%%5
\newcounter{aufgabenNummer}
\setcounter{aufgabenNummer}{1}


\NewDocumentCommand{\isLoesungen}{m}{%%
\ifthenelse{\equal{\kLoesungen}{true}}{%%
#1}{%%
}%% end ifthenelse
}%%

\NewDocumentCommand{\isAufgaben}{m}{%%
\ifthenelse{\equal{\kLoesungen}{false}}{%%
#1}{%%
}%% end ifthenelse
}%% end IS Aufgaben


\newcommand{\mmPapierZwei}[2]{\begin{tikzpicture}
%%  \draw[step=4mm,bbwMMFarbe,ultra thin]
%%  \draw[step=4mm,bbwMMFarbe,thick]
  \draw[step=5mm,gray,line width=0.02mm]
  (0, 0) grid ({#2}, {#1});
\end{tikzpicture}}%%



\NewDocumentCommand{\isMMPapier}{m}{%%
  \ifthenelse{\equal{\kMMPapier}{true}}{%%
    
    \mmPapierZwei{#1}{17.0}

}{%%
}%% end ifthenelse
}%%

\newcommand{\nurNiveauBMP}[1]{%%
\ifthenelse{\equal{\kAufgabenNiveau}{bmp}}{%%
#1
}%% end ifthenelse
}%%

\newcommand{\auchTrainingsAufgabe}[1]{%%
\ifthenelse{\equal{\kAufgabenNiveau}{bmp}}{%%
%% Aufgabennummern durchnummerieren, auch wenn keine Trainingsaufgaben
%% ausdedruckt werden
\stepcounter{aufgabenNummer}%%
}{%%
#1
}%% end ifthenelse
}%%

\newcommand{\fragenMarkierung}{}

%% Kompendium Aufgaben
%% #1 : Aufgabentext
%% #2 : Lösungstext
%% #3 : Optional, 

%% Jedes Kapitel beginnt mit neuen Buchstaben.
%% Algebra hat Aufgabne A1, A2. ,,, Funktionen F1, F2, ...
%%
\newcommand{\kAufgabenBuchstabe}{0}
\newcommand{\kMetaAufgabe}[3]{%%
%  \kAufgabenBuchstabe{}\,\arabic{aufgabenNummer}. \fragenMarkierung{}:
  \fragenMarkierung{}:
  \isLoesungen{{\color{blue}#2}}\isAufgaben{#1}\stepcounter{aufgabenNummer}
  \isMMPapier{#3}
  \vspace{2mm}
}%% end kaufgabe

\newcommand{\kNiveauAufgabe}[3]{%%
%%\nurNiveauBMP{\kMetaAufgabe{#1}{#2}{#3}}%%
\renewcommand{\fragenMarkierung}{\colorbox{red!30}{\kAufgabenBuchstabe{}\,\arabic{aufgabenNummer}.(B)}}
\kMetaAufgabe{#1}{#2}{#3}%%
}%% end kaufgabe

\newcommand{\kTrainingAufgabe}[3]{%%
\renewcommand{\fragenMarkierung}{\colorbox{blue!30}{\kAufgabenBuchstabe{}\,\arabic{aufgabenNummer}.(E)}}
\auchTrainingsAufgabe{\kMetaAufgabe{#1}{#2}{#3}}%%
}%% end kaufgabe


\renewcommand{\indexname}{\deu{Stichwortverzeichnis}\eng{Index}}%%



\newcommand{\LoesungsMenge}{L}

\def\gleichungZZ#1#2#3#4{%%
  $$\left|
  \begin{array}{rcl}
    {#1} &=& {#2}\\
    {#3} &=& {#4}
    \end{array}\right|$$}%%

\def\gleichungDD#1#2#3#4#5#6{%%
  $$\left|
  \begin{array}{rcl}
    {#1} &=& {#2}\\
    {#3} &=& {#4}\\
    {#5} &=& {#6}\\
    \end{array}\right|$$}%%


%% Aufgaben die entfernt werden, können durchaus im Kompendium
%% Qulltext bleiben.
\newcommand{\entfernteAufgabe}[3]{}
