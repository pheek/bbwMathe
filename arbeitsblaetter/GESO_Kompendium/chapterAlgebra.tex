{\section{Algebra}\index{Algebra}
  \setcounter{aufgabenNummer}{1}
  \renewcommand{\kAufgabenBuchstabe}{A}

\deu{\subsection{Zahlen}\index{Zahlmengen}}
\deu{\subsection{Numbers}\index{Numbers}}

%%%%%%%%%%%%%%%%%%%%%%%%%%%%%%%%%%%%%%%%%%%%%%%%%%%%%%%%%%%%%%%%%%%%%%
%% Aufgabe Zahlmengen
\kTrainingAufgabe{
\deu{Ordnen Sie die folgenden Zahlen der am weitesten links stehenden
Zahlmenge zu}\eng{Find the least powefull set for the given number
(the rightmost ist the most powerfull set of numbers)}:

$$\mathbb{N} \subset \mathbb{Z} \subset \mathbb{Q} \subset \mathbb{R}$$

\begin{multicols}{2}
\begin{enumerate}[label=\alph*)]
 \item$-\sqrt{\frac{16}{2}}$
 \item$|-\pi|$
 \item$1.\overline{25}$
 \item$\frac{\sqrt{8}}{\sqrt{2}}$
\end{enumerate}
\end{multicols}

}{%% Lösungsteil
\begin{multicols}{2}
\begin{enumerate}[label=\alph*)]
 \item$-\sqrt{\frac{16}{2}}$ : $\mathbb{R}$
 \item$|-\pi|$  : $\mathbb{R}$
 \item$1.\overline{25}$ : $\mathbb{Q}$
 \item$\frac{\sqrt{8}}{\sqrt{2}}$ : $\mathbb{N}$
\end{enumerate}
\end{multicols}
}{4}%% end kTrainingAufgabe

%%%%%%%%%%%%%%%%%%%%%%%%%%%%%%%%%%%%%%%%%%%%%%%%%%%%%%%%%%%%%%%%%%%%%%
%% Aufgabe Relationszeichken
\kTrainingAufgabe{%%
Setzen Sie zwischen die Terme (an die Stelle der drei Punkte) die
richtigen Relationszeichen ($<$, $=$, $>$):

\begin{multicols}{2}
\begin{enumerate}[label=\alph*)]
 \item$|5-2| ... |2-5|$
 \item$-\sqrt{3} ... -\sqrt{2}$ 
 \item$-3 ... |-3|$ 
 \item$\frac{1}{3} ... \frac{1}{4}$ 
 \item$\frac{2}{3} ... \frac{3}{4}$
 \item$2.\overline{9}$ ... 3
 \item$-\frac{1}{3} ... -\frac{1}{4}$ 
\end{enumerate}
\end{multicols}
}{%% Lösungsteil
\begin{multicols}{2}
\begin{enumerate}[label=\alph*)]
 \item$|5-2| =  |2-5|$
 \item$-\sqrt{3} < -\sqrt{2}$ 
 \item$-3 < |-3|$ 
 \item$\frac{1}{3} > \frac{1}{4}$
 \item$\frac{2}{3} < \frac{3}{4}$
 \item$2.\overline{9} = 3$
 \item$-\frac{1}{3} <-\frac{1}{4}$
\end{enumerate}
\end{multicols}
}{8}%% end Kaufgabe Training


\subsection{Terme}

\kTrainingAufgabe{
Benennen Sie die folgenden Terme mit dem richtigen Begriff
(Jeweils einer aus: \textit{Summe}, \textit{Differenz}, \textit{Produkt},
\textit{Quotient}, \textit{Potenz}):

\begin{multicols}{3}
\begin{enumerate}[label=\alph*)]
 \item $(3+x)\cdot{}2 + y$ 
 \item $(4a)^x$
 \item$4a^x$
 \item$(x-2y^4z)^2$
 \item$2a^7 - 4bc(z-1)^3$
 \item$(2a -b) : x + z : 4$
 \item$(a-4)(z+3)$ 
 \item$(2a-b):(x+z:4)$ 
\end{enumerate}
\end{multicols}
}{
\begin{multicols}{3}
\begin{enumerate}[label=\alph*)]
 \item $(3+x)\cdot{}2 + y$ :Summe
 \item $(4a)^x$ : Potenz
 \item$4a^x$ : Produkt
 \item$(x-2y^4z)^2$ : Potenz
 \item$2a^7 - 4bc(z-1)^3$ : Differenz
 \item$(2a -b) : x + z : 4$  : Summe
 \item$(a-4)(z+3)$   :Produkt
 \item$(2a-b):(x+z:4)$ :Quotient
\end{enumerate}
\end{multicols}
}{8}%% end kaufgabe

%%\subsection{\deu{Betrag}\eng{absolute value}}\deu{\index{Betrag}]  \eng{\index{absolute value}}
%%\subsection{\deu{Betrag}\eng{absolute value}}\index{\deu{Betrag}\eng{absolute value}}


Lösen Sie in der Grundmenge $\mathbb{Z}$:

\kTrainingAufgabe{
  
\begin{multicols}{2}
\begin{enumerate}[label=\alph*)]
 \item$|x+4| = 5$x
 \item$|2x-1| = 0$
 \item$|-x + 7| = 10$
 \item$|-x|=-1$ 
\end{enumerate}
\end{multicols}
}{ Lösungsteil
\begin{multicols}{2}
\begin{enumerate}[label=\alph*)]
 \item$|x+4| = 5$  : $\LoesungsMenge{}=\{1,-9\}$
 \item$|2x-1| = 0$ : $\LoesungsMenge{}=\{0.5\}$
 \item$|-x + 7| = 10$ :  $\LoesungsMenge{}=\{-3, 17\}$
 \item$|-x|=-1$ : $\LoesungsMenge{}=\{\}$
\end{enumerate}
\end{multicols}
}{4}

\deu{Vereinfachen Sie}\eng{simplify}:

\kTrainingAufgabe{$|-4|$}{$4$}{4}
\kTrainingAufgabe{$-\big| -5-|-8|  \big|$}{$-13$}{4}

\deu{Terme}\eng{terms}:

\kNiveauAufgabe{$T(x) = (-4)\cdot{}|x-8|\cdot{}(x^2)$\\
\deu{Berechnen Sie}\eng{calculate}: $$T(-2)$$ }{$80$}{4}%% end kaufgabe



\deu{\subsection{Runden}\index{runden}}
\eng{\subsection{round}\index{round}}

\deu{Runden Sie je auf zwei Dezimalen}
\eng{Round to two decimal places}:

\kTrainingAufgabe{4.555}{4.56}{4}

\kNiveauAufgabe{4.8956}{4.90}{4}
\kNiveauAufgabe{0.00499}{0.00}{4}




\deu{Runden Sie auf 2 Dezimalstellen in der wissenschaftlichen Notation:}
\eng{Translate: Runden Sie auf 2 Dezimalstellen in der
  wissenschaftlichen Notation:}
\kTrainingAufgabe{
\begin{multicols}{2}
\begin{enumerate}[label=\alph*)]
 \item 12.05649
 \item 0.028249
 \item 2\,498\,900
\end{enumerate}
\end{multicols}
}{%%
\begin{multicols}{2}
\begin{enumerate}[label=\alph*)]
 \item 12.05649  =1.21 $\cdot 10^1$
 \item 0.028249 =2.82 $\cdot 10^{-2}$
 \item 2\,498\,900 = 2.50 $\cdot 10^{6}$
\end{enumerate}
\end{multicols}
}%%



\subsection{Addition / Subtraktion}
%% Summenzeichen
\subsubsection{Summenzeichen}

\kTrainingAufgabe{%%
Schreiben Sie die Summanden hin und berechnen Sie den Wert des Ausdrucks:

\begin{enumerate}[label=\alph*)]
 \item $$\sum_{k=2}^{k=4}{(3k+2^k)}$$
 \item $$\sum_{i=4}^{i=7}(3i) + 6$$ 
\end{enumerate}
}{%%
\begin{enumerate}[label=\alph*)]
 \item $$\sum_{k=2}^{k=4}{(3k+2^k)}$$ : $(6+4) + (9+8) + (12+16) = 10+17+28=55$
 \item $$\sum_{i=4}^{i=7}(3i) + 6$$ : $((12) + (15) + (18) +(21) )+6=72$
\end{enumerate}
}{4}%%

\subsection{Multiplikation / Division}

Schreiben Sie als Summen:

\kTrainingAufgabe{%%
  \begin{multicols}{2}
\begin{enumerate}[label=\alph*)]
\item $\left(5m-\frac{1}{2}n\right)^2$
\item $(t-9s^2)\cdot(t+9s^2)$
  \item $(3a + b^4)^2$
  \end{enumerate}
\end{multicols}
}{%%
  ... noch keine Lösung ...
}{4}%%

  
\subsubsection{Faktorisieren}

Faktorzerlegung mit Hilfe der Binomischen Formeln\index{Binomische
  Formeln!Aufgaben}

\kTrainingAufgabe{%%
\begin{multicols}{2}
\begin{enumerate}[label=\alph*)]
\item $4x^2-y^2$
\item $36a^2 + 36ab + 9b^2$
\item $4x^4-28x^2y+49y^2$
\item $a^3-a$
\item $x^3-6x^2y+9xy^2$
  \item $z^4 -3^2$
\end{enumerate}
\end{multicols}
}{%%
  ... noch keine Lösung ...
}{4}%%


Faktorzerlegung durch planmäßiges Probieren

\kTrainingAufgabe{%%
\begin{multicols}{2}
\begin{enumerate}[label=\alph*)]
\item $a^2 + 2a -15$
\item $a^3-a^2-6a$
\item $a^2 - 20a + 75$
  \item $a^2 -19a + 48$
\end{enumerate}
\end{multicols}
}{%%
  ... noch keine Lösung ...
}{4}%%


Faktorzerlegung durch teilweises Ausklammern und Ausklammern von
Klammern (Zerlegen Sie in möglichst viele Faktoren):

\kTrainingAufgabe{%%
\begin{multicols}{2}
\begin{enumerate}[label=\alph*)]
\item $ab(x-2y)-b(x-2y)$
\item $x(y-2z)-(y-2z)$
\item $(a-2b)(m+n)+(a-2b)(3m+n)$
\item $3a-6ab+c-2bc$
  \item $ac-ad+bc-bd$
\end{enumerate}
\end{multicols}
}{%%
  ... noch keine Lösung ...
}{4}%%


\subsubsection{Bruchterme}


Bruchterme umformen

\kNiveauAufgabe{%%
\begin{multicols}{2}
\begin{enumerate}[label=\alph*)]
\item $\frac{st-3t^2}{s^2-st} + \frac{3s^2-3st-st+t^2}{s^2-2st+t^2}$
\item $\frac{\frac{a^2-16c^2}{8a^2}}{\frac{a-4c}{4a}}$
\item $\left(\frac{1}{a^2} - \frac{1}{b^2}\right) \cdot
    \left(\frac{a}{a+b} + \frac{b}{a-b}\right)$
\item $\left( \frac{a+4}{a} - \frac{b+4}{b}\right) : \frac{a-b}{a}$
    
\end{enumerate}
\end{multicols}
}{%%
  ... noch keine Lösung ...
}{4}%%


\subsection{Potenzen}

Schreiben Sie als Bruch ohne negative Exponenten:

\kTrainingAufgabe{%%
\begin{multicols}{2}
\begin{enumerate}[label=\alph*)]
\item $2x^{-3}$
  \item $ab^{-5}$
\end{enumerate}
\end{multicols}
}{%%
  ... noch keine Lösung ...
}{4}%%


Schreiben Sie folgende Bruchzahlen als Dreierpotenzen:

\kTrainingAufgabe{%%
\begin{multicols}{2}
\begin{enumerate}[label=\alph*)]
  \item $\frac{1}{9}$
  \item $\frac{1}{27}$
    \item $\frac{1}{81}$
\end{enumerate}
\end{multicols}
}{%%
  ... noch keine Lösung ...
}{4}%%


Lösen Sie die Exponentialgleichungen durch Erzeugen gleicher Basis:

\kTrainingAufgabe{%%
\begin{multicols}{3}
\begin{enumerate}[label=\alph*)]
\item $2^x=\frac{1}{8}$
\item $2^{2x}=\frac{1}{8}$
\item $4^x=\frac{1}{32}$
\item $3^x=\frac{1}{81}$
  \item $9^x=\frac{1}{3}$
  \item $27^{2x}=\frac{1}{9}$
  \item $\left(3^x\right)^6 = \frac{1}{81}$
    \item $50\cdot 5^{n-1} + 3\cdot 5^{n+1} = 5^x$
    \item $\frac{25^k}{125^3} = 5^x$
      \item $2^x=\frac{8^4}{4^8}$
\end{enumerate}
\end{multicols}
}{%%
  ... noch keine Lösung ...
}{4}%%



Wenden Sie die Potenzgesetze an und schreiben Sie so einfach wie
möglich:


\kNiveauAufgabe{%%
\begin{multicols}{3}
\begin{enumerate}[label=\alph*)]
\item $a\cdot a^{2x+3} : a^{1-x}$
\item $\left(\frac{x}{y}\right)^7 : \left(\frac{-x}{y}\right)^3$
\item $(-b)^6 \cdot (-b^8)$
\item $(-x^3)^4$
\item $\left( -(a^{-1})^{-2} \right)^6$
\item $(-a)^4 : (-a^{10})$
  \item $4^k \cdot \left( \frac{1}{2} \right)^k \cdot \left(
    \frac{1}{3} \right)^{-k}$
\end{enumerate}
\end{multicols}
}{%%
  ... noch keine Lösung ...
}{4}%%



\kTrainingAufgabe{%%
Vereinfachen Sie:
$$\left(\frac{3x^{-2}y^2}{4x^{-4}y^3}\right)^{-2} : \left(\frac{2x^{-1}}{3xy^{-2}}\right)^3$$
}{%%
  ... noch keine Lösung ...
}{4}%%


\subsubsection{Wurzeln}


Vereinfachen Sie folgende Terme. Resultat in Wurzelschreibweise:

\kTrainingAufgabe{%%
\begin{multicols}{2}
\begin{enumerate}[label=\alph*)]
\item $\left(\frac{1}{a}\right)^{-\frac{1}{4}}$
\item $\sqrt{x}\cdot \sqrt[3]{x^4} \cdot \sqrt[6]{x^3}$
  \item $3a\cdot \sqrt[3]{9a^2}$
\item $\sqrt[4]{b\cdot \sqrt[3]{b^2\cdot \sqrt{b}}}$
\end{enumerate}
\end{multicols}
}{%%
  ... noch keine Lösung ...
}{4}%%


\subsubsection{Logarithmen}

\textbf{Zehnerlogarithmen}


Berechnen Sie die Exponenten. Resultate exakt, wenn nötig mit Hilfe von
Zehnerlogarithmen:

\kTrainingAufgabe{%%
\begin{multicols}{2}
\begin{enumerate}[label=\alph*)]
\item $a^x=b$
\item $4^x=12$
  \item $4\cdot 3^{x+1}=6$
\end{enumerate}
\end{multicols}
}{%%
  ... noch keine Lösung ...
}{4}%%

\newpage
