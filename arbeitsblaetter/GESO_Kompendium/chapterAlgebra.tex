{\section{Algebra}\index{Algebra}
  \setcounter{aufgabenNummer}{1}
  \renewcommand{\kAufgabenBuchstabe}{A}

\kMatheNinjaLink{Grundlagen}{https://matheninja.ch/terme/}

\deu{\subsection{Zahlen}\index{Zahlmengen}}
\eng{\subsection{Numbers}\index{Numbers}}

%%%%%%%%%%%%%%%%%%%%%%%%%%%%%%%%%%%%%%%%%%%%%%%%%%%%%%%%%%%%%%%%%%%%%%
%% Aufgabe Zahlmengen

\kTrainingAufgabe{
\deu{Ordnen Sie die folgenden Zahlen allen möglichen Zahlmengen
  $\mathbb{N}$, $\mathbb{Z}$, $\mathbb{Q}$ oder $\mathbb{R}$
  zu}\eng{Find all suitable sets for the given number
(the rightmost ist the most powerfull set of numbers)}:
\\
\eng{$$\mathbb{N} \subset \mathbb{Z} \subset \mathbb{Q} \subset \mathbb{R}$$}
\begin{multicols}{2}
\begin{enumerate}[label=\alph*)]
 \item$-\sqrt{\frac{16}{2}}$
 \item$|-\pi|$
 \item$1.\overline{25}$
 \item$\frac{\sqrt{8}}{\sqrt{2}}$
\end{enumerate}
\end{multicols}
\kVerweiseAltesKompendium{6}{2}
\kPlatzFuerBerechnungen{8}
}{%% Lösungsteil
\begin{multicols}{2}
\begin{enumerate}[label=\alph*)]
 \item$-\sqrt{\frac{16}{2}}$ : $\mathbb{R}$
 \item$|-\pi|$  : $\mathbb{R}$
 \item$1.\overline{25}$ : $\mathbb{Q}, \mathbb{R}$
 \item$\frac{\sqrt{8}}{\sqrt{2}}$ : $\mathbb{N}, \mathbb{Z},
   \mathbb{Q}, \mathbb{R}$
\end{enumerate}
\end{multicols}
}%% end kTrainingAufgabe

%%%%%%%%%%%%%%%%%%%%%%%%%%%%%%%%%%%%%%%%%%%%%%%%%%%%%%%%%%%%%%%%%%%%%%
%% Aufgabe Relationszeichken
\kTrainingAufgabe{%%
\deu{Setzen Sie zwischen die Terme (an die Stelle der drei Punkte) die
  richtigen Relationszeichen ($<$, $=$, $>$)}
\eng{Set the correct symbol ($<$, $=$, $>$) between the numbers}:
\\
\begin{multicols}{2}
\begin{enumerate}[label=\alph*)]
 \item$|5-2| ... |2-5|$
 \item$-\sqrt{3} ... -\sqrt{2}$ 
 \item$-3 ... |-3|$ 
 \item$\frac{1}{3} ... \frac{1}{4}$ 
 \item$\frac{2}{3} ... \frac{3}{4}$
 \item$-\frac{1}{3} ... -\frac{1}{4}$
 \item$-\frac{8}{\sqrt{4}} ... (-2)^2$
\end{enumerate}
\end{multicols}
\kVerweiseAltesKompendium{6}{3}
\kPlatzFuerBerechnungen{8}
}{%% Lösungsteil
\begin{multicols}{2}
\begin{enumerate}[label=\alph*)]
 \item$|5-2| =  |2-5|$
 \item$-\sqrt{3} < -\sqrt{2}$ 
 \item$-3 < |-3|$ 
 \item$\frac{1}{3} > \frac{1}{4}$
 \item$\frac{2}{3} < \frac{3}{4}$
 \item$-\frac{1}{3} <-\frac{1}{4}$
 \item$-\frac{8}{\sqrt{4}} = (-2)^2$
\end{enumerate}
\end{multicols}
}%% end Kaufgabe Training


%%%%%%%%%%%%%%%%%%%%%%%%%%%%%%%%%%%%%%%%%%%%%%%%%%%%%%%%%%%%%%%%%%%%%%%%%%
%%%%%%%%%%%%%%%%%%%%%%% SECTION %%%%%%%%%%%%%%%%%%%%%%%%%%%%%%%%%%%%%%%%%%
%%%%%%%%%%%%%%%%%%%%%%%%%%%%%%%%%%%%%%%%%%%%%%%%%%%%%%%%%%%%%%%%%%%%%%%%%%


\subsection{\deu{Terme}\eng{Expressions}}\deu{\index{Terme}}\eng{\index{expressions}}
\kMatheNinjaLink{Termumformungen}{https://matheninja.ch/termumformungen/}

%% %%%%%%%%%%%%%%%%%%%%%%% Aufgabe %%%%%%%%%%%%%%%%%%%%%%%%%%%%%%%%%%%%%%%%%%%%%%%%

\kTrainingAufgabe{
\deu{Benennen Sie die folgenden Terme mit dem richtigen Begriff
(Jeweils einer aus: \textit{Summe}, \textit{Differenz}, \textit{Produkt},
  \textit{Quotient}, \textit{Potenz})}
\eng{Name the following expressions using the correct concept (Sum, Difference, Product, Quotient, Potency)}:
\\
\begin{multicols}{3}
\begin{enumerate}[label=\alph*)]
 \item $(3+x)\cdot{}2 + y$ 
 \item $(4a)^x$
 \item$4a^x$
 \item$(x-2y^4z)^2$
 \item$2a^7 - 4bc(z-1)^3$
 \item$(2a -b) : x + z : 4$
 \item$(a-4)(z+3)$ 
 \item$(2a-b):(x+z:4)$ 
\end{enumerate}
\end{multicols}
\kVerweiseAltesKompendium{6}{1}
\kPlatzFuerBerechnungen{8}
}{
\begin{multicols}{3}
\begin{enumerate}[label=\alph*)]
 \item $(3+x)\cdot{}2 + y$ :\deu{Summe}\eng{Sum}
 \item $(4a)^x$ : \deu{Potenz}\eng{Potency}
 \item$4a^x$ : \deu{Produkt}\eng{Product}
 \item$(x-2y^4z)^2$ : \deu{Potenz}\eng{Potency}
 \item$2a^7 - 4bc(z-1)^3$ : \deu{Differenz}\eng{Difference}
 \item$(2a -b) : x + z : 4$  : \deu{Summe}\eng{Sum}
 \item$(a-4)(z+3)$   : \deu{Produkt}\eng{Product}
 \item$(2a-b):(x+z:4)$ : \deu{Quotient}\eng{Quotient}
\end{enumerate}
\end{multicols}
}%% end kaufgabe


%% %%%%%%%%%%%%%%%%%%%%%%% Aufgabe %%%%%%%%%%%%%%%%%%%%%%%%%%%%%%%%%%%%%%%%%%%%%%%%

\kTrainingAufgabe{
  \kKommentar{neue Aufgabe}
  \eng{Given is the following term:}
  \deu{Gegegben ist der folgende Term:}
  
  $$T\left(x\right) = 2x^2+7x-1$$

\eng{Evaluate:}
\deu{Werten Sie aus:}

  a) $T(3)$

  b) $T(-3)$
}{ %%

  a) $38$

  b) $-4$
}

%%%%%%%%%%%%%%%%%%%%%%%%%%%%%%%%%%%%%%%%%%%%%%%%%%%%%%%%%%%%%%%%%%%%%%%%%%
%%%%%%%%%%%%%%%%%%%%%%% SECTION %%%%%%%%%%%%%%%%%%%%%%%%%%%%%%%%%%%%%%%%%%
%%%%%%%%%%%%%%%%%%%%%%%%%%%%%%%%%%%%%%%%%%%%%%%%%%%%%%%%%%%%%%%%%%%%%%%%%%


\subsubsection{\deu{Betrag}\eng{absolute value}}
\deu{\index{Betrag}}  \eng{\index{absolute value}}
%%\subsection{\deu{Betrag}\eng{absolute value}}\index{\deu{Betrag}\eng{absolute value}}



%% %%%%%%%%%%%%%%%%%%%%%%% Aufgabe %%%%%%%%%%%%%%%%%%%%%%%%%%%%%%%%%%%%%%%%%%%%%%%%

\entfernteAufgabe{
Lösen Sie in der Grundmenge $\mathbb{Z}$:
 \\
\begin{multicols}{2}
\begin{enumerate}[label=\alph*)]
 \item$|x+4| = 5$x
 \item$|2x-1| = 0$
 \item$|-x + 7| = 10$
 \item$|-x|=-1$ 
\end{enumerate}
\end{multicols}
\kVerweiseAltesKompendium{6}{4}
\kPlatzFuerBerechnungen{8}
}{ %%Lösungsteil
\begin{multicols}{2}
\begin{enumerate}[label=\alph*)]
 \item$|x+4| = 5$  : $\LoesungsMenge{}=\{1,-9\}$
 \item$|2x-1| = 0$ : $\LoesungsMenge{}=\{0.5\}$
 \item$|-x + 7| = 10$ :  $\LoesungsMenge{}=\{-3, 17\}$
 \item$|-x|=-1$ : $\LoesungsMenge{}=\{\}$
\end{enumerate}
\end{multicols}
}
\\
\kKommentar{Betragsglchg. nicht RLP}\\
\kKommentar{Ehem: Aufg. 4. Seite 6}\\

\kKommentar{Wenn schon Betragsgleichungen, dann bei Gleichungen!}
%% %%%%%%%%%%%%%%%%%%%%%%% Aufgabe %%%%%%%%%%%%%%%%%%%%%%%%%%%%%%%%%%%%%%%%%%%%%%%%

\kTrainingAufgabe{\deu{Vereinfachen Sie}\eng{simplify}:


  $|-4|$}{$4$}{4}


%% %%%%%%%%%%%%%%%%%%%%%%% Aufgabe %%%%%%%%%%%%%%%%%%%%%%%%%%%%%%%%%%%%%%%%%%%%%%%%

\kTrainingAufgabe{\deu{Vereinfachen Sie}\eng{simplify}:
\kKommentar{neue Aufgabe:}
  $\left| 2\cdot5 - 11 \right|$}{+1}


%% %%%%%%%%%%%%%%%%%%%%%%% Aufgabe %%%%%%%%%%%%%%%%%%%%%%%%%%%%%%%%%%%%%%%%%%%%%%%%

\kTrainingAufgabe{\deu{Vereinfachen Sie}\eng{simplify}:
\kKommentar{neue Aufgabe:}
  $1 - \big| 7- \left| 3 - 8 \right| \big|$}{...}


%% %%%%%%%%%%%%%%%%%%%%%%% Aufgabe %%%%%%%%%%%%%%%%%%%%%%%%%%%%%%%%%%%%%%%%%%%%%%%%

\kTrainingAufgabe{\deu{Vereinfachen Sie}\eng{simplify}:

  $-\big| -5-|-8|  \big|$}{$-13$}{4}

\deu{Terme}\eng{terms}:

%% %%%%%%%%%%%%%%%%%%%%%%% Aufgabe %%%%%%%%%%%%%%%%%%%%%%%%%%%%%%%%%%%%%%%%%%%%%%%%

\kNiveauAufgabe{\deu{Gegeben ist der folgende Term}\eng{The following
    expression is given:}
$$T(x) = (-4)\cdot{}|x-8|\cdot{}(x^2)$$
  \deu{Berechnen Sie}\eng{calculate}: $$T(-2)$$
  \kKommentar{Neue Aufgabe: Dafür Aufg. 4. S. 6 entfernt}\\
  \kKommentar{, da Betragsgleichungen nicht RLP.}
\kPlatzFuerBerechnungen{8}
}{$T(-2) = 80$}%% end kaufgabe

%% %%%%%%%%%%%%%%%%%%%%%%% Aufgabe %%%%%%%%%%%%%%%%%%%%%%%%%%%%%%%%%%%%%%%%%%%%%%%%

\kNiveauAufgabe{\deu{Gegeben ist der folgende Term}\eng{The following
    expression is given:}
$$T(x) = -x^2|4-x|\cdot{}|-x|$$
  \deu{Berechnen Sie}\eng{calculate}: $$T(3)$$
  \deu{Berechnen Sie}\eng{calculate}: $$T(-3)$$
  \kKommentar{Neue Aufgabe.}
\kPlatzFuerBerechnungen{8}
}{$T(3) = -27$ und $T(-3) = -189$}%% end kaufgabe

%%%%%%%%%%%%%%%%%%%%%%%%%%%%%%%%%%%%%%%%%%%%%%%%%%%%%%%%%%%%%%%%%%%%%%%%%%
%%%%%%%%%%%%%%%%%%%%%%% SECTION %%%%%%%%%%%%%%%%%%%%%%%%%%%%%%%%%%%%%%%%%%
%%%%%%%%%%%%%%%%%%%%%%%%%%%%%%%%%%%%%%%%%%%%%%%%%%%%%%%%%%%%%%%%%%%%%%%%%%

\deu{\subsection{Runden}\index{runden}}
\eng{\subsection{Round}\index{round}}
 \kVerweiseAltesKompendium{6./7.}{5-7}
\deu{Runden Sie je auf zwei Dezimalen}
\eng{Round to two decimal places}:

%% %%%%%%%%%%%%%%%%%%%%%%% Aufgabe %%%%%%%%%%%%%%%%%%%%%%%%%%%%%%%%%%%%%%%%%%%%%%%%

\kTrainingAufgabe{$2.245$
\kPlatzFuerBerechnungen{2}
}{$4.56$}
%% %%%%%%%%%%%%%%%%%%%%%%% Aufgabe %%%%%%%%%%%%%%%%%%%%%%%%%%%%%%%%%%%%%%%%%%%%%%%%

\kNiveauAufgabe{$2.995$
\kPlatzFuerBerechnungen{2}
}{$3.00$}
%% %%%%%%%%%%%%%%%%%%%%%%% Aufgabe %%%%%%%%%%%%%%%%%%%%%%%%%%%%%%%%%%%%%%%%%%%%%%%%

\kNiveauAufgabe{$2.9949$
\kPlatzFuerBerechnungen{2.5}
}{$2.99$}
%% %%%%%%%%%%%%%%%%%%%%%%% Aufgabe %%%%%%%%%%%%%%%%%%%%%%%%%%%%%%%%%%%%%%%%%%%%%%%%

\kNiveauAufgabe{$-2.9949$
\kPlatzFuerBerechnungen{2}
}{$-2.99$}
%% %%%%%%%%%%%%%%%%%%%%%%% Aufgabe %%%%%%%%%%%%%%%%%%%%%%%%%%%%%%%%%%%%%%%%%%%%%%%%

\kTrainingAufgabe{$\sqrt{3}$}{1.73}


%% %%%%%%%%%%%%%%%%%%%%%%% Aufgabe %%%%%%%%%%%%%%%%%%%%%%%%%%%%%%%%%%%%%%%%%%%%%%%%


\entfernteAufgabe{%%
\deu{Runden Sie auf 2 Dezimalstellen in der wissenschaftlichen Notation:}
\eng{Round to 2 decimal places in scientific notation:}
\begin{multicols}{2}
\begin{enumerate}[label=\alph*)]
 \item 12.05649
 \item 0.028249
 \item 2\,498\,900
\end{enumerate}
\end{multicols}
\kVerweiseAltesKompendium{7}{6./7.}
\kPlatzFuerBerechnungen{8}
}{%%
\begin{multicols}{2}
\begin{enumerate}[label=\alph*)]
 \item 12.05649  =1.21 $\cdot 10^1$
 \item 0.028249 =2.82 $\cdot 10^{-2}$
 \item 2\,498\,900 = 2.50 $\cdot 10^{6}$
\end{enumerate}
\end{multicols}
}%%
\kKommentar{Begriff «wissenschaftliche Schreibweise» RLP???}\\
\kKommentar{Begriff «signifikante Stellen» ???}\\

Meine Meinung: Der Lehrplan schreibt hier «Resultate sinnvoll runden
und auf Plausibilität überprüfen». Unsere Aufgaben in Prüfungen lauten
oft: Runden Sie auf zwei Dezimalen, was i. d. R. ein sinnloses
unterfangen ist; es sei denn, wir geben die Maßeinheit vor; doch genau
das sollten die Lernenden ja selbst können. Unsere Prüfungsaufgaben
sollten besser lauten:

\kKommentar{RUNDEN SIE SINNVOLL, od. «Runden Sie auf drei signifikante Stellen.»}

%%%%%%%%%%%%%%%%%%%%%%%%%%%%%%%%%%%%%%%%%%%%%%%%%%%%%%%%%%%%%%%%%%%%%%%
%%%%%%%%%%%%%%%%%%%%%%%%% SECTION %%%%%%%%%%%%%%%%%%%%%%%%%%%%%%%%%%%%%%
%%%%%%%%%%%%%%%%%%%%%%%%%%%%%%%%%%%%%%%%%%%%%%%%%%%%%%%%%%%%%%%%%%%%%%%

\deu{\subsection{Addition / Subtraktion}}\eng{\subsection{Addition / Subtraction}}
%% Summenzeichen
\deu{\subsubsection{Summenzeichen}} \eng{\subsubsection{Sum sign}}

%% %%%%%%%%%%%%%%%%%%%%%%% Aufgabe %%%%%%%%%%%%%%%%%%%%%%%%%%%%%%%%%%%%%%%%%%%%%%%%

\kTrainingAufgabe{
\deu{Schreiben Sie die Summanden einzeln hin und berechnen Sie den Wert des
  Ausdrucks}
\eng{Write down each summand and calculate the value of the expression}:

  $$\sum_{i=1}^6 i$$
}{
  $$=1+2+3+4+5+6 = 21$$
}

%% %%%%%%%%%%%%%%%%%%%%%%% Aufgabe %%%%%%%%%%%%%%%%%%%%%%%%%%%%%%%%%%%%%%%%%%%%%%%%

\kNiveauAufgabe{%%
\deu{Schreiben Sie die Summanden einzeln hin und berechnen Sie den Wert des Ausdrucks}
\eng{Write down each summand and calculate the value of the expression}:
\\

\kKommentar{Hier sind zwei neue Aufgaben in anderem Stil.}
\begin{multicols}{2}
\begin{enumerate}[label=\alph*)]
 \item $$\sum_{k=1}^{k}{(2^k+k)}$$
 \item $$\sum_{k=2}^{k}{(2k+3^k)}$$
 \item $$\sum_{i=1}^{4}{(x_i-2)}$$
 \item $$\sum_{i=1}^{3}{(2a^i+4)}$$
 \item $$\sum_{i=4}^{7}3i + 6$$
 \item $$\sum_{s=-2}^{2}s\cdot{}2^s$$
\end{enumerate}
\end{multicols}{2}

\kVerweiseAltesKompendium{7}{8}
\kPlatzFuerBerechnungen{8}
\kKommentar{Aufgabe a) b) c) und d) werden ja besser mit vorab durch }\\
\kKommentar{Ausklammern gelöst, daher ist die Aufgabenstellung fraglich.}\\
\kKommentar{Aufgabe e) besser entfernen,}\\
\kKommentar{da zum Verständnis wohl auch das}\\
\kKommentar{Produktzeichen eingeführt werden sollte.}
}{%%

\begin{enumerate}[label=\alph*)]
 \item $\sum_{k=1}^{4}{(2^k+k)} = (2^1+1) + (2^2+2) + (2^3+3) +
   (2^4+4) = 3 + 6 + 11 + 20 = 40$\\
   Wird i.d.R. so gelöst:\\
   $\sum_{k=1}^{4}{(2^k+k)} =
   \sum_{k=1}^{4}{(2^k)}+\sum_{k=1}^{4}{(k)} = (2+2^2+2^3+2^4) +
   \frac{4\cdot{}5}{2} = 30+10 = 40$

 \item $\sum_{k=2}^{4}{(2k+3^k)} = 2\cdot{}\sum_{k=2}^4{}(k) +
   \sum_{k=2}^4(3^k) = \\
   2\cdot{}(2+3+4) + (3^2+3^3+3^4) = 18 + (9+27+81) = 135$

 \item $\sum_{i=1}^{4}{(x_i-2)} = x_1+x_2+x_3+x_4 -8$
 \item $\sum_{i=1}^{3}{(2a^i+4)}  = 2(a + a^2 + a^3) + 12$
 \item $\sum_{i=4}^{i=7}(3i) + 6 = ((12) + (15) + (18) +(21) )+6=72$

 \item $\sum_{s=-2}^{2}s\cdot{}2^s = (-2\cdot{}2^{-2}) + (-2^{-1}) + (0) + (2^1) + (2\cdot{}2^2)=9$

\end{enumerate}
}%% end Aufgbae


%%%%%%%%%%%%%%%%%%%%%%%%%%%%%%%%%%%%%%%%%%%%%%%%%%%%%%%%%%%%%%%%%%%%%%%%%%
%%%%%%%%%%%%%%%%%%%%%%% SECTION %%%%%%%%%%%%%%%%%%%%%%%%%%%%%%%%%%%%%%%%%%
%%%%%%%%%%%%%%%%%%%%%%%%%%%%%%%%%%%%%%%%%%%%%%%%%%%%%%%%%%%%%%%%%%%%%%%%%%



\subsection{Multiplikation / Division}

%% %%%%%%%%%%%%%%%%%%%%%%% Aufgabe %%%%%%%%%%%%%%%%%%%%%%%%%%%%%%%%%%%%%%%%%%%%%%%%






\kTrainingAufgabe{
  \deu{Multiplizieren Sie aus und vereinfachen Sie so weit wie möglich:}
  \eng{Multiply and simplify as much as possible:}
\begin{multicols}{2}
\begin{enumerate}[label=\alph*)]
\item $\left(5m-\frac{1}{2}n\right)^2$
\item $(t-9s^2)\cdot(t+9s^2)$
\item $(3a + b^4)^2$
\item $(2a - \sqrt{3})(2a+\sqrt{3})$
\end{enumerate}
\end{multicols}

\kKommentar{Braucht es noch 2-3 weitere Trainingsaufgaben?}
\kVerweiseAltesKompendium{7}{9}
\kPlatzFuerBerechnungen{8}
}{%%
\begin{multicols}{2}
\begin{enumerate}[label=\alph*)]
\item $\left(5m-\frac{1}{2}n\right)^2 = 25m^2 -5mn + \frac14n^2$
\item $(t-9s^2)\cdot(t+9s^2) = t^2 + 81s^4$
\item $(3a + b^4)^2 = 9a^2 + 6ab^4 + b^8$
\item $4a^2 - 3$
\end{enumerate}
\end{multicols}
}%%


\kNiveauAufgabe{\eng{Multiply out and simplify as far as possible:}
\deu{Multiplizieren Sie aus und vereinfachen Sie so weit wie möglich:}
\begin{enumerate}[label=\alph*]
 \item $\left(x-3\right)\left(x+1\right) - \left(x+2\right)^2$
 \item $\left(x^2-3xy^4\right)^2$
 \item $ \left(x-2\right) \cdot \left(x+1\right) \cdot \left(x-1\right)$
 \item $ 2x^2\cdot \left(2x+5\right)^2$
\end{enumerate}

\kKommentar{neue Aufgabe}
}{%% Lösungen
\begin{enumerate}[label=\alph*)]
\item $-6x-7$
\item $x^4-6x^3y^4+9x^2y^8$
\item $x^3-2x^2-x+2$
\item $8x^4+40x^3+50x^2$
\end{enumerate}
}

%%%%%%%%%%%%%%%%%%%%%%%%%% Aufgabe %%%%%%%%%%%%%%%%%%%%%%%%%%%%%%%%%%%%%%%%%%%%%

%%%%%%%%%%%%%%%%%%%%%%%%%%%%%%%%%%%%%%%%%%%%%%%%%%%%%%%%%%%%%%%%%%%%%%%%%%
%%%%%%%%%%%%%%%%%%%%%%% SECTION %%%%%%%%%%%%%%%%%%%%%%%%%%%%%%%%%%%%%%%%%%
%%%%%%%%%%%%%%%%%%%%%%%%%%%%%%%%%%%%%%%%%%%%%%%%%%%%%%%%%%%%%%%%%%%%%%%%%%



\subsubsection{\deu{Faktorisieren}\eng{Factorize}}\deu{\index{Faktorzerlegung}}\eng{\index{factorize}}

\kMatheNinjaLink{Faktorisieren}{https://matheninja.ch/faktorisieren/}
%% %%%%%%%%%%%%%%%%%%%%%%% Aufgabe %%%%%%%%%%%%%%%%%%%%%%%%%%%%%%%%%%%%%%%%%%%%%%%%



\kKommentar{Einfache Einstiegsaufgbaen fehlten:}

\kTrainingAufgabe{
  
  \deu{Faktorisieren Sie so weit wie möglich:}
  \eng{Factorize as much as possible:}

\begin{enumerate}[label=\alph*)]
  \item $3x^3+6x^2+9x$
  \item $6ab-8ac-2a$
\end{enumerate}

\kPlatzFuerBerechnungen{4}
\kKommentar{neueAufgabe}
}{%% Lösugen
  \begin{enumerate}[label=\alph*)]
  \item $3x(x^2+2x+3)$
  \item  $2a(3b-4c+1)$
    \end{enumerate}
}%% end aufgabe

%%%%%%%%%%%%%%%%%%%%%%%%%% Aufgabe %%%%%%%%%%%%%%%%%%%%%%%%%%%%%%%%%%%%%%%%%%%%%



\kTrainingAufgabe{%%
\deu{Zerlegen Sie in möglichst viele Faktoren}\eng{Factorize the expression}:
\begin{multicols}{2}
\begin{enumerate}[label=\alph*)]
\item $4x^2-y^2$
\item $36a^2 + 36ab + 9b^2$
\item $4x^4-28x^2y+49y^2$
\item $a^3-a$
\item $x^3-6x^2y+9xy^2$
\item $z^4 -3^2$
\end{enumerate}
\end{multicols}
\kVerweiseAltesKompendium{7}{10}
\kPlatzFuerBerechnungen{8}
}{%%
\begin{multicols}{2}
\begin{enumerate}[label=\alph*)]
\item $4x^2-y^2 = (4x+y)(4x-y)$
\item $36a^2 + 36ab + 9b^2 = (6a+3b)^2$
\item $4x^4-28x^2y+49y^2 = (2x-7y)^2$
\item $a^3-a = a\cdot{}(a+1)(a-1)$
\item $x^3-6x^2y+9xy^2 = x\cdot{}(x-3y)^2$
\item $z^4 -3^2 = (z^2+w)(z^2-w)$
\end{enumerate}
\end{multicols}
}%%


%% %%%%%%%%%%%%%%%%%%%%%%% Aufgabe %%%%%%%%%%%%%%%%%%%%%%%%%%%%%%%%%%%%%%%%%%%%%%%%

\kTrainingAufgabe{%%
  \deu{Zerlegen Sie in möglichst viele Faktoren}\eng{Break the expressions down into factors}:\\
(\deu{Alle Aufgaben zum Zweiklammeransatz sind auch mit dem Taschenrechner
    lösbar.}\eng{All tasks can also be solved with the calculator.})
\begin{multicols}{2}
\begin{enumerate}[label=\alph*)]
\item $a^2 + 2a -15$
\item $a^3-a^2-6a$
\item $a^2 - 20a + 75$
  \item $a^2 -19a + 48$
\end{enumerate}
\end{multicols}
\kVerweiseAltesKompendium{7}{11}
\kPlatzFuerBerechnungen{8}
}{%%
\begin{multicols}{2}
\begin{enumerate}[label=\alph*)]
\item $a^2 + 2a -15 = (a+5)(a-3)$
\item $a^3-a^2-6a = a(a-3)(a+2)$
\item $a^2 - 20a + 75 = (a+15)(a+5)$
  \item $a^2 -19a + 48 = (a-16)(a-3)$
\end{enumerate}
\end{multicols}
}%%


%% %%%%%%%%%%%%%%%%%%%%%%% Aufgabe %%%%%%%%%%%%%%%%%%%%%%%%%%%%%%%%%%%%%%%%%%%%%%%%

\kTrainingAufgabe{%%
  \deu{Zerlegen Sie in Faktoren}\eng{Devide the terms into factors}:\\
\begin{multicols}{2}  
\begin{enumerate}[label=\alph*)]
\item $ab(x-2y)-b(x-2y)$
\item $(a-2b)(m+n)+(a-2b)(3m+n)$
\item $x(y-2z)-(y-2z)$
\item $3a-6ab+c-2bc$
%%  \item $ac-ad+bc-bd$
\end{enumerate}
\end{multicols}
\kVerweiseAltesKompendium{8}{12}
\kPlatzFuerBerechnungen{8}
}{%%
  \begin{multicols}{2}
\begin{enumerate}[label=\alph*)]
\item $ab(x-2y)-b(x-2y) = b(x-2y)(a-1)$
\item $(a-2b)(m+n)+(a-2b)(3m+n) = 2(a-2b)(2m+n)$
\item $x(y-2z)-(y-2z) = (y-2z)(x-1)$
\item $3a-6ab+c-2bc = (1-2b)(3a+c)$
%%  \item $ac-ad+bc-bd$
\end{enumerate}
\end{multicols}
}%%

%%%%%%%%%%%%%%%%%%%%%%%%%%%%%%%%%%%%%%%%%%%%%%%%%%%%%%%%%%%%%%%%%%%%%%%%%%
%%%%%%%%%%%%%%%%%%%%%%% SECTION %%%%%%%%%%%%%%%%%%%%%%%%%%%%%%%%%%%%%%%%%%
%%%%%%%%%%%%%%%%%%%%%%%%%%%%%%%%%%%%%%%%%%%%%%%%%%%%%%%%%%%%%%%%%%%%%%%%%%


\subsubsection{\deu{Bruchterme}\eng{Fractional terms}}
\kMatheNinjaLink{Brüche}{https://matheninja.ch/brueche/}

%% %%%%%%%%%%%%%%%%%%%%%%% Aufgabe %%%%%%%%%%%%%%%%%%%%%%%%%%%%%%%%%%%%%%%%%%%%%%%%



\kTrainingAufgabe{

\deu{Schreiben Sie das folgende Produkt bzw. die folgende Summe als vereinfachten Bruchterm.}
\eng{Write the following product or sum as a simplified rational expression.}

\deu{Was ist jeweils die Definitionsmenge $\definitionsmenge$?}
\eng{What is the domain $\definitionsmenge$ in each case?}

\begin{multicols}{2}
\begin{enumerate}[label=\alph*)]
 \item  $ \frac{2x + 3}{x + 2} \cdot \frac{x - 4}{x - 1}$
 \item $ \frac{2x + 3}{x + 2} + \frac{x - 4}{x - 1}$
\end{enumerate}
\end{multicols}
  
  \kKommentar{Neue Aufgabe: Schwierigkeitsgrad}
}{% Lösungen
\begin{multicols}{2}
\begin{enumerate}[label=\alph*)]  
  \item $\frac{2x^2-5x-12}{x^2+x-2}$  \deu{mit}\eng{width} $\definitionsmenge = \mathbb{R} \backslash \{-2; 1\}$
  \item $\frac{3x^2 -x -11}{x^2+X-2}$ \deu{mit}\eng{width} $\definitionsmenge = \mathbb{R} \backslash \{-2; 1\}$
  \end{enumerate}
\end{multicols}
}%% end aufgabe

%% %%%%%%%%%%%%%%%%%%%%%%% Aufgabe %%%%%%%%%%%%%%%%%%%%%%%%%%%%%%%%%%%%%%%%%%%%%%%%


\kTrainingAufgabe{
  \deu{Vereinfachen Sie den folgendnen Bruchterm.}
  \eng{Simplify the following fractional expression.}
  
\begin{multicols}{2}
 \begin{enumerate}[label=\alph*)]
  \item $\frac{\frac{28a^2}{x}}{\frac{14a}{2x^2}}$
  \item $\frac{\frac{ax^2}{y+2}}{\frac{2a}{y^2-4}}$
 \end{enumerate}
\end{multicols}

\kKommentar{neue Aufgabe zum Gesetz a:(x/y) = a(y/x).}
}{%% Lösung(en)
\begin{enumerate}[label=\alph*)]
 \item $\frac{28a^2}{x} :  \frac{14a}{2x^2}=\frac{28a^2}{x} \cdot{}  \frac{2x^2}{14a}= 4ax$
 \item $\frac{x}{y-2}$
\end{enumerate}
 
}%% end Aufgabe


%% %%%%%%%%%%%%%%%%%%%%%%% Aufgabe %%%%%%%%%%%%%%%%%%%%%%%%%%%%%%%%%%%%%%%%%%%%%%%%

\kNiveauAufgabe{%%
  \eng{Determine the domain and simplify as much as
    possible. Basic set:}  \deu{Bestimmen Sie die Definitionsmenge und kürzen Sie
    soweit wie möglich. Grundmenge:}
  $\grundmenge = \mathbb{R}$.
\begin{multicols}{2}
\begin{enumerate}[label=\alph*)]
\item $\frac{x^2-16}{x^2+x-20}$
\item $\frac{x^2+x-42}{x^2-4x-12}$
\end{enumerate}
\end{multicols}
\kKommentar{neue Aufgabe?}
\kPlatzFuerBerechnungen{8}
}{%%
\begin{multicols}{2}
\begin{enumerate}[label=\alph*)]
\item $\frac{x^2-16}{x^2+x-20}$ = $\frac{x+4}{x+5}$ :   $\definitionsmenge=\mathbb{R} \backslash \{-5; 4\}$
\item $\frac{x^2+x-42}{x^2-4x-12}$ = $\frac{x+7}{x+2}$ : $\definitionsmenge=\mathbb{R} \backslash \{-2; 6\}$ 
\end{enumerate}
\end{multicols}
}%%

  

%% %%%%%%%%%%%%%%%%%%%%%%% Aufgabe %%%%%%%%%%%%%%%%%%%%%%%%%%%%%%%%%%%%%%%%%%%%%%%%

\kNiveauAufgabe{%%
  \eng{Determine the domain and simplify as much as
    possible. Basic set:}
  \deu{Bestimmen Sie die Definitionsmenge und kürzen Sie
    soweit wie möglich. Grundmenge:}
  $\grundmenge = \mathbb{R}$.
\begin{multicols}{2}
\begin{enumerate}[label=\alph*)]
\item $\frac{st-3t^2}{s^2-st} + \frac{3s^2-3st-st+t^2}{s^2-2st+t^2}$
%%\item $\frac{\frac{a^2-16c^2}{8a^2}}{\frac{a-4c}{4a}}$
\item $\left(\frac{1}{a^2} - \frac{1}{b^2}\right) \cdot
    \left(\frac{a}{a+b} + \frac{b}{a-b}\right)$
\item $\left( \frac{a+4}{a} - \frac{b+4}{b}\right) : \frac{a-b}{a}$    
\end{enumerate}
\end{multicols}
\kVerweiseAltesKompendium{8}{13}
\kKommentar{Doppelbruch entfernt, da nicht RLP}\\
\kKommentar{Doppelburch gut zum Training, aber nicht RLP}
\kPlatzFuerBerechnungen{8}
}{%%
\begin{multicols}{2}
\begin{enumerate}[label=\alph*)]
\item $\frac{st-3t^2}{s^2-st} + \frac{3s^2-3st-st+t^2}{s^2-2st+t^2} = \frac{3s+t}s$
%%\item $\frac{\frac{a^2-16c^2}{8a^2}}{\frac{a-4c}{4a}}$
\item $\left(\frac{1}{a^2} - \frac{1}{b^2}\right) \cdot
    \left(\frac{a}{a+b} + \frac{b}{a-b}\right) = \frac{a+4c}{2a}$
\item $\left( \frac{a+4}{a} - \frac{b+4}{b}\right) : \frac{a-b}{a} = \frac{-4}b$    
\end{enumerate}
\end{multicols}
}%%


%%%%%%%%%%%%%%%%%%%%%%%%%%%%%%%%%%%%%%%%%%%%%%%%%%%%%%%%%%%%%%%%%%%%%%%%%%
%%%%%%%%%%%%%%%%%%%%%%% SECTION %%%%%%%%%%%%%%%%%%%%%%%%%%%%%%%%%%%%%%%%%%
%%%%%%%%%%%%%%%%%%%%%%%%%%%%%%%%%%%%%%%%%%%%%%%%%%%%%%%%%%%%%%%%%%%%%%%%%%



\subsection{\eng{Powers, Roots, Logarithms}\deu{Potenzen, Wurzeln und Logarithmen}}
\kMatheNinjaLink{\eng{Powers}\deu{Potenzen}}{https://matheninja.ch/potenzen/}

\subsubsection{\eng{Powers}\deu{Potenzen}}
%% %%%%%%%%%%%%%%%%%%%%%%% Aufgabe %%%%%%%%%%%%%%%%%%%%%%%%%%%%%%%%%%%%%%%%%%%%%%%%





%% %%%%%%%%%%%%%%%%%%%%%%% Aufgabe %%%%%%%%%%%%%%%%%%%%%%%%%%%%%%%%%%%%%%%%%%%%%%%%

\kTrainingAufgabe{%%
\eng{Write as a single fraction avoiding negative exponents}%%
\deu{Schreiben Sie als Bruch ohne negative Exponenten}:
\begin{multicols}{2}
\begin{enumerate}[label=\alph*)]
\item $2x^{-3}$
  \item $ab^{-5}$
\end{enumerate}
\end{multicols}
\kVerweiseAltesKompendium{8}{14}
\kPlatzFuerBerechnungen{2}
}{%%
\begin{multicols}{2}
\begin{enumerate}[label=\alph*)]
\item $2x^{-3} = \frac2{x^3}$
\item $ab^{-5} = \frac{a}{b^5}$
\end{enumerate}
\end{multicols}
}%%

%% %%%%%%%%%%%%%%%%%%%%%%% Aufgabe %%%%%%%%%%%%%%%%%%%%%%%%%%%%%%%%%%%%%%%%%%%%%%%%

\kTrainingAufgabe{
\kKommentar{Potenzen: Folgender Stil fehlt}

\deu{Vereinfachen Sie so weit wie möglich}
\eng{Simplify as much as possible}

a)
$$x^{\frac{1}{2}} \cdot x^{\frac{3}{4}}$$

b)
$$\frac{x^{\frac{2}{3} } }{x^{\frac{1}{4} } }$$

\kKommentar{neue Aufgabe}

}{%% Lösungen
  a)$$x^{\frac54} = \sqrt[4]{x^5}$$

  b) $$x^\frac{5}{12} = \sqrt[12]{x^5}$$
}

%% %%%%%%%%%%%%%%%%%%%%%%% Aufgabe %%%%%%%%%%%%%%%%%%%%%%%%%%%%%%%%%%%%%%%%%%%%%%%%


\kTrainingAufgabe{%%
  \eng{Simplify avoiding fractional terms}%
  \deu{Schreiben Sie die Terme ohne Quotienten}:
\begin{multicols}{3}
\begin{enumerate}[label=\alph*)]
\item $\frac{a}{b^2}$
  \item $\frac1{ab^2}$
  \item $\frac{x}{y^{-4}}$
\end{enumerate}
\end{multicols}
\kKommentar{neue Aufgabe?}
\kPlatzFuerBerechnungen{8}
}{%%
\begin{multicols}{3}
\begin{enumerate}[label=\alph*)]
\item $\frac{a}{b^2} = ab^{-2}$
  \item $\frac1{ab^2} = a^{-1}b^{-2}$
  \item $\frac{x}{y^{-4}} = xy^4$
\end{enumerate}
\end{multicols}
}%%


%% %%%%%%%%%%%%%%%%%%%%%%% Aufgabe %%%%%%%%%%%%%%%%%%%%%%%%%%%%%%%%%%%%%%%%%%%%%%%%

\kNiveauAufgabe{%%
\eng{Write as simple as possible (laws of powers may help)}
\deu{Wenden Sie Potenzgesetze an und schreiben Sie so einfach wie möglich}:
\begin{enumerate}[label=\alph*)]
\item $a\cdot{}a^{2x+3} : a^{1-x}$
\item $\left(\frac{x}y\right)^7 : \left(\frac{-x}y\right)^3$
\item $(-b)^{-6}\cdot{}(-b^8)$
\item $(-x^3)^4$
\item $\left(-\left(a^{-1}\right)^{-2}\right)^6$
\item $(-a)^4 : (-a^{10})$
%%  \item $4^k \cdot \left( \frac{1}{2} \right)^k \cdot \left( \frac{1}{3} \right)^{-k}$
\end{enumerate}
\kVerweiseAltesKompendium{9}{17}
\kPlatzFuerBerechnungen{8}
}{%%
\begin{enumerate}[label=\alph*)]
\item $a\cdot{}a^{2x+3} : a^{1-x} = a^{3x+3}$
\item $\left(\frac{x}y\right)^7 : \left(\frac{-x}y\right)^3 = \left(\frac{x}y\right)^4$
\item $(-b)^{-6}\cdot{}(-b^8) = -b^2$
\item $(-x^3)^4 = x^{12}$
\item $\left(-\left(a^{-1}\right)^{-2}\right)^6 = a^{12}$
\item $(-a)^4 : (-a^{10}) = -a^{-6}$
%%\item $4^k \cdot \left( \frac{1}{2} \right)^k \cdot \left(\frac{1}{3} \right)^{-k}$
\end{enumerate}
}%%

%% %%%%%%%%%%%%%%%%%%%%%%% Aufgabe %%%%%%%%%%%%%%%%%%%%%%%%%%%%%%%%%%%%%%%%%%%%%%%%

\kTrainingAufgabe{%%
  \deu{Vereinfachen Sie:}
  \eng{simplify:}
$$\left(\frac{3x^{-2}y^2}{4x^{-4}y^3}\right)^{-2} : \left(\frac{2x^{-1}}{3xy^{-2}}\right)^3$$
\kVerweiseAltesKompendium{9}{18}
\kPlatzFuerBerechnungen{8}
}{%%
$$\left(\frac{3x^{-2}y^2}{4x^{-4}y^3}\right)^{-2} :
  \left(\frac{2x^{-1}}{3xy^{-2}}\right)^3 = \frac{6x^2}{y^4}$$  
}%%


%% %%%%%%%%%%%%%%%%%%%%%%% Aufgabe %%%%%%%%%%%%%%%%%%%%%%%%%%%%%%%%%%%%%%%%%%%%%%%%

\kNiveauAufgabe{%%
  \deu{Vereinfachen Sie:}
  \eng{simplify:}
$$\frac{x-y}{x^2} \cdot{} \left( \frac1x - \frac1y \right)^{-2} : \left(   \frac{x^2-y^2}{x+y} \right)$$
\kVerweiseAltesKompendium{9}{18}
\kPlatzFuerBerechnungen{8}

\kKommentar{Neue Aufgabe: Niveau BMP}
}{%%
$$\frac{y^2}{(y-x)^2}$$
}%%

%%%%%%%%%%%%%%%%%%%%%%%%%%%%%%%%%%%%%%%%%%%%%%%%%%%%%%%%%%%%%%%%%%%%%%%%%%
%%%%%%%%%%%%%%%%%%%%%%% SECTION %%%%%%%%%%%%%%%%%%%%%%%%%%%%%%%%%%%%%%%%%%
%%%%%%%%%%%%%%%%%%%%%%%%%%%%%%%%%%%%%%%%%%%%%%%%%%%%%%%%%%%%%%%%%%%%%%%%%%


\subsubsection{\deu{Wurzeln}\eng{Roots}}

\kMatheNinjaLink{\deu{Wurzeln}\eng{Roots}}{https://matheninja.ch/wurzeln/}

\deu{Vereinfachen Sie folgende Terme. Resultat in Wurzelschreibweise:}
\eng{Simplify the following expressions. Write the results as roots:}

%% %%%%%%%%%%%%%%%%%%%%%%% Aufgabe %%%%%%%%%%%%%%%%%%%%%%%%%%%%%%%%%%%%%%%%%%%%%%%%





\kTrainingAufgabe{%%
\begin{multicols}{2}
\begin{enumerate}[label=\alph*)]
\item $\left(\frac{1}{a}\right)^{-\frac{1}{4}}$
\item $\sqrt{x}\cdot \sqrt[3]{x^4} \cdot \sqrt[6]{x^3}$
\item $3a\cdot \sqrt[3]{9a^2}$
\item $\sqrt[4]{b\cdot \sqrt[3]{b^2\cdot \sqrt{b}}}$
\item $a^{\frac13} \cdot{} \sqrt{ab} \cdot{} \sqrt[4]{b}$
\end{enumerate}
\end{multicols}
\kVerweiseAltesKompendium{9}{19}
\kPlatzFuerBerechnungen{8}
}{%%
\begin{multicols}{2}
\begin{enumerate}[label=\alph*)]
\item $\left(\frac{1}{a}\right)^{-\frac{1}{4}} = \sqrt[4]{a}$
\item $\sqrt{x}\cdot \sqrt[3]{x^4} \cdot \sqrt[6]{x^3} = \sqrt[3]{x^7}$
\item $3a\cdot \sqrt[3]{9a^2} = \sqrt[3]{(3a)^5}$
\item $\sqrt[4]{b\cdot \sqrt[3]{b^2\cdot \sqrt{b}}}  = \sqrt[24]{b^{11}}$
\item $a^{\frac13} \cdot{} \sqrt{ab} \cdot{} \sqrt[4]{b} = \sqrt[12]{a^{10}b^9}$
\end{enumerate}
\end{multicols}
}%%




%% %%%%%%%%%%%%%%%%%%%%%%% Aufgabe %%%%%%%%%%%%%%%%%%%%%%%%%%%%%%%%%%%%%%%%%%%%%%%%

\kNiveauAufgabe{
  
  \kKommentar{gemischte Wurzelausdrücke fehlen}

  \deu{Vereinfachen Sie so weit wie möglich und schreiben Sie in Potenz-
    oder Wurzelschreibweise:}
  \eng{Simplify as much as possible and write in exponent or radical notation:}

  \kKommentar{Noch andere Zahlen nehmen, da teilweise 1:1 aus Matheninja.}

\begin{multicols}{2}
\begin{enumerate}[label=\alph*)]
  \item $$\sqrt[5]{x^4\cdot{}\sqrt[2]{x^4}} \cdot{} \sqrt[4]{x^{-5}}$$

  \item $$\frac{\sqrt[4]{x^{\frac53}}}{x^{\frac16}} + \sqrt[3]{x^4}$$

 \item $$\sqrt[3]{x^4\cdot{} y^{8n}\cdot{}  \sqrt{y^{-36}}} \cdot{} x^{-\frac85}$$

 \item $$ \sqrt[3]{\frac{y^5}{x^6}}   : \sqrt[n]{\frac{x^{3n}}{y^{2n}}}$$
 \end{enumerate}
\end{multicols}

 \kKommentar{neue Aufgabe(n)}
}{%% Lösungen
\begin{enumerate}[label=\alph*)]
  \item $$\sqrt[5]{x^4\cdot{}\sqrt[2]{x^4}} \cdot{} \sqrt[4]{x^{-5}}
    = \sqrt[12]{x^5}$$

  \item $$\frac{\sqrt[4]{x^{\frac53}}}{x^{\frac16}} + \sqrt[3]{x^4} =
    \sqrt[4]{x} + x^3\sqrt{x}$$

 \item $$\sqrt[3]{x^4\cdot{} y^{8n}\cdot{}  \sqrt{y^{-36}}} \cdot{}
   x^{-\frac85} = \sqrt[15]{\frac{x^11y^{40n-90}}{x^{15}}}$$

 \item $$ \sqrt[3]{\frac{y^5}{x^6}}   :
   \sqrt[n]{\frac{x^{3n}}{y^{2n}}} = \sqrt[3]{\frac{y^9\cdot{}y^2}{x^{15}}}$$
 \end{enumerate}
}%% end Aufgabe



%% %%%%%%%%%%%%%%%%%%%%%%% Aufgabe %%%%%%%%%%%%%%%%%%%%%%%%%%%%%%%%%%%%%%%%%%%%%%%%



\kNiveauAufgabe{%%
\begin{multicols}{2}
\begin{enumerate}[label=\alph*)]
\item $\sqrt[3]{26\cdot{}\sqrt[4]{a^3}  + \sqrt[8]{a^6}}$
\end{enumerate}
\end{multicols}
\kPlatzFuerBerechnungen{8}

\kKommentar{neueAufgabe}

}{%%
\begin{multicols}{2}
\begin{enumerate}[label=\alph*)]
\item $\sqrt[3]{26\cdot{}\sqrt[4]{a^3}  + \sqrt[8]{a^6}} = \sqrt[4]{81a}$
\end{enumerate}
\end{multicols}
}%%



%%%%%%%%%%%%%%%%%%%%%%%%%%%%%%%%%%%%%%%%%%%%%%%%%%%%%%%%%%%%%%%%%%%%%%%%%%
%%%%%%%%%%%%%%%%%%%%%%% SECTION %%%%%%%%%%%%%%%%%%%%%%%%%%%%%%%%%%%%%%%%%%
%%%%%%%%%%%%%%%%%%%%%%%%%%%%%%%%%%%%%%%%%%%%%%%%%%%%%%%%%%%%%%%%%%%%%%%%%%


\subsubsection{\deu{Logarithmen}\eng{Logarithms}}
\kMatheNinjaLink{\deu{Logarithmen}\eng{Logarithms}}{https://matheninja.ch/logarithmen/}

\kKommentar{RLP: Hier sind Gleichungen bei Zehnerlogarithmen
  versteckt.}\\
\kKommentar{Aufgaben zu logarithmischen Skalen fehlen!}

\kKommentar{Lerplan schreibt $\frac{\lg(b)}{\lg(a)}$ sei
  entspr. Umschrift }\\
\kKommentar{von $a^x=b$. Doch die ist veraltet. Allg:}\\
\kKommentar{$a^x=b \Longleftrightarrow
  x=\log_a(b)$. $(a,b\in\mathbb{R}^{+}; a\ne 1 )$}

%% %%%%%%%%%%%%%%%%%%%%%%% Aufgabe %%%%%%%%%%%%%%%%%%%%%%%%%%%%%%%%%%%%%%%%%%%%%%%%

\kKommentar{Für Logarithmen fehlen elementare Aufgaben???}

  Es gilt:

$$a^x=b \Longleftrightarrow  \log_a(b) = x$$

\kTrainingAufgabe{
  
  Schreiben Sie in der Potenzschreibweise:

  \begin{enumerate}[label=\alph*)]
   \item $$\log_5(122+x) = 3 $$
   \item $$\log_x(8) = 3 $$
  \end{enumerate}

 \kKommentar{neue Aufgabe (nötig?)}
}{%% Lösung

  \begin{enumerate}[label=\alph*)]
  \item $$\log_5(122+x) =
    3 $$ $$\Longrightarrow$$ $$\doubleunderline{5^3 = 122+x}  \Longrightarrow 125=122+x \Longrightarrow x=3$$
  \item $$\log_x(8) = 3 $$ $$\Longrightarrow$$  $$\doubleunderline{x^3
    = 8} \Longrightarrow x=2$$
  \end{enumerate}

}

%% %%%%%%%%%%%%%%%%%%%%%%% Aufgabe %%%%%%%%%%%%%%%%%%%%%%%%%%%%%%%%%%%%%%%%%%%%%%%%



\kTrainingAufgabe{
  
  Schreiben Sie als Bruch unter der Verwendung des Logarithmus zur
  Basis 10 ($= \lg()$) und vereinfachen Sie so weit wie möglich:

  \begin{enumerate}[label=\alph*)]
   \item $$\log_{33}(122)$$
   \item $$\log_{0.5}(0.01)$$
   \item $$\log_4(1000)$$
  \end{enumerate}

 \kKommentar{neue Aufgabe (nötig?)}
}{%% Lösung


    \begin{enumerate}[label=\alph*)]
   \item $$\log_{33}(122)$$ $$\Longrightarrow  \frac{\lg(122)}{\lg(33)}$$
   \item $$\log_{0.5}(0.01)$$ $$\Longrightarrow
     \frac{\lg(0.01)}{\lg(0.5)} = \frac{-2}{\lg(0.5)}$$
   \item $$\log_{0.1}(1000)$$ $$\Longrightarrow  \frac{\lg(1000)}{\lg(0.1)} = \frac{3}{-1} = -3$$ 
  \end{enumerate}
}

%% %%%%%%%%%%%%%%%%%%%%%%% Aufgabe %%%%%%%%%%%%%%%%%%%%%%%%%%%%%%%%%%%%%%%%%%%%%%%%


\kTrainingAufgabe{
  Schreiben Sie in der Logarithmischen Schreibweise:

\kKommentar{Zwar nicht RLP, aber gut fürs Verständnis.}
  
  \begin{enumerate}[label=\alph*)]
  \item $$10^{x+2.5} = 1000$$
  \item $$10^{2x-3} = 0.01$$
  \end{enumerate}
  
  \kKommentar{neue Aufgabe (nötig?)}
}{%% Lösung

  \begin{enumerate}[label=\alph*)]
  \item $$\doubleunderline{10^{x+2.5} = 1000} \Longrightarrow  \lg(1000) = x+2.5  \Longrightarrow 3 = x+2.5 \Longrightarrow x=0.5$$
  \item $$10^{2x-3} = 0.01 \Longrightarrow \log_{10}(0.01) = 2x-3
    \Longrightarrow -2 = 2x-3 \Longrightarrow x=0.5$$
  \end{enumerate}

}


\newpage
