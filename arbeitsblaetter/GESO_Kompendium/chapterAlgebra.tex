{\section{Algebra}\index{Algebra}
\setcounter{aufgabenNummer}{1}

\deu{\subsection{Zahlen}\index{Zahlmengen}}
\deu{\subsection{Numbers}\index{Numbers}}

\kTrainingAufgabe{
\deu{Ordnen Sie die folgenden Zahlen der am weitesten links stehenden
Zahlmenge zu}\eng{Find the least powefull set for the given number
(the rightmost ist the most powerfull set of numbers)}:

$$\mathbb{N} \subset \mathbb{Z} \subset \mathbb{Q} \subset \mathbb{R}$$

\begin{multicols}{2}
\begin{enumerate}[label=\alph*)]
 \item$-\sqrt{\frac{16}{2}}$
 \item$|-\pi|$
 \item$1.\overline{25}$
 \item$\frac{\sqrt{8}}{\sqrt{2}}$
\end{enumerate}
\end{multicols}

}{%% Lösungsteil
\begin{multicols}{2}
\begin{enumerate}[label=\alph*)]
 \item$-\sqrt{\frac{16}{2}}$ : $\mathbb{R}$
 \item$|-\pi|$  : $\mathbb{R}$
 \item$1.\overline{25}$ : $\mathbb{Q}$
 \item$\frac{\sqrt{8}}{\sqrt{2}}$ : $\mathbb{N}$
\end{enumerate}
\end{multicols}
}%% end kTrainingAufgabe


\kTrainingAufgabe{%%
Setzen Sie zwischen die Terme (an die Stelle der drei Punkte) die
richtigen Relationszeichen ($<$, $=$, $>$):

\begin{multicols}{2}
\begin{enumerate}[label=\alph*)]
 \item$|5-2| ... |2-5|$
 \item$-\sqrt{3} ... -\sqrt{2}$ 
 \item$-3 ... |-3|$ 
 \item$\frac{1}{3} ... \frac{1}{4}$ 
 \item$\frac{2}{3} ... \frac{3}{4}$
 \item$2.\overline{9}$ ... 3
 \item$-\frac{1}{3} ... -\frac{1}{4}$ 
\end{enumerate}
\end{multicols}
}{%% Lösungsteil
\begin{multicols}{2}
\begin{enumerate}[label=\alph*)]
 \item$|5-2| =  |2-5|$
 \item$-\sqrt{3} < -\sqrt{2}$ 
 \item$-3 < |-3|$ 
 \item$\frac{1}{3} > \frac{1}{4}$
 \item$\frac{2}{3} < \frac{3}{4}$
 \item$2.\overline{9} = 3$
 \item$-\frac{1}{3} <-\frac{1}{4}$
\end{enumerate}
\end{multicols}
}{8}%% end Kaufgabe Training


\subsection{Terme}

\kTrainingAufgabe{
Benennen Sie die folgenden Terme mit dem richtigen Begriff
(Jeweils einer aus: \textit{Summe}, \textit{Differenz}, \textit{Produkt},
\textit{Quotient}, \textit{Potenz}):

\begin{multicols}{3}
\begin{enumerate}[label=\alph*)]
 \item $(3+x)\cdot{}2 + y$ 
 \item $(4a)^x$
 \item$4a^x$
 \item$(x-2y^4z)^2$
 \item$2a^7 - 4bc(z-1)^3$
 \item$(2a -b) : x + z : 4$
 \item$(a-4)(z+3)$ 
 \item$(2a-b):(x+z:4)$ 
\end{enumerate}
\end{multicols}
}{
\begin{multicols}{3}
\begin{enumerate}[label=\alph*)]
 \item $(3+x)\cdot{}2 + y$ :Summe
 \item $(4a)^x$ : Potenz
 \item$4a^x$ : Produkt
 \item$(x-2y^4z)^2$ : Potenz
 \item$2a^7 - 4bc(z-1)^3$ : Differenz
 \item$(2a -b) : x + z : 4$  : Summe
 \item$(a-4)(z+3)$   :Produkt
 \item$(2a-b):(x+z:4)$ :Quotient
\end{enumerate}
\end{multicols}
}{8}%% end kaufgabe

\subsection{Betrag}

\deu{Vereinfachen Sie}\eng{simplify}:

\kTrainingAufgabe{$|-4|$}{$4$}{4}
\kTrainingAufgabe{$-\big| -5-|-8|  \big|$}{$-13$}{4}

\deu{Terme}\eng{terms}:

\kNiveauAufgabe{$T(x) = (-4)\cdot{}|x-8|\cdot{}(x^2)$\\
\deu{Berechnen Sie}\eng{calculate}: $$T(-2)$$ }{$80$}{4}%% end kaufgabe



\deu{\subsection{Runden}\index{runden}}
\eng{\subsection{round}\index{round}}

\deu{Runden Sie je auf zwei Dezimalen}
\eng{Round to two decimal places}:

\kTrainingAufgabe{4.555}{4.56}{4}

\kNiveauAufgabe{4.8956}{4.90}{4}

\subsection{Addition / Subtraktion}

\subsection{Multiplikation / Division}

\subsubsection{Faktorisieren}

\subsubsection{Bruchterme}
