{\section{Algebra}\index{Algebra}
  \setcounter{aufgabenNummer}{1}
  \renewcommand{\kAufgabenBuchstabe}{A}

\kMatheNinjaLink{Grundlagen}{https://matheninja.ch/terme/}
  
\deu{\subsection{Zahlen}\index{Zahlmengen}}
\eng{\subsection{Numbers}\index{Numbers}}

%%%%%%%%%%%%%%%%%%%%%%%%%%%%%%%%%%%%%%%%%%%%%%%%%%%%%%%%%%%%%%%%%%%%%%
%% Aufgabe Zahlmengen
\kTrainingAufgabe{
\deu{Ordnen Sie die folgenden Zahlen allen möglichen Zahlmengen
  $\mathbb{N}$, $\mathbb{Z}$, $\mathbb{Q}$ oder $\mathbb{R}$
  zu}\eng{Find all suitable sets for the given number
(the rightmost ist the most powerfull set of numbers)}:
\\
\eng{$$\mathbb{N} \subset \mathbb{Z} \subset \mathbb{Q} \subset \mathbb{R}$$}
\begin{multicols}{2}
\begin{enumerate}[label=\alph*)]
 \item$-\sqrt{\frac{16}{2}}$
 \item$|-\pi|$
 \item$1.\overline{25}$
 \item$\frac{\sqrt{8}}{\sqrt{2}}$
\end{enumerate}
\end{multicols}
}{%% Lösungsteil
\begin{multicols}{2}
\begin{enumerate}[label=\alph*)]
 \item$-\sqrt{\frac{16}{2}}$ : $\mathbb{R}$
 \item$|-\pi|$  : $\mathbb{R}$
 \item$1.\overline{25}$ : $\mathbb{Q}, \mathbb{R}$
 \item$\frac{\sqrt{8}}{\sqrt{2}}$ : $\mathbb{N}, \mathbb{Z},
   \mathbb{Q}, \mathbb{R}$
\end{enumerate}
\end{multicols}
}{4}%% end kTrainingAufgabe

%%%%%%%%%%%%%%%%%%%%%%%%%%%%%%%%%%%%%%%%%%%%%%%%%%%%%%%%%%%%%%%%%%%%%%
%% Aufgabe Relationszeichken
\kTrainingAufgabe{%%
\deu{Setzen Sie zwischen die Terme (an die Stelle der drei Punkte) die
  richtigen Relationszeichen ($<$, $=$, $>$)}
\eng{Set the correct symbol ($<$, $=$, $>$) between the numbers}:
\\
\begin{multicols}{2}
\begin{enumerate}[label=\alph*)]
 \item$|5-2| ... |2-5|$
 \item$-\sqrt{3} ... -\sqrt{2}$ 
 \item$-3 ... |-3|$ 
 \item$\frac{1}{3} ... \frac{1}{4}$ 
 \item$\frac{2}{3} ... \frac{3}{4}$
 \item$-\frac{1}{3} ... -\frac{1}{4}$
 \item$-\frac{8}{\sqrt{4}} ... (-2)^2$
\end{enumerate}
\end{multicols}
}{%% Lösungsteil
\begin{multicols}{2}
\begin{enumerate}[label=\alph*)]
 \item$|5-2| =  |2-5|$
 \item$-\sqrt{3} < -\sqrt{2}$ 
 \item$-3 < |-3|$ 
 \item$\frac{1}{3} > \frac{1}{4}$
 \item$\frac{2}{3} < \frac{3}{4}$
 \item$-\frac{1}{3} <-\frac{1}{4}$
 \item$-\frac{8}{\sqrt{4}} = (-2)^2$
\end{enumerate}
\end{multicols}
}{8}%% end Kaufgabe Training


\subsection{\deu{Terme}\eng{Expressions}}\index{\deu{Terme}\eng{expressions}}

\kTrainingAufgabe{
\deu{Benennen Sie die folgenden Terme mit dem richtigen Begriff
(Jeweils einer aus: \textit{Summe}, \textit{Differenz}, \textit{Produkt},
  \textit{Quotient}, \textit{Potenz})}
\eng{Name the following expressions using the correct concept (Sum, Difference, Product, Quotient, Potency)}:
\\
\begin{multicols}{3}
\begin{enumerate}[label=\alph*)]
 \item $(3+x)\cdot{}2 + y$ 
 \item $(4a)^x$
 \item$4a^x$
 \item$(x-2y^4z)^2$
 \item$2a^7 - 4bc(z-1)^3$
 \item$(2a -b) : x + z : 4$
 \item$(a-4)(z+3)$ 
 \item$(2a-b):(x+z:4)$ 
\end{enumerate}
\end{multicols}
}{
\begin{multicols}{3}
\begin{enumerate}[label=\alph*)]
 \item $(3+x)\cdot{}2 + y$ :\deu{Summe}\eng{Sum}
 \item $(4a)^x$ : \deu{Potenz}\eng{Potency}
 \item$4a^x$ : \deu{Produkt}\eng{Product}
 \item$(x-2y^4z)^2$ : \deu{Potenz}\eng{Potency}
 \item$2a^7 - 4bc(z-1)^3$ : \deu{Differenz}\eng{Difference}
 \item$(2a -b) : x + z : 4$  : \deu{Summe}\eng{Sum}
 \item$(a-4)(z+3)$   : \deu{Produkt}\eng{Product}
 \item$(2a-b):(x+z:4)$ : \deu{Quotient}\eng{Quotient}
\end{enumerate}
\end{multicols}
}{8}%% end kaufgabe

%%\subsection{\deu{Betrag}\eng{absolute value}}\deu{\index{Betrag}]  \eng{\index{absolute value}}
%%\subsection{\deu{Betrag}\eng{absolute value}}\index{\deu{Betrag}\eng{absolute value}}



\entfernteAufgabe{
Lösen Sie in der Grundmenge $\mathbb{Z}$:
 \\
\begin{multicols}{2}
\begin{enumerate}[label=\alph*)]
 \item$|x+4| = 5$x
 \item$|2x-1| = 0$
 \item$|-x + 7| = 10$
 \item$|-x|=-1$ 
\end{enumerate}
\end{multicols}
}{ Lösungsteil
\begin{multicols}{2}
\begin{enumerate}[label=\alph*)]
 \item$|x+4| = 5$  : $\LoesungsMenge{}=\{1,-9\}$
 \item$|2x-1| = 0$ : $\LoesungsMenge{}=\{0.5\}$
 \item$|-x + 7| = 10$ :  $\LoesungsMenge{}=\{-3, 17\}$
 \item$|-x|=-1$ : $\LoesungsMenge{}=\{\}$
\end{enumerate}
\end{multicols}
}{4}

\deu{Vereinfachen Sie}\eng{simplify}:

\kTrainingAufgabe{$|-4|$}{$4$}{4}
\kTrainingAufgabe{$-\big| -5-|-8|  \big|$}{$-13$}{4}

\deu{Terme}\eng{terms}:

\kNiveauAufgabe{\deu{Gegeben ist der folgende Term}\eng{The following
    expression is given:}
$$T(x) = (-4)\cdot{}|x-8|\cdot{}(x^2)$$
\deu{Berechnen Sie}\eng{calculate}: $$T(-2)$$ }{$80$}{4}%% end kaufgabe



\deu{\subsection{Runden}\index{runden}}
\eng{\subsection{round}\index{round}}

\deu{Runden Sie je auf zwei Dezimalen}
\eng{Round to two decimal places}:

\kTrainingAufgabe{$2.245$}{$4.56$}{2}
\kNiveauAufgabe{$2.995$}{$3.00$}{2}
\kNiveauAufgabe{$2.9949$}{$2.99$}{2.5}
\kNiveauAufgabe{$-2.9949$}{$-2.99$}{2}
\kTrainingAufgabe{$\sqrt{3}$}{1.73}{2}




\entfernteAufgabe{%%
\deu{Runden Sie auf 2 Dezimalstellen in der wissenschaftlichen Notation:}
\eng{Round to 2 decimal places in scientific notation:}
\begin{multicols}{2}
\begin{enumerate}[label=\alph*)]
 \item 12.05649
 \item 0.028249
 \item 2\,498\,900
\end{enumerate}
\end{multicols}
}{%%
\begin{multicols}{2}
\begin{enumerate}[label=\alph*)]
 \item 12.05649  =1.21 $\cdot 10^1$
 \item 0.028249 =2.82 $\cdot 10^{-2}$
 \item 2\,498\,900 = 2.50 $\cdot 10^{6}$
\end{enumerate}
\end{multicols}
}{8}%%



\deu{\subsection{Addition / Subtraktion}}\eng{\subsection{Addition / Subtraction}}
%% Summenzeichen
\deu{\subsubsection{Summenzeichen}} \eng{\subsubsection{Sum sign}}

\kNiveauAufgabe{%%
\deu{Schreiben Sie die Summanden hin und berechnen Sie den Wert des
  Ausdrucks}
\eng{Write down the summands and calculate the value of the expression}:
\\
\begin{enumerate}[label=\alph*)]
 \item $$\sum_{k=1}^{k=4}{(2^k+k)}$$
 \item $$\sum_{k=2}^{k=4}{(3k+2^k)}$$
% \item $$\sum_{i=4}^{i=7}(3i) + 6$$ 
\end{enumerate}
}{%%
\begin{enumerate}[label=\alph*)]
 \item $$\sum_{k=1}^{k=4}{(2^k+k)} = (2^1+1) + (2^2+2) + (2^3+3) + (2^4+4) = 3 + 6 + 11 + 20 = 40$$
 \item $$\sum_{k=2}^{k=4}{(3k+2^k)}$$ : $(6+4) + (9+8) + (12+16) = 10+17+28=55$
% \item $$\sum_{i=4}^{i=7}(3i) + 6$$ : $((12) + (15) + (18) +(21) )+6=72$
\end{enumerate}
}{4}%%

\subsection{Multiplikation / Division}


\kTrainingAufgabe{%%
\deu{Schreiben Sie als Summen}\eng{Write as a sum}:
\begin{multicols}{2}
\begin{enumerate}[label=\alph*)]
\item $\left(5m-\frac{1}{2}n\right)^2$
\item $(t-9s^2)\cdot(t+9s^2)$
\item $(3a + b^4)^2$
\end{enumerate}
\end{multicols}
}{%%
\begin{multicols}{2}
\begin{enumerate}[label=\alph*)]
\item $\left(5m-\frac{1}{2}n\right)^2 = 25m^2 -5mn + \frac14n^2$
\item $(t-9s^2)\cdot(t+9s^2) = t^2 + 81s^4$
  \item $(3a + b^4)^2 = 9a^2 + 6ab^4 + b^8$
\end{enumerate}
\end{multicols}
}{4}%%

  
\subsubsection{\deu{Faktorisieren}\eng{factorize}}\index{\deu{Faktorzerlegung}\eng{factorize}}


\kTrainingAufgabe{%%
\deu{Zerlegen Sie in Faktoren}\eng{Factorize the expression}:
\begin{multicols}{2}
\begin{enumerate}[label=\alph*)]
\item $4x^2-y^2$
\item $36a^2 + 36ab + 9b^2$
\item $4x^4-28x^2y+49y^2$
\item $a^3-a$
\item $x^3-6x^2y+9xy^2$
\item $z^4 -3^2$
\end{enumerate}
\end{multicols}
}{%%
\begin{multicols}{2}
\begin{enumerate}[label=\alph*)]
\item $4x^2-y^2 = (4x+y)(4x-y)$
\item $36a^2 + 36ab + 9b^2 = (6a+3b)^2$
\item $4x^4-28x^2y+49y^2 = (2x-7y)^2$
\item $a^3-a = a\cdot{}(a+1)(a-1)$
\item $x^3-6x^2y+9xy^2 = x\cdot{}(x-3y)^2$
\item $z^4 -3^2 = (z^2+w)(z^2-w)$
\end{enumerate}
\end{multicols}
}{4}%%


\kTrainingAufgabe{%%
  \deu{Zerlegen Sie in Faktoren}\eng{Break the expressions down into factors}:\\
(\deu{Alle Aufgaben sind mit dem Taschenrechner
    lösbar.}\eng{All tasks can be solved with a calculator.})
\begin{multicols}{2}
\begin{enumerate}[label=\alph*)]
\item $a^2 + 2a -15$
\item $a^3-a^2-6a$
\item $a^2 - 20a + 75$
  \item $a^2 -19a + 48$
\end{enumerate}
\end{multicols}
}{%%
\begin{multicols}{2}
\begin{enumerate}[label=\alph*)]
\item $a^2 + 2a -15 = (a+5)(a-3)$
\item $a^3-a^2-6a = a(a-3)(a+2)$
\item $a^2 - 20a + 75 = (a+15)(a+5)$
  \item $a^2 -19a + 48 = (a-16)(a-3)$
\end{enumerate}
\end{multicols}
}{4}%%


\kTrainingAufgabe{%%
  \deu{Zerlegen Sie in Faktoren}\eng{Devide the terms into factors}:\\
\begin{enumerate}[label=\alph*)]
\item $ab(x-2y)-b(x-2y)$
\item $(a-2b)(m+n)+(a-2b)(3m+n)$
\item $x(y-2z)-(y-2z)$
\item $3a-6ab+c-2bc$
%%  \item $ac-ad+bc-bd$
\end{enumerate}
}{%%
\begin{enumerate}[label=\alph*)]
\item $ab(x-2y)-b(x-2y) = b(x-2y)(a-1)$
\item $(a-2b)(m+n)+(a-2b)(3m+n) = 2(a-2b)(2m+n)$
\item $x(y-2z)-(y-2z) = (y-2z)(x-1)$
\item $3a-6ab+c-2bc = (1-2b)(3a+c)$
%%  \item $ac-ad+bc-bd$
\end{enumerate}
}{4}%%


\subsubsection{\deu{Bruchterme}\eng{Fractional terms}}


\kNiveauAufgabe{%%
  \eng{Determine the domain and simplify as much as
    possible.}
  \deu{Bestimmen Sie die Definitionsmenge und kürzen Sie
    soweit wie möglich.}
  $G = \mathbb{R}$.
\begin{multicols}{2}
\begin{enumerate}[label=\alph*)]
\item $\frac{x^2-16}{x^2+x-20}$
\item $\frac{x^2+x-42}{x^2-4x-12}$
\end{enumerate}
\end{multicols}
}{%%
\begin{multicols}{2}
\begin{enumerate}[label=\alph*)]
\item $\frac{x^2-16}{x^2+x-20}$ = $\frac{x+4}{x+5}$ :   $D=\mathbb{R} \backslash \{-5; 4\}$
\item $\frac{x^2+x-42}{x^2-4x-12}$ = $\frac{x+7}{x+2}$ : $D=\mathbb{R} \backslash \{-2; 6\}$ 
\end{enumerate}
\end{multicols}
}{4}%%

  

\kNiveauAufgabe{%%
  \eng{Determine the domain and simplify as much as
    possible.}
  \deu{Bestimmen Sie die Definitionsmenge und kürzen Sie
    soweit wie möglich.}
  $G = \mathbb{R}$.
\begin{multicols}{2}
\begin{enumerate}[label=\alph*)]
\item $\frac{st-3t^2}{s^2-st} + \frac{3s^2-3st-st+t^2}{s^2-2st+t^2}$
%%\item $\frac{\frac{a^2-16c^2}{8a^2}}{\frac{a-4c}{4a}}$
\item $\left(\frac{1}{a^2} - \frac{1}{b^2}\right) \cdot
    \left(\frac{a}{a+b} + \frac{b}{a-b}\right)$
\item $\left( \frac{a+4}{a} - \frac{b+4}{b}\right) : \frac{a-b}{a}$    
\end{enumerate}
\end{multicols}
}{%%
\begin{multicols}{2}
\begin{enumerate}[label=\alph*)]
\item $\frac{st-3t^2}{s^2-st} + \frac{3s^2-3st-st+t^2}{s^2-2st+t^2} = \frac{3s+t}s$
%%\item $\frac{\frac{a^2-16c^2}{8a^2}}{\frac{a-4c}{4a}}$
\item $\left(\frac{1}{a^2} - \frac{1}{b^2}\right) \cdot
    \left(\frac{a}{a+b} + \frac{b}{a-b}\right) = \frac{a+4c}{2a}$
\item $\left( \frac{a+4}{a} - \frac{b+4}{b}\right) : \frac{a-b}{a} = \frac{-4}b$    
\end{enumerate}
\end{multicols}
}{4}%%


\subsection{\eng{Powers}\deu{Potenzen}}


\kTrainingAufgabe{%%
\eng{Write as a single fraction avoiding negative exponents}%%
\deu{Schreiben Sie als Bruch ohne negative Exponenten}:
\begin{multicols}{2}
\begin{enumerate}[label=\alph*)]
\item $2x^{-3}$
  \item $ab^{-5}$
\end{enumerate}
\end{multicols}
}{%%
\begin{multicols}{2}
\begin{enumerate}[label=\alph*)]
\item $2x^{-3} = \frac2{x^3}$
\item $ab^{-5} = \frac{a}{b^5}$
\end{enumerate}
\end{multicols}
}{4}%%


\kTrainingAufgabe{%%
  \eng{Simplify avoiding fractional terms}%
  \deu{Schreiben Sie die Terme ohne Quotienten}:
\begin{multicols}{3}
\begin{enumerate}[label=\alph*)]
\item $\frac{a}{b^2}$
  \item $\frac1{ab^2}$
  \item $\frac{x}{y^{-4}}$
\end{enumerate}
\end{multicols}
}{%%
\begin{multicols}{3}
\begin{enumerate}[label=\alph*)]
\item $\frac{a}{b^2} = ab^{-2}$
  \item $\frac1{ab^2} = a^{-1}b^{-2}$
  \item $\frac{x}{y^{-4}} = xy^4$
\end{enumerate}
\end{multicols}
}{4}%%


\kNiveauAufgabe{%%
\eng{Write as simple as possible (laws of powers may help)}
\deu{Wenden Sie Potenzgesetze an und schreiben Sie so einfach wie möglich}:
\begin{enumerate}[label=\alph*)]
\item $a\cdot{}a^{2x+3} : a^{1-x}$
\item $\left(\frac{x}y\right)^7 : \left(\frac{-x}y\right)^3$
\item $(-b)^{-6}\cdot{}(-b^8)$
\item $(-x^3)^4$
\item $\left(-\left(a^{-1}\right)^{-2}\right)^6$
\item $(-a)^4 : (-a^{10})$
%%  \item $4^k \cdot \left( \frac{1}{2} \right)^k \cdot \left( \frac{1}{3} \right)^{-k}$
\end{enumerate}
}{%%
\begin{enumerate}[label=\alph*)]
\item $a\cdot{}a^{2x+3} : a^{1-x} = a^{3x+3}$
\item $\left(\frac{x}y\right)^7 : \left(\frac{-x}y\right)^3 = \left(\frac{x}y\right)^4$
\item $(-b)^{-6}\cdot{}(-b^8) = -b^2$
\item $(-x^3)^4 = x^{12}$
\item $\left(-\left(a^{-1}\right)^{-2}\right)^6 = a^{12}$
\item $(-a)^4 : (-a^{10}) = -a^{-6}$
%%\item $4^k \cdot \left( \frac{1}{2} \right)^k \cdot \left(\frac{1}{3} \right)^{-k}$
\end{enumerate}
}{4}%%

\kTrainingAufgabe{%%
  \deu{Vereinfachen Sie:}
  \eng{simplify:}
$$\left(\frac{3x^{-2}y^2}{4x^{-4}y^3}\right)^{-2} : \left(\frac{2x^{-1}}{3xy^{-2}}\right)^3$$
}{%%
$$\left(\frac{3x^{-2}y^2}{4x^{-4}y^3}\right)^{-2} :
  \left(\frac{2x^{-1}}{3xy^{-2}}\right)^3 = \frac{6x^2}{y^4}$$  
}{4}%%


\subsubsection{\deu{Wurzeln}\eng{Roots}}

\deu{Vereinfachen Sie folgende Terme. Resultat in Wurzelschreibweise:}
\eng{Simplify the following expressions. Write the results as roots:}

\kTrainingAufgabe{%%
\begin{multicols}{2}
\begin{enumerate}[label=\alph*)]
\item $\left(\frac{1}{a}\right)^{-\frac{1}{4}}$
\item $\sqrt{x}\cdot \sqrt[3]{x^4} \cdot \sqrt[6]{x^3}$
\item $3a\cdot \sqrt[3]{9a^2}$
\item $\sqrt[4]{b\cdot \sqrt[3]{b^2\cdot \sqrt{b}}}$
\item $a^{\frac13} \cdot{} \sqrt{ab} \cdot{} \sqrt[4]{b}$
\item $\sqrt[3]{26\cdot{}\sqrt[4]{a^3}  + \sqrt[8]{a^6}}$
\end{enumerate}
\end{multicols}
}{%%
\begin{multicols}{2}
\begin{enumerate}[label=\alph*)]
\item $\left(\frac{1}{a}\right)^{-\frac{1}{4}} = \sqrt[4]{a}$
\item $\sqrt{x}\cdot \sqrt[3]{x^4} \cdot \sqrt[6]{x^3} = \sqrt[3]{x^7}$
\item $3a\cdot \sqrt[3]{9a^2} = \sqrt[3]{(3a)^5}$
\item $\sqrt[4]{b\cdot \sqrt[3]{b^2\cdot \sqrt{b}}}  = \sqrt[24]{b^{11}}$
\item $a^{\frac13} \cdot{} \sqrt{ab} \cdot{} \sqrt[4]{b} = \sqrt[12]{a^{10}b^9}$
\item $\sqrt[3]{26\cdot{}\sqrt[4]{a^3}  + \sqrt[8]{a^6}} = \sqrt[4]{81a}$
\end{enumerate}
\end{multicols}
}{4}%%


\newpage
