%%
%% Meta: Boxplot erstellen. Viele Beispiele
%%

\input{bmsLayoutPage}

%%%%%%%%%%%%%%%%%%%%%%%%%%%%%%%%%%%%%%%%%%%%%%%%%%%%%%%%%%%%%%%%%%

%\usepackage{amssymb} 
\usepackage{cancel}
\renewcommand{\metaHeaderLine}{Arbeitsblatt}
\renewcommand{\arbeitsblattTitel}{Boxplotdaten mit dem TI-30X Pro MathPrint}

\begin{document}%%
\arbeitsblattHeader{}

Diese Anleitung dient dazu, mit dem TI-30X Pro MathPrint
Taschenrechner die wesentlichen Daten für den Boxplot zu berechnen.

\section{Voraussetzung}
Die Datenreihe ist bereits vorhanden. Sie muss nicht unbedingt
sortiert sein:

4, 8, 5, 23, 9, 6, 11, 9, 6, 6, 8

\section{Data-Mode}
\begin{enumerate}
\item
Schalten Sie den Taschenrechner ein und drücken Sie \fbox{clear}, um
in die Standardanzeige (blinkender Cursor) zu gelangen. Eventuell
müssen Sie noch \fbox{2nd}\fbox{mode} (= quit) drücken, um in den
Standardmodus zu gelangen.

\item
Drücken Sie \fbox{data} um in den Eingabemodus für Datenreihen zu
gelangen.

\item
Mit Eingabe der Zahlen und anschließendem Drücken der
Taste \fbox{enter} werden die Daten in die Liste eingetragen.

\item Drücken Sie nach Eingabe des Letzten Wertes die
Tastenkombination \fbox{2nd}\fbox{data} (=stat-reg/distr) um in die
Auswertung zu gelangen. Wählen Sie die «1-VAR STATS» indem Sie die
Taste \fbox{2} drücken.

\item
Unter «1-VAR STATS» sind die Standardeinstellungen auf «DATA: L1» und
«FREQ: ONE». Ändern Sie hier nichts, sondern wählen Sie direkt «CALC»
aus.

\item
Mit Pfeiltasten gelangen Sie zu den Resultaten der
Auswertung. Kontrolle: Für die oben angegeben Werte erhalten Sie:
\begin{itemize}
\item minX = 4
\item Q1=6
\item Med=8
\item Q3=9
\item maxX=23\end{itemize}

\end{enumerate}

\end{document}
