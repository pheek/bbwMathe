\input{bbwLayoutPage}
\renewcommand{\bbwAufgabenBlockID}{A1F}

\renewcommand{\metaHeaderLine}{Aufgabenblatt}
\renewcommand{\arbeitsblattTitel}{Faktorisieren}


\begin{document}
\arbeitsblattHeader{}

Klammern Sie aus:
\begin{bbwAufgabenBlock}
\item $5a+10b = \LoesungsRaumLang{5(a+2b)}$
\item $x^2+x^3-4x^4 =\LoesungsRaumLang{x^2(1+x-4x^2)}$
\item $36a^5+18a^3+24a^4 = \LoesungsRaumLang{6a^3(6a^2+4a+3)}$
\item $15xy+14xz+16yz = $\LoesungsRaumLang{\textrm{Keine gemeinsamen Faktoren: Ist nicht faktorisierbar!}}
\item $\sqrt{2}ab+\sqrt{3}ac + \sqrt{4}ad= \LoesungsRaumLang{a(\sqrt{2}b + \sqrt{3}c + 2d)}$
\item $21\alpha\gamma\sigma + 6\alpha\pi\gamma + 27\bar{x}\gamma\alpha = \LoesungsRaumLang{3\alpha\gamma(7\sigma+2\pi+9\bar{x})}$

\end{bbwAufgabenBlock}

\platzFuerBerechnungenBisEndeSeite{}


%%%%%%%%%%%%%%%%%%%%%%%%%%%%%%%%%%%%%%%%%%%%%%%%%5

Klammern Sie $(-1)$ aus:


\begin{bbwAufgabenBlock}
\item $(x-3) = \LoesungsRaumLang{(-1)(3-x)}$
\item $-5a-7b = \LoesungsRaumLang{-(5a+7b)}$
\item $3+y = \LoesungsRaumLang{-(-3-y)}$
\item $-a+b-c-d = \LoesungsRaumLang{-(a-b+c+d)}$
\item $a-2b-(c+2) = \LoesungsRaumLang{-(-a+2b+c+2)}$
\end{bbwAufgabenBlock}

\platzFuerBerechnungenBisEndeSeite{}


%%%%%%%%%%%%%%%%%%%%%%%%%%%%%%%%%%%%%%%%%%%%%%%%%5

Klammern Sie identische Klammerausdrücke aus:


\begin{bbwAufgabenBlock}
\item $2(a+b) + r(a+b) = \LoesungsRaumLang{(2+r)(a+b)}$
\item $a^2(x+y+z) + x+y+1z = \LoesungsRaumLang{(a^2+1)(x+y+z)}$
\item $6a(x+z) - 2a(x+z) -4(x+z) + 8(x+z) = \LoesungsRaumLang{4(a+1)(x+z)}$
\item $(3-4r)(s+5t) + 4r(5t+s) \LoesungsRaumLang{3(s+5t)}$

\end{bbwAufgabenBlock}

\platzFuerBerechnungenBisEndeSeite{}


%%%%%%%%%%%%%%%%%%%%%%%%%%%%%%%%%%%%%%%%%%%%%%%%%5

%% Skript: Minus Eins ausklammern II
Erstellen Sie gleiche Klammerausdrücke durch Ausklammern von -1 und
faktorisieren Sie so weit wie möglich.

\begin{bbwAufgabenBlock}
\item $7(a-b) + 2(b-a) = \LoesungsRaumLang{5(a-b)}$
\item $\sqrt{2}(5-b) - \sqrt{3}(b-5)=\LoesungsRaumLang{(\sqrt{2}+\sqrt{3})(5-b)}$
\item $a^2(x-yz) - b^2(yz-x) = \LoesungsRaumLang{(a^2+b^2)(x-yz)}$
\item $a(x-y+z)+b(y-x-z) - c(z-y+x) =\LoesungsRaumLang{(a-b-c)(x-y+z)}$
\end{bbwAufgabenBlock}

\platzFuerBerechnungenBisEndeSeite{}




%%%%%%%%%%%%%%%%%%%%%%%%%%%%%%%%%%%%%%%%%%%%%%%%%5

%% Teilsummen
Mehrmaliges Ausklammern: Erstellen Sie erst identische Klammerausdrücke

\begin{bbwAufgabenBlock}
\item $ax+ay-(x+y)b = \LoesungsRaumLang{(a-b)(x+y)}$
\item $xz+2xs+yz+2ys = \LoesungsRaumLang{(x+y)(z+2s)}$
\item $2a+2b + ra+rb = \LoesungsRaumLang{(2+r)(a+b)}$
\item $xz-x+yz-y-tz+t = \LoesungsRaumLang{x+y-t)(z-1)}$
\item $z^3-z^2 + (z-1)a^2 = \LoesungsRaumLang{(z^2+a^2)(z-1)}$
\item $10ab+5a - 20b -10 = \LoesungsRaumLang{5(a-2)(2b+1)}$
\item $a(x-y) + by-bx = \LoesungsRaumLang{5(a-b)}$
\item $(b-3)(x-y)-x+y = \LoesungsRaumLang{(b-4)(x-y)}$
\item $x(r-s+t)+ ys-yr-yt - z(t-s+r) =\LoesungsRaumLang{(x-y-z)(r-s+t)}$

\item $a(x+y) - b(x-y) = \LoesungsRaumLang{\textrm{kann nicht Faktorisiert werden}}$

\end{bbwAufgabenBlock}


\platzFuerBerechnungenBisEndeSeite{}




%%%%%%%%%%%%%%%%%%%%%%%%%%%%%%%%%%%%%%%%%%%%%%%%%5


Faktorisieren Sie mit Hilfe der binomischen Formeln:


\begin{bbwAufgabenBlock}
\item $r^2-s^2 = \LoesungsRaumLang{(r+s)(r-s)}$
\item $9x^2 - 25y^2 = \LoesungsRaumLang{(3x+5y)(3x-5y)}$
\item $100x^2 -1 = \LoesungsRaumLang{(10x+1)(10x-1)}$
\item $1-f^6 = \LoesungsRaumLang{(1+f^3)(1-f^3)}$
\item $-16r^2 + 9s^2 = \LoesungsRaumLang{(3s+4r)(3s-4r)}$
\item $25a^4-2 =  \LoesungsRaumLang{\textrm{Kann ohne Wurzeln nicht faktorisiert werden}. Sonst: (5a^2 + \sqrt{2})(5a^2-\sqrt{2})}$
\item $b^2+2bc+c^2 = \LoesungsRaumLang{(b+c)^2}$
\item $4a^2+4a+1 = \LoesungsRaumLang{(2a+1)^2}$
\item $36x^2 + 84xy + 49y^2 = \LoesungsRaumLang{(6x+7y)^2}$
\item $-24a^2 - 24a -6 = \LoesungsRaumLang{-6(2a+1)^2}$
\item $a^2(xy+x) - b^2(xy+x) = \LoesungsRaumLang{(a+b)(a-b)(y+1)\cdot{}x}$
\item $25x^2 - (3p-6)^2 = \LoesungsRaumLang{(5x+3p-6)(5x-3p+6)}$
\item $(x+y)^2 - (x+y-z)^2 = \LoesungsRaumLang{(2x+2y-z)\cdot{}z)}$
\item $a^2+2a+1-25b^2 = \LoesungsRaumLang{(a+5b+1)(a-5b+1)}$
\item $1-x^2-2xy-y^2 = \LoesungsRaumLang{(x+y+1)(-x-y+1)}$
\item $x^4 - (x^2+x+1)^2 = \LoesungsRaumLang{(2x^2+x+1)(x+1)}$
\end{bbwAufgabenBlock}

\platzFuerBerechnungenBisEndeSeite{}


%%%%%%%%%%%%%%%%%%%%%%%%%%%%%%%%%%%%%%%%%%%%%%%%%5



Faktorisieren Sie (Tipp: Zweiklammeransatz):


\begin{bbwAufgabenBlock}
\item $x^2 + 12x + 20 = \LoesungsRaumLang{(x+10)(x+2)}$
\item $x^2 - 12x + 20 = \LoesungsRaumLang{(x-10)(x-2)}$
\item $x^2 - 8x - 20 = \LoesungsRaumLang{(x-10)(x+2)}$
\item $x^2 + 8x - 20 = \LoesungsRaumLang{(x+10)(x-2)}$
\item $x^2 + x - 20 = \LoesungsRaumLang{(x+5)(x-4)}$
\item $x^2 - x - 2 = \LoesungsRaumLang{(x+1)(x-2)}$
\item $a^2+3a+2 = \LoesungsRaumLang{(a+1)(a+2)}$
\item $x^2 + x + 1 = \LoesungsRaumLang{\textrm{Kann nicht faktorisiert werden}}$
\item $s^2 -2s - 48 = \LoesungsRaumLang{(s-8)(s+6)}$
\item $a^2 + a - 72 = \LoesungsRaumLang{(a-8)(a+9)}$
\item $x^2 - 11xy + 10y^2 = \LoesungsRaumLang{(x-1y)(x-10y)}$
\end{bbwAufgabenBlock}

\platzFuerBerechnungenBisEndeSeite{}


%%%%%%%%%%%%%%%%%%%%%%%%%%%%%%%%%%%%%%%%%%%%%%%%%5

Faktorisieren Sie in so viele Faktoren wie möglich
mit geeigneten Verfahren:


\begin{bbwAufgabenBlock}
\item $a^9-a^5 = \LoesungsRaumLang{a^5(a-1)(a+1)(a^2+1)}$
\item $a^2(x-y) + b^2(y-x) =\LoesungsRaumLang{(a+b)(a-b)(x-y)}$
\item $4x^3-12x^2-40x = \LoesungsRaumLang{4x(x-5)(x+2)}$
\item $18a^2b^2+12a^2bc + 2a^2c^2 = \LoesungsRaumLang{2a^2(3b+c)^2}$
\item $x+a-b^2x-ab^2 = \LoesungsRaumLang{(x+a)(1-b)(1+b)}$
\item $q-4r^2+p^2 + 4qr = $\LoesungsRaumLang{\textrm{Kann nicht faktorisiert werden.}}
\item $-243x^2 + 48x^6 = \LoesungsRaumLang{3x^2(2x-3)(2x+3)(4x^2+9)}$
\item $s^2+4t^2-(4st+m^2) = \LoesungsRaumLang{(s+2t+m)(s+2t-m)}$
\item $b(x-1)^3 + (3b-1)(x-1)^2 + (2b-2)(x-1) = \LoesungsRaumLang{(x-1)(bx-1)(x+1)}$
\end{bbwAufgabenBlock}

\platzFuerBerechnungenBisEndeSeite{}


%%%%%%%%%%%%%%%%%%%%%%%%%%%%%%%%%%%%%%%%%%%%%%%%%5

\end{document}
