%%
%% Übungen zum Bogenmaß mit Sinus Cosinus und Tangens
%%

\input{bbwSeite}

%%%%%%%%%%%%%%%%%%%%%%%%%%%%%%%%%%%%%%%%%%%%%%%%%%%%%%%%%%%%%%%%%%

\renewcommand{\metaHeaderLine}{Arbeitsblatt}
\renewcommand{\arbeitsblattTitel}{Uebungen zum Bogenmaß und den
trigonometrischen Funktionen}

\begin{document}%%
\arbeitsblattHeader{}


\section{Umrechnungen}

\subsection{Geben Sie im Bogenmaß an}

1. $90\degre = \LoesungsRaum{\frac{\pi}2}$

2. $30\degre = \LoesungsRaum{\frac{\pi}6}$

3. $390\degre = \LoesungsRaum{\frac{13\pi}6}$

4. $45\degre = \LoesungsRaum{\frac{\pi}4}$

5. $135\degre = \LoesungsRaum{\frac{3\pi}4}$

6. $-225\degre = \LoesungsRaum{-\frac{5\pi}4}$

Mit Taschenrechner

7. $17\degre \approx \LoesungsRaum{0.297 [\textrm{rad}]}$

8. $-199\degre \approx \LoesungsRaum{-3.473 [\textrm{rad}]}$



\subsection{Geben Sie im Gradmaß an}

1. $\frac{\pi}2 = \LoesungsRaum{90\degre}$

2. $\frac{3\pi}2 = \LoesungsRaum{270\degre}$

3. $-\frac{2\pi}3 = \LoesungsRaum{-120\degre}$

4. $\frac76\cdot{}\pi = \LoesungsRaum{210\degre}$

Mit Taschenrechner:

5. $5.3 \approx \LoesungsRaum{303.67\degre}$

6. $-2.8 \approx \LoesungsRaum{-160.43}$

\platzFuerBerechnungen{6.4}
\newpage


\subsection{Winkelberechnungen}

Berechnen Sie jeweils im Grad \textbf{und} im Bogenmaß und geben Sie
die Lösung im Intervall $[0\degre; 360\degre)$ (= $[0; 2\pi)$):

1. $\sin(\varphi) = 0.5$ $\Longrightarrow$
$\mathbb{L}_\varphi = \LoesungsRaum{30\degre, 150\degre}$ bzw. $= \LoesungsRaum{ \frac16\cdot{}\pi{}, \frac56\cdot{}\pi}$

2. $\sin(\varphi) = -\frac{\sqrt{2}}{2}$ $\Longrightarrow$
$\mathbb{L}_\varphi = \LoesungsRaum{225\degre, 315\degre}$ bzw. $= \LoesungsRaum{ \frac54\cdot{}\pi{}, \frac74\cdot{}\pi}$


3. $\cos(\varphi) = -\frac{\sqrt{3}}{2}$ $\Longrightarrow$
$\mathbb{L}_\varphi = \LoesungsRaum{60\degre, 300\degre}$ bzw. $= \LoesungsRaum{ \frac13\cdot{}\pi{}, \frac53\cdot{}\pi}$

4. $\tan(\varphi) = 1$ $\Longrightarrow$
$\mathbb{L}_\varphi = \LoesungsRaum{45\degre, 135\degre}$ bzw. $= \LoesungsRaum{ \frac14\cdot{}\pi{}, \frac54\cdot{}\pi}$

\subsection{Verhältnisse ohne Taschenrechner}

1. $\sin(\frac{\pi}3) = \LoesungsRaum{\frac{\sqrt{3}}2}$

2. $\sin(-\frac{\pi}4) = \LoesungsRaum{\frac{-\sqrt{2}}2}$

3. $\cos(-\frac{\pi}2) = \LoesungsRaum{0}$

4. $\tan(\frac{\pi}6) = \LoesungsRaum{\frac{\sqrt{3}}3}$

\platzFuerBerechnungen{8.4}

\end{document}
  
