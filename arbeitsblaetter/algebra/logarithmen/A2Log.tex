\input{bmsLayoutPage}
\renewcommand{\bbwAufgabenBlockID}{A2Log}

\renewcommand{\metaHeaderLine}{Logarithmen}
\renewcommand{\arbeitsblattTitel}{Algebra: Logarithmen}

\begin{document}
\arbeitsblattHeader{}

\section{Zehnerlogarithmen}

\textbf{\bbwAufgabenNummer{}.}
Berechnen Sie von Hand den Wert des Termes und prüfen Sie mit dem Taschenrechner:

\begin{bbwAufgabenBlock}
\item $\lg(1000) = \LoesungsRaumLang{3}$\abplz{2}
\item $\lg(10) = \LoesungsRaumLang{1}$\abplz{2}
\item $\lg(1) = \LoesungsRaumLang{0}$\abplz{2}
\item $\lg(0.1) = \LoesungsRaumLang{-1}$\abplz{2}
\item $\lg\left(\frac1{10^5}\right) = \LoesungsRaumLang{-5}$\abplz{2}
\item $\lg\left(10^{-6}\right) = \LoesungsRaumLang{-6}$\abplz{2}
\item $\lg\left(0.0001\right) = \LoesungsRaumLang{-4}$\abplz{2}
\item $\lg\left(\sqrt{10}\right) = \LoesungsRaumLang{0.5}$\abplz{2}
\item $\lg\left(\sqrt[5]{10^3}\right) = \LoesungsRaumLang{0.6}$\abplz{2}
\end{bbwAufgabenBlock}

%\platzFuerBerechnungenBisEndeSeite{}
\newpage

%%%%%%%%%%%%%%%%%%%%%%%%%%%%%%%%%%%%%%%%%%%%%%%%%5

\textbf{\bbwAufgabenNummer{}.}
Schreiben Sie die Gleichung in der $\lg$-Form und berechnen Sie $x$:

Beispiel:

$$10^x=\sqrt[5]{100} \Longrightarrow   x=\lg\left(\sqrt[5]{100}\right)
= \lg\left(10^{\frac25}\right) = \frac25$$

\begin{bbwAufgabenBlock}
\item $10^x = 10\,000$ $\LoesungsRaum{x=4}$\abplz{4}
\item $10^x = \sqrt[4]{10}$ $\LoesungsRaum{x=\frac14}$\abplz{4}
\item $10^x = \sqrt[7]{\frac1{10^3}}$ $\LoesungsRaum{x=\frac{-3}{7}}$\abplz{4}
\item $10^x = \frac{1}{\sqrt{10}}$ $\LoesungsRaum{x=\frac{-1}2}$\abplz{4}
\item $100^x = 10^5$ $\LoesungsRaum{x=2.5}$\abplz{4}
\end{bbwAufgabenBlock}

%%\platzFuerBerechnungenBisEndeSeite{}
\newpage

\textbf{\bbwAufgabenNummer{}.}
Schätzen Sie die folgenden Logarithmen und prüfen Sie mit dem
Taschenrechner. Zwischen welchen beiden ganzen Zahlen liegt der Logarithmus?


\begin{bbwAufgabenBlock}
\item $\lg(1001) \approx  \LoesungsRaum{3.0004}$\abplz{3.2}
\item $\lg(5) \approx  \LoesungsRaum{0.69897}$\abplz{3.2}
\item $\lg(0.02) \approx  \LoesungsRaum{-1.69897}$\abplz{3.2}
\item $\lg(10\cdot{}\pi) \approx  \LoesungsRaum{1.497}$\abplz{3.2}
\item $\lg\left(\left(10\right)^{10}\right)^{10} \approx  \LoesungsRaum{100 \text{ exakt }}$\abplz{3.2}

\end{bbwAufgabenBlock}
%%\platzFuerBerechnungenBisEndeSeite{}
\newpage

\textbf{\bbwAufgabenNummer{}.}
Vereinfachen Sie mit Hilfe des Gesetzes $10^{\lg(n)} = n$.


\begin{bbwAufgabenBlock}
\item $10^{\lg(5)} =  \LoesungsRaum{5}$\abplz{2}
\item $10^{1+\lg(4)} =  \LoesungsRaum{40}$\abplz{4}
\item $10^{3-\lg(0.4)} =  \LoesungsRaum{2\,500}$\abplz{4}
\item $10^{500-\lg(8^{500})} = \LoesungsRaum{1.25^{500} \approx 2.851\cdot{}10^{48}}$\abplz{6}
\end{bbwAufgabenBlock}


%%\platzFuerBerechnungenBisEndeSeite{}
\newpage

\textbf{\bbwAufgabenNummer{}.}
Bestimmen Sie $x$, indem Sie die logarithmische Gleichung erst in die
Potenz-Form bringen:
$$\lg(x) = p \Longleftrightarrow 10^p = x$$

Prüfen Sie Ihr Resultat, indem Sie das gefundene $x$ wieder in die
ursprüngliche Gleichung einsetzen.

\begin{bbwAufgabenBlock}
\item $\lg(x+1)= 3  \LoesungsRaum{x=999}$\abplz{4}
\item $\lg(\frac12 x - 14)= 2  \LoesungsRaum{x=228}$\abplz{4}
\item $\lg(2x+3)= 6  \LoesungsRaum{x=499\,998.5}$\abplz{4}
\item $\lg\left(\sqrt{x}\right)= 2  \LoesungsRaum{x=10\,000}$\abplz{4}
\item $\lg\left(\frac2{x+3}\right)=
1  \LoesungsRaum{x=\frac{-14}5}$\abplz{4}

challenge:
\item $\lg\left( x^2-2x+1.1 \right)= -1  \LoesungsRaum{x=1}$\abplz{4}

\end{bbwAufgabenBlock}

%%\platzFuerBerechnungenBisEndeSeite{}
\newpage
\section{andere Basen}

\textbf{\bbwAufgabenNummer{}.}
Schreiben Sie in der Potenzschreibweise und berechnen Sie $x$ von
Hand.
Prüfen Sie Ihr Resultat mit dem Taschenrechner.


\begin{bbwAufgabenBlock}
\item $x = \log_{10}(100)  \LoesungsRaum{x=2}$\abplz{2}
\item $x = \log_3(9)  \LoesungsRaum{x=2}$\abplz{2}
\item $x = \log_2(8)  \LoesungsRaum{x=3}$\abplz{2}
\item $x = \log_{25}(5)  \LoesungsRaum{x=0.5}$\abplz{2}
\item $x = \log_3(81)  \LoesungsRaum{x=4}$\abplz{2}
\item $x = \log_2\left(\frac1{16}\right)  \LoesungsRaum{x=-4}$\abplz{2}
\end{bbwAufgabenBlock}

%%\platzFuerBerechnungenBisEndeSeite{}
\newpage

\textbf{\bbwAufgabenNummer{}.}
Schreiben Sie in der logarithmischen Form und berechnen Sie die
Unbekannte anschließend von Hand oder mit dem Taschenrechner.

\begin{bbwAufgabenBlock}
\item $2^x = 10$   \LoesungsRaum{$x=\log_2(10) \approx 3.322$}\abplz{2}
\item $2^y = \frac{1}{64}$   \LoesungsRaum{$x=-6$}\abplz{3.2}
\item $2^z = \sqrt{2^6}$   \LoesungsRaum{$x=3$}\abplz{2}
\item $4^t= 1$   \LoesungsRaum{$x=0$}\abplz{2}
\end{bbwAufgabenBlock}



%%\platzFuerBerechnungenBisEndeSeite{}
\newpage

\textbf{\bbwAufgabenNummer{}.}
Bestimmen Sie die folgenden Logarithmen von Hand.

\begin{bbwAufgabenBlock}
\item $\log_4(16)     =    \LoesungsRaum{2}$\abplz{2}
\item $\log_{17}(1)     =    \LoesungsRaum{0}$\abplz{2}
\item $\log_4(2)     =    \LoesungsRaum{0.5}$\abplz{2}
\item $\log_{25}(\sqrt{5})     =    \LoesungsRaum{0.25}$\abplz{2}
\item $\log_{27}(9)     =    \LoesungsRaum{\frac23}$\abplz{2}
\end{bbwAufgabenBlock}

%%\platzFuerBerechnungenBisEndeSeite{}
\newpage

\textbf{\bbwAufgabenNummer{}.}
Bestimmen Sie die Basis $x$:

\begin{bbwAufgabenBlock}
\item $\log_x(8) = 3 \Longrightarrow x =    \LoesungsRaum{2}$\abplz{2}
\item $\log_x(0.125) = -3 \Longrightarrow x =    \LoesungsRaum{2}$\abplz{2}
\item $\log_x(2) = \frac12 \Longrightarrow x =    \LoesungsRaum{4}$\abplz{2}
\item $\log_x\left(\frac17\right) = -1 \Longrightarrow x =    \LoesungsRaum{7}$\abplz{2}
\item $\log_x\left(\frac1{5^4}\right) = \frac15 \Longrightarrow x =    \LoesungsRaum{4}$\abplz{2}

\end{bbwAufgabenBlock}


%%\platzFuerBerechnungenBisEndeSeite{}
\newpage

\textbf{\bbwAufgabenNummer{}.}
Bestimmen Sie den Numerus $x$:

\begin{bbwAufgabenBlock}
\item $\log_{10}(x) = 3 \Longrightarrow x =    \LoesungsRaum{1000}$\abplz{2}
\item $\log_{2}(x) = 5 \Longrightarrow x =    \LoesungsRaum{32}$\abplz{2}
\item $\log_{0.5}(x) = 16 \Longrightarrow x =    \LoesungsRaum{\frac1{2^{16}}\approx 15.3\cdot{}10^{-6}}$\abplz{2}
\item $\log_{3}(x) = -4 \Longrightarrow x =    \LoesungsRaum{\frac1{81}}$\abplz{2}
\end{bbwAufgabenBlock}



%%\platzFuerBerechnungenBisEndeSeite{}
\newpage

\textbf{\bbwAufgabenNummer{}.}
Berechnen Sie jeweils $x$:

\begin{bbwAufgabenBlock}
\item $x=\log_p(p) \Longrightarrow x =    \LoesungsRaum{1}$\abplz{1.6}
\item $x=\log_c(1) \Longrightarrow x =    \LoesungsRaum{0}$\abplz{1.6}
\item $x=\log_r(r^3) \Longrightarrow x =    \LoesungsRaum{3}$\abplz{1.6}
\item $x=\log_5\left(\sqrt[4]{5}\right) \Longrightarrow x =    \LoesungsRaum{0.25}$\abplz{3.2}
\item $x=\log_t\left(t^2\cdot{}\sqrt[3]{t}\right) \Longrightarrow x =    \LoesungsRaum{\frac73}$\abplz{3.2}
\item $x=\log_a\left( \frac1{a^{-2}\cdot{} \sqrt{a}} \right) \Longrightarrow x =    \LoesungsRaum{1.5}$\abplz{4}
\end{bbwAufgabenBlock}


%%\platzFuerBerechnungenBisEndeSeite{}
\newpage

\textbf{\bbwAufgabenNummer{}.}
Berechnen Sie jeweils $x$ mit dem Taschenrechner. Schreiben Sie die
Gleichung wenn nötig vorab in der Potenzschreibweise:

$$x=\log_b(p) \Longleftrightarrow b^x=p$$

\begin{bbwAufgabenBlock}
\item $x=log_2(5) \Longrightarrow x \approx    \LoesungsRaum{2.322}$\abplz{2}
\item $11.3 = \log_6(x) \Longrightarrow x \approx    \LoesungsRaum{621 \text{ Mio.}}$\abplz{2}
\item $12 = log_x(2) \Longrightarrow x \approx    \LoesungsRaum{1.059}$\abplz{2}
\item $x = log_{16}(15)  \Longrightarrow x \approx    \LoesungsRaum{0.9767}$\abplz{2}
\item $10 = \log_{1.5}(x)  \Longrightarrow x \approx    \LoesungsRaum{57.67}$\abplz{2}
\item $\frac34 = log_{x}\left(\frac52\right)  \Longrightarrow x \approx    \LoesungsRaum{3.393}$\abplz{2}\\

\noTRAINER{\newpage}

challenge:

\item $70=20 + 25 \cdot{} 1.43^{\frac{x}3}  \Longrightarrow x \approx    \LoesungsRaum{5.814}$\abplz{2}

\end{bbwAufgabenBlock}

%%\platzFuerBerechnungenBisEndeSeite{}
\newpage

\section{Logarithmus Naturalis}

$$\ln = \log_e$$
$$e\approx 2.71828 \text{ (= Eulersche Zahl)}$$



\textbf{\bbwAufgabenNummer{}.}
Berechnen Sie $x$ von Hand:

\begin{bbwAufgabenBlock}
\item $\e^x = \frac1{\sqrt[4]{\e^7}}  \Longrightarrow x =    \LoesungsRaum{\frac{-7}4}$\abplz{2}
\item $x = \ln\left(\e^3\right)  \Longrightarrow x =    \LoesungsRaum{3}$\abplz{2}
\item $\ln\left(\sqrt{x}\right)  = \frac12  \Longrightarrow x =    \LoesungsRaum{\e}$\abplz{4}
\item $\ln\left(x\cdot{} \e \right)  = -2  \Longrightarrow x =    \LoesungsRaum{\e^{-3}}$\abplz{4}
\noTRAINER{\newpage}
\item $\ln(x+1) = 2 \Longrightarrow x =    \LoesungsRaum{\e^2 - 1}$\abplz{3.2}
\item $x=\frac{\e^{\ln(2)+1}}{\e} \Longrightarrow x =    \LoesungsRaum{2}$\abplz{2}
\item $3 = \e^{\frac{x}4} \Longrightarrow x
=    \LoesungsRaum{4\cdot{} \ln(3) (\approx 4.394)}$\abplz{3.2}

\end{bbwAufgabenBlock}

%%\platzFuerBerechnungenBisEndeSeite{}
\newpage

\section{Anwenden der Logarithmengesetze}
\TRAINER{nur TALS}

\textbf{\bbwAufgabenNummer{}.}
Zerlegen Sie mit Hilfe der Logarithmengesetze so weit wie möglich.

Beispiel $$\log\left(a\cdot{}b^3\right) = \log(a) + 3\cdot{}\log(b)$$

\begin{bbwAufgabenBlock}
\item $\log_x(ab)  \LoesungsRaum{\log_x(a) + \log_x(b)}$\abplz{2}
\item $\log_x(3xy)  \LoesungsRaum{\log_x(3) + 1 + \log_x(y)}$\abplz{3.2}
\item $\ln(12u+4uv)  \LoesungsRaum{\ln(4) + \ln(u) + \ln(3+v)}$\abplz{3.2}
\item $\log_a(16p^8-p^4) \LoesungsRaum{4\cdot{}\log_a(p)+ \log_a(2p+1) + \log_a(2p-1)+ \log_a(4p^2+1)}$\abplz{6}
\noTRAINER{\newpage}

\item $\lg\left(\frac{c}{10}\right)  \LoesungsRaum{\lg(c) - 1}$\abplz{2}
\item $\log_a\left( m^2n^3 \right)  \LoesungsRaum{2\cdot{}\log_a(m) + 3\cdot{}\log_a(n)}$\abplz{2}
\item $\log_a\left( \frac{p^3}{a^2 \cdot{} \sqrt[3]{p^4}} \right)  \LoesungsRaum{\left(3-\frac43\right)\cdot{}\log_a(p)- 2}$\abplz{6}
\item
$\lg\left(\left(\frac{a^3}{b^4}\right)^{10}\right)  \LoesungsRaum{10\cdot{}\left(3 \cdot{}\lg(a) - 4\cdot{} \lg(b)\right)}$\abplz{6}
\noTRAINER{\newpage}

\item $\log \left( \sqrt[5]{a^2-b^2}\right)  \LoesungsRaum{\frac15\left(\log(a+b) + \log(a-b)\right)}$\abplz{6}

\end{bbwAufgabenBlock}

%%\platzFuerBerechnungenBisEndeSeite{}
\newpage


\textbf{\bbwAufgabenNummer{}.}
Fassen Sie zu einem einzigen Logarithmus zusammen und vereinfachen Sie
so weit wie möglich:

\begin{bbwAufgabenBlock}
\item $\lg(b^2) - \lg(b) - \lg(c) \LoesungsRaum{\lg\left(\frac{b}c\right)}$\abplz{3.2}
\item $\log(x^2-y^2) - \log(x+y) \LoesungsRaum{\log(x-y)}$\abplz{4}
\item $1+\lg(a)+2\lg(b)  \LoesungsRaum{\lg\left( 10ab^2 \right)}$\abplz{4}
\item $\ln\left(\e^2-1\right) + \frac16 - \ln(\e-1) \LoesungsRaum{\ln\left((e+1)\cdot{}\sqrt[6]{e}\right)}$\abplz{6}
\item $\log(2n-8) - \log\left(n^2+n-20\right) \LoesungsRaum{\log\left(\frac{2}{n+5}\right)}$\abplz{4}

\end{bbwAufgabenBlock}



\platzFuerBerechnungenBisEndeSeite{}%
\end{document}%
