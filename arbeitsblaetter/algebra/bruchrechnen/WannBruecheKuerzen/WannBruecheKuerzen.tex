%%
%% Meta: TI nSpire Einführung
%%       Ziel: Damit die Grundoperationen damit durchgeführt werden können.
%%             Damit man sich an den Rechner gewöhnt.
%%

\input{bmsLayoutPage}
\renewcommand{\bbwAufgabenBlockID}{A1wBk}%% Wann Brüche kürzen
%%%%%%%%%%%%%%%%%%%%%%%%%%%%%%%%%%%%%%%%%%%%%%%%%%%%%%%%%%%%%%%%%%

\usepackage{amssymb} %% für \blacktriangleright
\renewcommand{\metaHeaderLine}{Arbeitsblatt}
\renewcommand{\arbeitsblattTitel}{Wann Brüche kürzen?}

\begin{document}%%
\arbeitsblattHeader{}
\section{Grundrezept}

Brüche dürfen erst gekürzt werden, wenn sowohl der Zähler, wie auch
der Nenner je ein Produkt sind.

Geben Sie zunächst bei jedem Bruch an, um welche Termart (Summe,
Differenz, Produtk,...) es sich beim Zähler und beim Nenner handelt.

$a+3b+4$ = Summe

$4(a+2b)$ = Produkt

\begin{rezept}{Kürzen}{}
\begin{itemize}
\item Sind Zähler und Nenner identisch, so ist das Resultat $=1$.
\item Sind Zähler und Nenner je ein Produkt, dann dürfen jeweils die
selben Faktoren gekürzt werden.
\item Ist mindestens Zähler oder Nenner \textbf{kein Produkt}, dann
darf \textbf{nicht} gekürzt werden!
\end{itemize}
\end{rezept}

Geben Sie vom Zähler und vom Nenner den Termtyp (Summe, Differenz,
Produkt, ...) an und kürzen Sie, falls möglich. Folgen Sie den ersten
zwei Beispielen (Aufgabe 1a und 1b):

\begin{bbwAufgabenBlock}
\item  $\frac{7+x}{8+x} = \frac{\textrm{Summe}}{\textrm{Summe}} \mapsto $, nicht kürzbar

\item  $\frac{24\cdot{}(7+x)}{(8+x)\cdot{}9}
= \frac{\textrm{Produkt}}{\textrm{Produkt}} \mapsto $, theoretisch kürzbar, hier jedoch noch keine identischen Faktoren.

\item $\frac{7\cdot{}x}{7+x} = \frac{\LoesungsRaumLang{\textrm{Produkt}}}{\LoesungsRaum{\textrm{Summe}}} \mapsto $\LoesungsRaum{nicht kürzbar}

\item $\frac{a\cdot{}(-1)\sqrt{x+3}}{(x+3)\cdot{}(-1)\cdot{}b}
= \frac{\LoesungsRaumLang{\textrm{Produkt}}}{\LoesungsRaum{\textrm{Produkt}}} \mapsto
$\LoesungsRaum{identische Faktoren ($(-1)$) kürzen} \TRAINER{$\frac{a\sqrt{x+3}}{b(x+3)}$}

\item $\frac{2x-3b}{3b(2x)} = \frac{\LoesungsRaumLang{\textrm{Differenz}}}{\LoesungsRaum{\textrm{Produkt}}} \mapsto $\LoesungsRaum{nicht kürzbar}

\item $\frac{7\cdot{}(6x+3)}{a\cdot{}(6x+4)}
= \frac{\LoesungsRaumLang{\textrm{Produkt}}}{\LoesungsRaum{\textrm{Produkt}}} \mapsto
$\LoesungsRaum{leider keine identischen Faktoren} \TRAINER{$\mapsto$
nicht kürzbar}

\item $\frac{55axy+7}{7+55\cdot{}xy\cdot{}a}$ =
$\frac{\LoesungsRaumLang{\textrm{Summe}}}{\LoesungsRaum{\textrm{Summe}}}$  $\mapsto$
\LoesungsRaumLang{theoretisch nicht kürzbar, jedoch sind hier Zähler
und Nenner identisch. Lösung = 1.}

\item $\frac{7\cdot{}a(x-2)}{9\cdot{}(x-2)\cdot{}b} = \frac{\LoesungsRaumLang{\textrm{Produkt}}}{\LoesungsRaum{\textrm{Produkt}}} \mapsto $\LoesungsRaum{identische Faktoren ($x-2$) kürzen} \TRAINER{$\frac{7a}{9b}$}

\end{bbwAufgabenBlock}

\platzFuerBerechnungenBisEndeSeite{}
\TRAINER{\newpage}

Weitere Übungsaufgaben


\begin{bbwAufgabenBlock}




\item  $\frac{3\cdot{}(x-2)}{3x-2}
= \frac{\LoesungsRaumLang{\textrm{Produkt}}}{\LoesungsRaum{\textrm{Differenz}}} \mapsto
$ \LoesungsRaum{nicht kürzbar}

\item  $\frac{x^2a-c}{x^2\cdot{}(a-c)} = \frac{\LoesungsRaumLang{\textrm{Differenz}}}{\LoesungsRaum{\textrm{Produkt}}} \mapsto $ \LoesungsRaum{nicht kürzbar}

\item  $\frac{(3+x)\cdot{}a}{3x+ax} = \frac{\LoesungsRaumLang{\textrm{Produkt}}}{\LoesungsRaum{\textrm{Summe}}} \mapsto $ \LoesungsRaum{nicht kürzbar}

\item $\frac{4x\cdot{}(1\cdot{}x-1)}{(x+(-1))\cdot{}ax} = \frac{\LoesungsRaumLang{\textrm{Produkt}}}{\LoesungsRaum{\textrm{Produkt}}} \mapsto $\LoesungsRaum{identische Faktoren ($x-1$) kürzen} \TRAINER{$\frac{4}{a}$}

\item  $\frac{b-y}{b+y} = \frac{\LoesungsRaumLang{\textrm{Differenz}}}{\LoesungsRaum{\textrm{Summe}}} \mapsto $ \LoesungsRaum{nicht kürzbar}

\item  $\frac{2a}{3b}\cdot{}\frac{3b+x}{5c}
= \frac{\LoesungsRaumLang{\textrm{Produkt}}}{\LoesungsRaum{\textrm{Produkt}}} \cdot{} \frac{\LoesungsRaumLang{\textrm{\textbf{Summe}}}}{\LoesungsRaum{\textrm{Produkt}}}  \mapsto $ \LoesungsRaum{nicht kürzbar}

\item  $\frac{(b+2)^2}{2(b+2)}$ $=$ $\frac{\LoesungsRaumLang{\textrm{Potenz}}}{\LoesungsRaum{\textrm{Produkt}}}$
$\mapsto$ \LoesungsRaum{Produkt erst ausschreiben, dann kürzbar :} \LoesungsRaumLang{ $(b+2)^2 = (b+2)\cdot{}(b+2)$}


\item  $\frac{5-2ab}{-ab-ab+5}$ $=$ $\frac{\LoesungsRaumLang{\textrm{Differenz}}}{\LoesungsRaum{\textrm{Summer}}}$
$\mapsto$ \LoesungsRaum{Zwar keine Produkte, aber identische
Ausdrücke; somit ist das Resultat = 1}

\end{bbwAufgabenBlock}


\platzFuerBerechnungenBisEndeSeite{}
\TRAINER{\newpage}

Kürzen Sie direkt, falls möglich:

\begin{bbwAufgabenBlock}
\item $\frac{3x(x-1)}{2\cdot{}y(x-1)\cdot{}3} = \LoesungsRaumLang{\frac{x}{2y}}$

\item $\frac{a^2\cdot{}bx}{ab^2x^2} = \LoesungsRaumLang{\frac{a}{bx}}$

\item $\frac{(a-1)(a+1)}{(a-1)\cdot{}b\cdot{}(a+1)} = \LoesungsRaumLang{\frac{1}{b}}$

\item $\frac{(-1)\cdot{}(-1)^2}{(0-1)\cdot{}(1-0)^2} = \LoesungsRaumLang{1}$

\item $\frac{(-1)\cdot{}(a-b^2)}{(a^2-b)\cdot{}(1-0)} = \LoesungsRaumLang{\frac{b^2-a}{a^2-b}}$

\item $\frac{x-a}{a-x} = \LoesungsRaumLang{-1}$

\item $\frac{x-2+b}{b+x-2} = \LoesungsRaumLang{1}$

\item $\frac{2+x+c}{b+x+2} = \LoesungsRaumLang{\textrm{ kann nicht
gekürzt werden! }}$

\item $\frac{2a}{3b} \cdot \frac{3b+7c}{2a} = \LoesungsRaumLang{\frac{3b+7c}{3b}}$

\end{bbwAufgabenBlock}

\platzFuerBerechnungenBisEndeSeite{}

\end{document}
