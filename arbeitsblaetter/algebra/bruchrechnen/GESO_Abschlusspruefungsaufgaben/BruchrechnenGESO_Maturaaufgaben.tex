%%
%% Meta: TI nSpire Einführung
%%       Ziel: Damit die Grundoperationen damit durchgeführt werden können.
%%             Damit man sich an den Rechner gewöhnt.
%%

\input{bbwLayoutPage}


%%%%%%%%%%%%%%%%%%%%%%%%%%%%%%%%%%%%%%%%%%%%%%%%%%%%%%%%%%%%%%%%%%

\usepackage{amssymb} %% für \blacktriangleright
\renewcommand{\metaHeaderLine}{Arbeitsblatt}
\renewcommand{\arbeitsblattTitel}{Bruchrechnen (alte GESO Maturaaufgaben)}

\begin{document}%%
\arbeitsblattHeader{}
Vereinfachen Sie die folgenden Ausdrücke.

%%%%%%%%%%%%%%%%%%%%%%%%%%%%%%%%%%%%%%%%%%%%%%%%%%%%%%%%%%%%%%%%%%%%%%%%
$1.\ (2018)$

$$\left(1-\frac{x}{2y}\right) : \frac{4y^2 - x^2}{4y^2} $$
\TNT{4.8}{$ = \frac{2y}{2y+x}$}


%%%%%%%%%%%%%%%%%%%%%%%%%%%%%%%%%%%%%%%%%%%%%%%%%%%%%%%%%%%%%%%%%%%%%%%%
$2.\ (2018)$

$$\left(\frac{1}{b} - \frac{1}{a}\right) : \left(\frac{b-a}{a}\right)$$
\TNT{4.8}{$ = \frac{-1}{b}$}

%%%%%%%%%%%%%%%%%%%%%%%%%%%%%%%%%%%%%%%%%%%%%%%%%%%%%%%%%%%%%%%%%%%%%%%%
$3.\ (2018)$

$$\frac{6-3a}{b} : \frac{6a-12}{-a}$$
\TNT{4.8}{$= \frac{a}{2b}$}
\newpage

%%%%%%%%%%%%%%%%%%%%%%%%%%%%%%%%%%%%%%%%%%%%%%%%%%%%%%%%%%%%%%%%%%%%%%%%
$4.\ (2018)$

$$\left(\frac{y}{x} - \frac{x}{y}\right) : \left(\frac{1}{x}
- \frac{1}{y}\right)$$
\TNT{4.8}{$= x+y$}


%%%%%%%%%%%%%%%%%%%%%%%%%%%%%%%%%%%%%%%%%%%%%%%%%%%%%%%%%%%%%%%%%%%%%%%%
$5.\ (2017)$

$$\frac{1}{5}\left(a-\frac{b^2}{a}\right) : \frac{a+b}{a}$$
\TNT{4.8}{$ = \frac{1}{5}(a-b)$}

%%%%%%%%%%%%%%%%%%%%%%%%%%%%%%%%%%%%%%%%%%%%%%%%%%%%%%%%%%%%%%%%%%%%%%%%
$6.\ (2016)$

$$\frac{1.5x}{12-3x} - \frac{2.5x}{x-4}$$
\TNT{4.8}{$ = \frac{x+5}{2(4-x)}$}

\end{document}
