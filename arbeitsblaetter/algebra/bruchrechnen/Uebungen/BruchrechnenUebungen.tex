%%
%% Meta: TI nSpire Einführung
%%       Ziel: Damit die Grundoperationen damit durchgeführt werden können.
%%             Damit man sich an den Rechner gewöhnt.
%%

\input{bmsLayoutPage}

\newcommand{\LoesungsBlock}[1]{\noTRAINER{\TNTeop{}}\TRAINER{Lösung:#1
\vspace{1mm}
\hrule}}
%%%%%%%%%%%%%%%%%%%%%%%%%%%%%%%%%%%%%%%%%%%%%%%%%%%%%%%%%%%%%%%%%%

\usepackage{amssymb} %% für \blacktriangleright
\renewcommand{\metaHeaderLine}{Arbeitsblatt Bruchrechnen}
\renewcommand{\arbeitsblattTitel}{Zusammenfassung aller Übungen}

\newcommand{\TNTeopS}[1]{\TRAINER{#1}\noTRAINER{\TNTeop{}}}

\begin{document}%%
\arbeitsblattHeader{}

\tableofcontents{}

\newpage


\textbf{Vorgehen}


\begin{enumerate}
\item 
Lösen Sie ca. 10 Aufgaben aus den alten Aufnahmeprüfungen (Kapite I)
Wenn Sie davon gute 50 % lösen können, so gehen Sie zu Kapitel III oder IV, den
alten Abschlussprüfungen (III a) : GESO; III b): TALS)

\item  Haben Sie weniger als ca 50% gelöst, werden die Trainingsaufgaben Kap. II empfohle
Lösen Sie pro Aufgabennummer mindestens je die ersten zwei und die letzte Aufgabe
Wenn Sie mehr Training benötigen, so hat es genügend Übungsmaterial
in den weiteren Aufgabennummern.
\end{enumerate}

\newpage

\part{Aus alten Aufnahmeprüfungen}
\section*{Aufnahmeprüfung 2023}
Aufnahmeprüfung 2023 Serie e

2. b) Vereinfachen Sie so weit wie möglich:

$$\frac{(x+2)(x-4)}{5} : \frac{x^2 - 16}{10}$$
\LoesungsBlock{$\frac{2\cdot{}(x+2)}{x+4}$}
%%%%%%%%%%%%%%%%%%%%%%%%%%%%%%%%%%%%%%%%%%%%%%%%%%%%%%%%%%%%%%%%%%%%%%

Aufnahmeprüfung 2023 Serie d

2. a) Vereinfachen Sie so weit wie möglich:

$$\frac{x^2-9}{x+3} : \frac{x-3}{4}$$

\LoesungsBlock{$4$}
%%%%%%%%%%%%%%%%%%%%%%%%%%%%%%%%%%%%%%%%%%%%%%%%%%%%%%%%%%%%%%%%%%%%%%%%%%%
Beispielprüfung 2023

3. b) Vereinfachen Sie so weit wie möglich:

$$\frac{x^2+2xy+y^2}{x^2-16} : \frac{x+y}{3x-12}$$

\LoesungsBlock{$\frac{3(x-4)}{x-y}$}
%%%%%%%%%%%%%%%%%%%%%%%%%%%%%%%%%%%%%%%%%%%%%%%%%%%%%%%%%%%%%%%%%%%%%%%%%%%
\section*{Aufnahmeprüfung 2022}
Aufnahmeprüfung 2022 Serie B

2. a) Vereinfachen Sie so weit wie möglich:

$$\frac14 \left(4-\frac{x}2 \right) - \left( \frac{3x}{8} - \frac{3x}{2}\right)$$

\LoesungsBlock{$1+x$}
%%%%%%%%%%%%%%%%%%%%%%%%%%%%%%%%%%%%%%%%%%%%%%%%%%%%%%%%%%%%%%%%%%%%%%%%%%%
Aufnahmeprüfung 2022 Serie B

2. b) Vereinfachen Sie so weit wie möglich:

$$\frac{2a+10}{a^2+10a+25}$$

\LoesungsBlock{$\frac{2}{a+5}$}
%%%%%%%%%%%%%%%%%%%%%%%%%%%%%%%%%%%%%%%%%%%%%%%%%%%%%%%%%%%%%%%%%%%%%%%%%%%
Aufnahmeprüfung 2022 Serie A

1. c) Vereinfachen Sie so weit wie möglich:

$$\frac{\sqrt{130x^2-(7x)^2}}{5x} + \frac{6x}{\sqrt{25x^2}}$$

\LoesungsBlock{$3$}
%%%%%%%%%%%%%%%%%%%%%%%%%%%%%%%%%%%%%%%%%%%%%%%%%%%%%%%%%%%%%%%%%%%%%%%%%%%
Aufnahmeprüfung 2022 Serie A

2. a) Vereinfachen Sie so weit wie möglich:

$$\frac{x^2+4x}{x^2+5x+4}$$

\LoesungsBlock{$\frac{x}{x+1}$}
%%%%%%%%%%%%%%%%%%%%%%%%%%%%%%%%%%%%%%%%%%%%%%%%%%%%%%%%%%%%%%%%%%%%%%%%%%%
Aufnahmeprüfung 2022 Serie A

2. b) Vereinfachen Sie so weit wie möglich:

$$\frac18\left(x+\frac12\right)  - \frac{3x^2}{8} : \frac{12x}{4}$$

\LoesungsBlock{$\frac1{16}$}
%%%%%%%%%%%%%%%%%%%%%%%%%%%%%%%%%%%%%%%%%%%%%%%%%%%%%%%%%%%%%%%%%%%%%%%%%%%
\section*{Aufnahmeprüfung 2021}
Aufnahmeprüfung 2021 Serie B1

3. a) Vereinfachen Sie so weit wie möglich und schreiben Sie als
einzigen Bruch:

$$3-\frac{2x-5}{4}$$

\LoesungsBlock{$\frac{17}{4} - \frac{x}2 = \frac{17-2x}{4}$}
%%%%%%%%%%%%%%%%%%%%%%%%%%%%%%%%%%%%%%%%%%%%%%%%%%%%%%%%%%%%%%%%%%%%%%%%%%%
Aufnahmeprüfung 2021 Serie B1

3. b) Vereinfachen Sie so weit wie möglich.

$$\frac{a}2 + \frac{15a^2c}{7b} : \frac{20ac}{14b}$$

\LoesungsBlock{$2a$}
%%%%%%%%%%%%%%%%%%%%%%%%%%%%%%%%%%%%%%%%%%%%%%%%%%%%%%%%%%%%%%%%%%%%%%%%%%%
Aufnahmeprüfung 2021 Serie B1

3. c) Vereinfachen Sie so weit wie möglich.

$$\frac{3x+4}{x-7} : \frac{5x+10}{x^2-5x-14}$$

\LoesungsBlock{$\frac{3x+4}{5}$}
%%%%%%%%%%%%%%%%%%%%%%%%%%%%%%%%%%%%%%%%%%%%%%%%%%%%%%%%%%%%%%%%%%%%%%%%%%%
Aufnahmeprüfung 2021 Serie A2

3. a) Vereinfachen Sie so weit wie möglich und schreiben Sie als
einzigen Bruch.

$$5-\frac{2x-4}{7}$$

\LoesungsBlock{$\frac{39-2x}{7}$}
%%%%%%%%%%%%%%%%%%%%%%%%%%%%%%%%%%%%%%%%%%%%%%%%%%%%%%%%%%%%%%%%%%%%%%%%%%%
Aufnahmeprüfung 2021 Serie A2

3. b) Vereinfachen Sie so weit wie möglich.

$$16a\cdot{}\frac{b^2}{8} + 9a : \frac{3}{b^2}$$

\LoesungsBlock{$5ab^2$}
%%%%%%%%%%%%%%%%%%%%%%%%%%%%%%%%%%%%%%%%%%%%%%%%%%%%%%%%%%%%%%%%%%%%%%%%%%%
Aufnahmeprüfung 2021 Serie A2

3. c) Vereinfachen Sie so weit wie möglich.

$$\frac{4x-12}{x^2-5x+6} : \frac{3x+1}{x-2}$$

\LoesungsBlock{$\frac4{3x+1}$}
%%%%%%%%%%%%%%%%%%%%%%%%%%%%%%%%%%%%%%%%%%%%%%%%%%%%%%%%%%%%%%%%%%%%%%%%%%%
\section*{Aufnahmeprüfung 2020}
Aufnahmeprüfung 2020 Serie B2

3. a) Vereinfachen Sie so weit wie möglich und schreiben Sie als Bruchterm

$$\frac{5(x-4)}{4} - \frac{x+5}{6}$$

\LoesungsBlock{$\frac{13x-70}{12}$}
%%%%%%%%%%%%%%%%%%%%%%%%%%%%%%%%%%%%%%%%%%%%%%%%%%%%%%%%%%%%%%%%%%%%%%%%%%%
Aufnahmeprüfung 2020 Serie B2

3. b) Vereinfachen Sie so weit wie möglich.

$$\frac{8a^2}{2b} : \frac{a^2}{3b^2} - \frac{b}{5}$$

\LoesungsBlock{$\frac{59b}{5} = 11.8b$}
%%%%%%%%%%%%%%%%%%%%%%%%%%%%%%%%%%%%%%%%%%%%%%%%%%%%%%%%%%%%%%%%%%%%%%%%%%%
Aufnahmeprüfung 2020 Serie B2

3. c) Vereinfachen Sie so weit wie möglich.

$$\frac{x-5}{x^2+6x} \cdot{} \frac{x^2+7x+6}{x^2-25}$$

\LoesungsBlock{$\frac{x+1}{x(x+5)}$}
%%%%%%%%%%%%%%%%%%%%%%%%%%%%%%%%%%%%%%%%%%%%%%%%%%%%%%%%%%%%%%%%%%%%%%%%%%%
Aufnahmeprüfung 2020 Serie B1
(Aufgabe 3a = Aufgabe 3b von Serie B2 aus 2020)

3. b) Vereinfachen Sie so weit wie möglich und schreiben Sie als Bruchterm

$$\frac{3(x-2)}{4} - \frac{x+4}{6}$$

\LoesungsBlock{$\frac{7x-26}{12}$}
%%%%%%%%%%%%%%%%%%%%%%%%%%%%%%%%%%%%%%%%%%%%%%%%%%%%%%%%%%%%%%%%%%%%%%%%%%%
Aufnahmeprüfung 2020 Serie B1

3. c) Vereinfachen Sie so weit wie möglich.

$$\frac{x-4}{x^2+5x} \cdot{} \frac{x^2+6x+5}{x^2-16}$$

\LoesungsBlock{$\frac{x+1}{x(x+4)}$}
%%%%%%%%%%%%%%%%%%%%%%%%%%%%%%%%%%%%%%%%%%%%%%%%%%%%%%%%%%%%%%%%%%%%%%%%%%%
Aufnahmeprüfung 2020 Serie A2

3. a) Vereinfachen Sie so weit wie möglich und schreiben Sie als Bruchterm

$$\frac{7(x-1)}{9} - \frac{x+4}{6}$$

\LoesungsBlock{$\frac{11x-26}{18}$}
%%%%%%%%%%%%%%%%%%%%%%%%%%%%%%%%%%%%%%%%%%%%%%%%%%%%%%%%%%%%%%%%%%%%%%%%%%%
Aufnahmeprüfung 2020 Serie A2

3. b) Vereinfachen Sie so weit wie möglich.

$$\frac{19b}{3} - \frac{2a^2}{4b} : \frac{a^2}{6b^2}$$

\LoesungsBlock{$\frac{10b}{3}$}
%%%%%%%%%%%%%%%%%%%%%%%%%%%%%%%%%%%%%%%%%%%%%%%%%%%%%%%%%%%%%%%%%%%%%%%%%%%
Aufnahmeprüfung 2020 Serie A2

3. b) Vereinfachen Sie so weit wie möglich.

$$\frac{x^2-6x}{x^2+2x+1} \cdot{} \frac{x^2-1}{x-6}$$

\LoesungsBlock{$\frac{x(x-1)}{x+1}$}
%%%%%%%%%%%%%%%%%%%%%%%%%%%%%%%%%%%%%%%%%%%%%%%%%%%%%%%%%%%%%%%%%%%%%%%%%%%
Aufnahmeprüfung 2020 Serie A1

3. a) Vereinfachen Sie so weit wie möglich.

$$\frac{17a}{3} - \frac{2b^2}{4a}  :   \frac{b^2}{6a^2}$$

\LoesungsBlock{$\frac{8a}3$}
%%%%%%%%%%%%%%%%%%%%%%%%%%%%%%%%%%%%%%%%%%%%%%%%%%%%%%%%%%%%%%%%%%%%%%%%%%%
Aufnahmeprüfung 2020 Serie A1

3. b) Vereinfachen Sie so weit wie möglich und schreiben Sie als Bruchterm

$$\frac{5(x-1)}{6} - \frac{x+3}{9}$$

\LoesungsBlock{$\frac{13x-21}{18}$}
%%%%%%%%%%%%%%%%%%%%%%%%%%%%%%%%%%%%%%%%%%%%%%%%%%%%%%%%%%%%%%%%%%%%%%%%%%%
Aufnahmeprüfung 2020 Serie A1

3. c) Vereinfachen Sie so weit wie möglich

$$\frac{x^2-3x}{x^2+2x+1} \cdot{} \frac{x^2-1}{x-3}$$

\LoesungsBlock{$\frac{x(x-1)}{x+1}$}
%%%%%%%%%%%%%%%%%%%%%%%%%%%%%%%%%%%%%%%%%%%%%%%%%%%%%%%%%%%%%%%%%%%%%%%%%%%
\section*{Aufnahmeprüfung 2019}
Aufnahmeprüfung 2019 Serie B2

3. a) Vereinfachen Sie den Term so weit wie möglich

$$\frac{2x}{5} : \frac{4}{15} - \frac{7x}{6} \cdot{} \frac{3}{28}$$

\LoesungsBlock{$\frac{11x}8$}
%%%%%%%%%%%%%%%%%%%%%%%%%%%%%%%%%%%%%%%%%%%%%%%%%%%%%%%%%%%%%%%%%%%%%%%%%%%
Aufnahmeprüfung 2019 Serie B2

3. b) Vereinfachen Sie den Term so weit wie möglich

$$\frac{3(x-y)^2}{x+y} \cdot{} \frac{6xy+6y^2}{x^2-2xy+y^2}$$

\LoesungsBlock{$18y$}
%%%%%%%%%%%%%%%%%%%%%%%%%%%%%%%%%%%%%%%%%%%%%%%%%%%%%%%%%%%%%%%%%%%%%%%%%%%
Aufnahmeprüfung 2019 Serie B1

3. a) Vereinfachen Sie den Term und kürzen Sie die Resultate so weit wie möglich

$$\frac{5x}{2} : \frac{14}{4} - \frac{2x}{21} \cdot{} \frac{7}{6}$$

\LoesungsBlock{$\frac{5x}{9}$}
%%%%%%%%%%%%%%%%%%%%%%%%%%%%%%%%%%%%%%%%%%%%%%%%%%%%%%%%%%%%%%%%%%%%%%%%%%%
Aufnahmeprüfung 2019 Serie B1

3. b) Vereinfachen Sie den Term so weit wie möglich

$$\frac{4xy+4y^2}{x^2-2xy+y^2} \cdot{} \frac{6(x-y)^2}{x+y}$$

\LoesungsBlock{$24y$}
%%%%%%%%%%%%%%%%%%%%%%%%%%%%%%%%%%%%%%%%%%%%%%%%%%%%%%%%%%%%%%%%%%%%%%%%%%%
Aufnahmeprüfung 2019 Serie A2

3. a) Vereinfachen Sie den Term so weit wie möglich

$$\frac{2x}{3} \cdot{} \frac{9}{4} + \frac{3x}{2} : \frac{9}{16}$$

\LoesungsBlock{$\frac{25x}6$}
%%%%%%%%%%%%%%%%%%%%%%%%%%%%%%%%%%%%%%%%%%%%%%%%%%%%%%%%%%%%%%%%%%%%%%%%%%%
Aufnahmeprüfung 2019 Serie A2

3. b) Vereinfachen Sie den Term so weit wie möglich

$$\frac{4x^2-4xy}{x^2+2xy+y^2} \cdot{} \frac{5(x+y)^2}{x-y}$$

\LoesungsBlock{$20x$}
%%%%%%%%%%%%%%%%%%%%%%%%%%%%%%%%%%%%%%%%%%%%%%%%%%%%%%%%%%%%%%%%%%%%%%%%%%%
Aufnahmeprüfung 2019 Serie A1

3. a) Vereinfachen Sie den Term so weit wie möglich

$$\frac{3x}2 \cdot{} \frac49 + \frac{2x}3 : \frac89$$

\LoesungsBlock{$\frac{17x}{12}$}
%%%%%%%%%%%%%%%%%%%%%%%%%%%%%%%%%%%%%%%%%%%%%%%%%%%%%%%%%%%%%%%%%%%%%%%%%%%
Aufnahmeprüfung 2019 Serie A1

3. b) Vereinfachen Sie den Term so weit wie möglich

$$\frac{3(x+y)^2}{x-y} \cdot{} \frac{5x^2-5xy}{x^2+2xy+y^2}$$

\LoesungsBlock{$15x$}
%%%%%%%%%%%%%%%%%%%%%%%%%%%%%%%%%%%%%%%%%%%%%%%%%%%%%%%%%%%%%%%%%%%%%%%%%%%
\section*{Aufnahmeprüfung 2018}
Aufnahmeprüfung 2018 Serie B2

1. a) Vereinfachen Sie den Term so weit wie möglich

$$\frac{5x}{14} + \frac{14x}4 \cdot{} \frac17 - \frac{x}{28}$$

\LoesungsBlock{$\frac{23x}{28}$}
%%%%%%%%%%%%%%%%%%%%%%%%%%%%%%%%%%%%%%%%%%%%%%%%%%%%%%%%%%%%%%%%%%%%%%%%%%%
Aufnahmeprüfung 2018 Serie B2

2. ) Vereinfachen Sie den Term so weit wie möglich

$$\frac{x^2+10x+25}{x+5} + \frac{x^2+2x-8}{x+4}$$

\LoesungsBlock{$2x+3$}
%%%%%%%%%%%%%%%%%%%%%%%%%%%%%%%%%%%%%%%%%%%%%%%%%%%%%%%%%%%%%%%%%%%%%%%%%%%
Aufnahmeprüfung 2018 Serie B1

1. b) Vereinfachen Sie den Term so weit wie möglich

$$\frac{5x}{12} + \frac{14x}{4} \cdot{} \frac16 - \frac{x}{24}$$

\LoesungsBlock{$2x$}
%%%%%%%%%%%%%%%%%%%%%%%%%%%%%%%%%%%%%%%%%%%%%%%%%%%%%%%%%%%%%%%%%%%%%%%%%%%





\part{Übungsaufgaben}

\part*{III a) GESO Maturaaufgaben / Kompendium}
\part*{III b) TALS Maturaaufgaben / Kompendium}


\end{document}
