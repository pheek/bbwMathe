%%
%% Meta: TI nSpire Einführung
%%       Ziel: Damit die Grundoperationen damit durchgeführt werden können.
%%             Damit man sich an den Rechner gewöhnt.
%%

\input{bmsLayoutPage}
\renewcommand{\bbwAufgabenBlockID}{A1Br}

\ifisNURAUFGABEN{
\newcommand{\LoesungsBlock}[1]{\TRAINER{Lösung:#1
\vspace{1mm}
\hrule}}%% end new Command "LoesungsBlock"
\else
\newcommand{\LoesungsBlock}[1]{\noTRAINER{\TNTeop{}}\TRAINER{Lösung:#1
\vspace{1mm}
\hrule}}%% end new Command "LoesungsBlock"
\fi
%%%%%%%%%%%%%%%%%%%%%%%%%%%%%%%%%%%%%%%%%%%%%%%%%%%%%%%%%%%%%%%%%%

\usepackage{amssymb} %% für \blacktriangleright
\renewcommand{\metaHeaderLine}{Arbeitsblatt Bruchrechnen}
\renewcommand{\arbeitsblattTitel}{Zusammenfassung aller Übungen}

\newcommand{\TNTeopS}[1]{\TRAINER{#1}\noTRAINER{\TNTeop{}}}

\begin{document}%%
\arbeitsblattHeader{}

\begin{center}\textit{\tiny{V 0.99 5. Okt. 2024}}\end{center}

\tableofcontents{}

\newpage


\textbf{Vorgehen}


\begin{enumerate}
\item 
Lösen Sie ca. 5-10 Aufgaben aus den alten Aufnahmeprüfungen (Kapitel I)
Wenn Sie davon gute 50 \% lösen können, so gehen Sie zu Kapitel III, den
alten Abschlussprüfungen (III a) : GESO; III b): TALS)

\item  Haben Sie weniger als ca 50 \% gelöst, werden die Trainingsaufgaben Kap. II empfohlen.
Lösen Sie pro Aufgabennummer mindestens je die ersten zwei und die letzte Aufgabe.
Wenn Sie mehr Training benötigen, so hat es genügend Übungsmaterial
insbesondere bei der Addition und Subtraktion
(Kapitel \ref{bruchtermeAdditionSubtraktion} auf Seite \pageref{bruchtermeAdditionSubtraktion}).
\end{enumerate}

\newpage

\part{Aus alten Aufnahmeprüfungen}
\section*{Aufnahmeprüfung 2023}
Aufnahmeprüfung 2023 Serie e

2. b) Vereinfachen Sie so weit wie möglich:

$$\frac{(x+2)(x-4)}{5} : \frac{x^2 - 16}{10}$$
\LoesungsBlock{$\frac{2\cdot{}(x+2)}{x+4}$}
%%%%%%%%%%%%%%%%%%%%%%%%%%%%%%%%%%%%%%%%%%%%%%%%%%%%%%%%%%%%%%%%%%%%%%

Aufnahmeprüfung 2023 Serie d

2. a) Vereinfachen Sie so weit wie möglich:

$$\frac{x^2-9}{x+3} : \frac{x-3}{4}$$

\LoesungsBlock{$4$}
%%%%%%%%%%%%%%%%%%%%%%%%%%%%%%%%%%%%%%%%%%%%%%%%%%%%%%%%%%%%%%%%%%%%%%%%%%%
Beispielprüfung 2023

3. b) Vereinfachen Sie so weit wie möglich:

$$\frac{x^2+2xy+y^2}{x^2-16} : \frac{x+y}{3x-12}$$

\LoesungsBlock{$\frac{3(x-4)}{x-y}$}
%%%%%%%%%%%%%%%%%%%%%%%%%%%%%%%%%%%%%%%%%%%%%%%%%%%%%%%%%%%%%%%%%%%%%%%%%%%
\section*{Aufnahmeprüfung 2022}
Aufnahmeprüfung 2022 Serie B

2. a) Vereinfachen Sie so weit wie möglich:

$$\frac14 \left(4-\frac{x}2 \right) - \left( \frac{3x}{8} - \frac{3x}{2}\right)$$

\LoesungsBlock{$1+x$}
%%%%%%%%%%%%%%%%%%%%%%%%%%%%%%%%%%%%%%%%%%%%%%%%%%%%%%%%%%%%%%%%%%%%%%%%%%%
Aufnahmeprüfung 2022 Serie B

2. b) Vereinfachen Sie so weit wie möglich:

$$\frac{2a+10}{a^2+10a+25}$$

\LoesungsBlock{$\frac{2}{a+5}$}
%%%%%%%%%%%%%%%%%%%%%%%%%%%%%%%%%%%%%%%%%%%%%%%%%%%%%%%%%%%%%%%%%%%%%%%%%%%
Aufnahmeprüfung 2022 Serie A

1. c) Vereinfachen Sie so weit wie möglich:

$$\frac{\sqrt{130x^2-(7x)^2}}{5x} + \frac{6x}{\sqrt{25x^2}}$$

\LoesungsBlock{$3$}
%%%%%%%%%%%%%%%%%%%%%%%%%%%%%%%%%%%%%%%%%%%%%%%%%%%%%%%%%%%%%%%%%%%%%%%%%%%
Aufnahmeprüfung 2022 Serie A

2. a) Vereinfachen Sie so weit wie möglich:

$$\frac{x^2+4x}{x^2+5x+4}$$

\LoesungsBlock{$\frac{x}{x+1}$}
%%%%%%%%%%%%%%%%%%%%%%%%%%%%%%%%%%%%%%%%%%%%%%%%%%%%%%%%%%%%%%%%%%%%%%%%%%%
Aufnahmeprüfung 2022 Serie A

2. b) Vereinfachen Sie so weit wie möglich:

$$\frac18\left(x+\frac12\right)  - \frac{3x^2}{8} : \frac{12x}{4}$$

\LoesungsBlock{$\frac1{16}$}
%%%%%%%%%%%%%%%%%%%%%%%%%%%%%%%%%%%%%%%%%%%%%%%%%%%%%%%%%%%%%%%%%%%%%%%%%%%
\section*{Aufnahmeprüfung 2021}
Aufnahmeprüfung 2021 Serie B1

3. a) Vereinfachen Sie so weit wie möglich und schreiben Sie als
einzigen Bruch:

$$3-\frac{2x-5}{4}$$

\LoesungsBlock{$\frac{17}{4} - \frac{x}2 = \frac{17-2x}{4}$}
%%%%%%%%%%%%%%%%%%%%%%%%%%%%%%%%%%%%%%%%%%%%%%%%%%%%%%%%%%%%%%%%%%%%%%%%%%%
Aufnahmeprüfung 2021 Serie B1

3. b) Vereinfachen Sie so weit wie möglich.

$$\frac{a}2 + \frac{15a^2c}{7b} : \frac{20ac}{14b}$$

\LoesungsBlock{$2a$}
%%%%%%%%%%%%%%%%%%%%%%%%%%%%%%%%%%%%%%%%%%%%%%%%%%%%%%%%%%%%%%%%%%%%%%%%%%%
Aufnahmeprüfung 2021 Serie B1

3. c) Vereinfachen Sie so weit wie möglich.

$$\frac{3x+4}{x-7} : \frac{5x+10}{x^2-5x-14}$$

\LoesungsBlock{$\frac{3x+4}{5}$}
%%%%%%%%%%%%%%%%%%%%%%%%%%%%%%%%%%%%%%%%%%%%%%%%%%%%%%%%%%%%%%%%%%%%%%%%%%%
Aufnahmeprüfung 2021 Serie A2

3. a) Vereinfachen Sie so weit wie möglich und schreiben Sie als
einzigen Bruch.

$$5-\frac{2x-4}{7}$$

\LoesungsBlock{$\frac{39-2x}{7}$}
%%%%%%%%%%%%%%%%%%%%%%%%%%%%%%%%%%%%%%%%%%%%%%%%%%%%%%%%%%%%%%%%%%%%%%%%%%%
Aufnahmeprüfung 2021 Serie A2

3. b) Vereinfachen Sie so weit wie möglich.

$$16a\cdot{}\frac{b^2}{8} + 9a : \frac{3}{b^2}$$

\LoesungsBlock{$5ab^2$}
%%%%%%%%%%%%%%%%%%%%%%%%%%%%%%%%%%%%%%%%%%%%%%%%%%%%%%%%%%%%%%%%%%%%%%%%%%%
Aufnahmeprüfung 2021 Serie A2

3. c) Vereinfachen Sie so weit wie möglich.

$$\frac{4x-12}{x^2-5x+6} : \frac{3x+1}{x-2}$$

\LoesungsBlock{$\frac4{3x+1}$}
%%%%%%%%%%%%%%%%%%%%%%%%%%%%%%%%%%%%%%%%%%%%%%%%%%%%%%%%%%%%%%%%%%%%%%%%%%%
\section*{Aufnahmeprüfung 2020}
Aufnahmeprüfung 2020 Serie B2

3. a) Vereinfachen Sie so weit wie möglich und schreiben Sie als Bruchterm

$$\frac{5(x-4)}{4} - \frac{x+5}{6}$$

\LoesungsBlock{$\frac{13x-70}{12}$}
%%%%%%%%%%%%%%%%%%%%%%%%%%%%%%%%%%%%%%%%%%%%%%%%%%%%%%%%%%%%%%%%%%%%%%%%%%%
Aufnahmeprüfung 2020 Serie B2

3. b) Vereinfachen Sie so weit wie möglich.

$$\frac{8a^2}{2b} : \frac{a^2}{3b^2} - \frac{b}{5}$$

\LoesungsBlock{$\frac{59b}{5} = 11.8b$}
%%%%%%%%%%%%%%%%%%%%%%%%%%%%%%%%%%%%%%%%%%%%%%%%%%%%%%%%%%%%%%%%%%%%%%%%%%%
Aufnahmeprüfung 2020 Serie B2

3. c) Vereinfachen Sie so weit wie möglich.

$$\frac{x-5}{x^2+6x} \cdot{} \frac{x^2+7x+6}{x^2-25}$$

\LoesungsBlock{$\frac{x+1}{x(x+5)}$}
%%%%%%%%%%%%%%%%%%%%%%%%%%%%%%%%%%%%%%%%%%%%%%%%%%%%%%%%%%%%%%%%%%%%%%%%%%%
Aufnahmeprüfung 2020 Serie B1
(Aufgabe 3a = Aufgabe 3b von Serie B2 aus 2020)

3. b) Vereinfachen Sie so weit wie möglich und schreiben Sie als Bruchterm

$$\frac{3(x-2)}{4} - \frac{x+4}{6}$$

\LoesungsBlock{$\frac{7x-26}{12}$}
%%%%%%%%%%%%%%%%%%%%%%%%%%%%%%%%%%%%%%%%%%%%%%%%%%%%%%%%%%%%%%%%%%%%%%%%%%%
Aufnahmeprüfung 2020 Serie B1

3. c) Vereinfachen Sie so weit wie möglich.

$$\frac{x-4}{x^2+5x} \cdot{} \frac{x^2+6x+5}{x^2-16}$$

\LoesungsBlock{$\frac{x+1}{x(x+4)}$}
%%%%%%%%%%%%%%%%%%%%%%%%%%%%%%%%%%%%%%%%%%%%%%%%%%%%%%%%%%%%%%%%%%%%%%%%%%%
Aufnahmeprüfung 2020 Serie A2

3. a) Vereinfachen Sie so weit wie möglich und schreiben Sie als Bruchterm

$$\frac{7(x-1)}{9} - \frac{x+4}{6}$$

\LoesungsBlock{$\frac{11x-26}{18}$}
%%%%%%%%%%%%%%%%%%%%%%%%%%%%%%%%%%%%%%%%%%%%%%%%%%%%%%%%%%%%%%%%%%%%%%%%%%%
Aufnahmeprüfung 2020 Serie A2

3. b) Vereinfachen Sie so weit wie möglich.

$$\frac{19b}{3} - \frac{2a^2}{4b} : \frac{a^2}{6b^2}$$

\LoesungsBlock{$\frac{10b}{3}$}
%%%%%%%%%%%%%%%%%%%%%%%%%%%%%%%%%%%%%%%%%%%%%%%%%%%%%%%%%%%%%%%%%%%%%%%%%%%
Aufnahmeprüfung 2020 Serie A2

3. b) Vereinfachen Sie so weit wie möglich.

$$\frac{x^2-6x}{x^2+2x+1} \cdot{} \frac{x^2-1}{x-6}$$

\LoesungsBlock{$\frac{x(x-1)}{x+1}$}
%%%%%%%%%%%%%%%%%%%%%%%%%%%%%%%%%%%%%%%%%%%%%%%%%%%%%%%%%%%%%%%%%%%%%%%%%%%
Aufnahmeprüfung 2020 Serie A1

3. a) Vereinfachen Sie so weit wie möglich.

$$\frac{17a}{3} - \frac{2b^2}{4a}  :   \frac{b^2}{6a^2}$$

\LoesungsBlock{$\frac{8a}3$}
%%%%%%%%%%%%%%%%%%%%%%%%%%%%%%%%%%%%%%%%%%%%%%%%%%%%%%%%%%%%%%%%%%%%%%%%%%%
Aufnahmeprüfung 2020 Serie A1

3. b) Vereinfachen Sie so weit wie möglich und schreiben Sie als Bruchterm

$$\frac{5(x-1)}{6} - \frac{x+3}{9}$$

\LoesungsBlock{$\frac{13x-21}{18}$}
%%%%%%%%%%%%%%%%%%%%%%%%%%%%%%%%%%%%%%%%%%%%%%%%%%%%%%%%%%%%%%%%%%%%%%%%%%%
Aufnahmeprüfung 2020 Serie A1

3. c) Vereinfachen Sie so weit wie möglich

$$\frac{x^2-3x}{x^2+2x+1} \cdot{} \frac{x^2-1}{x-3}$$

\LoesungsBlock{$\frac{x(x-1)}{x+1}$}
%%%%%%%%%%%%%%%%%%%%%%%%%%%%%%%%%%%%%%%%%%%%%%%%%%%%%%%%%%%%%%%%%%%%%%%%%%%
\section*{Aufnahmeprüfung 2019}
Aufnahmeprüfung 2019 Serie B2

3. a) Vereinfachen Sie den Term so weit wie möglich

$$\frac{2x}{5} : \frac{4}{15} - \frac{7x}{6} \cdot{} \frac{3}{28}$$

\LoesungsBlock{$\frac{11x}8$}
%%%%%%%%%%%%%%%%%%%%%%%%%%%%%%%%%%%%%%%%%%%%%%%%%%%%%%%%%%%%%%%%%%%%%%%%%%%
Aufnahmeprüfung 2019 Serie B2

3. b) Vereinfachen Sie den Term so weit wie möglich

$$\frac{3(x-y)^2}{x+y} \cdot{} \frac{6xy+6y^2}{x^2-2xy+y^2}$$

\LoesungsBlock{$18y$}
%%%%%%%%%%%%%%%%%%%%%%%%%%%%%%%%%%%%%%%%%%%%%%%%%%%%%%%%%%%%%%%%%%%%%%%%%%%
Aufnahmeprüfung 2019 Serie B1

3. a) Vereinfachen Sie den Term und kürzen Sie die Resultate so weit wie möglich

$$\frac{5x}{2} : \frac{14}{4} - \frac{2x}{21} \cdot{} \frac{7}{6}$$

\LoesungsBlock{$\frac{5x}{9}$}
%%%%%%%%%%%%%%%%%%%%%%%%%%%%%%%%%%%%%%%%%%%%%%%%%%%%%%%%%%%%%%%%%%%%%%%%%%%
Aufnahmeprüfung 2019 Serie B1

3. b) Vereinfachen Sie den Term so weit wie möglich

$$\frac{4xy+4y^2}{x^2-2xy+y^2} \cdot{} \frac{6(x-y)^2}{x+y}$$

\LoesungsBlock{$24y$}
%%%%%%%%%%%%%%%%%%%%%%%%%%%%%%%%%%%%%%%%%%%%%%%%%%%%%%%%%%%%%%%%%%%%%%%%%%%
Aufnahmeprüfung 2019 Serie A2

3. a) Vereinfachen Sie den Term so weit wie möglich

$$\frac{2x}{3} \cdot{} \frac{9}{4} + \frac{3x}{2} : \frac{9}{16}$$

\LoesungsBlock{$\frac{25x}6$}
%%%%%%%%%%%%%%%%%%%%%%%%%%%%%%%%%%%%%%%%%%%%%%%%%%%%%%%%%%%%%%%%%%%%%%%%%%%
Aufnahmeprüfung 2019 Serie A2

3. b) Vereinfachen Sie den Term so weit wie möglich

$$\frac{4x^2-4xy}{x^2+2xy+y^2} \cdot{} \frac{5(x+y)^2}{x-y}$$

\LoesungsBlock{$20x$}
%%%%%%%%%%%%%%%%%%%%%%%%%%%%%%%%%%%%%%%%%%%%%%%%%%%%%%%%%%%%%%%%%%%%%%%%%%%
Aufnahmeprüfung 2019 Serie A1

3. a) Vereinfachen Sie den Term so weit wie möglich

$$\frac{3x}2 \cdot{} \frac49 + \frac{2x}3 : \frac89$$

\LoesungsBlock{$\frac{17x}{12}$}
%%%%%%%%%%%%%%%%%%%%%%%%%%%%%%%%%%%%%%%%%%%%%%%%%%%%%%%%%%%%%%%%%%%%%%%%%%%
Aufnahmeprüfung 2019 Serie A1

3. b) Vereinfachen Sie den Term so weit wie möglich

$$\frac{3(x+y)^2}{x-y} \cdot{} \frac{5x^2-5xy}{x^2+2xy+y^2}$$

\LoesungsBlock{$15x$}
%%%%%%%%%%%%%%%%%%%%%%%%%%%%%%%%%%%%%%%%%%%%%%%%%%%%%%%%%%%%%%%%%%%%%%%%%%%
\section*{Aufnahmeprüfung 2018}
Aufnahmeprüfung 2018 Serie B2

1. a) Vereinfachen Sie den Term so weit wie möglich

$$\frac{5x}{14} + \frac{14x}4 \cdot{} \frac17 - \frac{x}{28}$$

\LoesungsBlock{$\frac{23x}{28}$}
%%%%%%%%%%%%%%%%%%%%%%%%%%%%%%%%%%%%%%%%%%%%%%%%%%%%%%%%%%%%%%%%%%%%%%%%%%%
Aufnahmeprüfung 2018 Serie B2

2. ) Vereinfachen Sie den Term so weit wie möglich

$$\frac{x^2+10x+25}{x+5} + \frac{x^2+2x-8}{x+4}$$

\LoesungsBlock{$2x+3$}
%%%%%%%%%%%%%%%%%%%%%%%%%%%%%%%%%%%%%%%%%%%%%%%%%%%%%%%%%%%%%%%%%%%%%%%%%%%
Aufnahmeprüfung 2018 Serie B1

1. b) Vereinfachen Sie den Term so weit wie möglich

$$\frac{5x}{12} + \frac{14x}{4} \cdot{} \frac16 - \frac{x}{24}$$

\LoesungsBlock{$\frac{23x}{24}$}
%%%%%%%%%%%%%%%%%%%%%%%%%%%%%%%%%%%%%%%%%%%%%%%%%%%%%%%%%%%%%%%%%%%%%%%%%%%
Aufnahmeprüfung 2018 Serie B1

2. ) Vereinfachen Sie den Term so weit wie möglich

$$\frac{x^2+8x+16}{x+4} + \frac{x^2-3x-4}{x+1}$$

\LoesungsBlock{$2x$}
%%%%%%%%%%%%%%%%%%%%%%%%%%%%%%%%%%%%%%%%%%%%%%%%%%%%%%%%%%%%%%%%%%%%%%%%%%%
Aufnahmeprüfung 2018 Serie A2

1. b) Vereinfachen Sie den Term so weit wie möglich

$$\frac{3x}4 + \frac{10x}8 \cdot{} \frac12 - \frac{x}{16}$$

\LoesungsBlock{$\frac{21x}{16}$}
%%%%%%%%%%%%%%%%%%%%%%%%%%%%%%%%%%%%%%%%%%%%%%%%%%%%%%%%%%%%%%%%%%%%%%%%%%%
Aufnahmeprüfung 2018 Serie A2

2. ) Vereinfachen Sie den Term so weit wie möglich

$$\frac{x^2+4x+4}{x+2} + \frac{x^2+2x-15}{x-3}$$

\LoesungsBlock{$2x+7$}
%%%%%%%%%%%%%%%%%%%%%%%%%%%%%%%%%%%%%%%%%%%%%%%%%%%%%%%%%%%%%%%%%%%%%%%%%%%
Aufnahmeprüfung 2018 Serie A1

1. a) Vereinfachen Sie den Term so weit wie möglich

$$\frac{2x}9 + \frac{8x}6 \cdot{} \frac13 - \frac{x}{18}$$

\LoesungsBlock{$\frac{11x}{18}$}
%%%%%%%%%%%%%%%%%%%%%%%%%%%%%%%%%%%%%%%%%%%%%%%%%%%%%%%%%%%%%%%%%%%%%%%%%%%
Aufnahmeprüfung 2018 Serie A1

2. ) Vereinfachen Sie den Term so weit wie möglich

$$\frac{x^2+6x+9}{x+3} + \frac{x^2-3x-10}{x-5}$$

\LoesungsBlock{$2x+5$}
%%%%%%%%%%%%%%%%%%%%%%%%%%%%%%%%%%%%%%%%%%%%%%%%%%%%%%%%%%%%%%%%%%%%%%%%%%%
\section*{Aufnahmeprüfung 2017}
Aufnahmeprüfung 2017 Serie B2

1. ) Vereinfachen Sie den Term so weit wie möglich

$$\frac{b^2-8b+16}{b^2-7b+12}$$

\LoesungsBlock{$\frac{b-4}{b-3}$}
%%%%%%%%%%%%%%%%%%%%%%%%%%%%%%%%%%%%%%%%%%%%%%%%%%%%%%%%%%%%%%%%%%%%%%%%%%%
Aufnahmeprüfung 2017 Serie B2

2. ) Vereinfachen Sie den Term so weit wie möglich

$$\frac{5}{7x} : \frac{12}{\sqrt{49x^2}} + \frac{31x}{\sqrt{400x^2 - (16x)^2}}$$

\LoesungsBlock{$3$}
%%%%%%%%%%%%%%%%%%%%%%%%%%%%%%%%%%%%%%%%%%%%%%%%%%%%%%%%%%%%%%%%%%%%%%%%%%%
Aufnahmeprüfung 2017 Serie B1

1. ) Vereinfachen Sie den Term so weit wie möglich

$$\frac{a^2-5a+6}{a^2-6a+9}$$

\LoesungsBlock{$\frac{a-2}{a-3}$}
%%%%%%%%%%%%%%%%%%%%%%%%%%%%%%%%%%%%%%%%%%%%%%%%%%%%%%%%%%%%%%%%%%%%%%%%%%%
Aufnahmeprüfung 2017 Serie B1

2. ) Vereinfachen Sie den Term so weit wie möglich

$$\frac{\sqrt{81x^2}}{3} : \frac{2x}3 + \frac{\sqrt{169x^2 - (12x)^2}}{2x}$$

\LoesungsBlock{$7$}
%%%%%%%%%%%%%%%%%%%%%%%%%%%%%%%%%%%%%%%%%%%%%%%%%%%%%%%%%%%%%%%%%%%%%%%%%%%
Aufnahmeprüfung 2017 Serie A2

1. ) Vereinfachen Sie den Term so weit wie möglich

$$\frac{y^2+8y+16}{y^2-16}$$

\LoesungsBlock{$\frac{y+4}{y-4}$}
%%%%%%%%%%%%%%%%%%%%%%%%%%%%%%%%%%%%%%%%%%%%%%%%%%%%%%%%%%%%%%%%%%%%%%%%%%%
Aufnahmeprüfung 2017 Serie A2

2. ) Vereinfachen Sie den Term so weit wie möglich

$$\frac{19x}{\sqrt{(17x)^2 - 64x^2}} + \frac{\sqrt{121x^2}}{x^2} : \frac{15}{x}$$

\LoesungsBlock{$2$}
%%%%%%%%%%%%%%%%%%%%%%%%%%%%%%%%%%%%%%%%%%%%%%%%%%%%%%%%%%%%%%%%%%%%%%%%%%%
Aufnahmeprüfung 2017 Serie A1

1. ) Vereinfachen Sie den Term so weit wie möglich

$$\frac{x^2-25}{x^2+10x+25}$$

\LoesungsBlock{$\frac{x-5}{x+5}$}
%%%%%%%%%%%%%%%%%%%%%%%%%%%%%%%%%%%%%%%%%%%%%%%%%%%%%%%%%%%%%%%%%%%%%%%%%%%
Aufnahmeprüfung 2017 Serie A1

2. ) Vereinfachen Sie den Term so weit wie möglich

$$\frac{\sqrt{289x^2 - (15x)^2}}{3x} + \frac{2x^2}{\sqrt{9x^2}} : \frac{x}5$$

\LoesungsBlock{$6$}
%%%%%%%%%%%%%%%%%%%%%%%%%%%%%%%%%%%%%%%%%%%%%%%%%%%%%%%%%%%%%%%%%%%%%%%%%%%
\section*{Aufnahmeprüfung 2016}
Aufnahmeprüfung 2016 Serie B2

1. ) Vereinfachen Sie den Term so weit wie möglich. Das Resultat darf
keine Klammern enthalten.

$$ \frac{2(p-r)}{3a} \cdot{} \frac{6(r-p)}{24a}$$

\LoesungsBlock{$\frac{-p^2-2pr-r^2}{6a^2} $}

%%%%%%%%%%%%%%%%%%%%%%%%%%%%%%%%%%%%%%%%%%%%%%%%%%%%%%%%%%%%%%%%%
Aufnahmeprüfung 2016 Serie B2

2. ) Vereinfachen Sie den Term so weit wie möglich.

$$\frac{\sqrt{(4a)^2+4a^2+4a\cdot{}11a}}{21a} - \frac{3b}{\sqrt{(2b\cdot{}3)^2+45b^2}}$$

\LoesungsBlock{$ \frac1{21}$}

%%%%%%%%%%%%%%%%%%%%%%%%%%%%%%%%%%%%%%%%%%%%%%%%%%%%%%%%%%%%%%%%%
Aufnahmeprüfung 2016 Serie B1

1. ) Vereinfachen Sie den Term so weit wie möglich. Das Resultat darf
keine Klammern enthalten.

$$\frac{2(a+b)}{3b} \cdot{} \frac{3(b-a)}{4b}$$

\LoesungsBlock{$\frac12-\frac{a^2}{2b^2}$}

%%%%%%%%%%%%%%%%%%%%%%%%%%%%%%%%%%%%%%%%%%%%%%%%%%%%%%%%%%%%%%%%%
Aufnahmeprüfung 2016 Serie B1

2. ) Vereinfachen Sie den Term so weit wie möglich. 

$$\frac{\sqrt{(3c)^2+15c^2+5c\cdot{}5c}}{21c} - \frac{d}{\sqrt{(10d)^2 + 21d^2}}$$

\LoesungsBlock{$\frac8{33}$}

%%%%%%%%%%%%%%%%%%%%%%%%%%%%%%%%%%%%%%%%%%%%%%%%%%%%%%%%%%%%%%%%%
Aufnahmeprüfung 2016 Serie A2

1. ) Vereinfachen Sie den Term so weit wie möglich. 

$$\frac{3r^2}{-5p} : \frac{12r}{15p^2}$$

\LoesungsBlock{$\frac{-3rp}{4}$}

%%%%%%%%%%%%%%%%%%%%%%%%%%%%%%%%%%%%%%%%%%%%%%%%%%%%%%%%%%%%%%%%%
Aufnahmeprüfung 2016 Serie A2

2. ) Vereinfachen Sie den Term so weit wie möglich. 

$$\frac1{\sqrt{5a^2+22a\cdot{}2a}} + \frac1{\sqrt{(8a)^2-39a^2}}$$

\LoesungsBlock{$\frac{12}{35a}$}

%%%%%%%%%%%%%%%%%%%%%%%%%%%%%%%%%%%%%%%%%%%%%%%%%%%%%%%%%%%%%%%%%
Aufnahmeprüfung 2016 Serie A1

1. ) Vereinfachen Sie den Term so weit wie möglich. 

$$\frac{2a^2}{3b} : \frac{-4a}{9b^2}$$

\LoesungsBlock{$\frac{-3ab}2$}

%%%%%%%%%%%%%%%%%%%%%%%%%%%%%%%%%%%%%%%%%%%%%%%%%%%%%%%%%%%%%%%%%
Aufnahmeprüfung 2016 Serie A1

2. ) Vereinfachen Sie den Term so weit wie möglich. 

$$\frac1{\sqrt{5b^2+10b\cdot{}2b}} + \frac1{\sqrt{(10b)^2-19b^2}}$$

\LoesungsBlock{$\frac{14}{45b}$}
%%%%%%%%%%%%%%%%%%%%%%%%%%%%%%%%%%%%%%%%%%%%%%%%%%%%%%%%%%%%%%%   
\section*{Aufnahmeprüfung 2015}
Aufnahmeprüfung 2015 Serie B2

1. ) Vereinfachen Sie den Term so weit wie möglich und schreiben Sie
das Resultat als Bruchterm.

$$\frac{4f+e}8 - \frac{f-e}2$$

\LoesungsBlock{$\frac{5e}8$}
%%%%%%%%%%%%%%%%%%%%%%%%%%%%%%%%%%%%%%%%%%%%%%%%%%%%%%%%%%%%%%%   
Aufnahmeprüfung 2015 Serie B2

2. ) Vereinfachen Sie den Term so weit wie möglich

$$\frac{\sqrt{(5y)^2+3y\cdot{}8y}}2 - \frac{\sqrt{5y^2-y^2}}8$$

\LoesungsBlock{$\frac{13y}4$}
%%%%%%%%%%%%%%%%%%%%%%%%%%%%%%%%%%%%%%%%%%%%%%%%%%%%%%%%%%%%%%%   
Aufnahmeprüfung 2015 Serie B1

1. ) Vereinfachen Sie den Term so weit wie möglich und schreiben Sie
das Resultat als Bruchterm.

$$\frac{4c+3e}{9} - \frac{c+e}{3}$$

\LoesungsBlock{$\frac{c}9$}
%%%%%%%%%%%%%%%%%%%%%%%%%%%%%%%%%%%%%%%%%%%%%%%%%%%%%%%%%%%%%%%   
Aufnahmeprüfung 2015 Serie B1

2. ) Vereinfachen Sie den Term so weit wie möglich

$$\frac{\sqrt{(8x)^2+3x\cdot{}12x}}4 - \frac{\sqrt{15x^2+x^2}}3$$

\LoesungsBlock{$\frac{7x}6$}
%%%%%%%%%%%%%%%%%%%%%%%%%%%%%%%%%%%%%%%%%%%%%%%%%%%%%%%%%%%%%%%   
Aufnahmeprüfung 2015 Serie A2

1. ) Vereinfachen Sie den Term so weit wie möglich und schreiben Sie
das Resultat als Bruchterm.

$$\frac{7b}9 - \left( \frac{5b}6 - \frac{b}3 \right)$$

\LoesungsBlock{$\frac{5b}{18}$}
%%%%%%%%%%%%%%%%%%%%%%%%%%%%%%%%%%%%%%%%%%%%%%%%%%%%%%%%%%%%%%%   
Aufnahmeprüfung 2015 Serie A2

2. ) Vereinfachen Sie den Term so weit wie möglich

$$\frac{\sqrt{(12b)^2 - 63b^2}}{4ab} : \frac{\sqrt{35b^2 + b^2}}{2a}$$

\LoesungsBlock{$\frac3{4b}$}
%%%%%%%%%%%%%%%%%%%%%%%%%%%%%%%%%%%%%%%%%%%%%%%%%%%%%%%%%%%%%%%   
Aufnahmeprüfung 2015 Serie A1

1. ) Vereinfachen Sie den Term so weit wie möglich und schreiben Sie
das Resultat als Bruchterm.

$$\frac{5a}{12} - \left( \frac{7a}8 + \frac{a}4 \right)$$

\LoesungsBlock{$\frac{-17a}{24}$}
%%%%%%%%%%%%%%%%%%%%%%%%%%%%%%%%%%%%%%%%%%%%%%%%%%%%%%%%%%%%%%%   
Aufnahmeprüfung 2015 Serie A1

2. ) Vereinfachen Sie den Term so weit wie möglich

$$\frac{\sqrt{(13a)^2 = 25a^2}}{6ab} : \frac{\sqrt{10a^2 - a^2}}{2b}$$

\LoesungsBlock{$\frac4{3a}$}
\newpage
%%%%%%%%%%%%%%%%%%%%%%%%%%%%%%%%%%%%%%%%%%%%%%%%%%%%%%%%%%%%%%%


\part{Übungsaufgaben}

\section{Termwerte}
Berechnen Sie die Termwerte und vereinfachen Sie so weit wie
möglich. Gibt es Werte für Variable, die nicht eingesetzt werden dürfen?

\nextBbwAufgabenNummer{}
\begin{bbwAufgabenBlock}
\item $$T(x) := \frac{x^2-6x}{x-4}$$
Berechnen Sie damit
$$T(-2)$$
\LoesungsBlock{$\frac{-8}3$}

\item $$T(x):=\frac{7x-a}{x^2-25}$$
Berechnen Sie damit
$$T(-5)$$
\LoesungsBlock{$\not\in\mathbb{R}$ Wir dürfen in $\mathbb{R}$ nicht
durch Null teilen.}

\item $$T(x):= \frac{5x(a-x)(x+3)(x-4)}{25(a-x)(x+3)(x+6)}$$
Berechnen Sie damit
$$T(3)$$
\LoesungsBlock{$\frac{-1}{15}$ Sonderfälle. $T$ ist für $a=x$, $x=-3$
oder $x=-6$ nicht definiert. Für $T(3)$ ist also der Term für $a=3$
nicht definiert.}
\end{bbwAufgabenBlock}

\newpage
\section{Faktorisiert?}

Welche Brüche sind im Zähler und Nenner bereits faktorisiert? Kreuzen Sie an:
(Alle folgenden Brüche sind nicht kürzbar.)

\newcommand{\BoxTT}{\TRAINER{\fbox{\color{green} x}}\noTRAINER{\Box}}

\begin{bbwAufgabenBlock}
\item $\frac{7+x}{8+x}$ $\Box{}$ \TRAINER{f: $\frac{\text{\color{red}Summe}}{\text{\color{red}Summe}}$}

\item $\frac{24(7+x)}{(8+x)\cdot{}13}$ $\BoxTT$ \TRAINER{$\frac{\text{Produkt}}{\text{Produkt}}$}

\item $\frac{7x}{7+x}$ $\Box{}$ \TRAINER{f: $\frac{\text{Produkt}}{\text{\color{red}Summe}}$}

\item $\frac{a(-1)\sqrt{x+3}}{x+3\cdot{}4b}$ $\Box{}$ \TRAINER{f: $\frac{\text{Produkt}}{\text{\color{red}Summe}}$}

\item $\frac{x^2+1}{x-1}$ $\Box{}$ \TRAINER{f: $\frac{\text{\color{red}Summe}}{\text{\color{red}Differenz}}$}

\item $\frac{2x-3b}{3b\cdot{}(2x)}$ $\Box{}$ \TRAINER{f: $\frac{\text{\color{red}Differenz}}{\text{Produkt}}$}

\item $\frac{6(4-x)}{7(x+5)^2}$ $\BoxTT$ \TRAINER{$\frac{\text{Produkt}}{\text{Produkt}}$}

\item $\frac{x^2(a-c)}{x^2\cdot{}a -c}$ $\Box{}$ \TRAINER{f: $\frac{\text{Produkt}}{\text{\color{red}Differenz}}$}

\item $\frac{b\cdot{}\sqrt{a}(a^2-3)}{a^2\cdot{}\sqrt{b}(a-4)^2}$ $\BoxTT$ \TRAINER{$\frac{\text{Produkt}}{\text{Produkt}}$}

\item $\frac{8(x-4)}{-x-4}$ $\Box{}$ \TRAINER{f: $\frac{\text{Produkt}}{\text{\color{red}Differenz}}$}

\item $\frac{8(x-4)}{(-1)(x+4)}$ $\BoxTT$ \TRAINER{$\frac{\text{Produkt}}{\text{Produkt}}$}


\end{bbwAufgabenBlock}
\newpage
%%%%%%%%%%%%%%%%%%%%%%%%%%%%%%%%%%%%%%%%%%%%%%%%%%%%%%%%%%%%%%%%%%%%%%%%%%%%%%%%%%%%
\section{Zähler und Nenner Faktorisieren}
Faktorisieren Sie jeweils Zähler und Nenner:

\begin{bbwAufgabenBlock}
\item $$\frac{x^2+3x+2}{x^2+7x+12}$$
\LoesungsBlock{$$\frac{(x+1)(x+2)}{(x+3)(x+4)}$$}

\item $$\frac{x^2+2x+1}{x^2-2x+1}$$
\LoesungsBlock{$$\frac{(x+1)(x+1)}{(x-1)(x-1)}$$}

\item $$\frac{ac+ad+bc+bd}{x^2-1}$$
\LoesungsBlock{$$\frac{(a+b)(c+d)}{(x+1)(x-1)}$$}

\item $$\frac{xy+x+y+1}{xy-x-y+1}$$
\LoesungsBlock{$$\frac{(x+1)(y+1)}{(x-1)(y-1)}$$}

\item $$\frac{ab+ac}{ab+bc}$$
\LoesungsBlock{$$\frac{a(b+c)}{b(a+c)}$$}

\item $$\frac{\sqrt{2}\cdot{}ax^2-\sqrt{2}\cdot{}ay^2}{2bxy-2bx+2xy-2x}$$
\LoesungsBlock{$$\frac{\sqrt{2}a(x-y)(x+y)}{2x(b+1)(y-1)}$$}


\end{bbwAufgabenBlock}
\newpage
%%%%%%%%%%%%%%%%%%%%%%%%%%%%%%%%%%%%%%%%%%%%%%%%%%%%%%%%%%%%%%%%%%%%%%%%%%%%%%%%%%%%
\section{Kürzen}
Kürzen Sie so weit wie möglich. Faktorisieren Sie vorab wenn nötig.
Vor dem Kürzen müssen im Zähler sowie im Nenner je ein Produkt stehen.

\begin{bbwAufgabenBlock}
\item $$\frac{7ab}{14b}$$
\LoesungsBlock{$$\frac{a}2$$}

\item $$\frac{3ab^2}{9a^2b^4}$$
\LoesungsBlock{$$\frac1{3ab^2}$$}

\item $$\frac{-35r^3s^6}{-42r^4s^2}$$
\LoesungsBlock{$$\frac{5s^4}{6r}$$}

\item $$\frac{ax+ay}{bx+by}$$
\LoesungsBlock{$$\frac{a}b$$}

\item $$\frac{192rss-48st}{180rss-48st}$$
\LoesungsBlock{$$\frac{16rs-4t}{15rs-4t}$$}

\item $$\frac{(3-x)(8.4+y)}{(8.4-x}(3-x)$$
\LoesungsBlock{$$\frac{8.4+y}{8.4-x}$$}

\item $$\frac{7e-7f}{e^2-f^2}$$
\LoesungsBlock{$$\frac{7}{e+f}$$}

\item $$\frac{a^2+a2b+b^2}{a^2-b^2}$$
\LoesungsBlock{$$\frac{a+b}{a-b}$$}

\item $$\frac{xy+2x-y-2}{xy-x-y+1}$$
\LoesungsBlock{$$\frac{y+2}{y-1}$$}

\item $$\frac{x^2+x-6}{x^2-x-12}$$
\LoesungsBlock{$$\frac{x-2}{x-4}$$}

\item $$\frac{a^4-5a^3}{a^4-a^3-20a^2}$$
\LoesungsBlock{$$\frac{a}{a+4}$$}

\item $$\frac{x^2-xy-6y^2}{x^2-2xy-8y^2}$$
\LoesungsBlock{$$\frac{x-3y}{x-4y}$$}

\item $$\frac{ab-ac}{c-b}$$
\LoesungsBlock{$$-a$$}

\item $$\frac{x^3-x^2}{1-x^2}$$
\LoesungsBlock{$$\frac{-x^2}{1+x}$$}

\item $$\frac{ab-2cd}{2abcd}$$
\LoesungsBlock{Nicht weiter kürzbar: Zähler und Nenner sind bereits
vollständig faktorisiert.}

\vspace{3mm}
Challenge:

\item $$\frac{(a+4)^2-(b-1)^2}{a+13-(b+8)}$$
\LoesungsBlock{$$a+b+3$$}


\item $$\frac{49\varepsilon^2 - 4(\varepsilon+2)^2}{45\varepsilon^2 - 16(\varepsilon+1)}$$
\LoesungsBlock{$$1$$}


\end{bbwAufgabenBlock}
\newpage
%%%%%%%%%%%%%%%%%%%%%%%%%%%%%%%%%%%%%%%%%%%%%%%%%%%%%%%%%%%%%%%%%%%%%%%
\section{Erweitern auf vorgegebenen Nenner}
Erweitern Sie den vorgegebenen Bruch auf den vorgegebenen Nenner.

\begin{bbwAufgabenBlock}
\item
Bruch: $\frac{7x-4}{ab}$  Nenner: $a^2b^3$

\LoesungsBlock{$$\frac{(7x-4)ab^2}{a^2b^3}$$}

\item
Bruch: $\frac{x+2}{x-2}$  Nenner: $x^2-4x+4$

\LoesungsBlock{$$\frac{x^2-4}{(x-2)^2}$$}
\end{bbwAufgabenBlock}
%%%%%%%%%%%%%%%%%%%%%%%%%%%%%%%%%%%%%%%%%%%%%%%%%%%%%%%%%%%%%%%%%%%%%%%%%%%%
\section{Erweitern mit $-1$}
Erweitern Sie den Bruch mit $(-1)$:

\begin{bbwAufgabenBlock}
\item $$\frac{1-b}{-b-3}$$
\LoesungsBlock{$$\frac{b-1}{b+3}$$}


\item $$\frac{-5(-a+1)}{-s}$$
\LoesungsBlock{$$\frac{5(1-a)}s$$}
\end{bbwAufgabenBlock}
%%%%%%%%%%%%%%%%%%%%%%%%%%%%%%%%%%%%%%%%%%%%%%%%%%%%%%%%%%%%%%%%%%%%%%%%%%%%%%%%%
\section{Gleichnamig}
Machen Sie jeweils die beden gegebenen Brüche gleichnamig:

\begin{bbwAufgabenBlock}
\item $$\frac{7a}{33} \text{ und } \frac{2b}{44}$$
\LoesungsBlock{$$\frac{14a}{66} \text{ und } \frac{3b}{66}$$}

\item $$\frac{3}{a+1} \text{ und } \frac{4}{2a-4}$$
\LoesungsBlock{$$\frac{3(a-2)}{(a+1)(a-2)} \text{ und } \frac2{(a+1)(a-2)}$$}

\end{bbwAufgabenBlock}
\newpage
%%%%%%%%%%%%%%%%%%%%%%%%%%%%%%%%%%%%%%%%%%%%%%%%%%%%%%%%%%%%%%%%%%%%%%%%%%%%%%%%%%%
\section{Addition / Subtraktion}\label{bruchtermeAdditionSubtraktion}
Addieren bzw. subtrahieren Sie die folgenden Bruchterme. Schreiben Sie
das Resultat jeweils als Bruchterm.


\begin{bbwAufgabenBlock}
\item $$\frac73 - \frac34$$
\LoesungsBlock{$$\frac{19}{12}$$}

\item $$\frac14 + \frac{a}{3}$$
\LoesungsBlock{$$\frac{3+4a}{12}$$}

\item $$\frac13 - \frac1{12} + \frac{x}5$$
\LoesungsBlock{$$\frac{5+4x}{20}$$}

\item $$\frac{7x}{33} - \frac{12}{22} + \frac{2a}{55a}$$
\LoesungsBlock{$$\frac{35x-84}{165}$$}

\item $$\frac{a+2b}{a} - \frac{a-2b}{a}$$
\LoesungsBlock{$$\frac{4b}{a}$$}

\item $$\frac{4x}5 + \frac{9x}{13}$$
\LoesungsBlock{$$\frac{97}{65}\cdot{}x$$}

\item $$\frac{5a}{b^2} - \frac{-a}{2b}$$
\LoesungsBlock{$$\frac{10a+ab}{2b^2}$$}

\item $$7 - \frac{2a-3}3 $$
\LoesungsBlock{$$\frac{24-2a}3$$}

\item $$x + 2 - \frac{3x+x^2}{x}$$
\LoesungsBlock{$$-1$$}

\item $$\frac1a - \frac1{b-a}$$
\LoesungsBlock{$$\frac{b-2a}{a(b-a)}$$}

\item $$\frac{a}x + \frac{b}y + \frac{c}z$$
\LoesungsBlock{$$\frac{axy+bxz+cxy}{xyz}$$}

\item $$\frac{a}{5x-6} - \frac{b-2a}{12-10x}$$
\LoesungsBlock{$$\frac{b}{2(5x-6)}$$}

\item $$x - \frac{x^3-5}{x^2-5}$$
\LoesungsBlock{$$\frac{5((1-x)}{x^2-5}$$}

\item $$\frac{a+3b}{a^2+6ab+9b^2} - \frac{a-b}{5a+15b}$$
\LoesungsBlock{$$\frac{5-a+b}{5(a+3b)}$$}

\item $$\frac{m^2}{m-1} + \frac{6m^2}{12m}$$
\LoesungsBlock{$$\frac{m(3m-1)}{2(m-1)}$$}

\item $$\frac5x - \frac3{10} - \frac{x}{15}$$
\LoesungsBlock{$$\frac{150-9x-2x^2}{30x}$$}

\item $$\frac{a+6}{a^2+5a-14} + \frac{3-a}{a^2-4a+4}$$
\LoesungsBlock{$$\frac{9}{(a-2)^2\cdot{}(a+7)}$$}
 
\item $$\frac{2e-f}{12e^2+16ef} - \frac{1.5}{9e+12f}$$
\LoesungsBlock{$$\frac{2e-f-2}{4(3e+4f)}$$}


\item $$\frac{\alpha}{2\alpha -3\beta} + \frac{\beta}{3\beta-2\alpha}
- \frac{\alpha\beta}{4\alpha^2 - 9\beta^2}$$
\LoesungsBlock{$$\frac{2\alpha^2 - 3\beta^2}{(2\alpha-3\beta)(2\alpha+3\beta)}$$}


\end{bbwAufgabenBlock}
\newpage
%%%%%%%%%%%%%%%%%%%%%%%%%%%%%%%%%%%%%%%%%%%%%%%%%%%%%%%%%%%%%%%%%%%%%%%%%%%%%%%%%%%%
\section{Multiplikation}
Multiplizieren Sie die Bruchterme:

\begin{bbwAufgabenBlock}
\item $$\frac{a}b \cdot{} \frac{c-a}{a^2}$$
\LoesungsBlock{$$\frac{c-d}{ab}$$}

\item $$m \cdot{} \frac{n}{-m}$$
\LoesungsBlock{$$-n$$}

\item $$\frac{x-3}{6x^3-18x^2} \cdot{} \left(-3x^2\right)$$
\LoesungsBlock{$$\frac{-1}2$$}

\item $$(a-b) \cdot{} \frac{2a}{b-a}$$
\LoesungsBlock{$$-2a$$}

\item $$\frac{-xy}{4x-4y} \cdot{} \left(16y-16x\right)$$
\LoesungsBlock{$$4xy$$}

\item $$\frac{3x-3y}{2z} \cdot{} \frac{4z^2 + 2z}{2x^2-2y^2}$$
\LoesungsBlock{$$\frac{3(2z+1)}{2(x+y)}$$}
\end{bbwAufgabenBlock}
\newpage
%%%%%%%%%%%%%%%%%%%%%%%%%%%%%%%%%%%%%%%%%%%%%%%%%%%%%%%%%%%%%%%%%%%%%%%%%%%%%%%%%%%%
\section{Division}
Dividieren Sie durch Bruchterme bzw. dividieren Sie die Bruchterme:

\begin{bbwAufgabenBlock}
\item $$6 : \frac2{x}$$
\LoesungsBlock{$$3x$$}

\item $$m^2 : \frac4{m}$$
\LoesungsBlock{$$\frac{m^3}4$$}

\item $$\frac{27x^2y^3}{5z^2} : 81x^3y^4$$
\LoesungsBlock{$$\frac1{5xyz^@}$$}

\item $$\frac{-a}{-b} : \frac{-b}{c}$$
\LoesungsBlock{$$\frac{-ac}{b^2}$$}

\item $$\frac{a^2+2ab}{4a^2-4ab+b^2} : \frac{3ab+6b^2}{2a^2-2a-ab+b}$$
\LoesungsBlock{$$\frac{a(a-1)}{3b(2a-b)}$$}

\end{bbwAufgabenBlock}
\newpage
%%%%%%%%%%%%%%%%%%%%%%%%%%%%%%%%%%%%%%%%%%%%%%%%%%%%%%%%%%%%%%%%%%%%%%%%%%%%%%%%%5
\section{Doppelbrüche}
Vereinfachen Sie so weit wie möglich:

\begin{bbwAufgabenBlock}
\item $$\frac{\frac{a}{2b}}{\frac{2a}{3b}}$$
\LoesungsBlock{$$\frac34$$}

\item $$\frac{1}{\frac1f + \frac1g}$$
\LoesungsBlock{$$\frac{fg}{f+g}$$}

\item $$\frac{1}{1-\frac1a}$$
\LoesungsBlock{$$\frac{a}{a-1}$$}

\item $$\frac{x+\frac12}{x-\frac12}$$
\LoesungsBlock{$$\frac{2x+1}{2x-1}$$}

\end{bbwAufgabenBlock}
\newpage
%%%%%%%%%%%%%%%%%%%%%%%%%%%%%%%%%%%%%%%%%%%%%%%%%%%%%%%%%%%%%%%%%%%%%%%%%%%%%%%
\section{Gemischte Aufgaben}
Vereinfachen Sie so weit wie möglich:

\begin{bbwAufgabenBlock}
\item $$2xy \cdot{} \left( \frac{x}{2y} - \frac{y}{2x} \right)$$
\LoesungsBlock{$$(x+y)(x-y)$$}

\item $$\left( \frac1a + \frac{a}b \right) : \frac7{ab}$$
\LoesungsBlock{$$\frac{b+a^2}7$$}
\end{bbwAufgabenBlock}
\newpage

%%%%%%%%%%%%%%%%%%%%%%%%%%%%%%%%%%%%%%%%%%%%%%%%%%%%%%%%%%%%%%%%%%%%%%%%%%%%%%
%%%%%%%%%%%%%%%%%%%%%%%%%%%%%%%%%%%%%%%%%%%%%%%%%%%%%%%%%%%%%%%%%%%%%%%%%%%%%

\part*{III a) GESO Maturaaufgaben / Kompendium}


\section*{GESO Abschlussprüfung 2023}
\subsection*{Serie 1}
GESO Abschlussprüfung 2023 Serie 1 Aufgabe 1:

Vereinfachen Sie so weit wie möglich

$$\frac{2a+4}{a^2-3a-10} : \frac{6a}{3a-15}$$

\LoesungsBlock{$$\frac1a$$}

%%%%%%%%%%%%%%%%%%%%%%%%%%%%%%%%
\section*{GESO Abschlussprüfung 2022}
\subsection*{Serie 1}
GESO Abschlussprüfung 2022 Serie 1 Aufgabe 1:

Vereinfachen Sie so weit wie möglich

$$\left( \frac{a}{a+b} + \frac{b}{a-b} \right) : \frac{a^2+b^2}a$$

\LoesungsBlock{$$\frac{a}{(a+b)(a-b)}$$}



%%%%%%%%%%%%%%%%%%%%%%%%%%%%%%%%
\section*{GESO Abschlussprüfung 2020}
\subsection*{Serie 2}
GESO Abschlussprüfung 2020 Serie 2 Aufgabe 1:

Vereinfachen Sie so weit wie möglich

$$\left( \frac{-x^2-3}{x^2-1} + \frac{x-3}{x+1} \right) \cdot{}  \frac{x+1}{x}$$

\LoesungsBlock{$$\frac{-4}{x-1} $$}



%%%%%%%%%%%%%%%%%%%%%%%%%%%%%%%%
\section*{GESO Abschlussprüfung 2018}
\subsection*{Serie 4}
GESO Abschlussprüfung 2018 Serie 4 Aufgabe 1:

Vereinfachen Sie so weit wie möglich

$$\left( \frac{y}{x} - \frac{x}{y} \right) : \left( \frac1{x}  - \frac1{y}  \right)$$

\LoesungsBlock{$$x+y$$}



%%%%%%%%%%%%%%%%%%%%%%%%%%%%%%%%
\section*{GESO Abschlussprüfung 2018}
\subsection*{Serie 3}
GESO Abschlussprüfung 2018 Serie 3 Aufgabe 1:

Vereinfachen Sie so weit wie möglich

$$\frac{6-3a}{b} : \frac{6a-12}{-a}$$

\LoesungsBlock{$$\frac{a}{2b}$$}



%%%%%%%%%%%%%%%%%%%%%%%%%%%%%%%%
\section*{GESO Abschlussprüfung 2018}
\subsection*{Serie 2}
GESO Abschlussprüfung 2018 Serie 2 Aufgabe 1:

Vereinfachen Sie so weit wie möglich

$$\left(\frac1b - \frac1a \right) : \left( \frac{b-a}{a}  \right)$$

\LoesungsBlock{$$\frac{-1}{b}$$}



%%%%%%%%%%%%%%%%%%%%%%%%%%%%%%%%
\section*{GESO Abschlussprüfung 2018}
\subsection*{Serie 1}
GESO Abschlussprüfung 2018 Serie 1 Aufgabe 1:

Vereinfachen Sie so weit wie möglich

$$\left(1 - \frac{x}{2y} \right) : \frac{4y^2-x^2}{4y^2}$$

\LoesungsBlock{$$\frac{2y}{2y+x}$$}


%%%%%%%%%%%%%%%%%%%%%%%%%%%%%%%%
\section*{GESO Abschlussprüfung 2017}
\subsection*{Serie 1}
GESO Abschlussprüfung 2017 Serie 1 Aufgabe 1:

Vereinfachen Sie so weit wie möglich

$$\frac15\left( a - \frac{b^2}{a} \right) : \frac{a+b}{a}$$

\LoesungsBlock{$$\frac{a-b}5$$}

%%%%%%%%%%%%%%%%%%%%%%%%%%%%%%%%
\section*{GESO Abschlussprüfung 2016}
\subsection*{Serie 1}
GESO Abschlussprüfung 2016 Serie 1 Aufgabe 2:

Fassen Sie zusammen und schreiben Sie das Resultat möglichst einfach:

$$\frac{1.5x}{12-3x} - \frac{2.5x}{x-4}$$

\LoesungsBlock{$$-3x$$}
\newpage
%%%%%%%%%%%%%%%%%%%%%%%%%%%%%%%%%%%%%%%%%%%%%%%%%%%%%%%%%%%%%%%%%%%%%%%%%%%%%%%%%%%%%%%%%%%%%%%
\section*{GESO Kompendium}

Aufgabe 13: Bruchterme Umformen

13. a)
$$\frac{st-3t^2}{s^2-st} + \frac{3s^2-3st-st+t^2}{s^2-2st+t^2}$$

\LoesungsBlock{$$\frac{3(s+t)}s$$}

13. b)

$$\frac{\frac{a^2-16c^2}{8a^2}}{\frac{a-4c}{4a}}$$

\LoesungsBlock{$$\frac{a+4c}{2a}$$}

13. c)

$$\left(\frac1{a^2} - \frac1{b^2} \right) \cdot{} \left( \frac{a}{a+b} + \frac{b}{a-b}\right)$$

\LoesungsBlock{$$\frac{-(a^2+b^2)}{a^2b^2}$$}

13. d)

$$\left(\frac{a+4}a - \frac{b+4}b \right) : \frac{a-b}a$$

\LoesungsBlock{$$\frac{-4}b$$}


\newpage
%%%%%%%%%%%%%%%%%%%%%%%%%%%%%%%%%%%%%%%%%%%%%%%%%%%%%%%%%%%%%%%%%%%%%%%%%%%%%%%%%%
%%%%%%%%%%%%%%%%%%%%%%%%%%%%%%%%%%%%%%%%%%%%%%%%%%%%%%%%%%%%%%%%%%%%%%%%%%%%%%%%%%
\part*{III b) TALS Maturaaufgaben / Strukturaufgaben}

\section*{TALS alte Maturaaufgaben 2023}
\subsection*{2023 GLF / Serie 1 / Teil 1 / Aufgabe 1. a)}
TALS Abschlussprüfung Grundlagenfach (GLF) 2023 Serie 1 Teil 1 Aufgabe
1. a)

Vereinfachen Sie so weit wie möglich:

$$\frac{a^2}{-9}$$

\LoesungsBlock{$$\frac{a-3}{a-7}$$}

\subsection*{1. b)}
TALS Abschlussprüfung Grundlagenfach (GLF) 2023 Serie 1 Teil 1 Aufgabe
1. b)

Vereinfachen Sie so weit wie möglich:

$$\frac{a}{2a-2} - \frac{a+1}{3a-3}$$

\LoesungsBlock{$$\frac{a-2}{6(a-1)}$$}
%%%%%%%%%%%%%%%%%%%%%%%%%%%%%%%%%%%%%%%%%%%%%%%%%%%%%%%%%%%%%%%%%%%%%%%%%%%%%%%%%%
\subsection*{2023 GLF / Serie 1 / Teil 1 / Aufgabe 9.}
TALS Abschlussprüfung Grundlagenfach (GLF) 2023 Serie 1 Teil 1 Aufgabe
9.

Geben Sie den einen richtigen Pfad an. $x,y \in \mathbb{R}, x\ne y$

\noTRAINER{\bbwCenterGraphic{120mm}{img/TALS_GLF_2023_S1_T1_A9.png}}

$$T_1 \Longrightarrow T\LoesungsRaumLen{5mm}{{}_3} \Longrightarrow
T\LoesungsRaumLen{5mm}{{}_7} \Longrightarrow T\LoesungsRaumLen{5mm}{{}_{14}}$$

%%%%%%%%%%%%%%%%%%%%%%%%%%%%%%%%%%%%%%%%%%%%%%%%%%%%%%%%%%%%%%%%%%%%%%%%%%%%%%%%%%
\subsection*{2022 GLF / Serie 1 / Teil 1 / Aufgabe 1.}
TALS Abschlussprüfung Grundlagenfach (GLF) 2022 Serie 1 Teil 1 Aufgabe
1. a) Vereinfachen Sie so weit wie möglich:

$$\frac{a+b}{2a+2} - \ frac{a-b}{3a+3}$$

\LoesungsBlock{$$\frac{a+5b}{6(a+1)}$$}

%%%%%%%%%%%%%%%%%%%%%%%%%%%%%%%%%%%%%%%%%%%%%%%%%%%%%%%%%%%%%%%%55
\subsection*{2021 GLF / Serie 3 / Teil 1 / Aufgabe 1.}

1. a) Vereinfachen Sie so weit wie möglich:

$$\frac{a^2+2a-15}{45-15a}$$
\LoesungsBlock{$$\frac{-(a+5)}{15}$$}



%%%%%%%%%%%%%%%%%%%%%%%%%%%%%%%%%%%%%%%%%%%%%%%%%%%%%%%%%%%%%%%%55
\subsection*{2021 GLF / Serie 3 / Teil 1 / Aufgabe 5.}

5. Vereinfachen Sie so weit wie möglich:

$$\frac{\frac{4a+8}{12+4a} - \frac{a^2+1}{a^2+6a+9}}{\frac{a+1}{a+3}}$$

\LoesungsBlock{$$\frac5{a+3}$$}

%%%%%%%%%%%%%%%%%%%%%%%%%%%%%%%%%%%%%%%%%%%%%%%%%%%%%%%%%%%%%%%%55
\subsection*{2020 GLF / Serie 1 / Teil 1 / Aufgabe 5.}

5. Vereinfachen Sie so weit wie möglich:

$$\frac{y}{1+y} + 5\cdot{} \frac{y^2+5y}{5y^2-5} + \frac{2y+1}{1-y}$$

\LoesungsBlock{$$\frac1{y+1}$$}

%%%%%%%%%%%%%%%%%%%%%%%%%%%%%%%%%%%%%%%%%%%%%%%%%%%%%%%%%%%%%%%%55
\subsection*{2019 GLF / Serie 3 / Teil 1 / Aufgabe 1.}

1. a) Vereinfachen Sie so weit wie möglich:

$$\frac{9a^2-a}{1+3a}$$

\LoesungsBlock{$$a(3a-1)$$}

%%%%%%%%%%%%%%%%%%%%%%%%%%%%%%%%%%%%%%%%%%%%%%%%%%%%%%%%%%%%%%%%55
\subsection*{2019 GLF / Serie 3 / Teil 1 / Aufgabe 5.}

5.) Vereinfachen Sie so weit wie möglich:

$$\frac{5+x}{(x+7)(x-5)} + \frac{3x+35}{10-2x} \cdot{} \frac2{x(7+x)}$$

\LoesungsBlock{$$\frac1x$$}

%%%%%%%%%%%%%%%%%%%%%%%%%%%%%%%%%%%%%%%%%%%%%%%%%%%%%%%%%%%%%%%%55
\subsection*{2019 GLF / Serie 1 / Teil 1 / Aufgabe 1.}

1. b) Vereinfachen Sie so weit wie möglich:

$$\frac{-m^2+2mn-n^2}{3n-3m}$$

\LoesungsBlock{$$\frac{m-n}3$$}


%%%%%%%%%%%%%%%%%%%%%%%%%%%%%%%%%%%%%%%%%%%%%%%%%%%%%%%%%%%%%%%%55
\subsection*{2018 GLF / Serie 3 / Teil 1 / Aufgabe 1.}

1. b) Vereinfachen Sie so weit wie möglich:

$$\frac{m}{2(n-m)} + \frac{m+0.5n}{3(m-n)}$$

\LoesungsBlock{$$\frac{1.0}{6}$$}


%%%%%%%%%%%%%%%%%%%%%%%%%%%%%%%%%%%%%%%%%%%%%%%%%%%%%%%%%%%%%%%%55
\subsection*{2018 GLF / Serie 3 / Teil 1 / Aufgabe 5.}

5. ) Vereinfachen Sie so weit wie möglich:

$$\frac{- \frac{n}{3p} - \frac{2n-3p}{6n}}{\frac{2n-p}{6p} - \frac{p^2-n^2}{2pn+2n^2}}$$

\LoesungsBlock{$$-1$$}

%%%%%%%%%%%%%%%%%%%%%%%%%%%%%%%%%%%%%%%%%%%%%%%%%%%%%%%%%%%%%%%%55
\subsection*{2018 GLF / Serie 1 / Teil 1 / Aufgabe 5.}

5. ) Vereinfachen Sie so weit wie möglich:

$$\frac{x}{x+2} - \frac{x}{2-x} - \frac{x^2-3x+10}{x^2-4}  $$
\LoesungsBlock{$$\frac{x+5}{x+2}$$}

%%%%%%%%%%%%%%%%%%%%%%%%%%%%%%%%%%%%%%%%%%%%%%%%%%%%%%%%%%%%%%%%55
\subsection*{2017 GLF / Serie 3 / Teil 1 / Aufgabe 1.}

1. ) Vereinfachen Sie so weit wie möglich:

$$\frac{\frac{2a}{a-3}-\frac{a}{a+4}}{\frac{a+11}{a^2+a-12}}$$
\LoesungsBlock{$$a$$}

%%%%%%%%%%%%%%%%%%%%%%%%%%%%%%%%%%%%%%%%%%%%%%%%%%%%%%%%%%%%%%%%55
\subsection*{2017 GLF / Serie 1 / Teil 1 / Aufgabe 2.}

2. ) Vereinfachen Sie so weit wie möglich.

$$\frac{\left(\frac1{a+b} - \frac1a \right) \cdot{} \left(\frac{b}a + b\right) \cdot{} \left(a^2-b^2\right)}{\frac{a^2-ab+b^2}{a^2}-1}$$
\LoesungsBlock{$$b(a+1)$$}

%%%%%%%%%%%%%%%%%%%%%%%%%%%%%%%%%%%%%%%%%%%%%%%%%%%%%%%%%%%%%%%%55
\subsection*{2016 GLF / Serie 3 / Teil 1 / Aufgabe 1.}

1. ) Vereinfachen Sie so weit wie möglich.

$$\frac{4p}{4p^2-1} - \frac{\frac2p}{2-\frac1p} + \frac{p+3}{2p^2-5p-3}$$
\LoesungsBlock{$$\frac1{2p+1}$$}


%%%%%%%%%%%%%%%%%%%%%%%%%%%%%%%%%%%%%%%%%%%%%%%%%%%%%%%%%%%%%%%%55
\subsection*{2016 GLF / Serie 1 / Teil 1 / Aufgabe 2.}

2. ) Vereinfachen Sie so weit wie möglich:

$$\frac{1-6x-x^2}{\frac{x}{x-3} - \frac{1-7x}{x^2-4x+3}}$$

\LoesungsBlock{$$-x^2+4x-3$$}
\newpage
%%%%%%%%%%%%%%%%%%%%%%%%%%%%%%%%%%%%%%%%%%%%%%%%%%%%%%%%%%%%%%%%%%%%%
\section*{TALS Strukturaufgaben}

\subsection*{Strukturaufgabe 1. n}

1. n) Vereinfachen Sie so weit wie möglich:

$$\frac{9a^2-16b}{4b-3a}$$

\LoesungsBlock{$$-(3a+4b)$$}

%%%%%%%%%%%%%%%%%%%%%%%%%%%%%%%%5555
\subsection*{Strukturaufgabe 1. o}

1. o) Vereinfachen Sie so weit wie möglich:

$$\frac{r^2-8r+7}{r^2-2r+1}$$

\LoesungsBlock{$$\frac{r-7}{r-1}$$}

%%%%%%%%%%%%%%%%%%%%%%%%%%%%%%%%5555
\subsection*{Strukturaufgabe 1. q}

1. q) Vereinfachen Sie so weit wie möglich:

$$\frac{a}{a^2-ab} - \frac{b}{a^2-b^2}$$

\LoesungsBlock{$$\frac{a}{(a+b)(a-b)}$$}

%%%%%%%%%%%%%%%%%%%%%%%%%%%%%%%%5555
\subsection*{Strukturaufgabe 9. (Typ 1 Bruchrechnen)}

9. ) Vereinfachen Sie so weit wie möglich:

(Dies ist 1:1 die Aufgabe 2. aus der Serie 1 Teil 1 GLF 2016)

$$\frac{1-6x-x^2}{\frac{x}{x-3} - \frac{1-7x}{x^2-4x+3}}$$

\LoesungsBlock{$$-x^2+4x-3$$}


 %%%%%%%%%%%%%%%%%%%%%%%%%%%%%%%%5555
\subsection*{Strukturaufgabe 10. (Typ 1 Bruchrechnen)}

10. ) Vereinfachen Sie so weit wie möglich:

$$\frac{\frac{x+1}{x^2-1} + \frac{x}{x-1}}{\frac{x+x^2}{x^2-2x+1}}$$

\LoesungsBlock{$$\frac{x-1}x$$}

%%%%%%%%%%%%%%%%%%%%%%%%%%%%%%%%5555
\subsection*{Strukturaufgabe 11. (Typ 1 Bruchrechnen)}

11. ) Vereinfachen Sie so weit wie möglich:

$$\frac{5x+25}{x^3-25x} - \frac1{x-5}$$

\LoesungsBlock{$$\frac{-1}x$$}

%%%%%%%%%%%%%%%%%%%%%%%%%%%%%%%%5555
\subsection*{Strukturaufgabe 12. (Typ 1 Bruchrechnen)}

12. ) Vereinfachen Sie so weit wie möglich:

$$\frac{\left( \frac{1-5a}{a^2-5a+4} + \frac{a}{4-a} \right)  \cdot{} \left( 1-\frac4a \right)}{\frac1{1-a^2} - \frac1{4a}}$$

\LoesungsBlock{$$4(a+1)$$}

\newpage
%%%%%%%%%%%%%%%%%%%%%%%%%%%%%%%%%%%%%%%%%%%%%%%%%%%%%%%%%%%%%%%5555
\subsection*{Strukturaufgaben 12.1 (Typ 1 Bruchrechnen)}


Geben Sie den einen richtigen Pfad
an. $x \in \mathbb{R}\backslash \{ -7, -5, -2, 0\}$

\noTRAINER{\bbwCenterGraphic{160mm}{{img/TALS_GLF_Struktur_12.1}.png}}

$$T_1 \Longrightarrow T\LoesungsRaumLen{5mm}{{}_3} \Longrightarrow
T\LoesungsRaumLen{5mm}{{}_6} \Longrightarrow T\LoesungsRaumLen{5mm}{{}_{13}}$$
\newpage
%%%%%%%%%%%%%%%%%%%%%%%%%%%%%%%%%%%%%%%%%%%%%%%%%%%%%%%%%%%%%%%5555
\subsection*{Strukturaufgaben 12.2 (Typ 1 Bruchrechnen)}


Geben Sie den einen richtigen Pfad
an. $a \ne b$

\noTRAINER{\bbwCenterGraphic{160mm}{{img/TALS_GLF_Struktur_12.2}.png}}

$$T_1 \Longrightarrow T\LoesungsRaumLen{5mm}{{}_2} \Longrightarrow
T\LoesungsRaumLen{5mm}{{}_5} \Longrightarrow T\LoesungsRaumLen{5mm}{{}_{11}}$$


\end{document}
