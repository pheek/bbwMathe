%%
%% Meta: TI nSpire Einführung
%%       Ziel: Damit die Grundoperationen damit durchgeführt werden können.
%%             Damit man sich an den Rechner gewöhnt.
%%

\input{bmsLayoutPage}

\newcommand{\LoesungsBlock}[1]{\noTRAINER{\TNTeop{}}\TRAINER{Lösung: #1}}
%%%%%%%%%%%%%%%%%%%%%%%%%%%%%%%%%%%%%%%%%%%%%%%%%%%%%%%%%%%%%%%%%%

\usepackage{amssymb} %% für \blacktriangleright
\renewcommand{\metaHeaderLine}{Arbeitsblatt Bruchrechnen}
\renewcommand{\arbeitsblattTitel}{Zusammenfassung aller Übungen}

\newcommand{\TNTeopS}[1]{\TRAINER{#1}\noTRAINER{\TNTeop{}}}

\begin{document}%%
\arbeitsblattHeader{}

\tableofcontents{}

\newpage


\textbf{Vorgehen}


\begin{enumerate}
\item 
Lösen Sie ca. 10 Aufgaben aus den alten Aufnahmeprüfungen (Kapite I)
Wenn Sie davon gute 50 % lösen können, so gehen Sie zu Kapitel III oder IV, den
alten Abschlussprüfungen (III a) : GESO; III b): TALS)

\item  Haben Sie weniger als ca 50% gelöst, werden die Trainingsaufgaben Kap. II empfohle
Lösen Sie pro Aufgabennummer mindestens je die ersten zwei und die letzte Aufgabe
Wenn Sie mehr Training benötigen, so hat es genügend Übungsmaterial
in den weiteren Aufgabennummern.
\end{enumerate}

\newpage

\part{Aus alten Aufnahmeprüfungen}

Aufnahmeprüfung 2023 Serie e

2. b) Vereinfachen Sie so weit wie möglich:

$$\frac{(x+2)(x-4)}{5} : \frac{x^2 - 16}{10}$$
\LoesungsBlock{$\frac{2\cdot{}(x+2)}{x+4}$}
%%%%%%%%%%%%%%%%%%%%%%%%%%%%%%%%%%%%%%%%%%%%%%%%%%%%%%%%%%%%%%%%%%%%%%

Aufnahmeprüfung 2023 Serie d

2. a) Vereinfachen Sie so weit wie möglich:

$$\frac{x^2-9}{x+3} : \frac{x-3}{4}$$

\TNTeop{$4$}
%%%%%%%%%%%%%%%%%%%%%%%%%%%%%%%%%%%%%%%%%%%%%%%%%%%%%%%%%%%%%%%%%%%%%%%%%%%

\part{Übungsaufgaben}

\part*{III a) GESO Maturaaufgaben / Kompendium}
\part*{III b) TALS Maturaaufgaben / Kompendium}


\end{document}
