\input{bbwLayoutPage}
\renewcommand{\bbwAufgabenBlockID}{A1Be}

\renewcommand{\metaHeaderLine}{Aufgabenblatt}
\renewcommand{\arbeitsblattTitel}{Algebra: Betrag}

\begin{document}
\arbeitsblattHeader{}
Berechnen Sie den Wert des Terms

\begin{bbwAufgabenBlock}
\item $|13-5| = \LoesungsRaumLang{8}$\abplz{2}
\item $|5-13| = \LoesungsRaumLang{8}$\abplz{2}
\item $|-5| - |7| = \LoesungsRaumLang{-2}$\abplz{2}
\item $(-6)\cdot{}|-\frac{1}{3}| = \LoesungsRaumLang{-2}$\abplz{2}
\item $\big|2 - |5-6| - |3-8|\big| = \LoesungsRaumLang{4}$\abplz{2}
\item $\big| |4-8| - 11 \big| = \LoesungsRaumLang{7}$\abplz{2}
\end{bbwAufgabenBlock}

%\platzFuerBerechnungenBisEndeSeite{}
\newpage

%%%%%%%%%%%%%%%%%%%%%%%%%%%%%%%%%%%%%%%%%%%%%%%%%5

Lösen Sie die folgenden Gleichungen. Die Lösungsvariable ist stets $x$.

\begin{bbwAufgabenBlock}
\item $|x|     =  3   \Longrightarrow \lx =\{ \LoesungsRaum{-3; 3\}}$\abplz{4}
\item $|x+2|   =  0   \Longrightarrow \lx =\{ \LoesungsRaum{-2\}}$\abplz{4}
\item $|x+3|   =  7   \Longrightarrow \lx =\{ \LoesungsRaum{-10;4\}}$\abplz{4}
\item $|6-x|   =  9   \Longrightarrow \lx =\{ \LoesungsRaum{-3;15\}}$\abplz{4.8}\newpage
\item $|6-x|   = -5   \Longrightarrow \lx =\{ \LoesungsRaum{\}}$\abplz{4}
\item $|2-x|+3 = 18   \Longrightarrow \lx =\{ \LoesungsRaum{-13;17\}}$\abplz{4}
\item $|x+7|-5 = -6   \Longrightarrow \lx =\{ \LoesungsRaum{\}}$\abplz{4}
\item $|x|     = 11.4 \Longrightarrow \lx =\{ \LoesungsRaumLang{-11.4; 11.4\}}$\abplz{4}
\item $|x-3|   =  7   \Longrightarrow \lx =\{ \LoesungsRaumLang{-4; 10\}}$\abplz{3.2}

\end{bbwAufgabenBlock}

%%\platzFuerBerechnungenBisEndeSeite{}
\newpage
%% Weitere Seite: Challenge

\textbf{\bbwAufgabenNummer{}. Challenge:}

Berechnen Sie $-\sqrt{(-37)^2} + |-37| = \LoesungsRaum{-17+17= 0}$

\vspace{22mm}

Für welche $x\in\mathbb{R}$ gilt folgendes?
$$|-x| = -x$$
\TRAINER{$$\lx = ]-\infty; 0] = \mathbb{R}^{-}_0$$ mit anderen Worten: Die Gleichung ist
für alle negativen Zahlen inkl. Null gültig.}

\platzFuerBerechnungenBisEndeSeite{}%
\end{document}%
