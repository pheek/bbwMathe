%%
%% Meta: Arbeitsblatt zu Logarithmen
%%       Wie konnte man vor 200 Jahren komplizierte Multiplikationen
%%       ausführen.


\input{bmsLayoutPage}

%%%%%%%%%%%%%%%%%%%%%%%%%%%%%%%%%%%%%%%%%%%%%%%%%%%%%%%%%%%%%%%%%%

\renewcommand{\metaHeaderLine}{Logarithmen}
\renewcommand{\arbeitsblattTitel}{Multiplikation durch Addition}

\begin{document}%%
\arbeitsblattHeader{}

\section*{Auszug vierstellige Logarithmentabelle}

\begin{tabular}{|c|c|| c | c |}\hline
Numerus & Logarithmus & Numerus & Logarithmus\\\hline
0.1   & - 1        &   &  \\\hline
1     & 0          &   &  \\\hline
1.001 & 0.0004341  & 3.570 & 0.5527\\
1.002 & 0.0008677  & 3.571 & 0.5528\\
1.003 & 0.0013009  & 3.572 & 0.5529\\
...   & ...        & ...   & ... \\
1.155 & 0.06258    & 4.873 & 0.6878\\
1.156 & 0.06296    & 4.874 & 0.6879\\
1.157 & 0.06333    & 4.875 & 0.6880\\
...   & ...        & ...   & ...  \\
1.739 & 0.2403     & 6.638 & 0.8220\\
1.740 & 0.2405     & 6.639 & 0.8221\\
1.741 & 0.2408     & 6.640 & 0.8222\\
...   & ...        & ...   & ....\\
2.010 & 0.3032     & 9.997 & 0.99987\\
2.011 & 0.3034     & 9.998 & 0.99991\\
2.012 & 0.3036     & 9.999 & 1.0000\\\hline
      &            & 10    & 1\\\hline
\end{tabular}
 
\section*{Auftrag}
a) Multiplizieren Sie 1.156 mit 1.740 indem Sie deren Logarithmen
addieren und danach die Summe wieder in der Logarithmentabelle
nachschlagen:

$$a\cdot{}b = 10^{\log(a\cdot{}b)} = 10^{\log(a) + \log(b) } $$

\TNTeop{$$\log(1.156) + \log(1.740) = 0.06296 + 0.2405 = 0.30346$$
Nachschlagen bei «Logarithmus»: Das Produkt liegt zwischen $2.011$ und
$2.012$ (exakt $2.01144$) 
}

b) (challenge)
Multiplizieren Sie die drei Zahlen 4.873, 3.571 und 0.6640
miteinander, indem Sie die Logarithmen addieren.

\TNT{3.2}{$0.6878+0.5528+(0.8222-1) = 1.0628$

In der Tabelle 0.0628 suchen (da 1.0628 nicht vorkommt).

Das Resultat liegt zwischen $10\cdot1.155 = 11.55$ und 
$10\cdot1.156 = 11.56$

}%% END TNT
\end{document}
