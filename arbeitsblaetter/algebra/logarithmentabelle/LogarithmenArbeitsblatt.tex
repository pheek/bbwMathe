%%
%% Meta: Arbeitsblatt zu Logarithmen
%%       Wie konnte man vor 200 Jahren komplizierte Multiplikationen
%%       ausführen.


\input{bmsLayoutPage}

%%%%%%%%%%%%%%%%%%%%%%%%%%%%%%%%%%%%%%%%%%%%%%%%%%%%%%%%%%%%%%%%%%

\renewcommand{\metaHeaderLine}{Logarithmen}
\renewcommand{\arbeitsblattTitel}{Multiplikation durch Addition}

\begin{document}%%
\arbeitsblattHeader{}

\section*{Auszug vereinfachte Logarithmentabelle}

\begin{tabular}{|c|c|}\hline
Numerus & Logarithmus\\\hline
0.1   & - 1      \\\hline
1     & 0        \\\hline
1.001 & 0.0004341\\
1.002 & 0.0008677\\
...   & ...      \\
1.155 & 0.06258  \\
1.156 & 0.06296  \\
1.157 & 0.06333  \\
...   & ...      \\
1.739 & 0.2403   \\
1.740 & 0.2405   \\
1.741 & 0.2408   \\
...   & ...      \\
3.570 & 0.5527   \\
3.571 & 0.5528   \\
3.572 & 0.5529   \\
...   & ...      \\
4.873 & 0.6878   \\
4.874 & 0.6879   \\
4.875 & 0.6880   \\
...   & ...      \\
6.638 & 0.8220   \\
6.639 & 0.8221   \\
6.640 & 0.8222   \\
...   & ...      \\
9.998 & 0.9999   \\
9.999 & 1.000    \\\hline
10    & 1        \\\hline
\end{tabular}
 
\section*{Auftrag}
Multiplizieren Sie die drei Zahlen 4.873, 3.571 und 0.6640
miteinander, indem Sie die Logarithmen addieren.

\TNT{3.2}{$0.6878+0.5528+(0.8222-1) = 1.0628$

In der Tabelle 0.0628 suchen.

Das Resultat liegt zwischen $10\cdot1.155 = 11.55$ und 
$10\cdot1.156 = 11.56$

}%% END TNT
\end{document}
