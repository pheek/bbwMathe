\input{bmsLayoutPage}
\renewcommand{\bbwAufgabenBlockID}{A1rnd}

%%%%%%%%%%%%%%%%%%%%%%%%%%%%%%%%%%%%%%%%%%%%%%%%%%%%%%%%%%%%%%%%%%

\usepackage{amssymb} %% für \blacktriangleright
\renewcommand{\metaHeaderLine}{Runden}
\renewcommand{\arbeitsblattTitel}{Rationale Zahlen: Runden}

\begin{document}%%
\arbeitsblattHeader{}

\begin{center}V 0.1 2023 Dez. 16.\end{center}


\section{Motivation}

a) Ein Quadrat hat eine Seitenlänge von 292\,000. Wie lange ist die
Diagonale? Runden Sie sinnvoll.

\TNT{2}{Falls mit 6 signifikanten Stellen gerechnet: 412\,950}

b) Ein Quadrat hat eine Seitenlänge von 0.2920  Wie lange ist die
Diagonale? Runden Sie sinnvoll.

\TNT{1.6}{Falls mit 4 signifikanten Stellen gerechnet: 0.4130 (die
...29... wird auf 30 aufgerundet.)}

c) 
Die Zahlen 0.2920 und 292\,000 sind in der Messgröße «Länge» identisch! Nur in einer anderen
Maßeinheit angegeben. Daher müsste aber auch
das Runden der berechneten Diagonalen identisch ausfallen:

$$292\,000 [mm] = 0.2920 [km]$$

Was wäre nun «sinnvolles Runden»? Und was ist bei den beiden
vorangehenden Aufgaben schief gelaufen?

\TNT{3.2}{Nur der Zahl 0.2920 können wir die Signifikanz der 4. Stelle
, also der nachfolgenden 0 ansehen. Bei 292\,000 müssen wir mangels
weiterer Information allen 6 abgebildeten Stellen «glauben». Hier
fehlt die Angabe der Signifikanz.}

d) Berechnen Sie nun die Diagonale eines Quadrates mit der Seitenlänge
$2.920\cdot{}10^{-1}$ [km] und runden Sie nicht weniger und nicht
mehr, als die Genauigkeit der vorgegebenen Zahl angibt:

\TNT{2.4}{Die Diagonal misst 0.4130. Es sind vier Stellen signifikant:
die 4, die 1, die 3 und die nachfolgende 0.}


e) Berechnen Sie nun die Diagonale eines Quadrates mit der Seitenlänge
$2.920\cdot{}10^{5}$ [mm] und runden Sie nicht weniger und nicht
mehr, als die Genauigkeit der vorgegebenen Zahl angibt:

\TNTeop{Die Diagonal misst $4.130 \cdot{}10^5$ [mm]. Es handelt sich
um exakt die selben signifikanten Stellen, wie in der vorangehenden
Aufgabe.}%% end TNT


\newpage
\section{Aufgaben}

Runden Sie auf zwei Dezimalen (= Nachkommastellen):
\begin{bbwAufgabenBlock}
\item $33.333$ \LoesungsRaumLang{$33.33$}
\item $45.6789$ \LoesungsRaumLang{$45.68$}
\item $78.3976\%$ \LoesungsRaumLang{$78.40\%$}
\item $0.0149 \textrm{ m}$ \LoesungsRaumLang{$0.01 \textrm{ m}$}
\item $-6.9999$ \LoesungsRaumLang{$-7.00$}
\item $99.99492\%$ \LoesungsRaumLang{$99.99\%$}
\item $0.0049278 \textrm{ cm}$ \LoesungsRaumLang{$0.00 \textrm{ cm}$}
\item $0.0049278 \textrm{ cm} = 0.049278 \textrm{ mm}$ \LoesungsRaumLang{$0.05 \textrm{ mm}$}
\item $459.99499  \textrm{ kg}$ \LoesungsRaumLang{$459.99\textrm{ kg}$}
\item $459.995001 \textrm{ kg}$ \LoesungsRaumLang{$460.00\textrm{
kg}$}
\item $\frac{\sqrt3}{3}$ \LoesungsRaumLang{$0.58$}
\end{bbwAufgabenBlock}

\platzFuerBerechnungenBisEndeSeite{}
%%\TRAINER{\newpage}



Geben Sie jeweils die erste und die dritte gültige Stelle (=
signifikante Ziffer) an:

\begin{bbwAufgabenBlock}
\item $0.700324$ \LoesungsRaumLang{$7$ (= 1. sign. Stelle) und $0$ (= 3. sign. Stelle)}
\item $100.0386$ \LoesungsRaumLang{$1$ (= 1. sign. Stelle) und $0$ (= 3. sign. Stelle)}
\item $\frac73$ \LoesungsRaumLang{$2$ (= 1. sign. Stelle) und $3$ (= 3. sign. Stelle)}
\item $\frac{2.828}{\sqrt{2}}$ \LoesungsRaumLang{$1$ (= 1. sign. Stelle) und $9$ (= 3. sign. Stelle)}
\item $1+2+3+...+7+8$ \LoesungsRaumLang{$3$ (= 1. sign. Stelle) und $0$ (= 3. sign. Stelle)}
\end{bbwAufgabenBlock}

\platzFuerBerechnungenBisEndeSeite{}
\TRAINER{\newpage}

Die folgenden Zahlen sind (fast alle) in wissenschaftlicher Notation. Schreiben
Sie die Zahlen ohne Zehnerpotenzen.

\begin{bbwAufgabenBlock}
\item $3.847 \cdot{} 10^3$ \LoesungsRaumLang{$3847$}
\item $4.003\cdot{} 10^2$ \LoesungsRaumLang{$400.3$}
\item $0.07\cdot{}10^3$ \LoesungsRaumLang{$70$}
\item $7.14\cdot{}10^{-2}$\LoesungsRaumLang{$0.0714$}
\item $1.2345\cdot{}10^{-6}$ \LoesungsRaumLang{$0.0000012345$}
\item $6.7\cdot{}10^0$ \LoesungsRaumLang{$6.7$}
\end{bbwAufgabenBlock}

\platzFuerBerechnungenBisEndeSeite{}


Schreiben Sie in wissenschaftlicher Notation

\begin{bbwAufgabenBlock}
\item $144$ \LoesungsRaumLang{$1.44\cdot{}10^2$}
\item $12004$ \LoesungsRaumLang{$1.2004\cdot{}10^{4}$}
\item $12.3$ \LoesungsRaumLang{$1.23\cdot{} 10^{1}$}
\item $0.02$ \LoesungsRaumLang{$2\cdot{} 10^{-2}$}
\item $5^8$ \LoesungsRaumLang{$=390625= 3.90625\cdot{} 10^{5}$}
\item $7.1609$ \LoesungsRaumLang{$7.1609\cdot{} 10^{0}$}
\item $\sqrt{\sqrt{4096}}$ \LoesungsRaumLang{$8.\cdot{} 10^{0}$}
\end{bbwAufgabenBlock}

\platzFuerBerechnungenBisEndeSeite{}
%%\TRAINER{\newpage}

Interpretieren Sie Resultate vom Taschenrechner ($\e$-Schreibweise)

\begin{bbwAufgabenBlock}
\item $3.87\e7$ \LoesungsRaumLang{$38\,700\,000$}
\item $-7.653\e-2$ \LoesungsRaumLang{$-0.07653$}
\item $\pi^{30}$ \LoesungsRaumLang{$821',289\,330\,403\,000 \approx 821$ Billionen}
\end{bbwAufgabenBlock}


\aufgabenFarbe{Runden Sie die Zahl 3.21459 auf zwei Dezimalen: \LoesungsRaumLang{3.21}. Runden Sie die selbe Zahl 3.21459 auf drei Dezimalen: \LoesungsRaumLang{3.215} und runden Sie nun das gerundete Resultat auf zwei Dezimalen: \LoesungsRaumLang{3.22}. Was ist davon zu halten?}%% END Aufgabenfarbe

\TRAINER{Betrachten Sie beim Runden auf die $n$-te Stelle \textbf{nur}
die $n+1.$-Stelle. Beginnen Sie beim Runden nicht von ganz
rechts. 0.495 wird abgerundet, wie auch 0.499. Erst 0.5 wird aufgerundet!}

\platzFuerBerechnungenBisEndeSeite{}
\TRAINER{\newpage}


Geben Sie drei signifikante Stellen an. Dabei können
Rundungsfehler entstehen. Runden Sie bei dieser Aufgabe jedoch \textbf{nicht}.

\begin{bbwAufgabenBlock}
\item $4.392$ \LoesungsRaumLang{$4.39$}
\item $87.99$ \LoesungsRaumLang{$87.9$}
\item $0.014$ \LoesungsRaumLang{$0.0140$}
\item $11$ \LoesungsRaumLang{$11.0$}
\item $100.0001$ \LoesungsRaumLang{$100.$}
\item $0.002$ \LoesungsRaumLang{$0.00200$}
\end{bbwAufgabenBlock}

\platzFuerBerechnungenBisEndeSeite{}
%%\TRAINER{\newpage}

Runden Sie auf zwei signifikante Stellen

\begin{bbwAufgabenBlock}
\item $0.785$ \LoesungsRaumLang{$0.79$}
\item $3.049$ \LoesungsRaumLang{$3.0$}
\item $78.94$ \LoesungsRaumLang{$79.$}
\item $4.21\cdot{}\sqrt{2}$ \LoesungsRaumLang{$6.0$}
\item $\frac{253\cdot{}\pi}{10}$ \LoesungsRaumLang{$79.$}
\item $3\cdot{}15\,167$ \LoesungsRaumLang{$46\,000$}
\end{bbwAufgabenBlock}

\platzFuerBerechnungenBisEndeSeite{}
%%\TRAINER{\newpage}

Runden Sie auf vier signifikante Stellen und geben Sie das Resultat in
wissenschaftlicher Notation an:

\begin{bbwAufgabenBlock}
\item $123.123$ \LoesungsRaumLang{$1.231\cdot{} 10^{2}$}
\item $2000.762$ \LoesungsRaumLang{$2.001\cdot{} 10^{3}$}
\item $78.9262$ \LoesungsRaumLang{$7.893\cdot{} 10^{1}$}
\item $\pi^2$ \LoesungsRaumLang{$9.870\cdot{} 10^{0}$}
\item $\sqrt{7^2+8^2}$ \LoesungsRaumLang{$1.063\cdot{} 10^{1}$}
\end{bbwAufgabenBlock}

\platzFuerBerechnungenBisEndeSeite{}
\TRAINER{\newpage}



\end{document}
