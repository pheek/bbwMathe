\input{bmsLayoutPage}
\renewcommand{\bbwAufgabenBlockID}{A1ordn}

%%%%%%%%%%%%%%%%%%%%%%%%%%%%%%%%%%%%%%%%%%%%%%%%%%%%%%%%%%%%%%%%%%

\usepackage{amssymb} %% für \blacktriangleright
\renewcommand{\metaHeaderLine}{Ordnungsrelationen}
\renewcommand{\arbeitsblattTitel}{Arbeitsblatt}

\begin{document}%%
\arbeitsblattHeader{}


\section{Welches Zeichen?}
Setzen Sie eines der Zeichen $<$, $>$ bzw. $=$ zwischen die Zahlen:

\begin{bbwAufgabenBlock}
\item $\pi \LoesungsRaum{<} \sqrt{10}$ ($\pi$ ist die Kreiszahl.)

\item $-\sqrt{5} \LoesungsRaum{<} \sqrt{2}-\sqrt{13}$

\item $\e \LoesungsRaum{>} \frac{163}{60}$ ($\e$ ist die Eulersche Zahl.)

\item $\frac{10}3 \LoesungsRaum{>} 0.3333333333$


\item $8.\overline{9} \LoesungsRaum{=} 18.33 - \frac{933}{100}$

\item $-\sqrt{6} \LoesungsRaum{\ne} \sqrt{-6}$ \TRAINER{(Fangfrage, denn $\sqrt{-6}$ liegt außerhalb $\mathbb{R}.$)}

\end{bbwAufgabenBlock}

\platzFuerBerechnungenBisEndeSeite{}
%%\TRAINER{\newpage}


\section{Welche Aussagen sind wahr?}
Geben Sie an, ob es sich um eine wahre Aussage handelt:

\begin{bbwAufgabenBlock}

\item
$\pi > \e$ \TRAINER{Wahr}

\item
$\pi = 3.14$ \TRAINER{Falsch, zwar nur 1/2 Promille daneben, aber
  nicht gleich}

\item
$\pi = 3.14159265$ \TRAINER{Falsch, auch wenn schon besser}

\item
$\pi \approx 3.14259265$ \TRAINER{Falsch, falsche Genauigkeit wird vorgegaukelt}

\item
$\e \approx 2.7183$ \TRAINER{Wahr, wenn auch nur gerundet}

\item
$-\sqrt{3} < 0 - \sqrt{2}$ \TRAINER{Wahr}

\item
$\sqrt{\frac{8^8}{\e\cdot{}\pi}} > \sqrt{1.9646 \cdot{} 10^6}$ \TRAINER{Wahr}

\item
$0.1\overline{6} < \frac16$ \TRAINER{falsch, die sind gleich}

\end{bbwAufgabenBlock}

\platzFuerBerechnungenBisEndeSeite{}
\TRAINER{\newpage}




\section{Ordnen}
Ordnen Sie die folgenden Zahlen aufsteigend:

$\frac{10}3$; $\sqrt[6]{1073}$; $\e + \frac{17}{37}$; $\pi$;
$(-2.2)\cdot{}(-\frac32)$; $\sqrt{3^3}-2$; $\sqrt{10}$ 

\platzFuerBerechnungenBisEndeSeite{}


\section{Intervallschreibweise}
Im folgenden werden Anforderungen an die Variable $z$ definiert.
Markieren Sie die möglichen $z$-Werte auf einem Zahlenstrahl und geben
Sie diese Menge in der Intervallschreibweise an.


\begin{bbwAufgabenBlock}
\item $z<3.5$ \LoesungsRaumLang{$]-\infty; 3.5[$}

\item $z\ge -4$ und $1>z$ \LoesungsRaumLang{$[-4;1[$}

\item $z<8$ und $z\le -2$ \LoesungsRaumLang{$]-\infty;-2]$}

\item $2z<6$ oder $z\ge 4$ \LoesungsRaumLang{$]-\infty;3[ \cup  [4;\infty[$}

\end{bbwAufgabenBlock}


\platzFuerBerechnungenBisEndeSeite{}




\section{Mengenoperationen}
Zeichnen Sie die Mengen $A = [-4 ; 2.5[$ und $B = [-1.5; 3.5]$
je auf einen eigenen Zahlenstrahl übereinander.

Zeichnen Sie zu den folgenden Aufgaben die neue, gesuchte Menge auch auf
einen Zahlenstrahl und geben Sie die Lösung in der Intervallschreibweise an.


\begin{bbwAufgabenBlock}
\item Welche Zahlen gehören zu $A$ oder zu $B$ oder zu beiden Mengen?
(Vereinigungsmenge: $A\cup B$) \LoesungsRaumLang{$[-4;3.5]$}

\item Welche Zahlen gehören sowohl zu $A$, wie auch zu $B$?
(Schnittmenge $A\cap B$) \LoesungsRaumLang{$[-1.5; 2.5[$}

\item Welche Zahlen gehören zu $A$, aber nicht zu $B$?
(Differenzmenge $A\backslash B$) \LoesungsRaumLang{$[-4;-1.5[$}

\item Welche Zahlen gehören weder zu $A$, noch zu
$B$? (Komplementärmenge $\overline{(A\cup B)}$) \LoesungsRaumLang{$]-\infty; -4[ \cup ]3.5;\infty[$}

\item Welche Zahlen gehören entweder zu $A$ oder zu $B$, nicht aber zu
beiden? (Symmetrische Differenz $A\bigtriangleup B$) \LoesungsRaumLang{$[-4;-1.5[ \cup [2.5;3.5]$}

\end{bbwAufgabenBlock}


\platzFuerBerechnungenBisEndeSeite{}


\end{document}
