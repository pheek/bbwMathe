\input{bbwLayoutPage}
\renewcommand{\bbwAufgabenBlockID}{A1Te}

\renewcommand{\metaHeaderLine}{Aufgabenblatt}
\renewcommand{\arbeitsblattTitel}{Algebra: Terme}


\begin{document}
\arbeitsblattHeader{}
\section{Terme benennen}
Benennen Sie die folgenden Terme (Summe, Differenz, Produkt, Quotient,
Potenz (Wurzelterm)):

\begin{bbwAufgabenBlock}
\item $ax+2b$ \, \LoesungsRaum{Summe}\abplz{1.2}
\item $\frac{5+x}{5x}$ \, \LoesungsRaum{Quotient}\abplz{1.2}
\item $\sqrt{2x^3+5}$ \, \LoesungsRaum{Wurzelterm}\abplz{1.2}
\item $(a+b)^{c+d}$ \, \LoesungsRaum{Potenz}\abplz{1.2}
\item $(5-3y)c^8$ \, \LoesungsRaum{Produkt}\abplz{1.2}
\item $(\sqrt{x-3}+\sqrt{8-b})^2$ \, \LoesungsRaum{Potenz}\abplz{1.2}
\item $2a^{3-y}$ \, \LoesungsRaum{Produkt}\abplz{1.2}\newpage
\item $\frac15 - 2\cdot{}l^{x+1}$ \, \LoesungsRaum{Differenz}\abplz{1.2}
\item $(5s-t)^4+4$ \, \LoesungsRaum{Summe}\abplz{1.2}
\item $100:(2\cdot{}\sqrt{25})$ \, \LoesungsRaum{Quotient}\abplz{1.2}
\end{bbwAufgabenBlock}

%%\platzFuerBerechnungenBisEndeSeite{}
\newpage

%%%%%%%%%%%%%%%%%%%%%%%%%%%%%%%%%%%%%%%%%%%%%%%%%

\section{Werte von Zahltermen}
Berechnen Sie die Werte der folgenden Zahlteme und kontrollieren Sie
mit dem Taschenrechner:

\begin{bbwAufgabenBlock}
\item $-10^4$ $=$ $\LoesungsRaum{-10\,000}$\abplz{2}
\item $(-10)^5$ $=$ $\LoesungsRaum{-100\,000}$\abplz{2}
\item $(-100)^2$ $=$ $\LoesungsRaum{10\,000}$\abplz{2}
\item $-10^3$ $=$ $\LoesungsRaum{-1000}$\abplz{2}
\item $(-1)^{2n}$  (mit $n\in\mathbb{N}$)  $=$ $\LoesungsRaum{}$\abplz{2}
\end{bbwAufgabenBlock}

%%\platzFuerBerechnungenBisEndeSeite{}
\newpage

%%%%%%%%%%%%%%%%%%%%%%%%%%%%%%%%%%%%%%%%%%%%%%%%%%%%%%

\section{Zahlen einsetzen}
Setzen Sie die gegebenen Zahlen in die Terme ein und berechnen Sie die
Werte der Terme:

\begin{bbwAufgabenBlock}
\item $x^6$ für $x=-1$ $\LoesungsRaumLang{+1}$\abplz{2}
\item $-x^5$ für $x=-10$ $\LoesungsRaumLang{+100\,000}$\abplz{2}
\item $(-r)^3$ für $r=-2$ $\LoesungsRaumLang{8}$\abplz{2}
\end{bbwAufgabenBlock}
%%\platzFuerBerechnungenBisEndeSeite{}
\newpage

%%%%%%%%%%%%%%%%%%%%%%%%%%%%%%%%%%%%%%%%%%%%%%%%%%%%%%%%%%%
\section{Terme mit Namen}
Berechnen Sie die Werte der folgenden Terme:

\begin{bbwAufgabenBlock}
\item $A(r)=r^2\pi$ berechnen Sie $A(4)= \LoesungsRaumLang{16\pi\approx{}50.27}$\abplz{4}
\item $T(x)=-x^2\cdot{}x^1$ berechnen Sie $T(3)= \LoesungsRaumLang{-27}$\abplz{4}
\item $f(t)=-t^2\cdot{}t^3$ berechnen Sie $f(-2)= \LoesungsRaumLang{32}$\abplz{4}
\item $D(a;b;c)=b^2-4ac$ berechnen Sie $D(1;-2;-3)= \LoesungsRaumLang{16}$\abplz{4}\newpage
\item $T(b)=(c^2-b)^2$ berechnen Sie $T(c^2-x)= \LoesungsRaumLang{x^2}$\abplz{4}
\item $T(x)=(-2x)^2$ berechnen Sie $T(a-3)= \LoesungsRaumLang{(-2a+6)^2-2a=4a^2-26a+36}$\abplz{4}
\item $L(a,b,c)=\frac{-b+\sqrt{b^2-4ac}}{2a}$ berechnen Sie $L(4, -28, 49)= \LoesungsRaumLang{3.5}$\abplz{4}
\end{bbwAufgabenBlock}
\newpage%%\platzFuerBerechnungenBisEndeSeite{}

%%%%%%%%%%%%%%%%%%%%%%%%%%%%%%%%%%%%%%%%%%%%%%%%%%%%%%%%%%%
\section{Tabelle mit Termen}
\nextBbwAufgabenNummer{}

Füllen Sie die folgende Tabelle aus.


\begin{bbwFillInTabular}{|c||c|c|r|c|c|}
          & {\color{cyan}a)}        & {\color{cyan}b)}     &  {\color{cyan}c)}   & {\color{cyan}d)}\\\hline
  $x$     & 5                       & 3                    &  -1                 & 0\\\hline
  $s$     & 2                       & $\frac12$            &   0                 & -3\\\hline
  $t$     & 0                       & 2                    &  -1                 & 1\\\hline\hline
  $x^2$   & 25                      & \LoesungsRaum{9 }    &  \LoesungsRaum{1}   & \LoesungsRaum{0}\\\hline
  $x^3$   & \LoesungsRaum{125}      & \LoesungsRaum{27}    &  \LoesungsRaum{-1}  & \LoesungsRaum{0}\\\hline
  $-x^2$  & \LoesungsRaum{-25}      & \LoesungsRaum{-9 }   &  \LoesungsRaum{-1}  & \LoesungsRaum{0}\\\hline
  $s-t^2$ & \LoesungsRaum{2}        & \LoesungsRaum{-3.5}  &  \LoesungsRaum{-1}  & \LoesungsRaum{-4}\\\hline
  $10x^2$ & \LoesungsRaum{250}      & \LoesungsRaum{90}    &  \LoesungsRaum{10}  & \LoesungsRaum{0}\\\hline
  $\left(-\frac{t}{-s}\right)$
          & \LoesungsRaum{0}       & \LoesungsRaum{4}      &  \LoesungsRaum{geht nicht}  & \LoesungsRaum{$-\frac13$}\\\hline  
\end{bbwFillInTabular}

\platzFuerBerechnungenBisEndeSeite{}
%%%%%%%%%%%%%%%%%%%%%%%%%%%%%%%%%%%%%%%%%%%%%%%%%%%%%%%%%%%%%%%%%%%%%


\section{Geometrie}
Gegeben ist das folgende aus zwei Quadern zusammengesetzte Konstrukt.


\bbwCenterGraphic{8cm}{img/kasten.png}

Für das Volumen $V$ gilt: $V(x) = 66x^3$

\begin{bbwAufgabenBlock}
\item Berechnen Sie das Volumen für $x=7\textrm{ mm}$ $V(7 \textrm{
mm})  =  \LoesungsRaum{22638 \textrm{ mm}^3 \approx 23 \textrm{ cm}^3}$

\item Geben Sie einen Term (eine Formel) für die Oberfläche $S$ an:

$$S(x) = \LoesungsRaumLang{110x^2}$$

\item
Berechnen Sie die Oberfläche für $x=7 \textrm{ mm}$
$$S(7 \textrm{ mm}) = \LoesungsRaumLang{5390 \textrm{ mm}^2 \approx 54 \textrm{ cm}^2}$$
\end{bbwAufgabenBlock}
\platzFuerBerechnungenBisEndeSeite{}


%%%%%%%%%%%%%%%%%%%%%%%%%%%%%%%%%%%%%%%%%%%%%%%%%%%%%%%%%%%

\end{document}
