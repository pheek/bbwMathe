%%
%% Meta: TI nSpire Einführung
%%       Ziel: Damit die Grundoperationen damit durchgeführt werden können.
%%             Damit man sich an den Rechner gewöhnt.
%%

\input{bmsLayoutPage}

%%%%%%%%%%%%%%%%%%%%%%%%%%%%%%%%%%%%%%%%%%%%%%%%%%%%%%%%%%%%%%%%%%

\usepackage{amssymb} %% für \blacktriangleright
\renewcommand{\metaHeaderLine}{PC-Arbeitsblatt}
\renewcommand{\arbeitsblattTitel}{Arbeiten mit einem Formeleditor}

\begin{document}%%
\arbeitsblattHeader{}
\TRAINER{Benötigt ca. eine Lektion. Starte mit einem leeren Dokument.}
Erstellen Sie die folgenden Formeln mit einem Formeleditor.


\section{Einfache Formeln}

\begin{enumerate}
\item $a+x=5$\\ Beachten Sie die Variablen $a$ und $x$: Sie sind in
anderer Schrift und kursiv!\\Beispiel: 5 kg à $a$ Franken; oder das
x-fache von $x$.\TRAINER{ Variante 1 vorzeigen: Zuerst alles tippen, danach a+x=5
anwählen und mit «Einfügen» «Formel» in eine Formel verwandeln (=CTRL/SHIFT/*).}
\item $\frac{a}{b}$\TRAINER{ Variante 2 auch noch vorzeigen: nichts
anwählen und nun «Einfügen» «Formel», danach \zB Bruch wählen.}
\item $\sqrt[4]{2b}$
\item $x \in \mathbb{R}$\TRAINER{Unicode 0x211D = \&\#8477. Es gibt
mehrere Varianten Symbole einzufügen. Copy Paste von bestehendem
Dokument, Einfügen Symbol, Einfügen Formel und dann auf Symbole,
Einfügen Sonderzeichen, Einfügen Unicode...}
\item $\mediantilde{x}\ne \overline{x}$\TRAINER{Math Symbols Akzente suchen}
\end{enumerate}


\section{Zusammengesetzte Formeln}

\begin{enumerate}
\item $a+x=\frac{87\cdot(-a)}{\sqrt{x}}$
\item $\sqrt{2}x^2 - \sqrt{3}x = 0$
\item $\frac{-b \pm \sqrt{b^2-4ac}}{2a}$
\item
$\left(r-\frac{s}{1+x}\right)(p+q)=\left(r+\frac{s}{1+x}\right)(p-q)$
mit $p, q \in \mathbb{Q}$\TRAINER{ $\mathbb{Q}$ hat Unicode 211A}
\item Bei $0^{\circ}C$ gilt $\varphi/$Fahrenheit($^\circ{}F$) =
32\TRAINER{ $C$ muss nicht kursiv sein. ${}^\circ$ = ASCII B0.}
\item $f_{12}(n) = K_0\left(1+\frac{p}{100}\right)^n$
\item $P_B(A)=P(A|B) := \frac{P(A\cap{}B)}{P(B)}
= \frac{\frac{|A\cap{}B|}{|\Omega|}}{\frac{|B|}{|\Omega|}} = \frac{|A\cap{}B|}{|B|}$
\item Für die Kondensatorspannung gilt $U(t) = U_0\left(1-\e^{\frac{-t}{R\cdot{}C}}\right)$
\end{enumerate}


\end{document}
