%%
%% Meta: TI nSpire Einführung
%%       Ziel: Damit die Grundoperationen damit durchgeführt werden können.
%%             Damit man sich an den Rechner gewöhnt.
%%

\input{bmsLayoutPage}

%%%%%%%%%%%%%%%%%%%%%%%%%%%%%%%%%%%%%%%%%%%%%%%%%%%%%%%%%%%%%%%%%%

\usepackage{amssymb} %% für \blacktriangleright
\renewcommand{\metaHeaderLine}{Geradengleichung}
\renewcommand{\arbeitsblattTitel}{Vektorgeometrie in $\mathbb{R}^3$}

\begin{document}%%
\arbeitsblattHeader{}

\newcounter{aufgabennummer}
\setcounter{aufgabennummer}{1}


\newcommand\aufgabeNr[1]{\begin{samepage}%%
\subsection*{Aufgabe \arabic{aufgabennummer}}\,\,\\
#1
%%\abplz{3.2}
\end{samepage}%%
\stepcounter{aufgabennummer}%%
}%%


\section{Parameterdarstellung der Geradengleichung}

 Es bezeichne jeweils $g: \vec{r} = \vec{a} + t\cdot{}\vec{u}$, mit
 Stützvektor $\vec{a} = \overrightarrow{OA}$ und Richtungsvektor $\vec{u}$.


\aufgabeNr{Gegeben ist die Gerade

$$\vec{r}(t) = \Spvek{4;-2;6} + t\cdot{}\Spvek{-0.5;3;4} .$$

a) Welcher Punkt auf der Geraden wird durch den Parameter $t=2$
besschrieben?
\TNT{2}{P(t) = (3|4|14)}

b) Bestimmen Sie zwei weitere Punkte $P=P(t)$ auf $g$ und geben Sie explizit
den Parameter $t$ an.
\TNT{4}{$P(0) = (4|-2|6)$ und $P(-1) = (4.5 | -5 | 2)$}

c) Prüfen Sie, ob der Punkt $Q = (1|16|30)$ auf der Geraden $g$
liegt. Falls ja, geben Sie $t$ an.
\TNT{4}{Ja, mit $t=6$ liegt $Q$ auf der Geraden $g$.}
\noTRAINER{\newpage}
d) Prüfen Sie, ob der Punkt $R = (5.5| -11 | 6)$ auf der Geraden $g$
liegt. Falls ja, geben Sie $t$ an.
\TNT{4}{Nein, $R$ liegt nicht auf $g$. Beispielsweise läge $(5.5 | -11 | -6)$ auf $g$.}
}


\aufgabeNr{
%%Analog Marthaler S. 30 Aufg. 2. a) c)
Geben Sie eine Parametergleichung der Geraden $g$ an, welche durch die
Punkte $A$ und $B$ verläuft:

a) $A=(1|6)$; $B=(3|3)$
\TNT{4}{Beispiel $A$=Stützvektor. Dann ist $$g:\,\,\, \vec{r}(t) = \Spvek{1;6} + t\cdot{}\Spvek{3-1;3-6} = \Spvek{1;6} + t\cdot{}\Spvek{2;-3}$$}

b) $A=(1|-2|4)$; $B=(1|-5|2)$
\TNTeop{Beispiel $A$=Stützvektor. Dann ist $$g: \,\,\, \vec{r}(t)
= \Spvek{1;-2;4} + t\cdot{}\Spvek{1-1;-2-(-5);4-2} =  \Spvek{1;-2;4} + t\cdot{}\Spvek{0;3;2} $$}%% end TNTeop
}%% end Aufgabe


\noTRAINER{\newpage}
\aufgabeNr{
%% Analog Marthaler S. 30ff Aufg. 5

Gegeben ist die Gerade $$g: \,\,\, \vec{r}(t) = \overrightarrow{OA}+t\cdot{}\vec{u}= \Spvek{5;-1;0} + t\cdot{}\Spvek{\frac13; 2; 8}$$.

Bestimmen Sie eine weitere Gleichung der selben Geraden, welche die
folgende Eigenschaft hat:

a) Die neue Gleichung hat denselben Stützvektor $\overrightarrow{OA}$.
\TNT{2}{Wähle \zB $t=3$:
$$\vec{r}(t) = \Spvek{5;-1;0} + t\cdot{} \Spvek{1;6;24}$$}

b) Die neue Gleichung hat denselben Richtungsvektor $\vec{u}$.
\TNT{4}{Setze \zB $t=3$:
$$\vec{r}(t) =\Spvek{5+3 \cdot{} \frac13; -1 + 3\cdot{}2; 0 + 3\cdot{}8} + t \cdot{} \Spvek{\frac13;2;8}$$
$$ = \Spvek{6; 5; 24} + t \cdot{} \Spvek{\frac13;2;8}$$
}%% end TNT 
}%% end Aufgabe


\noTRAINER{\newpage}
\aufgabeNr{
(Grafik aus Marthaler Geometrie Seite 303 Aufg. 7)\\
Gegeben ist die folgende Pyramide in $\mathbb{R}^3$:

\bbwCenterGraphic{12cm}{img/Pyramide.png}\,

Die angegebenen Punkte sind jeweils Eckpunkte oder Mittelpunkte.

Bestimmen Sie mögliche Geradengleichungen der vier eingezeichneten Geraden.
\TNTeop{
$$g_1:  \vec{r}(t) = \Spvek{ 0;13;0} + t\cdot{} \Spvek{ 6; 6;-3}$$
$$g_2:  \vec{r}(t) = \Spvek{-2; 7;3} + t\cdot{} \Spvek{ 2;-2; 3}$$
$$g_3:  \vec{r}(t) = \Spvek{-2; 7;3} + t\cdot{} \Spvek{-4; 4; 0}$$
$$g_4:  \vec{r}(t) = \Spvek{ 0;13;0} + t\cdot{} \Spvek{ 6; 2;-3}$$
}%% end TNTeop
}%% end aufgabe

\aufgabeNr{
Gegeben ist die Funktionsgleichung der Geraden $g$. Bestimmen
Sie eine vektorielle Parametergleichung der Geraden $g$.

Gegeben:
$$y = 0.5x - 1.5$$

Gesucht
$$\vec{r}(t) = \overrightarrow{OA} + t \cdot{} \vec{u}$$
\TNT{4}{
Wähle \zB den $y$-Achsenabschnitt als Stützvektor und den Vektor vom
Stützvektor zur Nullstelle ($x_0=3$)als $\vec{u}$.

$$\vec{r}(t) = \Spvek{0; -1.5} + t \cdot{} \Spvek{3-0; 0-(-1.5)}
= \Spvek{0;-1.5} + t \cdot{} \Spvek{3;1.5}$$

}%% end TNT
}%% end Aufgabe 

.... hier weiter mit einer zweiten Aufgabe analog 8. b) aus Marthaler.
Bei der nächsten Aufgabe (Umkehrung) unbedingt auch eine senkrechte
Gerade einbauen! Diskussion: Mit der Parameterdarstellung sind auch
senkrechte Geraden im Koordinatensystem möglich.          
































\aufgabeNr{
%% Maturaprüfng 2022 Aufg. 4 (Teil 2 mit TR)
Gegeben sind die Gerade $g$ und die Punkte $A$ und $B$.

$$g : \vec{r} = \Spvek{2;1;1} + t\cdot{}\Spvek{0;2;1}\hspace{20mm} A=(6|4| 1), B=(-2|5|7)$$

Berechnen Sie die Punkte $P$ der Geraden $g$, für die gilt: $\angle
APB = 35\degre$.}

\end{document}
