%%
%% Meta: TI nSpire Einführung
%%       Ziel: Damit die Grundoperationen damit durchgeführt werden können.
%%             Damit man sich an den Rechner gewöhnt.
%%

\input{bbwLayoutPage}

%%%%%%%%%%%%%%%%%%%%%%%%%%%%%%%%%%%%%%%%%%%%%%%%%%%%%%%%%%%%%%%%%%

\usepackage{amssymb} %% für \blacktriangleright
\renewcommand{\metaHeaderLine}{Geradengleichung}
\renewcommand{\arbeitsblattTitel}{Vektorgeometrie in $\mathbb{R}^3$}

\begin{document}%%
\arbeitsblattHeader{}

\newcounter{aufgabennummer}
\setcounter{aufgabennummer}{1}


\newcommand\aufgabeML[2]{\begin{samepage}%%
Aufgabe \arabic{aufgabennummer}:\,\,\\
#1\\  #2

\abplz{3.2}
\end{samepage}
\stepcounter{aufgabennummer}%%
}%%


\section{Parameterdarstellung der Geradengleichung}

 Es bezeichne jeweils $g: \vec{r} = \vec{a} + t\cdot{}\vec{u}$, mit
 Stützvektor $\vec{a}$ und Richtungsvektor $\vec{u}$.


%% Maturaprüfng 2022 Aufg. 4 (Teil 2 mit TR)
\aufgabeML{
Gegeben sind die Gerade $g$ und die Punkte $A$ und $B$.

$$g : \vec{r} = \begin{pmatrix}2\\1\\1\end{pmatrix} +
t\cdot{}\begin{pmatrix}0\\2\\1\end{pmatrix}\hspace{20mm}  A=(6|4| 1),
B=(-2|5|7)$$

Berechnen Sie die Punkte $P$ der Geraden $g$, für die gilt: $\angle
APB = 35\degre$.}{}



\end{document}
