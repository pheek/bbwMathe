%%
%% Meta: TI nSpire Einführung
%%       Ziel: Damit die Grundoperationen damit durchgeführt werden können.
%%             Damit man sich an den Rechner gewöhnt.
%%

\input{bmsLayoutPage}

%%%%%%%%%%%%%%%%%%%%%%%%%%%%%%%%%%%%%%%%%%%%%%%%%%%%%%%%%%%%%%%%%%

\usepackage{amssymb} %% für \blacktriangleright
\renewcommand{\metaHeaderLine}{Geradengleichung}
\renewcommand{\arbeitsblattTitel}{Vektorgeometrie in $\mathbb{R}^3$}

\begin{document}%%
\arbeitsblattHeader{}

\newcounter{aufgabennummer}
\setcounter{aufgabennummer}{1}


\newcommand\aufgabeNr[1]{\begin{samepage}%%
\subsection*{Aufgabe \arabic{aufgabennummer}}
%\,\,\\
#1
%%\abplz{3.2}
\end{samepage}%%
\stepcounter{aufgabennummer}%%
}%%


\section{Parameterdarstellung der Geradengleichung}

 Es bezeichne jeweils $g: \vec{r} = \vec{a} + t\cdot{}\vec{u}$, mit
 Stützvektor $\vec{a} = \overrightarrow{OA}$ und Richtungsvektor $\vec{u}$.


\aufgabeNr{Gegeben ist die Gerade

$$\vec{r}(t) = \Spvek{4;-2;6} + t\cdot{}\Spvek{-0.5;3;4} .$$

a) Welcher Punkt auf der Geraden wird durch den Parameter $t=2$
beschrieben?
\TNT{2}{P(t) = (3|4|14)}

b) Bestimmen Sie zwei weitere Punkte $P=P(t)$ auf $g$ und geben Sie explizit
den Parameter $t$ an.
\TNT{4}{$P(0) = (4|-2|6)$ und $P(-1) = (4.5 | -5 | 2)$}

c) Prüfen Sie, ob der Punkt $Q = (1|16|30)$ auf der Geraden $g$
liegt. Falls ja, geben Sie $t$ an.
\TNT{4}{Ja, mit $t=6$ liegt $Q$ auf der Geraden $g$.}
\noTRAINER{\newpage}
d) Prüfen Sie, ob der Punkt $R = (5.5| -11 | 6)$ auf der Geraden $g$
liegt. Falls ja, geben Sie $t$ an.
\TNT{4}{Nein, $R$ liegt nicht auf $g$. Beispielsweise läge $(5.5 | -11 | -6)$ auf $g$.}
}%% end AufgabeNr


\aufgabeNr{
%%Analog Marthaler S. 30 Aufg. 2. a) c)
Geben Sie eine Parametergleichung der Geraden $g$ an, welche durch die
Punkte $A$ und $B$ verläuft:

a) $A=(1|6)$; $B=(3|3)$
\TNT{4}{Beispiel $A$=Stützvektor. Dann ist $$g:\,\,\, \vec{r}(t) = \Spvek{1;6} + t\cdot{}\Spvek{3-1;3-6} = \Spvek{1;6} + t\cdot{}\Spvek{2;-3}$$}

b) $A=(1|-2|4)$; $B=(1|-5|2)$
\TNTeop{Beispiel $A$=Stützvektor. Dann ist $$g: \,\,\, \vec{r}(t)
= \Spvek{1;-2;4} + t\cdot{}\Spvek{1-1;-2-(-5);4-2} =  \Spvek{1;-2;4} + t\cdot{}\Spvek{0;3;2} $$}%% end TNTeop
}%% end Aufgabe


\noTRAINER{\newpage}
\aufgabeNr{
%% Analog Marthaler S. 30ff Aufg. 5

Gegeben ist die Gerade: $$g: \,\,\, \vec{r}(t) = \overrightarrow{OA}+t\cdot{}\vec{u}= \Spvek{5;-1;0} + t\cdot{}\Spvek{\frac13; 2; 8}$$

Bestimmen Sie eine weitere Gleichung der selben Geraden, welche die
folgende Eigenschaft hat:

a) Die neue Gleichung hat denselben Stützvektor $\overrightarrow{OA}$.
\TNT{2}{Wähle \zB $t=3$:
$$\vec{r}(t) = \Spvek{5;-1;0} + t\cdot{} \Spvek{1;6;24}$$}

b) Die neue Gleichung hat denselben Richtungsvektor $\vec{u}$.
\TNT{4}{Setze \zB $t=3$:
$$\vec{r}(t) =\Spvek{5+3 \cdot{} \frac13; -1 + 3\cdot{}2; 0 + 3\cdot{}8} + t \cdot{} \Spvek{\frac13;2;8}$$
$$ = \Spvek{6; 5; 24} + t \cdot{} \Spvek{\frac13;2;8}$$
}%% end TNT 
}%% end Aufgabe


\noTRAINER{\newpage}
\aufgabeNr{
(Grafik aus Marthaler Geometrie Seite 303 Aufg. 7)\\
Gegeben ist die folgende Pyramide in $\mathbb{R}^3$:

\bbwCenterGraphic{12cm}{img/Pyramide.png}\,

Die Punkte $A$, $B$, $C$, $D$ und $S$ sind die Eckpunkte. Die anderen
angegebenen Punkte sind Mittelpunkte.

Bestimmen Sie mögliche Geradengleichungen der vier eingezeichneten Geraden.
\TNTeop{
$$g_1:  \vec{r}(t) = \Spvek{ 0;13;0} + t\cdot{} \Spvek{ 6; 6;-3}$$
$$g_2:  \vec{r}(t) = \Spvek{-2; 7;3} + t\cdot{} \Spvek{ 2;-2; 3}$$
$$g_3:  \vec{r}(t) = \Spvek{-2; 7;3} + t\cdot{} \Spvek{-4; 4; 0}$$
$$g_4:  \vec{r}(t) = \Spvek{ 0;13;0} + t\cdot{} \Spvek{ 6; 2;-3}$$
}%% end TNTeop
}%% end aufgabe


\noTRAINER{\newpage}

\aufgabeNr{
Gegeben ist die Funktionsgleichung der Geraden $g$ in $\mathbb{R}^2$. Bestimmen
Sie eine vektorielle Parametergleichung der Geraden $g$.

Gegeben:

a) $$g: \,\,\, y = 0.5x - 1.5$$

b $$g: \,\,\, y = \frac{-2}5 x $$

c) $$g:\,\,\, y = 1.8$$

Gesucht:
$$\vec{r}(t) = \overrightarrow{OA} + t \cdot{} \vec{u}$$
\TNTeop{
a)
Wähle \zB den $y$-Achsenabschnitt als Stützvektor und den Vektor vom
Stützvektor zur Nullstelle ($x_0=3$)als $\vec{u}$.

$$\vec{r}(t) = \Spvek{0; -1.5} + t \cdot{} \Spvek{3-0; 0-(-1.5)}
= \Spvek{0;-1.5} + t \cdot{} \Spvek{3;1.5}$$

b) Der $y$-Achsenabschnitt ist $0$. Somit kann der Nullvektor als
Stützvektor dienen. Wir wählen irgend einen Punkt $S$ auf der Geraden
als Stützvektor; \zB $S=\Spvek{1;\frac{-2}{5}}$.

$$\vec{r}(t) = \vec{0} + t\cdot{} S = t\cdot{} \Spvek{1;\frac{-2}{5}}$$



c) Der $y$-Achsenabschnitt funktioniert auch hier ($=1.8$). Jedoch hat
die konstante Funktion keine Nullstelle. Wir können einen beliebigen
zweiten Punkt auf der Geraden wählen oder wir nehmen als
Richtungsvektor $\vec{u} = \overrightarrow{e_x}$.

$$\vec{r}(t) = \Spvek{0;1.8} + t\cdot{} \Spvek{1;0}$$
}%% end TNT
}%% end Aufgabe 


\aufgabeNr{
Gegeben ist eine Parametergleichung der Geraden $g$. Bestimmen Sie die
Funktionsgleichung in der Form $y=a\cdot{}x+b$ für die Gerade.

a) $$\vec{r}(t) = \Spvek{3;1} + t\cdot{} \Spvek{-2;0.4}$$

b) $$\vec{r}(t) = \Spvek{5;3} + t\cdot{} \Spvek{0;-1.4}$$

\TNTeop{a)
$g: \,\,\, y= -0.2 x + 1.6 = \frac{-1}{5}x + \frac85$

b)
Die Gerade verläuft senkrecht im Koordinatensystem. Als
Funktionsgleichung ist dies nicht möglich.
Es gilt für alle $y$-Werte die Relation $$x = 5$$
}%% end tnt
}%% end AufgabeNr


\section{Gegenseitige Lage von Geraden}

\aufgabeNr{
%% Frommenwiler Geometrie Aufgabe 154
Bestimmen Sie die spezielle Lage der Geraden:

Spezielle Lagen sind:

\begin{itemize}
\item parallel oder zusammenfallend mit einer Achse
\item parallel oder zusammenfallend mit einer Ebene
\end{itemize}

a) $$\vec{r}(t) = \Spvek{ 0;0;5} + t\cdot{} \Spvek{ 0;0; 2}$$
b) $$\vec{r}(t) = \Spvek{-2;0;7} + t\cdot{} \Spvek{ 1;0; 0}$$
c) $$\vec{r}(t) = \Spvek{ 2;5;0} + t\cdot{} \Spvek{-2;5; 0}$$
d) $$\vec{r}(t) = \Spvek{ 4;1;3} + t\cdot{} \Spvek{ 0;1; 2}$$
e) $$\vec{r}(t) = \Spvek{ 0;0;0} + t\cdot{} \Spvek{ 2;4;-3}$$

\TNTeop{
a) die $z$-Achse

b) parallel zur $x$-Achse bei $z=7$ und $y=0$, also in der $xz$-Ebene

c) liegt in der $xy$-Ebene

d) parallel zur $yz$-Ebene bei $x=4$

e) durchstößt den Ursprung
}% end TNT
}%% end aufgabeNr


\aufgabeNr{
Gegeben ist die Gerade $g$:
$$r(t) = \Spvek{2; -1; 4} + t\cdot{} \Spvek{6; 5; -3}$$
Gesucht ist die gerade $h$, die parallel zu $g$ verläuft und durch den
Punkt $(7 | 7 |7)$ verläuft.

Geben Sie eine mögliche Gleichung der Geraden $h$ in Parameterdarstellung an.

\TNTeop{Nur der Stützvektor ändert.
$$h:\,\,\, \vec{r}(t) = \Spvek{7;7;7} + t \cdot{} \Spvek{6;5;-3}$$}%% end TNT
}%% end AufgabeNr


\noTRAINER{\newpage}
\aufgabeNr{
%% Marthaler Geometrie Aufg. 25 (Seite 303ff)
Gegeben sind die Geraden $g$, $h$, $i$ und $k$:

$$g: \,\,\, \vec{r}(t) = \Spvek{3;1} + t \cdot{} \Spvek{ 1; 2}$$
$$h: \,\,\, \vec{r}(t) = \Spvek{1;2} + t \cdot{} \Spvek{-2; 3}$$
$$i: \,\,\, \vec{r}(t) = \Spvek{4;3} + t \cdot{} \Spvek{-2;-4}$$
$$k: \,\,\, \vec{r}(t) = \Spvek{0;5} + t \cdot{} \Spvek{ 2;-3}$$

Analysieren Sie die gegenseitige Lage je zweier Geraden:
\TNTeop{
$g$ und $h$: schneidend

$g$ und $i$: zusammenfallend

$g$ und $k$: schneidend

$h$ und $i$: schneidend

$h$ und $k$:  zusammenfallend

$i$ und $k$: schneidend

}%% end TNT
}%% end AufgabeNr

\aufgabeNr{
%% Aufgabe 27 Marthaler Seite 306
Gegeben sind die Geraden $g: \,\,\, \vec{r}(t) = \Spvek{1;1;4} +
t\cdot{} \Spvek{-2;2;1}$ und $h$. Analysieren Sie deren gegenseitige
Lage und bestimmen Sie allfällige Schnittpunkte:


a) 
$$h: \,\,\, \vec{r}(t) = \Spvek{-1; 4; 2} + s \cdot{} \Spvek{ 1;  1;  3}$$

b) 
$$h: \,\,\, \vec{r}(t) = \Spvek{ 2; 4; 5} + s \cdot{} \Spvek{ 4; -4; -2}$$

c) 
$$h: \,\,\, \vec{r}(t) = \Spvek{ 3;-1; 3} + s \cdot{} \Spvek{-1;  1;0.5}$$

d) 
$$h: \,\,\, \vec{r}(t) = \Spvek{-1;-2; 5} + s \cdot{} \Spvek{ 2;0.5; -1}$$

\TNTeop{
a) windschief

b) parallel

c) identisch (jeder Punkt auf $g$ ist «Schnittpunkt»)

d) schneidend. Die Geraden schneiden sich bei $s=2$ und $t=-1$ in
$(3| -1 | 3)$

Lösungsweg Aufg. 27. Marthaler Vektorgeometrie Seite 306
} %% end TNTeop
}%% end AufgabeNr


\aufgabeNr{
%% RoMi Aufgabe 28 von Michaels Arbeitsblatt
Gegeben sind die Geraden

$$g:\,\,\,  \vec{r}(t) = \Spvek{k-9;0;1} + s\cdot{} \Spvek{2;1;1}$$
und
$$h:\,\,\,  \vec{r}(t) = \Spvek{  8;7;9} + t\cdot{} \Spvek{4;2;k}$$

a) Bestimmen Sie $k$ so, dass die beiden Geraden $g$ und $h$ parallel
sind. Begründen Sie, warum die Geraden mit diesem $k$ nicht identisch
sind.

b) Bestimmen Sie $k$ so, dass sich die beiden Geraden schneiden.


\TNTeop{a)
parallel heißt die Richtungsvektoren sind kollinear. Also
$$s\cdot{}\Spvek{2;1;1} = t\cdot{}\Spvek{4;2;k}$$
daher ist $t=2$ und somit $k=2$. Mit $k=2$ sind die Geraden parallel.

Versuchen wir nur schon die $x$ und $y$ Komponente nun gleichzusetzen,
so erhalten wir mit
\gleichungZZ{-7+2s}{8+4t}{s}{7+2t}
einen Widerspruch. Somit gibt es keine gemeinsamen Punkte mit $k=2$.

b)

Löse das Gleichungssystem
\gleichungDD{k-9+2s}{8+4t}{s}{7+2t}{1+s}{9+k\cdot{}t}

$k=3$, $t=-1$ und $s=5$

}%% end TNT eop

}%%


\section{Abstände}

\subsection{Abstand Gerade-Punkt}

\aufgabeNr{
Bestimmen Sie die Abstände des Punktes $A=(6; 9; -4)$ von den drei
Koordinatenachsen.


\TNTeop{
zur $x$-Achse:

Abstand = $\sqrt{9^2 + (-4)^2} = \sqrt{97} \approx 9.85$

zur $y$-Achse:

Abstand = $\sqrt{6^2 + (-4)^2} = \sqrt{52} \approx 7.21$

zur $z$-Achse:

Abstand = $\sqrt{6^2 + 9^2} =\sqrt{117} \approx 10.8$
}%% end TNT
}%% end AufgabeML


\aufgabeNr{
%% entspricht Marthaler S. 305ff Aufg. 20
Wie weit ist der Ursprung von der Geraden $g$ entfernt?
$$g:\,\,\,  \vec{r}(t) = \Spvek{-3; 2; 1} + t\cdot{} \Spvek{-2;-1;2}$$


Kontrollieren Sie Ihr Resultat mit dem Taschenrechner.

\TNTeop{
Setze das Skalarprodukt eines Vektors $\vec{a} \in g$ mit dem
Richtungsvektor gleich Null:
$$(-2)\cdot{}(-3-2t_a) + (-1)\cdot{}(2-t_a) + 2\cdot{}(1+2t_a) = 0$$

Somit ergibt sich $t_a= \frac{-2}3$.
Damit ist die Länge von $\vec{a}$ berechenbar:

$$|\vec{a}| = \sqrt{(-3+\frac43)^2 + (2+\frac23)^2 + (1-\frac43)^2}
= \frac13\cdot{}\sqrt{25+24+1} = \frac53\cdot{}\sqrt{2} \approx 2.357$$

}%% end TNTeop
}%% end Aufgabe Nr

\aufgabeNr{
%% Marthaler S. 305ff Aufg. 18

Wie groß ist der Abstand zwischen der Geraden $g$ und dem Punkt $P$?

Alle Aufgaben sind von Hand lösbar. Lösen Sie mindestens eine der
Aufgaben von Hand und kontrollieren Sie Ihr Resultat mit dem Taschenrechner.

a) $g:\,\,\, \vec{r}(t) = \Spvek{3;-3} + t\cdot{}\Spvek{-4;3} ; P=(0|2)$

b) $g:\,\,\, \vec{r}(t) =  t\cdot{}\Spvek{0;2} ; P=(5|3)$

c) $g:\,\,\, y=2x-4 ; P=(-1|5)$

d) $g:\,\,\, \vec{r}(t) =  \Spvek{2;3;-7} + t\cdot{}\Spvek{0;4;0} ; P=(6|3|-4)$

\TNTeop{
a) $2.2=\frac{11}5$


b) $3$


c) $11\cdot{}\frac{\sqrt{5}}5 \approx 4.9193$


d) $5$

(Lösungswege im OLAT zur Aufgabe 18. Seite 305 Geometrie Marthaler.)

}%% end TNTeop

}%% end Aufg. Nr


\aufgabeNr{%% entspricht Marthaler S. 305 Aufg. 21
Welcher Punkt $Q$ auf der Geraden $g$ hat den kleinsten Abstand zum
Punkt $P$?

$$P=(5|3|2); \,\,\,g:\,\,\, \vec{r}(t) = \Spvek{1;1;1} + t\cdot{} \Spvek{-1;-2;-3}$$

\TNTeop{
Erstens Differenzvektor:
$\vec{a}=\overrightarrow{PQ}=\Spvek{-t-4;-2t-2;-3t-1}$

Diesen Vektor senkrecht zum Richtungsvektor $\Spvek{-1;-2;-3}$ setzen:

$\vec{a}\circ \Spvek{-1;-2;-3} = 0$

$$(-1-4)(-1) + (-2t-2)(-2) + (-3t-1)(-3) = 0$$

Somit ist $t=\frac{-11}{14}$

Dies für $t$ in $g$ eingesetzt ergibt
$$Q=\left(\frac{25}{14} | \frac{18}{7} | \frac{47}{14}\right)$$
}%% end TNT eop
}%% end aufgabe Nr


\subsection{Abstand windschiefer Geraden}

\aufgabeNr{
Wie groß ist der Abstand der windschiefen Geraden $g$ und $h$
zueinander?

a) $$g: \,\,\,  \vec{r}(t) = \Spvek{1;2;1} + t\cdot{}\Spvek{-3;5;1};
     h: \,\,\,  \vec{r}(s) = \Spvek{4;11;3} + s\cdot{}\Spvek{1;1;-1}$$

b) $$g: \,\,\,  \vec{r}(t) = \Spvek{3;-3;8} + t\cdot{}\Spvek{0;1;0};
     h: \,\,\,  \vec{r}(s) = \Spvek{-1;5;2} + s\cdot{}\Spvek{3;1;1}$$

Berechnen Sie zunächst die Lösungen von Hand und geben Sie das
Resultat exakt (Brüche, Wurzeln, Logarithmen) an.

Prüfen Sie das Resultat anschließend mit dem Taschenrechner und geben
Sie jeweils vier signifikante Stellen an.

\TNTeop{
Siehe Lösungen OLAT Marthaler S. 307 Aufg. 35

a) Parameter: $t=1$ und $s=-3$; Abstand = $\sqrt{26} \approx5.099$

b) Parameter: $t=\frac{-35}3$ und $s=\frac{-10}3$; Abstand = $\frac{7}3\cdot{}\sqrt{6} \approx 5.715$
}
}%% end AufgabenNr

\end{document}
