%%
%% Meta: TI nSpire Einführung
%%       Ziel: Damit die Grundoperationen damit durchgeführt werden können.
%%             Damit man sich an den Rechner gewöhnt.
%%

\input{bmsLayoutPage}

%%%%%%%%%%%%%%%%%%%%%%%%%%%%%%%%%%%%%%%%%%%%%%%%%%%%%%%%%%%%%%%%%%

\usepackage{amssymb} %% für \blacktriangleright
\renewcommand{\metaHeaderLine}{Arbeitsblatt}
\renewcommand{\arbeitsblattTitel}{Vektorgeometrie in $\mathbb{R}^3$}

\begin{document}%%
\arbeitsblattHeader{}

\newcounter{aufgabennummer}
\setcounter{aufgabennummer}{1}


\newcommand\aufgabeML[2]{
\textbf{Aufgabe \arabic{aufgabennummer}}:\,\,\\
#1  #2

\abplz{4}

\hrule

\stepcounter{aufgabennummer}
}




\renewcommand\aufgabeML[2]{\begin{samepage}%%
\textbf{Aufgabe \arabic{aufgabennummer}:}\,\,\\
#1%%\\  \TRAINER{#2}
%%
\TRAINER{#2}%%abplz{5.2}
\noTRAINER{\mmPapierBisEndeSeite}
%%\end{samepage}
\stepcounter{aufgabennummer}%%
\end{samepage}%%
}%%



\section{Vektorgeometrie in $\mathbb{R}^3$}
\subsection{Komponenten}

\aufgabeML{Welcher Vektor $\vec{s}$ beschreibt die Verschiebung, welche
den Punkt $A$ auf $A'$ abbildet?

$A = (6  | 9 | -1.5)$

$A' = (8 | 5 | 3.5)$

\vspace{22mm}
}{

Lösung: $\vec{s}$ = \LoesungsRaum{$\Spvek{2;-4;5}$}%%
}


\aufgabeML{Gegeben sind die folgenden Vektoren:
$$\vec{a} = \Spvek{1;2;u}; \vec{b} = \Spvek{u;v;-1} \text{ und } \vec{c} = \Spvek{w; 2w; 0} $$

Berechnen Sie $u$, $v$ und $w$ so, dass $\vec{a} + \vec{b} + \vec{c} =
0$.
}{%%

Lösung: \noTRAINER{\vspace{15mm}}\TRAINER{1. Aus der dritten
Komponente folgt $u=1$. Nun folgt aus der ersten Komponente, dass
$w=-2$ ; danach kann das $v$ aus der zweiten Komponente errechnet
werden: $v=2$}}
\newpage
\aufgabeML{Bestimmen Sie die fehlenden Parameter, damit gilt $\vec{a} = 3\cdot{}\vec{b}$

$$\vec{a} = \Spvek{6; -t; m} \text{ und } \vec{b}= \Spvek{-\frac{s}{4}; 3(t-1); 6.8m}$$
}{

Lösung:

\TRAINER{Komponentenweise
$s=-8$, $t=\frac9{10}$, $m=0$}%% end TRAINER
}%% end AufgabeML


\newpage
\subsection{Längenprobleme}
\aufgabeML{
a) Bestimmen Sie den Betrag des Vektors $\vec{a}$:
$$\vec{b}=\Spvek{2;6;9}$$

b) Berechnen Sie:

$$\left|\Spvek{4;4;7}\right|$$


c) Bestimmen Sie mit dem Taschenrechner den Betrag des Vektors $\vec{c}$:
$$\vec{c}=\Spvek{3.5;-6;0}$$

%%\vspace{22mm}
}{
a) $|\vec{a}| = \LoesungsRaum{11}$

b) $|\vec{b}| = \LoesungsRaum{9}$

c) $|\vec{c}| \approx \LoesungsRaum{6.94622}$
}
\newpage

\aufgabeML{%%
a) Der Vektor $\vec{v}$ habe die Struktur $\Spvek{3;v_y;2}$. Es gilt
$|\vec{v}| = 7$. Berechnen Sie $v_y$.

b) 
Der Vektor $\vec{v}$ habe folgende Struktur:
$$\vec{v}=\Spvek{a;6;a+3}$$
Wir wissen, dass $$|\vec{v}|=\sqrt{125}$$
Berechnen Sie $a$. (Tipp: Zweiklammeransatz)
}{

Lösung: a)
\TNT{4}{
$$\sqrt{2^2 + 3^2 + v_y^2}=7$$
$$13 + v_y^2=49$$
$$v_y^2=36$$
$$L_{v_y} = \{-6; 6\}$$
}%% end TNT 4

Lösung b)
$\TRAINER{a=-8 \text{ oder } 5}$\\
\TRAINER{Es gilt
$$\sqrt{125} = \sqrt{a^2 + 6^2 + (a+3)^2}$$
Somit
$$125 = a^2 + 36 + a^2 + 6a + 9$$
Grundform:
$$a^2 + 3a -40 = 0$$
Zweiklammeransatz:
$$(a+8)(a-5) = 0$$

Somit $L_a = \{-8, 5\}$}
}

\newpage


\aufgabeML{Gegeben sind die Punkte:
$$A= (7 | 5.5 | -3) \text{ und } B=(5|-2|0)$$

Berechnen Sie die folgenden Terme:
$\overrightarrow{OA}$ und $\overline{AB}$

}{$$\overrightarrow{OA}
=\LoesungsRaum{\Spvek{7;5.5;-3}}$$
$$\overline{AB} \approx \LoesungsRaum{8.3217}$$}


\aufgabeML{Gegeben sind die Punkte $A=(5|1)$ und $B=(8|5)$.

a) Bestimmen Sie die Vektoren $\overrightarrow{OA}$,
$\overrightarrow{AB}$

b) Geben Sie die Länge des Vektors $\overrightarrow{BA}$ an.

}{

Lösung zu a) $$\overrightarrow{OA} = \LoesungsRaum{\Spvek{5;1}}$$

$$\overrightarrow{AB} = \LoesungsRaumLen{50mm}{\overrightarrow{OB} - \overrightarrow{OA}
=  \Spvek{8;5} - \Spvek{5;1} = \Spvek{3;4}}$$

Lösung zu b)

$$|\overrightarrow{AB}| = \LoesungsRaumLang{\sqrt{3^2+4^2} = \sqrt{25}= 5}$$
}%% end aufgabeML
\newpage

\aufgabeML{Bestimmen Sie alle Punkte der $xy$-Ebene, welche von den
beiden Punkten $A=(7|6)$ und $B=(-1|2)$ den selben Abstand haben.}{

Lösung: $$\Spvek{x;y} \text{ mit}$$

$$\LoesungsRaumLang{y = -2x + 10}$$

Lösugsweg:

Wir definieren $P := (x|y)$. Nun gelte $|\overrightarrow{AP}| =
|\overrightarrow{BP}|$. Mit den Zahlen:

$$\bigg|\Spvek{7-x;6-y}\bigg| = \bigg|\Spvek{-1-x;2-y}\bigg|$$
Wenn wir nun den Abstand rechnen, so erhalten wir:

$$\sqrt{(7-x)^2 + (6-y)^2} = \sqrt{(-1-x)^2 + (2-y)^2}$$
Beidseitig quadrieren und ausmultiplizieren:
$$49-14x+x^2 + 36-12y+y^2  =  1 + 2x + x^2  + 4 - 4y + y^2$$
Quadrate beidseitig subtrahieren und zusammenfassen:
$$49+36-1-4 = 8y + 16 x$$
Grundform:
$$y= -2x + 10$$
Alle Punkte auf der Geraden $y=-2x+10$ sind von $A$ und $B$ gleich weit entfernt.

}%% end aufgabeML





\aufgabeML{%%
Berechnen Sie den Umfang des Dreiecks
$$A=(7|2|6), B=(-4|3|-1), C=(0;0;8)$$
}{
Umfang $\approx \LoesungsRaumLang{30.9222}$

\TRAINER{
$$\overrightarrow{AB} = \Spvek{7-(-4);2-3;6-(-1)} = \Spvek{11;-1;7}$$
$$\overrightarrow{BC} = \Spvek{-4-0; 3-0; -1-8} = \Spvek{-4;3;-9}$$
$$\overrightarrow{AC} = \Spvek{7-0;2-0;6-8} = \Spvek{7;2;-2}$$
}%% end TRAINER

}%% end aufgabeML
\newpage


\aufgabeML{Bestimmen Sie die Länge der Seitenhalbierenden $s_a$ im
Dreieck $\Delta ABC$\\
(es gelte $e_x=e_y=e_z=1$):

$$A=(1|1|2); B=(3|4|5); C=(2|9|6)$$

}{Lösung: Die Länge misst \LoesungsRaumLang{ca. $6.6895$} (Einheiten)
\TRAINER{
Erstens den Mittelpunkt $M_a$ der Strecke $\overline{BC}$ rechnen:
$$M_a = (2.5| 6.5 | 5.5)$$
Nun den Vektor $\vec{s_a} = \overrightarrow{AM_a}$ bestimmen:
$$\vec{s_a} = \Spvek{2.5-1; 6.5-1; 5.5-2} = \Spvek{1.5; 5.5; 3.5}$$
TR: Dreidimensionaler Pythagoras:
$$|s_a| = \sqrt{1.5^2 + 5.5^2 + 3.5^2} \approx 6.6859$$
}%% end TRAINER
}%% end aufgabeML
\newpage
\aufgabeML{Berechnen Sie alle Punkte auf der $y$-Achse, welche vom
Punkt $P=(2|-3|4)$ fünf Einheiten entfernt sind.}{

Lösung:
\TRAINER{Ansatz Punkt = $\Spvek{0;y;0}$ Dann gilt

$$5 = \sqrt{(2-0)^2 + (-3-y)^2 + (4-0)^2}$$
$$25=4+(3+y)^2 + 16$$
$$y_{1,2} = \frac{-6\pm \sqrt{36-4\cdot{} 4}}{2}$$

$y=-3\pm \sqrt{5} \approx -0.764 \text{ bzw. } -5.24$}
}
\newpage
\aufgabeML{Bestimmen Sie mit dem Taschenrechner alle möglichen Punkte
auf der $x$-Achse, welche vom Punkt $A=(-4|3|12)$ die dreifache Entfernung wie
vom Punkt $B=(16|2|-1)$ haben.}{
Lösung: $\LoesungsRaumLang{}$
\TRAINER{Ansatz: Gesuchter Punkt $P$ hat Koordinaten $(x|0|0)$.

Dann gilt

$\overline{AP} = 3\cdot{} \overline{BP}$

TR: solver: $$\lx = \{10.15; 26.85\}$$
}%% end TRANIER
}%% end aufgabeML
\newpage

\aufgabeML{Gegeben sind die Drei Punkte $A=(1|2|-3)$, $B=(6|-2|4)$ und
$C=(2|3|-5)$.

Welcher Punkt in der $xz$-Ebene ist von all den drei Punkten gleich
weit entfernt?
}{Lösung \LoesungsRaumLang{$P=\Spvek{\frac{126}{17};0;\frac{-39}{17}}$}
}%% end aufgabe ML
\newpage

\subsection{Vielfaches im Raum}

\aufgabeML{Berechnen Sie $\vec{b}$ so, dass $\vec{b}$ den Betrag 5 hat
und dem Vektor $\vec{a}$ entgegengesetzt ist:
$$\vec{a} = \Spvek{4;0;-3}$$

\vspace{12mm}
}{$$\vec{b} = \LoesungsRaum{\Spvek{-0.8;0;0.6}}$$}
\newpage


\aufgabeML{
Gegeben sind die drei Vektoren $\vec{a}=\Spvek{1;2;0}$,
$\vec{b}=\Spvek{-4;-3;5}$ und
$\vec{c}=\Spvek{2;-3;1}$.

Berechnen Sie den Vektor $\vec{d}$, wenn gilt:

$$ -\vec{a} + 5\left( \vec{d}-\frac13\vec{b} \right) = 2\vec{c}$$

}{Der Lösungsvektor $\vec{d}$ ist gleich
\TRAINER{$\Spvek{-\frac13;-\frac95;\frac{31}{15}}$}

Kontrollieren Sie Ihr Ergebnis mit dem Taschenrechner.
}



\aufgabeML{Berechnen Sie den Vektor $\vec{v} = \vec{a}+2.5\vec{b}
-3\vec{c}$ mit:

$$\vec{a}= \Spvek{4;3;-1}; \vec{b}= \Spvek{1;-1;0} \vec{c}= \Spvek{-2;-1;1}$$}%%
{Lösung: \noTRAINER{\vspace{22mm}}\TRAINER{$$\vec{v} = \Spvek{4 + 2.5\cdot{}1 - 3\cdot{}(-2);3 +
2.5\cdot{}(-1) -3\cdot{}(-1);-1 +0 -3\cdot{}1}= \Spvek{12.5;3.5;-4}$$}}%%

\newpage

\aufgabeML{Kollinear:

Bestimmen Sie $x$ und $y$ so, dass $\vec{a}$ kollinear zu $\vec{b}$
ist:

$$\vec{a} = \Spvek{2;3;x}; \vec{b} = \Spvek{-x;5;y}$$
}{$x=\LoesungsRaum{\frac{-10}3}$ und $y = \LoesungsRaum{\frac{-50}9}$

\TRAINER{Es gilt $\vec{a} = \lambda\cdot{}\vec{b}$.
Aus der 2. Komponente folgt $3 = \lambda\cdot{}5$ und somit $\lambda =
\frac35$.\\

Dieses $\lambda$ setzen wir in die erste Komponente einund
erhalten $$2 = \frac35 \cdot{-x}$$
ergo $$x = \frac{-10}3$$

Nun können wir aus der 3. Komponente das $y$ berechnen:
$$x = \lambda\cdot{}y$$
$x$ und $\lambda$ einsetzen:
$$\frac{-10}3 = \frac35 \cdot{} y$$
und somit
$$y = \frac{-50}9$$}%% end TRAINER
}%% end aufgabeML

\aufgabeML{Gegeben seien die Vektoren $\vec{a}$, $\vec{b}$ und
$\vec{c}$.

Des weiteren ist die folgende Beziehung bekannt:

$$2\vec{c}-\frac14 \vec{b} - 2(\vec{a}+\vec{d})= 3(\vec{a}+\vec{b}+\vec{d})$$

Geben Sie den Vektor $\vec{d}$ als Linearkombination der Vektoren
$\vec{a}$, $\vec{b}$ und $\vec{c}$ an:
}{$$\vec{d}  = \LoesungsRaumLang{-5\vec{a} -\frac{13}4 \vec{b} + 2\vec{c}}$$}
\newpage

\aufgabeML{Bestimmen Sie die fehlenden Koordinaten, wenn beide der
folgenden Eigenschaften zutreffen sollten:

a) 
$$\vec{s} = \vec{a} + 2\vec{b} -5 \vec{c}$$
\textbf{und} b)\\
$$\vec{a} \text{ ist kollinear zu } \vec{c}$$
Für
$$\vec{a}=\Spvek{3;-2}; \vec{b} = \Spvek{b_x; b_y}; \vec{c}
= \Spvek{-2;c_y}; \vec{s} = \Spvek{5;5}$$}%%
{$b_x = \LoesungsRaum{-4}$, $c_y = \LoesungsRaum{\frac43}$ und $b_y
= \LoesungsRaum{\frac{41}6}$
\TRAINER{
Aus a) folgt:
\gleichungZZ{3+2b_x -5\cdot{}(-2)}{5}{-2+2\cdot{}b_y -5\cdot{}c_y}{5}
Aus der oberen Gleichung folgt:
$$2b_x + 10 = 2$$
und somit
$$b_x=-4$$
Weil $\vec{a}$ kollinear zu $\vec{b}$ folgt aus der 1. Komponente:
$$\vec{a} = \lambda\cdot{}\vec{b} \Longrightarrow 3 = \lambda \cdot{}
(-2) \Longrightarrow \lambda = \frac{-3}2$$
Wegen $-2 = \lambda \cdot{} c_y$ folgt, dass $c_y = \frac43$.
Dies können wir in die 2. Gleichung einsetzen:
$$2b_y -5\cdot{}\frac43 = 7$$
und somit
$$b_y = \frac{41}6$$
}%% end TRAINER
}%% end aufgabeML

\aufgabeML{%
Einheitsvektor:

Gegeben ist der Vektor $\vec{a}$:
$$\vec{a} = \Spvek{\sqrt{7};-2;3.8}$$

Skalieren Sie den Vektor $\vec{a}$ mit einem Skalar $s$ so, dass er die Länge 1 hat.


}{%
Lösung: $s=\LoesungsRaum{\frac{5\cdot{}\sqrt{159}}{318} \approx 0.1983}$
}
\newpage
\aufgabeML{Bestimmen Sie den Parameter $p$ so, dass es sich beim
folgenden um einen Einheitsvektor handelt:
$$\Spvek{0.7; k; \frac1{10}}$$
(Mit Taschenrechner auf 3 signifikante Stellen angeben.)


}{Lösung = \LoesungsRaumLang{$\sqrt{2}/2 \approx 0.707$}}

\newpage

\aufgabeML{
Berechnen Sie alle Punkte $A$ auf der Geraden $PQ$, die sieben Einheiten
von $P$ entfernt sind.

Gegeben: $$P=(3|9|1); Q=(6|8|-4)$$
}{Lösung \TRAINER{mit Einheitsvektor:

$$\Spvek{3;9;1} \pm \frac{\sqrt{35}}{5}\cdot{}\Spvek{3;-1;-5}$$

}%% end TRAINER
}%% end AufgabeML



\newpage
\subsection{Linearkombination}
\aufgabeML{Stellen Sie den Vektor $\vec{z}$ als Linearkombination von
$\vec{a}$, $\vec{b}$ und $\vec{c}$ dar.
$$\vec{a} = \Spvek{4;2;-7};
  \vec{b} = \Spvek{1;-4;-5};
  \vec{c} = \Spvek{4;0;-3};
  \vec{z} = \Spvek{48;48;84}$$

\vspace{12mm}
}{$$\vec{z} = \LoesungsRaum{42}\cdot{}\vec{a}
+ \LoesungsRaum{9}\cdot{}\vec{b} + \LoesungsRaum{-129}\cdot{}\vec{c}$$}
\newpage


\aufgabeML{Lässt sich der Vektor $\vec{a}$ als Linearkombination von
$\vec{u}$, $\vec{v}$ und $\vec{w}$ schreiben? Falls ja, geben Sie die
Linearkombination explizit an, falls nein, begründen Sie warum nicht.
$$\vec{a} = \Spvek{5;6;-7}$$

a) $$\vec{u} = \Spvek{1;0;1}; \vec{v} = \Spvek{0;1;1}; \vec{w} = \Spvek{1;1;0}$$

b) $$\vec{u} = \Spvek{1;0;0}; \vec{v} = \Spvek{1;0;1}; \vec{w} = \Spvek{0;0;1}$$

c) $$\vec{u} = \Spvek{1;0;0}; \vec{v} = \Spvek{1;1;0}; \vec{w} = \Spvek{1;1;1}$$
}{
Lösung zu a):\\
\noTRAINER{\vspace{30mm}}
\TRAINER{\gleichungDD{\alpha+\gamma}{5}{\beta+\gamma}{6}{\alpha+\beta}{-7}
$$\Longrightarrow \alpha = 5-\gamma; \beta = 6-\gamma$$
$$\alpha + \beta = -7 \Longrightarrow
5-\gamma+6-\gamma=-7 \Longrightarrow \gamma = 9$$
$$\Longrightarrow  \beta = -3; \alpha = -4$$
$$-4\vec{u} -3\vec{v} +9\vec{w} = \Spvek{5;6;-7}$$
}%% end TRAINER

Lösung zu b):\\
\noTRAINER{\vspace{30mm}}
\TRAINER{
Dies ist nicht lösbar, denn zum einen sind $\vec{u}$, $\vec{v}$ und
$\vec{w}$ nicht linear unabhängig und zum anderen kann die zweite
Komponente (die $y$-Koordinate) gar nicht 6 werden, denn die
$y$-Koordinate ist in allen drei gegebenen Vektoren = 0.\\
}%% end TRAINER

Lösung zu c):\\
\noTRAINER{\vspace{30mm}}
\TRAINER{
Beginne mit der $z$-Koordinate $\Longrightarrow \gamma=-7$. Verfahre
von unten her weiter mit der $y$-Koordinate...

$$-\Spvek{1;0;0} +13\Spvek{1;1;0}-7\Spvek{1;1;1} = \Spvek{5;6;-7}$$
}%% end TRAINER
}%% end aufgabeML
\newpage





\aufgabeML{Gegeben sind die Vektoren
$$\vec{a}= \Spvek{-5;3;-7},
 \vec{b} = \Spvek{-3;-3;2} \text{ und }
 \vec{c} = \Spvek{0;2;-2}$$


Berechnen Sie $\vec{d}$ so, dass
$\vec{a}+\vec{b}+\vec{c}+\vec{d}=\vec{0}$ ergibt.
\vspace{22mm}
}{$\vec{d} = \LoesungsRaum{\Spvek{8;-2;7}}$}

\subsection{lineare Abhängigkeit}


\aufgabeML{Sind die beiden folgenden Vektoren voneinander linear
abhängig?

$\vec{a} = \Spvek{2;12;7}$, $\vec{b} = \Spvek{-3;-18;10}$

}{Die beiden Vektoren sind linear \LoesungsRaumLang{unabhängig}.}

\newpage


\aufgabeML{Für welches $z$ sind die drei folgenden Vektoren
voneinander linear abhängig?

$\vec{a} = \Spvek{1;-1;3}$,
$\vec{b} = \Spvek{4;5;1}$
$\vec{c} = \Spvek{2;-4;z}$
}{
Lösung $z = \LoesungsRaum{\frac{76}9}$\TRAINER{, denn damit geht die
Gleichung $$s\cdot{}\vec{a} + t\cdot{}\vec{b} = \vec{c}$$ auf und
somit ist $\vec{c}$ eine Linearkombination von $\vec{a}$ und $\vec{b}$}.
}


\aufgabeML{%% Frommenwilere Geometrie Aufga 51 S. 189
Welche Vektoren sind voneinander linear abhängig?

$$\vec{p} = \Spvek{3;7}; \vec{q} = \Spvek{5;9} ; \vec{r} = \Spvek{4.5;10.5}$$
}{Lösung: \LoesungsRaumLang{$\vec{p}$ und $\vec{r}$, denn $\vec{r} = 1.5\cdot{}\vec{p}$}}

\aufgabeML{%% Frommenwilere Geometrie Aufga 51 S. 189
Welche Vektoren sind voneinander linear abhängig?

$$\vec{p} = \Spvek{1;4;3}; \vec{q} = \Spvek{-2;-8;6} ; \vec{r} = \Spvek{0.5;2;-1.5}$$
}{Lösung: \LoesungsRaumLang{$\vec{q}$ und $\vec{r}$, denn $\vec{q} = -4\cdot{}\vec{r}$}}




\end{document}
