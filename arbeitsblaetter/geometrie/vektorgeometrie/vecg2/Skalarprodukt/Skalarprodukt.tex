%%
%% Meta: TI nSpire Einführung
%%       Ziel: Damit die Grundoperationen damit durchgeführt werden können.
%%             Damit man sich an den Rechner gewöhnt.
%%

\input{bbwLayoutPage}

%%%%%%%%%%%%%%%%%%%%%%%%%%%%%%%%%%%%%%%%%%%%%%%%%%%%%%%%%%%%%%%%%%

\usepackage{amssymb} %% für \blacktriangleright
\renewcommand{\metaHeaderLine}{Skalarprodukt}
\renewcommand{\arbeitsblattTitel}{Vektorgeometrie in $\mathbb{R}^3$}

\begin{document}%%
\arbeitsblattHeader{}

\newcounter{aufgabennummer}
\setcounter{aufgabennummer}{1}


\newcommand\aufgabeML[2]{\begin{samepage}%%
\textbf{Aufgabe \arabic{aufgabennummer}:}\,\,\\
#1%%\\  \TRAINER{#2}
%%
\TRAINER{#2}%%abplz{5.2}
\noTRAINER{\mmPapierBisEndeSeite}
%%\end{samepage}
\stepcounter{aufgabennummer}%%
\end{samepage}%%
}%%


\section{Skalarprodukt}

 Es bezeichne jeweils $a = |\vec{a}|$.


\subsection{Formel zum Skalarprodukt}


\aufgabeML{ Gegeben sind die Vektoren $\vec{p}$ und $\vec{q}$ mit
$$p=37\hspace{11mm}q=42\hspace{1mm}\text{ und } \angle (\vec{p},\vec{q}) = 18\degre$$}%%
{Berechnen Sie das Skalarprodukt: $$\vec{p}\circ\vec{q} = \LoesungsRaum{37\cdot{}42\cdot\cos(18\degre)\approx 1477.9418}$$}


\aufgabeML{
Gegeben sind die Vektoren $\vec{r}$ und $\vec{s}$ mit: 
$$r=20\hspace{11mm}s=30\hspace{1mm}\text{ und } \angle
(\vec{r},\vec{s}) = 0\degre$$}{$$\vec{r}\circ\vec{s} = \LoesungsRaum{20\cdot{}30\cdot\cos(0\degre)=600}$$}


\aufgabeML{Berechnen Sie den Winkel zwischen $\vec{a}$ und $\vec{b}$.
$$\vec{a}\circ\vec{b}=18\hspace{11mm}a=9\hspace{11mm}b=8$$}{$$\angle(\vec{a},\vec{b})= \LoesungsRaum{=\arccos\left(\frac{\vec{a}\circ\vec{b}}{a\cdot{}b}\right)
  = \arccos{}\left(\frac{18}{9\cdot{}8}\right)\approx 75.52\degre}$$}

\aufgabeML{Berechnen Sie den Winkel zwischen $\vec{a}$ und $\vec{b}$.
$$\vec{a}\circ\vec{b}=56\hspace{11mm}a=7\hspace{11mm}b=8$$}{$$\angle(\vec{a},\vec{b})= \LoesungsRaum{=\arccos\left(\frac{\vec{a}\circ\vec{b}}{a\cdot{}b}\right)
  = \arccos{}\left(\frac{56}{7\cdot{}8}\right)\approx 0\degre}$$}



\aufgabeML{Berechnen Sie den Winkel zwischen $\vec{a}$ und $\vec{b}$.
$$\vec{a}\circ\vec{b}=0\hspace{11mm}a=1.41421\hspace{11mm}b=1.73205$$}{$$\angle(\vec{a},\vec{b})= \LoesungsRaum{=\arccos\left(\frac{\vec{a}\circ\vec{b}}{a\cdot{}b}\right)
  = \arccos{}\left(\frac{0}{a\cdot{}b}\right)\approx 90\degre}$$}

\aufgabeML{Berechnen Sie den Wert des Skalarproduktes, wenn bekannt
  isd, dass der Winkel zwischen den zwei Vektoren $\\vec{u}$ und
  $\vec{v}$ $45\degre$ beträgt und zusätzlich folgendes gilt:
 $$|\vec{u}|= 3 \text{ und } |\vec{v}|=4$$
}{...}


\aufgabeML{Seien $\vec{u}$ und $\vec{v}$ zwei Vektoren. Berechnen Sie
den Zwischenwinkel, wenn gilt:

a) $\vec{u}\circ{}\vec{v} = -u\cdot{}v$

b) $ \frac12ab = \vec{u}\circ{}\vec{v} $


}{...}

\aufgabeML{Zeigen Sie, mit der Definition des Skalarproduktes über den Cosinus, dass gilt:

$$\vec{u}\circ{}\vec{u} = u^2$$}{

Es gilt $$\vec{u}\circ{}\vec{v} = uv\cdot{}\cos(\gamma)$$ mit
$\gamma$  = Zwischenwinkel zwischen $\vec{u}$ und $\vec{v}$.

Wenn die Vektoren kollinear sind, so ist $\gamma=0$ und $\cos(\gamma)
= 1$

Somit ist

$$\vec{u}\circ{}\vec{u} = u\cdot{}u\cdot{}\cos(0\degre) = u\cdot{}u = u^2$$
}


\aufgabeML{%%
Gegeben sind die Vektoren $\vec{u} = \Spvek{2;0;-1}$, $\vec{v}
= \Spvek{1;-1;1}$ und $\vec{w} = \Spvek{0;1;-2}$.

Berechnen Sie mit dem Taschenrechner die folgende Summe:
$$(4\vec{u}\circ{}\vec{v}) + (\vec{u}\circ{}2\vec{w})
= \LoesungsRaum{8}$$
Tipp: \texttt{dotp} berechnet das Skalarprodukt auf dem tiNSpire.
}{%%
TR
}%% end AufgabeML





\aufgabeML{Für das Skalarproduk gilt das Distributivgesetz
folgendermaßen:

$$\vec{a}\circ{} (\vec{u} + \vec{v}) =  \vec{a}\circ{} \vec{u}
+ \vec{a}\circ{}\vec{v} $$
Berechnen Sie mit diesem Wissen den Winkel zwischen den Vektoren
$\vec{q}$ und $\vec{r}$, wenn folgendes gegeben ist:

$$q = 9 \text{, } r=2 \text{ und } \vec{r}\circ{}(\vec{q}-3\vec{r}) = 5$$
}{%% Lösungsteil
Es folgt:

$$\vec{r}\circ{}\vec{q} - 3\vec{r}\circ{}\vec{r} = 5$$

Wegen
$$\vec{r}\circ{}\vec{r} = r^2 = 4$$
gilt

$$\vec{r}\circ{}\vec{q} - 3\cdot{}4 = 5.$$

Daher:
$$\vec{r}\circ{}\vec{q} = 17 = rq\cdot{}\cos(\alpha)$$

Setzen wir unser $rq=2\cdot{}9=18$ ein, so erhalten wir:
$$17 = 18\cos{\alpha}$$

Nach $\alpha$ auflösen:
$$\alpha=\arccos\left(\frac{17}{18}\right) \approx 19.1881\degre$$

}

\subsection{Skalarprodukt aus gegebenen Vektoren berechnen}


\aufgabeML{Berechnen Sie $\vec{r}\circ\vec{s}$:
$$\vec{r} = \Spvek{6;-5} \hspace{11mm}\vec{s}=\Spvek{-6;-4}$$}%%
{$$\vec{r}\circ{}\vec{s} = \LoesungsRaum{-16}$$}

\aufgabeML{Berechnen Sie $\vec{r}\circ\vec{s}$:
$$\vec{r} = \Spvek{6;6;-3} \hspace{11mm} \vec{s}=\Spvek{-0.5;5;9}$$}%%
{$$\vec{r}\circ{}\vec{s} = \LoesungsRaum{0 \hspace{11mm}(=-3+30-27)}$$}
\TRAINER{\newpage}


\aufgabeML{%%
Bestimmen Sie das Skalarprodukt der Vektoren $\vec{u}$ und $\vec{v}$:
%% RoMi
$$\vec{u} = \Spvek{3;-4;6} \text{ und } \vec{v} = \Spvek{2;3;-1}$$
}{%%
}%% end AufgabeML



\aufgabeML{%%
Gegeben sind die Vektoren $\vec{a} = \Spvek{1;2}$,
$\vec{b}=\Spvek{-3;2}$, $\vec{c}=\Spvek{-4;-1}$ und $\vec{d}
= \Spvek{-1;3}$.

Berechnen Sie $(\vec{a} - 2\vec{c})\cdot{}(3\vec{b}-\vec{d})$
}{%%
 Lösung: $-60$
}%% end AufgabeML


\subsection{Parameter bestimmen, wenn Winkel gegeben}



\aufgabeML{Bestimmen Sie den Parameter $x$ von Hand so, dass die beiden
Vektoren $\vec{u}$ und $\vec{v}$ einen Winkel von 45 Grad
einschließen:
$$\vec{u} = \Spvek{1;1;0} \hspace{11mm} \vec{v}=\Spvek{2;1;x}$$}%%
{$$t = \LoesungsRaum{\pm 2}$$ }%% end aufgabeML
\TRAINER{..., denn $\cos{45\degre} = \frac{\sqrt2}2$ und dies ist
gleich $\frac{\vec{u}\circ\vec{v}}{uv} = \frac{1+2}{\sqrt{2}\cdot{}\sqrt{x^2+5}}$}



\aufgabeML{Bestimmen Sie den Parameter $s$ mit dem Taschenrechner so, dass die beiden
Vektoren $\vec{a}$ und $\vec{b}$ einen Winkel von 60 Grad
einschließen:
$$\vec{a} = \Spvek{2;3;4} \hspace{11mm} \vec{b}=\Spvek{5;s;8}$$}%%
{$$s = \LoesungsRaum{\frac{\sqrt{222691}-504}{7}\approx{}-4.5855}$$}
\TRAINER{Lösung mit TR: a und b definieren, dananach cos(60) =
dotp(a,b) / (norm(a) * norm(b)).}




\aufgabeML{%%
Berechnen Sie $x$ so, dass $\vec{a} = \Spvek{x;3;2}$ und $\vec{b}
= \Spvek{-4.5; 0; 1.5}$ einen Winkel von $62\degre$ einschließen.
}{%%
$x=-1.22$ (mit TR)
}%% end AufgabeML

\subsection{Winkel bestimmen}
\subsubsection{von Hand}

\aufgabeML{Bestimmen Sie von Handden Winkel zwischen Sie $\vec{u}$ und $ \vec{v}$:
$$\vec{u} = \Spvek{-1;3;5.5}\hspace{11mm}\vec{v}=\Spvek{2;-6;-11}$$}%%
{$$\angle(\vec{u},\vec{v}) = \LoesungsRaum{180\degre}$$
}%% end AufgabeML


\aufgabeML{Bestimmen Sie von Hand den Winkel zwischen Sie $\vec{u}$ und $ \vec{v}$:
$$\vec{u} = \Spvek{1;1;9} \hspace{11mm}\vec{v}= \Spvek{0.5;\frac12;4.5}$$}%%
{$$\angle(\vec{u},\vec{v}) = \LoesungsRaum{0\degre}$$}


\aufgabeML{Bestimmen Sie von Hand den Winkel zwischen Sie $\vec{u}$ und $ \vec{v}$:
$$\vec{u} = \Spvek{0;3;4} \hspace{11mm}\vec{v}= \Spvek{\sqrt{11};5;0}$$}%%
{$$\angle(\vec{u},\vec{v}) = \LoesungsRaum{60\degre}$$}

\aufgabeML{Bestimmen Sie von Hand den Winkel zwischen Sie $\vec{u}$ und $ \vec{v}$:
$$\vec{u} = \Spvek{-1;2;5} \hspace{11mm}\vec{v}= \Spvek{2;1;0}$$}%%
{$$\angle(\vec{u},\vec{v}) = \LoesungsRaum{90\degre}$$}

\subsubsection{mit Taschenrechner}

\aufgabeML{Bestimmen Sie den Winkel zwischen Sie $\vec{u}$ und $ \vec{v}$:
$$\vec{u} = \Spvek{-1;-2;-3} \hspace{11mm} \vec{v}=\Spvek{4;5;-6}$$}%%
{$$\angle(\vec{u},\vec{v}) = \LoesungsRaum{\arccos\left(\frac{2\cdot{}\sqrt{22}}{77}\right) \approx 83.00\degre}$$}


\aufgabeML{Bestimmen Sie den Winkel zwischen Sie $\vec{u}$ und $ \vec{v}$:
$$\vec{u} = \Spvek{-12;-7.5;-3.4} \hspace{11mm} \vec{v}=\Spvek{-4.5;2.4;3.8}$$}%%
{$$\angle(\vec{u},\vec{v}) = \LoesungsRaum{\arccos\left(\frac{2\cdot{}\sqrt{22}}{77}\right) \approx 75.6\degre}$$}

\subsection{Winkel zwischen Koordinatenachsen}



\aufgabeML{%%i
Welche drei Winkel bildet der Vektor $\vec{v}$ je mit den drei
Koordinatenachsen? Berechnen Sie von Hand!

$$\vec{v} = \Spvek{3\cdot{}\sqrt{3}; 0; 3}$$
}{%%
$x$-Achse: 30 Grad

$y$-Achse: 90 Grad

$z$-Achse: 60 Grad
}%% end AufgabeML



\aufgabeML{%%i
Ein Ortsvektor $\vec{r}$ mit Länge $v$ schließt mit der $x$- und der
$y$-Achse je einen Winkel von $\frac{\pi}3$ ein.

Bestimmen Sie von Hand den Winkel zwischen $\vec{r}$ und der $z$-Achse.


}{%%
Der Winkel misst $\frac{\pi}4$ (=$45\degre$ =$135\degre$, doch beim Winkel zwischen
zwei Geraden wird immer der kleinere genommen), egal, wie lang der Vektor ist.
}%% end AufgabeML


\subsubsection{Aufgaben mit Taschenrechner}



\aufgabeML{%%i
Welche Winkel bildet der Vektor $\vec{v}$ mit den drei
Koordinatenachsen?

$$\vec{v} = \Spvek{4;-11;1}$$
}{%%
Mit der $x$-Achse: 70.093 Grad

Mit der $y$-Achse: 20.547 Grad

Mit der $z$-Achse: 85.117 Grad
}%% end AufgabeML

\aufgabeML{%%i
Welche Winkel bildet der Vektor $\vec{v}$ mit den drei
Koordinatenachsen?

$$\vec{v} = \Spvek{6.5; 5.5; 4}$$
}{%%
Mit der $x$-Achse: 46.3 Grad

Mit der $y$-Achse: 54.2 Grad

Mit der $z$-Achse: 64.8 Grad
}%% end AufgabeML


\aufgabeML{%%i
Ein Ortsvektor bildet mit der $x$-Achse einen Winkel von $38\degre$
und mit der $y$-Achse einen von $117\degre$. Berechnen Sie den Winkel
mit der $z$-Achse.
}{%%
$\alpha_z = 65.427\degre$
}%% end AufgabeML



\aufgabeML{Berechnen Sie den Winkel $\alpha$ zwischen der $xy$-Ebene und dem
Vektor $\vec{q}$:

$\vec{q}=\Spvek{-3;6;2}$}%%
{$$\alpha=\LoesungsRaum{\arccos(\frac{\vec{q}\circ{}\vec{p}}{qp})\text{
mit p = Projektion auf xy-Ebene (-3;6;0)} \approx 16.6015\degre}$$
}


\subsection{Orthogonale Vektoren}
(Skalarprodukt = 0)
\subsubsection{Aufgaben von Hand}

\aufgabeML{
Gegeben sind Vektoren $\vec{a}$ und $\vec{b}$ mit
$$a=11.3\hspace{11mm}b=17.853\hspace{1mm}\text{ und } \angle
(\vec{a},\vec{b}) = 90\degre$$}{$$\vec{a}\circ\vec{b} = \LoesungsRaum{11.3\cdot{}17.853\cdot\cos(90\degre)=0}$$}

\aufgabeML{Bestimmen Sie den Parameter $t$ von Hand so, dass die beiden
Vektoren $\vec{p}$ und $\vec{q}$ einen Winkel von $\frac{\pi}2$
einschließen:
$$\vec{p} = \Spvek{1;t;2} \hspace{11mm} \vec{q}=\Spvek{3;1;t}$$}%%
{$$t = \LoesungsRaum{-1}$$ }%% end aufgabeML
\TRAINER{..., denn bei 90 Grad ist das Skalarprodukt = 0 und somit
muss $3t+3 = 0$ sein. Dies geht nur, wenn $t=-1$}



\aufgabeML{%%i
Bestimmen Sie sämtliche WVektoren, die senkrecht zum Vektor $\vec{v}$
stehen und die selbe Länge wie $\vec{v}$ haben:

$$\vec{v} = \Spvek{-3; 1}$$
}{%%
$\Spvek{1;3}$ und $\Spvek{-1;-3}$
}%% end AufgabeML

\aufgabeML{%%i
Bestimmen Sie den Parameter $n$, sodass die beiden folgenden Vektoren
senkrecht zueinaned stehen:

$$\Spvek{n;1;-3} \hspace{11mm}  \Spvek{4;-2;n}$$
}{%%
$n=2$
}%% end AufgabeML


\aufgabeML{%%i
Bestimmen Sie den Parameter $p$ so, adass die beiden Vektoren
senkrecht aufeinander stehen:

$$\vec{u} = \Spvek{p;2;3} \hspace{33mm}  \vec{v} = \Spvek{1;0;-2}$$
}{%%
$p=6$
}%% end AufgabeML


\aufgabeML{%%i
Bestimmen Sie den Parameter $p$ so, adass die beiden Vektoren
senkrecht aufeinander stehen:

$$\vec{u} = \Spvek{2p;2;-4} \hspace{33mm}  \vec{v} = \Spvek{p;-8;p}$$

(Tipp beim Lösen der quadratischen Gleichung: Zweiklammeransatz)
}{%%
$p_1=-2, p_2=4$
}%% end AufgabeML



\aufgabeML{%%i
Bestimmen Sie $x$ und $z$ so, dass der Vektor $\vec{r}=Spvek{x;3;z}$
senkrecht auf den beiden Vektoren steht:

$$\vec{u} = \Spvek{5;4;9} \hspace{33mm}  \vec{v} = \Spvek{2;5;-1.5}$$

(Tipp beim Lösen der quadratischen Gleichung: Zweiklammeransatz)
}{%%
$x=-6, z=2$
}%% end AufgabeML

\aufgabeML{%%i
Bestimmen Sie den Parameter $y$ so, dass das Dreieck $\triangle ABC$
bei $\beta (= \angle ABC)$ rechtwinklig wird:

$$A=(2|1|3); B=(4|y|1) \text{ und } C=(2|4|0)$$
}{%%
Lösungen: $y=2$ oder $y=3$

Weg:

$\vec{a} = \overrightarrow{BC}=\Spvek{-2;4-y;-1}$

$\vec{b} =\overrightarrow{AB} = \Spvek{2;y-1;-2}$

Skalarprodukt über die Komponenten bestimmen und = 0 setzen.
}%% end AufgabeML



\subsubsection{Aufgaben mit Taschenrechner}
\aufgabeML{%%i
Bestimmen Sie sämntliche Vektoren der Länge 6, die senkrecht zu
$\vec{u}$ und zu $\vec{v}$ stehen:

$$\vec{u} = \Spvek{2;1;-2}\hspace{11mm} \vec{v} = \Spvek{-1;0;4}$$
}{%%
Lösung: $\pm \Spvek{3.3; -4.94; 0.82}$
}%% end AufgabeML


\aufgabeML{%%i
Berechnen Sie $k$ so, dass $\vec{u}$ und $\vec{v}$ einen Winkel von
$90\degre$ einschließen:

$$\vec{u} = \Spvek{-1;k+2;3} \hspace{11mm} \vec{v} = \Spvek{2; k-3; k}$$
}{%%
$L_k = \{-4; 2\}$
}%% end AufgabeML







































\aufgabeML{%%i

}{%%

}%% end AufgabeML

\aufgabeML{%%i

}{%%

}%% end AufgabeML

\aufgabeML{%%i

}{%%

}%% end AufgabeML

\aufgabeML{%%i

}{%%

}%% end AufgabeML

\aufgabeML{%%i

}{%%

}%% end AufgabeML




\end{document}
