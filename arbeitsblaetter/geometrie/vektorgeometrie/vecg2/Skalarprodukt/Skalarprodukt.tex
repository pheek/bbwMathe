%%
%% Meta: TI nSpire Einführung
%%       Ziel: Damit die Grundoperationen damit durchgeführt werden können.
%%             Damit man sich an den Rechner gewöhnt.
%%

\input{bmsLayoutPage}

%%%%%%%%%%%%%%%%%%%%%%%%%%%%%%%%%%%%%%%%%%%%%%%%%%%%%%%%%%%%%%%%%%

\usepackage{amssymb} %% für \blacktriangleright
\renewcommand{\metaHeaderLine}{Arbeitsblatt}
\renewcommand{\arbeitsblattTitel}{Vektorgeometrie in $\mathbb{R}^3$: Skalarprodukt}

\begin{document}%%
\arbeitsblattHeader{}

\newcounter{aufgabennummer}
\setcounter{aufgabennummer}{1}


\newcommand\aufgabeML[2]{\begin{samepage}%%
\textbf{Aufgabe \arabic{aufgabennummer}:}\,\,\\
#1%%\\  \TRAINER{#2}
%%
\TRAINER{#2}%%abplz{5.2}
\noTRAINER{\mmPapierBisEndeSeite}
%%\end{samepage}
\stepcounter{aufgabennummer}%%
\end{samepage}%%
}%%


\section{Skalarprodukt}

 Es bezeichne jeweils $a = |\vec{a}|$.


\subsection{Formel zum Skalarprodukt}


\aufgabeML{ Gegeben sind die Vektoren $\vec{p}$ und $\vec{q}$ mit
$$p=37\hspace{11mm}q=42\hspace{1mm}\text{ und } \angle
(\vec{p},\vec{q}) = 18\degre$$
Berechnen Sie das Skalarprodukt!
}%%
{$$\vec{p}\circ\vec{q} = \LoesungsRaum{37\cdot{}42\cdot\cos(18\degre)\approx 1477.9418}$$}


\aufgabeML{
Gegeben sind die Vektoren $\vec{r}$ und $\vec{s}$ mit: 
$$r=20\hspace{11mm}s=30\hspace{1mm}\text{ und } \angle
(\vec{r},\vec{s}) = 0\degre$$
$$\vec{r}\circ\vec{s} = \LoesungsRaum{20\cdot{}30\cdot\cos(0\degre)=600}$$}{}


\aufgabeML{Berechnen Sie den Winkel zwischen $\vec{a}$ und $\vec{b}$.
$$\vec{a}\circ\vec{b}=18\hspace{11mm}a=9\hspace{11mm}b=8$$
$$\angle(\vec{a},\vec{b})= \LoesungsRaum{=\arccos\left(\frac{\vec{a}\circ\vec{b}}{a\cdot{}b}\right)
  = \arccos{}\left(\frac{18}{9\cdot{}8}\right)\approx 75.52\degre}$$}{}

\aufgabeML{Berechnen Sie den Winkel zwischen $\vec{a}$ und $\vec{b}$.
$$\vec{a}\circ\vec{b}=56\hspace{11mm}a=7\hspace{11mm}b=8$$

$$\angle(\vec{a},\vec{b})= \LoesungsRaum{
=\arccos  \left( \frac{\vec{a}\circ\vec{b}}{a\cdot{}b}  \right)
= \arccos \left( \frac{56}{7\cdot{}8} \right)
= 0\degre }$$}{}



\aufgabeML{Berechnen Sie den Winkel zwischen $\vec{a}$ und $\vec{b}$.
$$\vec{a}\circ\vec{b}=0\hspace{11mm}a=1.41421\hspace{11mm}b=1.73205$$

$$\angle(\vec{a},\vec{b})= \LoesungsRaum{=\arccos\left(\frac{\vec{a}\circ\vec{b}}{a\cdot{}b}\right)
  = \arccos{}\left(\frac{0}{a\cdot{}b}\right)\approx 90\degre}$$}{}

\aufgabeML{Berechnen Sie den Wert des Skalarproduktes, wenn bekannt
  ist, dass der Winkel zwischen den zwei Vektoren $\vec{u}$ und
  $\vec{v}$ $45\degre$ beträgt und zusätzlich folgendes gilt:
 $$|\vec{u}|= 3 \text{ und } |\vec{v}|=4$$
}{$$\vec{u} \circ \vec{v} = u\cdot{}v\cdot{}\cos(45\degre) = 3\cdot{}4\cdot{}\frac{\sqrt{2}}{2} = 6\cdot{}\sqrt{2} \approx 8.49$$}


\aufgabeML{Seien $\vec{u}$ und $\vec{v}$ zwei Vektoren. Berechnen Sie
den Zwischenwinkel, wenn gilt:

a) $\vec{u}\circ{}\vec{v} = -u\cdot{}v$

b) $ \frac12uv = \vec{u}\circ{}\vec{v} $


}{a) $\arccos(-1) = 180\degre = \pi$

b) $\arccos{0.5} = 60\degre = \frac{\pi}{3}$}

\aufgabeML{Zeigen Sie, mit der Definition des Skalarproduktes über den Cosinus, dass gilt:

$$\vec{u}\circ{}\vec{u} = u^2$$}{

Es gilt $$\vec{u}\circ{}\vec{v} = uv\cdot{}\cos(\gamma)$$ mit
$\gamma$  = Zwischenwinkel zwischen $\vec{u}$ und $\vec{v}$.

Wenn die Vektoren kollinear sind, so ist $\gamma=0$ und $\cos(\gamma)
= 1$

Somit ist

$$\vec{u}\circ{}\vec{u} = u\cdot{}u\cdot{}\cos(0\degre) = u\cdot{}u = u^2$$
}


\aufgabeML{%%
Gegeben sind die Vektoren $\vec{u} = \Spvek{2;0;-1}$, $\vec{v}
= \Spvek{1;-1;1}$ und $\vec{w} = \Spvek{0;1;-2}$.

Berechnen Sie mit dem Taschenrechner die folgende Summe:
$$(4\vec{u}\circ{}\vec{v}) + (\vec{u}\circ{}2\vec{w})
= \LoesungsRaum{8}$$
Tipp: \texttt{dotp} berechnet das Skalarprodukt auf dem tiNSpire.
}{%%
TR
}%% end AufgabeML





\aufgabeML{Für das Skalarprodukt gilt das Distributivgesetz
folgendermaßen:

$$\vec{a}\circ{} (\vec{u} + \vec{v}) =  \vec{a}\circ{} \vec{u}
+ \vec{a}\circ{}\vec{v} $$
Berechnen Sie mit diesem Wissen den Winkel zwischen den Vektoren
$\vec{q}$ und $\vec{r}$, wenn folgendes gegeben ist:

$$q = 9 \text{, } r=2 \text{ und } \vec{r}\circ{}(\vec{q}-3\vec{r}) = 5$$
}{%% Lösungsteil
Es folgt:

$$\vec{r}\circ{}\vec{q} - 3\vec{r}\circ{}\vec{r} = 5$$

Wegen
$$\vec{r}\circ{}\vec{r} = r^2 = 4$$
gilt

$$\vec{r}\circ{}\vec{q} - 3\cdot{}4 = 5.$$

Daher:
$$\vec{r}\circ{}\vec{q} = 17 = rq\cdot{}\cos(\alpha)$$

Setzen wir unser $rq=2\cdot{}9=18$ ein, so erhalten wir:
$$17 = 18\cos{\alpha}$$

Nach $\alpha$ auflösen:
$$\alpha=\arccos\left(\frac{17}{18}\right) \approx 19.1881\degre$$

}

\subsection{Skalarprodukt aus gegebenen Vektoren berechnen}


\aufgabeML{Berechnen Sie $\vec{r}\circ\vec{s}$:
$$\vec{r} = \Spvek{6;-5} \hspace{11mm}\vec{s}=\Spvek{-6;-4}$$}%%
{$$\vec{r}\circ{}\vec{s} = \LoesungsRaum{-16}$$}

\aufgabeML{Berechnen Sie $\vec{r}\circ\vec{s}$:
$$\vec{r} = \Spvek{6;6;-3} \hspace{11mm} \vec{s}=\Spvek{-0.5;5;9}$$}%%
{$$\vec{r}\circ{}\vec{s} = \LoesungsRaum{0 \hspace{11mm}(=-3+30-27)}$$}
\TRAINER{\newpage}


\aufgabeML{%%
Bestimmen Sie das Skalarprodukt der Vektoren $\vec{u}$ und $\vec{v}$:
%% RoMi
$$\vec{u} = \Spvek{3;-4;6} \text{ und } \vec{v} = \Spvek{2;3;-1}$$
}{%%
}%% end AufgabeML



\aufgabeML{%%
Gegeben sind die Vektoren $\vec{a} = \Spvek{1;2}$,
$\vec{b}=\Spvek{-3;2}$, $\vec{c}=\Spvek{-4;-1}$ und $\vec{d}
= \Spvek{-1;3}$.

Berechnen Sie $(\vec{a} - 2\vec{c})\cdot{}(3\vec{b}-\vec{d})$
}{%%
 Lösung: $-60$
}%% end AufgabeML


\subsection{Parameter bestimmen, wenn Winkel gegeben}



\aufgabeML{Bestimmen Sie den Parameter $x$ von Hand so, dass die beiden
Vektoren $\vec{u}$ und $\vec{v}$ einen Winkel von 45 Grad
einschließen:
$$\vec{u} = \Spvek{1;1;0} \hspace{11mm} \vec{v}=\Spvek{2;1;x}$$}%%
{$$t = \LoesungsRaum{\pm 2}$$ }%% end aufgabeML
\TRAINER{..., denn $\cos{45\degre} = \frac{\sqrt2}2$ und dies ist
gleich $\frac{\vec{u}\circ\vec{v}}{uv} = \frac{1+2}{\sqrt{2}\cdot{}\sqrt{x^2+5}}$}



\aufgabeML{Bestimmen Sie den Parameter $s$ mit dem Taschenrechner so, dass die beiden
Vektoren $\vec{a}$ und $\vec{b}$ einen Winkel von 60 Grad
einschließen:
$$\vec{a} = \Spvek{2;3;4} \hspace{11mm} \vec{b}=\Spvek{5;s;8}$$}%%
{$$s = \LoesungsRaum{\frac{\sqrt{222691}-504}{7}\approx{}-4.5855}$$}
\TRAINER{Lösung mit TR: a und b definieren, danach cos(60) =
dotp(a,b) / (norm(a) * norm(b)).}




\aufgabeML{%%
Berechnen Sie $x$ so, dass $\vec{a} = \Spvek{x;3;2}$ und $\vec{b}
= \Spvek{-4.5; 0; 1.5}$ einen Winkel von $62\degre$ einschließen.
}{%%
$x=-1.22$ (mit TR)
}%% end AufgabeML




\subsection{Winkel bestimmen}
\subsubsection{von Hand}

\aufgabeML{Bestimmen Sie von Hand den Winkel zwischen Sie $\vec{u}$ und $ \vec{v}$:
$$\vec{u} = \Spvek{-1;3;5.5}\hspace{11mm}\vec{v}=\Spvek{2;-6;-11}$$}%%
{$$\angle(\vec{u},\vec{v}) = \LoesungsRaum{180\degre}$$
}%% end AufgabeML


\aufgabeML{Bestimmen Sie von Hand den Winkel zwischen Sie $\vec{u}$ und $ \vec{v}$:
$$\vec{u} = \Spvek{1;1;9} \hspace{11mm}\vec{v}= \Spvek{0.5;\frac12;4.5}$$}%%
{$$\angle(\vec{u},\vec{v}) = \LoesungsRaum{0\degre}$$}


\aufgabeML{Bestimmen Sie von Hand den Winkel zwischen Sie $\vec{u}$ und $ \vec{v}$:
$$\vec{u} = \Spvek{0;3;4} \hspace{11mm}\vec{v}= \Spvek{\sqrt{11};5;0}$$}%%
{$$\angle(\vec{u},\vec{v}) = \LoesungsRaum{60\degre}$$}

\aufgabeML{Bestimmen Sie von Hand den Winkel zwischen Sie $\vec{u}$ und $ \vec{v}$:
$$\vec{u} = \Spvek{-1;2;5} \hspace{11mm}\vec{v}= \Spvek{2;1;0}$$}%%
{$$\angle(\vec{u},\vec{v}) = \LoesungsRaum{90\degre}$$}

\subsubsection{mit Taschenrechner}

\aufgabeML{Bestimmen Sie den Winkel zwischen Sie $\vec{u}$ und $ \vec{v}$:
$$\vec{u} = \Spvek{-1;-2;-3} \hspace{11mm} \vec{v}=\Spvek{4;5;-6}$$}%%
{$$\angle(\vec{u},\vec{v}) = \LoesungsRaum{\arccos\left(\frac{2\cdot{}\sqrt{22}}{77}\right) \approx 83.00\degre}$$}


\aufgabeML{Bestimmen Sie den Winkel zwischen Sie $\vec{u}$ und $ \vec{v}$:
$$\vec{u} = \Spvek{-12;-7.5;-3.4} \hspace{11mm} \vec{v}=\Spvek{-4.5;2.4;3.8}$$}%%
{$$\angle(\vec{u},\vec{v}) = \LoesungsRaum{\arccos\left(\frac{2\cdot{}\sqrt{22}}{77}\right) \approx 75.6\degre}$$}

\subsection{Winkel zwischen Koordinatenachsen}



\aufgabeML{%%i
Welche drei Winkel bildet der Vektor $\vec{v}$ je mit den drei
Koordinatenachsen? Berechnen Sie von Hand!

$$\vec{v} = \Spvek{3\cdot{}\sqrt{3}; 0; 3}$$
}{%%
$x$-Achse: 30 Grad

$y$-Achse: 90 Grad

$z$-Achse: 60 Grad
}%% end AufgabeML



\aufgabeML{%%i
Ein Ortsvektor $\vec{r}$ mit Länge $v$ schließt mit der $x$- und der
$y$-Achse je einen Winkel von $\frac{\pi}3$ ein.

Bestimmen Sie von Hand den Winkel zwischen $\vec{r}$ und der $z$-Achse.


}{%%
Der Winkel misst $\frac{\pi}4$ (=$45\degre$ =$135\degre$, doch beim Winkel zwischen
zwei Geraden wird immer der kleinere genommen), egal, wie lang der Vektor ist.
}%% end AufgabeML



\aufgabeML{%%
Ein Ortsvektor bildet mit der $y$-Achse einen Winkel von $45\degre$
und mit der $z;$-Achse einen Winkel von $60\degre$. Er hat eine Länge
von 6 Einheiten. Bestimmen Sie die Komponenten dieses Ortsvektors.
}{%%

$a_1 = \Spvek{3;3\sqrt{2};3}$

$a_2 = \Spvek{-3;3\sqrt{2};3}$
}%%

\subsubsection{Aufgaben mit Taschenrechner}



\aufgabeML{%%i
Welche Winkel bildet der Vektor $\vec{v}$ mit den drei
Koordinatenachsen?

$$\vec{v} = \Spvek{4;-11;1}$$
}{%%
Mit der $x$-Achse: 70.093 Grad

Mit der $y$-Achse: 20.547 Grad

Mit der $z$-Achse: 85.117 Grad
}%% end AufgabeML

\aufgabeML{%%i
Welche Winkel bildet der Vektor $\vec{v}$ mit den drei
Koordinatenachsen?

$$\vec{v} = \Spvek{6.5; 5.5; 4}$$
}{%%
Mit der $x$-Achse: 46.3 Grad

Mit der $y$-Achse: 54.2 Grad

Mit der $z$-Achse: 64.8 Grad
}%% end AufgabeML


\aufgabeML{%%i
Ein Ortsvektor bildet mit der $x$-Achse einen Winkel von $38\degre$
und mit der $y$-Achse einen von $117\degre$. Berechnen Sie den Winkel
mit der $z$-Achse.
}{%%
$\alpha_z = 65.427\degre$
}%% end AufgabeML



\aufgabeML{Berechnen Sie den Winkel $\alpha$ zwischen der $xy$-Ebene und dem
Vektor $\vec{q}$:

$\vec{q}=\Spvek{-3;6;2}$}%%
{$$\alpha=\LoesungsRaum{\arccos(\frac{\vec{q}\circ{}\vec{p}}{qp})\text{
mit p = Projektion auf xy-Ebene (-3;6;0)} \approx 16.6015\degre}$$
}


\subsection{Orthogonale Vektoren}
(Skalarprodukt = 0)
\subsubsection{Aufgaben von Hand}


\aufgabeML{Gegeben sind die Vektoren
$$\vec{a} = \Spvek{5;-3}; \vec{b} = \Spvek{b_x; b_y}.$$


a) Bestimmen Sie alle Vektoren, die zu $\vec{a}$ senkrecht stehen und den Betrag $a$
haben.

b) Bestimmen Sie alle Vektoren, die zu $\vec{a}$ senkrecht stehen.

c) Bestimmen Sie alle Vektoren, die zu $\vec{b}$ senkrecht stehen.
}{Lösungen: \\

a) $$L = \LoesungsRaumLen{55mm}{\left\{\Spvek{3;5}; \Spvek{-3;-5}\right\}}$$

b)  $$L = \LoesungsRaumLen{55mm}{\left\{k\cdot{}\Spvek{3;5}| k\in\mathbb{R}\backslash\{0\}\right\}}$$

c)  $$L = \LoesungsRaumLen{55mm}{\left\{k\cdot{}\Spvek{-b_y;+b_x}| k\in\mathbb{R}\backslash\{0\}\right\}}$$
 
}%% end aufgabeML



\aufgabeML{
Gegeben sind Vektoren $\vec{a}$ und $\vec{b}$ mit
$$a=11.3\hspace{11mm}b=17.853\hspace{1mm}\text{ und } \angle
(\vec{a},\vec{b}) = 90\degre$$ Berechnen Sie das Skalarprodukt.}{$$\vec{a}\circ\vec{b} = \LoesungsRaum{11.3\cdot{}17.853\cdot\cos(90\degre)=0}$$}

\aufgabeML{Bestimmen Sie den Parameter $t$ von Hand so, dass die beiden
Vektoren $\vec{p}$ und $\vec{q}$ einen Winkel von $\frac{\pi}2$
einschließen:
$$\vec{p} = \Spvek{1;t;2} \hspace{11mm} \vec{q}=\Spvek{3;1;t}$$}%%
{$$t = \LoesungsRaum{-1}$$ }%% end aufgabeML
\TRAINER{..., denn bei 90 Grad ist das Skalarprodukt = 0 und somit
muss $3t+3 = 0$ sein. Dies geht nur, wenn $t=-1$}



\aufgabeML{%%i
Bestimmen Sie sämtliche Vektoren, die senkrecht zum Vektor $\vec{v}$
stehen und die selbe Länge wie $\vec{v}$ haben:

$$\vec{v} = \Spvek{-3; 1}$$
}{%%
$\Spvek{1;3}$ und $\Spvek{-1;-3}$
}%% end AufgabeML

\aufgabeML{%%i
Bestimmen Sie den Parameter $n$, sodass die beiden folgenden Vektoren
senkrecht zueinander stehen:

$$\Spvek{n;1;-3} \hspace{11mm}  \Spvek{4;-2;n}$$
}{%%
$n=2$
}%% end AufgabeML


\aufgabeML{%%i
Bestimmen Sie den Parameter $p$ so, dass die beiden Vektoren
senkrecht aufeinander stehen:

$$\vec{u} = \Spvek{p;2;3} \hspace{33mm}  \vec{v} = \Spvek{1;0;-2}$$
}{%%
$p=6$
}%% end AufgabeML


\aufgabeML{%%i
Bestimmen Sie den Parameter $p$ so, dass die beiden Vektoren
senkrecht aufeinander stehen:

$$\vec{u} = \Spvek{2p;2;-4} \hspace{33mm}  \vec{v} = \Spvek{p;-8;p}$$

(Tipp beim Lösen der quadratischen Gleichung: Zweiklammeransatz)
}{%%
$p_1=-2, p_2=4$
}%% end AufgabeML



\aufgabeML{%%i
Bestimmen Sie $x$ und $z$ so, dass der Vektor $\vec{r}=\Spvek{x;3;z}$
senkrecht auf den beiden Vektoren steht:

$$\vec{u} = \Spvek{5;4;9} \hspace{33mm}  \vec{v} = \Spvek{2;5;-1.5}$$

(Tipp beim Lösen der quadratischen Gleichung: Zweiklammeransatz)
}{%%
$x=-6, z=2$
}%% end AufgabeML

\aufgabeML{%%i
Bestimmen Sie den Parameter $y$ so, dass das Dreieck $\triangle ABC$
bei $\beta (= \angle ABC)$ rechtwinklig wird:

$$A=(2|1|3); B=(4|y|1) \text{ und } C=(2|4|0)$$
}{%%
Lösungen: $y=2$ oder $y=3$

Weg:

$\vec{a} = \overrightarrow{BC}=\Spvek{-2;4-y;-1}$

$\vec{b} =\overrightarrow{AB} = \Spvek{2;y-1;-2}$

Skalarprodukt über die Komponenten bestimmen und = 0 setzen.
}%% end AufgabeML


\aufgabeML{%%
Gegeben sind die Punkte $A=(-5|1)$ und $B=(3|-3)$. Berechnen Sie die
Koordinaten des Punktes $C$, sodass $\overline{AC}$ einerseits genau
10 Einheiten lang wird und andererseits $\overline{AC}$ die Hypotenuse
im Dreieck $\triangle{} ABC$ wird.

Tipp 1: Machen Sie eine Skizze.

Tipp 2: Die entstehende quadratische Gleichung kann von Hand gelöst
werden, wenn die Gleichung in Grundform vereinfacht wird und danach
der Zweiklammeransatz verwendet wird.
}{%%
1. Lösung: $\overrightarrow{OC} = \Spvek{5;1}$

2. Lösung: $\overrightarrow{OC} = \Spvek{1;-7}$

}%% end AufgabeML



\subsubsection{Aufgaben mit Taschenrechner}
\aufgabeML{%%i
Bestimmen Sie sämtliche Vektoren der Länge 6, die senkrecht zu
$\vec{u}$ und zu $\vec{v}$ stehen:

$$\vec{u} = \Spvek{2;1;-2}\hspace{11mm} \vec{v} = \Spvek{-1;0;4}$$
}{%%
Lösung: $\pm \Spvek{3.3; -4.94; 0.82}
= \frac{\pm\sqrt{53}}{53}\cdot{}\Spvek{24; -36; 6}$
}%% end AufgabeML


\aufgabeML{%%i
Berechnen Sie $k$ so, dass $\vec{u}$ und $\vec{v}$ einen Winkel von
$90\degre$ einschließen:

$$\vec{u} = \Spvek{-1;k+2;3} \hspace{11mm} \vec{v} = \Spvek{2; k-3; k}$$
}{%%
$L_k = \{-4; 2\}$
}%% end AufgabeML


\subsection{Rechteck, Quadrat, Quader und Würfel}


\subsubsection{Rechteck}

\aufgabeML{%%i
Berechnen Sie den Winkel, unter dem sich die Diagonalen eines 15 cm
langen und 11 cm breiten Rechtecks schneiden.
}{%%
$72.508\degre$ bzw. $107.492\degre$
}%% end AufgabeML



\aufgabeML{%%i
Lösen Sie diese Aufgabe mit Hilfe des \textit{Solvers} und der
Funktion \texttt{dotp} mit Ihrem Taschenrechner:

Die Vektoren $\vec{a} = \Spvek{x;2;6}$ und $\vec{b}=\Spvek{3;6;z}$
spannen ein Rechteck auf, das doppelt so lang wie breit ist.

a) Sind beide Seiten ($a$ bzw. $b$) als die Längsseite des Rechtecks
denkbar? Begründen Sie Ihre Antwort.

b) Berechnen Sie den Flächeninhalt des Rechtecks.
}{%%
a) Das Gleichungssystem beinhaltet für beide Fälle die Gleichung
$\vec{a}\circ{}\vec{b} = 0$.

Mit Gleichung $|\vec{a}| = 2|\vec{b}|$ erhalten wir die Lösung
$x=\frac{-39}2$ und $z=\frac{31}4$. Wohingegen für $2|\vec{a}| =
|\vec{b}|$ keine reelle Lösung möglich ist (quadratische Gleichung
ohne Lösung).

b) Mit den gefundenen $x$ und $z$ Werten kann die Fläche mit
$|\vec{a}|\cdot{}|\vec{b}| = 210.125$ errechnet werden.
}%% end AufgabeML


\subsubsection{Quadrat}

\aufgabeML{%%i
Von einem Quadrat $ABCD$ kennt man die Ecke $B=(3|4|5)$ und den
Mittelpunkt $M=(1|3|2)$. Die Ecke $A$ liegt in der
$xy$-Ebene. Berechnen Sie mit dem Taschenrechner die Ecken $D$ und $A$.
}{%%
$D=(-1|2|-1)$

Für $A$ gibt es zwei Lösungen:

$$A_1 \approx (2.652 | 5.697 | 0)$$
und 
$$A_2 \approx (4.148 | 2.703 | 0)$$
}%% end AufgabeML




\aufgabeML{%%i
Die beiden Vektoren $\vec{a}=\Spvek{-3;1;z}$ und
$\vec{b}=\Spvek{1;y;3}$ spannen ein Quadrat auf. Wie groß ist der
Flächeninhalt und wie groß ist der Umfang dieses Quadrates?
}{%%

$(y_1 = -1.5; z_1=1.5); U_1=14; A_1=12.25$

$(y_2 = 0.75; z_2=0.75); U_2=13; A_2=10.563$
}%% end AufgabeML

\subsubsection{Quader}


\aufgabeML{Gegeben ist ein Quader mit der Grundfläche
$1.7 \times 2.4\text{m}^2$
und der Höhe $0.8 \text{m}$.\\
Welchen Winkel $\alpha$ schließt die Körperdiagonale zur Grundseite
ein? Berechnen Sie dies mit dem Taschenrechner.
}%%
{$$\alpha = \LoesungsRaum{
\alpha =
\arccos\left(\frac{\vec{a}\circ{}\vec{p}}{a\cdot{}p}\right)
= \arccos\left(\frac{2.4^2 + 1.7^2}{\sqrt{2.4^2+1.7^2+0.8^2}\cdot{}\sqrt{2.4^2+1.7^2}}\right)
\approx 15.21676\degre
}$$
Dabei ist $\vec{a}$ die Quaderdiagonale (Körperdiagonale) und $\vec{p}$
dessen Projektion auf die Grundfläche (= $xy$-Ebene).}


\aufgabeML{%%i
Im abgebildeten Quader $RSTUVWXY$ ist $\overline{RS}$ = 18 cm,
$\overline{ST}$ = 9 cm und $\overline{RV}$ = 13 cm. $M$ bezeichne den
Schnittpunkt der Raumdiagonalen. Berechnen Sie mit dem Taschenrechner die Winkel $\angle RMS$
und $\angle WMX$

\bbwCenterGraphic{100mm}{./img/Quader_RSTUVWXY.png}

}{%%

$\angle RMS = 97.407\degre$

$\angle WMX = 44.129\degre$
}%% end AufgabeML


\subsubsection{Würfel}

%% entspricht Frommenwiler S. 203 Aufg 109
%
\aufgabeML{Berechnen Sie den Winkel zwischen der Körperdiagonalen und
den Seitenflächen eines Würfels mit Hilfe des Skalarproduktes}{}%%
\TRAINER{$$\cos(\alpha) = \frac{\vec{a}\circ{}\vec{b}}{a\cdot{}
b} \Longrightarrow  \alpha
= \arccos\left(\frac{\sqrt{6}}{3}\right)\text{, mit } \vec{a}
= \Spvek{\sqrt{2};1}\text{ und } \vec{b} = \Spvek{\sqrt{2};1}$$
$$\Longrightarrow \alpha\approx 35.26\degre$$}%% end TRAINER




\subsection{Dreieck und Pyramide}

%% Maturaprüfung 2020 Aufgabe 2b):

\aufgabeML{Berechnen Sie im Dreieck mit den Eckpunkten $A=(3|1|4)$,
$B=(1|3|2)$ und $C=(2|2|1)$ den Winkel $\alpha = \angle BAC$:
}%%
{$$\alpha=\LoesungsRaum{\arccos\left(\frac{\vec{AB}\circ{}\vec{AC}}{|\vec{AB}|\cdot{}|\vec{AC}|}\right)
\approx 79.2\degre}$$
}





\aufgabeML{%%i
Von einem Dreieck $\triangle ABC$ sind die Koordinaten der drei
Eckpunkte bekannt:

$A=(3|-2|5), B=(7|6|10) \text{ und } C=(5|4|6)$

Berechnen Sie mit dem Taschenrechner den Flächeninhalt des Dreiecks $ABC$:
}{%%
1. Einen Winkel berechnen (\zB $\beta$): $\beta = 28.8...\degre$

2. Zwei anstoßende Seiten berechnen

$a=4.899...$ und $c=10.247...$

3. Allgemeine Flächenformel Fläche =
$\frac12\cdot{}a\cdot{}c\cdot{}\sin(\beta) \approx 12.1$
}%% end AufgabeML





\aufgabeML{%%i
In einer quadratischen Pyramide sind alle Kanten gleich
lang. Berechnen Sie den Winkel zwischen ...

a) ... einer Seiten- und einer Grundkante,

b) zwei gegenüberliegenden Seitenkanten und

c) zwei benachbarten Seitenkanten.
}{%%
a) 60 Grad; b) 90 Grad; c) 60 Grad
}%% end AufgabeML


\subsection{vermischte Aufgaben}


\aufgabeML{%%i
Die Ecken eines regulären Tetraeders liegen auf einer Kugel. Wie groß
ist das Volumenverhältnis

\begin{center}Kugel : Tetraeder\end{center}

Lösen Sie die Aufgabe a) mit dem Taschenrechner b) von Hand.

}{%% Lösungsteil
Es seien die Ecken des Tetraeders:
$A=(0|0|0); B=(1|0|0)$
Sei $C$ in der $xy$-Ebene, also
$C = (\frac12|\frac{\sqrt{3}}2|0)$

Die Koordinaten des Mittelpunktes der Grundseite ist
$G=(\frac12|\frac{\sqrt{3}}6|0)$

Die Spitze $S$ habe die Koordinaten $(\frac12|\frac{\sqrt{3}}6|z_S)$.

Ohne Beschränkung der Allgemeinheit ist die Seitenlänge = 1. Somit
auch $|\overline{AS}| = 1$, was zu $z_S = \frac{\sqrt{6}}{3}$ führt.

Der Radius $r$ ist also der Abstand jeder Ecke zum Mittelpunkt $M$ und es
gilt:

$|\overline{AM}| = |\overline{MS}|$

Dies bedeutet, die beiden folgenden Vektoren sind gleich lang ($z_M$ =
$z$-Koordinate des Mittelpunktes):

$$\Spvek{\frac12 ; \frac{\sqrt{3}}6; z_M} \text{ ist gleich lang wie
}  \Spvek{0;0;z_S-z_M}$$

Da wir $z_S = \frac{\sqrt{6}}3$ bereits kennen, können wir $z_M$
berechnen:

$$z_M = \frac{\sqrt{6}}{12}$$

Somit ist der Kugelradius = Höhe - $z$-Koordinate des Mittelpunktes

$r = h-\frac{\sqrt{6}}{12} = \frac{\sqrt{6}}4$

Kugelvolumen $V_K = \frac43\pi\cdot{}r^3 = \frac{\pi\sqrt{6}}{8}$

Pyramidenvolumen $V_T = \frac13 G\cdot{}h = \frac{\sqrt{3}\cdot{}\sqrt{6}}{3\cdot{}4\cdot{}3 = \frac{\sqrt{2}}{12}}$

Das gesuchte Volumen ist daher:

Kugel : Tetraeder = $$\frac{\pi\sqrt{6}}{8} : \frac{\sqrt{2}}{12} = \frac{3\sqrt{3}\cdot{}\pi}2$$
}%% end AufgabeML

\aufgabeML{%%i
Gegeben sind die Punkte $A=(12|12|-6)$ und $B=(15|6|3)$.

Welche Punkte $C$ der $xy$-Ebene bilden mit den Punkten $A$ und $B$ ein
gleichseitiges Dreieck?

Verwenden Sie die Vektor-Operationen des Taschenrechners.
}{%%
$C_1 = (4.2 | 6.6 | 0)$

$C_2 = (21 | 15 | 0)$
}%% end AufgabeML

\aufgabeML{%%i
Gegeben sind die beiden Vektoren $\vec{u} = \Spvek{1;2;x}$ und
$\vec{v} = \Spvek{3;-1;1}$.

Berechnen Sie $x$ so, dass der von $\vec{u}$ und $\vec{v}$
eingeschlossene Winkel minimal wird. Wie groß ist dieser minimale
Winkel?

}{%%
Zielfunktion $\varphi = f(x) := \arccos\left(\frac{3-2+x}{\sqrt{5+x^2}\cdot{}\sqrt{11}}\right)$

\texttt{fmin(f(x),x)} mit dem Taschenrechner liefert $x_{min} = 5$ und
somit $f(5) = 70.714\degre$.
}%% end AufgabeML

\aufgabeML{%%
Gegeben sind zwei Punkte $A=(7|8|-9)$ und $B=(-2|4|11)$.

Gesucht ist ein Punkt $P$ in der $xy$-Ebene (also $z_P=0$) der zum einen
von $A$ und von $B$ gleich weit entfernt ist und zum anderen soll der Winkel $\angle APB$ maximal werden.
}{%%

Möglicher Lösungsweg TR:
Definiere die Ortsvektoren $\overrightarrow{r_A} := \Spvek{7;8;-9}$
und $\overrightarrow{r_B} := \Spvek{-2;4;11}$ und den gesuchten
Ortsvektor $\overrightarrow{r_P} := \Spvek{x;y;0}$.
Somit müssen die beiden Vektoren $r_1=\overrightarrow{PA}$ und
$r_2=\overrightarrow{PB}$ gleich lang sein:

Ich löse diese Gleichung ($|r_1|=|r_2|$) mit dem \textit{Solver} nach $y$ auf und
erhalte die Geradengleichung $y=\frac{-(18x-53)}{8}$.

Nun definiere ich $y:=\frac{-(18x-53)}{8}$, sodass das $y$ an das $x$
gebunden ist.

Die Zielfunktion (Winkel) definiere ich als zweiparametrige Funktion:

$$f(x,y)
:= \arccos\left(\frac{\overrightarrow{r_1}\circ\overrightarrow{r_2}}{|\overrightarrow{r_1}|\cdot{}|\overrightarrow{r_2}|}\right)
=  \arccos\left(\frac{\text{dotp}(\overrightarrow{r_1}, \overrightarrow{r_2})}{\text{norm}(\overrightarrow{r_1})\cdot{}\text{norm}(\overrightarrow{r_2})}\right)$$

Mittels \texttt{fmax} maximiere ich die den Winkel durch den Parameter
$x$:

\texttt{fmax(f(x,y),x)} und ich erhalte das $x$:

$$x = \frac{125}{194}$$

Dies eingesetzt in die Funktionsgleichung der linearen Funktion
(Geradengleichung) in der $xy$-Ebene ergibt:

$$y = \frac{-(18x-53)}{8} = \frac{-(18\cdot{}\frac{125}{194}-53)}{8}
= \frac{502}{97}$$

Der Gesuchte Punkt ist also

$$P = \left(\frac{125}{194}\middle|\frac{502}{97}\middle|0\right)$$
}%%


\end{document}
