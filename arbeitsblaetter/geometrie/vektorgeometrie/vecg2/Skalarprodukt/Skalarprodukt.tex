%%
%% Meta: TI nSpire Einführung
%%       Ziel: Damit die Grundoperationen damit durchgeführt werden können.
%%             Damit man sich an den Rechner gewöhnt.
%%

\input{bbwLayoutPage}

%%%%%%%%%%%%%%%%%%%%%%%%%%%%%%%%%%%%%%%%%%%%%%%%%%%%%%%%%%%%%%%%%%

\usepackage{amssymb} %% für \blacktriangleright
\renewcommand{\metaHeaderLine}{Skalarprodukt}
\renewcommand{\arbeitsblattTitel}{Vektorgeometrie in $\mathbb{R}^3$}

\begin{document}%%
\arbeitsblattHeader{}

\newcounter{aufgabennummer}
\setcounter{aufgabennummer}{1}


\newcommand\aufgabeML[2]{\begin{samepage}%%
\textbf{Aufgabe \arabic{aufgabennummer}:}\,\,\\
#1%%\\  \TRAINER{#2}
%%
\TRAINER{#2}%%abplz{5.2}
\noTRAINER{\mmPapierBisEndeSeite}
%%\end{samepage}
\stepcounter{aufgabennummer}%%
\end{samepage}%%
}%%


\section{Skalarprodukt}

 Es bezeichne jeweils $a = |\vec{a}|$.


\subsection{Formel zum Skalarprodukt}


\aufgabeML{ Gegeben sind die Vektoren $\vec{p}$ und $\vec{q}$ mit
$$p=37\hspace{11mm}q=42\hspace{1mm}\text{ und } \angle (\vec{p},\vec{q}) = 18\degre$$}%%
{Berechnen Sie das Skalarprodukt: $$\vec{p}\circ\vec{q} = \LoesungsRaum{37\cdot{}42\cdot\cos(18\degre)\approx 1477.9418}$$}


\aufgabeML{
Gegeben sind die Vektoren $\vec{r}$ und $\vec{s}$ mit: 
$$r=20\hspace{11mm}s=30\hspace{1mm}\text{ und } \angle
(\vec{r},\vec{s}) = 0\degre$$}{$$\vec{r}\circ\vec{s} = \LoesungsRaum{20\cdot{}30\cdot\cos(0\degre)=600}$$}


\aufgabeML{Berechnen Sie den Winkel zwischen $\vec{a}$ und $\vec{b}$.
$$\vec{a}\circ\vec{b}=18\hspace{11mm}a=9\hspace{11mm}b=8$$}{$$\angle(\vec{a},\vec{b})= \LoesungsRaum{=\arccos\left(\frac{\vec{a}\circ\vec{b}}{a\cdot{}b}\right)
  = \arccos{}\left(\frac{18}{9\cdot{}8}\right)\approx 75.52\degre}$$}

\aufgabeML{Berechnen Sie den Winkel zwischen $\vec{a}$ und $\vec{b}$.
$$\vec{a}\circ\vec{b}=56\hspace{11mm}a=7\hspace{11mm}b=8$$}{$$\angle(\vec{a},\vec{b})= \LoesungsRaum{=\arccos\left(\frac{\vec{a}\circ\vec{b}}{a\cdot{}b}\right)
  = \arccos{}\left(\frac{56}{7\cdot{}8}\right)\approx 0\degre}$$}



\aufgabeML{Berechnen Sie den Winkel zwischen $\vec{a}$ und $\vec{b}$.
$$\vec{a}\circ\vec{b}=0\hspace{11mm}a=1.41421\hspace{11mm}b=1.73205$$}{$$\angle(\vec{a},\vec{b})= \LoesungsRaum{=\arccos\left(\frac{\vec{a}\circ\vec{b}}{a\cdot{}b}\right)
  = \arccos{}\left(\frac{0}{a\cdot{}b}\right)\approx 90\degre}$$}

\aufgabeML{Berechnen Sie den Wert des Skalarproduktes, wenn bekannt
  isd, dass der Winkel zwischen den zwei Vektoren $\\vec{u}$ und
  $\vec{v}$ $45\degre$ beträgt und zusätzlich folgendes gilt:
 $$|\vec{u}|= 3 \text{ und } |\vec{v}|=4$$
}{...}


\aufgabeML{Seien $\vec{u}$ und $\vec{v}$ zwei Vektoren. Berechnen Sie
den Zwischenwinkel, wenn gilt:

a) $\vec{u}\circ{}\vec{v} = -u\cdot{}v$

b) $ \frac12ab = \vec{u}\circ{}\vec{v} $


}{...}

\aufgabeML{Zeigen Sie, mit der Definition des Skalarproduktes über den Cosinus, dass gilt:

$$\vec{u}\circ{}\vec{u} = u^2$$}{

Es gilt $$\vec{u}\circ{}\vec{v} = uv\cdot{}\cos(\gamma)$$ mit
$\gamma$  = Zwischenwinkel zwischen $\vec{u}$ und $\vec{v}$.

Wenn die Vektoren kollinear sind, so ist $\gamma=0$ und $\cos(\gamma)
= 1$

Somit ist

$$\vec{u}\circ{}\vec{u} = u\cdot{}u\cdot{}\cos(0\degre) = u\cdot{}u = u^2$$
}


\aufgabeML{%%
Gegeben sind die Vektoren $\vec{u} = \Spvek{2;0;-1}$, $\vec{v}
= \Spvek{1;-1;1}$ und $\vec{w} = \Spvek{0;1;-2}$.

Berechnen Sie mit dem Taschenrechner die folgende Summe:
$$(4\vec{u}\circ{}\vec{v}) + (\vec{u}\circ{}2\vec{w})
= \LoesungsRaum{8}$$
Tipp: \texttt{dotp} berechnet das Skalarprodukt auf dem tiNSpire.
}{%%
TR
}%% end AufgabeML





\aufgabeML{Für das Skalarproduk gilt das Distributivgesetz
folgendermaßen:

$$\vec{a}\circ{} (\vec{u} + \vec{v}) =  \vec{a}\circ{} \vec{u}
+ \vec{a}\circ{}\vec{v} $$
Berechnen Sie mit diesem Wissen den Winkel zwischen den Vektoren
$\vec{q}$ und $\vec{r}$, wenn folgendes gegeben ist:

$$q = 9 \text{, } r=2 \text{ und } \vec{r}\circ{}(\vec{q}-3\vec{r}) = 5$$
}{%% Lösungsteil
Es folgt:

$$\vec{r}\circ{}\vec{q} - 3\vec{r}\circ{}\vec{r} = 5$$

Wegen
$$\vec{r}\circ{}\vec{r} = r^2 = 4$$
gilt

$$\vec{r}\circ{}\vec{q} - 3\cdot{}4 = 5.$$

Daher:
$$\vec{r}\circ{}\vec{q} = 17 = rq\cdot{}\cos(\alpha)$$

Setzen wir unser $rq=2\cdot{}9=18$ ein, so erhalten wir:
$$17 = 18\cos{\alpha}$$

Nach $\alpha$ auflösen:
$$\alpha=\arccos\left(\frac{17}{18}\right) \approx 19.1881\degre$$

}

\subsection{Skalarprodukt aus gegebenen Vektoren berechnen}


\aufgabeML{Berechnen Sie $\vec{r}\circ\vec{s}$:
$$\vec{r} = \Spvek{6;-5} \hspace{11mm}\vec{s}=\Spvek{-6;-4}$$}%%
{$$\vec{r}\circ{}\vec{s} = \LoesungsRaum{-16}$$}

\aufgabeML{Berechnen Sie $\vec{r}\circ\vec{s}$:
$$\vec{r} = \Spvek{6;6;-3} \hspace{11mm} \vec{s}=\Spvek{-0.5;5;9}$$}%%
{$$\vec{r}\circ{}\vec{s} = \LoesungsRaum{0 \hspace{11mm}(=-3+30-27)}$$}
\TRAINER{\newpage}


\aufgabeML{%%
Bestimmen Sie das Skalarprodukt der Vektoren $\vec{u}$ und $\vec{v}$:
%% RoMi
$$\vec{u} = \Spvek{3;-4;6} \text{ und } \vec{v} = \Spvek{2;3;-1}$$
}{%%
}%% end AufgabeML



\aufgabeML{%%

}{%%

}%% end AufgabeML


\aufgabeML{%%

}{%%

}%% end AufgabeML





\end{document}
