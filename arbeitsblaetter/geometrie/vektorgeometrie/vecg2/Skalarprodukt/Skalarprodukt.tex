%%
%% Meta: TI nSpire Einführung
%%       Ziel: Damit die Grundoperationen damit durchgeführt werden können.
%%             Damit man sich an den Rechner gewöhnt.
%%

\input{bbwLayoutPage}

%%%%%%%%%%%%%%%%%%%%%%%%%%%%%%%%%%%%%%%%%%%%%%%%%%%%%%%%%%%%%%%%%%

\usepackage{amssymb} %% für \blacktriangleright
\renewcommand{\metaHeaderLine}{Skalarprodukt}
\renewcommand{\arbeitsblattTitel}{Vektorgeometrie in $\mathbb{R}^3$}

\begin{document}%%
\arbeitsblattHeader{}

\newcounter{aufgabennummer}
\setcounter{aufgabennummer}{1}


\newcommand\aufgabeML[2]{\begin{samepage}%%
\textbf{Aufgabe \arabic{aufgabennummer}:}\,\,\\
#1\\  #2

\abplz{5.2}
\end{samepage}
\stepcounter{aufgabennummer}%%
}%%


\section{Skalarprodukt}

 Es bezeichne jeweils $a = |\vec{a}|$.

\subsection{Aufgaben ohne Taschenrechner}

\aufgabeML{Gegeben sind die Vektoren
$$\vec{a} = \Spvek{5;-3}; \vec{b} = \Spvek{b_x; b_y}.$$


a) Bestimmen Sie alle Vektoren, die zu $\vec{a}$ senkrecht stehen und den Betrag $a$
haben.

b) Bestimmen Sie alle Vektoren, die zu $\vec{a}$ senkrecht stehen.

c) Bestimmen Sie alle Vektoren, die zu $\vec{b}$ senkrecht stehen.
}{Lösungen: \\

a) $$L = \LoesungsRaumLen{55mm}{\left\{\Spvek{3;5}; \Spvek{-3;-5}\right\}}$$

b)  $$L = \LoesungsRaumLen{55mm}{\left\{k\cdot{}\Spvek{3;5}| k\in\mathbb{R}\backslash\{0\}\right\}}$$

c)  $$L = \LoesungsRaumLen{55mm}{\left\{k\cdot{}\Spvek{-b_y;+b_x}| k\in\mathbb{R}\backslash\{0\}\right\}}$$
 
}%% end aufgabeML


\aufgabeML{ Gegeben sind die Vektoren $\vec{p}$ und $\vec{q}$ mit
$$p=37\hspace{11mm}q=42\hspace{1mm}\text{ und } \angle (\vec{p},\vec{q}) = 18\degre$$}%%
{Berechnen Sie das Skalarprodukt: $$\vec{p}\circ\vec{q} = \LoesungsRaum{37\cdot{}42\cdot\cos(18\degre)\approx 1477.9418}$$}


\aufgabeML{\\
Gegeben sind die Vektoren $\vec{r}$ und $\vec{s}$ mit: 
$$r=20\hspace{11mm}s=30\hspace{1mm}\text{ und } \angle
(\vec{r},\vec{s}) = 0\degre$$}{$$\vec{r}\circ\vec{s} = \LoesungsRaum{20\cdot{}30\cdot\cos(0\degre)=600}$$}

\aufgabeML{\\
Gegeben sind Vektoren $\vec{a}$ und $\vec{b}$ mit
$$a=11.3\hspace{11mm}b=17.853\hspace{1mm}\text{ und } \angle
(\vec{a},\vec{b}) = 90\degre$$}{$$\vec{a}\circ\vec{b} = \LoesungsRaum{11.3\cdot{}17.853\cdot\cos(90\degre)=0}$$}

\aufgabeML{Berechnen Sie den Winkel zwischen $\vec{a}$ und $\vec{b}$.
$$\vec{a}\circ\vec{b}=18\hspace{11mm}a=9\hspace{11mm}b=8$$}{$$\angle(\vec{a},\vec{b})= \LoesungsRaum{=\arccos\left(\frac{\vec{a}\circ\vec{b}}{a\cdot{}b}\right)
  = \arccos{}\left(\frac{18}{9\cdot{}8}\right)\approx 75.52\degre}$$}


\TRAINER{\newpage}

\aufgabeML{Berechnen Sie den Winkel zwischen $\vec{a}$ und $\vec{b}$.
$$\vec{a}\circ\vec{b}=56\hspace{11mm}a=7\hspace{11mm}b=8$$}{$$\angle(\vec{a},\vec{b})= \LoesungsRaum{=\arccos\left(\frac{\vec{a}\circ\vec{b}}{a\cdot{}b}\right)
  = \arccos{}\left(\frac{56}{7\cdot{}8}\right)\approx 0\degre}$$}

\aufgabeML{Berechnen Sie den Winkel zwischen $\vec{a}$ und $\vec{b}$.
$$\vec{a}\circ\vec{b}=0\hspace{11mm}a=1.41421\hspace{11mm}b=1.73205$$}{$$\angle(\vec{a},\vec{b})= \LoesungsRaum{=\arccos\left(\frac{\vec{a}\circ\vec{b}}{a\cdot{}b}\right)
  = \arccos{}\left(\frac{0}{a\cdot{}b}\right)\approx 90\degre}$$}

\aufgabeML{Berechnen Sie $\vec{r}\circ\vec{s}$:
$$\vec{r} = \Spvek{6;-5} \hspace{11mm}\vec{s}=\Spvek{-6;-4}$$}%%
{$$\vec{r}\circ{}\vec{s} = \LoesungsRaum{-16}$$}

\aufgabeML{Berechnen Sie $\vec{r}\circ\vec{s}$:
$$\vec{r} = \Spvek{6;6;-3} \hspace{11mm} \vec{s}=\Spvek{-0.5;5;9}$$}%%
{$$\vec{r}\circ{}\vec{s} = \LoesungsRaum{0 \hspace{11mm}(=-3+30-27)}$$}
\TRAINER{\newpage}

\aufgabeML{Bestimmen Sie den Winkel zwischen Sie $\vec{u}$ und $ \vec{v}$:
$$\vec{u} = \Spvek{-1;-2;-3} \hspace{11mm} \vec{v}=\Spvek{4;5;-6}$$}%%
{$$\angle(\vec{u},\vec{v}) = \LoesungsRaum{\arccos\left(\frac{2\cdot{}\sqrt{22}}{77}\right) \approx 83.00\degre}$$}

\aufgabeML{Bestimmen Sie den Winkel zwischen Sie $\vec{u}$ und $ \vec{v}$:
$$\vec{u} = \Spvek{-1;3;5.5}\hspace{11mm}\vec{v}=\Spvek{2;-6;-11}$$}%%
{$$\angle(\vec{u},\vec{v}) = \LoesungsRaum{180\degre}$$}

\aufgabeML{Bestimmen Sie den Winkel zwischen Sie $\vec{u}$ und $ \vec{v}$:
$$\vec{u} = \Spvek{1;1;9} \hspace{11mm}\vec{v}= \Spvek{0.5;\frac12;4.5}$$}%%
{$$\angle(\vec{u},\vec{v}) = \LoesungsRaum{0\degre}$$}


\aufgabeML{Bestimmen Sie den Parameter $t$ von Hand so, dass die beiden
Vektoren $\vec{p}$ und $\vec{q}$ einen Winkel von 90 Grad
einschließen:
$$\vec{p} = \Spvek{1;t;2} \hspace{11mm} \vec{q}=\Spvek{3;1;t}$$}%%
{$$t = \LoesungsRaum{-1}$$ }%% end aufgabeML
\TRAINER{..., denn bei 90 Grad ist das Skalarprodukt = 0 und somit
muss $3t+3 = 0$ sein. Dies geht nur, wenn $t=-1$}



\aufgabeML{Bestimmen Sie den Parameter $x$ von Hand so, dass die beiden
Vektoren $\vec{u}$ und $\vec{v}$ einen Winkel von 45 Grad
einschließen:
$$\vec{u} = \Spvek{1;1;0} \hspace{11mm} \vec{v}=\Spvek{2;1;x}$$}%%
{$$t = \LoesungsRaum{\pm 2}$$ }%% end aufgabeML
\TRAINER{..., denn $\cos{45\degre} = \frac{\sqrt2}2$ und dies ist
gleich $\frac{\vec{u}\circ\vec{v}}{uv} = \frac{1+2}{\sqrt{2}\cdot{}\sqrt{x^2+5}}$}
\newpage


\aufgabeML{Bestimmen Sie den Parameter $s$ mit dem Taschenrechner so, dass die beiden
Vektoren $\vec{a}$ und $\vec{b}$ einen Winkel von 60 Grad
einschließen:
$$\vec{a} = \Spvek{2;3;4} \hspace{11mm} \vec{b}=\Spvek{5;s;8}$$}%%
{$$s = \LoesungsRaum{\frac{\sqrt{222691}-504}{7}\approx{}-4.5855}$$}
\TRAINER{Lösung mit TR: a und b definieren, dananach cos(60) =
dotp(a,b) / (norm(a) * norm(b)).}

%% entspricht Frommenwiler S. 203 Aufg 109
%
\aufgabeML{Berechnen Sie den Winkel zwischen der Körperdiagonalene und
den Seitenflächen eines Würfels mit Hilfe des Skalarproduktes}{}%%
\TRAINER{$$\cos(\alpha) = \frac{\vec{a}\circ{}\vec{b}}{a\cdot{}
b} \Longrightarrow  \alpha
= \arccos\left(\frac{\sqrt{6}}{3}\right)\text{, mit } \vec{a}
= \Spvek{\sqrt{2};1}\text{ und } \vec{b} = \Spvek{\sqrt{2};1}$$
$$\Longrightarrow \alpha\approx 35.26\degre$$}%% end TRAINER
\noTRAINER{\newpage}

\aufgabeML{Gegeben ist ein Quader mit der Grundfläche
$1.7 \times 2.4\text{m}^2$
und der Höhe $0.8 \text{m}$.\\
Welchen Winkel $\alpha$ schließt die Körperdiagonale zur Grundseite ein?
}%%
{$$\alpha = \LoesungsRaum{
\alpha =
\arccos\left(\frac{\vec{a}\circ{}\vec{p}}{a\cdot{}p}\right)
= \arccos\left(\frac{2.4^2 + 1.7^2}{\sqrt{2.4^2+1.7^2+0.8^2}\cdot{}\sqrt{2.4^2+1.7^2}}\right)
\approx 15.21676\degre
}$$
Dabei ist $\vec{a}$ die Quaderdiagonale (Körperdiagonale) und $\vec{p}$
dessen Projektion auf die Grundfläche (= $xy$-Ebene).}
\newpage

%% Maturaprüfung 2020 Aufgabe 2b):

\aufgabeML{Berechnen Sie im Dreieck mit den Eckpunkten $A=(3|1|4)$,
$B=(1|3|2)$ und $C=(2|2|1)$ den Winkel $\alpha = \angle BAC$:
}%%
{$$\alpha=\LoesungsRaum{\arccos\left(\frac{\vec{AB}\circ{}\vec{AC}}{|\vec{AB}|\cdot{}|\vec{AC}|}\right)
\approx 79.2\degre}$$
}


\aufgabeML{Berechnen Sie den Winkel $\alpha$ zwischen der $xy$-Ebene und dem
Vektor $\vec{q}$:

$\vec{q}=\Spvek{-3;6;2}$}%%
{$$\alpha=\LoesungsRaum{\arccos(\frac{\vec{q}\circ{}\vec{p}}{qp})\text{
mit p = Projektion auf xy-Ebene (-3;6;0)} \approx 16.6015\degre}$$
}






\end{document}
