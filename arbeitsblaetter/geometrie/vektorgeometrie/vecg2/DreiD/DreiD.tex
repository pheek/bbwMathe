%%
%% Meta: TI nSpire Einführung
%%       Ziel: Damit die Grundoperationen damit durchgeführt werden können.
%%             Damit man sich an den Rechner gewöhnt.
%%

\input{bbwLayoutPage}

%%%%%%%%%%%%%%%%%%%%%%%%%%%%%%%%%%%%%%%%%%%%%%%%%%%%%%%%%%%%%%%%%%

\usepackage{amssymb} %% für \blacktriangleright
\renewcommand{\metaHeaderLine}{Arbeitsblatt}
\renewcommand{\arbeitsblattTitel}{Vektorgeometrie in $\mathbb{R}^3$}

\begin{document}%%
\arbeitsblattHeader{}

\newcounter{aufgabennummer}
\setcounter{aufgabennummer}{1}


\newcommand\aufgabeML[2]{
Aufgabe \arabic{aufgabennummer}:\,\,\\
#1  #2

\abplz{4}

\stepcounter{aufgabennummer}
}


\section{Komponenten in drei D}

\aufgabeML{Gegeben sind die Vektoren
$$\vec{a}=\begin{pmatrix}-5\\3\\-7\end{pmatrix},
 \vec{b} = \begin{pmatrix}-3\\-3\\2\end{pmatrix} \text{ und }
 \vec{c} = \begin{pmatrix}0\\2\\-2\end{pmatrix}$$


Berechnen Sie $\vec{d}$ so, dass
$\vec{a}+\vec{b}+\vec{c}+\vec{d}=\vec{0}$ ergibt.
\vspace{22mm}
}{$\vec{d} = \LoesungsRaum{\begin{pmatrix}8\\-2\\7\end{pmatrix}}$}





\aufgabeML{Bestimmen Sie den Betrag des Vektors $\vec{a}$:
$$\vec{a}=\begin{pmatrix}3.5\\-6\\0\end{pmatrix}$$

%%\vspace{22mm}
}{$|\vec{a}| \approx \LoesungsRaum{6.94622199472}$}

\newpage

\aufgabeML{Gegeben sind die Punkte:
$$A= (7 | 5.5 | -3) \text{ und  } B=(5|-2|0)$$

Berechnen Sie die folgenden Vektoren und Abstände:

\vspace{22mm}
}{$$|\vec{OA}|
=\LoesungsRaum{\begin{pmatrix}7\\5.5\\-3\end{pmatrix}}$$
$$\overline{AB} \approx \LoesungsRaum{8.3217}$$}




\aufgabeML{Berechnen Sie $\vec{b}$ so, dass $\vec{b}$ den Betrag 5 hat
und dem Vektor $\vec{a}$ entgegengesetzt ist:
$$\vec{a} = \begin{pmatrix}4\\0\\-3\end{pmatrix}$$

\vspace{12mm}
}{$$\vec{b} = \LoesungsRaum{\begin{pmatrix}-0.8\\ 0 \\0.6\end{pmatrix}}$$}

\newpage

\aufgabeML{Stellen Sie den Vektor $\vec{z}$ als Linearkombination von
$\vec{a}$, $\vec{b}$ und $\vec{c}$ dar.
$$\vec{a} = \begin{pmatrix}4\\2\\-7\end{pmatrix};
  \vec{b} = \begin{pmatrix}1\\-4\\-5\end{pmatrix};
  \vec{c} = \begin{pmatrix}4\\0\\-3\end{pmatrix};
  \vec{z} = \begin{pmatrix}48\\48\\84\end{pmatrix}$$

\vspace{12mm}
}{$$\vec{z} = \LoesungsRaum{42}\cdot{}\vec{a}
+ \LoesungsRaum{9}\cdot{}\vec{b} + \LoesungsRaum{-129}\cdot{}\vec{c}$$}


\end{document}
