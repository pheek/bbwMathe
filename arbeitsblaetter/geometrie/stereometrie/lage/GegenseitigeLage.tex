%%
%% Meta: TI nSpire Einführung
%%       Ziel: Damit die Grundoperationen damit durchgeführt werden können.
%%             Damit man sich an den Rechner gewöhnt.
%%

\input{bmsLayoutPage}

%%%%%%%%%%%%%%%%%%%%%%%%%%%%%%%%%%%%%%%%%%%%%%%%%%%%%%%%%%%%%%%%%%

\usepackage{amssymb} %% für \blacktriangleright
\renewcommand{\metaHeaderLine}{Arbeitsblatt Stereometrie}
\renewcommand{\arbeitsblattTitel}{Schnitte im Raum}

\begin{document}%%
\arbeitsblattHeader{}

Graphiken Ch. Benz @ bms-w.ch

\newcounter{aufgabennummer}
\setcounter{aufgabennummer}{1}


\section{Schnitte im Raum}

\subsection{Würfel I}

Ein Würfel mit der Kantenlänge $a$ wird entlang der roten
Linie geschnitten.

Skizzieren Sie die resultierende Schnittebene und berechnen
Sie deren Kantenlängen in Abhängigkeit von $a$.

\noTRAINER{\bbwCenterGraphic{6cm}{img/A1.png}}

\TNTeop{\bbwCenterGraphic{8cm}{L1.png}}
%%%%%%%%%%%%%%%%555
\subsection{Würfel II}


Ein Würfel mit der Kantenlänge $a$ wird entlang der roten Linie
geschnitten. Die Punkte $P$ und $P'$ liegen in der Mitte der Kanten.

Skizzieren Sie die resultierende Schnittebene und berechnen
Sie deren Kantenlängen in Abhängigkeit von $a$.

\noTRAINER{\bbwCenterGraphic{6cm}{img/A2.png}}

\TNTeop{\bbwCenterGraphic{8cm}{L2.png}

$${\color{red}b} = \sqrt{a^2 + \left(\frac{a}2\right)^2}
= \sqrt{\frac{5a}{4}} = {\color{red}  \frac{\sqrt{5}}2\cdot{} a }$$
}
%%%%%%%%%%%%%%%%%%%%%%%%%%%%%%%%%%%%%%%%%%%%%%%%%%%%%%5
\subsection{Pyramide}
Eine gerade Pyramide mit quadratischer Grundfläche hat die
Grundseite $a$ und die Höhe $2a$. Dieser Pyramide ist ein Würfel
so einbeschrieben, so dass vier Ecken in der Grundfläche und
vier Ecken auf den Seitenkanten der Pyramide liegen.

a) Skizzieren Sie die resultierende Schnittebene, wenn entlang der
roten Linie geschnitten wird und berechnen Sie die Strecken der
Schnittebene in Abhängigkeit von $a$ oder $x$.

{\tiny{2018, Nullserie 1.2, Teil 1, Aufgabe 7 bzw. Strukturaufgaben SPF, S. 11: Nr. 40 (angepasst)}}

\noTRAINER{\bbwCenterGraphic{6cm}{img/A3a.png}}

\TNTeop{\bbwCenterGraphic{12cm}{L3a.png}

$${\color{red}b} = \sqrt{\left(\frac{\sqrt{2}}{2}\cdot{}a\right)^2 + (2a)^2}
= \sqrt{\frac12 a^2 + \frac82 a^2} = \sqrt{\frac92 a^2} = {\color{red}
\frac3{\sqrt{2}} \cdot a}$$
}%% end TNTeop
%%%%%%%%%%%%%%%%%%%%%%%%%%%%%%%%%%%%%%%%%%%%%%%%%%%%%%%%%%%%%%%
b) Gleich wie Aufgabe a), aber entlang der blauen Linie geschnitten.


\noTRAINER{\bbwCenterGraphic{6cm}{img/A3b.png}}

\TNTeop{\bbwCenterGraphic{12cm}{L3b.png}

$${\color{blue}b}  = \sqrt{\left(\frac12\cdot{}a\right)^2 + (2a)^2}
= \sqrt{\frac14a^2 + \frac{16}4a^2} = {\color{blue} \frac{17}2\cdot{}a}$$
}%% end TNTeop
%%%%%%%%%%%%%%%%%%%%%%%%%%%%%%%%%%%%%55
\subsection{Quader in Halbkugel}

Einer Halbkugel mit Radius $r$ wird ein Quader mit quadratischer
Grundfläche so einbeschrieben, dass er auf der Grundkreisfläche
der Halbkugel steht.

Halbkugel und Quader werden entlang der roten Linie geschnitten.

Skizzieren Sie die resultierende Schnittebene und berechnen Sie
die Strecken der Schnittebene in Abhängigkeit von $a$, $h$ oder $r$.

{\tiny{2019, Serie 1, Teil 2, Aufgabe 6 (angepasst)}}

\noTRAINER{\bbwCenterGraphic{6cm}{img/A4.png}}

\TNTeop{\bbwCenterGraphic{16cm}{L4.png}

}%% end TNT eop
%%%%%%%%%%%%%%%%%%%%%%%%%%%%%%%%%%%%%%%%%%%%%%%%%%%%%%%%%%%%%%%%
\subsection{Prisma in Halbkugel}
Einer Halbkugel mit Radius $r$ wird ein reguläres sechsseitiges
gerades Prisma mit der Grundkante $a$ einbeschrieben.

Halbkugel und Prisma werden entlang der roten Linie geschnitten.

Skizzieren Sie die resultierende Schnittebene und berechnen Sie
die Strecken der Schnittebene in Abhängigkeit von $a$, $h$ oder $r$.

{\tiny{2019, Serie 2, Teil 2, Aufgabe 5 (angepasst)}}

\noTRAINER{\bbwCenterGraphic{6cm}{img/A5.png}}

\TNTeop{\bbwCenterGraphic{16cm}{L5.png}

}%% end TNT eop
%%%%%%%%%%%%%%%%%%%%%%%%%%%%%%%%%%%%%%%%%%%%%%%%%%%%%%%%%%%
\subsection{Kugel in Kegel (Trichter)}

Eine Kugel mit Radius $r$ liegt in einem kegelförmigen Trichter.
Die Kugel berührt den Trichterdeckel. Die Mantellinie des Trichters
ist sechsmal so lang wie der Grundkreisradius $R$ des Trichters.

Trichter und Kugel werden entlang der roten Linie geschnitten.

Skizzieren Sie die resultierende Schnittebene und berechnen
Sie die Strecken der Schnittebene in Abhängigkeit von $r$ oder $R$.

{\tiny{2019, Serie 2, Teil 1, Aufgabe 7 (angepasst)}}

\noTRAINER{\bbwCenterGraphic{6cm}{img/A6.png}}

\TNTeop{\bbwCenterGraphic{16cm}{L6.png}

}%% end TNT eop
%%%%%%%%%%%%%%%%%%%%%%%%%%%%%%%%%%%%%%%%%%%%%%%%%%%%%%%%%%%%%%%
\subsection{Kugel in Pyramide}
Bei einer geraden Pyramide mit quadratischer Grundfläche haben alle
Kanten die Länge $a$. Dieser Pyramide ist eine Kugel einbeschrieben.

Pyramide und Kugel werden entlang der roten Linie geschnitten.

Skizzieren Sie die resultierende Schnittebene und berechnen
Sie die Strecken der Schnittebene in Abhängigkeit von $r$ oder $a$.

{\tiny{2018, Serie 2, Teil 2, Aufgabe 5 (angepasst)}}

\noTRAINER{\bbwCenterGraphic{6cm}{img/A7.png}}

\TNTeop{\bbwCenterGraphic{16cm}{L7.png}

$${\color{red} b} = \sqrt{a^2-\left(\frac12 a \right)^2} =
\sqrt{\frac44a^2-\frac14a^2} =
\sqrt{\frac34a^2}
= {\color{red} \frac{\sqrt{3}}{2} \cdot{} a}$$

}%% end TNT eop

\end{document}
