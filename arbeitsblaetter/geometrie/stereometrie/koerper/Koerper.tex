%%
%% Meta: TI nSpire Einführung
%%       Ziel: Damit die Grundoperationen damit durchgeführt werden können.
%%             Damit man sich an den Rechner gewöhnt.
%%

\input{bmsLayoutPage}

%%%%%%%%%%%%%%%%%%%%%%%%%%%%%%%%%%%%%%%%%%%%%%%%%%%%%%%%%%%%%%%%%%

\usepackage{amssymb} %% für \blacktriangleright
\renewcommand{\metaHeaderLine}{Arbeitsblatt Körper}
\renewcommand{\arbeitsblattTitel}{Begriffe}

\begin{document}%%
\arbeitsblattHeader{}

\noTRAINER{\bbwCenterGraphic{140mm}{img/BasicShapes.png}}
\TRAINER{\bbwCenterGraphic{80mm}{img/BasicShapes.png}}

\noTRAINER{$a= \text{ Grundkante }; h = \text{ Höhe };  r= \text{ Radius}$}

\begin{bbwFillInTabular}{|p{28mm}|p{50mm}|c|c|}\hline
Buchstabe  & Bezeichnung                     & $V$              & $S$                      \\\hline
\TRAINER{c}&\TRAINER{Kreiskegel}             & $\frac13 r^2\pi\cdot{}h$    & $r\pi\cdot{}(r+ \sqrt{h^2+r^2}) $            \\\hline
\TRAINER{B}&\TRAINER{regul. sechss. Prisma}  & $\frac32\cdot{}\sqrt{3}\cdot{}a^2\cdot{}h$       &   $3a\cdot{}\left(\sqrt{3}\cdot{}a + 2h \right)$            \\\hline
\TRAINER{E}&\TRAINER{regul. dreis. Prisma}   & $\frac{\sqrt{3}}{4}a^2\cdot{}h$        & $a\cdot{}\left(\frac{\sqrt{3}}{2}a + 3h \right) $            \\\hline
\TRAINER{A}&\TRAINER{Kugel}                  & $\frac43\pi r^3$ & $4\cdot{}\pi\cdot{}r^2$ \\\hline
\TRAINER{D}&\TRAINER{Würfel}                 & $a^3$            & $6\cdot{}a^2$            \\\hline
\TRAINER{C}&\TRAINER{Kreisyzlinder}          & $r^2\pi\cdot{}h$    & $2r\pi(r+h)$            \\\hline
\TRAINER{d}&\TRAINER{quadr. Pyramide}        & $\frac13 a^2\cdot{}h$    & $ a\cdot{}\left(a + 2\cdot{}\sqrt{\frac14a^2 + h^2}\right) $            \\\hline
\TRAINER{e}&\TRAINER{Tetraeder}              & $ \frac{\sqrt{2}}{12} a^3$   & $\sqrt{3}\cdot{} a^2$            \\\hline
\TRAINER{b}&\TRAINER{regul. sechss. Pyramide}& $\frac{\sqrt{3}}2 \cdot{}a^2 \cdot{}h$       & $3a\cdot{}\left( \frac{\sqrt{3}}2\cdot{}a + \sqrt{\frac{3a^2}{4} + h^2}\right)$            \\\hline
\end{bbwFillInTabular}%%
%%
\end{document}
