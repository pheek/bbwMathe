%%
%% Meta: TI nSpire Einführung
%%       Ziel: Damit die Grundoperationen damit durchgeführt werden können.
%%             Damit man sich an den Rechner gewöhnt.
%%

\input{bmsLayoutPage}

%%%%%%%%%%%%%%%%%%%%%%%%%%%%%%%%%%%%%%%%%%%%%%%%%%%%%%%%%%%%%%%%%%

\usepackage{amssymb} %% für \blacktriangleright
\renewcommand{\metaHeaderLine}{Arbeitsblatt Körper}
\renewcommand{\arbeitsblattTitel}{Begriffe}

\begin{document}%%
\arbeitsblattHeader{}


\begin{bbwFillInTabular}{|p{50mm}|p{50mm}|c|c|}\hline
Figur & Bezeichnung      & $V$              & $S$                      \\\hline
      & \TRAINER{Kugel}  & $\frac43\pi r^3$ & $4\cdot{}\pi\cdot{}r^2$ \\\hline
      & \TRAINER{Würfel} & $a^3$            & $6\cdot{}a^2$            \\\hline
      & \TRAINER{Kreisyzlinder} & $r^2\pi\cdot{}h$    & $2r\pi(r+h)$            \\\hline
      & \TRAINER{Kreiskegel} & $\frac13 r^2\pi\cdot{}h$    & $r\pi(r+ \sqrt{h^2+r^2}) $            \\\hline
      & \TRAINER{quadr. Pyramide} & $\frac13 a^2\cdot{}h$    & $ a\cdot{}\left(2 + \sqrt{\frac14a^2 + h^2}\right) $            \\\hline
      & \TRAINER{regul. dreis. Prisma} & $\frac{\sqrt{3}}{4}a^2\cdot{}h$
      & $\frac{\sqrt{3}}{2}a^2 + 3ah  $            \\\hline
      & \TRAINER{regul. dreis. Pyramide} & $ \frac{\sqrt{2}}{12} a^3$
      & $\sqrt{3} a^2$            \\\hline
      & \TRAINER{regul. sechss. Pyramide} & $a^2\cdot{}\frac{\sqrt{3}}2 \cdot{}h$       & $3a\cdot{}\left( \frac{a\sqrt{3}}2 + \sqrt{\frac{3a^2}{4} + h^2}\right)$            \\\hline
               
\end{bbwFillInTabular} 


\end{document}
