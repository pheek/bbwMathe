%%
%% Meta: Boxplot erstellen. Viele Beispiele
%%

\input{bbwSeite}

%%%%%%%%%%%%%%%%%%%%%%%%%%%%%%%%%%%%%%%%%%%%%%%%%%%%%%%%%%%%%%%%%%
\renewcommand{\metaHeaderLine}{Einführung Trainer}
\renewcommand{\arbeitsblattTitel}{Demo Latex Beispiele zu Koordinatensystemen}

\begin{document}%%
\arbeitsblattHeader{}

\section{Demos in tikzpicture}

Zu den folgenden Graphiken werden lediglich die bbw-tex-Layouts verwendet.

Die Befehle zur Erzeugung der Graphiken sind jeweils gleich darunter angegeben.

\subsection{Punkte und Stecken}
Mit bbw-Makros sind Stecken und Punkte im Koordinatensystem einfach zu zeichnen.

\bbwGraph{-2}{3}{-1}{3}{%%
 \bbwLine{-1,2}{2,1}{blue}%%
 \bbwDot{-1,2}{red}{east}{A}%%
 \bbwDot{2,1}{green}{west}{B}%%
 \bbwDot{1,2}{blue}{south}{C}%%
}%%

\begin{verbatim}
\bbwGraph{-2}{3}{-1}{3}{
 \bbwLine{-1,2}{2,1}{blue}
 \bbwDot{-1,2}{red}{east}{A}
 \bbwDot{2,1}{green}{west}{B}
 \bbwDot{1,2}{blue}{south}{C}
}
\end{verbatim}


\newpage
\subsection{Mischen mit TikzPictures}
Neben den bbw-Makros kann jeder tikz-draw-Befehl verwendet werden. Beachte das abschließende Semikonon (Strichpunkt ;) am Ende der Zeilen.

\bbwGraph{-3}{2}{-4}{3}{
 \bbwLine{-2,-1}{-2.5,2}{blue}
 \bbwLine{-1,0}{2,3}{red}
 \draw[thin,color=green] (-1,1)--(1,-4);
 \draw[thin] (0,1) circle(1.5);
 \draw [-{Latex[]},thick] (0,1) -- (1.3,0.3);
 \draw (0.76,0.76) node{$r$};
}
\begin{verbatim}
\bbwGraph{-3}{2}{-4}{3}{
 \bbwLine{-2,-1}{-2.5,2}{blue}
 \bbwLine{-1,0}{2,3}{red}
 \draw[thin,color=green] (-1,1)--(1,-4);
 \draw[thin] (0,1) circle(1.5);
 \draw [-{Latex[]},thick] (0,1) -- (1.3,0.3);
 \draw (0.76,0.76) node{$r$};
}
\end{verbatim}
\newpage

\subsection{Beliebige Funktionen}
%%       xmin,xmax,ymin,ymax,fct,domain
\bbwFunction{-3}{2}{-2}{5}{-\x*\x - \x + 4.5}{-2.5:1.5}
\begin{verbatim}
\bbwFunction{-3}{2}{-2}{5}{-\x*\x - \x + 4.5}{-2.5:1.5}
\end{verbatim}

Bei Funktionen sind die ersten vier Koordinaten die Grenzen des Koordinatensystems. \texttt{\{(-2.5:1.5)\}} hingegen bezeichnet den Definitonsbereich der Funktion.
Bemerkung: Eine \texttt{\\bbwFunction} benötigt
keine \texttt{bbwGraph}-Umgebung.
\newpage

Gleich noch ein Beispiel:

\bbwFunction{-1}{11}{-6}{4}{ln(-1+\x)}{1.005:11}
\begin{verbatim}
\bbwFunction{-1}{11}{-6}{4}{ln(-1+\x)}{1.005:11}
\end{verbatim}
\newpage


\subsection{Mehrere Funktionen im selben Graph}
Der Befehl \texttt{bbwFunc} kennzeichnet eine einzelne Funktion
innerhalb einer \texttt{bbwGraph}-Umgebung:

\bbwGraph{-4}{4}{-2}{2}{
  \bbwFunc{\x*0.2}{-3:3}
  \bbwFuncC{\x*0.1 + 0.5}{-3:3}{red}
}
\begin{verbatim}
\bbwGraph{-4}{4}{-2}{2}{
  \bbwFunc{\x*0.2}{-3:3}
  \bbwFuncC{\x*0.1 + 0.5}{-3:3}{red}
}
\end{verbatim}

\end{document}
