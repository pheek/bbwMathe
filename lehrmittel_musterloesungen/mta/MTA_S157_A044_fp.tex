%%
%% Musterlösung
%%

\input{bmsLayoutPage}

%%%%%%%%%%%%%%%%%%%%%%%%%%%%%%%%%%%%%%%%%%%%%%%%%%%%%%%%%%%%%%%%%%

%\usepackage{amssymb} 
\usepackage{cancel}
\renewcommand{\metaHeaderLine}{Lösungsweg (Vorschlag)}
\renewcommand{\arbeitsblattTitel}{Marthaler Algebra S. 157 Aufg. 44.}

\begin{document}%%
\arbeitsblattHeader{}

\begin{enumerate}
\item \textbf{Aufgabe verstehen}:

Zwei Kapitalien werden verzinst. Jedes Kapital hat
einen Anfangswert in CHF, einen Zinssatz, einen Zins und ein
Endkapital, sobald der Zins zum Anfangskapital dazu geschlagen wird.

Es gilt:  Zins = Kapital * Zinssatz / 100.

\item \textbf{Variable eindeutig einführen}:

$K_1$ sei das erste Kapital vor der Verzinsung (am Anfang) in CHF.

$K_2$ sei das zweite Kapital vor der Verzinsung (am Anfang) in CHF.

\item \textbf{Terme bilden}:

Jahreszins des 1. Kapitals = $K_1 * 0.04$

Jahreszins des 2. Kapitals = $K_2 * 0.05$

Endkapital des 1. Kapitals = $K_1 + K_1 * 0.04 = K_1 * 1.04$

Endkapital des 2. Kapitals = $K_2 + K_2 * 0.05 = K_2 * 1.05$

\item \textbf{Gleichungen aufstellen}:

Die beiden Endkapitalien (mit dazu geschlagenem Zins) sind gleich groß ...

$$K_1\cdot{} 1.04 = K_2 \cdot{} 1.05$$

... und die Summe der Zinse ist 1\,410.- CHF:
$$K_1\cdot{} 0.04 + K_2\cdot{} 0.05 = 1\,410$$

Das ergibt folgendes Gleichungssystem:

\gleichungZZ{K_1\cdot{} 1.04 - K_2\cdot{}1.05}{0}{K_1\cdot{} 0.04 +
K_2\cdot{} 0.05}{1\,410}

\item \textbf{Gleichung(en) Lösen}:
Das System kann der Taschenrechner mit \tiprobutton{2nd}\tiprobutton{tan_sys-solv} lösen:

$$K_1 = 15\,750.-$$
$$K_2 = 15\,600.-$$

\item \textbf{Probe}:
Nach einem Jahr: $K_1=15\,750$ wird auf $16380.-$ CHF ansteigen.

$K_1=15\,600$ wird auch auf $16380.-$ CHF ansteigen.

\item \textbf{Antwort}:

Das erste der Kapitalien war vor der Verzinsung \textbf{CHF $15\,750.-$}
wert, während das zweite Kapital den Wert \textbf{CHF $15\,600.-$} hatte.
\end{enumerate}

\end{document}
