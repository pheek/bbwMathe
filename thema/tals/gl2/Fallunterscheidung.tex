\subsection{Fallunterscheidung}

Aus den Strukturaufgaben:


\aufgabenFarbe{Berechnen Sie die Lösung für $x\in\mathbb{R}$ in Abhängigkeit von $k$ mit einer vollständigen Fallunterscheidung für alle Parameterwerte $k\in\mathbb{R}$.
}

$$kx^2 - x^2 + 3x = 2$$

\TNTeop{

  \begin{tabular}{c|c|c}
    A & B & C  \\\hline
    $k-1$ & $3$ & $-2$ 
    \end{tabular}

    $$D = 9-4\cdot{} (k-1)\cdot{}(-2)$$
    $$D = 9 + 8(k-1) = 8k + 1$$
Allgemeine Lösung:
    $$x_{1,2} = \frac{-3 \pm \sqrt{8k+1}}{2k-2}$$

 1. Soderfall Nenner: $k=1$ ist nicht möglich.

    Mit $k-1$ reduziert sich die ursprüngliche Gleichung jedoch auf:
    $$ 3x = 2 \Longrightarrow x = \frac23$$

2. Sonderfall: Diskriminante = 0:
Das heißt $8k+1=0$ und somit ist $k=\frac{-1}8$.

Somit erhalten wir für $x$ eingesetzt in die allgemeine Lösung nur noch eine Lösung:

$$x = \frac{-3\pm\sqrt{0}}{2\cdot{}\left(\frac{-1}{8} - 1\right)} = \frac43$$

3. Sonderfall: Diskriminante ist kleiner als 0:

$$8k+1 < 0 \Longleftrightarrow k <\frac{-1}8  \Longrightarrow  \lx=\{\}$$

Abgesehen von den Sonderfällen $k=1$ und $k<\frac{-1}8$ gilt die allgemeine Lösung.
    

    
    
}%% end TNT eop
%% \newpage implicit  
