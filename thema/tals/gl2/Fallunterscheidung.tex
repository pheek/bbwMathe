\subsection{Fallunterscheidung}
Einstiegsbeispiel: Ein Ball wird mit 30 m/s senkrecht nach oben geworfen.
Die Höhe kann wie folgt berechnet werden:

$$h(t)  = v_0 \cdot{} t -\frac{g}{2}\cdot{}t^2$$
Dabei ist $g$ die «Fallbeschleunigung»: $g\approx
9.8\,\frac{\text{m}}{\text{s}} / \text{ s}\, \approx 10$.

Mit Zahlen:

$$h(t) = 30\cdot{} t - \frac{10}2 \cdot{} t^2$$

Mit unseren bekannten Variablen:
$$h = 30x - 5x^2$$

Skizzieren Sie mit 
$y=h(t)$ in 10 m Schritten, $x=t$ Zeit in s:

\bbwGraph{-1}{10}{-1}{5}{
  \TRAINER{
      \bbwFunc{-0.5*(\x-3)*(\x-3)+4.5}{0:6}   
%%    \bbwFunc{-18*(\x-0.5)*(\x-0.5)+4.5){0:1}
    }
}%% end bbwGraph

\newpage

Wann ist der Ball 25 m hoch? Wann ist der Ball $h$ Meter hoch?

\TNTeop{
$$25 = 30x - 5x^2$$
  $$5x^2 - 30x + 25 = 0$$
  $$x^2 -6x + 5 = 0 \Longleftrightarrow (x-5)(x-1) = 0$$
  $$\lx=\{1 \text{ s } , 5 \text{ s }\}$$

  
  Je nach Höhe ergibt sich eine andere Anzahl der Lösungen für die Zeit:


  $$h = 30x - 5x^2 \Longleftrightarrow -5x^2 + 30x - h = 0$$

  Ist die Diskriminante = 0, so haben wir genau eine Lösung für die Zeit

  $$D = 30^2 - 4\cdot{} (-5) \cdot{} (-h) = 900 - 20h$$



  a) Berechnen $h$ für $D=0$  :

  $900 - 20h = 0 \Longrightarrow h = 45$

  b) was bedeutet dies? Bei $h=45$ Metern hat die Gleichung genau eine Lösung: Da ist der Ball ganz oben.
  $$h = 45 = 30x - 5x^2\,\,\,\, | : 5$$ 
  $$9 = 6x-x^2$$
  sortieren
  $$x^2 - 6x + 9 = 0$$
  $$(x-3)^2 = 0$$
  Nur eine Lösung, nämlich $x=t=3$ s.
  
  }%% end TNT



\newpage

Aus den Strukturaufgaben:


\aufgabenFarbe{Berechnen Sie die Lösung für $x\in\mathbb{R}$ in Abhängigkeit von $k$ mit einer vollständigen Fallunterscheidung für alle Parameterwerte $k\in\mathbb{R}$.
}

$$kx^2 - x^2 + 3x = 2$$

\TNTeop{

  \begin{tabular}{c|c|c}
    A & B & C  \\\hline
    $k-1$ & $3$ & $-2$ 
    \end{tabular}

    $$D = 9-4\cdot{} (k-1)\cdot{}(-2)$$
    $$D = 9 + 8(k-1) = 8k + 1$$
Allgemeine Lösung:
    $$x_{1,2} = \frac{-3 \pm \sqrt{8k+1}}{2k-2}$$

 1. Sonderfall Nenner: $k=1$ ist nicht möglich.

    Mit $k-1$ reduziert sich die ursprüngliche Gleichung jedoch auf:
    $$ 3x = 2 \Longrightarrow x = \frac23$$

2. Sonderfall: Diskriminante = 0:
Das heißt $8k+1=0$ und somit ist $k=\frac{-1}8$.

Somit erhalten wir für $x$ eingesetzt in die allgemeine Lösung nur noch eine Lösung:

$$x = \frac{-3\pm\sqrt{0}}{2\cdot{}\left(\frac{-1}{8} - 1\right)} = \frac43$$

3. Sonderfall: Diskriminante ist kleiner als 0:

$$8k+1 < 0 \Longleftrightarrow k <\frac{-1}8  \Longrightarrow  \lx=\{\}$$

Abgesehen von den Sonderfällen $k=1$ und $k<\frac{-1}8$ gilt die allgemeine Lösung.
    

    
    
}%% end TNT eop
%% \newpage implicit  


\subsection*{Aufgaben}

\olatLinkArbeitsblatt{GlQuad}{https://olat.bms-w.ch/auth/RepositoryEntry/6029786/CourseNode/111365560138736}{Aufgabe 13.}
