
\subsubsection{Quadratische Ergänzung (optional)}\index{quadratische Ergänzung}\index{Ergänzung!quadratische}
Grundform\TRAINER{ (Normalform ist, wenn $a=1$)}:
$$3x^2 - 11x - 4 = 0$$

\TNTeop{
$+4$, dann $:3 \Rightarrow$
%%
  $$x^2 - {\color{red}\frac{11}{3}}x  = \frac{4}{3}$$
  quadratisch ergänzen:
  $$x^2 - {\color{red}\frac{11}{3}}x + ({\color{ForestGreen}\frac{1}{2}}\cdot{\color{red}\frac{11}{3}})^2 = \frac{4}{3} + ({\color{ForestGreen}\frac{1}{2}}\cdot{\color{red}\frac{11}{3}})^2$$
%%
faktorisieren mit binomischer Formel:
  $$(x - \frac{1}{2}\cdot\frac{11}{3})^2 = \frac{4}{3}+ \frac{121}{36}$$
links Brüche multiplizieren, rechts ersten Bruch mit 12 erweitern:
  $$(x - \frac{11}{6})^2 = \frac{48}{36}+ \frac{121}{36}$$
%%
  $$(x - \frac{11}{6})^2 = \frac{169}{36}$$
%%
Wurzel ziehen (Achtung, es kann zwei Lösungen geben)
%%
   $$x - \frac{11}{6} = \pm\sqrt{ \frac{169}{36}}$$
%%
   $$x - \frac{11}{6} = \pm \frac{13}{6}$$
%%
   $$x  = \frac{11}{6} \pm \frac{13}{6}$$
%%
   Erste Lösung $x=\frac{24}{6} = +4$ und zweite Lösung $x =
   -\frac{2}{6} = -\frac{1}{3}$.
%%
   Probe: 4 und $-\frac{1}{3}$ einsetzen.
}%% END TNT eop

%%%%%%%%%%%%%%%%%%%%%%%%%%%%%%%%%%%

