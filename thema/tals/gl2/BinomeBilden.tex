\subsection{Binome}

$$x^2+6x+9 = 19$$

\TNTeop{Binomische Formel (!):
(Hinweis: Beidseitig -9 ist nicht falsch, führt hier jedoch in eine Sackgasse)

Die Sackgasse zeigen, falls nicht schon im Einstiegsbeispiel $(x-2)^2=49$ gezeigt.

$$x^2+6x+9 = 19$$

$$(x+3)^2 = 19$$

  Ziehe die Wurzel und ... 

  $$\sqrt{(x+3)^2} = \sqrt{19}$$

... beachte die 2. Lösung.
  $$|x+3| = \sqrt{19}$$

$$x+3 = \pm\sqrt{19}$$

  $$x = -3 \pm \sqrt{19}$$

  $$\lx = \{-3-\sqrt{19}; -3 + \sqrt{19}\} \approx \{-7.359; 1.359\}$$
}
\newpage

Selbständig:

$$x^2 -10x + 25 = 17$$
\TNT{7.2}{
$$(x-5)^2 = 17$$

$$\Longrightarrow$$

$$x-5 = \pm \sqrt{17}$$

$$\Longrightarrow$$

$$x=5\pm\sqrt{17}$$

$$\Longrightarrow  \lx = \{5-\sqrt{17}; 5+\sqrt{17}\}$$


}


$$x^2 -4x + 4 = -16$$
\TNTeop{
$$(x-2)^2 = -16$$
$$x-2 = \pm\sqrt{-16}$$

$$\Longrightarrow  \lx = \{\}$$


}

%%%%%%%%%%%%%%%%%%%%%%%%%%%%%%%%%%%%%%%%%%%%%%%%%%%
\newpage
