\subsubsection{Binome}

$$x^2+6x+9 = 19$$

Algorithmus:


1.

\TNT{2}{Binomische Formel:
$$(x+3)^2 = 19$$
}

2.
\TNT{2}{Wurzel ziehen / 2. Lösung nicht vergessen
  $$x+3 = \pm \sqrt{19}$$
  rechts 0  $\Longrightarrow$ Eine Lösung\\
  rechts <0 $\Longrightarrow$ keine Lösungen
}

3.
\TNT{2}{ausgleichen:
$$x = -3 \pm \sqrt{19}$$
}

4. 
\TNT{2}{Lösungsmenge angeben:
$$\lx = \{-3-\sqrt{19}; -3+\sqrt{19}\} \approx \{-7.3589; 1.3589\} $$
}

Bemerkung
\TNTeop{Sackgasse:

Hinweis: Beidseitig -9 ist nicht falsch, führt hier jedoch in eine Sackgasse:

$$x^2 +6x = 10\, ???$$



}
\newpage

\TRAINER{ganz selbständig}

$$x^2 -10x + 25 = 17$$
\TNT{7.2}{
$$(x-5)^2 = 17$$

$$\Longrightarrow$$

$$x-5 = \pm \sqrt{17}$$

$$\Longrightarrow$$

$$x=5\pm\sqrt{17}$$

$$\Longrightarrow  \lx = \{5-\sqrt{17}; 5+\sqrt{17}\}$$


}


$$x^2 -4x + 4 = -16$$
\TNTeop{
$$(x-2)^2 = -16$$
$$x-2 = \pm\sqrt{-16}$$

$$\Longrightarrow  \lx = \{\}$$


}

%%%%%%%%%%%%%%%%%%%%%%%%%%%%%%%%%%%%%%%%%%%%%%%%%%%
\newpage
