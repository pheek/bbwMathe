\subsubsection{Binome}

$$x^2+6x+7 = 18$$

\TNTeop{Binomische Formel (!):
(Hinweis: Beidseitig -9 ist nicht falsch, führt hier jedoch in eine Sackgasse)

Die Sackgasse zeigen, falls nicht schon im Einstiegsbeispiel $(x-2)^2=49$ gezeigt.


Binom vervollständigen\TRAINER{ (al-gabr = «ganz machen», «wieder
herstellen», «vervollständigen»}
$$x^2+6x+9 = 20$$

$$(x+3)^2 = 20$$

  Ziehe die Wurzel und beachte die 2. Lösung:

  $$\sqrt{(x+3)^2} = \sqrt{20} ( \approx \pm 4.47)$$

  $$|x+3| = \sqrt{20}$$

$$x+3 = \pm\sqrt{20}$$

  $$x = -3 \pm \sqrt{20}$$

  $$\lx = \{-3-\sqrt{20}; -3 + \sqrt{20}\} = \{-3-2\sqrt{5}; -3+2\sqrt{5}\} \approx \{-7.47; 1.47\}$$
}
\newpage

Selbständig:

$$x^2 -10x + 25 = 17$$
\TNT{7.2}{
$$(x-5)^2 = 17$$

$$\Longrightarrow$$

$$x-5 = \pm \sqrt{17}$$

$$\Longrightarrow$$

$$x=5\pm\sqrt{17}$$

$$\Longrightarrow  \lx = \{5-\sqrt{17}; 5+\sqrt{17}\}$$


}


$$x^2 -4x + 4.03 = 0.03$$
\TNTeop{
Ausgleichen ($-0.03$):
$$x^2 -4x + 4 = 0$$
$$(x-2)^2 = 0$$

$$\Longrightarrow$$

$$x-2 = 0$$

$$\Longrightarrow$$

$$x= 2$$

$$\Longrightarrow  \lx = \{2\}$$


}

%%%%%%%%%%%%%%%%%%%%%%%%%%%%%%%%%%%%%%%%%%%%%%%%%%%
\newpage
