\subsubsection{Binome}

$$x^2+6x+9 = 20$$

\TNTeop{Binomische Formel (!):
(Hinweis: Beidseitig -9 ist nicht falsch, führt hier jedoch in eine Sackgasse)

  $$(x+3)^2 = 20$$

  Ziehe die Wurzel und beachte die 2. Lösung:

  $$x+3 = \pm\sqrt{20}$$

  -3:

  $$x = -3 \pm \sqrt{20}$$

  $$\lx = \{-3-\sqrt{20}; -3 + \sqrt{20}\}$$
}
\newpage

Selbständig:

$$x^2 -10x + 25 = 17$$
\TNT{7.2}{
$$(x-5)^2 = 17$$

$$\Longrightarrow$$

$$x-5 = \pm \sqrt{17}$$

$$\Longrightarrow$$

$$x=5\pm\sqrt{17}$$

$$\Longrightarrow  \lx = \{5-\sqrt{17}; 5+\sqrt{17}\}$$


}


$$x^2 -4x + 4 = 0$$
\TNTeop{
$$(x-2)^2 = 0$$

$$\Longrightarrow$$

$$x-2 = 0$$

$$\Longrightarrow$$

$$x= 2$$

$$\Longrightarrow  \lx = \{2\}$$


}

%%%%%%%%%%%%%%%%%%%%%%%%%%%%%%%%%%%%%%%%%%%%%%%%%%%

\subsection*{Aufgaben}

\olatLinkArbeitsblatt{Quadratische
    Gleichungen}{https://olat.bbw.ch/auth/RepositoryEntry/572162090/CourseNode/107315676197926}{Weiter, wo Sie aufgehört hatten}

\TRAINER{Lernzielkontrolle: Kahoot: Quadratische Gleichungen.}%%
\newpage
