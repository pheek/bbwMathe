%%
%% 2019 07 04 Ph. G. Freimann
%% Ergänzung für TALS zu Quadratischen Gleichungen I (Ohne Parameter)
%%

\subsection{Quadratische Gleichungen II (mit Parametern)}\index{Gleichungen!quadratische mit Parametern}
\sectuntertitel{Welches ist denn nun die Gesuchte?}

\TadBMTA{172}{10.4}
%%%%%%%%%%%%%%%%%%%%%%%%%%%%%%%%%%%%%%%%%%%%%%%%%%%%%%%%%%%%%%%%%%%%%%%%%%%%%%%%%
\subsection*{Lernziele}

\begin{itemize}
\item Taschenrechner: Mit Parameter(n)
\item Taschenrechner: Visualisierung, Interpretation
\end{itemize}
\newpage

\subsubsection{Vorzeigeaufgabe}
Lösen Sie die folgende Gleichung ohne Berücksichtigung von
Spezialfällen, die durch die Wahl der Parameter ($\lambda$ bzw. $b$) auftreten
könnten:

$$bx^2 + \sqrt{2} x = \lambda b x + \lambda \sqrt{2}$$

\TNTeop{
Alles nach links nehmen:
$$bx^2 + \sqrt{2}x - \lambda b x - \lambda \sqrt{2} = 0$$
$A$, $B$ und $C$ bestimmen, dazu zuerst $x$ ausklammern:
\\
$$b\cdot{}x^2 + (\sqrt{2} - \lambda b) \cdot x - \lambda \sqrt{2} = 0$$
$$\Longrightarrow  A = b; B = (\sqrt{2} - \lambda{} b); C = -\lambda \sqrt{2}$$
Einsetzen in $\frac{-B \pm \sqrt{B^2-4AC}}{2A} $:
$$x_{1,2} = \frac{\lambda b - \sqrt{2} \pm \sqrt{(2-2\sqrt{2}\lambda b) + \lambda^2 b^2 + 4b\lambda\sqrt{2}}}{2b}$$
$$  = \frac{\lambda b - \sqrt{2} \pm \sqrt{2+2\sqrt{2}\lambda b+ \lambda^2 b^2 }}{2b} $$
binomische Formel: 
  $$  = \frac{\lambda b - \sqrt{2} \pm \sqrt{(\sqrt{2} + \lambda b)^2 }}{2b} $$
  $$  = \frac{\lambda b - \sqrt{2} \pm (\sqrt{2} + \lambda b)}{2b}$$
  $$x_1 = \frac{-\sqrt{2}}{b}; x_2 = \frac{2\lambda b}{2b} = \lambda $$
}%% END TNT
%%\newpage

\subsection{Aufgaben}
%\AadBMTA{182ff}{13. a) b)}%%\TALSAadBMTA{99}{ab 279}
\TALS{\olatLinkArbeitsblatt{GlQuad}{https://olat.bms-w.ch/auth/RepositoryEntry/6029786/CourseNode/111365560138736}{Aufgabe 9.}
  }%% end GESO

\newpage

\subsection*{Typ «genau eine Lösung»}

Bestimmen Sie den Parameter $b$ so, dass die folgende Gleichung genau
eine Lösung hat:
$$x^2+b+2 = -2bx$$
\TNTeop{
  Vorzeigeaufgabe: S. 183.: 16. b):%%
  Grundform:
  $$x^2+2bx+b+2=0$$
  Diskriminante = 0 setzen:
  $$4b^2 - 4(b+2) = 0$$
  Grundform $b$:
  $$b^2-b-2 = 0$$
  $$\LoesungsMenge{}_b = \{-1; 2\}$$
}%% END TNT


\subsection*{Aufgaben}
\TALS{\olatLinkArbeitsblatt{GlQuad}{https://olat.bms-w.ch/auth/RepositoryEntry/6029786/CourseNode/111365560138736}{Aufgabe 10.}
  }%% end GESO
\newpage
%%%%%%%%%%%%%%%%%%%%%%%5
\subsubsection{Taschenrechner}
Das selbe Beispiel können wir ganz einfach auch mit dem Taschenrechner lösen:
\begin{beispiel}{Taschenrechner}{}
  Bestimmen Sie den Parameter $b$ so, dass die quadratische
  Gleichung $$x^2+b+2=-2bx$$
  genau eine Lösung hat:

  \TNT{2}{\texttt{solve($x^2+b+2=-2bx$, x)} liefert

  $$x_{1,2} = \pm\sqrt{b^2-b-2} - b$$}
  
  nun die Diskriminante suchen

  \TNT{2}{\texttt{D = $b^2-b-2$}}

  jetzt Diskriminante = 0 setzen:

  \TNT{2}{\texttt{solve($b^2-b-2$ = 0,b)}

    liefert

  $$b_{1,2} = -1 \text{ oder } 2$$}%% end tnt
\end{beispiel}%%  


\subsection*{Aufgaben}

\TALS{\olatLinkArbeitsblatt{GlQuad}{https://olat.bms-w.ch/auth/RepositoryEntry/6029786/CourseNode/111365560138736}{Aufgabe 10.}
  }%% end GESO

%\AadBMTA{182ff}{16. a) c) d)}
\newpage
