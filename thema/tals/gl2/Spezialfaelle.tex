%%
%% 2020 03 27 Ph. G. Freimann
%%

\subsubsection{Spezialfall $c=0$}

Beispiel
$$5x^2 + 3x = 0$$

\TNT{8}{
Faktorisieren
$$x(5x+3) = 0$$
$$\lx= \{-\frac35; 0\}$$
}%% END TNT


Wenn eine quadratische Gleichung der Form
$$ax^2 +bx = 0$$
gegeben ist, so kann man einfach ein $x$ ausklammern:

$$x(ax+b)=0$$
 Die Gleichung ist erfüllt, wenn nun entweder $x$ selbst oder aber
 der Klammerausdruck $(ax+b)$ Null werden. Somit haben wir sofort zwei
 Lösungen gefunden:
 $$\lx=\left\{0, \frac{-b}{a}\right\}$$

 %%\TALSAadBMTA{95ff}{265. a) b) c) e) g)}
 \newpage

 
 \subsubsection{Spezialfall $b=0$ (reinquadratische Gleichungen)}
\TadBMTA{166}{10.2.1}

Beispiel
$$5x^2 - 3 = 0$$

\TNT{8}{
Quadrat separieren:
$$5x^2 = 3$$
$$x^2 = \frac35$$
Wurzel ziehen:
$$\lx= \left\{-\sqrt{\frac35}; +\sqrt{\frac35}\right\}$$
}%% END TNT




Ist die quadratische Gleichung in der Form
 $$ax^2 + c = 0$$
 gegeben, so gibt es ebenfalls eine einfache Lösungsformel. Es folgt:
 $$ax^2 = -c$$
 und daraus:
 $$x^2 = \frac{-c}{a}$$

 Die Lösungsmenge ist also schlicht
 $$\lx=\left\{ + \sqrt{\frac{-c}{a}}, -\sqrt{\frac{-c}{a}} \right\}.$$

\newpage

 \subsubsection{Faktorisierte Form}

$$(x-7)(2x+8) = 0$$

\TNTeop{$\lx=\{-4; 7\}$, danach unbedingt Probe vorzeigen.\vspace{32mm}%%
}%% END TNTeop
%%\TALSAadBMTA{95ff}{266. b) c) d)}
\newpage
