%% Trigonometrie III
%% Arkusfunktionen
%% 2020 - 12 - 21 φ@bms-w.ch

\section{Arkusfunktionen}\index{Arkusfunktionen}

Einstiegsüberlegung: Von einem Winkel $\varphi$ ist gerade einmal sein Sinuswert bekannt: $0.6 = \sin(\varphi)$. Wie groß ist der Winkel?

\trigsysDsin{}


\TNT{6.4}{Hier gibt es unendlich viele Lösungen. Im Bereich 0 bis 180$\degre$ sind es jedoch genau zwei Lösungen: 1. Lösung: $\varphi=\arcsin(0.6) \approx 36.87$ und 2. Lösung: $\varphi = 180\degre - \arcsin(0.6) \approx{} 143.13$}

Mit dem Taschenrechner haben Sie zwei Varianten diesen Winkel zu berechnen:

A) \texttt{arcsin(0.6)}

B) \texttt{solve(0.6=sin(x), x)}

Was fällt Ihnen bei den Lösungen auf?
\newpage



\subsection{Arcussinus}
Auch die Umkehrungen der Sinus-, Cosinus- und Tangensbeziehungen
können stückweise als Funktionen aufgefasst werden.

Füllen Sie die folgende Tabelle aus und übertragen Sie die Werte ins nachstehende
Koordinatensystem (Winkel im Bogenmaß):


%\begin{bbwFillInTabular}{l|l|l|l||l|l}
%  \multicolumn{4}{c||}{Winkel} & \multicolumn{2}{c}{$\sin()$-Wert} \\\hline
%  Grad & \multicolumn{3}{c||}{Bogenmaß} & \multicolumn{2}{c}{}\\\hline
%  & exakt &  & Dezimalzahl & Dezimalzahl & exakt \\\hline\hline
%  $-30^\circ$ & \TRAINER{$-\frac{\pi}6$} & \TRAINER{$\arcsin(-\frac12)$} & \TRAINER{$-0.52...$} & \TRAINER{$-0.5$} & \TRAINER{$-\frac12$}\\\hline
%  $0^\circ$ & $ \TRAINER{0} $ & \TRAINER{$\arcsin(0)$} & \TRAINER{$0$} & \TRAINER{$0$} & \TRAINER{$0$}\\\hline
%  $30^\circ$ & $ \frac{\pi}6 $ & $\arcsin\left(\frac12\right)$ \noTRAINER{\,\,\,\,\,\,\,\,\,\,}& $0.52...$ & $0.5$ & $\frac12$\\\hline
%  $45^\circ$ & $ \TRAINER{\frac{\pi}4} $ & \TRAINER{$\arcsin\left(\frac{\sqrt{2}}2\right)$} & \TRAINER{$0.785...$} & \TRAINER{$0.7071...$} & \TRAINER{$\frac{\sqrt{2}}2$}\\\hline
%  $60^\circ$ & $ \TRAINER{\frac{\pi}3} $ & \TRAINER{$\arcsin \left( \frac{\sqrt{3}}2 \right) $} & \TRAINER{$1.047...$} & \TRAINER{$0.8660...$} & \TRAINER{$\frac{\sqrt{3}}2$}\\\hline
%  $90^\circ$ & $ \TRAINER{\frac{\pi}2} $ & \TRAINER{$\arcsin(1)$} &   \TRAINER{$1.570796...$} & \TRAINER{$1.000...$} & \TRAINER{$1$}\\\hline
%  $120^\circ$ & $ \TRAINER{\frac{2\pi}3} $ & \TRAINER{$\pi-\arcsin(\frac{\sqrt{3}}{2})$} &   \TRAINER{$2.094...$} & \TRAINER{$0.8660...$} & \TRAINER{$\frac{\sqrt{3}}2$}\\\hline
%\end{bbwFillInTabular}


\begin{bbwFillInTabular}{r||r|r||rcr}
Winkel      & \multicolumn{2}{c||}{$\sin$-Wert} & \multicolumn{3}{c}{Winkel} \\\hline
$\varphi$   & $\sin(\varphi)$                    & dezimal & $\arcsin$        & $=$  & Winkel         \\\hline
$-30\degre$ & $\frac{-1}2$                       & $-0.5$  & $\arcsin(-0.5)$  & $=$  & $-30\degre$    \\\hline 
$  0\degre$ & $0$                                & $0$     & $\arcsin(0)$     & $=$  & $0\degre$		   \\\hline 
$ 30\degre$ & $\frac12$                          & $0.5$   & $\arcsin(0.5)$   & $=$  & $30\degre$	   \\\hline 
$ 60\degre$ & \TRAINER{$\frac{\sqrt{3}}2$}       & \TRAINER{$0.8660$}   & \TRAINER{$\arcsin(0.8660)$}   & $=$  & \TRAINER{$60\degre$}	   \\\hline 
$ 90\degre$ & \TRAINER{$1$}       & \TRAINER{$1$}   & \TRAINER{$\arcsin(1)$}   & $=$  & \TRAINER{$90\degre$}	   \\\hline 
$120\degre$ & \TRAINER{$\frac{\sqrt{3}}2$}       & \TRAINER{$0.8660$}   & \TRAINER{$\arcsin(0.8660)$}   & $=$  & \TRAINER{$60\degre$}	   \\\hline 
$135\degre$ & \TRAINER{$\frac{\sqrt{2}}2$}       & \TRAINER{$0.7071$}   & \TRAINER{$\arcsin(0.7071)$}   & $=$  & \TRAINER{$45\degre$}	   \\\hline 
$150\degre$ & \TRAINER{$\frac12$}       & \TRAINER{$0.5$}   & \TRAINER{$\arcsin(0.5)$}   & $=$  & \TRAINER{$30\degre$}	   \\\hline 
$180\degre$ & \TRAINER{$$}       & \TRAINER{$0$}   & \TRAINER{$\arcsin(0)$}   & $=$  & \TRAINER{$0\degre$}	   \\\hline 
\end{bbwFillInTabular}







\TNTeop{Achtung: Für den $\sin$-Wert 0.8660... existieren mehrere
  Winkel! Die Umkehrung der Sinusfunktion ist nur \textbf{stückweise}
  definiert!}

\newpage

\subsubsection{Graph der Arcussinus-Funktion}
\noTRAINER{\bbwCenterGraphic{12cm}{tals/trig3/img/ArcsinLeer.png}}
\TRAINER{\bbwCenterGraphic{12cm}{tals/trig3/img/Arcsin.png}}

Geben Sie den Definitionsbereich der Arcus-Sinus-Funktion an:

$$\DefinitionsMenge{} = \LoesungsRaum{\left[-1; 1\right]}$$

Geben Sie den Wertebereich der Arcus-Sinus-Funktion einmal im Bogenmaß,
einmal im Gradmaß an:
$$\Wertebereich{} =  \LoesungsRaum{[-90\degre; 90\degre] = \LoesungsRaum{\left[-\frac{\pi}{2}; \frac{\pi}{2}\right]}  }$$
\newpage


\subsection*{Referenzaufgabe}

Winkel-Aufgabe diesmal anhand der Funktionsgraphen und nicht (nur) am Einheitskreis.

Vorzeigeaufgabe: Für welche(n) Winkel $\varphi$ gilt \fbox{$\sin(\varphi) = -\sin(-41.7\degre)$}?

\TRAINER{Lösung: $\varphi = 41.7\degre$ und $180\degre-41.7\degre=138.3\degre$ und alle periodischen Vielfachen: $\varphi\pm n\cdot{}360\degre$}

\trigsysDsin{}

\subsection*{Aufgaben}

%%\TALSAadBFWG{102ff}{85. a) c)}
\aufgabenFarbe{
Mit Taschenrechner:\\
a) $\sin(\alpha) = 0.574$ \TRAINER{ $\alpha\approx 35.03\degre$ und $\alpha\approx 144.97\degre$} \\
b) $-0.985 = \sin(\beta)$ \TRAINER{ $\beta\approx 279.94\degre $ und  $\beta\approx 260.06\degre$ }\\
Ohne Taschenrechner\TRAINER{ (Niveau Grundlagenfach)}:\\
c) $\sin(76\degre) = -\sin(\gamma)$ \TRAINER{ $\gamma =284\degre$ und $\gamma = 256\degre$}\\
d) $\sin(265\degre) = -\sin(\delta)$ \TRAINER{ $\delta=85\degre$ und $\delta=95\degre$}\\
e) $\sin(-53\degre) = \sin(\varepsilon)$ \TRAINER{ $\varepsilon =   233\degre$ und $\varepsilon = 307\degre$} \\
f) $-\sin(\omega) = \sin(-113\degre)$ \TRAINER{ $\omega = 67\degre$ und $\omega = 113\degre$} \\
}
%%\TALSAadBFWG{102ff}{85. e) f) g) und h)}


\TNTeop{}
%%\newpage

\subsection{Arcuscosinus}
Zeichnen Sie die Umkehrfunktion des Cosinus und geben Sie den Definitionsbereich an.

\noTRAINER{\bbwCenterGraphic{12cm}{tals/trig3/img/ArccosLeer.png}}
\TRAINER{\bbwCenterGraphic{12cm}{tals/trig3/img/Arccos.png}}


$$\DefinitionsMenge{} = \LoesungsRaum{[-1; 1]}$$

Geben Sie den Wertebereich der Arcus-Sinus-Funktion einmal im Bogenmaß,
einmal im Gradmaß an:
$$\Wertebereich{} = \LoesungsRaum{[0; \pi]}  =  \LoesungsRaum{[0; 180\degre]}$$
\newpage
\subsection*{Aufgaben}

\trigsysDcos{}

\aufgabenFarbe{Mit Taschenrechner:\\
%%\TALSAadBFWG{102ff}{86. b)}
a) $-0.546 = \cos(\alpha)$ \TRAINER{ $\alpha \approx 123.09\degre$  und $\alpha\approx 236.91$}\\
Ohne Taschenrechner:\\
%%\TALSAadBFWG{102ff}{86. c) und e)}
b) $\cos(\beta) = -\cos(37\degre)$ \TRAINER{ $\beta = 143\degre$ und $\beta = 217\degre $}\\
c) $\cos(-\gamma) = \cos(96\degre)$ \TRAINER{ $\gamma = 96\degre$ und $\gamma = 264\degre$}\\
d) $-\cos(-24\degre) = \cos(\delta)$ \TRAINER{ $\delta=156\degre$ und  $\delta=204\degre$}\\
e) $ \sin(72\degre) = -\cos(\varphi)$ \TRAINER{ $\varphi=162\degre$ und  $\varphi=198\degre$}\\
f) $-\sin(\epsilon) = \cos(52)$ \TRAINER{ $\epsilon=281\degre$ und  $\epsilon=322\degre$}\\
g) $sin(\alpha)\cdot{}\cos(\alpha) = 0$ \TRAINER{ Alle Nullstellen von  $\sin$ bzw. $\cos$: $0\degre$, $90\degre$, $180\degre$, $270\degre$}\\
}%% end aufgabenFrabe

\aufgabenFarbe{Vereinfachen Sie:\\
  \begin{itemize}
  \item
    $\cos(180\degre + x) = \TRAINER{-\cos(x)}$
  \item
    $\cos(\frac\pi2 + x) = \TRAINER{-\sin(x)}$
  \item
    $-\sin(270\degre + x) = \TRAINER{\cos(x)}$
  \end{itemize} 
}%% end aufgabenFrabe
\TNTeop{}
%%\newpage


\subsection{Arcustangens}
Zeichnen Sie die Umkehrfunktion des Tangens (im Bogenmaß) und geben Sie den Definitionsbereich an.

\noTRAINER{\bbwCenterGraphic{16cm}{tals/trig3/img/ArctanLeer.png}}
\TRAINER{\bbwCenterGraphic{16cm}{tals/trig3/img/Arctan.png}}


$$\DefinitionsMenge{} = \LoesungsRaum{\mathbb{R}}$$

Geben Sie den Wertebereich der Arcus-Sinus-Funktion einmal im Bogenmaß,
einmal im Gradmaß an:
$$\Wertebereich{} = \LoesungsRaum{\left]-90\degre; 90\degre\right[ = \LoesungsRaum{\left]-\frac{\pi}{2}; \frac{\pi}{2}\right[} }$$

Beachten Sie, dass im Wertebereich $\Wertebereich{}$ die Grenzen ($-90\degre$und $+90\degre$) nicht enthalten sind. Warum ist das so?

\subsection*{Aufgaben}
\aufgabenFarbe{
  Mit Taschenrechner:\\
%%\TALSAadBFWG{102ff}{87. a)}
a) $\tan(\alpha) = 0.706$ \TRAINER{ $\alpha\approx 35.22 \degre$ und $\alpha\approx 215.22\degre$}\\
b) $\sin(\beta) = 0.3$ \TRAINER{ $\beta\approx 17.458 \degre$ und $\beta\approx 162.54\degre$}\\
c) $\tan(\gamma) = 30$ \TRAINER{ $\gamma\approx 88.091\degre$ und $\gamma\approx 268.09\degre$}\\
Ohne Taschenrechner:\\
%%\TALSAadBFWG{102ff}{87. c)}
d) $-\tan(\alpha) = -tan(-13\degre)$ \TRAINER{ $\alpha = 167\degre$ und $\alpha=347\degre$}\\
e) $\cos(\beta) = 45$ \TRAINER{ Keine Lösung}\\
f) $\tan(\varphi) = \cos(\pi)$ \TRAINER{$\varphi = \frac{3\pi}4$ oder
  $\varphi = \frac{7\pi}4$}
}
\TNTeop{}
%%\newpage

\subsubsection{Für Applikationsentwickler\LoesungsRaum{\,($\mathrm{atan2}()$)}}

Wir betrachten ein kleines Computerspiel, bei dem von einer
Abschussrampe im Punkt $O=(0|0)$ in einem Winkel $\varphi$ ein Geschoss
abgeschossen wird (siehe folgende beiden Graphiken). Die Längen sind
in Einheiten von \texttt{px}\footnote{Ein \texttt{px} ist ein
  sog. \textit{picture element} (kurz \textit{Pixel}) und ist
  ursprünglich die kleinste adressierbare Einheit auf dem Computerdisplay.} gegeben.

\begin{tabular}{p{11cm}c}
\textbf{Problem I}: Gegeben ist ein Winkel von $107\degre$ von der
Abschussrampe aus gesehen und gesucht sind die Koordinaten des Punktes
$P=(x_P|y_P)$, der in der Entfernung von 350 [px].
&
\raisebox{-4cm}{\includegraphics[width=5cm]{tals/trig3/img/atan2PunktGesucht.jpg}}\\
\end{tabular}

Lösung: $P=( \LoesungsRaumLang{350[px]\cdot{}\cos(107\degre)\approx{}-102[px]} | \LoesungsRaumLang{350[px]\cdot{}\sin(107\degre)\approx{}335[px]})$


\begin{tabular}{p{11cm}c}
\textbf{Problem II}: Gegeben ist neben dem Ursprung $O = (0 | 0)$ ein
weiterer Punkt $P=(x_P|y_P)$. In welchem Winkel $\varphi$ befindet sich
$P$ von $O$ aus gesehen? Gesucht ist der Winkel $\varphi$ im
mathematisch positiven Sinne.

&
\raisebox{-4cm}{\includegraphics[width=5cm]{tals/trig3/img/atan2WinkelGesucht.jpg}}\\
\end{tabular}

Beispiele und Lösung:

\begin{tabular}{l|l|l}
  Punkt          & Winkel                  & allgemeine Formel? \\ \hline
  $P=(+10 |  60)$ & $\varphi = \LoesungsRaum{80.5\degre}$  & $\varphi=\LoesungsRaum{\arctan(\frac{y}{x})}$ \\ \hline
  $P=(-10 |  60)$ & $\varphi = \LoesungsRaum{99.46\degre}$ & $\varphi=\LoesungsRaum{180\degre - \arctan(\frac{y}{x})}$ \\ \hline
  $P=(  0 |  70)$ & $\varphi = \LoesungsRaum{90\degre}$    & $\varphi=$\LoesungsRaum{Sonderfall, sonst Division durch 0} \\ \hline
  $P=(-20 |   0)$ & $\varphi = \LoesungsRaum{180\degre}$   & $\varphi=\LoesungsRaum{\arctan(\frac{y}{x}) + 180\degre}$ \\ \hline
  $P=(x_P | y_P)$ & allgemeiner Fall?      & $\varphi=\LoesungsRaum{\text{\textbf{atan2}}(\textbf{y}, \textbf{x})}$\\ \hline
\end{tabular}

\TNT{2.8}{
Beispiel unter \texttt{html-css-js.com}: im HTML-Fenster: \texttt{<p
  id='test'/>} und im javascript-Fenster (tippen, nicht copy, da
versteckte Sonderzeichen vorhanden):

\texttt{document.getElementById('test').innerHTML = Math.atan2(60, -10)}%%

Im Gradmaß: \texttt{180 * Math.atan2(60, -10) / Math.PI}%%
}%% END TNT

