%% Trigonometrie III
%% Periodische Lösungen
%% 2020 - 12 - 21 φ@bbw.ch

\section{Periodische Lösungen}\index{periodische Lösungen}

\subsection*{Lernziele}

Periodische Lösungen bei trigonometrischen Gleichungen.

\begin{itemize}
\item Periodische Lösungen bei trigonometrischen Gleichungen
\item Arkusfunktionen interpretieren und periodische Lösungen angeben
\item Substitution bei trigonometrischen Gleichungen
\end{itemize}
\newpage


\subsection{Einstiegsbeispiel}
Das folgende Beispiel stammt 1:1 aus dem Lehrplan:

Lösen Sie die folgende Gleichung im Definitionsbereich $x \in \DefinitionsMenge{} = [ 0, 2\pi]$ und finden Sie die Lösungsmenge für $x$:

$$\cos\left(2x+\frac{\pi}{9}\right) = 0.5$$

Die Gleichung hat offensichtlich periodische Lösungen, welche wir
entweder am Einheitskreis, mittels der Cosinus-Funktion oder mit dem
Taschenrechner herausfinden und schlussendlich auf den vorgegebenen
Definitionsbereich einschränken müssen.

\subsubsection{Eine Lösung von Hand}

Substituiere \TRAINER{$y=2x+\frac{\pi}9$:}
$$\cos(y) = 0.5 \Longrightarrow y=\LoesungsRaumLen{30mm}r{\frac{\pi}3} \left(=60\degre\right)$$
Rücksubstitution:
$$ 2x+\frac{\pi}9 = \TRAINER{\frac{\pi}3}$$
\TNT{4}{
  $$  18x + \pi = 3\pi$$
  $$x=\frac{2\pi}{18} = \frac{\pi}9 $$
}%%%

Probe:
\TNTeop{
  $x= \frac{\pi}9$

  $\cos\left(2\cdot{}\frac\pi9 + \frac\pi9\right) =
  \cos\left(\frac{3\pi}9\right) = \cos\left(\frac{\pi}3\right) = 0.5$
}
%% \newpage

\subsubsection{Alle Lösungen im Definitionsbereich}
\textbf{Lösungsidee:} Substitution:
Wir setzen wieder $y := 2x + \frac{\pi}{9}$ und erhalten somit:

\TNT{2}{
$$y:= 2x+\frac{\pi}{3} \Longrightarrow \cos(y) = 0.5.$$
}


Aus dem Einheitskreis lesen wir die Lösung für $y$ ab:


\TNT{2}{$y_1= \frac{\pi}{3}$  und   $y_2=\frac{5\pi}{3}$
}


Alle periodischen Lösungen (auch negative!):

\TNT{4}{
  $$y_1 ={\color{blue} \frac{\pi}{3} + z\cdot{}2\pi: z\in\mathbb{Z}}$$
  
  $$y_2 = {\color{green}\frac{5\pi}{3}+ z\cdot{}2\pi: z\in\mathbb{Z}}$$
}


Rücksubstituton $y_1$:
\TNT{6}{
  $${\color{blue}y_1= 2x_1 + \frac{\pi}{9}}$$

  $${\color{blue}\frac\pi3 + z\cdot{}2\pi = 2x_1 + \frac{\pi}{9}}$$
  $${\color{blue}\frac{2\pi}9 + z\cdot{}2\pi = 2x_1}$$
  $${\color{blue}\frac{\pi}9  + z\cdot{}\pi = x_1}$$
}%% end TNT

Analog ${\color{ForestGreen}x_2}$:
\TNTeop{
${\color{ForestGreen}\frac{5\pi}{3} + z\cdot{}2\pi = 2x_2 +
    \frac{\pi}{9}}$.

  Ebenfalls durch Abziehen von $\frac{\pi}{9}$ und Halbieren ergibt sich

  $${\color{ForestGreen}x_2 = \frac{7\pi}{9} + z\cdot{}\pi.}$$
}%% end TNTeop
%% implicit \newpage


Als allerletztes gilt es noch durch «Abzählen» herauszufinden, welche
der Lösungen im vorgegebenen Definitionsbereich $\DefinitionsMenge{} =
[0, 2\pi[$ liegen:

    $$x_1= \frac\pi9 + z\cdot{}\pi$$
    $$x_2= \frac{7\pi}9 + z\cdot{}\pi$$

    Für ${\color{blue}x_1}$ sind dies:

    \TNT{4}{${\color{blue}\frac{\pi}{9}}$ und
      ${\color{blue}\frac{\pi}{9} + 1\cdot{}\pi}$}%% end TNT

    
      \vspace{5mm}
      
      Für ${\color{ForestGreen}x_2}$ sind dies:

      \TNT{4}{
      ${\color{ForestGreen}\frac{7\pi}{9}}$ und
        ${\color{ForestGreen}\frac{7\pi}{9} + 1\cdot{} \pi}$.
      }%% end TNT

      \vspace{5mm}

      $$\lx = \LoesungsRaumLen{80mm}{\left\{ \frac\pi9; \frac{7\pi}9;
        \frac{10\pi}9; \frac{16\pi}9\right\}}$$
      
\newpage


\subsection{Referenzaufgabe}
Berechnen Sie zwei Lösungen für $x$ im Intervall $[0,2\pi]$ für die Gleichung
$$\sin\left(3x+\frac{\pi}{3}\right) = -\frac{\sqrt{3}}{2}.$$
\TNTeop{1. Substitution: $y:=3x+\frac{\pi}{3}$\\
  \hrule
  2. $\sin(y)=\frac{-\sqrt{3}}{2}$\\
  Daraus lesen wir $y$ bei $240\degre$ und bei $300\degre$ ab. Oder im
  Bogenmaß: $y=\frac43\pi$ oder $y=\frac53\pi$\\
  \hrule
  3. Periodische Lösungen:
  $y_1 = \frac43\pi + z\cdot{}2\pi$ und $y_2=\frac53\pi +
  z\cdot{}2\pi$ mit $z\in\mathbb{Z}$\\
  \hrule
  4. Resub $y_1$:\\
  $$y_1: \frac{4\pi}3 + z\cdot{}2\pi = 3x + \frac{\pi}3$$
  Daraus ergibt sich $x_1$:
  $$x_1 = \frac{\pi}9(3+6z) = ...; \frac{3\pi}9; \frac{9\pi}9; \frac{15\pi}9; \frac{21\pi}9; ...$$

 \vspace{5mm}

 Resub $y_2$:\\
  $$y_2: \frac{5\pi}3 + z\cdot{}2\pi = 3x + \frac{\pi}3$$
  Daraus ergibt sich $x_2$:
  $$x_2 = \frac{2\pi}9(2+3z) = ...; \frac{4\pi}9; \frac{10\pi}9;
  \frac{16\pi}9; \frac{22\pi}9; ...$$
  \hrule
  5. Einschränken auf $[0; 2\pi[$:
  $$x\in\left\{\frac{3\pi}9, \frac{4\pi}9,\frac{9\pi}9, \frac{10\pi}{9}, \frac{15\pi}{9},\frac{16\pi}{9}\right\}$$

}%% END TNTeop
%% implicit \newpage

    \TNT{12}{... Platz für weitere Berechnungen ...}

\begin{rezept}{Periodische Lösungen}{}
\begin{itemize}
    \item Substitution
    \item Lösen, Ablesen
    \item periodische Lösungen angeben
    \item Rücksubstitution
    \item Lösung durch «Abzählen» auf Definitionsbereich einschränken
\end{itemize}
\end{rezept}%%

\newpage

\subsection*{Aufgaben}

\olatLinkTALSStrukturaufgabenGLF{Teil 1}{5}{3. o) n)}
%%\aufgabenFarbe{Strukturaufgaben Grundlagenfach Aufg. 3. o)}

\AadBMTG{138}{4. a) b) }

\aufgabenFarbe{Finden Sie alle Lösungen der Gleichung\\
  $\tan\left(\frac{x}2 - \pi\right) = - \sqrt{3}$\\
  Im Definitionsbereich $[0;2\pi]$.
}%% end aufgabenFarbe

\TNTeop{ Lösung der vorherigen Aufgabe:
  $$\tan(y) = -\sqrt{3} \Longrightarrow y_1=\frac56\pi + z\cdot{}2\pi
  \text{ und } y_2 = \frac23\pi + z\cdot{} 2\pi$$

  $$y_1: \frac{x}2 -\pi = \frac{5\pi}3 + 2z\pi$$
  Hieraus resultiert nur eine Lösung für $z=-1$ im Definitionsbereich,
  nämlich $x=\frac{4\pi}3$.
  
  $$y_2: \frac{x}2 - \pi = \frac{2\pi}3 + 2z\pi$$
  Hieraus resultiert keine Lösung im Definitionsbereich.

  ergo:

  $$\lx = \left\{\frac{4\pi}3\right\}$$
}%% end TNTeop
%% implicit
%%\newpage
