\subsubsection{Für Applikationsentwickler\LoesungsRaum{\,($\mathrm{atan2}()$)}}

Wir betrachten ein kleines Computerspiel, bei dem von einer
Abschussrampe im Punkt $O=(0|0)$ in einem Winkel $\varphi$ ein Geschoss
abgeschossen wird (siehe folgende beiden Graphiken). Die Längen sind
in Einheiten von \texttt{px}\footnote{Ein \texttt{px} ist ein
  sog. \textit{picture element} (kurz \textit{Pixel}) und ist
  ursprünglich die kleinste adressierbare Einheit auf dem Computerdisplay.} gegeben.

\begin{tabular}{p{11cm}c}
\textbf{Problem I}: Gegeben ist ein Winkel von $107\degre$ von der
Abschussrampe aus gesehen und gesucht sind die Koordinaten des Punktes
$P=(x_P|y_P)$, der in der Entfernung von 350 [px].
&
\raisebox{-4cm}{\includegraphics[width=5cm]{tals/trig3/img/atan2PunktGesucht.jpg}}\\
\end{tabular}

Lösung: $P=( \LoesungsRaumLang{350[px]\cdot{}\cos(107\degre)\approx{}-102[px]} | \LoesungsRaumLang{350[px]\cdot{}\sin(107\degre)\approx{}335[px]})$


\begin{tabular}{p{11cm}c}
\textbf{Problem II}: Gegeben ist neben dem Ursprung $O = (0 | 0)$ ein
weiterer Punkt $P=(x_P|y_P)$. In welchem Winkel $\varphi$ befindet sich
$P$ von $O$ aus gesehen? Gesucht ist der Winkel $\varphi$ im
mathematisch positiven Sinne.

&
\raisebox{-4cm}{\includegraphics[width=5cm]{tals/trig3/img/atan2WinkelGesucht.jpg}}\\
\end{tabular}

Beispiele und Lösung:

\begin{tabular}{l|l|l}
  Punkt          & Winkel                  & allgemeine Formel? \\ \hline
  $P=(+10 |  60)$ & $\varphi = \LoesungsRaum{80.5\degre}$  & $\varphi=\LoesungsRaum{\arctan(\frac{y}{x})}$ \\ \hline
  $P=(-10 |  60)$ & $\varphi = \LoesungsRaum{99.46\degre}$ & $\varphi=\LoesungsRaum{180\degre - \arctan(\frac{y}{x})}$ \\ \hline
  $P=(  0 |  70)$ & $\varphi = \LoesungsRaum{90\degre}$    & $\varphi=$\LoesungsRaum{Sonderfall, sonst Division durch 0} \\ \hline
  $P=(-20 |   0)$ & $\varphi = \LoesungsRaum{180\degre}$   & $\varphi=\LoesungsRaum{\arctan(\frac{y}{x}) + 180\degre}$ \\ \hline
  $P=(x_P | y_P)$ & allgemeiner Fall?      & $\varphi=\LoesungsRaum{\text{\textbf{atan2}}(\textbf{y}, \textbf{x})}$\\ \hline
\end{tabular}

\TNT{2.8}{
Beispiel unter \texttt{html-css-js.com}: im HTML-Fenster: \texttt{<p
  id='test'/>} und im javascript-Fenster (tippen, nicht copy, da
versteckte Sonderzeichen vorhanden):

\texttt{document.getElementById('test').innerHTML = Math.atan2(60, -10)}%%

Im Gradmaß: \texttt{180 * Math.atan2(60, -10) / Math.PI}%%
}%% END TNT
