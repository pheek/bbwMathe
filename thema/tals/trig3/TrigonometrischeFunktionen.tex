%% Trigonometrie III
%% Trigonometrische Funktionen
%% 2020 - 12 - 21 φ@bbw.ch

\section{Trigonometrische
  Funktionen}\index{Funktionen!trigonometrische}\index{Trigonometrische
  Funktionen}

Lernziele:
\begin{itemize}
\item Aufzeichnen von $\sin()$-, $\cos()-$ und $\tan()$-Funktionen
\item Sie können trigonometrische Funktionen strecken, stauchen und
  verschieben (Amplitude $\widehat{y}$, Frequenz $\omega$, Phase
  $\varphi$): $y=\widehat{y}\cdot{}\sin(\omega\cdot{}(x+\varphi))$
\end{itemize}


\subsection{Periodische Vorgänge}
Machen Sie Beispiele von periodischen Vorgängen
\begin{itemize}
  \item \TRAINER{Wellen: Schallwellen / Wasser}
    \noTRAINER{......................................}
  \item \TRAINER{Tageslänge, Jahresrythmus}
    \noTRAINER{......................................}
  \item \TRAINER{Gezeiten}
    \noTRAINER{......................................}
  \item \TRAINER{Puls}
    \noTRAINER{......................................}
  \item \TRAINER{Wechselstrom}
    \noTRAINER{......................................}
  \item \TRAINER{Thermostatregelung (Kühlschrank)}
    \noTRAINER{......................................}
\end{itemize}
\newpage

\subsection{Sinus-, Cosinusfunktionen}
Sinus, Cosinus und Tangens können nicht nur als Verhältnisse angesehen
werden, sondern auch als Funktionen abgebildet werden.
Zeichnen Sie zu jedem angegebenen Winkel von $-270\degre - 450\degre$ den zugehörigen
\textbf{Sinus}-Wert ins folgende Koordinatensystem:

\noTRAINER{\trigsysD}\TRAINER{\trigsysDsin}

%%\bbwCenterGraphic{10cm}{tals/trig3/img/coordSystem0-450.png}


Zeichnen Sie zu jedem angegebenen Winkel von $-270\degre - 450\degre$ den zugehörigen
\textbf{Cosinus}-Wert ins folgende Koordinatensystem:

\noTRAINER{\trigsysD}\TRAINER{\trigsysDcos}
%%\bbwCenterGraphic{10cm}{tals/trig3/img/coordSystem0-450.png}

\newpage

\subsection{Tangensfunktion}
Zeichnen Sie zu jedem der angegebenen Winkel von $0\degre - 450\degre$ den zugehörigen
\textbf{Tangens}-Wert ins folgende Koordinatensystem:

\bbwCenterGraphic{10cm}{tals/trig3/img/coordSystem0-450.png}

\subsection{Definitions- und Wertebereiche}

Geben Sie von Sin-, Cos- und Tangensfunktion die Definitions- und
Wertebereiche an:

\begin{bbwFillInTabular}{|l|l|c|r|} \hline
  Funktion & Definitionsbereich $\DefinitionsMenge{}$ & Wertebereich
  $\Wertebereich{}$ & Nullstellen\\ \hline
  $\sin()$ & \TRAINER{$\mathbb{R}$} & \TRAINER{$[-1,1]$} &
  \TRAINER{$z\cdot{}180\degre| z\in\mathbb{Z}$} \\ \hline
  $\cos()$ & \TRAINER{$\mathbb{R}$} & \TRAINER{$[-1,1]$}& \TRAINER{$90+  z\cdot{}180\degre| z\in\mathbb{Z}$}\\ \hline
  $\tan()$ & \TRAINER{$\mathbb{R}\backslash\{90\degre +
    z\cdot{}180\degre| z\in\mathbb{Z} \}$} & \TRAINER{$\mathbb{R}$} & \TRAINER{$z\cdot{}180\degre| z\in\mathbb{Z}$}\\ \hline
\end{bbwFillInTabular}

\begin{bemerkung}{}{}
  Die Polstellen des Tangens sind genau die Nullstellen des Cosinus, denn es gilt ja:
  $$tan = \frac{sin}{cos}$$
\end{bemerkung}

\newpage


\subsection{Amplitude, Frequenz und Phase}\index{Amplitude}\index{Frequenz}\index{Phase}
\aufgabenFarbe{Lösen Sie als Einstieg das Blatt «Transformationen Aufgaben» im OLAT von M.\,\,Rohner.}


Danach betrachten wir die folgende Funktion:

$$y = f(x) = 0.7 \cdot \sin(1.5\cdot(x - 20\degre))$$

\renewcommand{\bbwFunctionColour}{gray!30}\noTRAINER{\trigsysDsin{}}
\renewcommand{\bbwFunctionColour}{blue}\TRAINER{\trigsysDFct{2*0.7*sin(1.5*(50*\x-20))}}

Zum Skizzieren beginnen wir mit uns bekannten Winkeln. Winkeln also, bei denen wir die Sinus-Werte bereits kennen:

$$\sin(0\degre)  = 0  ; \sin(30\degre) = 0.5; \sin(90\degre) = 1  $$

\textbf{Leider}: Für $x=90\degre$ \zB erhalten wir also ... $$f(x) = f({\color{ForestGreen}90\degre}) = 0.7\cdot\sin(1.5\cdot({\color{ForestGreen}90\degre}-20\degre)) = 0.7\cdot{}\sin(105\degre)$$ ... und das macht die Sache nicht wirklich einfach.


Wir müssen also irgendwie erreichen, dass der Term $$(1.5\cdot(x-20\degre))$$
entweder $0\degre$, $30\degre$ oder $90\degre$ ist, dann können
wir $$f(x) := 0.7\cdot\sin(1.5\cdot(x-20\degre))$$ einfach berechnen.


\newpage
\subsubsection{Charakteristische Punkte}

\begin{itemize}
\item  $0.7\cdot\sin(0\degre) = 0.7\cdot\sin(1.5\cdot(x-20\degre))$:
  Löse $0\degre = 1.5\cdot(x-20\degre)$
  $$\longrightarrow x=20\degre$$
  Somit ist $f(20\degre) = 0.7\cdot{}\sin(0\degre) = 0$.
\item  $0.7\cdot\sin(30\degre) = 0.7\cdot\sin(1.5\cdot(x-20\degre))$:
  Löse $30\degre = 1.5\cdot(x-20\degre)$
  $$\longrightarrow 40\degre$$
  Somit ist $f(40\degre) = 0.7\cdot{}\sin(30\degre) = 0.7\cdot{}0.5 = 0.35$.
\item  $0.7\cdot\sin(90\degre) = 0.7\cdot\sin(1.5\cdot(x-20\degre))$:
  Löse $90\degre = 1.5\cdot(x-20\degre)$
  $$\longrightarrow 80\degre$$
  Somit ist $f(80\degre) = 0.7\cdot{}\sin(90\degre) = 0.7\cdot{}1 = 0.7$.
\end{itemize}
Die anderen Winkel folgen durch Symmetrieüberlegungen.


Was fällt Ihnen auf?
\TNTeop{
  Die Funktion hat eine Amplitude von 0.7

  Die Funktion ist um 20 Grad nach rechts verschoben

  Die Frequenz nimmt zu um Faktor 1.5
}%% END TNTeop
%%\newpage

\subsection{Referenzaufgaben}
1. Sie möchten die \textbf{Sinusfunktion} um 50\% in $y$-Richtung strecken und um
$30\degre$ nach rechts verschieben.

a) Wie sieht die Funktion aus?

\renewcommand{\bbwFunctionColour}{gray!30}\noTRAINER{\trigsysDsin{}}
\renewcommand{\bbwFunctionColour}{blue}\TRAINER{\trigsysDFct{3*sin(\x*50-30)}}

%%\bbwCenterGraphic{10cm}{tals/trig3/img/coordSystem0-450.png}

b) Wie lautet die Funktionsgleichung

$$y = \LoesungsRaum{1.5 \cdot{} \sin(x - 30\degre)}$$



2. Sie möchten die Sinusfunktion so verschieben, dass sich die neuen Nullstellen bei
$-90\degre$, $90\degre$, $270\degre$, ... befinden.

a) Wie sieht die Funktion aus?

\noTRAINER{\renewcommand\bbwFunctionColour{gray!30}\trigsysDsin{}}
\TRAINER{\renewcommand\bbwFunctionColour{blue}\trigsysDcos{}}

b) was fällt Ihnen auf?

\begin{gesetz}{Sinus vs. Cosinus}{}
  Es gilt: $$\cos(\varphi)=\sin(\varphi+90\degre)=\sin\left(\varphi + \frac\pi2\right)$$
\end{gesetz}
\newpage

3. Sie möchten die \textbf{Cosinusfunktion} so verändern, dass sich die
Nullstellen nicht mehr bei $-90\degre$, $90\degre$, $270\degre$
etc. befinden, sondern bei $-60\degre$, $60\degre$, $180\degre$
etc. also bei $60\degre \pm$ Vielfachen von
$120\degre$. Nullstellenmenge = $\{60\degre+z\cdot{}120\degre|
z\in\mathbb{Z}\}$

Machen sie eine Skizze

\renewcommand\bbwFunctionColour{gray!30}

%%%\noTRAINER{\trigsysDsin{}}\TRAINER{\trigsysDFct{2*0.7*sin(1.5*(50*\x-20))}}

\noTRAINER{\trigsysDcos{}}{\TRAINER{\trigsysDFct{ 2*0.7*cos(1.5*(50*\x))  }}
%%\bbwCenterGraphic{10cm}{tals/trig3/img/coordSystem0-450.png}

Wie lautet die Funktionsgleichung?

$$y = \LoesungsRaum{\cos\left(x\cdot\frac{90\degre}{60\degre}\right)}$$
\newpage




\subsubsection{Allgemeine Form der Sinusfunktion}\index{Sinusfunktion!allgemeine Form}

Die allgemeine Sinusfunktion (ohne Verschiebung in $y$-Richtung und
ohne Phasenverschiebung) lautet:
$$y=\widehat{y}\cdot{}\sin(\omega\cdot{}x)$$

Dabei sind:
\begin{itemize}

\item
  Amplitude $\widehat{y}$: Ausschlag in $y$-Richtung.
\item
  Frequenz $\omega$ (kleines Omega): Je größer $\omega$, umso schneller wechselt die
  Funktion die Werte von positiv nach negativ und umgekehrt.
\end{itemize}

Wird zusätzlich eine Phasenverschiebung in $x$-Richtung vorgenommen, so lautet die Gleichung:

\begin{gesetz}{}{}
  Es gilt
  $$y=\widehat{y}\cdot{}\sin(\omega\cdot{}(x-\varphi))$$

mit
\begin{itemize}

\item
  Amlitude $\widehat{y}$
\item
  Frequenz $\omega$
\item
  Phasenverschiebung\footnote{Dabei wird der Funktionsgraph in positiver $x$-Richtung um
    $\varphi$ verschoben.} $\varphi$
\end{itemize}
\end{gesetz}


\subsection*{Aufgaben}

\aufgabenFarbe{Zeichnen Sie die Funktion $y=\sin\left(x+\frac{3\pi}{2}\right)\TRAINER{=-\cos(x)}$.}

Im folgenden die Aufgaben im Geometriebuch. Diese kommen im Algebrabuch in leicht abgewandelter Form auch vor:

%%\TALSAadBFWG{116ff}{165. a) e) g), 167., 179.}
\TALSAadBMTG{140ff}{13. a) b) , 15., 18. b) [b) mit
    \texttt{geogebra.org}] 18. a) , 23. und 21. b) c) e) challenge: d)}

\TRAINER{\TALS{Aufgabe 21. d) ist der Einstieg in die Arkusfunktionen}}
