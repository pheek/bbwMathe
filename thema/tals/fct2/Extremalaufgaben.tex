%%
%% Maximierungsaufgaben 2d (Marthaler)
%%

\section{Extremalaufgaben}\index{Extremalaufgaben}

\TadBMTA{271}{15.6}

\subsection{Referenzaufgabe}
Durch die Funktionsgleichung $y=(x-3)^2+2$ ist die Parabel $f$ definiert.
Unter den deren Graphen wird ein Rechteck im ersten
Quadranten so eingepasst,
dass zwei Seiten auf den beiden Koordinatenachsen liegen. Die dem Ursprung
gegenüberliegende Ecke liegt auf dem Graphen der Funktion aber links
vom Scheitelpunkt.

Bestimmen Sie die maximal mögliche Rechtecksfläche mit dem Taschenrechner.

\TNTeop{
  Vorbereitung


  Skizze:
  \bbwGraph{-1}{4}{-1}{6}{
    \bbwFunc{(\x-3)*(\x-3)+2}{1:4}
  }
  
  $f(x) := (x-3)^2+2$

  sog. Zielfunktion:
  $A = A(x) := x \cdot{} f(x)$

  1. Weg: Graph zu a(x) zeichnen und Graph analysieren und maximales
  $x$ bei 1.42265 «ablesen».
  
  2. Weg: xMax := fMax(a(x),x,0,3), dann

  Am Schluss das gefundene $x$ in die Zielfunktion einsetzen $aMax := a(xMax)$

  Lösung $x_{\text{max}} \approx 1.42265$ und $a_{\text{max}}
  \approx{} 6.385$
  
}

%%\newpage

\subsection*{Aufgaben}
\AadBMTA{281ff}{60., 55.a), 57. a), 58., 59., 62.}

\newpage
