%%
%% Maximierungsaufgaben 2d (Marthaler)
%%

\section{Extremalaufgaben}\index{Extremalaufgaben}

\TadBMTA{271}{15.6}

\subsection{Referenzaufgabe}
Gegeben ist die Gerade $g$ mit $y=-0.5x+3.5$. Unter dem Graphen von $g$
wird im ersten Quadranten ein rechtwinkliges Dreieck einbeschrieben,
wie die Abbildung zeigt:

  \bbwGraph{-1}{8}{-1}{4}{
    \bbwFunc{-0.5*\x+3.5}{-0.25:7.5}
    \bbwLine{0,0}{3.5,1.75}{red}
    \bbwLine{0,0}{3.5,0}{red}
    \bbwLine{3.5,0}{3.5,1.75}{red}
    \bbwLetter{3.5,-0.3}{x_\Delta}{red}
    \TRAINER{
      \bbwFunc{-0.25*(\x-3.5)*(\x-3.5)+49/16}{-0.1:7.1}%%
      \bbwLetter{4.2,0.7}{h(x_\Delta)}{red}
      \bbwLetter{5,3}{A(x)}{blue}
    }%% end TRAINER
  }
  
  Welche maximale Dreiecksfläche ist unter diesen Bedingungen möglich?

  1. \textbf{Hauptbedingung}\index{Hauptbedingung!Extremwertaufgaben}

  \TNT{0.8}{Zu maximieren ist die \textbf{Fläche}.}


  2. \textbf{Nebenbedingung}\index{Nebenbedingung!Extremwertaufgaben}

  \TNTeop{Die senkrechte Kathete $h$ darf nicht höher als der Graph von
  $g$ sein: $h(x) \le g(x)$. Wir setzen $h(x) = -0.5x + 3.5$. }

3. \textbf{Zielfunktion}\index{Zielfunktion!Extremwertaufgaben} (für welches Ziel soll die Optimierung erreicht werden)

\TNT{6}{Die Zielfunktion ist hier die \textbf{Fläche $A$ in Abhängigkeit von $x$}.
    $$A(x) = \frac12\cdot{}  x \cdot{} h(x)$$ mit $h(x) = g(x)$.
    Somit
    $$A(x) = \frac12 \cdot{} x \cdot{} (-0.5x+3.5) = -0.25 \cdot{} x^2
    + 1.75 x = \frac{-1}4 x^2 + \frac74\cdot{}x$$
  }

  4. Zielfunktion $A(x)$ \textbf{maximieren} (Dies liefert die \textbf{Extremstelle}\index{Extremstelle!Extremwertaufgaben})

  \TNT{6}{
    Um $A(x) =  \frac{-1}4 x^2 + \frac74\cdot{}x $ zu maximieren, suchen wir den
    Scheitelpunkt. Es gilt
    $$x_S = x_\Delta = \frac{-b}{2a} =
    \frac{-\left(\frac{7}4\right)}{2\cdot{} \left(\frac{-1}4\right)}
    = 3.5 = \frac72$$
  }

  5. \textbf{Extremwert}\index{Extremwert!Extremwertaufgaben} rechnen

  \TNT{4}{
    $$A(x_\Delta) = \frac12\cdot{} x_\Delta \cdot{} h(x_\Delta) =
    \frac12 \cdot{} \frac72 \cdot{} \frac{7}{4} = \frac{49}{16} = 3.0625$$
  }

  Schlusssatz:

  \TNTeop{Die maximal mögliche Fläche misst $\frac{49}{16}$ (Einheitsquadrate).}
  
\newpage

\subsection*{Aufgaben}
\AadBMTA{281ff}{60., 55.a), 57. a), 59., 62. (optional 52. und 57. c))}


\subsection{Taschenrechneraufgabe}
Durch die Funktionsgleichung $y=(x-3)^2+2$ ist die Parabel $f$ definiert.
Unter deren Graphen wird ein Rechteck im ersten
Quadranten so eingepasst,
dass zwei Seiten auf den beiden Koordinatenachsen liegen. Die dem Ursprung
gegenüberliegende Ecke liegt auf dem Graphen der Funktion aber links
vom Scheitelpunkt.

Bestimmen Sie die maximal mögliche Rechtecksfläche mit dem Taschenrechner.

\TNTeop{
  Vorbereitung


  Skizze:
  \bbwGraph{-1}{4}{-1}{6}{
    \bbwFunc{(\x-3)*(\x-3)+2}{1:4}
  }
  
  $f(x) := (x-3)^2+2$

  sog. Zielfunktion:
  $A = A(x) := x \cdot{} f(x)$

  1. Weg: Graph zu a(x) zeichnen und Graph analysieren und maximales
  $x$ bei $1.42265$ «ablesen».
  
  2. Weg: xMax := fMax(a(x),x,0,3), dann

  Am Schluss das gefundene $x$ in die Zielfunktion einsetzen $aMax := a(xMax)$

  Lösung $x_{\text{max}} \approx 1.42265$ und $a_{\text{max}}
  \approx{} 6.385$
}

\newpage
