\section{Berührende Graphen}\index{Graphen!berührende}\index{berührende Graphen}

\textbf{Einführungsbeispiel}


Gegeben ist die Parabel $f: y=\frac{1}{4}x^2 -\frac12x +\frac14$ und von einer
Geraden $g$ ist der $y$-Achsenabschnitt $b = -2$ gegeben.

Gesucht ist von der Geraden $g$ die Steigung $a$ so, dass die
Gerade die Parabel tangiert; also genau in einem Punkt berührt.

Wo (in welchem Punkt $S=(x_S|y_S)$) tangiert also die Gerade $g$ die Parabel $f$?

In der folgenden Skizze sind drei mögliche Geraden mit $y$-Achsenabschnitt
$-2$ gezeichnet. Nur eine dieser drei Geraden \textit{tangiert} die Parabel.

\bbwGraph{-4}{4}{-3}{5}{
  \draw[thick,color=blue,variable=\x,domain=-3.5:4] plot ({\x},   {0.25*\x*\x -0.5*\x + 0.25});

  \draw[color=red,variable=\x,domain=-3:5] plot ({\x},{2*\x -2});
  \draw[color=red,variable=\x,domain=-3:5] plot ({\x},{0.5*\x - 2});
  \draw[color=cyan,thick,variable=\x,domain=-3:5] plot ({\x},{1*\x  -2});

  \bbwDot{3, 1}{green}{north}{S}

  %%%\draw[thick,color=blue,variable=\x,domain=-1:5] plot ({\x}, {0.5*\x*\x - 2*\x + 3}); 
  %%\draw[color=red,variable=\x,domain=-1:5] plot ({\x},{0.5*\x + 1});
  %%\draw[color=red,variable=\x,domain=-1:5] plot ({\x},{0.5*\x - 1.5});
  %%\draw[color=cyan,thick,variable=\x,domain=-1:6] plot ({\x},{0.5*\x  -0.125});
  %%\bbwDot{2.5, 1.125}{green}{north}{P}
}
\newpage
\textbf{Lösungsidee}: Wähle als Parameter die Steigung $a$  (der Geraden)
und berechne die Schnittpunkte mit $f(x) = g(x)$. Nur für ein $a$ gibt
es genau eine Lösung (Diskriminante = 0). Den $x$-Wert dieses
Schnittpunkts nennen wir $x_S$:

\TNT{3.2}{
 
\begin{tabular}{rclr}
$f(x_S)$                      & $=$ & $g(x_S)$                              & \\
$\frac14x^2-\frac12x+\frac14$ & $=$ & $ax-2$                                & \\
$0$                           & $=$ & $\frac14x^2+(-\frac12-a)-\frac94$     & (I)\\
\end{tabular}
\vspace{10mm}
}%% END TNT

Die Diskriminante $D=B^2 - 4AC$ muss gleich 0 sein. 

\TNT{7.2}{
$A = \frac14$, $B = -\frac12-a$ und $C = \frac{-9}4$.

  $$D=0=B^2-4AC = (-\frac12-a)^2  - (4\cdot{}A\cdot{}C)$$
  $$0 = (a^2+a+\frac14) - (4\cdot\frac14\cdot{}\frac{-9}4)$$
$$\Longrightarrow 0 = a^2 + a - 2$$
$$\Longrightarrow a_1 = 1 \textrm{ und  } a_2 = -2$$
\vspace{40mm}
}%% END TNT
\newpage

Für den \textbf{Schnittpunkt} müssen wir nun nur doch das gefundene
$a$ in die Gleichung (I) einsetzen ($a_1=1$):

\TNT{6}{
$S_1: a_1 = 1:$
$$0 = \frac14x^2+ (-\frac12 - 1) - \frac94$$
  $$\Longrightarrow x_1 = \frac{\frac32\pm\sqrt{D}}{2\cdot{}\frac14} = \frac32 : \frac12= 3$$

Bem. Der Term $\sqrt{D}$ kann weggelassen werden, denn $D$ wurde ja so
konstruiert, dass $D=0$ sei.


Das $y$ finden wir einfach durch Einsetzen von $x$ in den
Funktionsterm der Geradengleichnug.

$$y_1 = a_1x_1 - 2 = 1\cdot{}3-2 = 1$$
und somit ist
$$S_1 = (3 | 1)$$
}%% END TNT



Zweiter \textbf{Schnittpunkt} ($a_2 = -2$):


\TNT{6}{
$S_2: a_2 = -2:$
$$0 = \frac14x^2+ (-\frac12 - (-2)) - \frac94$$
$$\Longrightarrow x_2 = \frac{-\frac32\pm\sqrt{D}}{2\cdot{}\frac14} = -\frac32 : \frac12= -3$$


$$y_2 = a_2x_2+b = -2\cdot{}3-2 = 4$$
und somit ist
$$S_2 = (-3 | 4)$$

}%% END TNT
\newpage
\TRAINER{(Je nach Zeit wäre 699. a) noch eine Vorzeigeaufgabe.)}

\subsection{Aufgaben}
\TRAINER{Achtung bei 702., dass nicht das Gleichungssystem bereits
  nach dem Gleichsetzen der Kurven aufgestellt wird. Erst nach dem
  Null-Setzen der Diskriminante erhalten wir eine gültige Gleichung
  für das Gleichungssystem.}
%%\TALSAadBMTA{190}{698., 702., 703., 705.}
\TALSAadBMTA{278}{40. a) c), 42. a), 43. a), 44. a)}
\newpage


\subsection{Grenzwerte und Steigungsfunktion (Optional)}

Wie macht das ein CAS, dass es den tiefsten bzw. den höchsten Punkt
einer Funktion bestimmen kann? Hier ein Erklärungsversuch am Beispiel
der quadratischen Funktion.
Betrachten wir die allgemeine quadratische Funktion $$p: y=ax^2 + bx +
c$$
Mit dem selben $a$ und dem selben $b$ kann ich eine Gerade $s$
definieren, die ich die \textbf{Steigungsfunktion}\footnote{Diese
  Steigungsfunktion wird in der Mathematik die
  \textbf{Ableitung}\index{Ableitung} genannt. Genau genommen handelt
  es sich nicht um die Steigung der Parabel, sondern um die Steigung
  einer im Punkt $P=(X_P|f(x_P))$ angelegter Tangente.} nenne:
$$s: y= 2ax+b$$
\newpage


Diese Steigungsfunktion $s$ gibt in jedem Punkt $x$ die Steigung der
Parabel $p$ an.

\begin{beispiel}{Parabel}{}
  Gegeben ist die Parabel $p: y=2.5x^2 - 3x + 6.5$. Wo hat diese Parabel
  ihren Tiefpunkt? Im Scheitelpunkt $x_S =
  \frac{-b}{2a} = \frac{-(-3)}{2\cdot{}2.5}$. Dies ist gleichzeitig die Nullstelle der
  Steigungsfunktion $s: y= 2\cdot{}2.5x - 3$.

  Wir können damit aber auch die Steigung in einem ganz anderen Punkt
  \zB für $x=10$ berechnen. Die Parabel $p$ hat an der Stelle $x=10$
  die Steigung, die durch die Steigungsfunktion im Punkt $10$
  ermittelt wird: $s(10) = 2\cdot{}10\cdot{2.5} - 3 = 47$.
  
\end{beispiel}

\begin{beispiel}{Gerade gesucht}{}
Gegeben ist die Parabel $p: y=4x^2 -6x + 3$. Gesucht ist die Gerade
$g: y=ax+b$, sodass die Gerade die Parabel bei $x=7$ berührt.

\begin{enumerate}
\item Die Steigungsfunktion $s$ lautet: $s: y=2\cdot{}4x - 6$
\item Für $x=7$ hat die Steigungsfunktion den Wert 50, und somit hat
  die Parabel bei $x=7$ die Steigung 50.
\item Um den Berührungspunkt $B$ zu finden, setzen wir 7 diesen in $p$
  ein: $p(7)= 4\cdot{}49-6\cdot{}7+3 = 157$; und somit $B=(\LoesungsRaum{7}|\LoesungsRaum{157})$.
\item Die gesuchte Gerade $g: y=ax+b$ hat also auch die Steigung 50
  und verläuft durch den Berührungspunkt $B=(7|157)$. Also
  $157=50\cdot{}7+b$. Somit ist das gesuchte $b = 157-50\cdot{}7=-193$
  und die gesuchte Funktionsgleichung der Geraden lautet: $y = 50x-193$
  \end{enumerate}
\end{beispiel}
\newpage

\subsubsection{Beweis der Formel der Steigungsfunktion}

Betrachten wir auf auf der $x$-Achse zwei benachbarte Punkte $x$ und
$x+\Delta$. Die Funktionswerte lauten $f(x)$ und $f(x+\Delta)$. Die
\textbf{Steigung} kann nun für eine Gerade bestimmt werden durch:

\bbwCenterGraphic{7cm}{tals/fct3/img/Steigungsfunktion.jpg}

$$\frac{f(x+\Delta) - f(x)}{\Delta}$$

Dies gilt für eine Parabel $p: y=ax^2+bx+c$ näherungsweise auch:
$$\frac{f(x+\Delta)-f(x)}{\Delta} = \frac{(a(x+\Delta)^2 + b(x+\Delta)
  + c) - (ax^2 + bx +c)}{\Delta}=2ax+a\Delta+b$$

Wenn wir nun das $\Delta$ gegen Null gehen lassen, also immer kleinere
Werte einsetzen, so verschwindet der Term $a\Delta$ fast und unsere
Formel stimmt annähernd --- jedoch präzise genug, um dies als Beweis
gelten zu lassen.

Ein anderer Beweis wäre die Tangente effektiv einzusetzen und diese so
zu wählen, dass es genau einen Schnittpunkt gibt, was wieder darauf
zurückführt, dass die Diskriminante = Null gesetzt werden muss. Dann
sind wir exakt, aber der Beweis ist ungleich aufwändiger.
\newpage

