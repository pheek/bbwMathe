
\subsection{Winkel im rechtwinkligen Dreieck: $\sin()$, $\cos()$ und
  $\tan()$}\index{Sinus}\index{Cosinus}\index{Kosinus}\index{Tangens}


\subsection*{Lernziele}
\begin{itemize}
 \item Ähnliche rechtwinklige Dreiecke haben die selben Winkel
 \item Sinus, Cosinus, Tangens
\end{itemize} 
\TadBMTG{90}{6.3}


\olatLinkArbeitsblatt{Tangens}{https://olat.bbw.ch/auth/RepositoryEntry/572162090/CourseNode/102901169721727}{3-5 Höhenmessungen}



\olatLinkArbeitsblatt{Sinus/Cosinus/Tangens}{https://olat.bbw.ch/auth/RepositoryEntry/572162090/CourseNode/101586242856840}{Täglich
5-10 Minuten}

\newpage
%% Sinushand

\subsection{Merkregel der wichtigsten Sinus-Werte}\index{Sinushand}

Anstelle des Ablesens im Gleichseitigen Dreieck kann man sich die wichtigsten Werte
auch mit folgender ``Sinushand'' merken:

\TRAINER{\bbwCenterGraphic{12cm}{tals/trig1/img/sinushand.png}}%% END TRAINER
\noTRAINER{\vspace{98mm}}%%


\TRAINER{%%
\begin{tabular}{r|l|c|l}
      & $sin()$                              & $\cos()$             & $\tan()=\frac{\sin()}{\cos()}$              \\\hline
   0  & $\frac{\sqrt{\textbf{0}}}{2} = 0$    & 1                    &   0                                         \\\hline
  30  & $\frac{\sqrt{\textbf{1}}}{2} = 0.5 $ & $\frac{\sqrt{3}}{2}$ &  $\frac{1}{\sqrt{3}} = \frac{\sqrt{3}}{3}$  \\\hline
  45  & $\frac{\sqrt{\textbf{2}}}{2}$        & $\frac{\sqrt{2}}{2}$ &   1                                         \\\hline
  60  & $\frac{\sqrt{\textbf{3}}}{2}$        & 0.5                  &  $\sqrt{3}$                                 \\\hline
  90  & $\frac{\sqrt{\textbf{4}}}{2} = 1$    & 0                    & nicht definiert
\end{tabular}
}%% END TRAINER

\noTRAINER{%%
\begin{tabular}{r|l|c|l}
      & $sin()$  & $\cos()$  &  $\tan()=\frac{\sin()}{\cos()}$        \\\hline
   0  &          &           &         \\\hline
  30  &          &           &         \\\hline
  45  &          &           &         \\\hline
  60  &          &           &         \\\hline
  90  &          &           &         \\
\end{tabular}
}%% END noTRANIER
\newpage

\subsection*{Aufgaben}
\TALSAadBFWG{84ff}{2. alle 3. a) b) e) 4. a) c) 6. a) 8. 9. 15. 23. 26. a) b) c) 30.}%
\TALSAadBFWG{96}{59}
\GESOAadBMTA{????}{????}
\newpage
