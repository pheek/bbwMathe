%%
%% Trigonometrie aufgaben ohne Rechner
%% Finde die richtige zugehörige Formel
%%

\begin{frage}[1]
  In einem \textbf{rechtwinkligen} Dreieck ist die Kathete a 3.9cm lang und der Winkel $\alpha$ misst $37.4^\circ$.
  
\begin{center}
  \includegraphics[width=5cm]{P_TRIG/img/Dreieck374_39.png}
\end{center}
Wie berechnet sich nun die Hypothenuse $c$?


%% #1: Lösung true false
%% #2: Formel
\newcommand{\MCTrigFrage}[2]{\ifstrequal{#1}{true}{\TRAINER{x}\noTRAINER{$\Box$}}{$\Box$} #2}

\begin{tabular}{|c|c|c|}
  \hline
  \MCTrigFrage{true}{$c=3.9\cdot{}\sin(37.4)$} &%
  \MCTrigFrage{false}{$c=\frac{\sin(37.4)}{3.9}$} &%
  \MCTrigFrage{false}{$c=\frac{3.9}{\sin(37.4)}$}\\%
  \hline
  \MCTrigFrage{false}{$c=3.9\cdot{}\cos(37.4)$} &%
  \MCTrigFrage{false}{$c=\frac{\cos(37.4)}{3.9}$} &%
  \MCTrigFrage{false}{$c=\frac{3.9}{\cos(37.4)}$}\\%
  \hline
  \MCTrigFrage{false}{$c=3.9\cdot{}\tan(37.4)$} &%
  \MCTrigFrage{false}{$c=\frac{\tan(37.4)}{3.9}$} &%
  \MCTrigFrage{false}{$c=\frac{3.9}{\tan(37.4)}$}\\%
  \hline
\end{tabular}

\platzFuerBerechnungen{3.2}

\end{frage}%
