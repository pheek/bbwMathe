%%
%% Trigonometrische Funktionen, die von Hand (ohne TR) gelöst werden können.
%%

\begin{frage}[1]
  Lösen Sie die folgende Gleichung ohne Taschenrechner nach $x$ auf und berechnen Sie $x$:

  $$cos(60^\circ)=\frac{4.3}{x}$$
  \vspace{7mm}
  $$x = \LoesungsRaum{8.6}$$

  \platzFuerBerechnungen{5.2}
\end{frage}


%\begin{frage}[1]
% Berechnen Sie die Länge der Strecke $x$ ohne Taschenrechner: 
%\begin{center}
%\raisebox{-1cm}{\includegraphics[width=6cm]{p_img/trigo/aufgabe60grad35mm.png}}
%\end{center}
%
%$$x=..................\TRAINER{70mm = 7cm}$$
%  
%\platzFuerBerechnungen4}
%\end{frage}


\begin{frage}[2]
  Gegeben ist ein rechtwinkliges Dreieck mit den Seiten $a$, $b$ und
  $c$. $c$ ist die Hypotenuse. Der Winkel $\alpha$ liegt hier wie
  üblich gegenüber der Seite $a$.
  Gegeben ist die Seite $a = 9cm$ und der Winkel $\alpha = 51^\circ$.
  Schreiben Sie die Formel für die Hypotenuse $c$ auf und setzen Sie
  die gegebenen Zahlen ein. Geben Sie nur die Formel mit Zahlen an; im Stil von
  $c = 3.7 \cdot tan(38^\circ)$:
  
  \vspace{7mm}
  
  $$c = \LoesungsRaum{\frac{9cm}{sin(51^\circ)}}$$

  \platzFuerBerechnungen{8}
\end{frage}


\begin{frage}[1]
 Berechnen Sie die Länge der Strecke $x$ ohne Taschenrechner: 
\begin{center}
\raisebox{-1cm}{\includegraphics[width=3cm]{P_TRIG/img/dreiecke/aufgabe30grad4_6cm.png}}
\end{center}

$$x=\LoesungsRaum{2.3cm}$$

\platzFuerBerechnungen{4}
\end{frage}
  

