%%
%% 2019 07 04 Ph. G. Freimann
%%

\newpage
\section{Das allgemeine Dreieck}\index{Dreieck!allgemeines}

\subsection{Fläche im allgemeinen Dreieck}\index{Fl\"ache!Dreieck}
\TALS{(S. 96 Kap. 2.1.3 \cite{frommenwiler18geom})}

\bbwCenterGraphic{6cm}{tals/trig1/img/dreieck_sws.png}

Im allgemeinen Dreieck kann die Fläche A wie folgt berechnet werden:
$$A = \frac{1}{2}\cdot{}b\cdot{}c\cdot{}sin(\epsilon)$$
Also die Hälfte von Seite mal Seite mal Sinus des Zwischenwinkels.
\newpage


\subsection{Steigung vs. Winkel}\index{Steigung}

\bbwCenterGraphic{4.5cm}{tals/trig1/img/starkeSteigung.jpg}

Steigungen werden üblicherweise in \% angegeben. Das obige
Verkehrsschild «starke Steigung» gibt 10\% an. Das heißt:

\TNT{3.6}{Auf einen horizontalen Meter, steigt die Straße um 10cm (=
  10\%).
\vspace{3cm}}

Um den Winkel zu berechnen, kann der Tangens verwendet werden:
$$\frac{0.10m}{1.00m} = \tan(\sigma)$$
In obigem Beispiel ist der Steigungswinkel also durch den $\arctan()$
berechenbar:

$$\sigma = \arctan\left(\frac{0.1m}{1.0m}\right) = \arctan(0.1) \approx 5.71\degre$$
\newpage
