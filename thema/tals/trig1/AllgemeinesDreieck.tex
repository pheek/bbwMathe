%%
%% 2019 07 04 Ph. G. Freimann
%%

\newpage
\section{Das allgemeine Dreieck}\index{Dreieck!allgemeines}

\subsection{Fläche im allgemeinen Dreieck}\index{Fl\"ache!Dreieck}
\TadBMTG{115}{7.4}
%%\TALS{(S. 96 Kap. 2.1.3 \cite{frommenwiler18geom})}

\bbwCenterGraphic{9cm}{tals/trig1/img/dreieck_sws.png}

\begin{gesetz}{Fläche allgemeines Dreieck}{}
  
Im allgemeinen Dreieck kann die Fläche $A$ wie folgt berechnet werden: 
$$A = \frac{1}{2}\cdot{}b\cdot{}c\cdot{}sin(\varepsilon)$$

«Die Hälfte von Seite mal Seite mal Sinus des Zwischenwinkels»

\end{gesetz}

\TNT{4}{Herleitung:
$A=\frac12\cdot{} c\cdot{}h_c$. Doch $h_c$ kann mit dem Sinus berechnet werden: $$h_c  : b = \sin(\varepsilon)$$ und somit ist $A=\frac12\cdot{}c\cdot{} (b\cdot{}\sin(\varepsilon)$.)
}%% END TNT

\subsection*{Aufgaben}
\AadBMTG{123}{35. und 37.}
\newpage

