%% Einbinden in Trigo: Rechtwinkliges Dreieck, aber auch
%% in Planimetrie: Satz des Pythagoras
%%
%% Hier:
%%   * Allgemeine Form
%%   * Höhensatz
%%   Nicht dabei:
%5     -Sinus/Cosinus/Tangens (dies ist nur in Trigo)
%5     -Höhensatz (der kommt nur bei der Planimetrie)



%% Load this only once (the first occurence)!
%% see here: https://tex.stackexchange.com/questions/195157/is-there-any-analog-to-pragma-once-in-latex
\ifcsname XX_Pythagoras.tex\endcsname
  \expandafter\endinput
\fi
\expandafter\gdef\csname XX_Pythagoras.tex\endcsname{loaded}



%%%%%%%%%%%%%%%%%%%%%%%%%%%%%%%%%%%%%%%%%%%%%%%%%%
\newpage

\subsection{Rechtwinkliges Dreieck}

\definecolor{qqwuqq}{rgb}{0,0.39,0}
\definecolor{xdxdff}{rgb}{0.49,0.49,1}
\definecolor{qqqqff}{rgb}{0,0,1}
\begin{tikzpicture}[line cap=round,line join=round,>=triangle 45,x=1.0cm,y=1.0cm]
\clip(-0.54,0.17) rectangle (5.83,4.51);
\draw [shift={(3.35,3.58)},color=qqwuqq,fill=qqwuqq,fill opacity=0.1] (0,0) -- (-154.75:0.95) arc (-154.75:-64.75:0.95) -- cycle;
\draw (0,2)-- (4.56,1.02);
\draw (0,2)-- (3.35,3.58);
\draw (3.35,3.58)-- (4.56,1.02);
\fill[color=qqwuqq,fill=qqwuqq,fill opacity=0.1] (3.16,3.05) circle (0.03);
\begin{scriptsize}
\fill [color=qqqqff] (0,2) circle (1.5pt);
\draw[color=qqqqff] (0.25,2.42) node {$A$};
\fill [color=qqqqff] (4.56,1.02) circle (1.5pt);
\draw[color=qqqqff] (4.81,1.44) node {$B$};
\fill [color=xdxdff] (3.35,3.58) circle (1.5pt);
\draw[color=xdxdff] (3.61,4.01) node {$C$};
\draw[color=qqwuqq] (2.37,2.49) node {$90\textrm{\degre}$};
\end{scriptsize}
\end{tikzpicture}

\begin{gesetz}{Rechtwinkliges Dreieck}{}
Im \textbf{rechtwinkligen} Dreieck misst der Winkel gegenüber der
längsten Seite 90\degre.
\end{gesetz}
\newpage



\subsection{Satz des Pythagoras (Repetition)}\index{Pythagoras!Satz des|textbf}\index{Satz des Pythagoras|textbf}

\fbox{\parbox{\textwidth}{
$$a^2 + b^2 = c^2$$
}}

\TNTeop{Platz für graphischen Beweis\vspace{3cm}}



%%%%%%%%%%%%%%%%%%%%%%%%%%%%%%%%%%%%%%%%%%%%55
\subsection{Höhensatz}\index{Höhensatz}

Im rechtwinkligen Dreieck gilt:
$$h^2=p\cdot{}q$$

Beweise:

\TNT{6.4}{Beweis mit Pythagoras: $h^2 = a^2 - p^2$ und $h^2 = b^2 -
  q^2$.\\
Somit gilt
$2h^2 = a^2 + b^2 - p^2 - q^2 = c^2 - p^2 - q^2 = (p+q)^2 -p^2 - q^2 =
  p^2 +2pq + q^2 - p^2 - q^2 = 2pq$\\
Daraus folgt:$$h^2=pq$$ Nun brauchen wir nur noch auf beiden Seiten die
Wurzel zu ziehen.

Beweis mit Ähnlichkeit: $p:h = h:q$, somit folgt der Satz
automatisch.}

\subsection*{Aufgaben}
\AadBMTG{37}{8., 9., 15., }

\newpage

\subsection{Spezielle Dreiecke}\index{Dreiecke!spezielle}

\subsubsection*{Lernziele}
\begin{itemize}
\item 45\degre: Das halbe Quadrat
\item 30\degre/60\degre: Das halbe gleichseitige Dreieck
\end{itemize}

\TadBMTG{33}{2.4.1}


\subsection*{Aufgaben}
\AadBMTG{37}{7., 11., 12., 13., 19., 20., 21., 24., 26., 27., 28.,
29., 30., 33., 34., 40. und 41. }
\newpage
