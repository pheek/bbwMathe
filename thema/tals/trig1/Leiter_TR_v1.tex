%%
%% Leiter: Ein Beispiel aus der Praxis:
%%

\begin{frage}[2]
  Eine Leiter ist an einer Wand angelehnt (aufgestellt).
  Die Leiter ist in einem Winkel von $72.3^{\circ}$ platziert.
  Sie ist am Boden 1.24m von der Wand entfernt aufgestellt.
  Berechnen Sie die Länge der Leiter ($l$), und berechnen Sie weiter die
  Höhe ($h$) an der Wand, in welcher die Leiter auf die Wand trifft:
\begin{center}
\raisebox{-1cm}{\includegraphics[width=4.5cm]{P_TRIG/img/leiter/Leiter.png}}
\end{center}

Geben Sie alle Resultate auf exakt \textbf{drei} signifikante Stellen
an.

$$l \approx \LoesungsRaum{4.08m}$$

$$h \approx \LoesungsRaum{3.89m}$$

\platzFuerBerechnungen{10}
\end{frage}
  
