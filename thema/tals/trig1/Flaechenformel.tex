%%
%% 2019 07 04 Ph. G. Freimann
%%

\newpage
\section{Flächenformel im allgemeinen
Dreieck}\index{Dreieck!Flächenformel
allgemeines}\index{Flächenformel!allgemeines Dreieck}

%%\subsection{Fläche im allgemeinen Dreieck}\index{Fl\"ache!Dreieck}
\TadBMTG{115}{7.4}
%%\TALS{(S. 96 Kap. 2.1.3 \cite{frommenwiler18geom})}

\bbwCenterGraphic{9cm}{tals/trig1/img/FlaechenFormel.png}

\TNT{4}{Herleitung:
$A=\frac12\cdot{} q\cdot{}h_q$.

Das $h_q$ kann mit dem Sinus
berechnet werden:

$$h_q  : p = \sin(\varepsilon) \Longrightarrow{}    h_q=p\cdot{}\sin(\varepsilon)$$

Somit ist $A=\frac12\cdot{}q\cdot{} (p\cdot{}\sin(\varepsilon))$.
}%% END TNT


\begin{gesetz}{Fläche allgemeines Dreieck}{}
  
Im allgemeinen Dreieck kann die Fläche $A$ wie folgt berechnet werden: 
$$A = \LoesungsRaumLen{50mm}{\frac{1}{2}\cdot{}p\cdot{}q\cdot{}\sin(\varepsilon)}$$

«Die Hälfte von Seite mal Seite mal Sinus des Zwischenwinkels»

\end{gesetz}

Beispiel

\includegraphics[width=50mm]{tals/trig1/img/Flaechenformel.png}

\TNTeop{$A = \frac{1}{2} \cdot{} 7 \cdot{}
5 \cdot{} \sin(48\degre) \approx 13 \text{ cm}^2$}
%%\newpageA
\newpage

\subsection*{Aufgaben}
\AadBMTG{123}{35. und 37.}
\newpage

