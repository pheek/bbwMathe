%% 2020 12 25 Ph. G. Freimann
%% Tals SPF Gleichungen III 1/2: Wurzelgleichungen durch Quadrieren
%%

\subsection{Typ: Produkt = Null}

%%%%%%%%%%%%%%%%%%%%%%%%%%%%%%%%%%%%%%%%%%%%%%%%%%%%%%%%%%%%%%%%%%%%%%%%%%%%%%%%%

%%\TadBMTA{}{}
%%\TALS{(\cite{frommenwiler17alg} S.115 (Kap. 2.4.3))}
%%\GESO{(\cite{marthaler21alg}       S.??? (Kap. ???))}

\subsubsection{Einstiegsbeispiel}

Finden Sie die Lösungsmenge folgender Gleichung:

$$x^4 + 7x^2 - 44 = 0$$

\TNT{6}{
  1. Faktorzerlegung
  $$(x^2 + 11)\cdot{}(x^2-4) = 0$$
  ... weiter ...
  $$(x^2 + 11)\cdot{}(x+2)\cdot{}(x-2) = 0$$

    Nun reicht es, wenn einer der Faktoren = 0 ist:

    $$\lx=\{-2; 2\}$$
}

\newpage
\subsection*{Aufgaben}
\aufgabenFarbe{Finden Sie die Lösungsmenge $\lx$ für die folgende
  Gleichung: $$(x^2+x-2)(x+3)=0$$}%% END Aufgabenfrabe
\TNT{2.4}{Faktorisiert: $$(x+2)(x-1)(x+3)=0$$
  $$\lx=\{-3, -2, 1\}$$}%% END TNT

\aufgabenFarbe{Finden Sie von Hand die Lösungsmenge $\lx$ für die folgende
  Gleichung, wenn $\sin$ im Bogenmaß gerechnet wird:
  $$ x^2\cdot{} \sin(x) = 4\cdot{} \sin(x)  $$}%% END Aufgabenfrabe}
  
\TNT{4}{$$(x^2-4)\cdot{}\sin(x) = 0$$
  $$\sin(x) \cdot{} (x-2) \cdot{} (x+2) = 0$$
  $$\lx = \{-2;2\} \cup \{n\cdot{}\pi | n\in \mathbb{N}\}$$}
%%\TALSAadBMTA{115}{351. d), 352. f) [Tipp: Substitution] g), 353. a) b) c) h)}
%%\GESOAadBMTA{???}{???}

