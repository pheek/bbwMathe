%%
%% 2020 12 25 Ph. G. Freimann
%% Polynomgleichungen höheren Grades (TALS SPF)
%%

\section{Polynom(un)gleichungen}\index{Polynomgleichungen}


%%%%%%%%%%%%%%%%%%%%%%%%%%%%%%%%%%%%%%%%%%%%%%%%%%%%%%%%%%%%%%%%%%%%%%%%%%%%%%%%%
\subsection*{Lernziele}

\begin{itemize}
\item Produktform (Typ: Produkt gleich «Null»)
\item Taschenrechner
\end{itemize}

\newpage


%%
%% 2019 07 04 Ph. G. Freimann
%%

\subsection{Polynome}\index{Polynom}
\sectuntertitel{Poly = viel; Nom = Name}

%%\TALSTadBFWA{12}{1.1.4}
\TadBMTA{18}{1.4}
%%%%%%%%%%%%%%%%%%%%%%%%%%%%%%%%%%%%%%%%%%%%%%%%%%%%%%%%%%%%%%%%%%%%%%%%%%%%%%%%%
\subsubsection*{Lernziele}

\begin{itemize}
\item Grad, Grundform und Koeffizienten
\end{itemize}


Eine spezielle Form von Termen sind die
\textbf{Polynome}.\footnote{Siehe dazu auch im Anhang das
  Summenzeichen auf Seite \pageref{Summenzeichen}.}


\begin{beispiel}{Polynom}{}
  
  Binom: $T(x) = \LoesungsRaumLang{5x^4 + 3x}$
  \leserluft{}
  
  Trinom: $T(z) = \LoesungsRaumLang{7bz^6 + \sqrt{2}z^3 + 44}$

  (Dies
  ist auch ein Binom in $b$: $= 7z^6\cdot{} b^1 + (\sqrt{2}z^3 + 44) \cdot{} b^0$)
  \end{beispiel}


\begin{definition}{Polynom}{definition_polynom}
  Unter einem \textbf{Polynom} verstehen wir einen Term in einer Variable
  $T(x)$ in der Gestalt:

  \begin{tabular}{rrlllll}\index{$\sum{}$ Summe} 
   $T({\color{red}x}) = \sum\limits_{{\color{blue}i}=0}^{n}{{\color{ForestGreen}a}_{\color{blue}i}{\color{red}x}^{\color{blue}i}}$ &=& ${\color{ForestGreen}a}_{\color{blue}0}$ &+ ${\color{ForestGreen}a}_{\color{blue}1}{\color{red}x}$ &+ ${\color{ForestGreen}a}_{\color{blue}2}{\color{red}x}^{\color{blue}2}$ &+ $...$ &+ ${\color{ForestGreen}a}_{\color{blue}n}{\color{red}x}^{\color{blue}n}$\\
    &(=& ${\color{ForestGreen}a}_{\color{blue}0}{\color{red}x}^{\color{blue}0}$ &+ ${\color{ForestGreen}a}_{\color{blue}1}{\color{red}x}^{\color{blue}1}$ &+ ${\color{ForestGreen}a}_{\color{blue}2}{\color{red}x}^{\color{blue}2}$ &+ $...$ &+ ${\color{ForestGreen}a}_{\color{blue}n}{\color{red}x}^{\color{blue}n}$)
\end{tabular}

    \end{definition}
Dabei bezeichnet n ($\in \mathbb{N}$) den \textbf{Grad} des Polynoms,
während die $a_i$ Koeffizienten in $\mathbb{R}$ sind.

\begin{bemerkung}{}{}Bei Polynomen kommt die Variable weder im Nenner, noch im
  Exponenten, noch unter einer Wurzel vor.\end{bemerkung}

\subsection{Nomenklatur}
Polynome ersten Grades (\zB $ 3x-1$) nennen wir \textbf{lineare}
Polynome\index{Polynom!lineares}.

\GESO{Weitere Definitionen zu \textit{quadratischen} und
  \textit{kubischen} Polynomen finden wir im Buch \cite{marthaler21alg}
  auf Seite 19.}

\textbf{Achtung} Kein Polynom ist $x^2 - \sqrt{x}$, denn $x$ kommt in
der Wurzel vor. Ebensowenig ist $x - \frac{5}{x^2}$ ein Polynom, denn
es lässt sich nicht in die Grundform verwandeln.
\newpage


\subsection{Beispiele}

%%\renewcommand{\arraystretch}{2}
\begin{bbwFillInTabular}{|r|c|l|}\hline
\multirow{2}{*}{$3yx^2 + 3x$} & Binom  in $x$ & $(3y)\cdot{}x^2 + 3\cdot{}x^1$\\\cline{2-3}
                              & Binom  in $y$ & $(3x^2)\cdot{}y^1 + (3x)\cdot{}y^0$\\\hline
$3m$                    & \LoesungsRaum{Monom in $m$} & \LoesungsRaumLang{$3\cdot{} m^1$}\phantom{xxxxxxxxxxxxxxxx}\\\hline
$4m^2 + 3m$             & \LoesungsRaum{Binom  in $m$} & \LoesungsRaumLang{$4\cdot{}m^2 + 3\cdot{} m^1$}\\\hline
$5t^2 + 2$              & \LoesungsRaum{Binom  in $t$} & \LoesungsRaumLang{$5\cdot{}t^2 + 2\cdot{}t^0$}\\\hline
\multirow{3}{*}{$5a^2 + 4am +2rm + m^3$} & \LoesungsRaum{Trinom in $a$} & \LoesungsRaumLang{$5\cdot{}a^2 + 4m\cdot{}a^1 + (2rm+m^3)\cdot{}a^0$}\\\cline{2-3}
                        & \LoesungsRaum{Trinom in $m$} & \LoesungsRaumLang{$1\cdot{}m^3 + (4a+2r)\cdot{}m^1 + 5a^2\cdot{}m^0$}\\\cline{2-3}
                        & \LoesungsRaum{Binom  in $r$} & \LoesungsRaumLang{$2m\cdot{}r^1 + (5a^2+4am+m^3)\cdot{}r^0$}\\\hline
\end{bbwFillInTabular} 

\subsection*{Aufgaben}
%%\TALSAadBMTA{13}{10 und 11}
\AadBMTA{25}{30. a) c) e) f)}
\newpage

%% 2020 12 25 Ph. G. Freimann
%% Tals SPF Gleichungen III 1/2: Wurzelgleichungen durch Quadrieren
%%

\subsubsection{Typ: Produkt = Null}

%%%%%%%%%%%%%%%%%%%%%%%%%%%%%%%%%%%%%%%%%%%%%%%%%%%%%%%%%%%%%%%%%%%%%%%%%%%%%%%%%

\TALS{(\cite{frommenwiler17alg} S.115 (Kap. 2.4.3))}
%%\GESO{(\cite{marthaler21alg}       S.??? (Kap. ???))}

\subsection{Einstiegsbeispiel}

Finden Sie die Lösungsmenge folgender Gleichnug:

$$x^4 + 7x^2 - 44 = 0$$

\TNT{6}{
  1. Faktorzerlegung
  $$(x^2 + 11)\cdot{}(x^2-4) = 0$$
  ... weiter ...
  $$(x^2 + 11)\cdot{}(x+2)\cdot{}(x-2) = 0$$

    Nun reicht es, wenn einer der Faktoren = 0 ist:

    $$\lx=\{-2; 2\}$$
  }

\subsection*{Aufgaben}
\aufgabenFarbe{Finden Sie die Lösungsmenge $\lx$ für die folgende
  Gleichung: $$(x^2+x-2)(x+3)=0$$}%% END Aufgabenfrabe
\TNT{2.4}{Faktorisiert: $$(x+2)(x-1)(x+3)=0$$
  $$\lx=\{-3, -2, 1\}$$}%% END TNT

%%\TALSAadBMTA{115}{351. d), 352. f) [Tipp: Substitution] g), 353. a) b) c) h)}
%%\GESOAadBMTA{???}{???}

\newpage

\subsection{Taschenrechner}
Beispiel

$$x^3 + 70 \ge{} 4x^2 + 31$$

Verschiedene Lösungsansätze



a) Taschenrechner: \LoesungsRaumLang{\texttt{solve($x^3+70 \ge 4x^2 +
    31$, x)}}

\leserluft{}

b) Graphisch auch mit TR (ablesen aus Graph):

\TNT{2}{\texttt{g(x) := $x^3 + 70$} und \texttt{h(x) := $4x^2 + 31$}
  
Lies ab, wo nun $g(x) \ge h(x)$\vspace{6mm}
}%% ENd TNT

  
c) Die harte Tour\footnote{Solche und ähnliche Verfahren wendet auch der
  Taschenrechner an.}:

\TNT{14}{
  \begin{enumerate}
  \item Wie bei einer quadratischen Gleichung alles auf eine Seite
    nehmen: $$x^3 - 4x^2 -31x + 70 \ge{} 0$$

  \item Annähern, pröbeln  ($x=0, x=1, x=2$)
    Hier finden wir, dass $x=2$ die Ungleichung (bzw. die Gleichung) löst.

  \item Ziel: Gleichung vom Typ «Produkt gleich Null»:
    $$(x-2)\cdot(x + ???)\cdot(x + ???) = 0$$

  \item (optional Vorzeigen) Polynomdivison:
    $$(x^3 - 4x^2 - 31x + 70) = (x-2) \cdot (???)$$
    $$(x^3 - 4x^2 - 31x + 70) : (x-2) = (x^2 + 2x - 35)$$

  \item Lösung: $$x^3 - 4x^2 -31x + 70 = (x-2)\cdot(x-5)\cdot(x+7)$$
    Somit gilt $$-7, 2 \textrm{ und } 5 \in \lx $$
\end{enumerate}
}%% END TNT
\newpage

\subsubsection{halbgraphische Methode}
... weiter im Beispiel ...
\TNT{14}{
Halbgraphische Methode mit $(x-2)$, $(x-5)$ und $(x+7)$ zeichnen und
interpretieren:
\bbwCenterGraphic{15cm}{tals/gl3_1/img/HalbgraphischeMethode.png}
\vspace{3cm} 
}%% END TNT



%%Erstellen Sie zu einer der folgenden Aufgaben eine Musterlösung für OLAT!

\subsection*{Aufgaben}
%%\TALSAadBMTA{157ff}{555. (TR), 559., 561., 562., 564., 568., 572., 574.,
%%  575. und Seite 203: 761., 762 und 763. \TRAINER{764. gehört zu den
%%    Ungleichungen (anderes Skript).}}
%%\TALSAadBMTA{203}{761-764}
\TALSAadBMTA{131}{22. b), 23. b)}
\TALSAadBMTA{184}{32. e), 33. b)}

