\newpage

\begin{rezept}{... durch direktes Logarithmieren.}{}
$$8^{x-1}=5\cdot{}7^{x+2}$$
\end{rezept}

  \TNT{16}{
  Erst mal auf beiden Seiten logarithmieren, mit einem Logarithmus zu beliebiger Basis:
  $$\log(8^{x-1}) = \log(5\cdot{}7^{x+2})$$
  Log Gesetz (Produktregel):
  $$\log(8^{x-1}) =\log(5)+ \log(7^{x+2})$$
  Log Gesetz (Potenzregel)\TRAINER{\footnote{Ab hier ist es eine
      lineare Gleichung}}:
  $$(x-1)\cdot{}\log(8) = \log(5) + (x+2)\cdot{}\log(7)$$
  ausmultiplizieren
  $$\log(8)\cdot{}x-\log(8) = \log(5) + \log(7)\cdot{}x+2\cdot{}\log(7)$$
  und $x$ auf eine Seite bringen
  $$\log(8)\cdot{}x-\log(7)\cdot{}x =\log(5)+ 2\cdot{}\log(7)+\log(8)$$
  $x$ ausklammern
  $$x\cdot{}(\log(8)-\log(7)) = \log(5) + 2\cdot{}\log(7)+\log(8)$$
  und durch die Klammer $(\log(8)-\log(7))$ teilen
  $$x = \frac{\log(5) + 2\cdot{}\log(7)+\log(8)}{\log(8)-\log(7)} \approx 56.77092$$
}

\TALS{\olatLinkArbeitsblatt{Exponentialgleichungen [GL\_Ex]}{https://olat.bms-w.ch/auth/RepositoryEntry/6029786/CourseNode/106029166654550}{3.}}

\newpage



\subsubsection{$\ln$: Logarithmus Naturalis}

Geben Sie mit Hilfe des natürlichen Logarithmus $\ln()$ an:

$$7^x = 5$$
\TNT{16}{
$$\ln(7^x) = \ln(5)$$ 
$$x\cdot{}\ln(7) = \ln(5)$$ 
$$x = \frac{\ln(5)}{\ln(7)}$$ 
}%% end TNT

\TALS{\olatLinkArbeitsblatt{Exponentialgleichungen [GL\_Ex]}{https://olat.bms-w.ch/auth/RepositoryEntry/6029786/CourseNode/106029166654550}{4.}}

\newpage



\subsubsection{Substitution}

Finden Sie die Lösungsmenge $\lx$:

$$2\cdot{} 5^x - 5^{2x} = 1$$

\TNT{10}{
  Substituiere $u := 5^x$

  $$2u -u^2 = 1$$
  Quadratische Gleichung:
  $$0 = u^2 -2u + 1$$
  $$0 = (u-1)(u-1)$$
  daraus folgt:
  $$u= 1$$
  Rücksubstitution:
  $$1=5^x$$
  Und daraus:
  $$x=0 ; \lx=\{0\}$$

}%% end TNT

\TALS{\olatLinkArbeitsblatt{Exponentialgleichungen [GL\_Ex]}{https://olat.bms-w.ch/auth/RepositoryEntry/6029786/CourseNode/106029166654550}{5.}}

