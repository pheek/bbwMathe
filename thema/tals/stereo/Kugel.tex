\section{Kugel}\index{Kugel}

\theorieTALSGeom{166}{3.3.3}

\subsection*{Lernziele}
\begin{itemize}
\item Volumen der Kugel
\item Oberfläche der Kugel
\item Einbeschriebene und umschriebene Körper
\item Kugelsektor (optional)
\item Kuglesegment (optional)
\item Kugelhaube (Klaotte)
\item Kugelschicht
\end{itemize}
\newpage

\subsection{Kugelkörper}

\bbwCenterGraphic{5cm}{tals/stereo/img/Sphaere.png}

\begin{definition}{Kugel}{}\index{Kugel}\index{Sphäre}\index{Vollkugel}
  Die Menge aller Punkte, welche von einem Punkt $M$ (dem \textbf{Mittelpunkt}) den selben Abstand ($r$ = Radius) haben, nennen wir die \textbf{Kugelfläche}, \textbf{Kugeloberfläche} oder \textbf{Sphäre}.
  Ist das Innere der Kugel mitgemeint, so sprechen wir vom \textbf{Kugelkörper} oder von der \textbf{Vollkugel}.
\end{definition}
\newpage

\subsection{Kugelvolumen}\index{Volumen!Kugel}\index{Kugel!Volumen}

\TRAINER{Video Volumen/Oberflächenformel:\\\texttt{https://www.youtube.com/watch?v=jQoHp\_P8Y0A}}

\begin{gesetz}{Volumen}{}

  $V$ = Volumen der Kugel

  $r$ = Radius der Kugel

  $$V = \frac43\cdot{}\pi\cdot{}r^3$$
  \end{gesetz}

Herleitung:\\
\TNT{16}{
  Zeiche eine Halbkugel. (Lasse rechts Platz.)

  Wähle beliebige Höhe $h$ in beiden Figuren.

  Damit entsteht eine Kreisfläche mit Radius $a$, wobei nach Pythagoras gilt:
  $$a^2 + h^2 = r^2$$
  und somit
  $$a^2 = r^2 - h^2$$
  Der \textbf{Flächeninhalt} dieses Kreises (auf Höhe $h$ ist $\pi a^2$ = $\pi (r^2 - h^2)$ und somit $\pi r^2 = \pi h^2$.

  Somit ist der Flächeninhalt aber auch ein Kreisring mit Außenradius $r$ und Innenhardius $h$.

  Wenn wir das in jeder Höhe machen, entsteht ein Vergleichskörper, der aus einem Kreiszylinder mit ausgeschnittenem Kreiskegel besteht.

  Volumen Halbkugel = (dank Cavalieri) Volumen Vergeichskörper = Volumen Kreszylinder - Volumen Kreiskegel = $r^2\pi r - \frac13 r^2\pi r = \frac23 r^3 \pi $

  Somit ist das Volumen der ganzen Kugel = 2 $\cdot$ Volumen Halbkugel = $V = \frac43\pi r^3$.

  
}%% END TNT
\newpage


\subsection{Kugeloberfläche}
\begin{gesetz}{Oberfläche}{}

  $S$ = Oberfläche der Sphäre (Surface)

  $r$ = Radius der Kugel

  $$S = 4\cdot{}\pi\cdot{}r^2$$
  \end{gesetz}

\subsection*{Aufgaben}
\TALSGeomAadB{167}{177., 179., 180. (Pyramide in Kugel), 181. (Zylinder in Kugel), 182. (Kugel in Kegel) und 200. (Kugelschale)}
\newpage
