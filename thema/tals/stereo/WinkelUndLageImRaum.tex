\section{Länge, Flächen, Winkel}\index{Winkel!im Raum}\index{Flächel!im Raum}\index{Länge!im Raum}
%%\sectuntertitel{Flachwitz?}

\TRAINER{TODO: Hier vielleicht besser drei Kapitel erstellen a) Nur Längen und Flächen von Schntittebenen, b) Winkel, c) Ein- und Umschriebene Körper}

\subsection{Lage Im Raum}
\TALSTadBMTG{178}{11.2}
%%\TALSTadBFWG{134}{3.1.1\TRAINER{auch S. 135 behandeln}}

\subsubsection*{Lernziele}
\begin{itemize}
\item zusammenfallend, parallel, senkrecht, schneidend, windschief
\item Schnittpunkt, Durchstoßpunkt, Schnittgerade 
\end{itemize}

%%\TALSAadBFWG{136}{6. b) c) und g), 9.}
\newpage


\subsection{Winkel im Raum}


\subsection*{Lernziele}
\begin{itemize}
\item Winkel in Körpern berechnen
\end{itemize}


Einstieg und Repetition: Berechnen Sie den Winkel $\alpha$ im folgenden Dreieck.

\bbwCenterGraphic{5cm}{tals/stereo/img/einstiegTrigo.png}

$$\alpha= \LoesungsRaumLang{ \arctan\left(\frac{\sqrt2}2\right) \approx 35.26\degre  }$$

\TALSTadBMTG{179}{11.2.3}
%%\TALSTadBFWG{138}{3.1.2}
\newpage

\subsubsection{Referenzaufgabe: Winkel im Quader}

Berechnen Sie im folgenden Quader den Winkel $\beta = \angle ABC$:
\bbwCenterGraphic{10cm}{tals/stereo/img/ReferenzaufgabenWinkelImQuader.png}
\TNT{2.0}{Idee: Den «richtigen» Schnitt finden und danach mit trigonometrischen
  Methoden die Aufgabe lösen.}%% END TNT

Lösungsweg:
\TNT{4.4}{\begin{itemize}
  \item Schnitt des Quaders durch die Ebene, welche durch das Dreieck
    $\triangle ABC$ bestimmt ist.
  \item Berechne (mit Pythagoras) die Strecke $\overline{AB}$:\\
    
    $|\overline{AB}| = \sqrt{3.5^2 + 2.5^2}$.
  \item Im $\triangle ABC$ gilt $\tan(\beta) = \frac{5.0}{|\overline{AB}|}$. Und daraus ergibt sich
    $$\beta = \arctan\left(\frac{5.0}{\sqrt{3.5^2 + 2.5^2}}\right) \approx 49.30\degre$$
\end{itemize}}

\subsection*{Aufgaben}
%%\TALSAadBFWG{139}{21. a) b) und c) und 22.}

\TALSAadBMTG{182}{2. c)}
\newpage
\subsection{Lage von Punktmengen}
Gegenseitige Lage von Punkten, Geraden und Ebenen.

\TALSTadBMTG{178}{11.2.2} \TRAINER{Selbststudium der Begriffe: Windschief,
  parallel, schneidend, Durchstoßpunkt}

\subsubsection{Winkel zwischen Gerade und Ebene}
\TALSTadBMTG{179}{11.2.3} 



\subsection*{Aufgaben}
\TALSAadBMTG{182}{2. a) b), 3. a) b) }
\TALSAadBMTG{233ff}{14. und \textit{challenge:} 13., 15.}

\olatLinkArbeitsblatt{Schnittebenen}{https://olat.bms-w.ch/auth/RepositoryEntry/6029786/CourseNode/109275405553951}{1.1  - 1.7}
\newpage
