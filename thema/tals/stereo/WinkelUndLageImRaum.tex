\section{Winkel im Raum}\index{Winkel im Raum}
\sectuntertitel{}


\subsection{Lage Im Raum}
\TALSTadBFWG{134}{3.1.1\TRAINER{auch S. 135 behandeln}}

\subsubsection*{Lernziele}
\begin{itemize}
\item windschief
\item Durchstoß
\end{itemize}

\TALSAadBFWG{136}{6. b) c) und g), 9.}
\newpage


\subsection{Winkel im Raum}


\subsection*{Lernziele}
\begin{itemize}
\item Winkel in Körpern berechnen
\end{itemize}


Einstieg und Repetition: Berechnen Sie den Winkel $\alpha$ im folgenden Dreieck.

\bbwCenterGraphic{5cm}{tals/stereo/img/einstiegTrigo.png}

$$\alpha= \LoesungsRaumLang{ \arctan(\frac{\sqrt2}2) \approx 35.26\degre  }$$

\TALSTadBFWG{138}{3.1.2}
\newpage



\subsubsection{Referenzaufgabe: Winkel im Quader}

Berechnen Sie im folgenden Quader den Winkel $\beta = \angle ABC$:
\bbwCenterGraphic{10cm}{tals/stereo/img/ReferenzaufgabenWinkelImQuader.png}
\TNT{2.0}{Idee: Den «richtigen» Schnitt finden und danach mit trigonometrischen
  Methoden die Aufgabe lösen.}%% END TNT

Lösungsweg:
\TNT{4.4}{\begin{itemize}
  \item Schnitt des Quaders durch die Ebene, welche durch das Dreieck
    $\triangle ABC$ bestimmt ist.
  \item Berechne (mit Pythagoras) die Strecke $\overline{AB}$:\\
    
    $|\overline{AB}| = \sqrt{3.5^2 + 2.5^2}$.
  \item Im $\triangle ABC$ gilt $\tan(\beta) = \frac{5.0}{|\overline{AB}|}$. Und daraus ergibt sich
    $$\beta = \arctan(\frac{5.0}{\sqrt{3.5^2 + 2.5^2}}) \approx 49.30\degre$$
\end{itemize}}

\subsection*{Aufgaben}
\TALSAadBFWG{139}{21. a) b) und c) und 22.}
\newpage

