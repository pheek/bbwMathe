\section{Schiefe und stumpfe Körper}\index{schief!Körper}\index{stumpf!Körper}

Pyramide : \TadBMTG{215}{14.1 Pyramidenstumpf}

Kreiskegel: \TadBMTG{217}{14.2 Kreiskegelstumpf}


\bbwCenterGraphic{8cm}{tals/stereo/img/SchiefUndStumpf.png}

\begin{bemerkung}{Deckfläche}{}
Die Deckfläche beim stumpfen allgemeinen Kegel ist parallel zur
Grundfläche und ist dieser (in der Form) ähnlich.
\end{bemerkung}

\begin{bemerkung}{gerade}{}
  Der «gerade» allgemeine Kegel kann nur als \textbf{gerader Kegel}
  bezeichnet werden, wenn die Grundfläche eine eindeutige Mitte
  aufweist (Parallelogramm, Ellipse, ...). Die Kegelspitze steht dann senkrecht über dieser Mitte.
\end{bemerkung}
  \newpage


\subsection{Kreiskegelstumpf / Pyramidenstumpf}
\TRAINER{Optional?}


\bbwCenterGraphic{6.5cm}{tals/stereo/img/VolumenKegelstumpf.png}


\begin{gesetz}{Volumen Pyramidenstumpf}{}

  $G$ = Grundfläche; $D$ = Deckfläche

  $h$ = Höhe des Stumpfs
  
  $$ V = \frac13 \cdot{} \left(G + \sqrt{GD} + D \right) \cdot{} h $$

\end{gesetz}
\TRAINER{Bemerkung: Mit dieser Formel ist auch \begin{itemize}\item $V=Gh$ für $G=D$
  (Zylinder/Prismen) und \item 
  $V=\frac13Gh$ für $D=0$ (Koni) \end{itemize} erschlagen.}
\newpage

... Beweis der Volumenformel für stumpfe Körper (optional) ...
\TNTeop{
Zur Volumenberechnung betrachten wir den «Ersatzkörper», der noch nicht beschnitten ist.


Gegeben: $G$ = Grundfläche; $D$ = Deckfläche und $h$ = Höhe (des Stumpfs)

Hilfsgröße $H$ = Höhe des Ersatzkörpers

Die Wurzeln der Flächen verhalten sich wie die Strecken in ähnlichen Körpern. Daher:
\begin{center}
  \[
\begin{array}{rcl}
  \sqrt{G} : H &=&\sqrt{D} : (H-h)\\
  H\sqrt{D} &=& \sqrt{G}(H-h) = \sqrt{G}H - \sqrt{G} h\\
  -H\sqrt{D} + \sqrt{G}\cdot{}H &=& \sqrt{G}\cdot{}h\\
      H\cdot{}(\sqrt{G} - \sqrt{D}) &=& \sqrt{G}\cdot{}h\\
      H &=& \frac{\sqrt{G}\cdot{}h}{\sqrt{G}-\sqrt{D}}
  \end{array}
\]
\end{center}

Und somit für das Volumen erhalten wir:

\begin{center}
  \[
\begin{array}{rcl}
  3V &=& GH - D\cdot{} (H-h)\\
  &=& GH - DH + Dh = H\cdot{}(G-D) + D\cdot{}h\\
  &=& \frac{\sqrt{G}\cdot{}h}{\sqrt{G}-\sqrt{D}} \cdot{} (G-D) + D\cdot{}h\\
  &=& \frac{(\sqrt{G} + \sqrt{D})\cdot{}\sqrt{G}h}{(\sqrt{G}+\sqrt{D})\cdot{}(\sqrt{G} - \sqrt{D})}\cdot{}(G-D) + D\cdot{}h\\
  &=& \frac{(\sqrt{G} + \sqrt{D})\cdot{}\sqrt{G}h}{G-D}\cdot{}(G-D) + D\cdot{}h\\
  &=& (\sqrt{G} + \sqrt{D})\cdot{}\sqrt{G}h + D\cdot{}h\\
  &=& G\cdot{}h + \sqrt{G}\sqrt{D}h + D\cdot{}h
\end{array}
\]
\end{center}
}%% END TNT
\newpage

\subsection*{Aufgaben}
\TALSAadBMTG{221ff}{5. (quadratische Pyramide), 17. a) [schwimmender Kegel]}%%

\TRAINER{ev. noch Aufgabe 16. S. 223 im Geometriebuch (Musterlösung ausstehend) }


\newpage
