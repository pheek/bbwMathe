\section{Allgemeiner Zylinder (Prisma und Kreiszylinder)}\index{Zylinder}\index{Prisma}

\bbwCenterGraphic{5cm}{tals/stereo/img/AllgemeinerZylinder.png}

\begin{gesetz}{Volumen allgemeiner Zylinder (inkl. Prisma)}{}\index{Volumen!Zylinder ink. Prisma}

  \begin{center}\fbox{$V = \LoesungsRaumLang{G\cdot{}h}$}\end{center}
\end{gesetz}

\begin{gesetz}{Oberfläche allgemeiner Zylinder (inkl. Prisma)}{}\index{Oberfläche!Zylinder inkl. Prisma}

  \begin{center}\fbox{$S = \LoesungsRaumLang{G + D + M  = 2\cdot{}G + U\cdot{}h}$}\end{center}
\end{gesetz}

\begin{bemerkung}{Mantel}{}
  
  Für die Mantelfäche beim allgemeinen «geraden» Zylinder gilt:

  $M = \LoesungsRaumLang{h \cdot U}$
\end{bemerkung}


\newpage
\subsection{Quader und Würfel}\label{QuaderUndWuerfel}
Die beiden speziellsten Prismen sind der Würfel und der Quader.

Im Spezialfall «Würfel» gilt:\index{Würfel}
\begin{gesetz}{Würfel}{}\\
  Volumen $V = G\cdot{} h = \LoesungsRaum{a^3}$\\
  Oberfläche $S=2\cdot{}G + M = \LoesungsRaum{6\cdot{}a^2}$\\
  Raumdiagonale $d = \LoesungsRaum{\sqrt{3\cdot{}a^2}} = \LoesungsRaum{a\cdot{}\sqrt{3}}$
\end{gesetz}

\bbwCenterGraphic{12cm}{tals/stereo/img/WuerfelUndQuader.png}

Der beliebige Quader hat die Seiten $a$, $b$ und $c$:
\begin{gesetz}{Quader}{}\index{Quader}\\
  Volumen $V = G\cdot{} h = \LoesungsRaum{(a\cdot{}b)\cdot{}h} = \LoesungsRaum{a\cdot{}b\cdot{}c}$\\
  Oberfläche $S=2\cdot{}G + M = \LoesungsRaumLang{2\cdot{}ab + ac + bc + ac + bc} = \LoesungsRaum{2\cdot{}(ab + ac + bc)}$\\
  Raumdiagonale $d = \LoesungsRaumLang{\sqrt{a^2 + b^2 + c^2}}$
\end{gesetz}

\subsection*{Aufgaben}
%%\TALSAadBFWG{139ff}{21. d) und 24. a) b)}
\TALSAadBMTG{194}{5.}
\newpage


\subsection{Prisma}\index{Prisma}
%%Prisma: \TALSTadBFWA{141}{3.2.1}
Prisma: \TALSTadBMTG{183}{12.1}

\bbwCenterGraphic{18cm}{tals/stereo/img/toblerbms.png}
\begin{center}Bildherkunft offizielle Webseite: \texttt{https://toblerone.fr}\end{center}

\bbwCenterGraphic{5cm}{tals/stereo/img/AllgemeinesPrisma.png}

\begin{bemerkung}{}{}
  Alle Seitenflächen eines geraden Prismas sind Rechtecke mit derselben Höhe $h$.
  \end{bemerkung}

\subsection*{Aufgaben}

%%\TALSAadBMTA{142ff}{32., 33., 57. und 60.}
\TALSAadBMTG{194}{11., 14., 23., 29.}
\newpage


\subsection{Kreiszylinder}\index{Zylinder!Kreiszylinder}\index{Kreiszylinder}

%%Kreiszylinder: \TALSTadBFWA{158}{3.3.1}
Kreiszylinder: \TALSTadBMTG{190}{12.2}

\bbwCenterGraphic{8cm}{tals/stereo/img/KreiszylinderVolumen.png}

\subsubsection{Volumen}

\begin{bemerkung}{Grundfläche}{}
  Die \textbf{Grundfläche} im Kreisyzlinder ist eine Kreisfläche:

  $r$ = Zylinderradius
  
  $$G = \LoesungsRaum{r^2\pi}$$
\end{bemerkung}

\begin{gesetz}{Volumen}{}
  Das \textbf{Volumen} des Zylinders ist gegeben durch

  $r$ = Zylinderradius

  $G$ = Grundfläche (= $r^2\pi$)

  $$V = G\cdot{}h = \LoesungsRaum{r^2\pi\cdot{} h}$$
\end{gesetz}
\newpage

\subsubsection{Oberfläche}
\bbwCenterGraphic{12cm}{tals/stereo/img/Kreiszylinder.png}


\begin{gesetz}{Oberfläche}{}
  Die \textbf{Oberfläche} $S$ ist hier im speziellen:

  $S$ = Oberfläche

  $r$ = Zylinderradius

  $h$ = Zylinderhöhe

  $G$ = Grundfläche $=r^2\pi$ = Deckfläche $D$

  $M$ = Mantelfläche $=2r\pi\cdot{}h$
  
  $$S =\LoesungsRaumLang{G+D+M = 2\cdot{}G + U\cdot{}h = 2\cdot{}r^2\pi + 2r\pi\cdot{}h = 2r\pi\cdot{}(r + h)}$$
\end{gesetz}



\begin{bemerkung}{Mantelfläche}{}
  Die \textbf{Mantelfläche} im Kreisyzlinder kann am einfachsten durch eine «Abwicklung» erklärt werden. Dabei entsteht ein Rechteck:

  $U$ = Umfang der Grundfläche $= 2r\pi$
  
  $$M = U\cdot{}h = \LoesungsRaum{2r\pi\cdot{}h}$$
\end{bemerkung}


\newpage

\subsection*{Aufgaben}
%%\TALSAadBFWG{159}{136}

\TALSAadBMTG{198}{36., 41. a)}
\newpage
