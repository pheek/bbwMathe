%% Fallunterscheidung bei linearen Gleichungen

\subsection{Fallunterscheidung}\index{Fallunterscheidung!lineare Gleichungen}

Gegeben ist die folgende Gleichung in $x$ mit Parametern:

$$a\cdot{}x = b$$

Die generelle Lösung ist:

$x = \frac{b}a$

\TNT{20mm}{
  falls $a\ne 0$  gilt  $\lx = \left\{\frac{b}a\right\}$
}% end TNT

Ist aber $a=0$, so sieht es anders aus:

$0\cdot{} x = b$

\TNT{30mm}{
  $$b=0    \Longrightarrow \lx = \mathbb{R}$$
  $$b\ne 0 \Longrightarrow \lx = \{\}$$
  }%% end TNT


\TALS{\olatLinkArbeitsblatt{Lineare Gleichungen}{https://olat.bms-w.ch/auth/RepositoryEntry/6029786/CourseNode/110662976730861
}{Aufgabe 5.}}

