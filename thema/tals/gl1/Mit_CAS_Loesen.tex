%%
%% 2019 11 14 Ph. G. Freimann
%%

\subsection{Gleichungen mit CAS lösen}\index{CAS@Computer Algebra System}\index{Computer Algebra System@CAS}
Damit weniger Fehler geschehen, aber auch, um schneller an Resultate zu gelangen werden in der Praxis Gleichungen fast ausschließlich mit einem Computer-Algebra-System (CAS) gelöst.
Wir wählen eine einfache Gleichung, damit wir a) nicht viel tippen müssen und b) das Resultat auch sofort überprüfen können. Gegeben ist also die folgende Gleichung:

$$3x+1=2x$$

Typischerweise bieten sich die folgenden drei Lösungswege an\footnote{Die drei Verfahren wurden mit dem Rechner ``TI-nspire II-CX CAS'' getestet.}:
\begin{itemize}
\item solve($3x+1=2x$, $x$)
\item $\text{gl1} := 3x+1=2x$\\
  solve($\text{gl1}$, $x$)
  

\end{itemize}


  
\subsection*{Aufgaben}
Lösen Sie diesmal \textbf{mit dem Taschenrechner}:

\TALS{\olatLinkArbeitsblatt{Lineare
    Gleichungen}{https://olat.bms-w.ch/auth/RepositoryEntry/6029786/CourseNode/110662976730861}{Aufgabe 1. b) c) d) f), 2. a) b)}
  \TRAINER{Solche Aufgaben später direkt als neue Nummer aufs Arbeitsblatt integrieren.}
}



\newpage
