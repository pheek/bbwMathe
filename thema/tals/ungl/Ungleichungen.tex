%% Ungleichungen
%% 2021 - 06 -24 φ@bbw.ch

\section{Ungleichungen}\index{Ungleichungen}\index{Gleichungen!Ungleichungen}

\TadBMTA{112}{7.3}

%%%%%%%%%%%%%%%%%%%%%%%%%%%%%%%%%%%%%%%%%%%%%%%%%%%%%%%%%%%%%%%%%%%%%%%%%%%%%%%%%
\subsection*{Lernziele}

\begin{itemize}
\item Spezielle Äquivalenzumformungen
\item Lösungsmenge in Intervallnotation
\item Graphische Lösungsverfahren
\end{itemize}

\newpage
\subsection{Einstiegsbeispiel}
Finden Sie die Lösungsmenge $\lx$:

$$\frac{-3(2x-7) - x}{5} \ge  \frac{-9-x}{5}$$


\TNTeop{
  $\cdot{}5$
  $$-3(2x-7) -x\ge -9 -x$$

  Erklärung $\cdot5$ (größer/kleiner Zeichen selbst einsetzen lassen):
  
  \begin{tabular}{|c|c|}\hline
    Drei Fälle & $\cdot5$\\
    $ 7 \ge  3$& $ 35 \ge  15$ (ok)\\\hline
    $ 7 \ge -1$& $ 35 \ge  -5$ (ok)\\\hline
    $-2 \ge -7$& $-10 \ge -35$ (ok)\\\hline
    \end{tabular}
  
  Beidseitig $+x$
  
  $$-3(2x-7) \ge -9$$
  $:(-3)$

  $$2x-7 \le 3$$
  
  Erklärung $:(-3)$ (größer/kleiner Zeichen selbst einsetzen lassen):

    \begin{tabular}{|c|c|}\hline
    Drei Fälle & $:(-3)$\\
    $6 > 3$  & $ -2 < -1$ (ok)\\\hline
    $6 > -3$ & $ -2 <  1$ (ok)\\\hline
    $-3 > -6$& $  1 <  2$ (ok)\\\hline
    \end{tabular}

  
  $+7$
  $$2x\le 10$$
    $:2$
    $$x\le 5$$
    Lösungsmenge
  $$\lx = ]-\infty;5]$$
}%% End TNTeop
%%\newpage

  
\begin{gesetz}{}{}
  Äquivalenzumformungen bei Ungleichungen.\index{Äquivalenzumformungen!Ungleichungen}

  \begin{itemize}
	\item Termumformungen (unabhängig links und rechts des Gleichheitszeichens, sofern der Definitionsbereich des Terms nicht verändert wird.)
	\item $\pm    c\ (c \in \mathbb{R})$
	\item $\pm    c\cdot{}x^n\ (c \in \mathbb{R})$
	\item $\cdot\ c\ (c \in \mathbb{R}^+\backslash 0)$
	\item $:      c\ (c \in \mathbb{R}^+\backslash 0)$
\end{itemize}

Bei den folgenden Umformungen dreht sich das Ungleichheitszeichen:
\begin{itemize}
	\item $\cdot\ c\ (c \in \mathbb{R}^-\backslash 0)$
	\item $:      c\ (c \in \mathbb{R}^-\backslash 0)$
  \end{itemize}
  
  \end{gesetz}

\subsection*{Aufgaben}

\AadBMTA{114}{7. a), 8. a) und 9. a) c)}

\newpage
\subsection{Quadratische Ungleichungen}

Einstiegsbeispiel:

$$-2x^2 < -6$$

\TNT{6}{
  1. Gegenzahl
  $$2x^2 > 6$$
  2. Durch 2 teilen:
  $$x^2 > 3$$
  3. a) positive Wurzel:
  $$x>\sqrt{3}$$
  3. b) negative Wurzel:
  $$x < -\sqrt{3}$$

  4. Lösungsmenge:
  $$\lx = ]-\infty; -\sqrt{3}] \cup [+\sqrt{3};\infty[$$
}%% END TNT


    
    \begin{gesetz}{}{}
      Ein beidseitiges positives und negatives Wurzelziehen bei einer
      Ungleichung kehrt bei der negativen Variante das
      Ungleichheitszeichen um.
  \end{gesetz}

\newpage


\subsubsection{Lösung mit Funktionsgraph}
Wir betrachten wieder $-2x^2 < -6$. Dies ist, wie wir bereits wissen,
äquivalent zu: 
$$x^2 > 3$$
\TNTeop{
  $$\Longleftrightarrow x^2-3 > 0$$
  $$\textrm{mit }f(x) := x^2-3$$
  $$\Longleftrightarrow f(x) > 0$$
  Zeichne $f(x)$ und suche die Nullstellen.
  \bbwCenterGraphic{8cm}{tals/ungl/img/parabel.png}

  Grenzen der Lösungen nun ablesen bei $-\sqrt{3}$ und $+\sqrt{3}$.

  $$\lx = ]-\infty;-\sqrt{3}[ \phantom{...}\cup\phantom{...} ]\sqrt{3};\infty[$$
}%% END TNTeop
%%\newpage

%%\subsection*{Aufgaben}

%%\TALSAadBMTA{186}{670.}


\subsubsection{Graphische Lösungsmethode (Optional)}

Bsp.:

Für welche $x$ gilt:
\noTRAINER{$$(2x+2) \cdot{} \left(\frac{x}2 -1.5\right) > 0$$}
\TRAINER{$$\overbrace{{\color{black}(2x+2)}}^{A(x)}\cdot{}\overbrace{{\color{black}\left(\frac{x}2 -1.5\right)}}^{B(x)}{\color{black} > 0}$$}

\TNT{8}{
  \bbwCenterGraphic{15cm}{tals/ungl/img/plusminus.png}
}%% END TNT

$$\lx = \LoesungsRaumLang{]-\infty;-1[ \,\,\,\, \cup \,\,\,\,  ]3; \infty[     }$$

\subsection*{Aufgaben}

\AadBMTA{131}{22. a) c) e)}
