%%
%% 2019 07 04 Ph. G. Freimann
%%

\section{Kreisberechnungen}\index{Kreisberechnungen}\index{Berechnungen am Kreis}
\sectuntertitel{Warum sind Seeräuber so schlecht in Geometrie? --- Weil sie $\pi$ raten!}
%%%%%%%%%%%%%%%%%%%%%%%%%%%%%%%%%%%%%%%%%%%%%%%%%%%%%%%%%%%%%%%%%%%%%%%%%%%%%%%%%


\subsection*{Lernziele}

\begin{itemize}
\item einfache Kreisberechnungen
\item Kreisring
\item Sehne (und Sekante)
\item Segment und Sektor
\end{itemize}

\TadBMTG{56}{4}


\begin{gesetz}{Kreislinie}{}\index{PI@$\pi$ (Pi)}\index{$\pi$}
  Die Länge der Kreislinie wird aus dem Durchmesser $d=2r$  mittels Kreiszahl
  $\pi$ berechnet. Es gilt für den Umfang $U$:
  $$U = 2r\pi = d\pi$$
\end{gesetz}

\begin{gesetz}{Kreisfläche}{}\index{Kreisfläche}
  Die Kreisfläche $A$ eines Kreises mit Radius $r$ wird wie folgt
  berechnet:
  $$A = r^2\pi$$
\end{gesetz}
Herleitung
\TNTeop{Pizza: Schneiden, auslegen = halbe Rechtecksfläche.}

%%%%%%%%%%%%%%%%%%%%%%%%%%%%%%%%%%5
\subsection{Kreisteile}
\begin{gesetz}{Kreisring}{}\index{Kreisring}
  Die Kreisringfläche ist die Differenz der umgebenden Kreisfläche
  (Radius $R$) und
  der inneren Kreisfläche (Radius $r$):

  $$A = A(R) - A(r) = R^2\pi - r^2\pi = (R^2-r^2)\pi$$
\end{gesetz}

\begin{gesetz}{Kreisbogen und
    Kreissektor}{}\index{Kreissektor}\index{Kreisbogen}
  Für den Sektorwinkel $\varphi$ werden Bogen $b$ und Sektorfläche
  $A_{SK}$ wie folgt berechnet:
  $$b = 2r\pi \cdot{}\frac{\varphi}{360\degre} =
  r\pi\cdot{}\frac{\varphi}{180\degre}$$
  $$A_{SK} = r^2\pi\cdot{}\frac{\varphi}{360\degre} = \frac12\cdot{}b\cdot{}r$$
  \end{gesetz}

Begründung
\TNT{10}{Bogen: Es gilt das Verhältnis
  $$b : 2r\pi = \alpha : 360\degre$$

  Analog Fläche:
$$A : r^2\pi = $$
}

Für das Kreissegment Siehe \cite{marthaler20geom} Seite 61 Kap. 4.2.3.



\newpage
\subsection*{Aufgaben}

Kreisfläche und Kreisumfang:

%% Kreisfläche
%%\TALSAadBFWG{43ff (Kreisfläche)}{159. 160. 165. 168.}
\AadBMTG{63}{1., 2. c) d), 4. und 11.}

%% Kreissektor und Segment
%%\TALSAadBFWG{48ff(Kreissektor)}{179. 180. 183. 185. 187. 188. 189. 192. 194.}
Bogenmaß:

\AadBMTG{65}{12. c), 13. (Bogenmaß), 14.}

Kreissektor:

\AadBMTG{66}{15. b), 17.}

Beliebige Winkel (Repetition: Flächenformel für allgemeine Dreiecke)

\AadBMTG{123}{39. a) b) c) d)}%% Aufgabe schon vorgelöst
%% Kreissegment
%%\TALSAadBFWG{51ff (Kreissegment)}{197. 198. 200. 204. 206.}


%% Vermischte Aufgaben
%%\TALSAadBFWG{53ff (vermischte Aufgaben)}{209. 215. 219. 223.}
