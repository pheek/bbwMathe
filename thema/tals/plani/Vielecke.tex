%%
%% 2019 07 04 Ph. G. Freimann
%%
\section{Vielecke und deren Elemente}

\sectuntertitel{Sei $A$ ein Punkt $Q$; wir wollen ihn $M$ nennen}


\subsection{Elementare Objekte}\index{Elementare Objekte}
%%%%%%%%%%%%%%%%%%%%%%%%%%%%%%%%%%%%%%%%%%%%%%%%%%%%%%%%%%%%%%%%%%%%%%%%%%%%%%%%%

\subsection*{Lernziele}

\begin{itemize}
\item Vierecke
  \begin{itemize}
  \item Quadrat und Rechteck
  \item Rhombus (Raute) und Parallelogramm
  \item Trapez
  \end{itemize}
\item Dreiecke
  \begin{itemize}
  \item allgemeines Dreieck
  \item spezielle Dreiecke
  \end{itemize}

\item Elemente:
  \begin{itemize}
  \item Höhen, Seiten- und Winkelhalbierende und
    Mittelsenkrechte im Dreieck,
  \item Mittellinie im Trapez,
  \item Kreis: Tangente
  \end{itemize}
\item Zusammenhänge
  \begin{itemize}
  \item
    Umfang
  \item
    Flächeninhalt 
  \end{itemize}

\end{itemize}
\TadBMTG{43}{3}

\newpage
\subsubsection{Begriffe}

Skizzieren Sie die folgenden Begriffe

\begin{tabular}{rp{90mm}}
  Höhe $h_a$, $h_b$ und $h_c$ & Senkrechte (Lot) auf die Seiten.\\\hline
  
  Winkelhalbierende $w_a$, $w_b$ und $w_c$ & halbiert den Winkel. Drei
    Winkelhalbierende im Dreieck schneiden sich im Innkreismittelpunkt\\\hline

    Seitenhalbierende $s_a$, $s_b$ und $s_c$ & halbieren die
  gegenüberliegenden Seiten und sind somit die
  Schwerlinien\index{Schwerlinie}\\\hline
  
  Diagonale & In Vierecken verbinden Diagonale die genenüberliegenden
  Ecken\\\hline

  Mittellinie (im Trapez) & Auch Mittelparallele: Gibt den Mittelwert
  der beiden parallelen Seiten an. Fläche $A = mh = \frac{a+b}2h$
\end{tabular}

\TNTeop{}
\newpage

\begin{rezept}{Geometrische Aufgaben}{}
Um geometrische Aufgaben zu lösen, hat bei mir folgendes meist geholfen:

\begin{enumerate}
\item Machen Sie eine Skizze
\item Machen Sie eine möglichst genaue Konstruktion
\item Geben Sie Gegebenem und Gesuchtem Namen
\item Verwenden Sie Farben für Gegebenes
\item Verwenden Sie die selben Farben (od. Symbole) für die selben Streckenlängen, Winkel, Flächen
\item Bei Aufgaben mit berührenden Kreisen: Verbinden Sie die
  Mittelpunkte und verlängern Sie durch den Berührungspunkt
\item Suchen Sie rechtwinklige Dreiecke (Pythagoras)

\end{enumerate}
\end{rezept}


\subsection*{Aufgaben}
\AadBMTG{51}{2. b), 4. (Rechteck), 5. b) (Parallelogramm) 8. (Rhombus), 12. (Quadrat), 22. (Umkreis) , 25. (Inkreis/Trapez)}

\aufgabenFarbe{Strukturaufgaben: 3. y) z) aa) bb)}

\olatLinkArbeitsblatt{Strukturaufgaben}{https://olat.bbw.ch/auth/RepositoryEntry/572162090/CourseNode/107682599240642}{VT1\_6. (Trapez  mit speziellen Winkeln)}


\aufgabenFarbe{Maturitätsprüfung Grundlagenfach 2022 (Teil 1 ohne
  Taschenrechner): Aufgabe 8 (Sechseck)}





\GESOAadBMTA{???}{???}
\newpage


%%
%% 2019 07 04 Ph. G. Freimann
%%

%%\subsection{Elemente und Zusammenhänge im Vieleck}
%%\sectuntertitel{$H_2O$ ist kein Element}
%%%%%%%%%%%%%%%%%%%%%%%%%%%%%%%%%%%%%%%%%%%%%%%%%%%%%%%%%%%%%%%%%%%%%%%%%%%%%%%%%

%%\TadBMTG{43}{3}
%%\TALSTadBFWG{1.1.2}{1.1.2}


%%\subsection*{Vermischte Aufgaben}
%%\TALSAadBFWG{???}{???}
%%Das Buch hat hierzu (mit Ausnahme der Tangente \TALSTadBFWG{33}{1.2.5}) keine zusammengefasste Theorie: Die Aufgaben dazu sind verstreut in den übrigen Kapiteln zur Planimetrie zu finden (Seite 10 - 82 \cite{frommenwiler18geom}).
%%\TALSAadBFWG{10ff}{2. (1) (2) (3) 4. }
%%\TALSAadBFWG{16ff (Winkel am Kreis)}{23. a) c) e) }
%%\TALSAadBFWG{27ff (Kreisberührung)}{81. 82. 83. 86.}
%%\TALSAadBFWG{33ff (Tangenten)}{106. 107.}
%%\TALSAadBFWG{34ff (vermischte Aufgaben)}{113. }
%%\GESOAadBMTA{???}{???}
%%\newpage
