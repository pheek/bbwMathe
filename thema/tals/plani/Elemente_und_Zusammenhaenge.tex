%%
%% 2019 07 04 Ph. G. Freimann
%%
\IstinVieleckeEingebundenWirdNichtVerwendet!

\section{Elemente und Zusammenhänge im Vieleck}
\sectuntertitel{$H_2O$ ist kein Element}
%%%%%%%%%%%%%%%%%%%%%%%%%%%%%%%%%%%%%%%%%%%%%%%%%%%%%%%%%%%%%%%%%%%%%%%%%%%%%%%%%

\TadBMTG{43}{3}
%%\TALSTadBFWG{1.1.2}{1.1.2}


\subsection*{Lernziele}

\begin{itemize}
\item Elemente:
  \begin{itemize}
  \item Höhen, Seiten- und Winkelhalbierende und
    Mittelsenkrechte im Dreieck,
  \item Mittellinie im Trapez,
  \item Kreis: Tangente
  \end{itemize}
\item Zusammenhänge
  \begin{itemize}
  \item
    Umfang
  \item
    Flächeninhalt 
  \item
    Abstände
  \end{itemize}
\end{itemize}



\subsection*{Aufgaben}
%%\TALSAadBFWG{???}{???}
Das Buch hat hierzu (mit Ausnahme der Tangente \TALSTadBFWG{33}{1.2.5}) keine zusammengefasste Theorie: Die Aufgaben dazu sind verstreut in den übrigen Kapiteln zur Planimetrie zu finden (Seite 10 - 82 \cite{frommenwiler18geom}).
%%\TALSAadBFWG{10ff}{2. (1) (2) (3) 4. }
%%\TALSAadBFWG{16ff (Winkel am Kreis)}{23. a) c) e) }
%%\TALSAadBFWG{27ff (Kreisberührung)}{81. 82. 83. 86.}
\TALSAadBFWG{33ff (Tangenten)}{106. 107.}
\TALSAadBFWG{34ff (vermischte Aufgaben)}{113. }
\GESOAadBMTA{???}{???}
\newpage
