%%
%% 2019 07 04 Ph. G. Freimann
%%
\subsection{Strahlensätze}\index{Strahlensätze}
\sectuntertitel{Frau zum Arzt: Das Röntgenbild von meinem Mann können Sie sich sparen: Ich durchschaue ihn auch so.}

%%%%%%%%%%%%%%%%%%%%%%%%%%%%%%%%%%%%%%%%%%%%%%%%%%%%%%%%%%%%%%%%%%%%%%%%%%%%%%%%%
\subsection*{Lernziele}
\begin{itemize}
  \item Anwenden der Strahlensätze
\end{itemize}

\TALSTadBMTG{68}{5.1 und 5.2}


\TRAINER{\bbwCenterGraphic{8cm}{tals/plani/img/strahlensatz.png}}
\noTRAINER{\bbwCenterGraphic{8cm}{tals/plani/img/strahlensatzLeer.png}}

Die drei folgenden Dreiecke sind ähnlich (Symbol $\sim{}$)

\TNT{1.6}{$$\Delta ( a, p, b)  \sim{} \Delta (a+x, q, b+y)  \sim{}
  \Delta (x, q-p, y)$$}

Somit gelten \zB{} folgende Ähnlichkeiten...

\TNT{1.6}{$$a : (a+x) = p : q = b : (b+y)$$}

...aber auch:

\TNT{1.6}{$$a:x = b:y$$}
\newpage

\aufgabenFarbe{Begründen Sie, warum in der folgenden Skizze gilt, dass
  $a:b = x:y$}

\TRAINER{\bbwCenterGraphic{8cm}{tals/plani/img/strahlensatz2.png}}
\noTRAINER{\bbwCenterGraphic{8cm}{tals/plani/img/strahlensatz2Leer.png}}

\TNT{5.6}{
  Die unteren beiden Dreiecke haben eine gemeinsame Seite zu den
  beiden oberen Dreiecken.
  $$a: b = s : (s+t) = x:y$$
  \vspace{22mm}
}%% END TNT


\begin{bemerkung}{Außenglieder, Innenglieder}{}\index{Innenglieder!Brüche}\index{Außenglieder!Brüche}
Das Produt der Innenglieder ist gleich dem Produkt der Außenglieder:
$${\color{ForestGreen} a}:{\color{red} b}={\color{gray} c}:{\color{blue} d}  \Longleftrightarrow \frac{{\color{ForestGreen} a}}{{\color{red} b}}=\frac{{\color{gray} c}}{{\color{blue} d}} \Longrightarrow {\color{red} b}\cdot{}{\color{gray} c}={\color{ForestGreen} a}\cdot{}{\color{blue} d}$$0
\end{bemerkung}



\subsection*{Aufgaben}
%%\TALSAadBFWG{57 \zB}{225. a) b) 226. 231.}
%% Aufgabe 3 nicht mehr, da im Buch falsch
\AadBMTG{79}{4., 5., 6., 7.}
\aufgabenFarbe{Strukturaufgaben: 3. z) aa)}

\GESOAadBMTA{???}{???}
\newpage
