%%
%% 2019 07 04 Ph. G. Freimann
%%
\subsection{Strahlensätze}\index{Strahlensätze}
\sectuntertitel{Frau zum Arzt: Das Röntgenbild von meinem Mann können Sie sich sparen: Ich durchschaue ihn auch so.}

%%%%%%%%%%%%%%%%%%%%%%%%%%%%%%%%%%%%%%%%%%%%%%%%%%%%%%%%%%%%%%%%%%%%%%%%%%%%%%%%%
\subsection*{Lernziele}
\begin{itemize}
  \item Anwenden der Strahlensätze
\end{itemize}

\TALSTadBMTG{68}{5.1 und 5.2}


\TRAINER{\bbwCenterGraphic{8cm}{tals/plani/img/strahlensatz.png}}
\noTRAINER{\bbwCenterGraphic{8cm}{tals/plani/img/strahlensatzLeer.png}}

Die drei folgenden Dreiecke sind ähnlich (Symbol $\sim{}$)

\TNT{1.6}{$$\Delta ( a, p, b)  \sim{} \Delta (a+x, q, b+y)  \sim{}
  \Delta (x, q-p, y)$$}

Somit gelten \zB{} folgende Ähnlichkeiten...

\TNT{1.6}{$$a : (a+x) = p : q = b : (b+y)$$}

...aber auch:

\TNT{1.6}{$$a:x = b:y$$}
\newpage


\subsection*{Aufgaben}
%%\TALSAadBFWG{57 \zB}{225. a) b) 226. 231.}
%% Aufgabe 3 nicht mehr, da im Buch falsch
\AadBMTG{79}{4., 5., 6., 7.}
\GESOAadBMTA{???}{???}
\newpage
