\subsection{Kreisberührung}\index{Kreisberührung!Planimetrie}


\bbwCenterGraphic{8cm}{tals/plani/img/HalbkreisMitRadien.png}

In obigem Halbkreis ist rechts ein Bogen mit demselben Kreisradius $r$
eingezeichnet. Links ist ein kleiner Kreis mit Radius $k$
einbeschrieben.

Drücken Sie $k$ durch $r$ aus!
\newpage

%% Lösungsseite
\TRAINER{\bbwCenterGraphic{8cm}{tals/plani/img/HalbkreisMitRadienMitLoesungen.png}}

\noTRAINER{\bbwCenterGraphic{12cm}{tals/plani/img/HalbkreisOhneRadien.png}}


\TNTeop{
  Pythagoras in den beiden Dreiecken:

  $$(r+x)^2 + k^2 = (r+k)^2$$
  $$x^2 + k^2 = (r-k)^2$$

  Von Hand (obwohl mit TR rascher)
  Erste Gleichung ausmultiplizieren und vereinfachen:
  
  $$(I)  \hspace{18mm} 2rx + x^2 = 2rk$$
  $$(II) \hspace{18mm}  x^2 = r^2-2rk \Longrightarrow  x = \sqrt{r^2-2rk}$$
  (II) in (I) einsetzen:
  $$2r\sqrt{r^2-2rk} + r^2 -2rk = 2rk$$
  zusammenfassen:
  $$2r\sqrt{r^2-2rk} = 4rk-r^2$$
  durch $r$ teilen
  $$2\sqrt{r^2-2rk} = 4k-r$$
  quadrieren
  $$4(r^2-2rk) = 16k^2-8rk+r^2$$
  ausmultiplizieren
  $$4r^2 - 8rk = 16rk^2-8rk+r^2$$
  zusammenfassen
  $$3r^2 = 16k^2$$
  durch 16 teilen und die Wurzel ziehen:
  $$k^2 = \frac{3r^2}{16} \Longrightarrow   k = \frac{\sqrt{3}r}{4}$$
  
}%% END TNT
\newpage


\begin{rezept}{Kreisberührung}{}
  \begin{itemize}

  \item Wählen Sie zwei Kreise, die sich \textbf{berühren}.

  \item Verbinden Sie den Berührungspunkt mit den beiden
    Kreismittelpunkten.

  \item Der Abstand der Mittelpunkte ist die Summe\footnote{bzw.: Bei
    sich von «innen» berührenden Kreisen gilt: Der Abstand der
    Mittelpunkte ist die \textbf{Differenz} der beiden Radien.} der beiden Radien.

  \item  Meist sind die Strecken zwischen den beiden Mittelpunkten
  \textbf{Hypothenusen} in rechtwinkligen Dreiecken. Dann verwenden Sie den
  \textbf{Satz des Pythagoras}.

    \end{itemize}
    
\end{rezept}

\subsection*{Aufgaben}

\AadBMTG{39ff}{28. und eine Teilaufgabe aus Aufg. 33. und zwei Teilaufgaben
  aus Aufg. 34.}


\olatLinkArbeitsblatt{Strukturaufgaben}{https://olat.bbw.ch/auth/RepositoryEntry/572162090/CourseNode/107682599240642}{VT1\_3. (Kreise
  im Kreis)}


\newpage
