
\subsection{Kreisberührung}\index{Kreisberührung!Planimetrie}


\bbwCenterGraphic{8cm}{tals/plani/img/HalbkreisMitRadien.png}

In obigem Halbkreis ist rechts ein Bogen mit demselben Kreisradius $r$
eingezeichnet. Links ist ein kleiner Kreis mit Radius $k$
einbeschrieben.

Geben Sie $k$ in Abhängikeit von $r$ an.\newpage

\TRAINER{\bbwCenterGraphic{8cm}{tals/plani/img/HalbkreisMitRadienMitLoesungen.png}}

\noTRAINER{\bbwCenterGraphic{12cm}{tals/plani/img/HalbkreisOhneRadien.png}}


\TNTeop{
  Pythagoras in den beiden Dreiecken:

  $$(r+x)^2 + k^2 = (r+k)^2$$
  $$x^2 + k^2 = (r-k)^2$$

  Von Hand (obwohl mit TR rascher)
  Erste Gleichung ausmultiplizieren und vereinfachen:
  
  $$(I)  \hspace{18mm} 2rx + x^2 = 2rk$$
  $$(II) \hspace{18mm}  x^2 = r^2-2rk \Longrightarrow  x = \sqrt{r^2-2rk}$$
  (II) in (I) einsetzen:
  $$2r\sqrt{r^2-2rk} + r^2 -2rk = 2rk$$
  zusammenfassen:
  $$2r\sqrt{r^2-2rk} = 4rk-r^2$$
  quadrieren
  $$4r^2(r^2-2rk) = 16r^2k^2-8r^3k+r^4$$
  ausmultiplizieren
  $$4r^4 - 8r^3k = 16r^2k^2-8r^3k+r^4$$
  zusammenfassen
  $$3r^4 = 176r^2k^2$$
  wurzel ziehen
  $$\sqrt{3} r^2 = 4rk$$
  ergo
  $$k = \frac{\sqrt{3}r^2}{4r} = \frac{\sqrt{3}r}{4}$$
  
}%% END TNT
\newpage


\begin{rezept}{Kreisberührung}{}
  Wählen Sie zwei Kreise, die sich \textbf{berühren}.

  Von sich berührenden Kreisen werden die \textbf{Mittlpunkte und der Berührungspunkt}
    miteinander verbunden. Die drei Punkte liegen immer auf einer Geraden.

  Meist sind die Strecken zwischen den beiden Mittelpunkten
  \textbf{Hypothenusen} in rechtwinkligen Dreiecken. Dann verwenden Sie den
  \textbf{Satz des Pythagoras}.
\end{rezept}

\subsection*{Aufgaben}

\AadBMTG{39ff}{28. und eine Teilaufgabe aus Aufg. 33. und zwei Teilaufgaben
  aus Aufg. 34.}
\newpage
