\subsection{Kreisberührung}\index{Kreisberührung!Planimetrie}


\bbwCenterGraphic{8cm}{tals/plani/img/HalbkreisMitRadien.png}

In obigem Halbkreis ist rechts ein Bogen mit demselben Kreisradius $r$
eingezeichnet. Links ist ein kleiner Kreis mit Radius $k$
einbeschrieben.

Drücken Sie $k$ durch $r$ aus!
\newpage

%% Lösungsseite
\TRAINER{\bbwCenterGraphic{8cm}{tals/plani/img/HalbkreisMitRadienMitLoesungen.png}}

\noTRAINER{\bbwCenterGraphic{12cm}{tals/plani/img/HalbkreisOhneRadien.png}}


\TNTeop{
  Pythagoras in den beiden Dreiecken
  (o.\,B.\,d.\,A.\footnote{o.\,B.\,d.\,A. = ohne Beschränkung der Allgemeinheit}: $r=1$):

  $$(1+x)^2 + k^2 = (1+k)^2$$
  $$x^2 + k^2 = (1-k)^2$$

  Von Hand (obwohl mit TR rascher)
  Erste Gleichung ausmultiplizieren und vereinfachen:
  
  $$(I)  \hspace{18mm} 1+2x+x^2 + k^2 = 1+2k+k^2 \Longrightarrow 2x + x^2 = 2k$$
  $$(II) \hspace{18mm}  x^2 = 1-2k \Longrightarrow  x = \sqrt{1-2k}$$
  (II) in (I) einsetzen:
  $$2\sqrt{1-2k} + 1 -2k = 2k$$
  zusammenfassen:
  $$2\sqrt{1 - 2k} = 4k-1$$
  durch $2$ teilen
  $$\sqrt{1-2k} =  2k-\frac12$$
  quadrieren (rechts: binom. Formel)
  $$1-2k = 4k^2-2k+\frac14$$
  $+2k$, dann zusammenfassen
  $$\frac34  = 4k^2$$
  durch 4 teilen und die Wurzel ziehen:
  $$k^2 = \frac3{16} \Longrightarrow   k = \frac{\sqrt{3}}4$$
  Da $r=1$ folgt $k = \frac{\sqrt{3}}4 \cdot{}r \approx 0.433 \cdot{} r$

  
}%% END TNT
\newpage


\begin{rezept}{Kreisberührung}{}
  \begin{itemize}

  \item Wählen Sie zwei Kreise, die sich \textbf{berühren}.

  \item Verbinden Sie den Berührungspunkt mit den beiden
    Kreismittelpunkten.

  \item Der Abstand der Mittelpunkte ist die Summe\footnote{bzw.: Bei
    sich von «innen» berührenden Kreisen gilt: Der Abstand der
    Mittelpunkte ist die \textbf{Differenz} der beiden Radien.} der beiden Radien.

  \item  Meist sind die Strecken zwischen den beiden Mittelpunkten
  \textbf{Hypothenusen} in rechtwinkligen Dreiecken. Dann verwenden Sie den
  \textbf{Satz des Pythagoras}.

    \end{itemize}

\end{rezept}

\subsection*{Aufgaben}

\AadBMTG{39ff}{28. und eine Teilaufgabe aus Aufg. 33. und zwei Teilaufgaben
  aus Aufg. 34.}

\olatLinkArbeitsblatt{Strukturaufgaben}{https://olat.bms-w.ch/auth/RepositoryEntry/6029786/CourseNode/107682599240642}{VT1\_3. (Kreise
  im Kreis)}

\newpage
