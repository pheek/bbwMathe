%%
%% 2019 07 04 Ph. G. Freimann
%%

\section{Satz des Pythagoras}\index{Pythagoras!Planimetrie}\index{Satz des Pythagoras!Planimetrie}
\sectuntertitel{Warum sitzen die Mathematiker im Winter immer in eine
  Ecke? --- Weil dort bestimmt 90 Grad ist.}
%%%%%%%%%%%%%%%%%%%%%%%%%%%%%%%%%%%%%%%%%%%%%%%%%%%%%%%%%%%%%%%%%%%%%%%%%%%%%%%%%
\TadBMTG{29}{2.3.2}

\subsection*{Lernziele}

\begin{itemize}
  \item Formel des Pythagoras
\end{itemize}

Repetition \ifisALLINONE{\totalref{satzDesPythagoras}}\fi
\begin{bemerkung}{Pythagoras}{}
  Im rechtwinkligen Dreieck gilt:
  $$a^2+b^2=c^2$$
\end{bemerkung}

%%\TALSAadBFWG{23}{56. 57. 58. 60. 63. }
\subsection*{Aufgaben}
\AadBMTG{37}{11., 12., 13. und 31.}
%%\GESOAadBMTA{???}{???}
\newpage
\subsection{Spezielle Dreiecke}
Repetition: Halbes Quadrat und halbes gleichseitiges Dreieck \ifisALLINONE{\totalref{spezielleDreiecke}}\fi

\noTRAINER{\bbwCenterGraphic{12cm}{tals/plani/img/regulaer.png}}
\TRAINER{\bbwCenterGraphic{12cm}{tals/plani/img/regulaerTrainer.png}}


\TNT{10}{
  Beziehungen im Quadrat:

  $a$ = Seitenlänge

  $d$ = Diagonale

  $\frac{d}2$ = halbe Diagonale

  Beziehungen:  $a = \frac{d}2 \cdot{} \sqrt{2}; d = a\cdot{} \sqrt{2}$

  \vspace{20mm}

  Beziehungen im gleichseitigen (regulären) Dreieck:

  $a$ = Seitenlänge

  $h$ = Höhe

  $\frac{a}2$ = halbe Seitenlänge

  Beziehungen:

 $h = \frac{a}2 \cdot{} \sqrt{3};  a=2 \cdot{} \frac{h}{\sqrt{3}} $

  \vspace{20mm}

}%% END TNT

\subsection*{Aufgaben}
\AadBMTG{38ff}{19., 20., 24., 26., 29., 38., 40. und 41.}
\newpage

\subsection{Kreisberührung}\index{Kreisberührung!Planimetrie}


\bbwCenterGraphic{8cm}{tals/plani/img/HalbkreisMitRadien.png}

In obigem Halbkreis ist rechts ein Bogen mit demselben Kreisradius $r$
eingezeichnet. Links ist ein kleiner Kreis mit Radius $k$
einbeschrieben.

Geben Sie $k$ in Abhängikeit von $r$ an.\newpage

\TRAINER{\bbwCenterGraphic{8cm}{tals/plani/img/HalbkreisMitRadienMitLoesungen.png}}

\noTRAINER{\bbwCenterGraphic{8cm}{tals/plani/img/HalbkreisOhneRadien.png}}


\TNTeop{
  Pythagoras in den beiden Dreiecken:

  $$x^2 + k^2 = (r-k)^2$$
  $$(r+x)^2 + k^2 = (r+k)^2$$

  Von Hand (obwohl mit TR rascher)
  Erste Gleichung ausmultiplizieren und vereinfachen:
  
  $$(I)   \hspace{18mm}  x^2 = r^2-2rk \Longrightarrow  x = \sqrt{r^2-2rk}$$
  $$(II)  \hspace{18mm} 2rx + x^2 = 2rk$$
  (I) in (II) einsetzen:
  $$2r\sqrt{r^2-2rk} + r^2 -2rk = 2rk$$
  zusammenfassen:
  $$2r\sqrt{r^2-2rk} = 4rk-r^2$$
  quadrieren
  $$4r^2(r^2-2rk) = 16r^2k^2-8r^3k+r^4$$
  ausmultiplizieren
  $$4r^4 - 8r^3k = 16r^2k^2-8r^3k+r^4$$
  zusammenfassen
  $$3r^4 = 176r^2k^2$$
  wurzel ziehen
  $$\sqrt{3} r^2 = 4rk$$
  ergo
  $$k = \frac{\sqrt{3}r^2}{4r} = \frac{\sqrt{3}r}{4}$$
  
}%% END TNT
\newpage


\begin{rezept}{Kreisberührung}{}
  Wählen Sie zwei Kreise, die sich \textbf{berühren}.

  Von sich berührenden Kreisen werden die \textbf{Mittlpunkte miteinander verbunden}.

  Meist sind die Strecken zwischen den beiden Mittelpunkten
  \textbf{Hypothenusen} in rechtwinkligen Dreiecken. Dann verwenden Sie den
  \textbf{Satz des Pythagoras}.
\end{rezept}

\subsection*{Aufgaben}

\AadBMTG{39ff}{28. und eine Teilaufgabe aus Aufg. 33. und zwei Teilaufgaben
  aus Aufg. 34.}
\newpage
