%%
%% 2019 07 04 Ph. G. Freimann
%%

\section{Satz des Pythagoras}\index{Pythagoras!Planimetrie}\index{Satz des Pythagoras!Planimetrie}
\sectuntertitel{Warum sitzen die Mathematiker im Winter immer in eine
  Ecke? --- Weil dort bestimmt 90 Grad ist.}
%%%%%%%%%%%%%%%%%%%%%%%%%%%%%%%%%%%%%%%%%%%%%%%%%%%%%%%%%%%%%%%%%%%%%%%%%%%%%%%%%
\TadBMTG{29}{2.3.2}

\subsection*{Lernziele}

\begin{itemize}
  \item Formel des Pythagoras
\end{itemize}

Repetition \ifisALLINONE{\totalref{satzDesPythagoras}}\fi
\begin{bemerkung}{Pythagoras}{}
  Im rechtwinkligen Dreieck gilt:
  $$a^2+b^2=c^2$$
\end{bemerkung}

%%\TALSAadBFWG{23}{56. 57. 58. 60. 63. }
\subsection*{Aufgaben}
\AadBMTG{37}{11., 12., 13. und 31.}
%%\GESOAadBMTA{???}{???}
\newpage
\subsection{Spezielle Dreiecke}
Repetition: Halbes Quadrat und halbes gleichseitiges Dreieck \ifisALLINONE{\totalref{spezielleDreiecke}}\fi

\bbwCenterGraphic{12cm}{tals/plani/img/regulaer.png}


\TNT{10}{
  Beziehungen im Quadrat:

  $a$ = Seitenlänge

  $d$ = Diagonale

  $\frac{d}2$ = halbe Diagonale

  Beziehungen:  $a = \frac{d}2 \cdot{} \sqrt{2}; d = a\cdot{} \sqrt{2}$

  \vspace{35mm}

  Beziehungen im gleichseitigen (regulären) Dreieck:

  $a$ = Seitenlänge

  $h$ = Höhe

  $\frac{a}2$ = halbe Seitenlänge

  Beziehungen:

 $h = \frac{a}2 \cdot{} \sqrt{3};  a=2 \cdot{} \frac{h}{\sqrt{3}} $

  \vspace{35mm}

}%% END TNT

\subsection*{Aufgaben}
\AadBMTG{38ff}{19., 20., 24., 26., 28., 29., 38., 40. und 41.}
\newpage

\subsection{Kreisberührung}\index{Kreisberührung!Planimetrie}


\bbwCenterGraphic{8cm}{tals/plani/img/KleinsterKreis.png}
In obigem Kreis sind zwei kleinere Kreise einbeschrieben. Berechnen
Sie den Radius $k$ des kleinsten Kreises aus dem gegebenen Radius $r$ des
großen Kreises.
\TNT{12}{
  \bbwCenterGraphic{8cm}{tals/plani/img/KleinsterKreisLoesung.png}
  $$\Delta A: x^2 + k^2 = (r-k)^2$$
  $$\Delta B: x^2 + (\frac{r}2 - k)^2 = (\frac{r}2 + k)^2$$
  Ausmultiplizieren und 2. Gleichung von 1. Gleichung subtrahieren:
  $$k^2 - \frac{r^2}4 +rk - k^2  =r^2  -2rk - \frac{r^2}4 - rk$$\
  $$-\frac{r^2}4 + rk = r^2 - 3rk - \frac{r^2}4$$
  $$rk = r^2 - 3rk$$
  $$4k = r$$
  $$k = \frac{r}4$$
}%% END TNT
\newpage


\begin{rezept}{Kreisberührung}{}
  Wählen Sie zwei Kreise, die sich \textbf{berühren}.

  Von sich berührenden Kreisen werden die \textbf{Mittlpunkte miteinander verbunden}.

  Meist sind die Strecken zwischen den beiden Mittelpunkten
  \textbf{Hypothenusen} in rechtwinkligen Dreiecken. Dann verwenden Sie den
  \textbf{Satz des Pythagoras}.
\end{rezept}

\subsection*{Aufgaben}

\AadBMTG{39ff}{28., 33., 34.}
\newpage
