%%
%% 2019 07 04 Ph. G. Freimann
%%

\section{Satz des Pythagoras}\index{Pythagoras!Planimetrie}\index{Satz des Pythagoras!Planimetrie}
\sectuntertitel{Warum sitzen die Mathematiker im Winter immer in eine
  Ecke? --- Weil dort bestimmt 90 Grad ist.}
%%%%%%%%%%%%%%%%%%%%%%%%%%%%%%%%%%%%%%%%%%%%%%%%%%%%%%%%%%%%%%%%%%%%%%%%%%%%%%%%%
\TadBMTG{29}{2.3.2}

\subsection*{Lernziele}

\begin{itemize}
  \item Formel des Pythagoras
\end{itemize}

Repetition \ifisALLINONE{\totalref{satzDesPythagoras}}\fi
\begin{bemerkung}{Pythagoras}{}
  Im rechtwinkligen Dreieck gilt:
  $$a^2+b^2=c^2$$
\end{bemerkung}

%%\TALSAadBFWG{23}{56. 57. 58. 60. 63. }
\subsection*{Aufgaben}
\AadBMTG{37}{11., 12., 13. und 31.}
%%\GESOAadBMTA{???}{???}
\newpage
\subsection{Spezielle Dreiecke}
Repetition: Halbes Quadrat und halbes gleichseitiges Dreieck \ifisALLINONE{\totalref{spezielleDreiecke}}\fi

\noTRAINER{\bbwCenterGraphic{12cm}{tals/plani/img/regulaer.png}}
\TRAINER{\bbwCenterGraphic{12cm}{tals/plani/img/regulaerTrainer.png}}


\TNT{10}{
  Beziehungen im Quadrat:

  $a$ = Seitenlänge

  $d$ = Diagonale

  $\frac{d}2$ = halbe Diagonale

  Beziehungen:  $a = \frac{d}2 \cdot{} \sqrt{2}; d = a\cdot{} \sqrt{2}$

  \vspace{10mm}

  Beziehungen im gleichseitigen (regulären) Dreieck:

  $a$ = Seitenlänge

  $h$ = Höhe

  $\frac{a}2$ = halbe Seitenlänge

  Beziehungen:

 $h = \frac{a}2 \cdot{} \sqrt{3};  a=2 \cdot{} \frac{h}{\sqrt{3}} $

  \vspace{10mm}

}%% END TNT


\newpage
\subsection*{Aufgaben}
\AadBMTG{38ff}{19., 20., 24., 26., 29., 38., 40. und 41.}

\olatLinkArbeitsblatt{Strukturaufgaben}{https://olat.bms-w.ch/auth/RepositoryEntry/6029786/CourseNode/107682599240642}{VT1\_4. (Tangenten mit 120\degre) und VT1\_5 (Spezielle Winkel)}

\newpage
