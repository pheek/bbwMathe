\subsection{Gleichungssysteme mit Parametern}\index{Parameter!Gleichungssysteme}\index{lineare Gleichungssysteme!mit Parametern}
Das folgende Gleichungssystem hat offensichtlich vier und nicht wie
üblich zwei Variable. Wenn wir das System hingegen nach $x$ und $y$
auf"|lösen, so können wir diese beiden Größen
\textbf{in Abhängigkeit} der beiden Parameter\footnote{Parameter =
  Bei- oder Nebenmaß} ($p$ und $q$) ausdrücken.

\gleichungZZ{3x-5y}{p-5q}{5x+6y}{16p+6q}

Zum Lösen können wir wieder mit der Additionsmethode vorgehen, indem
wir die erste Gleichung mit 6 und die zweite Gleichung mit 5 multiplizieren\footnote{Danke Melisa für den Tipp: Natürlich könnte man auch zuerst die erste Gleichung mit 5 und die zweite Gleichung mit 3 multiplizieren, um so das $x$ zu eliminieren. Doch mit Melisas Trick, fällt auch das $q$ direkt weg.
}:

\TNT{4}{\gleichungZZ{18x-30y}{6p-30q}{25x+30y}{80p+30q}}

Nach Addition der beiden Gleichungen erhalten wir

\TNT{2}{$43x=86p$,}

was uns zu $x=\LoesungsRaum{2p}$ bringt.
Einsetzen in eine der beiden ursprünglichen Gleichungen liefert $y=\LoesungsRaum{p+q}$.
Somit ist die Lösungsmenge


$$\LoesungsMenge{}_{(x;y)} = \LoesungsRaumLang{\left\{(2p; p+q)\right\}}$$


\subsection*{Aufgaben}
%%\TALSAadBMTA{127ff}{392. a) b), 394. a)}
%%\AadBMTA{151}{10. a) b) c) und d)}

%%  \AadBMTA{151}{8. d) h) und 9. a)}
\olatLinkArbeitsblatt{Gleichungssysteme}{https://olat.bms-w.ch/auth/RepositoryEntry/6029786/CourseNode/112603866140343}{Kap. 10: a) b) c) d) e)}
\newpage
  

\newpage
