\subsection{Sonderfälle (Fallunterscheidung)}\index{Fallunterscheidung!Gleichungssysteme}\index{Sonderfälle!Gleichungssysteme}\index{lineare Gleichungssysteme!Sonderfälle, Fallunterscheidung}

Berechnen Sie die Lösung für $(x,y)$ in Abhängigkeit von $a$:

\gleichungZZ{ax+ay}{1}{x-ay}{-1}

\TNTeop{
  Additionsverfahren:
  
  $$ax+x=0$$
  $$x(a+1)=0$$
  $$\Longrightarrow x=0, y=\frac1a$$


  $$\LoesungsMenge{}_{(x;y)} = \left\{\left( 0 ; \frac1a \right)\right\}$$

  Sonderfälle (SF)

  \textbf{SF1}: Lösung für $a=0$ kann die Lösung für $y$ nicht
  stimmen (Division durch 0).
  Setzen wir $a=0$ in die erste Gleichung ein, so entsteht eine
  Falschaussage:

  Für $a=0$ gilt also $\LoesungsMenge{}_{(x;y)} = \{\}$
  
  \textbf{SF2}: Lösung für $a+1=0$ gilt: $x$ ist beliebig und $a=-1$

  Aus der 2. Gleichung folgt mit $a=-1$: $x+y=-1$ und somit $y=-x-1$

}%% END TNTeop

%%%%%%%%%%%%% \ newpage %%%%%%%%%%%%%%%%%%

\subsection*{Aufgaben}
%\TALSAadBMTA{153}{20. a) c) e)}
\olatLinkArbeitsblatt{Gleichungssysteme}{https://olat.bms-w.ch/auth/RepositoryEntry/6029786/CourseNode/112603866140343}{Kap. 10.1: a) bis f)}


\newpage
