\subsection{Vektoren im Taschenrechner}\index{Taschenrechner!Vektorgeometrie}

\subsubsection{Eingabe}
Vektoren im Taschenrechner werden entweder mit eckigen Klammern und
einem Strichpunkt (Semikolon) wie folgt eingegeben\TRAINER{ (oder mit der Newline-Taste \nspirebutton{newline})}:
$$a := [3; 1]$$
oder direkt mit dem Vektor-Befehl bei den mathematischen Symbolen:

\bbwCenterGraphic{10cm}{tals/vecg1/img/TR_eingabe.png}

\subsubsection{Vektoraddition, Vektorsubtraktion}
Vektoren können mit $+$ bzw. $-$ addiert bzw. subtrahiert
werden:
\bbwCenterGraphic{3cm}{tals/vecg1/img/TR_plusminus.png}

\subsubsection{Gegenvektor}
Der Gegenvektor wird mit einem Minuszeichen erzeugt:
$$a:=\left(3\atop 1\right) \Longrightarrow -a = \left( -3 \atop
-1\right)$$
\newpage

\subsubsection{Mulitplikation mit Skalar (Vielfaches)}\index{Skalare Multiplikation!Taschenrechner}\index{Multiplikation!mit Skalar (TR)}

Um einen Vektor mit einem Skalar zu multiplizieren, wird beim
Taschenrechner das übliche Multiplikationszeichen verwendet.

\bbwCenterGraphic{3cm}{tals/vecg1/img/TR_skalarmultiplikation.png}

\subsection*{Aufgaben}
\AadBMTG{264}{27. a) bis e) [ohne f)]}
\newpage


\subsection{Länge im kartesischen Koordinatensystem}

Die Länge der Vektoren wird mittels «Pythagoras» berechnet.

Sei $\vec{a}  = \Spvek{x_a;y_a}$. Somit ist die Länge von
    $\vec{a}$ wie folgt zu berechnen:

    \begin{gesetz}{Betrag, Länge}{}
      Betrag von $\vec{a}$ := Länge von $\vec{a}$

      $$a = |\vec{a}| = \sqrt{x_a^2 + y_a^2}$$
      \end{gesetz}
    Notationen:

    \begin{beispiel}{}{}
      $$ \vec{a}= \Spvek{3;1}$$
        $$|\vec{a}| = \LoesungsRaumLang{\sqrt{3^2+1^2} = \sqrt{10}\approx 3.162}$$
      \end{beispiel}
    

    \begin{bemerkung}{}{}
      Im Taschenrechner werden die Vektoren mit eckigen Klammern
      definiert:

      \texttt{a := [3; 1]}

      Die Länge (Betrag) wird mit dem Befehl \texttt{norm} ermittelt:

      \texttt{norm(a)}
    \end{bemerkung}
    \subsection*{Aufgaben}
    \TALSAadBMTG{262ff}{9., 10., 19. a) 20. A, 21. a), 38.}
\newpage

\newpage

\subsubsection{Linearkombination finden}\index{Linearkombination!Taschenrechner}
Will ich einen Vektor mittels Linearkombination
\totalref{linearkombination} zerlegen, so geschieht das einfach mit dem
«solver». Bei gegebenen Vektoren $\vec{a}$, $\vec{b}$ und $\vec{c}$ sei also
wieder $s$ und $t$ gesucht, so dass gilt:
$$\vec{c} = s\cdot{}\vec{a} + t\cdot{}\vec{b}$$

\bbwCenterGraphic{90mm}{tals/vecg1/img/TR_linearkombination.png}

\subsection*{Aufgaben}
\AadBMTG{264}{27. f), 40. a) b), 42. 43. a) c)}

\newpage


Algebraisch kann man auch zeigen, dass eine Zerlegung in zwei Vektoren
nicht immer möglich ist:
\TNTeop{
Begründung Gleichungssystem:
  
  $$\vec{c} = s\cdot{}\vec{a\vphantom{b}} + t\cdot{}\vec{b}$$ heißt:
  \gleichungZZ{c_x}{s\cdot{}a_x + t\cdot{}b_x}{c_y}{s\cdot{}a_y +
    t\cdot{}b_y}

  Die Lösung (\zB mittels Einsetz-Verfahren) ist

%%  $$\det{} = a_xb_y - a_yb_x$$

  $$s = \frac{b_yc_x - b_xc_y}{a_xb_y-a_yb_x}$$
  und
  $$t = \frac{a_xc_y - a_yc_x}{a_xb_y-a_yb_x}$$

  Ist der Nenner Null, so gibt es keine (oder keine
  eindeutige) Zerlegung.
}%% END TNTeop
%%\newpage
