\section{Vektorbegriff}\index{Vektor}

%%\TALSTadBFWG{175}{4}
%%Weitere Theorie: S. 176-178, S. 181, S. 187 und S. 189
\TALSTadBMTG{235}{16.1}

\subsection*{Lernziele}
\begin{itemize}
\item Vektor als Repräsentant
\item Betrag
\item Richtung
\item Skalar vs. Vetkor
\end{itemize}

\begin{bemerkung}{Vektor}{}
  Der Begriff «Vektor» kommt aus dem Lateinischen und bedeutet so viel
  wie «Träger» oder «Fahrer».
\end{bemerkung}
\newpage
\subsection{Motivation}
%%\def\mycomyc{cm}
\bbwCenterGraphic{\defaultTextBreiteCM}{tals/vecg1/img/DreiSchwimmer.png}

Wer erreicht das andere Ufer am raschesten?
\TNTeop{

  Für die Geschwindigkeiten ist lediglich die «$y$»-Koordinate
  zuständig. Daneben rechnen wir erst alles in km / h.

Schwimmer A: $y = 2\cdot{}\sqrt2 $ (km/h) und somit erreicht er
das andere Ufer nach
$t=\frac{s}{v}=0.2 : (2\cdot{}\sqrt2) \approx 0.07071 [\text{h}] = 4.24 \text{ min.}$

Schwimmer B: $y = 3 $ (km/h) und somit erreicht er
das andere Ufer nach
$t=\frac{s}{v}=0.2 : 3 = 4 \text{ min.}$

Schwimmer(in) C: $y = 3.5\cdot{}\frac{\sqrt{3}}{2}$ (km/h) und somit erreicht sie
das andere Ufer nach
$t=\frac{s}{v}=0.2 : (1.75\cdot{}\sqrt{3}) \approx 3.96 \text{ min.}$

}%% END TNTeop
%%\newpage


Wo landen die drei Schwimmer am anderen Ufer?

\TNTeop{
  Bei allen dreien «addieren» wir die Pfeile. Das entstehende Dreieck
  ist ähnlich wie ein Dreieck über den ganzen Fluss. $x$ sei der
  «Drift» nach rechts.
  
  Schwimmer A: $\frac{6-2\cdot{}\sqrt{2}}{2\cdot{}\sqrt2}
  =\frac{x}{0.2} \Longrightarrow x\approx 224 \text{ m}$
  
  Schwimmer B: $\frac63 =\frac{x}{0.2} \Longrightarrow x = 400 \text{ m}$
  
  Schwimmer(in) C: $\frac{1.75 + 6}{1.75\cdot{}\sqrt3}=\frac{x}{0.2} \Longrightarrow x\approx 511 \text{ m}$

  \bbwCenterGraphic{13cm}{tals/vecg1/img/DreiSchwimmerLoesung.png}
  
  }%% END TNTeop
%%\newpage

\subsection{Freie Vektoren}\index{Freie Vektoren}\index{Vektoren!freie}

\subsubsection{Notation}\index{Notation!Vektor}\index{Vektor!Notation}

Ein Vektor wird entweder mit einem Pfeil über dem Buchstaben
${\color{blue}\vec{a}}$ oder aber mit einem Pfeil über der zwei
Punkebezeichnungen ${\color{red}\overrightarrow{{PQ}}}$ angegeben:

\bbwGraphLeer{-3}{6}{-1}{3}{
\bbwLetter{2,2}{\vec{a}}{blue}
\draw [->,blue] (1,1) -- (4,2);

\bbwLetter{-1.5,1}{P}{red}
\bbwLetter{-2.5,3}{Q}{red}
\draw [->,red] (-1,1) --(-2,3);
}%% END bbwGraph


\subsection{Richtung}\index{Richtung!eines Vektors}
Die \textbf{Richtung} eines Vektors wird nicht durch die Pfeillänge,
sondern durch den Strahl ab dem Anfangspunkt durch den Endpunkt des
Vektors festgelegt: «Wohin die Route führt.»


\subsection{Skalar}\index{Skalar}
Ein \textbf{Skalar} ist eine ungerichtete Größe, wie \zB die 2.5 in
``2.5 kg''.
Ein \textbf{Vektor} hingegen hat (mit Ausnahme des Nullvektors) immer
eine Richtung.

\begin{tabular}{ll}\hline
  Skalare Größen& Vektorielle Größen\\\hline

  Fläche [a]                   & Wind (Richtung inkl. Geschwindigkeit)\\
  Volumen [l, $\text{dm}^3$] & Kraft (Richtung und Stärke [N])\\
  Masse [kg]                   & Gewicht [N]\\
  Druck [Pa]                   & Strahlen (Licht, $\gamma$)\\
  Temperatur [K]               & Flussrichtung und Geschwindigkeit
\end{tabular}
\newpage

%% Marthaler hat keine entsprechenden einführenden Aufgaben ?
%%\subsection*{Aufgaben}\
%%\TALSAadBFWG{178ff}{1., 3. und 4.}
%%\newpage



\subsubsection{Repräsentant}\index{Repräsentant!Vektorgeometrie}
Betrachten Sie die beiden folgenden (freien) Vektoren ${\color{blue} \vec{a}}$ und
${\color{red}\vec{b}}$, welche beide durch mehrere Repräsentanten\index{Repräsentant!Vektor}
(Vertreter\index{Vertreter!Vektor}) eingezeichnet sind:

\bbwGraphLeer{-4}{7}{-3}{3}{
\bbwLetter{3.5,3}{\vec{a}}{blue}
\draw [->,blue] (1,1) -- (4,2);
\draw [->,blue] (2,2) -- (5,3);
\draw [->,blue] (-3.5,-1) -- (-0.5,0);
\draw [->,blue] (2,0.5) -- (5,1.5);
\bbwLetter{-1,3}{\vec{b}}{red}
\draw [->,red] (-1,1) --(-2,3);
\draw [->,red] (-1,-2) --(-2,0);
\draw [->,red] (2.5,-0.5) --(1.5,1.5);
\draw [->,red] (5,-3) --(4,-1);
\draw [->,red] (7,0.5) --(6,2.5);
}%% END bbwGraph

Alle {\color{red} roten} Repräsentanten bezeichnen ein und denselben
\textbf{freien} Vektor. (Ananlog gilt das für den {\color{blue} blauen} freien Vektor.)

\newpage


\subsection{Betrag (Länge) von Vektoren}\index{Betrag!eines Vektors}


%%\TALSTadBFWG{176}{4.1}
\TALSTadBMTG{236}{16.1.2}


\begin{definition}{Vektor}{}
  Ein \textbf{Vektor} besteht aus einer Länge und einer Richtung.
\end{definition}


    \begin{definition}{Länge}{}
      Ein Vektor $\vec{a}$ von Punkt $A$ nach $B$ hat den Betrag (= Länge)

      $$a = |\vec{a}| = \overline{AB} = \left|\overrightarrow{AB}\right|$$
    \end{definition}

\TRAINER{meteoblue.com - Wind Map zeigen: Blau = kleiner Betrag,
  Orange = großer Betrag}

\subsection*{Aufgaben}\
%%\TALSAadBFWG{178ff}{2.}
\TALSAadBMTG{245}{1., 2., 3. a) c)}
\newpage
