%% Vektoren in Koordinatensystemen
%% 2020 - 12 - 25 ph. g. freimann @bms-w.ch

\section{Koordinatensysteme}

\subsection*{Lernziele}
\begin{itemize}
  \item Vektoren in kartesischen Koordinaten
\item Vektoren in der Polarform
  zeichnen
\item Umrechnen von Kartesischen Koordinaten zu Polarkoordinaten und
  umgekehrt
\item Vektoren in Kartesischen Koordinaten sowie in Polarkoordinaten
  im Taschenrechner 
\end{itemize}
\newpage

\subsection{Kartesische Koordinaten}

\begin{definition}{Kartesische Koordinaten}{}
  Vektoren im Kartesischen Koordinatensystem werden durch ihre $x$-
  bzw. $y$-Komponente angegeben:

  $$\vec{v} = \Spvek{x_v;y_v}$$
\end{definition}

Betrachten Sie nochmals die beiden folgenden (freien) Vektoren ${\color{blue} \vec{a}}$ und
${\color{red}\vec{b}}$:

\bbwGraph{-4}{7}{-3}{3}{
\bbwLetter{3.5,3}{\vec{a}}{blue}
\draw [->,blue] (1,1) -- (4,2);
\bbwLetter{-1,3}{\vec{b}}{red}
\draw [->,red] (-1,1) --(-2,3);
}%% END bbwGraph

Tragen Sie die fehlenden Werte in die Tabelle ein\footnote{Der
  mathematisch positive Winkel wird ab der $x$-Achse im
  Gegenuhrzeigersinn gemessen.}:

%%\renewcommand{\arraystretch}{2}
\begin{bbwFillInTabular}{|c|c|c|}\hline
                 & ${\color{blue}\vec{a}}$   & ${\color{red}\vec{b}}$   \\\hline
  $x$-Komponente & \TRAINER{3}\noTRAINER{\hspace{10em}}      & \TRAINER{-1}\noTRAINER{\hspace{10em}}   \\\hline
  $y$-Komponente & \TRAINER{1}      & \TRAINER{2}     \\\hline
  Betrag\index{Betrag!eines Vektors} (=Länge) & \TRAINER{$\sqrt{10}$}     & \TRAINER{$\sqrt{5}$}     \\\hline
%  math. pos. Winkel  & \TRAINER{$\arctan{}\left(\frac13\right)\approx
%    18.43\degre$} & \TRAINER{$90\degre +
%    \arctan{}\left(\frac12\right)\approx 116.6\degre$}               \\\hline
\end{bbwFillInTabular}

Koordinatenschreibweise von $\color{blue}\vec{a}$ und $\color{red}\vec{b}$:\,\,
$\vec{a} = \left( \TRAINER{3}\noTRAINER{\,\,\,\,} \atop \TRAINER{1} \right)$
$\vec{b} = \left( \TRAINER{-1}\noTRAINER{\,\,\,\,\,} \atop \TRAINER{2} \right)$
\newpage
\subsubsection{Addition in kartesischen Koordinaten}
Vektoren im kartesischen Koordinatensystem werden addiert, indem ihre
$x$- bzw $y$-Komponenten separat addiert werden:

\begin{gesetz}{Addition}{}\index{Addition!Vektoren}\index{Vektoraddition}
$$\vec{a} + \vec{b} =   \left(x_a \atop y_a \right)  + \left( x_b \atop y_b \right) =
  \left( x_a + x_b \atop y_a + y_b \right)$$
  \end{gesetz}

\begin{beispiel}{Addition}{}
  $$\vec{a} = \left(3\atop 1\right) \text{ und } \vec{b} = \left(-1
  \atop 2\right)$$
  $$\Longrightarrow \vec{a} + \vec{b} = \left(3 + (-1) \atop 1 +
  2\right) = \left(2 \atop 3\right)$$
  \end{beispiel}

\begin{definition}{Ortsvektor}{}\index{Ortsvektor}
  Derjenige Repräsentant eines Vektors, welcher im Ursprung $\mathcal{O} = (0|0)$
  startet, wird \textbf{Ortsvektor} genannt.
\end{definition}

\bbwGraph{-1}{5}{-1}{4}{
\bbwLetter{3.5,0.5}{\vec{a}}{blue}
\draw [->,blue] (1,0.5) -- (4,1.5);
\bbwLetter{4.5,2.5}{\vec{b}}{blue}
\draw [->,blue] (4,1.5) -- (3,3.5);
\TRAINER{%%
\bbwLetter{2,2.5}{\vec{c}}{red}
\draw [->,red] (1,0.5) -- (3,3.5);
}%% end TRAINER
}%% END bbwGraph



\subsection*{Aufgaben}    
%%    \TALSAadBFWG{188}{46. a) c) und d), 47. und 48.}
\TALSAadBMTG{261}{3. a) c) e) g), 4., 6., 8., }
\newpage



\subsubsection{Taschenrechner}

Was fällt bei folgender Gleichung auf?
$$72x^2 - 605 x + 848 = 0$$

  Der TI-30X Pro MathPrint kann quadratische Gleichungen mit Zahlen direkt auflösen.
  Doch dazu einige Tipps:
  
  \begin{tabular}{|c|p{13cm}|}
    \hline
    Tasten & Bemerkung \\
    \hline
    \tiprobutton{2nd}\tiprobutton{cos_poly-solv} & Starte das Lösen
    von quadratischen Gleichungen. Lösung obiger Gleichung: $\lx=\left\{\frac{53}8,\frac{16}9\right\}$\\
    \hline
    \tiprobutton{2nd}\tiprobutton{mode_quit}      & Beende den Poly-Solver.\\
    \hline
    \tiprobutton{neg} & Negative Zahlen nicht mit dem Subtraktionsoperator sondern mit der Vorzeichen-Taste eingeben.\\
    \hline
    \textbf{ACHTUNG} & Das Resultat lässt beim TI-30X Pro negative Wurzeln zu! Dies wird in der Lösung mit $i$ angegeben ($i$ steht für die imaginäre Einheit: $i=\sqrt{-1}$). Das $i$ kann aber je nach Modus erst gesehen werden, wenn mit der Pfeiltaste ganz nach rechts \textit{gescrollt} wurde.
    
    Steht ein $i$ in der Lösung des Taschenrechners, so gibt es keine reellen Lösungen und wir schreiben

    $\lx=\{\}$\\
    \hline
  \end{tabular}


\begin{rezept}{}{}
Mit dem Taschenrechner reduzieren sich Aufgaben (ohne Parameter) auf
das \textbf{Umformen} einer quadratischen Gleichung in die \textbf{Grundform}.
\end{rezept}


  \newpage
  Lösen Sie mit dem Taschenrechner:
$$5x^2 - 7x + 2 = 0$$
  \TNT{2}{$x_{1,2} = 1, 0.4$\vspace{16mm}}
  $$x^2 -6x  = - 11$$
  \TNT{2}{Keine Lösung $\lx=\{\}$}

Lösen Sie: $$4.1x^2+6x=-15$$
\TNT{2}{Auch hier verschwindet das $i$.}

  
$$1001x^2 + x  = -1$$
\TNT{2}{$x_{1,2} = \text{keine reelle Lsg. Das }i\text{ verschwindet
    aus dem Sichtfeld}$\vspace{16mm}}%% END TNT
\newpage

\subsection*{Aufgaben}

\olatLinkArbeitsblatt{GlQuad}{https://olat.bms-w.ch/auth/RepositoryEntry/6029794/CourseNode/111365559999287}{Aufgabe 12.}

%\GESOAadBMTA{182}{ 8. e) i), 9. c) und 10. b) e)}

\olatLinkGESOKompendium{2.3.2.}{16}{46. bis 49.}

und Textaufgaben:

\olatLinkGESOKompendium{2.3.4.}{17ff}{56. bis 58.}

\newpage


\newpage

