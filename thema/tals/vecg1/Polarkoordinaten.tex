\subsection{Polarkoordinaten (optional)}\index{Polarkoordinaten}
\sectuntertitel{Opposite of a polarbear? A cartesian bear!}

\bbwCenterGraphic{16cm}{tals/vecg1/img/polen.jpg}

\newpage

\subsubsection{Winkel in Vektoren (optional)}

Kim schwimmt über den 200 m breiten Fluss mit einer Geschwindigkeit
von 0.9 m/s und einem Winkel gegenüber dem Ufer (gegen die Strömung,
welche 2.6 m/s beträgt)
von 62\degre.

Wo und an welcher Stelle wird Kim auf der anderen Flussseite ankommen?

a) Skizzieren Sie die Geschwindigkeiten als Vektoren

\bbwCenterGraphic{14cm}{tals/vecg1/img/FlussLeer.png}

b) Geben Sie für Fließ- und Schwimmrichtung die Vektoren (Meter pro Sekunde)
an:
$$\vec{f} = \LoesungsRaumLen{30mm}{\Spvek{2.6; 0}}; \vec{s} =
\LoesungsRaumLen{30mm}{0.9\cdot{}\Spvek{-\cos(62\degre);\sin(62\degre)}
  =  0.9\cdot{}\Spvek{\cos(118\degre);\sin(62\degre)}}$$

c) Es sei $t$ die Zeit (in Sekunden) und $x$ die $x$-Koordinate der
Ankunft auf der anderen Flusseite. Diskutieren Sie die folgende
Gleichung:

$$t\cdot{} (\vec{s}+\vec{f}) = \Spvek{x;200}$$

d) Lösen Sie obige Gleichung mit dem Taschenrechner nach $t$ und $x$
auf.

Lösung für die Zeit $t = \LoesungsRaumLen{25mm}{251}$ s und für $x =
\LoesungsRaumLen{25mm}{548}$ m.

\vspace{2mm}
e)
Lösen Sie mit diesem Wissen die drei Fluss-Probleme A, B und C der
Einstiegsaufgabe \totalref{woLandenDieSchwimmer}.
\newpage


Anstelle der $x$- und der $y$-Komponente können wir mit dem selben
Informationsgehalt auch den Winkel ($\varphi$)
(in mathematisch positiver Richtung) und die Länge ($r$) eines Vektors
angeben.

\bbwCenterGraphic{10cm}{tals/vecg1/img/polar.png}

Dieser Winkel wird im mathematisch positiven Sinne ab der $x$-Achse
angegeben. Der Vektor
$\Spvek{0;1}$ hat somit den Winkel $90\degre$.
  \begin{definition}{Polarkoordinaten}{}
    Vektoren in Polarkoordinaten werden in der Reihenfolge (Länge |
    Winkel) angegeben:
    $$\vec{a} = (r | \varphi)$$
    Dabei ist $r$ die Länge und $\varphi$ der Winkel im
    Gegenuhrzeigersinn ab der $x$-Achse gemessen.
  \end{definition}

  \begin{beispiel}{Polarkoordinaten}{}
    $$\vec{a} = (2 | 60\degre) = \LoesungsRaum{\left(1\atop \sqrt{3}\right)}$$
  \end{beispiel}
  
  \begin{bemerkung}{Nullvektor}{}\index{Nullvektor}
    Der Nullvektor hat keine Richtung.
    \end{bemerkung}
  \newpage
  
\subsubsection{Transformation (optional)}\index{Transformation!Polar-
    vs. Kartesische Koordinaten}
  Die Umrechnung von Polar- zu kartesischen
  Koordinaten und umgekehrt wird Transformation genannt.
  
  Dank unseren Freunden \textit{Sinus} und \textit{Cosinus} ist die
  Transformation aus Polarkoordinaten relativ einfach:
  \begin{rezept}{Transformation «polar» nach «kartesisch»}{}

    Gegeben:  Winkel $\varphi$ und Länge $r$

    Gesucht: $x$ und $y$ in kartesischen Koordinaten

    $$x = r\cdot{}\cos(\varphi)$$
    $$y = r\cdot{}\sin(\varphi)$$
  \end{rezept}

  Die Umkehrung ist etwas komplizierter, denn dabei müssen wir
  beachten, in welchen Quadranten die Pfeilspitze des Ortsvektors
  zeigt.
  Zum Glück nimmt uns das der Taschenrechner ab.

  \subsubsection{Kartesische Koordinaten aus Polarkoordinaten}
  Der Taschenrechner verwandelt Polarkoordinaten automatisch immer in
  kartesische Koordinaten um:

    \bbwCenterGraphic{5cm}{tals/vecg1/img/Polar2Kartesisch.png}
    \newpage

    
\subsubsection{Polarkoordinaten aus kartesischen}
  \begin{beispiel}{kartesisch zu polar}{}
    Gegeben der Vektor $\vec{a} = \left(-\sqrt{3} \atop 1\right)$:

    Eine Skizze zeigt uns rasch, dass es sich um $150\degre$ und eine
    Länge von $2$ handeln muss.

    \bbwCenterGraphic{5cm}{tals/vecg1/img/Kartesisch2Polar.png}
    \end{beispiel}

Rechnen Sie in kartesische Koordinaten um:

\begin{tabular}{rcl}
  $(2   | 30\degre)$ & $=$ & \LoesungsRaum{$\left(\sqrt3\atop 1\right)$}\\
  $(321 | 270\degre)$ & $=$ & \LoesungsRaum{$\left(0\atop -321\right)$}\\
  $(0   | 103.83346\degre)$ & $=$ & \LoesungsRaum{$\left(0\atop 0\right)$}
\end{tabular}

Rechnen Sie in Polarkoordinaten um:

\begin{tabular}{rcl}\vspace{2mm}
  $\left(2\atop 2 \right)$ & $=$ & \LoesungsRaum{$(2\cdot{}\sqrt2|45\degre)$}\\\vspace{2mm}
  $\left(0\atop -13 \right)$ & $=$ & \LoesungsRaum{$(13|270\degre)$}\\\vspace{2mm}
  $\left(-5\atop 1 \right)$ & $=$ & \LoesungsRaum{$(\sqrt{26}\cdot{} | {180\degre - \arctan(\frac15)}) \approx (5.099 | 168.69\degre)$}\\\vspace{2mm}
  $\left(0\atop 0 \right)$ & $=$ & \LoesungsRaum{$(0|18.35\degre) = (0|299.68\degre)= ...$} \\\vspace{2mm}
  
\end{tabular}

Berechnen Sie den Abstand der beiden Endpunkte der Ortsvektoren $(3|30\degre)$ und $(2|45\degre)$:

\TNTeop{

  Abstand $\approx 1.1870$

  $$a := (3; < 30)$$
  $$b := (2; <45)$$
  $$c:=b-a$$
  $$\text{norm}(c)$$
}
%% implicit \newpage


\subsubsection{Drei Schwimmer im Taschenrechner}
Mit dem Taschenrechner wird unsere Einstiegs-Aufgabe mit den drei
Schwimmern stark vereinfacht. Wir geben die Geschwindigkeit und
Richtung des Schwimmers «A» in Polarkoordinaten an
(\textbf{\texttt{schwimmera}}).

Die Geschwindigkeit und Richtung des Flusses können wir auch in kartesischen
Koordinaten angeben (\texttt{\textbf{fluss}}).

Der resultierende Vektor aus Schwimmer und Fluss ist einfach
die Summe der beiden Vektoren.

Wir suchen nun also ein Vielfaches ($t$) der resultierenden
Geschwindigkeit, sodass in $y$-Richtung gerade die 0.2 km erreicht
wird. Die «Verschiebung» am anderen Ufer in $x$-Richtung bezeichnen
wir einfach mit $x$.

\bbwCenterGraphic{12cm}{tals/vecg1/img/TR_DreiSchwimmer.png}

\begin{bemerkung}{}{}
  Auch wenn es nur eine Gleichung mit zwei Unbekannten ist, dürfen wir
  nicht vergessen, das Vektoren in der Ebene aus zwei Komponenten
  bestehen, somit haben wir genau genommen auch ein lineares
  Gleichungssystem mit zwei Gleichungen und zwei Unbekannten zu lösen.
  \end{bemerkung}
\newpage
  
\subsection*{Aufgaben}

Berechnen Sie mit dem Taschenrechner auch noch die Ankunftszeiten ($t$) und die Orte am anderen Ufer
($s$) für die beiden anderen Schwimmenden (B und C) aus der
Einstiegsaufgabe zur Vektorgeometrie.


\TALSAadBMTG{248}{19.}
%%\TALSAadBMTA{181ff}{18., 21. und 24.}
%%Mit Taschenrechner:
%%\TALSAadBMTA{181ff}{23.}
\newpage
