\section{Elementare Betragsgleichungen}\index{Betrag!Gleichung}\index{elementare Betragsgleichungen}\index{Gleichnug!Betragsgleichung}

\textbf{Repetition}

Der Abstand von Zürich ($z$)  nach Bern ($b$) ist natürlich gleich groß, wie der
Abstand von Bern nach Zürich. Mathematisch könnte das etwa so
dargestellt werden:

$$|z-b| = |b-z|$$

Bei Differenzen sprechen wir i.\,d.\,R. nur vom positiven Abstand. Der
Betrag ist definiert als

\begin{definition}{Betrag}{}

  $$|a| = \left\{  \begin{aligned} a & \text{ falls } a > 0 \\ (-a) &
    \text{ sonst. }  \end{aligned} \right.$$
\end{definition}


\begin{definition}{Betragsgleichung}{}
  Unter einer Betragsgleichung verstehen wir eine Gleichung, bei der
  die Gesuchte im Betrag vorkommt.
\end{definition}

\begin{beispiel}{Betragsgleichung}{}
  $$|x-3| - 4x = x$$
\end{beispiel}

\newpage
\subsection{Einstiegsbeispiele}
Eine elementare Betragsgleichung könnte nun so aussehen:

$$|A| = 5$$

Zum Lösen haben wir zwei Fälle.

\TNTeop{
\textbf{1. Fall:} Ist nämlich $A>0$; hier hat der Betrag keinen Einfluss. So lautet die Gleichung einfach
$A=5$ und wir haben damit auch schon eine Lösung gefunden.

\textbf{2. Fall:} Ist hingegen $A \le{} 0$, so müssen wir von $A$ die
\textbf{Gegegzahl}\index{Gegenzahl} einsetzen:
$$-(A) = 5$$
Dies führt uns sofort auf $A=-5$.

Daher lautet die Lösungsmenge $\mathbb{L}_A = \{-5; 5\}$

}%% end TNTeop
%%%%%%%%%%%%%%%%%%%%%%%%%%%%%%%%%%%%%%%%%%%%%%%%%%%%%%%%%%

Ein zweites Beispiel sieht nun etwas komplizierter aus:

  $$|x-3| - 4x = x$$

\TNTeop{
Als erstes separieren wir den Betrag:

  $$|x-3| = 5x$$

\textbf{1. Fall:} $(x-3)> 0$ führt zu $$x-3 = 5x$$ (minus
$x$) $$-3=4x$$ und somit $$x_1 = \frac{-3}4$$

\textbf{2. Fall:} $(x-3) \le{} 0$. Jetzt nehmen wir die Gegenzahl des
Betragsargumentes: $$-(x-3) = 5x$$ (plus
$x$) $$3=6x$$ und somit $$x_2 = \frac12$$

\textbf{Probe 1:}
Die Probe bei Betragsgleichungen ist unerläßlich:

$$|x_1 -3 | \stackrel{?}{=} 5x$$
Bitte nachrechnen $|-\frac34 -3 | \stackrel{?}{=}
5\cdot{}\frac{-3}4$. Dies ist nicht möglich, weil die rechte Seite der
Gleichung negativ wird.

NEIN! Die Lösung $x_1$ fällt also weg.

\textbf{Probe 2:}

$$|x_2 -3 | \stackrel{?}{=} 5x$$
$$\left|\frac12 -3 \right| \stackrel{?}{=} 5\cdot{} \frac12$$
$$|-2.5| \stackrel{!}{=} 2.5$$

Dies ist korrekt und es bleibt
$$\lx = \left\{\frac12 \right\}$$


}%% end TNTeop
%%%%%%%%%%%%%%%%%%%%%%%%%%%%%%%%%%%%%%%%%%%%%%%%%%%%%%%%%%
\begin{rezept}{Betragsgleichung}{}
  \begin{enumerate}
  \item Betrag separieren (alleine auf eine Seite nehmen)
  \item Zwei Gleichungen erstellen, einmal einfach Betragszeichen
    weglassen und einmal mit der Gegenzahl $|A(x)|$ wird $-(A(x))$.
  \item Beide Gleichungen separat Lösen
  \item Mit beiden Resultaten die Probe machen
  \end{enumerate}
  
\end{rezept}

\subsection*{Aufgaben}

\AadBMTA{259}{49. Lösen Sie die Aufgaben erstens mit obigem Rezept und
zweitens Graphisch, indem Sie je den Term links und den Term
rechts des Gleichheitszeichens als Funktionsterm auffassen.}
