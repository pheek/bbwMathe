%%
%% 2019 07 04 Ph. G. Freimann
%%

\section{Logarithmusfunktionen}\index{Funktionen!Logarithmus}\index{Logarithmusfunktionen}
\sectuntertitel{$e^x$}
%%%%%%%%%%%%%%%%%%%%%%%%%%%%%%%%%%%%%%%%%%%%%%%%%%%%%%%%%%%%%%%%%%%%%%%%%%%%%%%%%
\subsection*{Lernziele}

\begin{itemize}
\item Umkehrung der Exponentialfunktion
\item Nullstellen, Definitionsbereich
\end{itemize}

\TALS{(\cite{frommenwiler17alg} S.227 (Kap. 3.11))}
\GESO{(\cite{marthaler17}       S.329 (Kap. 19.2))}


\subsection{Definition}
Die Funktion $f(x): x \mapsto y = log(x)$ ist eine
Logarithmusfunktion.

\subsection*{Aufgaben}
\TALSAadB{227ff}{856-879}
\GESOAadB{340ff}{42-51}

