%%
%% 2019 07 04 Ph. G. Freimann
%%

\section{Logarithmusfunktionen}\index{Funktionen!Logarithmus}\index{Logarithmusfunktionen}
\sectuntertitel{Die Logarithmusfunktion meidet das Fitnessstudio, um
  nicht noch weiter abzuflachen.}

%%%%%%%%%%%%%%%%%%%%%%%%%%%%%%%%%%%%%%%%%%%%%%%%%%%%%%%%%%%%%%%%%%%%%%%%%%%%%%%%%

\TadBMTA{329}{19.2}
%%\TALS{(\cite{frommenwiler17alg} S.227 (Kap. 3.11))}
%%\GESO{(\cite{marthaler21alg}       S.329 (Kap. 19.2))}


\subsection{Definition}
\begin{definition}{Logarithmusfunktion}{}
  Die Funktion $f(x): x \mapsto y = \log(x)$ ist eine
  Logarithmusfunktion.
  \end{definition}
\newpage

\subsubsection{Umkehrung der Exponentialfunktion}\index{Umkehrfunktion!der Exponentialfunktion}

Skizzieren Sie $f(x) = 2^x$ und $g(x) = \log_2(x)$ ins selbe
Koordinatensystem und markieren Sie die charakteristischen Punkte. Was
sind jeweils Definitions- und Wertebereiche? Was ändert sich, wenn Sie
bei $f$ und $g$ statt der Zahl $2$ die Zahl $3$ verwenden?

\bbwGraph{-3}{4}{-3}{4}{%%  0.69315 = ln(2)
\TRAINER{  \bbwFunc{exp(\x*0.69315)}{-3:1.95}}
\TRAINER{  \bbwFunc{ln(\x)  / 0.69315}{0.2:3.99}}
  
}%%

$$
\mathbb{D}_f  = \LoesungsRaumLen{20mm}{\mathbb{R}};
\mathbb{W}_f  = \LoesungsRaumLen{20mm}{\mathbb{R}^{+} \backslash \{0\}};
\mathbb{D}_g  = \LoesungsRaumLen{20mm}{\mathbb{R}^{+} \backslash \{0\}};
\mathbb{W}_g  = \LoesungsRaumLen{20mm}{\mathbb{R}}
$$

\TNTeop{
  Bei der Exponentialfunktion $f$ ist 1 der $y$-Achsenabschnitt. Bei
  $g$ ist 1 die Nullstelle.

  Die beiden Funktionen $f$ und $g$ sind sich gegenseitig
  Umkehrfunktionen und somit das Spiegelbild an der Diagonalen $x=y$
  im Koordinatensystem.  
}
\newpage


\subsection*{Aufgaben}
%%\TALSAadBMTA{227ff}{856-879}
\AadBMTA{339ff}{41. a) c), 42. a) c), 44. a) b), 50. a) c) und
  optional 51.}

  \olatLinkTALSStrukturaufgabenSPF{Basiskenntnisse Funktionen Teil 1}{4}{3.}
  \olatLinkTALSStrukturaufgabenSPF{Basiskenntnisse Funktionen Teil 2}{17}{53. und 54.}

%%\GESOAadBMTA{340ff}{42-51}
