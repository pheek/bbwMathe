%% 2020 12 25 ph. g. Freimann
%%
\section{Längenprobleme}

\subsection*{Lernziele}
\begin{itemize}
\item Längen
\item Skalarmultiplikation
\end{itemize}

Im Kartesischen Koordinatensystem kann die Länge von Vektoren einfach
mit dem Satz des Pythagoras ermittelt werden.

\begin{gesetz}{in $\mathbb{R}^2$}{}
  $$|\vec{a}| = \sqrt{a_x^2 + a_y^2}$$
\end{gesetz}

\begin{gesetz}{in $\mathbb{R}^3$}{}
  $$|\vec{a}| = \sqrt{a_x^2 + a_y^2 + a_z^2}$$
\end{gesetz}


\subsection*{Aufgaben}
\TALSAadBFWG{196}{77. a), 83.}

\newpage


\subsection{Skalarmultiplikation im Raum}
Wie bereits in der Ebenen können Vektoren auch im Raum verlängert oder
verkürzt werden. Diese Operation nennen wir auch hier 

\begin{definition}{Skalarmultiplikation}{}
  Sei ein Vektor $\vec{a}$ in der Ebene oder im Raum gegeben. Weiter
  sei $x$ eine beliebige reelle Zahl. Dann bezeichnen wir mit
  
  $$\vec{b} = x \cdot{} \vec{\vphantom{b}a}$$
  eine Multiplikation mit einem Skalar. Der neue Vektor $\vec{b}$ ist
  der um $x$ gestreckte Vektor $a$. 
\end{definition}

\begin{beispiel}{Skalarmultiplikation}{}
  $$2.5\cdot{} \begin{pmatrix} 4 \\ -2.5 \\ 1.1 \end{pmatrix} \;
  = \begin{pmatrix}\TRAINER{2.5\cdot{} 4}\noTRAINER{\hspace{25mm}}  \\ \TRAINER{2.5\cdot{} (-2.5)} \\ \TRAINER{2.5\cdot{} 1.1} \end{pmatrix}
  = \begin{pmatrix}\TRAINER{10}\noTRAINER{\hspace{12mm}} \\ \TRAINER{-6.25} \\ \TRAINER{2.65} \end{pmatrix}$$
\end{beispiel}


\subsection*{Aufgaben}
\TALSAadBFWG{197}{86.}
