%% 2020 12 25 ph. g. Freimann
%%
\section{Längenprobleme}

\subsection*{Lernziele}
\begin{itemize}
\item Längen
\item Multiplikation mit einem Skalar (Vielfaches)
\end{itemize}

Im kartesischen Koordinatensystem kann die Länge von Vektoren einfach
mit dem Satz des Pythagoras ermittelt werden.

\begin{gesetz}{in $\mathbb{R}^2$}{}
  $$|\vec{a}| = \left| \Spvek{a_x;a_y}  \right|   = \sqrt{a_x^2 + a_y^2}$$
\end{gesetz}

\begin{gesetz}{in $\mathbb{R}^3$}{}
  $$|\vec{a}| =  \left|\Spvek{a_x;a_y;a_z}\right| = \sqrt{a_x^2 + a_y^2 + a_z^2}$$
\end{gesetz}

Bemerkung Taschenrechner: Im Taschenrechner kann die Länge (Distanz
\texttt{d}) mit der \texttt{norm}\index{norm!Vektorlänge} ermittelt
werden:

\texttt{d:=norm(vec)}

\begin{definition}{Einheitsvektor}{}\index{Einheitsvektor}
  Ein Vektor der Länge 1 heißt \textbf{Einheitsvektor}.
\end{definition}

\subsection*{Aufgaben}
%%\TALSAadBFWG{196}{77. a), 83.}
\aufgabenFarbe{Arbeitsblatt Aufg 3., 4., 5., 6., 7., Taschenrechner: 8. und 9.}
\newpage


\subsection{Vielfaches im
  Raum}\index{Vielfaches!Vektoren}\index{Multiplikation!mit Skalar}\index{skalare Multiplikation}
Wie bereits in der Ebenen können Vektoren auch im Raum durch eine
\textbf{Multiplikation mit einem Skalar} verlängert oder
verkürzt werden. Diese Operation nennen wir auch hier 

\begin{definition}{Multiplikation mit einem Skalar}{}
  Sei ein Vektor $\vec{a}$ in der Ebene oder im Raum gegeben. Weiter
  sei $k$ eine beliebige reelle Zahl. Dann bezeichnen wir mit
  
  $$\vec{b} = k \cdot{} \vec{\vphantom{b}a}$$
  eine Multiplikation mit einem Skalar. Der neue Vektor $\vec{b}$ ist
  der um $k$ gestreckte Vektor $a$.
\end{definition}

\begin{beispiel}{Vielfaches}{}

  $$2.5\cdot{} \Spvek{4;-2.5;1.1} \;
  = \Spvek{\TRAINER{2.5\cdot{} 4}\noTRAINER{\hspace{25mm}} ;    \TRAINER{2.5\cdot{} (-2.5)} ; \TRAINER{2.5\cdot{} 1.1}  }
  = \Spvek{\TRAINER{10}\noTRAINER{\hspace{12mm}} ; \TRAINER{-6.25} ; \TRAINER{2.65}}$$
\end{beispiel}

«Das Entgegengesetz»

\begin{bemerkung}{Streckung mit Richtungsänderung}{}
  Ist $k<0$, so ist die Richtung von $k\cdot{}\vec{a}$ der Richtung von $\vec{a}$ entgegengesetzt.
\end{bemerkung}


\subsection*{Aufgaben}
%%\TALSAadBFWG{197}{86.}
%%\TALSAadBMTG{264}{30. a) b) }

\aufgabenFarbe{Arbeitsblatt Aufg. 10., 11., 12., 13., 14., 15. und 16. (Taschenrechner)}
