%% 2020 12 25 ph. g. Freimann
%%
\section{Skalarprodukt}\index{Skalarprodukt}



% Weglassen, da nur ein Lernziel und so gibt es mehr Platz auf der Seite
%\subsection*{Lernziele}
%\begin{itemize}
%\item Skalarprodukt
%\end{itemize}

\subsection{Komponentenweise}\index{Komponentenweise!Skalarprodukt}
\TALSTadBMTG{266}{18}

%%\TALSTadBFWG{201}{4.5}

Ein Geschäft verkauft seine Produkte zu folgenden Preisen.

$$\vec{p} = \Spvek{3.20; 2.40; 1.80; 1.75; 4.20} \text{ entsprechende Produkte: }
\Spvek{\text{Brot};\text{Milch (lt)}; \text{Mehl (kg)}; \text{Zucker
    (kg)}; \text{Teigwaren}}$$

Vera $\vec{v}$ und Walter $\vec{w}$ kaufen dem entsprechend wie folgt ein. Was sind die
jeweiligen Totalpreise?

$$\vec{v} = \Spvek{1;0;0;2;2}; \vec{w} = \Spvek{2;1;1;2;0} $$

\TNT{6}{Vera: $1\cdot{} 3.20 + 0\cdot{} 2.40 + 0\cdot{}1.80 + 2\cdot{}
  1.75 + 2\cdot{}4.20 = 15.10$

Walter: $2\cdot{}3.20 + 1\cdot{}2.40 + 1\cdot{}1.80 + 2\cdot{}1.75 +
0\cdot{}4.20 = 14.10$}

\vspace{3mm}
Verka kauft für CHF \LoesungsRaumLen{30mm}{15.10} und Walter für CHF
\LoesungsRaumLen{30mm}{14.10} ein.

\TRAINER{Vektoren sind nicht auf $\mathbb{R}^2$ oder $\mathbb{R}^3$ beschränkt!}
\newpage

\begin{definition}{Skalarprodukt}{}\index{Skalarprodukt|textbf}
  $$\vec{a\vphantom{b}}\circ\vec{b} := a_1\cdot{}b_1 + a_2 \cdot{} b_2
  + a_3\cdot{}b_3 + ...$$
\end{definition}

\begin{gesetz}{Skalarprodukt ist ein Skalar}{}
  Das Skalarprodukt zwischen zwei Vektoren liefert eine skalare Größe.
\end{gesetz}

Berechnen Sie mit diesem Wissen die folgenden Skalarprodukte:

\vspace{3mm}


$$\Spvek{1;1;2} \circ \Spvek{3;4;-3.5} = \LoesungsRaumLen{30mm}{0}$$
$$\Spvek{\frac1{2\pi};\sqrt{3};\sqrt{2}} \circ \Spvek{\pi;
  \sqrt{3};\frac{7}{\sqrt{8}}} = \LoesungsRaumLen{30mm}{7}$$

Berechnen Sie den fehlenden Parameter $p$ so, dass das Skalarprodukt = 0 wird:
\vspace{3mm}

$$\Spvek{1;2;3}\circ{}\Spvek{-1;p;\sqrt{2}} \Longrightarrow p =
\LoesungsRaumLen{30mm}{\frac{1-3\sqrt{2}}{2} \approx -1.6213}$$


\newpage


\subsection{Skalarprodukt und Winkel}\index{Skalarprodukt!Winkel}

\bbwCenterGraphic{6cm}{tals/vecg2/img/skalarprodukt.png}

\TNTeop{
  $\vec{b} + \vec{c} = \vec{a} \Longrightarrow   \vec{c} = \vec{a} - \vec{b}$

  $\Longrightarrow   c_x = a_x - b_x \text{ und } c_y = a_y - b_y$

  $a = |\vec{a}|, b=|\vec{b}| \text{ und } c = |\vec{c}|$

  Pythagoras:
  $c^2 = c_x^2 + c_y^2$

  $a_x-b_x$ anstelle von $c_x$ einsetzen (analog $c_y=a_y-b_y$):

  $c^2 = (a_x-b_x)^2 + (a_y - b_y)^2$

  Wir erinnern an den Cosinussatz: $c^2 = a^2 + b^2 -
  2ab\cdot{}\cos(\gamma)$

  Setzen wir beide Gleichungen gleich (via $c^2$), so erhalten wir:

  $(a_x-b_x)^2 + (a_y-b_y)^2 = a^2 + b^2 - 2ab\cdot{}\cos(\gamma)$

  Rechts Pythagoras: $a^2 = a_x^2 + a_y^2$ und $b^2 = b_x^2 + b_y^2$.
  
  $(a_x-b_x)^2 + (a_y-b_y)^2 = \overbrace{a_x^2 + a_y^2}^{|\vec{a}|^2=a^2}
  + \overbrace{b_x^2 + b_y^2}^{b^2} - 2ab\cdot{}\cos(\gamma)$

  jetzt links ausmultiplizieren:
  
  $a_x^2 - 2a_xb_x +b_x^2 + a_y^2 - 2a_yb_y + b_y^2 = a_x^2 + a_y^2 +
  b_x^2 + b_y^2 - 2ab\cdot{}\cos(\gamma)$

  nun beginnt der Spaß mit dem Wegstreichen:

  $-2a_xb_x - 2a_yb_y = -2ab\cdot{}\cos(\gamma)$

  und noch kürzen mit $-2$:
  $$ a_xb_x + a_yb_y = ab\cdot{}\cos(\gamma)$$
  Bem.: Dieser Beweis funktioniert exakt gleich im $\mathbb{R}^3$, es
  fallen dann fünf weitere Summanden an, von denen sich vier aber auch
  wieder wegstreichen:

  \texttt{https://www.youtube.com/watch?v=RQ\_7laap-rM}
  }
\newpage
\subsubsection{Gesetze}
Dies gilt auch in $\mathbb{R}^3$:

\begin{gesetz}{}{}
  Für zwei Vektoren $\vec{a}$ und $\vec{b}$ gilt:\\
  $$\LoesungsRaumLen{60mm}{a_x\cdot{}b_x + a_y\cdot{}b_y  + a_z\cdot{}b_z} = \LoesungsRaumLen{60mm}{|\vec{a}|\cdot{}|\vec{b}|\cdot{}\cos(\gamma)}$$
\end{gesetz}
 

Nach obigem Gesetz sind die beiden folgenden Definitionen identisch:
\begin{definition}{Skalarprodukt im $\mathbb{R}^3$}{}\index{Skalarprodukt|textbf}
  $$\vec{a\vphantom{b}}\circ\vec{b} := a_x\cdot{}b_x + a_y \cdot{} b_y + a_z\cdot{}b_z$$
oder mit $\gamma$ als Zwischenwinkel von $\vec{a}$ und $\vec{b}$:
  $$\vec{a\vphantom{b}}\circ\vec{b} := |\vec{\vphantom{b}a}| \cdot{} |\vec{b}| \cdot{}\cos({\gamma})$$
\end{definition}



\begin{gesetz}{Winkel und Skalarprodukt}{}
Für den Zwischenwinkel zwischen $\vec{a}$ und $\vec{b}$ gilt:
  $$\cos(\gamma) = \LoesungsRaumLen{35mm}{\frac{\vec{a\vphantom{b}}\circ\vec{b}}{|\vec{\vphantom{b}a}|\cdot |\vec{\vphantom{b}b}|}}$$
  bzw.:
  $$\gamma = \LoesungsRaumLen{45mm}{\arccos{} \left(\frac{\vec{a\vphantom{b}}\circ\vec{b}}{|\vec{\vphantom{b}a}|\cdot |\vec{\vphantom{b}b}|}\right)}$$
  
\end{gesetz}
\newpage
\begin{bemerkung}{\texttt{dotp}}{}\index{dotp|texttt}
Mit dem Taschenrechner TI nSpire CAS\footnote{CAS: Computer-Algebra-System} wird das Skalarprodukt mit

\begin{center}\textbf\texttt{dotp(a, b)}\end{center}

berechnet.

\end{bemerkung}

optional:

\begin{bemerkung}{Skalarprodukt als Fläche (optional)}{}
  Das Skalarprodukt kann auch als Fläche angesehen werden:
  
  \weblink{Geogebra: Skalarprodukt als Fläche:\\
    (https://www.geogebra.org/m/AEbtvksE)}{https://www.geogebra.org/m/AEbtvksE}
\end{bemerkung}
\newpage


\subsubsection{Referenzaufgabe}
Berechnen Sie den fehlenden Parameter $p$, wenn Sie wissen, dass die
beiden Vektoren den Winkel $59\degre$ einschließen:

$$\vec{a} = \Spvek{5;3;p} \hspace{11mm} \vec{b} = \Spvek{6;p;9}$$

\TNTeop{

  Es gilt:

  $$\vec{a}\circ{}\vec{b} = 5\cdot{}6 + 3\cdot{}p + p\cdot{}9 =
  \left| \vec{a} \vphantom{\vec{b}}\right| \cdot{} \left|\vec{b} \right| \cdot{} \cos(59\degre)$$

  
a)

    TR: solve ($\cos(59) = \frac{5\cdot{}6 + 3p + p\cdot{} 9
    }{\text{norm}(a) \cdot{} \text{norm}(b)}$, p)

b)

    TR: solve ($\cos(59) = \frac{\text{dotp}(a, b)}{\text{norm}(a)\cdot{}\text{norm}(b)}$, p)

    
    Es gibt zwei Lösungen: p1 = 0.20926... und  p2=22.60099
  }%%
  %% implicit newpage

  Selbständig:

Gegeben sind die beiden Vektoren $\vec{a}$ und $\vec{b}$:

$$\vec{a} = \Spvek{9;8;1}; \vec{b} = \Spvek{11;12;t}$$

Bestimmen Sie den Parameter $t$ so, dass der Winkel zwischen den
beiden Vektoren $19\degre$ wird.

Lösen Sie die Aufgabe mit dem CAS-Taschenrechner.

\TNTeop{
  \bbwCenterGraphic{8cm}{tals/vecg2/img/ParameterT.png}
}%% implicit EOP 
%%%%%%%%%%%%%%%%%%%%%%%%%%%%%%%%%%%%%%%%%%%%%%%%%%%%%%%%%%%%%%%%%%%%%%%%%%%%%%%%%%%%%%
\subsection*{Aufgaben}

  \olatLinkArbeitsblatt{Vecg. II Skalarprodukt}{https://olat.bms-w.ch/auth/RepositoryEntry/6029786/CourseNode/108600440643739}{Kap. 1.1
    bis 1.5 die Aufgaben 1., 2., 3., 6., 7., 11., 15., 18. (von Hand), 22. (mit TR), 24. (von Hand), 27. (mit TR)} %% end link
Lösen Sie je die erste und zweite Aufgabe aus jedem Kapitel. Die weiteren Aufgaben
sind Übungsmaterial und gedacht, sich für die Prüfungen vorzubereiten.

%%\TALSAadBFWG{202}{102. a) d) e), 103. a) c), 104. a) c), 106. a),
%%107. a), 108., 109., 110. a), 113.}

%%\TALSAadBMTG{282ff}{1. a) c) e), 2. a) c) e), 4. a), 7. a) b) c), 8.,
%%  10. d), 20.}


\newpage

\subsection{Orthogonale Vektoren}\index{orthogonal!Vektor}


Einstiegsaufgabe:

a) Geben Sie einen Vektor an, der zum Vektor $\vec{a} = \Spvek{-3;2}$
orthogonal steht.

b) Geben Sie alle Vektoren an, die senkrecht zum obigen Vektor $\vec{a}$ sind.

c) Geben Sie alle Vektoren an, die im rechten Winkel zum Vektor $\vec{v} = \Spvek{v_x;
  v_y}$ stehen.

\TNTeop{
  a) Entweder $\Spvek{2;3}$ oder $\Spvek{-2;-3}$. (und alle Vielfachen davon)

  b) $\left\{\vec{v} \middle| \vec{v} = k\cdot{}\Spvek{2;3} \text{ und } k\in\mathbb{R}\right\}$

  c) $\left\{\vec{v} \middle| \vec{v} = k\cdot{}\Spvek{-v_y; v_x} \text{ und } k\in\mathbb{R}\right\}$

}



\begin{gesetz}{Orthogonale Vektoren}{}
  Zwei von Null verschiedene Vektoren $\vec{a}$ und $\vec{b}$ sind genau dann
  \textbf{orthogonal}, wenn das Skalarprodukt der beiden Vektoren
  Null ergibt.


  $$\vec{\vphantom{b}a} \perp \vec{b} \Longleftrightarrow
  \vec{\vphantom{b}a}\circ{}\vec{b} = 0$$
\end{gesetz}

Referenzaufgabe: Berechnen Sie $p$ von Hand so, dass die Vektoren $\vec{a}$ und $\vec{b}$ orthogonal zu einander stehen:

$$\vec{a} = \Spvek{7; 6}; \vec{b} = \Spvek{p; 2}$$


\TNT{4}{
  $$7p+12=0 \Longrightarrow p = \frac{-12}{7} \approx -1.71$$
}

\subsection*{Aufgaben}

  \olatLinkArbeitsblatt{Vecg. II Skalarprodukt}{https://olat.bms-w.ch/auth/RepositoryEntry/6029786/CourseNode/108600440643739}{Kap. 1.6
    bis 1.9 Aufgaben 31., 32., 33., 37., 41.} %% end link
Lösen Sie je die erste Aufgabe jedem dem Kapitel. Die weiteren Aufgaben
sind Übungsmaterial und gedacht, sich für die Prüfungen vorzubereiten.

%%\TALSAadBFWG{205}{123. b), 124.}
