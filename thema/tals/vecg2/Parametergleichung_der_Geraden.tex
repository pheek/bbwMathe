%% 2020 12 25 ph. g. Freimann
%%
\section{Parametergleichung der Geraden}\index{Gerade!Parametergleichung}\index{Parametergleichung!der Geraden}

\subsection*{Lernziele}
\begin{itemize}
\item Parametergleichung
\end{itemize}

\TALSTadBMTG{287}{19}
\newpage
\subsection{Ortsvektor}

\begin{definition}{Ortsvektor}{}
  Sei $O=(0|0|0)$ der Koordinatenursprung und $P=(x_P|y_P|z_P)$ ein Punkt im
  kartesischen Koordinatensystem.
  
  Ein \textbf{Ortsvektor}\index{Ortsvektor} $\vec{p}$ eines Punktes
  ist ein Repräsentant des Vektors $\overrightarrow{OP}$:
  $$\vec{p} = \overrightarrow{OP}=\begin{pmatrix}x_P\\y_P\\z_P\end{pmatrix}$$
  bzw.
  $$P = O + \vec{p}$$
\end{definition}

\begin{beispiel}{Ortsvektor}{}
  Sei
  $$P=(\LoesungsRaum{4}|\LoesungsRaum{-3}|\LoesungsRaum{s})$$
  ein
gegebener Punkt. So ist
 $$\vec{p}
= \begin{pmatrix}\LoesungsRaum{4}\\\LoesungsRaum{-3}\\\LoesungsRaum{s}\end{pmatrix}$$
der zugehörige \textbf{Ortsvektor}.
\end{beispiel}

\newpage
\subsection{Parametergleichnug}
Jede Gerade $\vec{r}$ in der Ebene oder im Raum kann durch die
folgende Gleichung dargestellt werden:

\begin{definition}{Parametergleichung}{}
  Bei gegebenem Referenzpunkt $A$ auf der Geraden und einer gegebenen
  Richtung $\vec{u}$ kann die Gerade $\vec{r}$ geschrieben werden als.
  
  $$\vec{r} = \vec{r}(t) = \vec{r}_A + t\cdot{} \vec{u}$$

  Dabei ist $t\in\mathbb{R}$ beliebig und $\vec{r}_A$ ist der
  Ortsvektor zum Referenzpunkt $A$.
\end{definition}

\subsection*{Aufgaben}

\TRAINER{Mögliche Aufgaben:

  Bestimmen Sie 3 Punkte auf einer gegebenen Geraden.

  Prüfen Sie, ob Punkt $P$ auf der Geraden $g$ liegt.

}

\TALSAadBMTG{303ff}{2. a) c), 5. a) b) c), 8., 9., 14. a), 18. a) b) c)
  d), 19., 20., 21. a), 25., 35.}
\newpage%
