%% 2020 12 25 ph. g. Freimann
%%
\section{Parametergleichung der Geraden}\index{Gerade!Parametergleichung}\index{Parametergleichung!der Geraden}

\subsection*{Lernziele}
\begin{itemize}
\item Parametergleichung
\end{itemize}

\TALSTadBMTG{287}{19}
\newpage
\subsection{Ortsvektor}

\begin{definition}{Ortsvektor}{}
  Sei $O=(0|0|0)$ der Koordinatenursprung und $A=(A_x|A_y|A_z)$ ein Punkt im
  kartesischen Koordinatensystem.
  
  Ein \textbf{Ortsvektor}\index{Ortsvektor} $\vec{a}$ eines Punktes
  ist derjenige Repräsentant des Vektors $\overrightarrow{OA}$, dessen
  Startpunkt bei $O$ liegt. 
  $$\overrightarrow{a_A} := \overrightarrow{OA}=\Spvek{A_x;A_y;A_z}$$
\end{definition}

\begin{bemerkung}{Ortsvektor}{}\index{Ortsvektor}
  Vektoren bezeichnen üblicherweise nur die Länge und die Richtung
  eines \textit{Pfeils}. Man versteht unter dem Vektor die Menge
  \textbf{aller} Repräsentanten.

  Um jedoch den spezifischen Repräsentanten anzugeben,
  der vom Ursprung zum Punkt $A$ führt, wird entweder explizit vom
  \textbf{Ortsvektor} gesprochen oder der Vektor wird dem Namen des
  Punktes als Index versehen:
  \vspace{5mm}
  
  \begin{bbwFillInTabular}{ll}
    Bezeichnung & Notation\\\hline
    Ortsvektor & $\overrightarrow{a_A} = \overrightarrow{OA}$ \\
    alle übrigen Repräsentanten & $\vec{a}$ 
  \end{bbwFillInTabular} 

\end{bemerkung}


\begin{beispiel}{Ortsvektor}{}
  Sei
  $$P=(\LoesungsRaum{4}|\LoesungsRaum{-3}|\LoesungsRaum{s})$$
  ein
gegebener Punkt. So ist
 $$\overrightarrow{v_P}
= \noTRAINER{\hspace{15mm}} \TRAINER{\Spvek{4;-3;s}}$$
der zugehörige \textbf{Ortsvektor}.
\end{beispiel}

\newpage
\subsection{Parametergleichung}

Skizze:

\noTRAINER{\bbwCenterGraphic{16cm}{tals/vecg2/img/ParameterDarstellung.png}}
\TRAINER{\bbwCenterGraphic{16cm}{tals/vecg2/img/ParameterDarstellungTrainer.png}}

\begin{bemerkung}{Parameter $t$}{}
  Für jedese $t\in\mathbb{R}$ wird mit
  $$\overrightarrow{OA} + t\cdot{} \vec{u}$$
  ein weiterere Punkt auf der Geraden bestimmt.
  \end{bemerkung}
\newpage

Jede Gerade $\vec{r}$ in der Ebene oder im Raum kann durch die
folgende Gleichung dargestellt werden:

\begin{definition}{Parametergleichung}{}
  Bei gegebenem Referenzpunkt $A$ auf der Geraden und einer gegebenen
  Richtung $\vec{u}$ kann die Gerade $\vec{r}$ geschrieben werden als.
  
  $$\vec{r} = \vec{r}(t) = \overrightarrow{a_A} + t\cdot{} \vec{u}$$

  oder kurz:

  \begin{center}\fbox{\fbox{$g:\hspace{7mm} \vec{r}(t) = \vec{a} + t\cdot{}\vec{u}$}}\end{center}
  
  Dabei ist
  \begin{itemize}
    \item $g$ der Name der Geraden,
    \item $t\in\mathbb{R}$ der beliebig wählbare Parameter,
    \item $\overrightarrow{a_A} = \vec{a} = \overrightarrow{OA}$ der
      \textbf{Ortsvektor} zu einem Referenzpunkt $A$ auf der Geraden,
    \item $\vec{u}$ ein
      \textbf{Richtungsvektor}\index{Richtungsvektor} der Geraden und
    \item $\vec{r}(t)$ die vektorielle Variable; eine mögliche
      Parameterdarstellung der Geraden $g$.
  \end{itemize}
  
\end{definition}

\subsection*{Aufgaben}

\TRAINER{Mögliche Aufgaben:

  Bestimmen Sie 3 Punkte auf einer gegebenen Geraden.

  Prüfen Sie, ob Punkt $P$ auf der Geraden $g$ liegt.

}

\olatLinkArbeitsblatt{Geradengleichung}{https://olat.bms-w.ch/auth/RepositoryEntry/6029786/CourseNode/110262782958559}{Kap. 1
(Aufg 1.-6.)}
%% \TALSAadBMTG{303ff}{2. a) c), 5. a) b) c), 8., 9., 14. a), 35.}
\newpage%
