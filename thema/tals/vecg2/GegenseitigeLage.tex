%% 2020 12 25 ph. g. Freimann
%%
\section{Gegenseitige Lage von Geraden}\index{gegenseitige Lage!von Geraden}

\subsection*{Lernziele}
\begin{itemize}
  \item die Gegenseitige Lage zweier Geraden bestimmen (identisch,
    zusammenfallend
    \index{identisch!Geradengleichung}\index{zusammenfallend!Geradengleichung},
    parallel\index{parallel!Geradengleichung},
    schneidend\index{schneidend!Geradengleichung}, windschief\index{windschief!Geradengleichung})
\end{itemize}

\subsection{Entscheidungshilfe}
Das folgende Diagramm gibt uns eine Entscheidungshilfe, um die
gegenseitige Lage zweier Geraden zu bestimmen:

\vspace{3mm}

\begin{tabular}{p{60mm}|p{50mm}|p{50mm}}
  & Die Richtungsvektoren sind kollinear. &  \textbf{nicht} kollinear\\\hline
Die\,Geraden\,haben mind.\,einen\,gemeinsamen Punkt.  & zusammenfallend, identisch& schneidend\\\hline
\textbf{k}einen\,gemeinsamen Punkt & parallel & windschief
\end{tabular}

\begin{definition}{windschief}{}
  Zwei Geraden stehen \textbf{windschief} (zueinander), wenn sie nicht
  parallel sind und auch keinen gemeinsamen Schnittpunkt aufweisen.
\end{definition}

\begin{bemerkung}{Gemeinsamer Punkt}{}
  Zwei Geraden haben mind. einen gemeinsamen Punkt, wenn auf einer der
  beiden Geraden mind. ein Stützpunkt der anderen Geraden liegt.  
\end{bemerkung}

\begin{bemerkung}{Erst Parallelität prüfen}{}
Tipp: Es lohnt sich ab und zu, zunächst zu prüfen, ob zwei Geraden parallel
sind.

Ist dies nämlich der Fall, so ist für die Identität (zusammenfallende
Geraden) nur noch zu prüfen, ob ein beliebiger Stützpunkt auf der
jeweils anderen Geraden liegt.
\end{bemerkung}

\newpage
\subsection{Referenzaufgabe}
Prüfen Sie die gegenseitige Lage von $g$ und $h$:

$$g:\,\,\, \vec{r}(t) = \Spvek{1;1;0} + t\cdot{} \Spvek{1;0;1} = \TRAINER{\Spvek{1+t;1;t}}$$

$$h:\,\,\, \vec{r}(t) = \Spvek{0;1;1} + s\cdot{} \Spvek{2;2;2} = \TRAINER{\Spvek{2s;1+2s;1+2s}}$$

a) Sind sie parallel (Richtungsvektoren kollinear)?

$$\Spvek{1;0;1} \stackrel{?}{=} k\cdot{} \Spvek{2;2;2}$$
\TNT{2}{
Aus der $y$ Komponente folgt, dass $k=0$ sein muss, und somit stimmen
die $x$-Komponenten nicht überein.
Also sind sie nicht parallel.}%% end TNT

b) Gibt es einen gemeinsamen Schnittpunkt?
Prüfe:
$$\Spvek{1+t;1;t} \stackrel{?}{=} \Spvek{2s;1+2s;1+2s} $$

\TNT{8}{Aus der $y$-Komonente folgt $s=0$. Ergo:
  $$\Spvek{1+t;1;t} \stackrel{?}{=} \Spvek{0;1;1}$$
  Aus der $z$-Komonente folgt $t=1$.
  $$\Spvek{2;1;1} \stackrel{?}{=} \Spvek{0;1;1}$$
  Falsch, somit sind sie nicht schneidend.

  Also \textbf{windschief}!
}

%%Mit folgendem Entscheidungsbaum können wir leicht sehen, ob zwei
%%geraden \textit{identisch}, \textit{parallel}, \textit{schneidend}
%%oder \textit{windschief} sind:

%%\bbwCenterGraphic{18cm}{tals/vecg2/img/WindschiefEntscheidung.png}



\newpage


\subsection*{Aufgaben}


\olatLinkArbeitsblatt{Geradengleichung}{https://olat.bms-w.ch/auth/RepositoryEntry/6029786/CourseNode/110262782958559}{Kap. 2
  (Aufg. 7. - 11.)}


\begin{rezept}{Gegenseitige Lage von Geraden}{}
Gegeben sind zwei Gerade $g:\,\,\, \vec{r_1}(t) = \vec{a} +
t\cdot{}\vec{u}$ und $h:\,\,\, \vec{r_2}(s) = \vec{b} +
s\cdot{}\vec{v}$.

a) Sind die Geraden kollinear? Das heißt: Der eine Richtungsvektor ist
ein Vielfaches des anderen Richtungsvektors.

b1) Falls kollinear:
 Prüfe, ob der Stützvektor der einen Geraden auf der zweiten Geraden
 liegt. Liegt $\vec{a}$ auf $\vec{r_2}(s)$?

 Falls ja, sind die beiden Geraden \textbf{zusammenfallend}, ansonsten
 lediglich \textbf{parallel}.

 b2) Falls nicht kollinear:
 Prüfe, ob es einen gemeinsamen Schnittpunkt gibt:
 Gibt es ein $t$ und ein $s$, sodass $\vec{a} + t\cdot{} \vec{u} =
 \vec{b} + s\cdot{} \vec{v}$?

 Falls ja, sind die geraden \textbf{schneidend}, ansonsten \textbf{windschief}\index{windschief}.


\end{rezept}

%%\TALSAadBMTG{305ff}{25., 27.}
\newpage
