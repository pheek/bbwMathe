%% 2020 12 25 ph. g. Freimann
%%
\section{Gegenseitige Lage von Geraden}

\subsection*{Lernziele}
\begin{itemize}
\item identisch
\item parallel
\item schneidend
\item windschief
\end{itemize}


\subsection{Entscheidungshilfe}
Mit folgendem Entscheidungsbaum können wir leicht sehen, ob zwei
geraden \textit{identisch}, \textit{parallel}, \textit{schneidend}
oder \textit{windschief} sind:

\bbwCenterGraphic{18cm}{tals/vecg2/img/WindschiefEntscheidung.png}


\subsection*{Aufgaben}

\TRAINER{
  mögliche Aufgaben:

  Prüfen Sie, ob zwei Geraden einen gemeinsamen Schnittpunkt haben.
  
  Prüfen Sie, ob zwei Geraden parallel sind.
}

\TALSAadBMTG{305ff}{25., 27.}
