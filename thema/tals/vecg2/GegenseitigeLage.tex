%% 2020 12 25 ph. g. Freimann
%%
\section{Gegenseitige Lage von Geraden}\index{gegenseitige Lage!von Geraden}

\subsection*{Lernziele}
\begin{itemize}
  \item die Gegenseitige Lage zweier Geraden bestimmen (identisch,
    zusammenfallend
    \index{identisch!Geradengleichung}\index{zusammenfallend!Geradengleichung},
    parallel\index{parallel!Geradengleichung},
    schneidend\index{schneidend!Geradengleichung}, windschief\index{windschief!Geradengleichung})
\end{itemize}

\subsection{Entscheidungshilfe}
Das folgende Diagramm gibt uns eine Entscheidungshilfe, um die
gegenseitige Lage zweier Geraden zu bestimmen:

\vspace{3mm}

\begin{tabular}{p{60mm}|p{50mm}|p{50mm}}
  & Die Richtungsvektoren sind kollinear. &  \textbf{nicht} kollinear\\\hline
Die\,Geraden\,haben mind.\,einen\,gemeinsamen Punkt.  & zusammenfallend, identisch& schneidend\\\hline
\textbf{k}einen\,gemeinsamen Punkt & parallel & windschief
\end{tabular}

\begin{definition}{windschief}{}
  Zwei Geraden stehen \textbf{windschief} (zueinander), wenn sie nicht
  parallel sind und auch keinen gemeinsamen Schnittpunkt aufweisen.
\end{definition}

\begin{bemerkung}{Gemeinsamer Punkt}{}
  Zwei Geraden haben mind. einen gemeinsamen Punkt, wenn auf einer der
  beiden Geraden mind. ein Stützpunkt der anderen Geraden liegt.  
\end{bemerkung}

\begin{bemerkung}{Erst Parallelität prüfen}{}
Tipp: Es lohnt sich also, zunächst zu prüfen, ob zwei Geraden parallel
sind.

Ist dies nämlich der Fall, so ist für die Identität (zusammenfallende
Geraden) nur noch zu prüfen, ob ein beliebiger Stützpunkt auf der
jeweils anderen Geraden liegt.
\end{bemerkung}




%%Mit folgendem Entscheidungsbaum können wir leicht sehen, ob zwei
%%geraden \textit{identisch}, \textit{parallel}, \textit{schneidend}
%%oder \textit{windschief} sind:

%%\bbwCenterGraphic{18cm}{tals/vecg2/img/WindschiefEntscheidung.png}



\newpage


\subsection*{Aufgaben}

\TRAINER{
  mögliche Aufgaben:

  Prüfen Sie, ob zwei Geraden einen gemeinsamen Schnittpunkt haben.
  
  Prüfen Sie, ob zwei Geraden parallel sind.
}

\olatLinkArbeitsblatt{Geradengleichung}{https://olat.bms-w.ch/auth/RepositoryEntry/6029786/CourseNode/110262782958559}{Kap. 2
(Aufg. 7. - 11.)}
%%\TALSAadBMTG{305ff}{25., 27.}
\newpage
