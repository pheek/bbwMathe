%% 2020 12 25 ph. g. Freimann
%%
\section{Koordinatenbezogene Vektoren}\index{Vektoren!koordinatenbezogen}

\subsection*{Lernziele}
\begin{itemize}
\item Darstellung im Koordinatensystem
\item Addition und Subtraktion via Koordinaten
\end {itemize}

\subsection{Dreidimensionales Koordinatensystem}

Betrachten wir im dreidimensionalen den Vektor
$$\vec{a} = \Spvek{1;2.5;-2}$$

  \bbwCenterGraphic{7cm}{tals/vecg2/img/dreidimensional.png}

  \begin{bemerkung}{Rechtshändiges System}{}
Gebräuchlich ist das \textbf{rechtshändige} Koordinatensystem, bei dem
die dritte Achse, die $z$-Achse nach oben zeigt, wenn $x$ nach rechts
und $y$ nach hinten zeigt (s. Abbildung).
  \end{bemerkung}



\subsection*{Aufgabe}
\aufgabenFarbe{Geben Sie die obigen Koordinaten in
  \texttt{geogebra.org} ein und betrachten Sie das System aus Sicht
  der $x$-, $y$- und $z$-Achse.}

  \newpage

\subsection{Addition, Subtraktion}
\begin{gesetz}{Vektoraddition}{}
  Vektoren im Dreidimensionalen werden wie im Zweidimensionalen
komponentenweise addiert bzw. subtrahiert:

$$\Spvek{4;7;-2} - \Spvek{2;2.5;-1.1} = 
    \Spvek{\TRAINER{4-2}\noTRAINER{\hspace{20mm}}; \TRAINER{7-2.5} ;
      \TRAINER{-2 - (-1.1)}} = 
    \Spvek{\TRAINER{2}\noTRAINER{\hspace{12mm}} ; \TRAINER{4.5} ; \TRAINER{-0.9}}$$

\end{gesetz}

\subsection*{Aufgaben}
%%\TALSAadBFWG{195}{74. a), 79.}
%%\aufgabenFarbe{Aufgabenblatt Vektorgeometrie. Aufgabe 1. und 2.}

\olatLinkArbeitsblatt{Vecg II Komponenten}{https://olat.bbw.ch/auth/RepositoryEntry/572162090/CourseNode/108600437225393}{1. und
2. }%% END 
