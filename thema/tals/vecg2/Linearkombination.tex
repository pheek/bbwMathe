%% 2020 12 25 ph. g. Freimann
%%
\section{Linearkombination}\index{Linearkombination}

\TadBMTG{240}{16.2.3}

Häufig wollen wir Kräfte, Geschwindigikeiten, kurz Richtungen in
Abhängigkeit von gegebenen Richtungen berechnen.

Wenn zum Beispiel drei Kräfte auf einen Körper wirken, so ist jede
Kraft immer durch eine Kombination der beiden anderen Kräfte
darstellbar.

\begin{definition}{Linearkombination}{}
  Sind $\vec{a\vphantom{b}}$ und $\vec{b}$ gegeben so ist für alle Zahlen $x$ und
  $y$ der Verktor
  $$\vec{c} = x\cdot{}\vec{a\vphantom{b}} + y\cdot{}\vec{b}$$
  eine Linearkombination von $\vec{a\vphantom{b}}$ und $\vec{b}$.

  Dies gilt für jede beliebige Anzahl von Summanden:

  Sind $\vec{a_1}$, $\vec{a_2}$, $\vec{a_3}$, ... gegebene Vektoren.

  Und sind $x_1$, $x_2$, $x_3$, ... beliebige reelle Zahlen, so ist

  $$\vec{b} = x_1\cdot{}\vec{a_1} +  x_2\cdot{}\vec{a_2} +
  x_3\cdot{}\vec{a_3} + ...$$
  eine Linearkombination der gegebenen Vektoren $\vec{a_1}$, $\vec{a_2}$, $\vec{a_3}$, ... 
  \end{definition}


\begin{gesetz}{Linearkombination}{}
Oft wollen wir einen Vektor $\vec{c}$ durch eine Linearkombination von
zwei gegebenen Vektoren ausdrücken.

Gegeben: $\vec{a\vphantom{b}}$, $\vec{b}$ und $\vec{c\vphantom{b}}$

Gesucht $t$ und $s$ in $\mathbb{R}$, sodass

$$\vec{c} = t\cdot{}\vec{a\vphantom{b}} + s\cdot{}\vec{b}$$

\end{gesetz}
\newpage

\begin{beispiel}{Linearkombination}{}
  Es seien $\vec{c} = \left(3 \atop 5\right)$, $\vec{a} = \left(7 \atop
  9\right)$ und $\vec{b} = \left(4 \atop 3\right)$ gegebene Vektoren.

  Nun soll $\vec{c}$ als \textbf{Linearkombination} der Vektoren
  $\vec{a\vphantom{b}}$ und $\vec{b}$ dargestellt werden. Gesucht sind also zwei
  Zahlen $s$ und $t$ mit:

  $$\left(3 \atop 5\right) = s\cdot{} \left(7 \atop 9\right) +
  t\cdot{} \left(4 \atop 3\right)$$

  Dies entspricht einem linearen Gleichungssystem:
  \gleichungZZ{3}{7s + 4t}{5}{9s + 3t}
\end{beispiel}

\begin{bemerkung}{Linearkombination}{}
  Sind $\vec{a\vphantom{b}}$ und $\vec{b}$ zwei \textbf{kolineare} Vektoren
  (d.\,h. $\vec{a\vphantom{b}} = x\cdot{}\vec{b}$), so kann der Vektor $\vec{c}$
  nicht, oder \textbf{nicht eindeutig als Linearkombination} von $\vec{a\vphantom{b}}$ und
  $\vec{b}$ geschrieben werden.

  Dies entspricht bei linearen Funktionen der Suche nach dem
  Schnittpunkt von zwei \textbf{parallelen} Geraden.
  \end{bemerkung}

\newpage
