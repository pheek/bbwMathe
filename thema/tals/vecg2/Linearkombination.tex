%% 2020 12 25 ph. g. Freimann
%%
\section{Linearkombination}\index{Linearkombination}

\TadBMTG{240}{16.2.3}

Häufig wollen wir Kräfte, Geschwindigkeiten, kurz Richtungen in
Abhängigkeit von gegebenen Richtungen berechnen.

Wenn zum Beispiel drei Kräfte in einer Ebene auf einen Körper wirken, so ist jede
Kraft i.\,d.\,R durch eine Kombination der beiden anderen Kräfte
darstellbar.

\begin{definition}{Linearkombination}{}
 %%  Sind $\vec{a\vphantom{b}}$ und $\vec{b}$ gegeben so ist für alle Zahlen $x$ und
 %%  $y$ der Verktor
 %%  $$\vec{c} = x\cdot{}\vec{a\vphantom{b}} + y\cdot{}\vec{b}$$
 %%  eine Linearkombination von $\vec{a\vphantom{b}}$ und $\vec{b}$.

 %% gilt für jede beliebige Anzahl von Summanden:
 
  Sind $\vec{a_1}$, $\vec{a_2}$, $\vec{a_3}$, ... gegebene Vektoren.

  Und sind $x_1$, $x_2$, $x_3$, ... beliebige reelle Zahlen, so ist

  $$\vec{b} = x_1\cdot{}\vec{a_1} +  x_2\cdot{}\vec{a_2} +
  x_3\cdot{}\vec{a_3} + ...$$
  eine Linearkombination der gegebenen Vektoren $\vec{a_1}$, $\vec{a_2}$, $\vec{a_3}$, ... 
\end{definition}


\begin{beispiel}{Linearkombination}{}
Oft wollen wir einen Vektor $\vec{c}$ durch eine Linearkombination von
gegebenen Vektoren ausdrücken.

Gegeben: $\vec{a\vphantom{b}}$, $\vec{b}$ und $\vec{c\vphantom{b}}$

Gesucht $t$ und $s$ in $\mathbb{R}$, sodass

$$\vec{c} = t\cdot{}\vec{a\vphantom{b}} + s\cdot{}\vec{b}$$
gilt.

\end{beispiel}

\newpage

\begin{beispiel}{Linearkombination}{}
  Es seien $\vec{c} = \left(3 \atop 5\right)$, $\vec{a} = \left(7 \atop
  9\right)$ und $\vec{b} = \left(4 \atop 3\right)$ gegebene Vektoren.

  Nun soll $\vec{c}$ als \textbf{Linearkombination} der Vektoren
  $\vec{a\vphantom{b}}$ und $\vec{b}$ dargestellt werden. Gesucht sind also zwei
  Zahlen $s$ und $t$ mit:

  $$\left(3 \atop 5\right) = s\cdot{} \left(7 \atop 9\right) +
  t\cdot{} \left(4 \atop 3\right)$$

  Dies entspricht einem linearen Gleichungssystem:
  \gleichungZZ{3}{7s + 4t}{5}{9s + 3t}

  Für welches Paar $(s, t)$ ist dies möglich?
  
  \TNT{4}{solve mit TR, am besten gleich in der Vektor-Schreibweise}%%
  
  Die Kombination ist möglich, wenn $s=\LoesungsRaum{\frac{11}{15}}$ und\\
  $t=\LoesungsRaum{-\frac{8}{15}}$ gesetzt wird.
\end{beispiel}

\begin{bemerkung}{Linearkombination}{}
  Sind $\vec{a\vphantom{b}}$ und $\vec{b}$ zwei \textbf{kollineare} Vektoren
  (d.\,h. $\vec{a\vphantom{b}} = x\cdot{}\vec{b}$), so kann der Vektor $\vec{c}$
  nicht, oder \textbf{nicht} eindeutig als \textbf{Linearkombination} von $\vec{a\vphantom{b}}$ und
  $\vec{b}$ geschrieben werden.

  Dies entspricht bei linearen Funktionen der Suche nach dem
  Schnittpunkt von zwei \textbf{parallelen} Geraden.
  \end{bemerkung}


\subsection*{Aufgaben}
%%\TALSAadBFWG{196}{82. b)}
\olatLinkArbeitsblatt{Vecg II
  Komponenten}{https://olat.bms-w.ch/auth/RepositoryEntry/6029786/CourseNode/108600437225393}{22.,
  24. und 25.}%% END 

\newpage
\subsection{Lineare Abhängigkeit}\index{linear abhängig}\index{Abhängigkeit!lineare}

Ich brauche zwei Vektoren, um eine Ebene und drei Vektoren, um der
Raum aufzuspannen.


\begin{bemerkung}{Drei Dimensionen}{}
  Die drei Einheitsvektoren
  $$\vec{e}_1=\Spvek{1;0;0} ;
  \vec{e}_2=\Spvek{0;1;0} \text{ und }
  \vec{e}_3=\Spvek{0;0;1}$$
spannen den dreidimensionalen Raum derart auf, dass jeder weitere
erdenkliche Vektor als
Linearkombination aus diesen drei Vektoren geschrieben werden kann:

$$\Spvek{7;3;-1.4} = 
  \LoesungsRaum{  7   }\cdot{}\Spvek{1;0;0} + 
  \LoesungsRaum{  3   }\cdot{}\Spvek{0;1;0} +
  \LoesungsRaum{(-1.4)}\cdot{}\Spvek{0;0;1}$$
\end{bemerkung}

\begin{gesetz}{Einheitsvektoren}{}
  Die drei Einheitsvektoren sind voneinander linear unabhängig, d.\,h.:

  Es ist nicht möglich, einen der drei Einheitsvektoren aus einer
  Linearkombination der anderen beiden Einheitsvektoren darzustellen.
\end{gesetz}

\begin{definition}{Linear abhängig}{}
  Ein Vektor $\vec{b}\ne \vec{0}$ heißt von den Vektoren $\vec{a}_1, \vec{a}_2,
  \vec{a}_3, ... , \vec{a}_n$ \textbf{linear abhängig}, wenn es eine
  Linearkombination mit $x_i \in \mathbb{R}$ gibt, für die gilt:

  $$\vec{b} = \sum_{i=1}^{n}x_i\cdot{}\vec{a}_i$$
  \end{definition}
\newpage
\subsubsection{Beispiele}%%\index{unabhängig!linear}

\begin{beispiel}{Linear abhängig}{}
  Der Vektor $\Spvek{-12;-13;29}$ ist von den drei Vektoren 
    Vektoren $\Spvek{1;2;7}$, $\Spvek{2;1;9}$
    und $\Spvek{5;5;3}$  linear abhängig, denn
      er kann als Linearkombination dieser drei
      Vektoren geschrieben werden:
      $$\Spvek{-12;-13;29} = 2\cdot{}\Spvek{1;2;7} + 3\cdot{}
      \Spvek{2;1;9} -4\cdot{} \Spvek{5;5;3}$$
\end{beispiel}

\begin{beispiel}{Linear unabhängig}{}
  Der Vektor $\Spvek{4;5;6}$ ist von den beiden
    Vektoren $\Spvek{1;0;0}$
    und $\Spvek{0;1;0}$  linear \textbf{un}abhängig, denn
      er kann \textbf{nicht} als Linearkombination dieser beiden
      Einheitsvektoren geschrieben werden.
\end{beispiel}

\subsection*{Aufgaben}
%%\TALSAadBFWG{196}{ 80. a)}
\olatLinkArbeitsblatt{Vecg II
  Komponenten}{https://olat.bms-w.ch/auth/RepositoryEntry/6029786/CourseNode/108600437225393}{26. bis
  29.}%% end
