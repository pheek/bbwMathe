%% 2020 12 25 ph. g. Freimann
%%
\section{Abstände}

\subsection*{Lernziele}
\begin{itemize}
\item Abstand eines Punktes zu einer Geraden
\item Abstand zwischen zwei parallelen Geraden im Raum
\item Abstand zwischen zwei windschiefen Geraden im Raum
\end{itemize}

\TALSTadBMTG{298}{19.6}
\newpage

\subsection{Anwendung Skalarprodukt und Parametergleichung}

Eine klassische Anwendung des Skalarproduktes ist die
Abstandsbestimmung zwischen einer Geraden und einem Punkt oder
allgemein zwischen zwei (windschiefen) Geraden im Raum.
\newpage

\subsubsection{Abstand Gerade / Punkt}\index{Abstand!Gerade/Punkt}

\begin{beispiel}{Abstand}{}
Gegeben ist der Ortsvektor $\vec{a}=\left(1 \atop 2\right)$, der einen Punkt auf der
Geraden $g$ repräsentiert und ein Richtungsvektor
$\vec{u}=\left(3\atop 1\right)$, der
die Richtung der Geraden angibt.

Des weiteren ist ein Punkt $P=(8|1)$ gegeben.

Gesucht ist der Abstand des Punktes $P$ zur Geraden $g$, was einem Vektor
$\vec{s}$ gleichkommt, der senkrecht zur Geraden $g$ steht und dessen Länge
$s=|\vec{s}|$ gerade den Abstand zwischen Gerade und Punkt angibt.
\end{beispiel}

\TNTeop{Der Abstand wird senkrecht gemessen. Somit existiert ein
  Punkt $Q$ auf $g$, sodass $\overrightarrow{QP}$ senkrecht zum
  Richtungsvektor $\vec{u}$ steht.
  
  Zum Begriff senkrecht machen wir uns das Skalarprodukt zu nutze.

  Ich bezeichne $\vec{s} := \overrightarrow{QP}$ und $\vec{p} = \overrightarrow{OP}$.

  Somit gelten:

  $$\vec{a} + t\cdot{}\vec{u} + \vec{s} = \vec{p}$$
  $$\vec{s} \perp \vec{u} \Longleftrightarrow \vec{s}\circ\vec{u} = 0$$
  Dies lösen wir entweder mit dem Taschenrechner, oder wir stellen das
  Gleichungssystem komponentenweise auf (3. Gleichung: Skalarprodukt=0):

  $$\vec{s} = \vec{p} - (\vec{a} + t\cdot{}\vec{u})$$

  \gleichungDD{s_x}{8-3t-1}{s_y}{1-1t-2}{3s_x+1s_y}{0}
  Mit $s_y=-3s_x$ ergibt sich
  \gleichungZZ{s_x}{7-3t}{-3s_x}{-1-t}
  Daraus folgt $\vec{s}=\left(1\atop -3\right)$ und somit ist der
  Abstand $s=\sqrt{10}$. (Bem.: $t=2$)
}

\subsection*{Aufgaben}
%%\AadBMTG{305}{18. a) b) c) d), 19., 20. und 21.}
\olatLinkArbeitsblatt{Geradengleichung}{https://olat.bms-w.ch/auth/RepositoryEntry/6029786/CourseNode/110262782958559}{Kap. 3.1 (Aufg. 12. - 15.)}
\newpage


\subsubsection{Abstand windschiefer Geraden}

Gegeben sind zwei windschiefe Geraden in der Parameter-Schreibweise:

$$g:\,\,\,  \vec{r}(t) = \vec{a}+t\cdot{}\vec{u}$$

Dabei ist $\vec{a} = \overrightarrow{OA}$ ein Stützvektor der Geraden
und $\vec{u}$ ein Richtungsvektor.

Nennen wir unsere zwei Geraden $g_1$ und $g_2$:
$$g_1:\,\,\,\, \vec{r}_1(t) =  \vec{a}_1+t\cdot{}\vec{u}_1 \text{ mit } \vec{a}_1
= \Spvek{3;2;1} \text{ und } \vec{u}_1= \Spvek{2;2;0}$$
$$g_2:\,\,\,\,  \vec{r}_2(t)\vec{a}_2+s\cdot{}\vec{u}_2 \text{ mit } \vec{a}_2
= \Spvek{7;2;0} \text{ und } \vec{u}_2= \Spvek{1;0;1}$$

\begin{rezept}{Grundidee Abstand}{}

  1. Der Abstandsvektor $\vec{d}$ stehe senkrecht auf $\vec{u}_1$
  \textbf{und} senkrecht auf $\vec{u}_2$.

  2. Der Abstandsvektor $\vec{d}$ (distance) zeigt von einem Punkt der einen Geraden auf
  einen Punkt der zweiten Geraden: $$\vec{d}=\vec{r}_1(t) - \vec{r}_2(s)$$

  3. Dies ergibt uns drei Gleichungen:

  $$\vec{d}\perp\vec{u}_1 \Longleftrightarrow{}  \vec{d}\circ{}\vec{u}_1 = 0$$
  $$\vec{d}\perp\vec{u}_2 \Longleftrightarrow{}  \vec{d}\circ{}\vec{u}_2 = 0$$
  $$\vec{d}=\vec{r}_1(t)-\vec{r}_2(s)$$

  4. Der Abstand ist dann durch die \textbf{Länge} (Norm) des Vektors gegeben:
  $$\text{len} = \text{norm}(\vec{d})$$
  
\end{rezept}
\newpage


Lösung der Abstands-Aufgabe zwischen zwei windschiefen Geraden mit dem
Taschenrechner:
\TRAINER{}

\bbwCenterGraphic{12cm}{tals/vecg2/img/abstandTR.png}

$$d = |\vec{d} | = \frac{5\sqrt{3}}{3} \approx 2.88675$$
\newpage
\subsubsection*{Referenzaufgabe von Hand}

Gegeben sind die Geraden
%% beispiel von hier: https://www.mathelike.de/abiturskript-mathematik-bayern/2-geometrie/2.4-abstandsbestimmungen/2.4.3-abstand-windschiefer-geraden.html
$$g: \,\,\, \vec{r}(t) = \Spvek{2;6;2} + t\cdot{}\Spvek{0;1;0} \text{
  und }  h: \,\,\, \vec{r}(t) = \Spvek{6;-2;8} + s\cdot{}\Spvek{-3;1;0} $$

\TNTeop{
  1. $P$ und $Q$ auf $g$ und $h$ wählen:
  $$P = \Spvek{2;6+ t;2} \text{ und } Q = \Spvek{6-3s;-2+s;8} $$

  2. $\overrightarrow{PQ}$ bestimmen:
  $$\overrightarrow{PQ} = \Spvek{(2)-(6-3s); (6+t)-(-2+s); (2)-(8)} =
  \Spvek{3s-4; 8+t-s; -6}$$

  3. $\overrightarrow{PQ}$  senkrecht auf beide Richtungsvektoren
  stellen (je Skalarprodukt = Null setzen):
  $$(3s-4)\cdot{}0     + (8+t-s)\cdot{}1+ (-6)\cdot{} 0 = 0 \Longleftrightarrow t-s=-8$$
  und 
  $$(3s-4)\cdot{}(-3)  + (8+t-s)\cdot{}1+ (-6)\cdot{} 0 = 0 \Longleftrightarrow -10s + t = -20$$

  4. Gleichungssystem nach $s$ und $t$ auflösen:
  $$s=\frac43 \text{ und } t = \frac{-20}3$$
  
  5. $s$ und $t$ in  $\overrightarrow{PQ}$  einsetzen:
  $\overrightarrow{PQ} = \Spvek{3\cdot{}\frac43 -4 ; 8+\frac{-20}3
    -\frac43 ; -6} = \Spvek{0;0;-6}$

  6. Länge (mit Pythagoras) besitmmen
  $$d = \sqrt{0^2 + 0^2 + (-6)^2}  = 6$$
}%% end TNTeop

\newpage
\begin{rezept}{Abstand von Hand}{}
  \begin{enumerate}
  \item $P$ und $Q$ auf $g$ und $h$ wählen (Ansatz $s$ und $t$ bekannt)
  \item $\overrightarrow{PQ}$ bestimmen in Abhängikgkeit von $s$ und $t$.
  \item $\overrightarrow{PQ}$  senkrecht auf beide Richtungsvektoren
  stellen (je Skalarprodukt = Null setzen)
  \item Gleichungssystem nach $s$ und $t$ auflösen:
  \item $s$ und $t$ in  $\overrightarrow{PQ}$  einsetzen:
  \item Länge (Norm) bestimmen (Pythagoras)
  \end{enumerate}
\end{rezept}

\newpage

\subsection*{Aufgaben}

\olatLinkArbeitsblatt{Geradengleichung}{https://olat.bms-w.ch/auth/RepositoryEntry/6029786/CourseNode/110262782958559}{Kap. 3.2 (Aufg. 16.)}

\TALSAadBMTG{307ff}{35. und Flugzeug: 38. bis 43.}

