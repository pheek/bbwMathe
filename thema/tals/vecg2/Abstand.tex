%% 2020 12 25 ph. g. Freimann
%%
\section{Abstände}

\subsection*{Lernziele}
\begin{itemize}
\item Abstände von Geraden im Raum
\end{itemize}

\subsection{Anwendung Skalarprodukt und Parametergleichnug}

Eine klassische Anwendung des Skalarproduktes ist die
Abstandsbestimmung zwischen einer Geraden und einem Punkt oder
allgemein zwischen zwei Geraden im Raum.

\begin{beispiel}{Abstand}{}
Gegeben ist der Ortsvektor $\vec{r_A}=\left(1 \atop 2\right)$, der einen Punkt auf der
Geraden $\vec{r}$ repräsentiert und ein Richtungsvektor
$\vec{u}=\left(3\atop 1\right)$, der
die Richtung der Geraden angibt.

Des weiteren ist ein Punkt $P=(8|1)$ gegeben. Gesucht ist nun ein Vektor
$\vec{a}$ der senkrecht zur Geraden $\vec{r}$ steht und dessen Länge
$|\vec{a}|$ gerade den Abstand zwischen Gerade und Punkt angibt.
\end{beispiel}


\TALSTadBMTG{298}{19.6}

\TALSAadBMTG{308}{38. bis 43.}
