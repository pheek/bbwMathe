%% 2020 12 25 ph. g. Freimann
%%
\section{Abstände}

\subsection*{Lernziele}
\begin{itemize}
  \item Abstand eines Punktes zu einer Geraden
\item Abstand zwischen zwei parelleln Geraden im Raum
\item Abstand zwischen zwei windschiefen Geraden im Raum
\end{itemize}

\TALSTadBMTG{298}{19.6}
\newpage

\subsection{Anwendung Skalarprodukt und Parametergleichnug}

Eine klassische Anwendung des Skalarproduktes ist die
Abstandsbestimmung zwischen einer Geraden und einem Punkt oder
allgemein zwischen zwei (windschiefen) Geraden im Raum.
\newpage

\subsubsection{Abstand Gerade / Punkt}\index{Abstand!Gerade/Punkt}

\begin{beispiel}{Abstand}{}
Gegeben ist der Ortsvektor $\vec{r}=\left(1 \atop 2\right)$, der einen Punkt auf der
Geraden $g$ repräsentiert und ein Richtungsvektor
$\vec{a}=\left(3\atop 1\right)$, der
die Richtung der Geraden angibt.

Des weiteren ist ein Punkt $P=(8|1)$ gegeben.

Gesucht ist nun ein Vektor
$\vec{s}$ der senkrecht zur Geraden $g$ steht und dessen Länge
$s=|\vec{s}|$ gerade den Abstand zwischen Gerade und Punkt angibt.
\end{beispiel}

\TNTeop{ Der Abstand wird senkrecht gemessen. Somit existiert ein
  Punkt $Q$ auf $g$, sodass $\vec{QP}$ senkrecht zu $\vec{a}$ steht.
  
  Zum Begriff senkrecht machen wir uns das Skalarprodukt zu nutze.

  Ich bezeichne $\vec{s} := \overrightarrow{QP}$ und $\vec{p} = \overrightarrow{OP}$.

  Somit gelten:

  $$\vec{r} + t\cdot{}\vec{a} + \vec{s} = \vec{p}$$
  $$\vec{s} \perp \vec{a} \Longleftrightarrow \vec{s}\circ\vec{a} =
  0$$
  Dies lösen wir entweder mit dem Taschenrechner, oder wir stellen das
  Gleichungssystem komponentenweise auf:

  $$\vec{s} = \vec{p} - t\cdot{}\vec{a} - \vec{r}$$
  

  \gleichungDD{s_x}{8-3t-1}{s_y}{1-1t-2}{3s_x+1s_y}{0}
  Mit $s_y=-3s_x$ ergibt sich
  \gleichungZZ{s_x}{7-3t}{-3s_x}{-1-t}
  Daraus folgt $\vec{s}=\left(1\atop -3\right)$ und somit ist der
  Abstand $s=\sqrt{10}$. (Bem.: $t=2$)
}

\subsection*{Aufgaben}
\AadBMTG{305}{18. a) b) c) d), 19., 20. und 21.}
\newpage


\subsubsection{Abstand windschiefe Geraden}

Gegeben sind zwei windschiefe Geraden in der Parameter-Schreibweise:

$$g:  S+t\cdot{}\vec{r}$$

Dabei ist $\vec{s} = \overrightarrow{OS}$ der Stützvektor der Geraden
und $\vec{r}$ der Richtungsvektor.

Nennen wir unsere zwei Geraden $g_1$ und $g_2$:
$$g_1:  S_1+t\cdot{}\vec{r}_1 \text{ mit } S_1
= \begin{pmatrix}3\\2\\1\end{pmatrix} \text{ und } \vec{r}_1= \begin{pmatrix}2\\2\\0\end{pmatrix}$$
$$g_2:  S_1+s\cdot{}\vec{r}_2 \text{ mit } S_2
= \begin{pmatrix}2\\2\\0\end{pmatrix} \text{ und } \vec{r}_2= \begin{pmatrix}1\\0\\1\end{pmatrix}$$

\begin{rezept}{Grundidee Abstand}{}

  1. Der Abstandsvektor $\vec{d}$ stehe senkrecht auf $\vec{r}_1$
  \textbf{und} senkrecht auf $\vec{r}_2$.

  2. Der Abstandsvektor zeigt von einem Punkt der einen Geraden auf
  einen Punkt der zweiten Geraden: $$\vec{d}=g_1(t)-g_2(s)$$

  3. Dies ergibt uns drei Gleichungen:

  $$\vec{d}\perp\vec{r}_1 \Longleftrightarrow{}  \vec{d}\circ{}\vec{r}_1 = 0$$
  $$\vec{d}\perp\vec{r}_2 \Longleftrightarrow{}  \vec{d}\circ{}\vec{r}_2 = 0$$
  $$\text{len} = \text{norm}(\vec{d})$$
  
\end{rezept}

Auf der nächsten Seite zeige ich, wie dies im Taschenrechner direkt
eingegeben werden kann:
\newpage

Lösung der Abstands-Aufgabe zwischen zwei windschiefen Geraden mit dem
Taschenrechner:

\bbwCenterGraphic{12cm}{tals/vecg2/img/abstandTR.png}

$$|\vec{d}| = \frac{5\sqrt{3}}{3} \approx 2.88675$$
\newpage

\TALSAadBMTG{307ff}{35. und Flugzeug: 38. bis 43.}
