%% 2020 12 25 ph. g. Freimann
%%
\section{Norm}

\subsection*{Lernziele}
\begin{itemize}
\item Vektoren Normieren
\end{itemize}

\begin{gesetz}{Vektor Normieren}{}
  Sei $\vec{a}$ ein Vektor der Länge $a$.
  Somit ist $\frac1{|a|} = \frac1{|\vec{a}|}$ der Kehrwert der Länge
  des Vektors.

  Der Vektor $$\vec{n} := \frac1{|\vec{a}|}\cdot{}\vec{a}$$ hat somit
  die Länge 1 und ist der \textit{normierte} Vektor zu $\vec{a}$.
  \end{gesetz}

\begin{bemerkung}{Nullvektor}{}
  Zum Nullvektor $\vec{0}$ gibt es keinen normierten Vektor.
  \end{bemerkung}
