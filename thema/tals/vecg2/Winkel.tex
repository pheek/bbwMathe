%% 2020 12 25 ph. g. Freimann
%%
\section{Winkel}\index{Winkel!bei Vektoren}

\subsection*{Lernziele}
\begin{itemize}
\item Skalarprodukt
\end{itemize}

\subsection{Skalarprodukt}\index{Skalarprodukt}

%%\TALSTadBFWG{201}{4.5}
\TALSTadBMTG{266}{18}
\bbwCenterGraphic{5cm}{tals/vecg2/img/skalarprodukt.png}

Herleitung:

\TNT{12}{
  $\vec{b} + \vec{c} = \vec{a} \Longrightarrow   \vec{c} = \vec{a} - \vec{b}$

  $\Longrightarrow   c_x = a_x - b_x \text{ und } c_y = a_y - b_y$

  $a = |\vec{a}|, b=|\vec{b}| \text{ und } c = |\vec{c}|$

  Pythagoras:
  $c = \sqrt{c_x^2 + c_y^2}$ und somit $c^2 = c_x^2 + c_y^2$

  $a_x-b_x$ anstelle von $c_x$ einsetzen (analog $c_y=a_y-b_y$):

  $c^2 = (a_x-b_x)^2 + (a_y - b_y)^2$

  Wir erinnern an den Cosinussatz: $c^2 = a^2 + b^2 -
  2ab\cdot{}\cos(\gamma)$

  Setzen wir beide Gleichungen gleich (via $c^2$), so erhalten wir:

  $(a_x-b_x)^2 + (a_y-b_y)^2 = a^2 + b^2 - 2ab\cdot{}\cos(\gamma)$

  $(a_x-b_x)^2 + (a_y-b_y)^2 = \overbrace{a_x^2 + a_y^2}^{|\vec{a}|^2=a^2}
  + \overbrace{b_x^2 + b_y^2}^{b^2} - 2ab\cdot{}\cos(\gamma)$

  jetzt links ausmultiplizieren:
  
  $a_x^2 - 2a_xb_x +b_x^2 + a_y^2 - 2a_yb_y + b_y^2 = a_x^2 + a_y^2 +
  b_x^2 + b_y^2 - 2ab\cdot{}\cos(\gamma)$

  nun beginnt der Spaß mit dem Wegstreichen:

  $-2a_xb_x - 2a_yb_y = -2ab\cdot{}\cos(\gamma)$

  und somit:
  }
\newpage
\subsubsection{Gesetze}
\begin{gesetz}{}{}
  Mit $a = |\vec{a}|$ und $b = |\vec{b}|$ gilt:\\
  $a_x\cdot{}b_x + a_y\cdot{}b_y = a\cdot{}b\cdot{}\cos(\gamma)$
\end{gesetz}
 

Nach obigem Gesetz sind die beiden folgenden Definitionen identisch:
\begin{definition}{Skalarprodukt}{}\index{Skalarprodukt}
  $$\vec{a\vphantom{b}}\circ\vec{b} := a_x\cdot{}b_x + a_y \cdot{} b_y$$
oder mit $a=|\vec{a}|$, $b=|\vec{b}|$ und $\gamma$ = Zwischenwinkel:
  $$\vec{a\vphantom{b}}\circ\vec{b} := a\cdot{}b\cdot{}\cos({\gamma})$$
\end{definition}


\begin{gesetz}{Winkel und Skalarprodukt}{}\index{Skalarprodukt!Zwischenwinkel}
  $$\cos(\gamma) = \frac{\vec{a\vphantom{b}}\circ\vec{b}}{a\cdot b}$$
  bzw.:
  $$\gamma = \arccos{} \left(\frac{\vec{a\vphantom{b}}\circ\vec{b}}{a\cdot b}\right)$$
  
\end{gesetz}

\begin{bemerkung}{}{}
  Genau dann, wenn zwei Vektoren $\vec{a}$ und $\vec{b}$ senkrecht
  zueinander stehen, ist das Skalarprodukt null.

  $$\gamma=90\degre   \Longleftrightarrow  \vec{a\vphantom{b}}\circ\vec{b}=0$$
  \end{bemerkung}


\subsection*{Aufgaben}
%%\TALSAadBFWG{202}{102. a) d) e), 103. a) c), 104. a) c), 106. a),
%%107. a), 108., 109., 110. a), 113.}

\TALSAadBMTG{282ff}{1. a) c) e), 2. a) c) e), 4. a), 7. a) b) c), 8.,
  10. d), 20.}

\newpage
