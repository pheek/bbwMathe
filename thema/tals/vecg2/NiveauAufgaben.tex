\section{Niveau Aufgaben}


\subsection{Referenzaufgabe}

(Maturaprüfng 2022 Aufg. 4 Teil 2 mit TR)

Gegeben sind die Gerade $g$ und die Punkte $A$ und $B$.

$$g : \vec{r} = \Spvek{2;1;1} + t\cdot{}\Spvek{0;2;1}\hspace{20mm} A=(6|4| 1), B=(-2|5|7)$$

Berechnen Sie mit dem Taschenrechner die Punkte $P$ der Geraden $g$, für die gilt: $\angle
APB = 35\degre$.


\TNTeop{
  Lösung Maturaprüfung 2022 Aufgabe 4 Teil 2 mit Taschenrechner:

Ansatz: $P=(2|1+2t|1+s)$:

$$\overrightarrow{PA} = \Spvek{4;3-2t;-t}; \overrightarrow{PB}
= \Spvek{-4;4-2t;6-t}$$
Solve:
$$\frac{\overrightarrow{PA}\circ{}\overrightarrow{PB}}{|\overrightarrow{PA}|\cdot{}|\overrightarrow{PB}|}
= \cos(35\degre)$$

Lösung
$$t_1 \approx -4.64 \text{ und } t_2\approx 8.80$$
$$P_1 \approx (2 | -8.29 | -3.64) \text{ und } P_2\approx (2|18.6|9.80)$$
}%% end TNTeop



\subsection{Vermischte Aufgaben}
Aufgaben Vektorgeometrie 2 mit Taschenrechner:

\olatLinkTALSStrukturaufgabenSPF{Vektorgeometrie Teil 2}{17ff}{55. bis 61.}

\olatLinkTALSStrukturaufgabenSPF{Vektorgeometrie Teil 2 Typ 6:}{21}{71. bis 75.}

