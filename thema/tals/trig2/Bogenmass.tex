%%
%% 2019 07 04 Ph. G. Freimann
%%

\subsection{Grad- und Bogenmaß}\index{Gradmaß}\index{Bogenmaß}

%%\TALSTadBFWG{46}{1.3.2}
\TadBMTG{87}{6.1}

Das Bogenmaß ist eine alternative Einteilung des Kreises zum
klassischen $360\degre$-Gradmaß. Dabei wird der volle Kreis nicht in
$360\degre$ Grad sondern im Verhältnis zum Radius dargestellt. Mit
anderen Worten: Bei gegebenem Winkel $\alpha$ ist das Bogenmaß nichts
anderes als die Länge des Bogens zum Winkel $\alpha$ gemessen im Einheitskreis\index{Einheitskreis}.

\bbwGraphic{5cm}{tals/trig2/img/bogenmass.png}\hspace{20mm}entspricht \hspace{20mm}\bbwGraphic{7cm}{tals/trig2/img/gradmass.png}


\begin{bbwFillInTabular}{c|c}\hline
  Bogenlänge                                  &  Winkel  \\\hline
  $b = 2\pi$                                  & $\beta = \LoesungsRaumLen{30mm}{360\degre}$ \\\hline
  $b = \frac{\pi}{2}$                         &  $\beta = \LoesungsRaumLen{30mm}{ 90\degre}$ \\\hline
  $b = \frac{\pi}{3}$                         & $\beta = \LoesungsRaumLen{30mm}{ 60\degre}$ \\\hline
  $b = \LoesungsRaumLen{30mm}{\frac{\pi}{4}}$ & $\beta = 45\degre$ \\\hline
  \end{bbwFillInTabular}


\newpage

\begin{definition}{Bogenmaß}{}
$$360\degre \entspricht 2\cdot\pi \text{\,\,rad}$$
\end{definition}

\begin{bemerkung}{rad}{}
Der Winkel im Bogenmaß wird nicht in der Maßeinheit Grad (${}\degre$)
sondern in $\text{rad}$ angegeben.
\end{bemerkung}


\begin{beispiel}{Bogenmaß}{}

$$30\degre = \frac{30\degre}{360\degre}\cdot2\pi = \frac{30\degre\cdot{}\pi}{180\degre} = \frac{1}{6} \text{ rad}$$
$$\frac{2}{3}\pi \text{ rad} = \frac{2}{3} \pi \cdot{} \frac{360\degre}{2\pi} = \LoesungsRaumLen{80mm}{\frac{2}{3}\pi \cdot{} \frac{180\degre}{\pi} = \frac{2}{3}\cdot{}180\degre = 120\degre}$$
  $$1 \text{ rad } = \LoesungsRaumLen{40mm}{\frac{180}{\pi}\degre \approx 57.30\degre}$$
$$1\degre = \LoesungsRaumLen{50mm}{\frac{\pi}{180} \text{ rad } \approx 0.01745 \text{ rad }}$$
\end{beispiel}

  
\begin{gesetz}{Bogenmaß}{}

  Umwandeln ins Bogenmaß:
  $$\alpha\degre = \left(\frac{\alpha\pi}{180}\right) \text{ rad }$$
  Umwandeln ins Gradmaß:
$$\stackrel{\frown}{\alpha} \text{rad} = \left(\frac{\alpha\cdot{}180}{\pi}\right)^\circ$$
\end{gesetz}

\newpage


\begin{beispiel}{Bogenmaß}{}
 Zum Beispiel entspricht
$\frac{3}{4}\pi\,\text{rad}$ unseren bekannten $\LoesungsRaum{135}\degre$ in der
(bereits sumerischen/babylonischen) $360\degre$-Einteilung.

 $$\frac34\pi \text{ rad } = \LoesungsRaumLang{\left(\frac{\frac34\pi\cdot{}180}{\pi}\right)^\circ} = \LoesungsRaum{135}\degre$$
\end{beispiel}

\begin{bemerkung}{}{}
  Sind bei Winkelfunktionen $\cos$, $\sin$ oder $\tan$ Winkel gegeben,
  so müssen wir immer zuerst überlegen, ob wir die Größen im Grad-
  oder im Bogenmaß vor uns haben.
  Eine Aufgabe im Stil
  $$0.75 = \sin(x + 2\pi + 40\degre + 1.5)$$
  \textbf{ist nicht sinnvoll}, da bei $1.5$ nicht klar ist, ob es sich nun um eine Zahl im Grad- oder im Bogenmaß handelt.
\end{bemerkung}%%


\subsection*{Aufgaben}

\olatLinkArbeitsblatt{Bogenmaß}{https://olat.bms-w.ch/auth/RepositoryEntry/6029786/CourseNode/103430829669712}{1.1 bis 1.5}
\newpage


\subsection{Bogenlänge und Fläche}
Die Bogenlänge ist proportional zum Radius.

Daher kann die Bogenänge (und auch die Sektorfläche) mit dem Bogenmaß einfacher berechnet werden als im Gradmaß.

\begin{gesetz}{Bogenlänge und Sektorfläche}{}
  Ist von einem Sektor der Radius $r$ und der Sektorwinkel $\stackrel{\frown}{\alpha}$ im Bogenmaß gegeben, so lassen sich Bogenlänge $b$ und Sektorfläche $A$ wie folgt berechnen:

  Bogenlänge $b$: 
  $$b = \LoesungsRaumLen{50mm}{r \cdot{} \stackrel{\frown}{\alpha}}$$

  Sektorfläche $A$:
  $$A = \LoesungsRaumLen{50mm}{\frac12 \cdot{} r\cdot{} b = \frac{1}{2} \cdot{} r^2 \stackrel{\frown}{\alpha}}$$
\end{gesetz}

Beispiel:

Berechnen Sie Bogenlänge $b$ und Fläche $A$ eines Kreissektors mit Radius $r = 1.4 \text{ cm }$ und Sektorwinkel $\stackrel{\frown}{\alpha} = 0.36$ (im Bogenmaß).


\TNTeop{
  Bogenlänge:

  $b = r \cdot{} \stackrel{\frown}{\alpha} = 1.4 \cdot{} 0.36 = 0.504 \text{ cm} $

  Fläche

  $A = \frac12 \cdot{} r^2\stackrel{\frown}{\alpha} = \frac12 \cdot{}1.4^2 \cdot{} 0.36 = 0.3528 \text{ cm}^2$
}%% end TNT eop

%%%%%%%%%%%%%%%%%%%%%%%%%%%%%%%%%%%%%%%%%%%%%%%%555
\subsection*{Aufgaben}

\aufgabenFarbe{Wie groß ist der Sektorwinkel eines Sektors mit Radius $r=1$ und Sektorfläche $A=1$?}
\TNT{4}{$A = \frac12 \cdot{} r^2 \stackrel{\frown}{\alpha}$

  $\Longrightarrow$

  $1 = \frac{1}{2} \cdot{} 1 \cdot{} \stackrel{\frown}{\alpha}$

  $\Longrightarrow$

  $\stackrel{\frown}{\alpha} = 2 \text{ rad} \approx 114.6\degre$
}

\olatLinkTALSStrukturaufgabenGLF{Teil 1}{3}{3. m) v) x)}

\AadBMTG{97ff}{1. b), 2. b) und 10. b)}

\AadBMTG{103}{40. [Winkel sind im Bogenmaß gegeben]}
%%\TALSAadBFWG{46}{175. a) b) d) 176. a) f) i) und das Aufgabenblatt im OLAT}
%%\AadBMTG{99}{10.}

\AadBMTG{175}{24.}
\newpage
