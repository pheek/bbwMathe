%%
%% 2019 07 04 Ph. G. Freimann
%%
\newcommand{\einheitskreis}{
\definecolor{qqwuqq}{rgb}{0,0.79,0}
\definecolor{qqqqff}{rgb}{0,0,1}
\definecolor{qqzzzz}{rgb}{0,0.6,0.6}
\definecolor{ffwwqq}{rgb}{1,0.4,0}
\definecolor{qqccww}{rgb}{0,0.8,0.4}
\definecolor{uququq}{rgb}{0.25,0.25,0.25}
\definecolor{xdxdff}{rgb}{0.49,0.49,1}
\begin{tikzpicture}[line cap=round,line join=round,>=triangle 45,x=2.5cm,y=2.5cm]
\draw[->,color=black] (-1.5,0) -- (1.5,0);
\foreach \x in {-1.5,-1,-0.5,0.5,1}
\draw[shift={(\x,0)},color=black] (0pt,2pt) -- (0pt,-2pt) node[below] {\footnotesize $\x$};
\draw[->,color=black] (0,-1.5) -- (0,1.5);
\foreach \y in {-1.5,-1,-0.5,0.5,1}
\draw[shift={(0,\y)},color=black] (2pt,0pt) -- (-2pt,0pt) node[left] {\footnotesize $\y$};
\draw[color=black] (0pt,-10pt) node[right] {\footnotesize $0$};
\clip(-1.5,-1.5) rectangle (1.5,1.5);
\draw [shift={(0,0)},color=qqwuqq,fill=qqwuqq,fill opacity=0.1] (0,0) -- (0:0.28) arc (0:46.14:0.28) -- cycle;
\draw(0,0) circle (2.5cm);
\draw [domain=-1.5:1.5] plot(\x,{(-0-0.72*\x)/-0.69});
\draw (0.69,-1.5) -- (0.69,1.5);
\draw (1,-1.5) -- (1,1.5);
\draw (1.12,0.6) node[anchor=north west,color=qqccww] {t};
\draw (0.50,0.42) node[anchor=north west,color=ffwwqq] {s};
\draw (0.30,0.21) node[anchor=north west,color=qqzzzz] {c};
\draw [line width=1.2pt,color=qqccww] (1,1.04)-- (1,0);
\draw [line width=1.2pt,color=ffwwqq] (0.69,0.72)-- (0.69,0);
\draw [line width=1.6pt,color=qqzzzz] (0.69,0)-- (0,0);
\begin{scriptsize}
\fill [color=xdxdff] (0.69,0.72) circle (1.5pt);
\draw[color=xdxdff] (0.49,0.71) node {$P$};
\fill [color=uququq] (0.69,0) circle (1.5pt);
\draw[color=uququq] (0.8,0.12) node {$Xp$};
%%\fill [color=uququq] (1,1.04) circle (1.5pt);
%%\draw[color=uququq] (1.22,0.95) node {$Yp$};
\fill [color=uququq] (0.69,0.72) circle (1.5pt);
\draw[color=uququq] (0.82,0.7) node {$Yp$};
\fill [color=qqqqff] (0,0) circle (1.5pt);
\draw(0.20,0.09) node {$\Phi$};
\end{scriptsize}
\end{tikzpicture}\par
}%% END Command \einheitskreis

%%%%%%%%%%%%%%%%%%%%%%%%%%%%%%%%%%%%%%%%%%%%%%%%%%%%%%%%%%%%%%%%%%%%%%%%%%%%%%%%%
\section{Winkel am Einheitskreis}\index{Einheitskreis|textbf}
\sectuntertitel{Wie groß ist eigentlich der Einheitskreis?}


\subsection*{Lernziele}

\begin{itemize}
\item Einheitskreis
\item Werte exakt ablesen
\item Grad- und Bogenmaß
\end{itemize}

\TadBMTG{104}{7.1}
\newpage
\subsection{Einheitskreis}\index{Einheitskreis}

\TRAINER{\einheitskreis{}: NUR SINUS ZEICHNEN!}
%%\noTRAINER{Skizze Kreis mit Radius $r=1$:\vspace{84mm}}

\noTRAINER{
  \bbwCenterGraphic{175mm}{tals/trig2/img/Einheitskreis.png}
}



\subsection{Sinus als Verhältnis}\index{Sinus!als!im Einheitskreis}
Egal, wie groß der Einheitskreis gewählt wird, der $\sin()$ eines
gegebenen Winkels $\Phi$ ist immer gleich lang, denn er wird im Maß
des Kreises angegeben.


\begin{gesetz}{Einheitskreis: Sinus}{}
  Die Strecke $s$ kann im Einheitskreis auch als $\frac{s}1$ aufgefasst werden -- also
  als Gegenkathete ($s$) dividiert durch Hypotenuse ($r=1$). Somit
  gilt im Einheitskreis:
  $$s = \sin(\Phi)$$
\end{gesetz}

\begin{bemerkung}{Ordinate\index{Ordinate} = Sinus}{}
  Die Ordinate (= $y$-Koordinate) eines Punktes auf dem Einheitskreis ist
  der Sinus des zugehörigen Winkels.
\end{bemerkung}

\begin{bemerkung}{Sinus und Cosinus}{}
Der $\sin{}$, liegt immer zwischen -1 und
+1. Dies stimmt auch für den $\cos{}$.
\end{bemerkung}


\newpage

\subsection{Analog Cosinus}\index{Cosinus!im Einheitskreis}

\TRAINER{\einheitskreis{}: NUR Cosinus ZEICHNEN!}
%%\noTRAINER{Skizze Kreis mit Radius $r=1$:\vspace{84mm}}

\noTRAINER{
  \bbwCenterGraphic{175mm}{tals/trig2/img/Einheitskreis.png}
}


\begin{bemerkung}{Cosinus}{}
  Die Strecke $c$ kann auch als $\frac{c}1$ aufgefasst werden -- also
  als Ankathete ($c$) dividiert durch Hypotenuse ($r=1$). Somit
  gilt
  $$c = \cos(\Phi)$$
  und auch, dass die Abszisse (=$x$-Koordinate) eines Punktes auf dem
  Einheitskreis der Cosinus des zugehörigen Winkels darstellt.
\end{bemerkung}

\begin{bemerkung}{Abszisse\index{Abszisse} = Cosinus}{}
  Die Abszisse (= $x$-Koordinate) eines Punktes auf dem Einheitskreis ist
  der Cosinus des zugehörigen Winkels.
\end{bemerkung}

\newpage
\subsection{Tangens}\index{Tangens!im Einheitskreis}

\TRAINER{\einheitskreis{}: sin cos und tan zeichnden!}
%%\noTRAINER{Skizze Kreis mit Radius $r=1$:\vspace{84mm}}

\noTRAINER{
  \bbwCenterGraphic{175mm}{tals/trig2/img/Einheitskreis.png}
}
%%\einheitskreis{}


Ähnlichkeit: 
Betrachten wir in obigem Bild die Strecke $t$ (von der $x$-Achse zum
Punkt $Y_P$, so sehen wir, dass sie genau das Verhältnis angibt, das
auch zwischen den Strecken $s$ und $c$ vorliegt. 
Somit können wir den Tangens definieren als

$$\tan(\phi) = t$$

Ebenso gilt:
\begin{gesetz}{Tangens}{}
$$\tan(\phi) = t = \frac{t}{1} =  \frac{s}{c} = \frac{\sin(\phi)}{\cos(\phi)}$$
\end{gesetz}
\newpage

\subsection{Zwei Winkel}
Zu jedem $\sin()$-, $\cos()$- und $\tan()$-Wert existieren im
Einheitskreis je zwei Winkel (also Winkel zwischen $0^{\circ{}}$ und
$360^{\circ{}}$).


\noTRAINER{
  \bbwCenterGraphic{80mm}{tals/trig2/img/Einheitskreis.png}
}


\noTRAINER{
  \bbwCenterGraphic{80mm}{tals/trig2/img/Einheitskreis.png}
}


\noTRAINER{
  \bbwCenterGraphic{80mm}{tals/trig2/img/Einheitskreis.png}
}

\TRAINER{\bbwCenterGraphic{100mm}{tals/trig2/img/ZweiWinkel.png}}

\newpage
\subsection*{Aufgaben}
Berechnen Sie je die beiden Winkel ($\gamma_1, \gamma_2$,
$\sigma_1, \sigma_2$ und $\tau_1, \tau_2$) für die folgenden Werte mit Hilfe des
Einheitskreises und prüfen Sie Ihre Resultate mit dem Taschenrechner:


\begin{tabular}{|c|c|c|}
  $\cos(\gamma)=\frac{\sqrt{3}}{2}$ & $\sin(\sigma) = \frac{-\sqrt{2}}{2}$ & $\tan(\tau) = -1$\\
\end{tabular}

\TNT{6}{Lösungen: $\gamma = 30\degre; 330\degre$, $\sigma =
  225\degre;315\degre$ und $\tau=135\degre;315\degre$. \vspace{50mm}}




\olatLinkArbeitsblatt{Einheitskreis}{https://olat.bbw.ch/auth/RepositoryEntry/572162090/CourseNode/102901229792483}{Aufg. 1. - 3. }


%%\TALSAadBFWG{101ff}{83. a) c),  84., 85. [$\sin$] a) c) e) f) g) und   h), 86. [$\cos$] b) c) e) und h) 87. [$\tan$] a) und c)}
%% \GESOAadBMTA{????}{????} %% GESO haben noch keine Geometrie
\AadBMTG{118ff}{8., 3. a) c) d), 4. c) d) , 5. b) d) , 14. a)}
\newpage


\subsubsection{Weitere trigonometrische Beziehungen}
Des weiteren gelten die folgenden Beziehungen, welche auch aus dem
Einheitskreis abgelesen werden können:
\begin{gesetz}{Sinus}{}
 Für den Sinus gilt:

  $$\sin(-\varphi) = - \sin(\varphi)$$
  \end{gesetz}

\begin{gesetz}{Cosinus}{}
 Für den Cosinus gilt:
  $$\cos(-\varphi) = \cos(\varphi)$$
  \end{gesetz}

\begin{gesetz}{Sinus und Cosinus}{}
 Für Sinus und Cosinus gilt:
  $$\cos( \varphi ) = \sin( \varphi + 90\degre )$$
  \end{gesetz}
\newpage


\subsubsection{Referenzaufgabe}
Berechnen Sie die Lösungsmenge $\lx$ für
$$\cos(x) = -\sin(211\degre)$$
im Definitionsbereich $\mathbb{D}=[0; 360\degre]$.

%%?//???????????????????????????????????????????????????????
%%WAS STAND DA bei 1??????

\TNTeop{
  1. $$\cos(x) = \sin(-211\degre)$$

  2. Generell gilt:

  $\cos(x) = \sin(x+90\degre)$

  3. $\cos(x)$ aus (1.) und (2.) gleichsetzen:

$$\sin(-211\degre) = \sin(x+90\degre)$$

4. 
Somit erhalten wir für $x$ sofort eine Lösung nämlich

$$211\degre = x+ 90\degre$$
also
$$x=-301\degre$$

5. $x$ auf den Definitionsbereich einschränken:

Der $\cos$ ist an $\varphi=0$ gespiegelt (gerade), somit ist $x_1=+301\degre$ sofort auch eine
Lösung.

6. Der $\cos$ ist symmetrisch gegenüber $180\degre$ und somit ist $x_2
= 360\degre-301\degre = 59\degre$ auch eine Lösung.

$$\Longrightarrow \lx = \{59\degre, 301\degre\}$$
}%% END TNTeop
%%\newpage



\subsection*{Aufgaben}
\olatLinkTALSStrukturaufgabenGLF{Teil 1}{3}{3. k) m) p) q) t) v) x)}
%%\aufgabenFarbe{Strukturaufgaben S. 3 Aufg. 3. k) m) p) q) t) v) x)}
%%\TALSAadBFWG{124}{183. a) b) c) und d), 185. a) und mit TR (solve)
%%  185. b), 188. a) b) c) d) und e) und 189. a) mit geeigneter Substitution}

\TALSAadBMTG{173}{1. a) b) c)}
\TALSAadBMTG{175}{16. a) b) e), 17. a) b) c) d), 24. a)}

%%\TALSAadBMTA{234ff}{885.ff}
