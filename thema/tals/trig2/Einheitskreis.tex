%%
%% 2019 07 04 Ph. G. Freimann
%%

%%%%%%%%%%%%%%%%%%%%%%%%%%%%%%%%%%%%%%%%%%%%%%%%%%%%%%%%%%%%%%%%%%%%%%%%%%%%%%%%%
\section{Winkel am Einheitskreis}\index{Einheitskreis|textbf}
\sectuntertitel{Wie groß ist eigentlich der Einheitskreis?}


\subsection*{Lernziele}

\begin{itemize}
\item Einheitskreis
\item Werte exakt ablesen
\item Grad- und Bogenmaß
\end{itemize}

\TadBMTG{104}{7.1}
\newpage
\subsection{Drehsinn}
Winkel werden im \textbf{mathematisch positiven Sinne} (=
Gegenuhrzeigersinn) gemessen.

Die positive $x$-Achse hat den Drehwinkel $0\degre$.

Aufgaben:

Geben Sie vom folgenden Punkt $P$ jeweils seinen mathematischen
Winkel an:


\begin{tabular}{cc}\makecell{
  $P=(1 | 1)\TRAINER{=45\degre}$\\
  \miniEinheitskreis{}} &

  \makecell{$P=(-1 | 1)\TRAINER{=135\degre}$\\
  \miniEinheitskreis{}} \\


   \makecell{ $P=(1 | -1)\TRAINER{=315\degre = -45\degre}$\\
  \miniEinheitskreis{}} &

  \makecell{$P=(\sqrt{3} | -1)\TRAINER{330\degre = -30\degre}$\\
  \miniEinheitskreis{}} \\

\end{tabular}

\newpage


\begin{tabular}{cc}\makecell{
  $P=(-4 | -4) \TRAINER{=225\degre=-135\degre}$\\
  \miniEinheitskreis{}} &

  \makecell{$P=\left( \frac{\sqrt{3}}2 \bigg| 1\right)\TRAINER{=30\degre}$\\
  \miniEinheitskreis{}} \\

  \makecell{$P=\left( \frac{\sqrt{-3}}2 \bigg| 1\right)\TRAINER{=150\degre}$\\
  \miniEinheitskreis{}} &

  \makecell{$P=(-1 | 0)\TRAINER{=180\degre = -180\degre}$\\
  \miniEinheitskreis{}} \\

\end{tabular}


\newpage
\subsection{Einheitskreis}\index{Einheitskreis}

\TRAINER{\einheitskreis{}: Trainer: Hier nur $s$=Sinus einzeichnen!}
%%\noTRAINER{Skizze Kreis mit Radius $r=1$:\vspace{84mm}}

\noTRAINER{
  \bbwCenterGraphic{175mm}{tals/trig2/img/Einheitskreis.png}
}



\subsection{Sinus als Verhältnis}\index{Sinus!als!im Einheitskreis}
Egal, wie groß der Einheitskreis gewählt wird, der $\sin()$ eines
gegebenen Winkels $\varphi$ ist immer gleich lang, denn er wird im Maß
des Kreises angegeben.


\begin{gesetz}{Einheitskreis: Sinus}{}
  Die Strecke $s$ kann im Einheitskreis auch als $\frac{s}1$ aufgefasst werden -- also
  als Gegenkathete ($s$) dividiert durch Hypotenuse ($r=1$). Somit
  gilt im Einheitskreis:
  $$s = \sin(\varphi)$$
\end{gesetz}

\begin{gesetz}{Ordinate\index{Ordinate} = Sinus}{}
  Die Ordinate (= $y$-Koordinate) eines Punktes auf dem Einheitskreis ist
  der Sinus des zugehörigen Winkels.
\end{gesetz}

\begin{bemerkung}{Sinus und Cosinus}{}
Der $\sin{}$ liegt immer zwischen -1 und
+1. Dies stimmt auch für den $\cos$\-.
\end{bemerkung}


\newpage

\subsection{Analog Cosinus}\index{Cosinus!im Einheitskreis}

\TRAINER{\einheitskreis{}: TRAINER: Hier nur $c$=Cosinus einzeichnen!}
%%\noTRAINER{Skizze Kreis mit Radius $r=1$:\vspace{84mm}}

\noTRAINER{
  \bbwCenterGraphic{175mm}{tals/trig2/img/Einheitskreis.png}
}


\begin{bemerkung}{Cosinus}{}
  Die Strecke $c$ kann auch als $\frac{c}1$ aufgefasst werden -- also
  als Ankathete ($c$) dividiert durch Hypotenuse ($r=1$). Somit
  gilt
  $$c = \cos(\varphi).$$
\end{bemerkung}

\begin{bemerkung}{Abszisse\index{Abszisse} = Cosinus}{}
  Die Abszisse (= $x$-Koordinate) eines Punktes auf dem Einheitskreis ist
  der Cosinus des zugehörigen Winkels.
\end{bemerkung}

\newpage
\subsection{Tangens}\index{Tangens!im Einheitskreis}

\TRAINER{\einheitskreisT{}: sin cos und tan zeichnen!}
%%\noTRAINER{Skizze Kreis mit Radius $r=1$:\vspace{84mm}}

\noTRAINER{
  \bbwCenterGraphic{175mm}{tals/trig2/img/Einheitskreis.png}
}
%%\einheitskreis{}


Ähnlichkeit: 
Betrachten wir in obigem Bild die Strecke $t$ (von der $x$-Achse zum
Punkt $T=(1|t)$, so sehen wir, dass sie genau das Verhältnis angibt, das
auch zwischen den Strecken $s$ und $c$ vorliegt. 
Somit können wir den Tangens definieren als

$$\tan(\varphi) = t.$$

Ebenso gilt:
\begin{gesetz}{Tangens}{} Betrachte das große Dreieck:
  
  $$\TRAINER{\frac{\text{Gegenkathete}}{\text{Ankathete}}}\noTRAINER{\hspace{40mm}}
  =  \TRAINER{\frac{t}1}\noTRAINER{\hspace{15mm}} =
  \TRAINER{\frac{s}{c}}\noTRAINER{\hspace{15mm}} =
\frac{\sin(\varphi)}{\cos(\varphi)} = t =  
  \tan(\varphi) $$
\end{gesetz}
\newpage

\subsection{Zwei Winkel}
Zu jedem $\sin()$-, $\cos()$- und $\tan()$-Wert existieren im
Einheitskreis je zwei Winkel (also Winkel zwischen $0^{\circ{}}$ und
$360^{\circ{}}$).

Beispiel $\sin(\varphi) = 0.6$:
\noTRAINER{
  \bbwCenterGraphic{90mm}{tals/trig2/img/Einheitskreis.png}
}

Beispiel $\cos(\varphi) = -0.4$:
\noTRAINER{
  \bbwCenterGraphic{90mm}{tals/trig2/img/Einheitskreis.png}
}

Beispiel $\tan(\varphi) = 1.2$:
\noTRAINER{
  \bbwCenterGraphic{90mm}{tals/trig2/img/Einheitskreis.png}
}

\TRAINER{\bbwCenterGraphic{80mm}{tals/trig2/img/ZweiWinkel.png}}

\newpage
\subsection*{Aufgaben}
Berechnen Sie je die beiden Winkel ($\gamma_1, \gamma_2$,
$\sigma_1, \sigma_2$ und $\tau_1, \tau_2$) für die folgenden Werte mit Hilfe des
Einheitskreises und prüfen Sie Ihre Resultate mit dem Taschenrechner:


\begin{tabular}{|c|c|c|}
  $\cos(\gamma)=\frac{\sqrt{3}}{2}$ & $\sin(\sigma) = \frac{-\sqrt{2}}{2}$ & $\tan(\tau) = -1$\\
\end{tabular}

\TNT{6}{Lösungen: $\gamma = 30\degre; 330\degre$, $\sigma =
  225\degre;315\degre$ und $\tau=135\degre;315\degre$. \vspace{50mm}}




\olatLinkArbeitsblatt{Einheitskreis}{https://olat.bbw.ch/auth/RepositoryEntry/572162090/CourseNode/102901229792483}{Aufg. 1. - 3. }


%%\TALSAadBFWG{101ff}{83. a) c),  84., 85. [$\sin$] a) c) e) f) g) und   h), 86. [$\cos$] b) c) e) und h) 87. [$\tan$] a) und c)}
%% \GESOAadBMTA{????}{????} %% GESO haben noch keine Geometrie
\AadBMTG{118ff}{8., 3. a) c) d), 4. c) d) , 5. b) d) , 14. a)}
\newpage


\subsubsection{Weitere trigonometrische Beziehungen}
Des weiteren gelten die folgenden Beziehungen, welche auch aus dem
Einheitskreis abgelesen werden können:
\begin{gesetz}{Sinus}{}
 Für den Sinus gilt:

  $$\sin(-\varphi) = - \sin(\varphi)$$
  \end{gesetz}

\begin{gesetz}{Cosinus}{}
 Für den Cosinus gilt:
  $$\cos(-\varphi) = \cos(\varphi)$$
  \end{gesetz}

\begin{gesetz}{Sinus und Cosinus}{}
 Für Sinus und Cosinus gilt:
  $$\cos( \varphi ) = \sin(90\degre -  \varphi )$$
\end{gesetz}

Beweis letztes Gesetz exemplarisch:

\TNTeop{
\bbwCenterGraphic{12cm}{tals/trig2/img/SinCosAmEinheitskreis.png}
}

\newpage


\subsubsection{Referenzaufgabe}
Berechnen Sie die Lösungsmenge $\lx$ für
$$\cos(x) = -\sin(211\degre)$$
im Definitionsbereich $\DefinitionsMenge{}=[0; 360\degre]$.

%%?//???????????????????????????????????????????????????????
%%WAS STAND DA bei 1??????

\TNTeop{
  am besten am Einheitskreis ablesen. Ansonsten rechnerisch:
  
  1. Sinus ist gespiegelt am Nullpunkt $$\cos(x) = \sin(-211\degre)$$
     (Begriff «gerade/ungerade» Funktionen noch nicht einführen: Kommt
  bei den Potenzfunktionen im Schwerpunktfach.)
  
  2. Generell gilt:

  $\cos(x) = \sin(90\degre - x)$

  3. $\cos(x)$ aus (1.) und (2.) gleichsetzen:

$$\sin(-211\degre) = \sin(90\degre - x)$$

4. 
Somit erhalten wir für $x$ sofort eine Lösung, nämlich

$$ - 211\degre = 90\degre - x$$
$$   211\degre = x - 90\degre$$

also
$$x=301\degre$$

5. $x$ auf den Definitionsbereich einschränken:

Der $\cos$ ist an der $x$-Achse gespiegelt, somit ist
$x_1=-311\degre = 59\degre$ sofort auch eine
Lösung.

$$\Longrightarrow \lx = \{59\degre, 301\degre\}$$
}%% END TNTeop
%%\newpage



\subsection*{Aufgaben}
\olatLinkTALSStrukturaufgabenGLF{Teil 1}{5}{3. k) p) q) t) u)}
%%\aufgabenFarbe{Strukturaufgaben S. 3 Aufg. 3. k) m) p) q) t) v) x)}
%%\TALSAadBFWG{124}{183. a) b) c) und d), 185. a) und mit TR (solve)
%%  185. b), 188. a) b) c) d) und e) und 189. a) mit geeigneter Substitution}


% Goniometrie:
%%\TALSAadBMTG{173}{1. a) b) c)}
\TALSAadBMTG{175}{16. a) b) e)}

%%\TALSAadBMTA{234ff}{885.ff}
