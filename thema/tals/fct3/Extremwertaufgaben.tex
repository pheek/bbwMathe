

%%%%%%%%%%%%%%%%%%%%%%%%%%%%%%%%%%%%%%%%%%%%%%%%%%%%%%%%%%%%%%%%
\subsection{Extremwertaufgaben}



\subsubsection{Aus alter Maturprüfung}

\aufgabenFarbe{Gegeben ist die Funktion $f(x) = x\cdot{}(3-\sqrt{x})$,
  $x\in[0;\infty[$.\\
  a) Bestimmen Sie die Nullstellen und das  Extremum der Funktion
  $f$.\\
  b) Im ersten Quadranten, zwischen dem  Graphen und der
  $x$-Achse ist ein  rechtwinkliges Dreieck $ABC$ einbeschrieben.  Der
  rechte Winkel ist in der Ecke $B$.  Punkt $A$ liegt im Ursprung, $B$
  auf der  $x$-Achse und $C$ auf dem Graphen von $f$. Berechnen Sie
  die Koordinaten des  Punktes $C$ so, dass der Flächeninhalt des
  Dreiecks maximal wird.
}%% END aufgabenFarbe

    
\bbwCenterGraphic{8cm}{tals/fct3/img/Maximieren.png}
\TNTeop{
   a) solve$(f(x)=0,x)$ Somit sind die Nullstellen bei 0 und 9\\
     $\text{fmax}(f(x),x)$ liefert $x=4$ ist Maximalstelle (und auch
     Maximalwert ($f(4)=4$)\\
   b) $\text{fmax}(0.5\cdot{}x\cdot{}f(x), x)$ liefert $x = 5.76$ und
   $f(5.76) = 3.456$}%% End TNTeop
\newpage
\subsection*{Aufgaben}
\AadBMTA{311}{49.}
\AadBMTA{321}{17.}
