\section{Wurzelfunktionen}\index{Wurzelfunktionen}\index{Funktionen!Wurzelfct.}


\subsection*{Lernziele}

\begin{itemize}
\item Wurzelfunktion als Umkehrfunktion
\item Punkte-Aufgaben
\item Symmetrieeigenschaften und Scheitelform
\end{itemize}

%%\TALSTadBFWA{209}{3.8}
\newpage

\subsection{Motivation}

Die geforderte Länge $l$ (in m) eines Pendels lässt sich wie folgt ermitteln:

$$l = g \cdot{} \left(\frac{t}{2\pi}\right)^2$$


Dabei ist $t$ die gewünschte Schwingungsdauer in
Sekunden\footnote{Gemessen hin und zurück.} und $g$ die
Erdbeschleunigung (9.81).

Berechnen Sie die Schwingdauer für ein Pendel von 44 cm Länge.

\vspace{4mm}

Die Schwingdauer bei 44 cm beträgt ca. \LoesungsRaum{1.3} Sekunden.

\TNT{3.2}{
  $$0.44 = 9.81\cdot{}  \left( \frac{t}{2\pi}\right)^2$$
  $$\sqrt{\frac{0.44}{9.81}} = \frac{t}{2\pi}$$
  $$t\approx 1.33$$
}

Geben Sie die Schwingungsdauer $t$ in Abhängigkeit einer beliebigen
Länge $l$ des Pendels an.

$$t = t(l) = \LoesungsRaumLen{40mm}{2\pi \cdot{} \sqrt{\frac{l}{g}}}$$

\TNT{3.2}{
  Herleitung
  $$4 = g\cdot{}  \left( \frac{t}{2\pi}\right)^2$$
  $$\sqrt{\frac{l}{g}} = \frac{t}{2\pi}$$
  $$2\pi\cdot{}\sqrt{\frac{l}{g}} = t$$
}



\newpage
%%%%%%%%%%%%%%%%%%%%%%%%%%%%%%%%%%%%%%%%%%%%%%%%%%%%%%%%%%%%%%%%%%%%%%%%%%%%%%%%%%%%%%%%%%%%%55
\TadBMTA{300}{17.2}

Skizzieren Sie $f: y = \sqrt{x}$ und $g: y=\sqrt[5]{x}$ im Bereich $[-1; 9]$ indem Sie für jeden 0.5-er $x$-Wert die zugehörigen $y$-Werte berechnen:

\bbwGraph{-2}{9.5}{-1}{3.5}{%%
  \TRAINER{\bbwFunc{sqrt(\x)}{0:9}}
  \TRAINER{\bbwFunc{pow(\x, 0.2)}{0:9}}
}%%

\begin{bemerkung}{Wurzelfunktion}{}
Für große $x$ Werte gilt: Je größer der Wurzelexponent $n$, umso
\LoesungsRaumLang{flacher} wird der Graph der Wurzelfunktion

$$y=\sqrt[n]{x}.$$ 

\end{bemerkung}


\begin{definition}{Wurzelfunktion}{}
  Der Definitionsbereich der Wurzelfunktion beschränkt sich auf die
  positiven reellen Zahlen:
  $$\DefinitionsMenge{} = \mathbb{R}_0^{+} = [0;\infty[$$
      Der Wertebereich beschränkt sich somit auch nur auf die positive $y$-Achse:
  $$\Wertebereich{} = \mathbb{R}_0^{+} = [0;\infty[$$
      
\end{definition}
\newpage

%%%%%%%%%%%%%%%%%%%%%%%%%%%%%%%%%%%%%%%%%%%
\subsection{Wurzelfunktion als Umkehrfunktion}\index{Umkehrfunktion!der Wurzelfunktion}\index{Funktionen!Umkehrfunktion}

Skizzieren Sie $f: y = \sqrt[2]{x}$ und $g: y=x^2$ im Bereich $[-2; 9]$ indem Sie für jeden 0.5-er $x$-Wert das jeweils zugehörige $y$ berechnen:

\bbwGraph{-2}{9.5}{-2}{9}{%%
  \TRAINER{\bbwFuncC{\x*\x}{-2:0}{red, very thick}}
  \TRAINER{\bbwFuncC{\x*\x}{0:3}{blue, very thick}}
  \TRAINER{\bbwFuncC{pow(\x, 0.5))}{0:9}{blue}}
  \TRAINER{\bbwFuncC{-pow(\x, 0.5))}{0:9}{red}}
}%%

\begin{bemerkung}{Umkehrfunktion}{}
  Eine \textbf{Umkehrfunktion} ist das Spiegelbild der der Funktion
  gespiegelt an der Geraden \LoesungsRaum{$y=x$}.
\end{bemerkung}  

\begin{bemerkung}{Einschränkung}{}
Funktionen, welche beim Spiegeln an $x=y$ mehrere «Funktionswerte»
erhalten würden, sind nur \textbf{eingeschränkt} umkehrbar.
\end{bemerkung}

\newpage
\subsection*{Aufgaben}
Skizzieren Sie
a) $y=x^3$ und $y=\sqrt[3]{x}$ ins selbe Koordinatensystem:

\bbwGraph{-7}{7}{-5}{5}{
  \TRAINER{\bbwFuncC{exp(0.333*ln(\x))}{0.01:7}{blue}}
  \TRAINER{\bbwFuncC{\x*\x*\x}{-1.7:1.7}{red}}
}

b) Was fällt auf?

\TNT{2}{Im positiven Ast ist $\sqrt[3]{x}$ die Umkehrung von $x^3$
  (gespiegelt an der Halbierenden des 1. Quadranten.}

c) Geben Sie von $y=x^3$ und von $y=\sqrt[3]{x}$ jeweils den
Definitionsbereich und den Werteberech an.

\TNT{4}{
  $$x^3: \mathbb{D}=\mathbb{W}=\mathbb{R}$$

  $$\sqrt[3]{x}: \mathbb{D}=\mathbb{W}=\mathbb{R}^+_0$$
  Obschon die Funktion auch Umkehrbar ist, ist die Wurzel von
  negativen Zahlen in $\mathbb{R}$ nicht definiert.
}%% end TNT

\newpage


%%\TALSAadBMTA{213ff}{801. für $n=2,3$, 802. mit Geogebra oder TR}
\subsection{Punkte-Aufgaben}
Die Wurzelfunktion $f: y= a\cdot{} \sqrt[n]{x}$ gehe durch die Punkte
$A= (16 | -4)$ und $B=(625 | -10)$. Finden Sie die Parameter $a$ und
$n$.

\TNT{6}{Idee: Beide nach $a$ auflösen und gleichsetzen. $n=4$ und $a=-2$.}


\subsection*{Aufgaben}
\AadBMTA{310}{42. a) c)}
\newpage


\subsection{Transformationen an der Wurzelfunktion}

\subsubsection{Wurzelfunktionen skizzieren}


Geben Sie den Definitionsbereich der folgenden Funktionsterme an und skizzieren Sie:

$$\sqrt{x}; \sqrt{3x}; \sqrt{x-3}; \sqrt{x} - 3;
\sqrt{-x}; -\sqrt{x+3} + 2; 2\sqrt{x}$$

\subsection*{Aufgaben}
\AadBMTA{}{41. c) f)}


\newpage


\subsubsection{Parameter und «Scheitelform»}\index{Scheitelform!Wurzelfunktion}
$$y = a\cdot{} \sqrt{b\cdot{}x + c} + d$$

Was bedeuten die Parameter $a$, $b$, $c$ und $d$. Nutzen Sie die
Erkenntnisse aus der vorangehenden Aufgabe.

\TNT{6}{
  $a$: Formfaktor (Streckung entlang der $y$-Achse

  $b$: Stauchung entlang der $x$-Achse

  $c$: Verschiebung gegüber der $x$-Achse

  $d$: Verschiebung entlang der $y$-Achse
}

Skizzieren Sie a) $y=2\sqrt{x}$ und $y=\sqrt{4x}$.

Was fällt auf?

\TNTeop{
  Die Faktoren $a$ und $b$ sind austauschbar. Entweder kann $a$, oder
  $b$ ermittelt werden, dies sobald der andere der beiden Faktoren
  bekannt ist.

  $$a\sqrt{bx}  = (a\sqrt{b})\cdot{}\sqrt{x} = \sqrt{a^2bx}$$
}%% end TNTeop
%%%%%%%%%%%%%%%%%%%%%%%%%%%%%%%%%%%%%%%%%%%%%%%%%%%%%%%%%%%%%%%%%%55


\begin{definition}{Wurzelfunktion in «Scheitelform»}{}

  $$f: y = a\cdot{}\sqrt{\pm(x-X_S)} + Y_S$$

\end{definition}

\begin{gesetz}{Scheitelform}{}
  Bei der Funtkion
  $$f: y = a\cdot{}\sqrt{\pm(x-X_S)} + Y_S$$
liegt der Scheitelpunkt bei $S=(X_S | Y_S)$ und der Formfaktor $a$ kann
entweder durch Einsetzen in die Funktionsgleichung oder durch ablesen
bei $x=X_S+1$ ermittelt werden.

Das $\pm$ unter der Wurzel wird zu einem Minus ($-$), sofern der
Wurzelast nach links abgeht.
\end{gesetz}

\newpage



\subsubsection{Wurzelfunktionen ablesen}

Geben Sie von den folgenden Graphen\footnote{Ursprüngliche Graphik: M. Mohr} den zugehörigen Funktionsterm
in der Scheitelform an und vereinfachen Sie danach so weit wie möglich: $$y=a\cdot{}\sqrt{\pm(x - X_S)}+Y_S$$

\bbwCenterGraphic{150mm}{tals/fct3/img/WurzelfunktionenAblesen.png}
\TNTeop{
  $f: y=1 \cdot{} \sqrt{-(x-4)} + 0 = \sqrt{4-x}$,
  
  $g: y=3 \cdot{} \sqrt{+(x-(-5)} + 0 = 3\cdot{} \sqrt{x+5}$,

  $h: y=-\sqrt{2} \cdot{} \sqrt{+(x+0)} + 5 = -\sqrt{2x} + 5$ und

  $i: y=2\cdot{} \sqrt{-(x-4)} + 3 = 2\cdot{} \sqrt{4-x} + 3$
}%% end TNT

%%\newpage implicit


\subsection*{Aufgaben}

%%\TALSAadBMTA{213ff}{803. n), 804. h), 805. c), 806. c), 807. a) c) 808. b)}
\TALSAadBMTA{310}{40., 43. g), 44. b), 45. a) }

\olatLinkTALSStrukturaufgabenSPF{Teil 2}{14ff}{46. und 50.}
\aufgabenFarbe{Strukturaufgaben SPF (V4.0) S. 14ff: Aufg. 46. und 50.}

\newpage

