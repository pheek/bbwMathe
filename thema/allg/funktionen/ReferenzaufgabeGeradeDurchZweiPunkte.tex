%% 
\subsubsection{Gerade durch zwei gegebene Punkte}\index{Punkt auf Geraden}\index{Gerade!Punkt auf}
\TRAINER{\TALS{ev. Einstiegsaufgaben Frommenwiler: 606. a) c)}}

Seien die Punkte $P=(9|6)$ und $Q=(-6|3)$ gegeben.
Gesucht ist die Funktionsgleichung $f: y=ax+b$ (namentlich $a$ und $b$), sodass
der Graph der Funktion $f$ durch beide Punkte führt.


\textbf{Rezept I: Graphisch}\\

\vspace{1mm}

\bbwGraph{-7}{10}{-1}{7}{
%
\TRAINER{%% Steigungsdreieck: 
 \bbwLine{-7,2.8}{10,6.2}{green}
 \bbwDot{-6,3}{blue}{south}{Q}
 \bbwDot{9,6}{blue}{south}{P}
  \bbwLine{-6,3}{9,3}{blue}
  \bbwLine{9,6}{9,3}{blue}
}%% end TRAINER
}%% end bbwGraph

\TNTeop{
1. Zeichnen\\
2. Steigungsdreieck zeichnen $V=3, H=15$\\
3. Steigung rechnen $a = \frac{V}{H} = \frac3{15} = \frac15 = 0.2$\\
4. $b$ «ablesen». Hier Dreisatz: $b = 3 + y^*$ mit $H : V = 6 :
y^* \Longleftrightarrow  6V = H\cdot{}y^*$ \\
    $\Longrightarrow y^* = 6\cdot{} V : H = 6\cdot{} a =
    6\cdot{} \frac15 = 1.2$\\
    ergo\\
    $b = 3 + y^* = 3+1.2 = 4.2.$

5. Funktionsgleichung $y = ax+b = 0.2 x + 4.2$
}%% end TNTeop
%
%\raisebox{2cm}{\includegraphics[width=16cm]{allg/funktionen/img/GeradeDurchZweiPunkte.jpg}}%% END TNT EOP 
%
\newpage

\textbf{Rezept II: Rechnerisch, algbraisch \GESO{(optional)}}\\

Seien wieder $P=(9|6)$ und $Q=(-6|3)$.
\vspace{1mm}

a)

Das $a$ kann durch das «Steigungsdreieck» berechnet werden:

$$a = \frac{\text{V}}{\text{H}}=\TALS{\frac{\Delta y}{\Delta x}}=\frac{y_Q-y_P}{x_Q-x_p} = \LoesungsRaumLang{\frac{3 - 6}{(-6) - 9} = \frac{-3}{-15} = \frac{1}{5}}$$

b)

Setzen wir $a=\frac{1}{5}$ und einen der beiden Punkte (\zB $P=(9|6)$) in die
allgemeine Funktionsgleichung $y=ax+b$ ein, so erhalten wir

\TNT{4}{$$y=\frac15x+b$$
$$6=\frac15\cdot{}9 + b$$
$$b=6 - \frac15 \cdot{} 9 = 4.2$$}

Die gesuchte Geradengleichung lautet also:

$$f: y=\LoesungsRaumLang{\frac{1}{5}\cdot{}x + 4.2}$$

\begin{rezept}{}{}
Um die Funktionsgleichung $$y=ax+b$$ zu finden,
  wird erst $a$ aus dem Steigungsdreieck berechnet und danach dieses
  $a$ und die Koordinaten eines Punktes in die Funktionsgleichung
  $y=ax+b$ eingesetzt. Diese Gleichung wird nach $b$ aufgelöst.
\end{rezept}
\newpage

\GESO{
\textbf{Rezept III: Mit dem Taschenrechner}

Wieder seien die beiden Punkte $P=(9|6)$ und $Q=(-6|3)$ gegeben.

Tippen Sie auf dem Taschenrechner \tiprobutton{data}, um mit der
Eingabe der beiden Punkte zu beginnen. Geben Sie in die
Spalte \texttt{L1} die $x$-Werte und in die Spalte \texttt{L2} die
$y$-Werte ein:

\begin{tabular}{c|c|}
 \texttt{L1}& \texttt{L2} \\\hline
 \noTRAINER{.....}\TRAINER{9}  & \noTRAINER{.....}\TRAINER{6}\\
 \noTRAINER{.....}\TRAINER{-6}  & \noTRAINER{.....}\TRAINER{3}
\end{tabular}

Wählen Sie mit \tiprobutton{2nd} \tiprobutton{data_stat-reg-distr}
unter \texttt{STAT-REG} die Nummer 4: «\texttt{LinReg} $ax+b$».
Lassen Sie die Werte auf den nächsten Seite so stehen wie sie sind:

xDATA: L1

yDATA: L2

Mittels \texttt{CALC} (\tiprobutton{enter}) erhalten Sie $a=0.2$ und $b=4.2$.

}%% END GESO

\TALS{

Dies kann auch mit dem Taschenrechner gelöst werden; denn für alle Punkte $P$ auf $f$ gilt ja $P(x|y) = P(x|f(x))$:

$$f(x):=a\cdot{}x + b$$
\[
%%\begin{equation}%% but equation makes a number (1)
    gls:= \left\{\begin{array}{@{}lr@{}}
        f(9) = 6\\
        f(-6) = 3
        \end{array}\right.
\]
%%\end{equation}

  $$solve(gls,\{a, b\})$$
}%%

\subsection*{Aufgaben}

\AadBMTA{252ff}{16. a), 18. a) und 19.}

\GESO{\olatLinkArbeitsblatt{Lineare
  Funktionen}{https://olat.bbw.ch/auth/RepositoryEntry/572162163/CourseNode/102901171745363}{7.3, 7.4
  und 7.5}}
\TALS{\olatLinkArbeitsblatt{Lineare
  Funktionen}{https://olat.bbw.ch/auth/RepositoryEntry/572162090/CourseNode/106131926046623}{7.3, 7.4
  und 7.5}}

\newpage%%
