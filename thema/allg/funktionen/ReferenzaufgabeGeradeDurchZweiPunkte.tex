%% 
\subsubsection{Gerade durch zwei gegebene Punkte}\index{Punkt auf Geraden}\index{Gerade!Punkt auf}
\TRAINER{\TALS{ev. Einstiegsaufgaben Frommenwiler: 606. a) c)}}

Seien die Punkte $P=(10|6)$ und $Q=(-5|3)$ gegeben.
Gesucht ist die Funktionsgleichung $f: y=ax+b$ (namentlich $a$ und $b$), sodass
der Graph der Funktion $f$ durch beide Punkte führt.


\textbf{Rezept I: Graphisch}\\

\vspace{1mm}


\noTRAINER{
\bbwGraph{-6}{11}{-1}{7}{
\bbwLine{-6,2.8}{11,6.2}{green}
\bbwDot{-5,3}{blue}{north}{Q}
\bbwDot{10,6}{blue}{north}{P}
}%% end bbwGraph
}%% end noTRAINER

\TNTeop{%%
\raisebox{2cm}{\includegraphics[width=16cm]{allg/funktionen/img/GeradeDurchZweiPunkte.jpg}}}%% END TNT EOP 

\newpage

\textbf{Rezept II: Rechnerisch, algbraisch \GESO{(optional)}}\\

\vspace{1mm}

\begin{rezept}{}{}
Um die Funktionsgleichung $$y=ax+b$$ zu finden,
  werden die $x$- und $y$-Koordinaten der beiden Punkte in die Gleichung eingesetzt und die beiden Gleichungen werden nach $a$ und $b$ aufgelöst. 
\end{rezept}

a)

Das $a$ kann durch das «Steigungsdreieck» berechnet werden:

$$a = \frac{y_Q-y_P}{x_Q-x_p} = \LoesungsRaumLang{\frac{3 - 6}{(-5) - 10} = \frac{-3}{-15} = \frac{1}{5}}$$

b)

Setzen wir $a=\frac{1}{5}$ und einen der beiden Punkte (\zB $P=(10|6)$) in die
allgemeine Funktionsgleichung $y=ax+b$ ein, so erhalten wir

\TNT{2}{$$y=\frac15x+b$$
$$6=\frac15\cdot{}10 + b$$
$$b=4$$}

Die gesuchte Geradengleichung lautet also:

$$f: y=\LoesungsRaumLang{\frac{1}{5}\cdot{}x + 4}$$

\GESO{\newpage
\textbf{Rezept III: Mit dem Taschenrechner}
Wieder seien die beiden Punkte $P=(10|6)$ und $Q=(-5|3)$ gegeben.

Tippen Sie auf dem Taschenrechner \tiprobutton{data}, um mit der
Eingabe der beiden Punkte zu beginnen. Geben Sie in die
Spalte \texttt{L1} die $x$-Werte und in die Spalte \texttt{L2} die
$y$-Werte ein:

\begin{tabular}{c|c|}
 \texttt{L1}& \texttt{L2} \\\hline
 \noTRAINER{.....}\TRAINER{10}  & \noTRAINER{.....}\TRAINER{6}\\
 \noTRAINER{.....}\TRAINER{-5}  & \noTRAINER{.....}\TRAINER{3}
\end{tabular}

Wählen Sie mit \tiprobutton{2nd} \tiprobutton{data_stat-reg-distr}
unter \texttt{STAT-REG} die Nummer 4: «\texttt{LinReg} $ax+b$».
Lassen Sie die Werte auf den nächsten Seite so stehen wie sie sind:

xDATA: L1

yDATA: L2

Mittels \texttt{CALC} (\tiprobutton{enter}) erhalten Sie $a=0.2$ und $b=4$.

}%% END GESO

\TALS{\newpage}


\TALS{%%
Dies kann auch mit dem Taschenrechner gelöst werden; denn für alle Punkte $P$ auf $f$ gilt ja $P(x|y) = P(x|f(x))$:

$$f(x):=a\cdot{}x + b$$
\[
%%\begin{equation}%% but equation makes a number (1)
    gls:= \left\{\begin{array}{@{}lr@{}}
        f(10) = 6\\
        f(-5) = 3
        \end{array}\right.
\]
%%\end{equation}

  $$solve(gls,\{a, b\})$$
}%%

\subsection*{Aufgaben}

\AadBMTA{252ff}{13. a), 14. a), 15. a), 16. a), 18. a), 19., 25. a)}

\newpage%%
