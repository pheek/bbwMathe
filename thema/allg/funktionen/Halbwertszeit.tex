%%
%% Halbwertszeit
\subsection{Halbwertszeit, Verdopplungszeit}\index{Halbwertszeit}\index{Verdopplungszeit}


\begin{definition}{Halbwertszeit}{}
Die Zeitspanne, in der sich ein Wert bei exponentiellem Zerfall halbiert, nennen wir
\textbf{Halbwertszeit} und bezeichnen diese Zeit mit:

$$T_{1/2}$$
Ansatz:
$$\frac12 \cdot{} f(0) = f(T_{1/2})$$
\end{definition}


\textbf{Einstiegsbeispiel}

\begin{beispiel}{Halbwertszeit (A)}{}
Strontium 90 hat eine Halbwertszeit von 29 Tagen. Um wie viel Prozent
nimmt Strontium 90 innerhalb von 50 Tagen ab?

\TNT{8}{
  Ansatz:
  $$a = \frac{1}{2}$$
  $$\tau=29 \text{ Tage}$$
  $$f(t) = 100\% \cdot{} \left(\frac{1}{2}\right)^{\frac{t}{29}}$$

  50 Tage einsetzen:

  $$f(50) = 100\% \cdot{} \left(\frac{1}{2}\right)^{\frac{t}{50}}
  \approx 0.30268$$
  Somit nimmt Strontium innerhalb von 50 Tagen um ca. 70 \% ab.

  \vspace{20mm}
}

\end{beispiel}



Ist die Halbwertszeit $T = T_{1/2}$ bekannt, so gilt mit
$\tau=T_{1/2}$ und mit $a=\frac12$:

\begin{gesetz}{Stoffmenge}{}
  Ist die Halbwertszeit $T_{1/2}$ und die anfängliche Stoffmenge $b$
  bekannt, so kann die Stoffmenge zu jedem Zeitpunkt $t$ mit der
  folgenden Funktion $f$ angegeben werden:
  \leserluft{}
  \leserluft{}  
  $$f(t) = \LoesungsRaumLen{50mm}{b\cdot{}\left(\frac12\right)^{\frac{t}{T_{1/2}}}}$$
\end{gesetz}
\newpage



\subsection*{Aufgaben}
Uran 235 (U-235) hat eine Halbwertszeit von 704 Millionen Jahren. Wie viel ist
von einem Kilogramm U-235 nach 100 Millionen Jahren noch vorhanden?

\TNT{4}{
  
  $$f(100 \cdot{}10^6) =
  1 \text{ kg} \cdot{}
  \left( \frac{1}{2} \right)^{ \frac{100\cdot{} 10^6}{704 \cdot{} 10^6}}
  \approx 906 \text{ g}
  $$

}



Der Luftdruck ab Meereshöhe nimmt innerhalb von 5400 Metern
(Höhenunterschied) um jeweils 50\% ab (abgesehen von der
Wetterdynamik). Bei 0 M. ü. M. (Meeresspiegel) ist der
durchschnittliche Luftdruck ca. bei 1\,013.25 hPa (Hektopascal).

Wie hoch ist der durchschnittliche Luftdruck hier in Winterthur auf
einer Meereshöhe von 444\,m?


\TNT{4}{
  
  $$f(400) = 1\,013.25 \text{ hP } \cdot{}  \left( \frac{1}{2} \right)^{\frac{400}{5400}}
  \approx 962.54 \text{ hP }
  $$

}


\newpage

\subsubsection*{Wie berechnen wir die Halbwertszeit?}

\noTRAINER{\bbwCenterGraphic{85mm}{allg/funktionen/img/exp/HalbwertszeitLeer.png}}
  \TRAINER{\bbwCenterGraphic{75mm}{allg/funktionen/img/exp/Halbwertszeit.png}}

\begin{beispiel}{Halbwertszeit (B)}{}
Beispiel: Ein Stoff nimmt innerhalb von sieben Tagen auf 80\% ab. Wie
groß ist seine Halbwertszeit $T_{1/2}$?
\end{beispiel}

\TNTeop{
  Generell gilt: $f(t) = b\cdot{}a^\frac{t}\tau$

  Gesucht $T = T_{1/2} = $ \, Halbwertszeit

  Ansatz (S. Grafik)

  $$\frac{f(0)}{2} = f(T)$$

  mit ($f(0) = b$) gilt

  $$\frac{b}{2} = f(T)$$

  $$\frac{b}{2}  = b \cdot{} a^{\frac{T}{\tau}}$$

  Auflösen

  $$\frac{1}{2} = a^{\frac{T}{\tau}}$$

  $$\frac{T}{\tau} = \log_a\left(\frac{1}{2}\right)$$

  $$T_{1/2} = T = \tau\cdot{}\log_a\left(\frac{1}{2}\right)$$

  80 \% und 7 Tage einsetzen:

  $$T_{1/2} = 7 \cdot{} \log_{0.8} \left(\frac{1}{2}\right) \approx
  21.74 \text{ (Tage)}$$

}%% END TNTeop



\begin{gesetz}{Halbwertszeit}{}
  Die Halbwertszeit $T_{1/2}$ berechnet sich wie folgt:
  $$T_{1/2} = \LoesungsRaumLen{70mm}{\tau \cdot{} \log_a\left(\frac12\right)}$$
  Dabei ist $a$ der Abnahmefaktor pro Zeiteinheit $\tau$.
\end{gesetz}


\subsection*{Aufgaben}

Das Isotop C14 (Kohlenstoff) zerfällt innerhalb von 200 Jahren auf
97.61\%. Wie groß ist seine Halbwertszeit?

\TNT{6}{
  $$T = T_{1/2} = \tau \cdot{} \log_a\left(\frac{1}{2}\right)$$

  $$a = 97.61\% = 0.9761$$
  
  $$= 200 \cdot{} \log_{0.9761}\left(\frac{1}{2}\right) \approx 5731
  \text{ (Jahre)}$$
}%end TNT


Das radioaktive Isotop Beryllium 7 (Be7) nimmt innerhalb von 10 Tagen um
12.23 \% ab. Wie groß ist seine Halbwertszeit?

\TNT{6}{
  $$T = T_{1/2} = \tau \cdot{} \log_a\left(\frac{1}{2}\right)$$

  $$a = 100\% - 12.23\% = 87.77\% = 0.8777$$
  
  $$= 10 \cdot{} \log_{0.8777}\left(\frac{1}{2}\right) \approx 53.13
  \text{ (Tage)}$$
}%% end TNT













\newpage

\GESO{Optional:}





\subsection{Verdopplungszeit}\index{Verdopplungszeit|textbf}

Analog gilt das Gesetz zur Verdopplung (s. obiges Beispiel Bevölkerung Irlands):
\begin{gesetz}{Verdopplungszeit}{}
  $$T_2 = \LoesungsRaumLen{40mm}{ \tau \cdot{} \log_a(2)}$$

  $a$: Zunahmefaktor pro Zeiteinheit $\tau$.
\end{gesetz}

(optional:)

\begin{bemerkung}{Verdopplungszeit}{}
  Ansatz:$$2 \cdot{} f(0) = f(T_2)$$
  Funktionsterm einsetzen:
  %%  $$2 \cdot{} b \cdot{} a^{\frac{0)\tau} = b \cdot{} a^{\frac{T_2}\tau}$$
  $$2 \cdot{} b \cdot{} a^{\frac{0}{\tau}} = b \cdot{} a^{\frac{T_2}{\tau}}$$
    $$2 = a^{\frac{T_2}\tau} \Longrightarrow  T_2 = \tau \cdot{} \log_a(2)$$
\end{bemerkung}
\newpage

%\begin{gesetz}{Vervielfachungszeit $T_q$}{}
%  Die Vervielfachungszeit $T_q$ kann durch die Vervielfachung $q$ ausgedrückt werden. Es gilt:
%  $$T_q = \tau\cdot{}\log_a(q)$$
%  und
%  $$f(t) = b\cdot{} q^{\frac{t}{T_q}}$$
%  \end{gesetz}

\subsection*{Aufgaben}

\GESO{\olatLinkArbeitsblatt{Exponentialfunktionen}{https://olat.bms-w.ch/auth/RepositoryEntry/6029794/CourseNode/106029175831971}{Kap. 1.6:
    Halbwertszeit/Verdopplungszeit: Aufg. 22. - 25.}}
\TALS{\olatLinkArbeitsblatt{Exponentialfunktionen}{https://olat.bms-w.ch/auth/RepositoryEntry/6029786/CourseNode/106029175777725}{Kap. 1.6:
    Halbwertszeit/Verdopplungszeit: Aufg. 22. - 25.}}

\AadBMTA{352ff}{3. (Taucherin), 4. [Glasfaser ohne Teilaufgabe
    Eindringtiefe 4. b)] und 5. b) (radioaktiver Zerfall)}

\olatLinkGESOKompendium{3.4}{28ff}{42., 43. und 47.}

\GESO{\aufgabenFarbe{
    Maturaprüfung 2020, Aufg 11 (Käfer)\\
    Maturaprüfung 2018 (Serie 4), Aufg 10 (Cäsium 137)\\
    Maturaprüfung 2018 (Serie 2), Aufg 11 (Plutonium)\\
    Maturaprüfung 2018 (Serie 1), Aufg 11 (radioaktive Substanz)\\
    Maturaprüfung 2016, Aufg. 9 (Jod-131)
}}

\olatLinkGESOKompendium{3.4}{27ff}{32. bis 47.}

\newpage
