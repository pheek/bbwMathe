%% Sirup-Beispiel
\subsection{Mischtank}\index{Mischtank}\index{Sirup}\label{sirup_beispiel}
Wird ein volles Glas Wasser in ein gleich großes volles Glas Sirup geschüttet, so

%%\TRAINER{\bbwCenterGraphic{5cm}{allg/alg/potenzen_wurzeln/img/Schwapp.png}}%%
%%\noTRAINER{\bbwCenterGraphic{5cm}{allg/alg/potenzen_wurzeln/img/SchwappOhneFormel.png}}
\bbwCenterGraphic{9cm}{allg/funktionen/img/exp/Sirup.png}%%
\begin{center}{\footnotesize Bildlegende: Sirup mischen}\end{center}

geschieht erst mal etwas eher klebriges:
\begin{itemize}
  \item Das Wasser verdrängt den Sirup und
  \item das Sirupglas schwappt über.
\end{itemize}

Wenn man nun gleichzeitig im Sirupglas
umrührt, so mischt sich das Wasser mit dem Sirup und je länger man
Wasser einschüttet, umso verdünnter wird der Sirup.


Wie viel Sirup bleibt nun im Sirupglas?

\TNT{2.4}{
Am Ende bleibt ein Anteil von $$\frac1\e \approx 0.367\,879 = 36.7879\%$$ Sirup im
Glas.

M.\,a.\,W.: Ein Verhältnis von  Sirup:Wasser = $\left(\frac{1}{\e}\right) :
\left(1-\frac{1}{\e}\right)$

\vspace{1.5cm}
}

Diese Konstante wird auch in großen chemischen Mischtanks verwendet.
\newpage


\textbf{Begründung:}\\
1. Gedanke: Jedes eingefüllte Glas, vermindert die vorhandene
Sirupkonzentration um den selben Faktor. Ergo handelt es sich um
einen exponentiellen Zerfall.

\leserluft

2. Gedanke: Wir tauschen drei Mal $\frac13$ aus. Nehmen also im
\begin{itemize}
\item \textbf{ersten Schritt} $\frac13$ des Sirups weg (und ersetzen diesen mit Wasser).
  Es bleiben $\frac23$ Sirup. Den Rest füllen wir mit Wasser auf.
\item Im \textbf{zweiten Schritt} nehmen wir $\frac13$ des Gemisches
weg; es verbleiben also $\frac23$ von $\frac23$ an
Sirup-Konzentrat. Der Rest wird immer wieder mit Wasser aufgefüllt. Mit
anderen Worten: Es bleiben $\frac23 \cdot \frac23
= \left(\frac23\right)^2$ an Sirup\footnote{Man könnte hier auch argumentieren mit: «Wir nehmen von den $\frac23$ einen Drittel weg»: $\frac23 - (\frac13$ von $\frac23)$ = $\frac23 - (\frac13 \cdot\frac23) = \frac23 \cdot(1-\frac13)=\frac23\cdot\frac23$}.
\item Im \textbf{dritten Schritt} entnehmen wir wieder $\frac13$ des
Gemisches; es verbleiben wieder $\frac23$ vom bisherigen Sirup, also
$\frac23$ von $(\frac23)^2$ also $\left(\frac23\right)^3$.

Beim dreistufigen Gedankenexperiment verbleiben
$\left(\frac23\right)^3 = \left(1-\frac13\right)^3$ der ursprünglichen Konzentration.
\end{itemize}
\leserluft

3. Gedanke: Das Experiment vom vorherigen Gedanken können wir natürlich auch mit immer kleineren\TALS{, sogenannten infinitesimalen,} Schritten durchführen.
Mit Centilitern \zB im dl-Glas ersetzen wir 10 Mal je $\frac1{10}$. 
So verbleibt am Schluss $\left(1-\frac{1}{10}\right)^{10}\approx 0.35$ Sirup.

\GESO{Wenn wir (\zB mit dem Taschenrechner) die Schrittanzahl immer weiter vergrößern (und somit die pro Schritt ausgetauschte Menge immer verkleinern), so ergibt sich für 1000 Schritte ein Verhältnis von $\left(1-\frac{1}{1000}\right)^{1000}\approx 0.3677 \approx \frac1{\e}$. }
\TALS{Wenn wir die Schritte permanent erhöhen (und gegen Unendlich gehen lassen), so erhalten wir den Grenzwert (lat. Limes) von

$$\lim_{n\rightarrow\infty} \left(1-\frac{1}{n}\right)^n = \frac1{\e}$$
}
\newpage

\textbf{Aufgabe 1: Sirup}\\
Wie viel Wasser muss eingeschüttet werden, damit das auf der Flasche
angegebene Verhältnis von 1:6 (1 Teil Sirup, 6 Teile Wasser) zustande
kommt?

\TNT{8}{
Bei 1x Schütten, erhalten wir $\left(\frac{1}{\e}\right)^1$ Anteil Sirup.

Bei 2x Schütten, erhalten wir $\left(\frac{1}{\e}\right)^2$ Anteil Sirup.

Somit erhalten wir den Siebtel (1:6 = $\frac17$-Anteil) indem wir die
folgende Exponentialgleichung lösen:

$$\frac17 = \left(\frac{1}{\e}\right)^n$$
Diese Gleichung lösen wir, indem wir beidseitig logarithmieren und so
erhalten wir den einzuschüttenden Teil $$n=\ln(7)\approx{1.946}.$$
}%% END TNT

\textbf{Aufgabe 2: Minerige-Haus}\\
Wenn wir also wissen wollen, wie viel Luft in ein Minergiehaus
eingepumpt werden muss, damit nur noch 1 Promille der alten Luft
vorhanden ist, so erhalten wir (unter der Annahme, dass sich die alte
und die neue Luft permanent mischen) ...

$$\text{Volumen Neuluft} = \text{Wohnungsvolumen} \cdot{} \LoesungsRaum{\ln(1000)}$$
\TNTeop{
  $$\ln(1000) \approx 6.9$$

  Wir pumpen also ca. 7x das Wohnungsvolumen an Luft hinein, bis nur
  noch 1 Promille der ursprünglichen Luft vorhanden ist.
} %% end TNT


\newpage
