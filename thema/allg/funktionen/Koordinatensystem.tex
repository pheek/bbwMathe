%%
%% 2019 07 04 Ph. G. Freimann
%%

\section{Koordinatensystem}\index{Koordinatensystem!kartesisches}\index{Kartesisches
  Koordinatensystem}
\sectuntertitel{Was man nicht gut beschreiben kann, kann man auch
  nicht messen (René Descartes: franz. Philosoph und Mathematiker).}
%%%%%%%%%%%%%%%%%%%%%%%%%%%%%%%%%%%%%%%%%%%%%%%%%%%%%%%%%%%%%%%%%%%%%%%%%%%%%%%%%
\subsection*{Lernziele}

\begin{itemize}
\item $x$-Achse und $y$-Achse\index{Achse}\index{x-Achse}\index{y-Achse}
\item Einheitsstrecke ($e_x$, $e_y$)\index{Einheitsstrecke!im Koordinatensystem}
\item kartesisch (nach René Descartes)\index{kartesisch}
\item Quadrant \index{Quadrant}
\item Ursprung\index{Ursprung}, Nullpunkt
\end{itemize}

%%\TALS{(\cite{frommenwiler17alg} S. 165 (Kap. 3.1))}
%%\GESO{(\cite{marthaler21alg}       S.211 (Kap. 13.1))}
\TadBMTA{211}{13.1}
\newpage

\subsection{Das kartesische Koordinatensystem}
In einem Koordinatensystem kann man Zahlenpaare $(x; y)$ als Punkte
darstellen.

\begin{definition}{Punkt im Koordinatensystem}{}
  Punkte in der $x|y$-Ebene können als geordnete Zahlenpaare aufgefasst
  werden. Sie werden bezeichnet als

  $$P=(x|y)$$
  bzw.
  $$P=(x;y).$$

\end{definition}


\begin{bemerkung}{}{}
  Auch wenn in einigen Lehrbüchern  (s. \cite{frommenwiler17alg})die Bezeichnung $P(x/y)$ für Punkte gebräuchlich ist, ist
  sie jedoch die unglücklichste von allen. Gerade bei $P=(1/2,5)$
  treten im deutschsprachigen Raum Verwechslungen auf. Ist $P=(1/2,5)$
  nun gleich $P=(1 | 2.5)$ oder gleich $P=(\frac{1}{2} | 5)$?
  Daher verwende ich lieber den senkrechten Strich, was auch in den
  BMS Abschlussprüfungen so gehandhabt wird. Im Lehrbuch (\cite{marthaler21alg}) wird jedoch
  konsequent das Semikolon ($;$ = Strichpunkt) verwendet, was auch
  nicht zu Verwechslungen führen kann.
\end{bemerkung}
\newpage

\subsection{Bezeichnungen im Koordinatensystem}

\TRAINER{\begin{center}\index{Quadrant}\index{$x$-Achse}\index{y-Achse}\index{Ursprung}
  \includegraphics[width=16cm]{allg/funktionen/img/Koordinatensystem.png}
  \end{center}
  \vspace{15mm}
}
\noTRAINER{\bbwGraph{-7}{8}{-4}{5}{}}

Dabei bezeichnen $e_x$ und $e_y$ die Einheiten (\zB cm, km, CHF, kg, ...) der entsprechenden Achsen.

Die $x$-Koordinate wird auch Abszisse\index{Abszisse} und die $y$-Koordinate wird auch Ordinate\index{Ordinate} genannt.

\subsection*{Aufgaben}

%%\TALSAadBMTA{165}{577. e) f) 578. b) 579. a) c) 580. a) c) 581. b) 582.}
\TALS{%% TALS
\olatLinkArbeitsblatt{Koordinatensystem}{https://olat.bbw.ch/auth/RepositoryEntry/572162090/CourseNode/108030907897983}{1.1
  bis 1.5}

Zusätzlich:

\AadBMTA{228ff.}{10. a) c) }
}%% end TALS
\GESO{%%
\olatLinkArbeitsblatt{Koordinatensystem}{https://olat.bbw.ch/auth/RepositoryEntry/572162163/CourseNode/108030907866742}{1.1 bis 1.5}
  %%\AadBMTA{228ff.}{3.a) b), 5. a) d) e)}%%
}%% end GESO

\newpage
