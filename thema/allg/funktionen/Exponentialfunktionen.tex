%%
%% 2019 07 04 Ph. G. Freimann
%%

\section{Exponentialfunktionen}\index{Funktion!Exponentialfunktion}\index{Exponentialfunktionen}
\sectuntertitel{Go viral!}
%%%%%%%%%%%%%%%%%%%%%%%%%%%%%%%%%%%%%%%%%%%%%%%%%%%%%%%%%%%%%%%%%%%%%%%%%%%%%%%%%
\subsection*{Lernziele}

\begin{itemize}
\item Definition Exponentialfunktion
\item Koeffizienten interpretieren
\item Graph: Symmetrien, Polstellen, Asymptoten, Schnittpunkte mit
  Achsen
  \item Basiswechsel
\end{itemize}

\TALS{(\cite{frommenwiler17alg} S.215 (Kap. 3.10))}
\GESO{(\cite{marthaler21alg}       S.322 (Kap. 19))}
\newpage

\subsection{Punkte einsetzen}
Wie bei den linearen Funktionen oder bei den Potenzfunktionen können auch Exponentialfunktionen gefunden werden, wenn bereits Punkte auf dem Graphen bekannt sind:

\textbf{Referenzaufgabe}

Finden Sie die Parameter $a$ und $b$ der Exponentialfunktion
$$f: y=b\cdot{}a^x$$
wenn Sie wissen, dass die Funktion durch die Punkte $P=(2|4)$ und $Q=(-1|2)$ verläuft:

\TNTeop{
  Zunächst wie immer die Punkte in die Gleichung einsetzen:

  \gleichungZZ{4}{b\cdot{}a^2}{2}{b\cdot{}a^{-1}}

  Einsetzverfahren: Zum Beiaspiel aus der ersten Gleichung das $b$ ermitteln...
  $$b = \frac4{a^2} \hspace{2cm} (I)$$
  ... und in die zweite Gleichung einsetzen:
  $$2 = \frac4{a^2} \cdot{} a^{-1}$$
  $$2 = 4\cdot{}a^{-3}$$
  $$1 = 2\cdot{}a^{-3} | \cdot{} a^3$$
  $$a^3 = 2| \sqrt[3]{}$$
  $$a = \sqrt[3]2$$
  Dies setzen wir noch in $(I)$ ein:
  $$b = \frac4{a^2} = \frac4{(\sqrt[3]{a})^2} \approx 2.52$$
  Ergo: $$f(x) = b\cdot{}a^x = \frac4{(\sqrt[3]{a})^2}  \cdot{} \left(\sqrt[3]2\right)^x$$
}


\newpage


\subsection{Basiswechsel}
Die Algen (im Türlersee) verdoppeln sich alle fünf Tage.

$$\tau = 5; a= 2$$

Die Exponentialfunktion bei Startwert $b$ lautet also:

\TNT{2.4}{$$f(t) = b\cdot{} 2^\frac{t}5$$}


Um wie viel nehmen sie ...

\begin{enumerate}
\item ... alle 15 Tage zu?
\item ... alle 7 Tage zu?
\item In wie vielen Tagen verdreifachen sie sich?
\item Wie sieht es mit der Basis $\e$ aus? Mit welchem $m$ kann man die
  Zunahme als $f(t) = b \cdot{} \e^{m\cdot{}t}$ angeben?
\end{enumerate}
\newpage


\textbf{Lösungen}

@ 1.: $\tau_2=15$:
\TNT{3.2}{$2^\frac15 = a_2^\frac1{15}$ Nun auf beiden Seiten ``hoch
  15'' und somit

  $a_2 = (2^{\frac15})^{15} = 2^3 = 8$.

Ergo $b\cdot{}2^\frac{t}5 = b\cdot{} 8^\frac{t}{15}$}

@ 2.: $\tau_2=7$:
\TNT{3.2}{$2^\frac15 = a_2^\frac1{7}$ Nun auf beiden Seiten ``hoch
  7'' und somit

  $a_2 = (2^{\frac15})^{7} = 2^{\frac75} \approx 2.639$.

Ergo $b\cdot{}2^\frac{t}5 \approx b\cdot{} 2.639^\frac{t}{7}$}

@ 3.: $a_2 = 3$ (verdreifachen):
\TNT{4}{$2^\frac15 = 3^\frac{1}{\tau_2}$ (Definition Logarithmus):

  $\frac1{\tau_2} = \log_3\left(2^\frac15\right)$

  $\tau_2 = \frac1{\log_3\left(2^\frac15\right)}\approx 7.925$ Tage.

Also: Sie verdreifachen sich ca. alle 8 Tage.

  Ergo: $b\cdot{}2^\frac{t}5 \approx b\cdot{}3^\frac{t}{7.925}$
}%% end TNT

@ 4.: $b\cdot{}a^\frac{t}\tau = b\cdot{}\e^{m\cdot{}t}$ (Basis $\e$):
\TNT{4.8}{
  Beidseitig durch $b$ teilen: \\
  $a^\frac{t}\tau = \e^{m\cdot{}t}$ (Zahlen einsetzen)\\
  $2^\frac{t}5 = \e^{m\cdot{}t}$ (beidseitig $\sqrt[t]{}$) :\\
  $a^\frac15 = \e^m$ (beidseitig $\log_{\e}() = \ln()$):\\

  $\ln(a^\frac1\tau)=\ln(2^\frac15) = \ln(\e^m) = m \approx 0.1386$ und somit:
  $$f(t) \approx b\cdot{} \e^{0.1386\cdot{} t} \approx
  b\cdot{}2^\frac{t}5$$
  Bedeutung: Gerade $g(t): y=0.1386\cdot{}t+b$ zeichnen: Dies ist die
  Tangente an die Exponentialfunktion $b\cdot{}2^\frac{t}{5} = b\cdot{}4^\frac{t}{10}$
}%% END TNT


\subsection*{Aufgaben}
\TALSAadBMTA{215ff}{809-830}
\GESOAadBMTA{335ff}{11. a) b), 15. a) b) [Zeichnen mit \texttt{geogebra.org}], 17. a) b) e), 22. a) b), 23. d), 25. a) b)}

\newpage

\begin{center}\fbox{
    \textbf{ $\tau$ und $a$ sind voneinander abhängig!}}
  \end{center}

Es gilt:
\begin{center}\fbox{$b\cdot{}a^{\frac{t}{\tau}} = b\cdot{}a_2^{\frac{t}{\tau_2}} =
    b\cdot{}\e^{m\cdot{}t}$}
\end{center}

... oder nach ziehen der $t$-ten Wurzel:

\begin{gesetz}{Basiswechsel}{}

  \begin{center}\fbox{\fbox{$a^{\frac{1}{\tau}} =     a_2^{\frac{1}{\tau_2}} = \e^{m}$}}\end{center}

 
  $a=$ Wachstumsfaktor\\
  $\tau=$ Beobachtungszeitraum\\
  $a_2, \tau_2=$ Entsprechend neuer Wachstumsfaktor $a_2$ bzw. neues
  Beobachtungsintervall $\tau_2$ nach dem Basiswechsel\\
  $\e=$ Eulersche Konstante (2.718...)\\
  \hrule
  \vspace{3mm}
  $a_2 = \left(a^\frac1\tau\right)^{\tau_2}$\\
  \vspace{3mm}
  $\tau_2=1 : \left(\log_{a_2}\left(a^\frac1\tau\right)\right)$\\
  \vspace{3mm}
  $m=\ln\left(a^\frac1\tau\right) =  \ln\left(a_2^\frac1{\tau_2}\right)$\\
\end{gesetz}

\begin{bemerkung}{Konstante $\e$}{}
  Mit der Eulerschen Konstanten $\e$ als Basis wird lediglich ein einziger
  Parameter (meist $m$ oder $q$) verwendet, um die Form der
  Exponentialfunktion zu beschreiben. Eine Abhängigkeit von $a$ zu
  $\tau$ fällt weg.

  Dabei ist $m$ die Tangentensteigung an die Funktion $$f(t) = \e^{mt}$$
  im Punkt $$\left(0\middle|1\right).$$
  \end{bemerkung}

\newpage


\subsection{Verschiebung und Streckung \GESO{(optional)}}

Eine Verschiebung der Exponentialfunktion $y=b\cdot{}a^x$ in der Zeit ($x$-Richtung) kann auch in Form einer Veränderung der Startfaktors $b$ umgeschrieben werden.

Verschieben wir \zB $$y=2^x$$ um fünf Einheiten nach rechts, so liest sich die neue Funktionsgleichung wie folgt:
$$y=2^{x-5}.$$

(Zeichnen Sie in \texttt{geogebra.org} a) $y=2^x$ und b) $y=2^{x-5}$.)

Dies kann jedoch auch umgeschrieben werden:

$$2^{x-5} = 2^x \cdot{} 2^{-5} = 2^{-5} \cdot{} 2^x = \frac{1}{2^5} \cdot{} 2^x =
\frac{1}{32}\cdot{}2^x$$

\bbwCenterGraphic{8cm}{allg/funktionen/img/exp/verschiebung_gleich_streckung.png}
Bildlegende: Eine Verschiebung ($x$-Richtung) der Exponentialfunktion entspricht einer Stauchung ($y$-Richtung) der selben Exponentialfunktion.

\TALS{Es gilt hier $$a^{x-b}=\frac{a^x}{k}$$ mit
$k=a^b$ und mit $b=\log_a(k)$.}

\newpage

\subsection{Bedeutung der Zahl $\e$ (optional)}

\paragraph{$45\degre$ auf der $y$-Achse:}
Die E-Funktion $y=\e^x$ hat die $45\degre$-Steigung genau auf der $y$-Achse; sie beginnt im Gegensatz zu anderen Exponentialfunktionen genau bei $x=0$ so «richtig zu wachsen»\TALS{\footnote{Wenn wir in einem Punkt $x$ den Wert der $\e$-Funktion ($\e^x$) betrachten, so ist ihr Funktionswert exakt gleich der Steigung einer Tangente bei diesem $x$-Wert.}}.

\bbwGraph{-4}{3}{-1}{5}{
  \bbwFunc{pow(2.71828,\x)}{-3.5:1.5}
  \bbwFuncC{\x+1}{-2:2}{green}
}%% end graph

Dies ist ein weiterer Grund, warum die Zahl $\e$ eine solche Bedeutung bei
exponentiellen Prozessen einnimmt.


\TALS{
\paragraph{Tangente an die E-Funktion}\index{E-Funktion}
Als \textbf{E-Funktion} bezeichnen wir die Exponentialfunktion zur
Basis $\e \approx 2.71828172846$; also:
$$f(x) = \e^x$$
Diese Funktion ist von allen Exponentialfunktionen insofern speziell,
als dass für jedes $x$ der Funktionswert $y=\e^x$ genau die Steigung
der Tangente an $\e^x$ im Punkt $P_x=\left(x | \e^x\right)$ angibt.

So ist die Tangentensteigung im Punkt $(0|\e^0) = (0 | 1)$ genau 1 und die Steigung im
Punkt $(1|\e^1) = (1|\e)$ genau $\e$. 

}%% END TALS

\newpage

\subsubsection{Beispiel: Wissenschaftliche Publikation (optional)}
In einem wissenschaftlichen Journal ist angegeben, dass eine Population annähernd folgendes Verhalten an den Tag lege [$t$ = Tage]:
$$y = 5.6\cdot{} \e^{\frac{t}{4.1}}$$

Berechnen Sie die tägliche Zunahme $a$ und geben Sie somit die Funktionsgleichung ohne Bruch im Exponenten ($\tau = 1$) und mit dem neu berechneten $a$ in der Basis an:

$$y = 5.6\cdot{} \e^{\frac{t}{4.1}} = b\cdot{}a^t \approx \LoesungsRaum{5.6} \cdot{} \LoesungsRaum{1.2762}^t$$

\TNTeop{Tipp: Setzen Sie $t=0$ und $t=1$ ein. $$a = \e^{\frac{1}{4.1}}$$
\vspace{3cm}}



\subsubsection*{Aufgaben}
\GESO{\olatLinkArbeitsblatt{Exponentialfunktionen}{https://olat.bbw.ch/auth/RepositoryEntry/572162163/CourseNode/106029175831971}{Kap. 2.: E-Funktion}}
\TALS{\olatLinkArbeitsblatt{Exponentialfunktionen}{https://olat.bbw.ch/auth/RepositoryEntry/572162090/CourseNode/106029175777725}{Kap. 2.: E-Funktion}}


\newpage
