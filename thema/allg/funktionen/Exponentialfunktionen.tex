%%
%% 2019 07 04 Ph. G. Freimann
%%

\section{Exponentialfunktionen}\index{Funktion!Exponentialfunktion}\index{Exponentialfunktionen}
\sectuntertitel{Go viral!}
%%%%%%%%%%%%%%%%%%%%%%%%%%%%%%%%%%%%%%%%%%%%%%%%%%%%%%%%%%%%%%%%%%%%%%%%%%%%%%%%%
\subsection*{Lernziele}

\begin{itemize}
\item Definition Exponentialfunktion
\item Koeffizienten interpretieren
\item Graph: Symmetrien, Polstellen, Asymptoten, Schnittpunkte mit
  Achsen
  \item Basiswechsel
\end{itemize}

\TadBMTA{322}{19}
%%\TALS{(\cite{frommenwiler17alg} S.215 (Kap. 3.10))}
%%\GESO{(\cite{marthaler21alg}       S.322 (Kap. 19))}
\newpage

\subsection{Aussehen von Exponentialfunktionen}


Zeichnen Sie $f: y=2^x$ und $g: y=1.4^x$ ins selbe Koordinatensystem:


\bbwGraph{-5}{5}{-1}{5}{
\TRAINER{  \bbwFuncC{pow(2.0,\x)}{-3.5:2.1}{blue}
  \bbwFuncC{pow(1.4,\x)}{-3.5:3}{green}
  \bbwLetter{1,3}{$f$}{blue}
  \bbwLetter{2.5,2}{$g$}{green}
}%% end TRAINER
}%% end graph

\begin{definition}{Exponentialfunktion}{}\index{Exponentialfunktionen!Definition}
  Eine Funktion der Form $$f(x): x \mapsto a^x$$
  bzw. $$y = a^x$$
  mit $a\in \mathbb{R}^{+}\backslash\{1\}$ heißt \textbf{Exponentialfunktion}.
\end{definition}


\begin{bemerkung}{Zunahmefaktor}{}\index{Zunahmefaktor|textbf}
Der Parameter $a$ gibt den Wachstumsfaktor pro Zeiteinheit $e_x$ an.
\end{bemerkung}

\newpage

\subsection{Formen der Darstellung}
Zeichnen Sie die Funktionen $f: y=2^x$, $g: y=2^{-x}$ und $h: y=\left(\frac12\right)^x$ in dasselbe Koordinatensystem:

\bbwGraph{-6}{6}{-1}{5}{
  \TRAINER{\bbwFuncC{exp(0.69314718*\x)}{-6:2}{blue}}
  \TRAINER{\bbwFuncC{exp(-0.69314718*\x)}{-2:6}{red} }
  \TRAINER{\bbwLetter{1.5,4}{2^x}{blue}}
  \TRAINER{\bbwLetter{-5,4}{g(x)=2^{-x}=\left(\frac{1}{2}\right)^x=h(x)}{red}}
}

Bemerkung:
\TNT{2}{$$2^{-x} = \frac1{2^x} = \frac{1^x}{2^x}= \left(\frac12\right)^x$$}
%%\newpage



\subsubsection{Spiegelung an der $y$-Achse\GESO{ (optional)}}

\begin{bbwFillInTabular}{c|c}
  Original & Spiegelbild \\\hline
  $f(x)$ & $g(x)$ \\\hline
  $a^x$  & \TRAINER{$a^{-x}$}\noTRAINER{\hspace{3cm}}  \\\hline
  $a^x$  & \TRAINER{$\left({\frac1a}\right)^{x}$}\noTRAINER{\hspace{3cm}}  \\
  \end{bbwFillInTabular}


\newpage

\subsection{Verschiebung und Streckung \GESO{(optional)}}

Eine Verschiebung der Exponentialfunktion $y=b\cdot{}a^x$ in der Zeit ($x$-Richtung) kann auch in Form einer Veränderung der Startfaktors $b$ umgeschrieben werden.

Verschieben wir \zB $$y=2^x$$ um \textbf{fünf Einheiten nach rechts}, so liest sich die neue Funktionsgleichung wie folgt:
\TNT{1.2}{$$y=2^{x-5}.$$}

Zeichnen Sie noch $y=2^{x-5}$:

\TRAINER{\bbwCenterGraphic{10cm}{allg/funktionen/img/exp/verschiebung_gleich_streckung.png}
}%%
\noTRAINER{\bbwFunction{-1}{10}{-1}{5}{exp(0.69315*\x)}{-1:2.322}
}%% end NoTrainer


$2^{x-5}$ kann jedoch auch umgeschrieben werden:

\TNT{2}{
$$2^{x-5} = 2^x \cdot{} 2^{-5} = 2^{-5} \cdot{} 2^x = \frac{1}{2^5} \cdot{} 2^x =
\frac{1}{32}\cdot{}2^x$$
}%% end TNT

Eine Verschiebung ($x$-Richtung) der Exponentialfunktion entspricht einer Stauchung ($y$-Richtung) der selben Exponentialfunktion.

\TALS{Es gilt hier $$a^{x-b}=\frac1k \cdot{} a^x$$ mit
$k=a^b$ und mit $b=\log_a(k)$.}

\TALS{
\AadBMTA{336}{18. bis 20.}
}%% end TALS


\newpage


\subsection{Punkte einsetzen\GESO{ (optional)}}
\TRAINER{Hier bei Infortmatikern-Klassen ev. KI zeigen: \href{https://github.com/pheek/genetischerAlgorithmus}{genetischer Algorithmus}}

Wie bei den linearen Funktionen oder bei den Potenzfunktionen können auch Exponentialfunktionen gefunden werden, wenn bereits Punkte auf dem Graphen bekannt sind:

\textbf{Referenzaufgabe}

Finden Sie die Parameter $a$ und $b$ der Exponentialfunktion
$$f: y=b\cdot{}a^x$$
wenn Sie wissen, dass die Funktion durch die Punkte $P=(2|4)$ und $Q=(-1|2)$ verläuft:

\TNTeop{
  In Gleichung einsetzen:

  \gleichungZZ{4}{b\cdot{}a^2}{2}{b\cdot{}a^{-1}}

  Einsetzverfahren: Zum Beispiel aus der zweiten Gleichung das $b$ ermitteln...
  $$b = 2 a (III)$$
  ... und in die erste Gleichung einsetzen:

$$4 = 2a\cdot{} a^2$$
  $$2 = a^3$$
  $$a=\sqrt[3]{2}$$
  In (III) einsetzen:

  $$b = 2 \sqrt[3]{2}$$

  Ergo: $$f(x) = b\cdot{}a^x = 2\cdot{}\sqrt[3]{2} \cdot{} \left(\sqrt[3]2\right)^x \approx 2.5198 \cdot{} 1.2599^x$$
}


\TALS{
  \subsection*{Referenzaufgabe}
  Gesucht $q$, sodass $y=e^{q\cdot{}x}$ durch durch den Punkt
  $\left(2\middle|\frac1{\e}\right)$ verläuft.

  \TNTeop{
    $$\frac1{\e} = e^{q\cdot{}2}$$
    $$e^{-1} =  e^{2q}$$
    Exponentenverlgeich
    $$-1 = 2q$$
    $$q = \frac{-1}2$$
  }%% end tntEOP
  \newpage}%% end TALS


\subsection*{Aufgaben}
%%\AadBMTA{334}{10. a) b)\TALS{ e)}\GESO{ f)}}
\GESO{\olatLinkArbeitsblatt{Exponentialfunktionen}{https://olat.bms-w.ch/auth/RepositoryEntry/6029794/CourseNode/106029175831971}{Kap. 2.1: E-Funktion / Punkte-Aufgaben Aufg. 26. a) b) ($e^{qt}$ optional) und Aufg. 27.}}
\TALS{\olatLinkArbeitsblatt{Exponentialfunktionen}{https://olat.bms-w.ch/auth/RepositoryEntry/6029786/CourseNode/106029175777725}{Kap. 2.1: E-Funktion / Punkte-Aufgaben Aufg. 26. a) b) und Aufg. 27.}}

\AadBMTA{335 ff}{11. a) b),\\15. a) b) [Zeichnen mit \texttt{geogebra.org}],\\ 23. d), 25. a) b)}
\TRAINER{Diese Aufgabe 23. d) noch ins Aufgabenblatt integrieren}
\newpage
