
%%%%%%%%%%%%%%%%%%%%%%%%%%%%%%%%%%%%%%%%%%%%%%%%%%%%%%%%%%%%%%%%%%%%%%%%%%%%%%%%%%%

\subsection{Zusammenfassungen und Überblick}\index{Potenzfunktionen!Überblick}

$$y = a\cdot{}x^z$$

$z$ gerade: Achsensymmetrischer Graph an der $y$-Achse

$z$ ungerade: Punktsymmetrischer Graph am Ursprung $O=(0|0)$

Ist ${\color{ForestGreen}a>0}$\TRAINER{ (grün in der Graphik)}, so sind Graphen mit
geraden Exponenten nach oben geöffnet bzw. mit ungeraden Exponenten im
Quadranten I und III.

Ist ${\color{red}a<0}$\TRAINER{ (rot/orange in der Graphik)}, so sind Graphen mit
geraden Exponenten nach unten geöffnet bzw. mit ungeraden Exponenten
im Quadranten II und IV.


\TRAINER{%
  \includegraphics[width=8.5cm]{allg/funktionen/img/potenzfct/potenzFunktionenGerade.png}\hfill{}\includegraphics[width=8.5cm]{allg/funktionen/img/potenzfct/potenzFunktionenUngerade.png}
}%% END Trainer
\noTRAINER{%
  \includegraphics[width=8.5cm]{allg/funktionen/img/potenzfct/potenzFunktionenLeer.png}\hfill{}\includegraphics[width=8.5cm]{allg/funktionen/img/potenzfct/potenzFunktionenLeer.png}
}%% END Trainer


\subsection*{Aufgaben}
\GESO{\olatLinkArbeitsblatt{Potenzfunktionen zuordnen}{https://olat.bms-w.ch/auth/RepositoryEntry/6029794/CourseNode/111658377908379}{(alle Beispiele)}}%% END olatLinkArbeitsblatt
\TALS{\olatLinkArbeitsblatt{Potenzfunktionen zuordnen}{https://olat.bms-w.ch/auth/RepositoryEntry/6029786/CourseNode/111658377974876}{(alle Beispiele)}}%% END olatLinkArbeitsblatt



\AadBMTA{307}{25. b) c), 26. a)\TALS{ b)}, 27. b) c)\TALS{, 33*}}

\newpage

