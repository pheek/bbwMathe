
%% 2019 07 04 Ph. G. Freimann
%%

\section{Wachstum und Zerfall}\index{Wachstum}\index{Zerfall}
\sectuntertitel{Sagt ein großer Stift zum kleinen Stift: ``Wachsmalstift!''}
%%%%%%%%%%%%%%%%%%%%%%%%%%%%%%%%%%%%%%%%%%%%%%%%%%%%%%%%%%%%%%%%%%%%%%%%%%%%%%%%%
\TRAINER{Video Mathe Mann}%%
\subsection*{Lernziele}

\begin{itemize}
\item Zinseszins
\item Wachstums-, Zerfallsprozesse
\item Verdoppelungs- und Halbwertszeiten
%%\item Basiswechsel
\end{itemize}

\TALS{(\cite{frommenwiler17alg} S.221 (Kap. Exponentielles Wachstum))}
\TALS{(\cite{frommenwiler17alg} S.223 (Kap. Exponentielle Abnahme))}
\TALS{(\cite{frommenwiler17alg} S.225 (Kap. Zinseszins))}
\GESO{(\cite{marthaler21}       S.342 (Kap. 20))}


\subsection{Beispiele}
Bei ungebremsten Prozessen sprechen wir dann von einer exponentieller
Zunahme, wenn die Zunahme pro Zeiteinheit jeweils proportional zur aktuellen Anzahl ist.

\begin{itemize}
\item \Lueckentext{Zinseszins}
\item \Lueckentext{Frequenzen in der temperierten Stimmung (Musik). Zunahme der Frequenz pro Halbtonschritt.}
\item \Lueckentext{Keime in der Kuhmilch; Ansteckungsbedingte Krankheitsfälle (\zB viral)}
\item \Lueckentext{Algenbefall in Teichen}
\item \Lueckentext{(ungebremstes) Bevölkerungswachstum / bzw. Tierpopulation}
\item \Lueckentext{\dotfill}
\end{itemize}
\newpage

\subsection{Einstiegsbeispiel Taschengeld}

Bei Familie Cash kann man aussuchen, wie sich sein Taschengeld über
die Jahre «vermehrt». Mani wählt Variante A. Bei Varante A erhält man
CHF 1.- im ersten Jahr, CHF 2.- im 2. Jahr, CHF 3.- im 3. Jahr und so
weiter bis zur abgeschlossenen Grundbildung im 13. Jahr CHF 13.-.

Carla wählt Variante B. Bei der Variante B erhält Carla auch CHF 1.-
im ersten Jahr, dann aber jedes Jahr 30\% mehr, als im Vorjahr.

a) Wird Carla vor Ende der Grundbildung jemals mehr als Mani erhalten?

\LoesungsRaumLang{Ja, im 10. Schuljahr}

b) Wer hat über alle Jahre mehr Taschengeld?

\LoesungsRaumLang{Carla wird mehr haben}

c) Skizzieren Sie beide Varianten:

\TRAINER{\bbwCenterGraphic{16cm}{allg/funktionen/img/taschengeldAusgefuellt.png}}
\noTRAINER{\bbwCenterGraphic{16cm}{allg/funktionen/img/taschengeld.png}}


\newpage

\TRAINER{Optional: Zeige mit Geogebra (\texttt{geogebra.org}) $x^2$ vs. $1.2^x$. Fazit:
  Exponentielles Wachstum überholt jegliche Potenzfunktion.}


\subsubsection{Rate vs. Faktor I}\index{Rate!Zunahme}\index{Faktor!Zunahme}\index{Zunahmefaktor}\index{Zunahmerate}

Vervollständigen Sie die folgende Tabelle:

\begin{tabular}{|c|c|}\hline
  Wachstums\textbf{rate} & Wachstums\textbf{faktor}\\\hline
  15\%                   & 1.15\\\hline
  8\%                    & \LoesungsRaum{1.08}\\\hline
  \LoesungsRaum{100\%}   & 2\\\hline
  -3\%                   & \LoesungsRaum{0.97}\\\hline
  \LoesungsRaum{-90\%}   & 0.1\\\hline%%
\end{tabular} 

\newpage
\begin{definition}{Exponentialfunktion}{}\index{Exponentialfunktion!Definition}
  Eine Funktion der Form $$f(x): x \mapsto a^x$$
  bzw. $$y = a^x$$
  heißt \textbf{Exponentialfunktion}.
\end{definition}

Zeichnen Sie $f: y=2^x$ und $g: y=1.4^x$ ins selbe Koordinatensystem:


\bbwGraph{-5}{5}{-1}{5}{
\TRAINER{  \bbwFuncC{pow(2.0,\x)}{-3.5:2.1}{blue}
  \bbwFuncC{pow(1.4,\x)}{-3.5:3}{green}
  \bbwLetter{1,3}{$f$}{blue}
  \bbwLetter{2.5,2}{$g$}{green}
}%% end TRAINER
}%% end graph



\begin{bemerkung}{Wachstumsfaktor}{}\index{Wachstumsfaktor}
Der Parameter $a$ gibt den Wachstumsfaktor pro Zeiteinheit $e_x$ an.
\end{bemerkung}

\newpage

\subsection{Exponentialfunktion mit anderen Startwerten}

Obige Exponentialfunktion hat für den Wert $x=0$ immer den $y$-Wert = 1. Da dies in der Praxis aber meist nicht der Fall ist, gib es die folgende \textbf{allgemeine Form der Exponentialfunktion}:

\begin{definition}{allgemeine Exponentialfunktion}{}
$$f: y = b\cdot{}a^x$$

Auch hier gibt der Parameter $a$ den Wachstumsfaktor pro Zeiteinheit $e_x$ an.

Dabei ist $b$ der Startwert zum Zeitpunkt $x$ = 0.
\end{definition}


\noTRAINER{\bbwCenterGraphic{8cm}{allg/funktionen/img/exp/b_faktor.png}}
\TRAINER{\bbwCenterGraphic{8cm}{allg/funktionen/img/exp/b_faktor_trainer.png}}

\begin{bemerkung}{Wachstumsfaktor}{}{}
Werden zwei $x$ Positionen mit Differenz 1 (=$e_x$) betrachtet, so sind
die zugehörige $y$-Werte um \textbf{Faktor} $a$ auseinander.
\end{bemerkung}

Begründung:
\TNT{4}{
  Gegeben $x_1$ und $x_2 = x_1 + 1$. So ist

  $y_1 = b\cdot{}a^{x_1}$ und $y_2 = b\cdot{}a^{x_2}$.

  Setzen wir nun $x_1 + 1$ für $x_2$ ein, so erhalten wir:

  $$y_2 = f(x_2) = b\cdot{}a^{x_2} = b\cdot{} a^{x_1+1} =  b\cdot{}a^{x_1} \cdot{} a^1 = y_1\cdotp{} a$$
}
\newpage


\GESO{
  \subsection*{Aufgaben}
  Als Vorbereitung zur allgemeinen Wachstumsfunktion:
  \GESOAadB{207}{10. (Algen)}
}


\subsection{Einstiegsbeispiel Wachstumsprozesse}

\bbwCenterGraphic{17cm}{allg/funktionen/img/tuerlersee2.jpg}
\begin{center}{\small Legende: Türlersee April 2022}\end{center}

Der Türlersee ist ein kleiner See im Repischtal. Seine Oberfläche
begann sich vor einigen Jahrzehnten stark mit Algen\footnote{Genau
  genommen handelte es sich um Zooplanktonbiomasse zwischen 1982 und 1994, doch als
  Idee zur Exponentialfunktion sollte ein ungefähres Flächenmodell reichen.} zu bedecken.

Anfänglich (zum Zeitpunkt $t=0$) waren gerade mal 20$m^2$ bedeckt. Doch nach fünf Tagen hatte sich diese Fläche verdoppelt und nach weiteren fünf Tagen nochmals verdoppelt (also insgesamt vervierfacht).

Füllen Sie die (prognostizierte) Wertetabelle für 30 Tage ein:

\def\spaceX{\,\,\,\,\,\,\,\,\,\,}
\newcommand\tuerlerB[1]{\noTRAINER{\spaceX}\TRAINER{#1}}
\begin{tabular}{l|c|c|c|c|c|c|c}
  $t$:  & 0 & 5 & 10 & 15 & 20 & 25 & 30 \\
  \hline
  $m^2$ & \tuerlerB{20}  & \tuerlerB{40}  &   \tuerlerB{80}  &  \tuerlerB{160}  &  \tuerlerB{320}  &  \tuerlerB{640}  &  \tuerlerB{1280} \\
\end{tabular}

\newpage
Zeichnen Sie die Algenpopulation als Graph in eine Koordinatensystem
(beginnen Sie mit dem Ursprung ganz links unten. Eine $x$-Einheit,
also ein Tag,  nach rechts entspricht einem Häuschen und in $y$-Richtung nehmen Sie 2 Häuschen für 100$m^2$ Algenfläche:

\TNT{10}{\bbwGraphic{10cm}{allg/funktionen/img/tuerlerAlgen.png}}

Geben Sie eine Funktionsgleichung an, welche das Wachstum der Algenpopulation beschreibt:

\begin{center}
  $f:\,\,\, y=\LoesungsRaumLang{20\cdot{} 2 ^{\frac{t}{5}}}$
  \end{center}
\TRAINER{
Dabei bezeichnet 20 den Startwert, 2 den Zunahmefaktor und 5 die Beobachtungszeitspanne.}%%

\newpage



\subsubsection{Grundform exponentieller Wachstumsprozesse}

Die allgemeine Formel für exponentielle Prozesse lautet:

\begin{definition}{Wachstumsprozess}{}
  $$y = b\cdot{}a^{\frac{t}{\tau}}$$
  bzw.
  $$f(t) = b\cdot{}a^{\frac{t}{\tau}}$$
  
\end{definition}


\begin{bemerkung}{}{}
Oft wird bei Wachstumsprozessen die Zeitachse auch mit $t$ statt $x$ bezeichnet: $t$ steht für \textit{time}.
\end{bemerkung}

\newpage

\subsubsection{Graphische Erläuterung}

\bbwCenterGraphic{11cm}{allg/funktionen/img/exp/exponentielles_wachstum.png}

Dabei sind
\begin{itemize}
\item $b=f(0)$ der Anfangsbestand zum Zeitpunkt $t$ = 0.
\item $\tau$ ist die typische Zeitspanne zwischen zwei Beobachtungszeitpunkten (zum Beispiel zwischen Wert und dessen Verdopplung). Beispiele:
  \begin{itemize}
  \item Verdoppeln in 3 Stunden: $a=2$ und $\tau = 3 [\textrm{h}]$
  \item Verfünf"|fachen einer Viertelstunde (= 15 Minuten): $a=5$ und
    $\tau=\frac{1}{4} [\textrm{h}]$
  \end{itemize}
  Meist ist die Beobachtungszeit gleichzeitig die Maßeinheit (\zB
  Veränderung pro Stunde). Dann ist unser $\tau=1$ und die Formel
  vereinfacht sich zu $f(t) = b\cdot{}a^t$.
\item $a$ ist der Zunahmefaktor zwischen zwei Beobachtungszeitpunkten (Beispiel $a=2$ bei Verdoppelungsprozessen).
  $a$ berechnet sich durch den Quotienten zwischen zwei
  Beobachtungswerten $a = \frac{f(\tau)}{f(0)} =\frac{m}{b}$. Dabei
  ist $m$ der Messwert zum Zeitpunkt $\tau$.
\item $f(t)=b\cdot{}a^{\frac{t}{\tau}}$ ist der Wert (Anzahl, Fläche,
  Bestand, ...) zum Zeitpunkt
  $t$. 
\end{itemize}

Bemerkung: 
\TNT{2.4}{
$\frac{m}b = \frac{f(\tau)}{f(0)}=\frac{b\cdot{}a^{\frac{\tau}{\tau}}}{b\cdot{}a^{\frac0{\tau}}}
  = \frac{a^1}{a^0} = \frac{a}{1} 
  = a$}%% END TNT

\newpage

\subsection{Referenzaufgabe}\index{Irland!Bevölkerungswachstum}
Irland hatte 1990 3.51 Mio. Einwohner. Im Jahr 2019 waren es bereits 4.93 Mio.

\textbf{Frage 1}: Was prognostizieren Sie für das Jahr 2025, wenn Sie von einem exponentiellen Wachstum ausgehen?

\TNT{12}{
  1. Skizze:Ersichtlich: $\tau=29$, 1990 = $t_0=0$, 2019 = $t_1=29$, 3.51 und 4.93 (Rest irrelevant)

  $b=3.51$ [Mio EW] = Startwert (Taschenrechner!).

  $a=\frac{4.93}{3.51} = 1.404...$ pro 29 Jahre (Taschenrechner!). 

  $\tau$ = 29 Jahr, Einheit = 1 Jahr ($t$ in Jahren gemessen).
  
  $$f(t) = b\cdot{}a^\frac{t}{\tau}=b\cdot{}a^\frac{t}{29}$$
  
    Jahr 2025: $t=35$:

    $$f(35) = b\cdot{}a^\frac{35}{29}\approx 5.28899$$

  }%% END TNT
\newpage

\textbf{Frage 2}: In wie vielen Jahren hat sich die Bevölkerung verdoppelt?

\TNT{12}{
  Gesucht $T_2$ = Verdopplungszeitpunkt.

  $$f(T_2) = 2\cdot{}f(0)$$

  Funktionsgleichung einsetzen:

  $$b\cdot{}a^\frac{T_2}{\tau} = 2\cdot{} b\cdot{} a^\frac0\tau$$

  Weil $a^\frac0\tau = 1$ folgt:

  $$b\cdot{}a^\frac{T_2}{\tau} = 2\cdot{} b$$

  Durch $b$ teilen:

  $$a^\frac{T_2}{\tau} = 2$$

  Def. Logarithmus:

  $$\frac{T_2}{\tau} = \log_a(2)$$

$$T_2 = \tau\cdot{}\log_a(2) = 29\cdot{}\log_a(2) \approx 59.17$$
Die Bevölkerung wird sich voraussichtlich alle 59.17 Jahren
verdoppeln.

}%% END Trainer
\newpage

\subsection*{Aufgaben}
\TALSAadB{221ff}{831 - 839}
\GESOAadB{338}{28. Bakterien}
\GESOAadB{352ff}{2. (Hasenpopulation), 7., 1. (optional)}
\GESO{\aufgabenFarbe{Kompendium: S. 27ff: 32., 34., 38., 40., 44., 45. und 46.}}
\GESO{\aufgabenFarbe{Nullserie 2: Aufgabe 8.}}
\GESO{\aufgabenFarbe{Maturaprüfung 2017, Aufg. 12 (Raupen)\\
Maturaprüfung 2018 (Serie 3), Aufg. 11 (Müll)
}}

\newpage


\subsection{Exponentieller Zerfall}\label{zerfallsfunktion}
Die Funktion $f(t): t \mapsto y = d^{-t}$ ist eine
Exponentialfunktion, die gegen Null geht.
\bbwFunction{-4}{4}{-1}{8}{exp(-\x)}{-2:4}

\newpage

\subsubsection{Beispiele}
\begin{itemize}
	\item \Lueckentext{Zinsliche Abschreibungen (\zB Wert eines Autos)}
	\item \Lueckentext{Radioaktiver Zerfall}
	\item \Lueckentext{Lichtintensität in Medium (Gas / Flüssigkeit / Glasfaser), dies gilt vertikal, wie auch horizontal}
	\item \Lueckentext{Atmosphärischer Luftdruck in Metern über Meer}
  \item \Lueckentext{Entladen einer Batterie bzw. eines Kondensators}
  \item \Lueckentext{Sauerstoffkonzentration in Seen (\zB Herbst bei kontinuierlicher Abnahme)}
  \item \Lueckentext{Abnahme des Bierschaums im Glas}
  \item \Lueckentext{Mischen, wie im Sirup-Beispiel\totalref{sirup_beispiel}}
  \item \Lueckentext{«Halbwertszeit des Wissens» ;-)}
  \item \Lueckentext{\dotfill}
\end{itemize}

\newpage



Zeichnen Sie die Funktionen $f: y=2^x$, $g: y=2^{-x}$ und $h: y=\left(\frac12\right)^x$ in dasselbe Koordinatensystem:

\bbwGraph{-6}{6}{-1}{5}{
  \TRAINER{\bbwFuncC{exp(0.69314718*\x)}{-6:2}{blue}}
  \TRAINER{\bbwFuncC{exp(-0.69314718*\x)}{-2:6}{red} }
  \TRAINER{\bbwLetter{1.5,4}{2^x}{blue}}
  \TRAINER{\bbwLetter{-5,4}{g(x)=2^{-x}=\left(\frac{1}{2}\right)^x=h(x)}{red}}
}

Bemerkung: \TRAINER{$$2^{-x} = \frac1{2^x} = \left(\frac12\right)^x$$}
\newpage
Die allgemeine Formel für Zerfallsprozesse lautet:

\begin{definition}{Zerfall}{}
$$y = b\cdot{}a^{\frac{t}{\tau}}$$
  bzw.
  $$f(t) = b\cdot{}a^{\frac{t}{\tau}}$$
\end{definition}

\bbwCenterGraphic{10cm}{allg/funktionen/img/exp/exponentieller_zerfall.png}

Dabei sind
\begin{itemize}
\item $b$ der Anfangsbestand zum Zeitpunkt $t$ = 0.
\item $\tau$ ist die Zeitspanne zwischen zwei typischen Beobachtungszeitpunkten (Zum Beispiel zwischen Wert und dessen Halbierung).
\item $a=\frac{f(\tau)}{f(0)}$ ist der Abnahmefaktor ($0<a<1$)
  zwischen zwei Beobachtungszeitpunkten (im Abstand von $\tau$). Bei
  Halbierungsprozessen \zB wäre $a=0.5$\footnote{Ist $a=\frac12$, so
  nennen wir das $\tau$ die «Halbwertszeit»\index{Halbwertszeit}, denn
  in dieser Zeitspanne halbiert sich der Funktionswert.}.
\end{itemize}

\begin{bemerkung}{Wachstum vs. Zerfall}{}

  Der einzige Unterschied bei Wachstums- bzw Zerfallsprozessen ist der
  Faktor $a$:

  \begin{itemize}
  \item $a>1$: Wachstum
  \item $0<a<1$: Zerfall
    \end{itemize}
  
  \end{bemerkung}
\newpage

\subsubsection{Umkehrung (Optional)}
Anstelle eines  positiven Exponenten ($\frac{+t}{\tau}$) kann genauso
gut der Kehrwert des Faktors $a$ genommen werden. Dann wird der
Exponent negativ, dafür wird die Basis $> 1$. Seien $m_1$ bzw. $m_2$
zwei Beobachtungswerte.

Mit $a = \frac{m_2}{m_1}$ und $d := \frac{m_1}{m_2} = \frac{1}{a}$, gilt

\begin{center}
  \fbox{$b\cdot{} a^{\frac{+t}{\tau}} = b\cdot{}d^{\frac{-t}{\tau}} $}
\end{center}

\TALS{Beweis:}\GESO{Begründung: So gilt \zB bei Halbierungsprozessen, dass $a=0.5$ und $d=2$:}

\TALS{ $b\cdot{}a^{\frac{t}{\tau}} = b\cdot{} \left(\frac{m_2}{m_1}\right)^{\frac{t}{\tau}} = b\cdot{} \left(\frac{m_1}{m_2}\right)^{\frac{-t}{\tau}} =  b\cdot{}d^{\frac{-t}{\tau}} $}
\GESO{ $b\cdot{}0.5^{\frac{t}{\tau}} = b\cdot{} \left(\frac{1}{2}\right)^{\frac{t}{\tau}} = b\cdot{} \left(\frac{2}{1}\right)^{\frac{-t}{\tau}} =  b\cdot{}2^{\frac{-t}{\tau}} $}
\newpage

\subsection{Rate vs. Faktor
  II}\index{Rate}\index{Fatkor}\index{Zunahmefaktor}\index{Zunahmerate}
\totalref{RateZins1}

Den Unterschied von Zinsfuß (= Rate) und Zinsfaktor kennen wir bereits aus der Zinsrechnung.

So entspricht eine Zunahme von 12\% einem\\
Aufzinsungsfaktor von \LoesungsRaumLang{1.12}.

Wenn jedoch eine Beobachtung einer Zunahme von, sagen wir, 70\% innerhalb einer Viertelstunde beobachtet wird, so können wir uns fragen, um wie viel die Zunahme (als Rate oder Faktor) pro Zeiteinheit (hier Stunden) ist.


Füllen Sie dazu folgende Tabelle aus. Dabei bedeuten

\begin{tabular}{lp{14cm}}\hline
  Einheit & Stunden, Minuten, Meter, ... \\\hline
  $\tau$  & In dieser Zeitspanne (Stunden, Meter, ...) wird beobachtet \\\hline
  $p$     & Zunahme\textbf{rate}\index{Zunahmerate}\index{Rate} während $\tau$ Einheiten in \%. Ist $p$ negativ, handelt es sich um eine Abnahme\\\hline
  $a_\tau$ & Zunahme\textbf{faktor}\index{Zunahmefaktor} während $\tau$ Einheiten. Ist $a<1$, handelt es sich um einen Abnahmefaktor\\\hline
  $a_E$ (Formel)   & Zunahme pro Einheit (als Faktor). Aufgeschrieben als Formel\\\hline
  $\approx a_E$ (Zahl)  & Zunahme pro Einheit (als Näherungswert).\\\hline
  $p_E$   & Prozentuale Zunahme pro Zeiteinheit\\\hline
  \end{tabular} 

\leserluft{}
\leserluft{}
%% temporäres Platzhalterchen 
\newcommand{\ph}[1]{\noTRAINER{...........}\TRAINER{#1}}

%%\renewcommand{\arraystretch}{1.7}
$$f(t) = a_\tau^{\frac{t}\tau} = a_E^t$$
%% probably turn off auto-fill-mode in emacs when editing long lines
\begin{bbwFillInTabular}{|l|l|l|l|l|l|l|}\hline
  Einheit & $\tau$            &  $p$         & $a_\tau$         & $a_E$ (Formel)           &  $\approx a_E$    &$p_E$            \\\hline\hline
  h       &  $3$              &  56\%        & $1.56$           & $1.56^\frac13$            &  1.1598           & 11.598\%        \\\hline 
  h       &  $\frac14 = 0.25$ &  70\%        & \ph{1.7}         & \ph{ $1.7^\frac1{0.25}$}  &  \ph{8.3521}      & \ph{735.21\%}   \\\hline 
  h       &  $\frac12$        &  \ph{50\%}   & 1.5              &  $1.5^\frac1{0.5}$        &  2.25             & \ph{125\%}      \\\hline 
  Tage    &  $5$              & 100\%        & \ph{2}           & \ph{$2^\frac1{5}$}        &  \ph{1.1487}      & \ph{14.87}\%    \\\hline 
  Min.    &  $2$              & \ph{-60}\%   & 0.4              & \ph{$0.4^\frac1{2}$}      &  \ph{0.6325}      & \ph{-36.75}\%   \\\hline 
  m       &  $12$             & \ph{4.5}\%   & \ph{1.045}       & $1.045^\frac1{12}$        &  \ph{1.00367}     & \ph{0.3675}\%   \\\hline
  Wochen  & \ph{$2$}          & \ph{$200$}\% & \ph{3}           & $3^\frac12$               & \ph{1.73205}      & \ph{73.205}\%   \\\hline
  Jahr    & \ph{$\frac1{12}$} & \ph{$1$}\%   & \ph{$1.01$}      & $1.01^{\frac1{1/12}}$      &  \ph{$1.127$}     & \ph{$12.7$}\%   \\\hline
\end{bbwFillInTabular} 


\newpage
%% Sirup-Beispiel
\subsection{Mischtank}\index{Mischtank}\index{Sirup}\label{sirup_beispiel}
Wird ein Glas Wasser in ein Glas Sirup geschüttet, so

\TRAINER{\bbwCenterGraphic{5cm}{allg/alg/potenzen_wurzeln/img/Schwapp.png}}%%
\noTRAINER{\bbwCenterGraphic{5cm}{allg/alg/potenzen_wurzeln/img/SchwappOhneFormel.png}}

geschieht erst mal etwas eher klebriges:
\begin{itemize}
  \item Das Wasser verdrängt den Sirup und
  \item das Sirupglas schwappt über.
\end{itemize}

Wenn man nun gleichzeitig im Sirupglas
umrührt, so mischt sich das Wasser mit dem Sirup und je länger man
Wasser einschüttet, umso verdünnter wird der Sirup.


Wie viel Sirup bleibt im Glas?

\TNT{2.4}{
Am Ende bleibt ein
Verhältnis von Wasser : Sirup = $\left(1-\frac{1}{e}\right) : \left(\frac{1}{e}\right)$
\vspace{1.5cm}
}

Diese Konstante wird oft in großen chemischen Mischtanks verwendet,
gibt aber auch ein Maß an, wenn \zB in einer Minergie-Wohnung die Luft
ausgetauscht wird. Wenn nämlich das Volumen der Wohnung einmal neu hineingepumpt (bzw. weggeblasen) wurde während sich alte die Luft im Haus permanent mit der neuen vermischt, so ist noch ein Anteil von \TRAINER{$\frac{1}{e}$}\noTRAINER{ ..... } der alten Luft im Haus.
\newpage


\textbf{Begründung:}\\
1. Gedanke: Jedes eingefüllte Glas, vermindert die vorhandene
Sirupkonzentration um den selben Faktor. Ergo handelt es sich um
einen exponentiellen Zerfall.

\leserluft

2. Gedanke: Wir tauschen drei Mal $\frac13$ aus. Nehmen also im
\begin{itemize}
\item \textbf{ersten Schritt} $\frac13$ des Sirups weg (und ersetzen diesen mit Wasser).
  Es bleiben $\frac23$ Sirup. Den Rest füllen wir mit Wasser auf.
\item Im \textbf{zweiten Schritt} nehmen wir $\frac13$ des Gemisches
weg; es verbleiben also $\frac23$ von $\frac23$ an
Sirup-Konzentrat. Der Rest wird immer wieder mit Wasser aufgefüllt. Mit
anderen Worten: Es bleiben $\frac23 \cdot \frac23
= \left(\frac23\right)^2$ an Sirup\footnote{Man könnte hier auch argumentieren mit: «Wir nehmen von den $\frac23$ einen Drittel weg»: $\frac23 - (\frac13$ von $\frac23)$ = $\frac23 - (\frac13 \cdot\frac23) = \frac23 \cdot(1-\frac13)=\frac23\cdot\frac23$}.
\item Im \textbf{dritten Schritt} entnehmen wir wieder $\frac13$ des
Gemisches; es verbleiben wieder $\frac23$ vom bisherigen Sirup, also
$\frac23$ von $(\frac23)^2$ also $\left(\frac23\right)^3$.

Beim dreistufigen Gedankenexperiment verbleiben
$\left(\frac23\right)^3 = \left(1-\frac13\right)^3$ der ursprünglichen Konzentration.
\end{itemize}
\leserluft

3. Gedanke: Das Experiment vom vorherigen Gedanken können wir natürlich auch mit immer kleineren\TALS{, sogenannten infinitesimalen,} Schritten durchführen.
Mit Centilitern \zB im dl-Glas ersetzen wir 10 Mal je $\frac1{10}$. 
So verbleibt am Schluss $\left(1-\frac{1}{10}\right)^{10}\approx 0.35$ Sirup.

\GESO{Wenn wir (\zB mit dem Taschenrechner) die Schrittanzahl immer weiter vergrößern (und somit die pro Schritt ausgetauschte Menge immer verkleinern), so ergibt sich für 1000 Schritte ein Verhältnis von $\left(1-\frac{1}{1000}\right)^{1000}\approx 0.3677 \approx \frac1{\e}$. }
\TALS{Wenn wir die Schritte permanent erhöhen (und gegen Unendlich gehen lassen), so erhalten wir den Grenzwert (lat. Limes) von

$$\lim_{n\rightarrow\infty} \left(1-\frac{1}{n}\right)^n = \frac1{\e}$$
}
\newpage

\textbf{Aufgabe 1: Sirup}\\
Wie viel Wasser muss eingeschüttet werden, damit das auf der Flasche
angegebene Verhältnis von 1:6 (1 Teil Sirup, 6 Teile Wasser) zustande
kommt?

\TNT{8}{
Bei 1x Schütten, erhalten wir $\left(\frac{1}{\e}\right)^1$ Anteil Sirup.

Bei 2x Schütten, erhalten wir $\left(\frac{1}{\e}\right)^2$ Anteil Sirup.

Somit erhalten wir den Siebtel (1:6 = $\frac17$-Anteil) indem wir die
folgende Exponentialgleichung lösen:

$$\frac17 = \left(\frac{1}{\e}\right)^n$$
Diese Gleichung lösen wir, indem wir beidseitig logarithmieren und so
erhalten wir den einzuschüttenden Teil $$n=\ln(7)\approx{1.946}.$$
}%% END TNT

\textbf{Aufgabe 2: Minerige-Haus}\\
Wenn wir also wissen wollen, wie viel Luft in ein Minergiehaus
eingepumpt werden muss, damit nur noch 1 Promille der alten Luft
vorhanden ist, so erhalten wir
\TNTeop{
  $$\text{Volumen Neuluft} = \text{Wohnungsvolumen}\cdot{}\ln(1000)$$
  $$\ln(10000) \approx 6.9$$
} %% end TNT


\newpage



\subsection*{Aufgaben}
\TALSAadB{223ff}{840-847}
\GESOAadB{338}{29. (Bierschaum)}
\GESOAadB{354}{9. (Bauchspeicheldrüse)}
\GESOAadB{353}{6 (C-14 Methode)}
\GESO{\aufgabenFarbe{Kompendium S. 27ff Kap. 3.4.1 33., 35., 36., 39. und 41.}}
\newpage

\subsection{Halbwertszeit, Verdopplungszeit}\index{Halbwertszeit}\index{Verdopplungszeit}

\begin{definition}{Halbwertszeit}{}
Die Zeitspanne, in der sich eine Menge halbiert, nennen wir
\textbf{Halbwertszeit} und bezeichnen diese Zeit mit:

$$T_{1/2}$$
\end{definition}

Die Halbwertszeit wird inbesondere
  beim radioaktiven Zerfall verwendet: nach wie viel tausend Jahren strahlt
  ein Stoff nur noch die Hälfte.

  Herleitung. Es gilt ja:
  $$b=f(0) \textrm{ bei unserer Notation } f(t) = b\cdot{}a^\frac{t}\tau$$
  Und es soll gelten:
  $$\frac{b}2 = f(T_{1/2})$$

  Daraus ergibt sich:
  $$\frac{b}2 = b\cdot{}a^{\frac{T_{1/2}}{\tau}}$$

    Nach Auflösen ergibt sich:

    \TNT{4}{
      Durch $b$ teilen:
      $$\frac12 = a^{\frac{T_{1/2}}{\tau}}$$
      logarithmieren:
      $$\log_a\left(\frac12\right) = \frac{T_{1/2}}{\tau}$$
      mit $\tau$ multiplizieren...
      $$\tau\cdot{}\log_a\left(\frac12\right) = T_{1/2}$$
    }%% END TNT

    
\begin{gesetz}{Halbwertzszeit}{}
  $$T_{1/2} = \tau\cdot{}\log_a\left(\frac12\right)$$
\end{gesetz}

\newpage
Beispiel: Ein Stoff nimmt innerhalb von sieben Tagen auf 80\% ab. Wie
groß ist seine Halbwertszeit $T_{1/2}$?

\TNT{6}{
  $\tau=7$ und $a = 0.8$. Somit:

  $$T_{1/2} = 7\cdot{} \log_{0.8}\left(\frac12\right) \approx 21.74
  \textrm{ Tage}$$

  Bemerkung: $$f(t) =b\cdot{}0.8^\frac{t}7 \approx b\cdot{}\left(\frac12\right)^\frac{t}{21.74}$$
}%% END TNT

\begin{gesetz}{Stoffmenge}{}
  $$f(t) = b\cdot{}\left(\frac12\right)^\frac{t}{T_{1/2}}$$
\end{gesetz}

\newpage
\GESO{Optional:}

Analog gilt das Gesetz zur Verdopplung (s. obiges Beispiel Bevölkerung Irlands):
\begin{gesetz}{Verdopplungszeit}{}
  $$T_2 = \tau\cdot{}\log_a(2)$$
\end{gesetz}

\GESO{\subsection*{Aufgaben}}
\GESOAadB{352ff}{3. (Taucherin), 4. [Glasfaser ohne Teilaufgabe Eindringtiefe 4. b)] und 5. (radioaktiver Zerfall)}

\GESO{\aufgabenFarbe{Kompendium S. 28ff: Aufg 42., 43. und 47.}}

\GESO{\aufgabenFarbe{
    Maturaprüfung 2018 (Serie 4), Aufg 10 (Cäsium 137)\\
    Maturaprüfung 2018 (Serie 2), Aufg 11 (Plutonium)\\
    Maturaprüfung 2018 (Serie 1), Aufg 11 (radioaktive Substanz)\\
    Maturaprüfung 2016, Aufg. 9 (Jod-131)
}}

\newpage

