\subsection{Rate vs. Faktor II (optional)}\index{Rate}\index{Fatkor}\index{Zunahmefaktor}\index{Zunahmerate}
\totalref{RateZins1}

Rep.: eine Zunahme von 12\% entspricht einem Aufzinsfaktor von \LoesungsRaumLang{1.12}.

Wenn jedoch eine Beobachtung einer Zunahme von \zB 70\% innerhalb einer Viertelstunde beobachtet wird, so können wir uns fragen, um wie viel die Zunahme pro Zeiteinheit (hier Stunden) ist.
\newpage
Füllen Sie dazu folgende Tabelle aus. Dabei bedeuten

\begin{tabular}{lp{14cm}}\hline
  Einheit & Stunden, Minuten, Meter, ... \\\hline
  $\tau$  & In dieser Zeitspanne (Stunden, Meter, ...) wird beobachtet \\\hline
  $p$     & Zunahme\textbf{rate}\index{Zunahmerate}\index{Rate} während $\tau$ Einheiten in \%. Ist $p$ negativ, handelt es sich um eine Abnahme\\\hline
  $a$ & Zunahme\textbf{faktor}\index{Zunahmefaktor} während
  $\tau$ Einheiten. Ist $a<1$, handelt es sich um einen Abnahmefaktor
  ($a = a_{\tau} = 1+\frac{p}{100\%}$)\\\hline
  $A$ (Formel)   & Zunahme pro Einheit (als Faktor). Aufgeschrieben
  als Formel $=a^{\frac1{\tau}}$\\\hline
  $\approx A$ (Zahl)  & Zunahme pro Einheit (als Näherungswert).\\\hline
  $p_E$   & Prozentuale Zunahme pro Zeiteinheit $=(A-1)\cdot{}100\%$\\\hline
  \end{tabular} 

\leserluft{}
\leserluft{}
%% temporäres Platzhalterchen 
\newcommand{\ph}[1]{\noTRAINER{...........}\TRAINER{#1}}

%%\renewcommand{\arraystretch}{1.7}
$$f(t) = a^{\frac{t}\tau} = \left(a^{\frac1\tau}\right)^t= A^t$$
\TNT{1.6}{1. Zeile: Eine Zunahme von 56\% pro 3h entspricht einer
  Zunahme von 15.98\% pro Stunde.}

%% probably turn off auto-fill-mode in emacs when editing long lines
\begin{bbwFillInTabular}{|l|l|l|l|l|l|l|}\hline
  Einheit & $\tau$            &  $p$         & $a$          & $A$ (Formel)             &  $\approx A$     &$p_E$            \\\hline\hline
  h       &  $3$              &  $56$\%      & $1.56$       & $1.56^\frac13$            &  $1.1598$         & $15.98$\%       \\\hline 
  m       &  $2$              &  $-10$\%     & \ph{0.9}     &     \ph{$0.9^\frac1{2}$}  &  \ph{0.9487}      & \ph{-5.13}\%    \\\hline 
  h       &  $\frac14 = 0.25$ &  $70$\%      & \ph{1.7}     & \ph{ $1.7^\frac1{0.25}$}  &  \ph{8.3521}      & \ph{735.21}\%   \\\hline 
  h       &  $\frac12$        &  \ph{50}\%   & $1.5$        &  $1.5^\frac1{0.5}$        &  $2.25$           & \ph{125}\%      \\\hline 
  Tage    &  $5$              & $100$\%      & \ph{2}       & \ph{$2^\frac1{5}$}        &  \ph{1.1487}      & \ph{14.87}\%    \\\hline 
  Min.    &  $2$              & \ph{-60}\%   & $0.4$        & \ph{$0.4^\frac1{2}$}      &  \ph{0.6325}      & \ph{-36.75}\%   \\\hline 
  m       &  $12$             & \ph{4.5}\%   & \ph{1.045}   & $1.045^\frac1{12}$        &  \ph{1.00367}     & \ph{0.3675}\%   \\\hline
  Wochen  & \ph{$2$}          & \ph{$200$}\% & \ph{3}       & $3^\frac12$               & \ph{1.73205}      & \ph{73.205}\%   \\\hline
  Jahr    & \ph{$\frac1{12}$} & \ph{$1$}\%   & \ph{$1.01$}  &  $1.01^{\frac1{1/12}}$     &  \ph{$1.127$}     & \ph{$12.7$}\%   \\\hline
\end{bbwFillInTabular}%
\newpage
