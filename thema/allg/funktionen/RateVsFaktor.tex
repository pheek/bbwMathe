\subsection{Rate vs. Faktor
  II}\index{Rate}\index{Fatkor}\index{Zunahmefaktor}\index{Zunahmerate}
\totalref{RateZins1}

Den Unterschied von Zinsfuß (= Rate) und Zinsfaktor kennen wir bereits aus der Zinsrechnung.

So entspricht eine Zunahme von 12\% einem\\
Aufzinsungsfaktor von \LoesungsRaumLang{1.12}.

Wenn jedoch eine Beobachtung einer Zunahme von, sagen wir, 70\% innerhalb einer Viertelstunde beobachtet wird, so können wir uns fragen, um wie viel die Zunahme (als Rate oder Faktor) pro Zeiteinheit (hier Stunden) ist.


Füllen Sie dazu folgende Tabelle aus. Dabei bedeuten

\begin{tabular}{lp{14cm}}\hline
  Einheit & Stunden, Minuten, Meter, ... \\\hline
  $\tau$  & In dieser Zeitspanne (Stunden, Meter, ...) wird beobachtet \\\hline
  $p$     & Zunahme\textbf{rate}\index{Zunahmerate}\index{Rate} während $\tau$ Einheiten in \%. Ist $p$ negativ, handelt es sich um eine Abnahme\\\hline
  $a$     & Zunahme\textbf{faktor}\index{Zunahmefaktor} während $\tau$ Einheiten. Ist $a<1$, handelt es sich um einen Abnahmefaktor\\\hline
  $a_E$ (Formel)   & Zunahme pro Einheit (als Faktor). Aufgeschrieben als Formel\\\hline
  $\approx a_E$ (Zahl)  & Zunahme pro Einheit (als Näherungswert).\\\hline
  $p_E$   & Prozentuale Zunahme pro Zeiteinheit\\\hline
  \end{tabular} 

\leserluft{}
\leserluft{}
%% temporäres Platzhalterchen 
\newcommand{\ph}[1]{\noTRAINER{...........}\TRAINER{#1}}

%%\renewcommand{\arraystretch}{1.7}

%% probably turn off auto-fill-mode in emacs when editing long lines
\begin{bbwFillInTabular}{|l|l|l|l|l|l|l|}\hline
  Einheit & $\tau$            &  $p$         & $a$         & $a_E$ (Formel)           &  $\approx a_E$    &$p_E$            \\\hline\hline
  h       &  $\frac14 = 0.25$ &  70\%        & \ph{1.7}    & \ph{ $1.7^\frac1{0.25}$} &  \ph{8.3521}      & \ph{735.21\%}   \\\hline 
  h       &  $\frac12$        &  \ph{50\%}   & 1.5         &  $1.5^\frac1{0.5}$       &  2.25             & \ph{125\%}      \\\hline 
  Tage    &  $5$              & 100\%        & \ph{2}      & \ph{$2^\frac1{5}$}       &  \ph{1.1487}      & \ph{14.87}\%    \\\hline 
  Min.    &  $2$              & \ph{-60}\%   & 0.4         & \ph{$0.4^\frac1{2}$}     &  \ph{0.6325}      & \ph{-36.75}\%   \\\hline 
  m       &  $12$             & \ph{4.5}\%   & \ph{1.045}  & $1.045^\frac1{12}$       &  \ph{1.00367}     & \ph{0.3675}\%   \\\hline
  Wochen  & \ph{$2$}          & \ph{$200$}\% & \ph{3}      & $3^\frac12$              & \ph{1.73205}      & \ph{73.205}\%   \\\hline
  Jahr    & \ph{$\frac1{12}$} & \ph{$1$}\%   & \ph{$1.01$} & $1.01^{\frac1{1/12}}$     &  \ph{$1.127$}     & \ph{$12.7$}\%   \\\hline
\end{bbwFillInTabular} 
