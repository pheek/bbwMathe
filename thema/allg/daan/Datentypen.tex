%%
%% 2019 07 04 Ph. G. Freimann
%%

\section{Datentypen und Skalen}\index{Datentypen}\index{Skala}
\sectuntertitel{Mit Statistik kann man alles beweisen...}

\GESOTadBMTA{363}{21}
\TALSTadBFWA{284}{4.1}
%%%%%%%%%%%%%%%%%%%%%%%%%%%%%%%%%%%%%%%%%%%%%%%%%%%%%%%%%%%%%%%%%%%%%%%%%%%%%%%%%
\subsection*{Lernziele}

\begin{itemize}
  \item Grundlagen
  \item Tabellenkalkulation
  \item Datentypen (Merkmalsausprägung) und Skalen
\end{itemize}

\bbwCenterGraphic{16cm}{allg/daan/img/Dreun.png}

\newpage



\subsection{Merkmalstypen}
Merkmale werden in verschiedene \textbf{Kategorien}\index{Kategorie} und \textbf{Skalen}\index{Skala} eingeteilt. Hier ein Versuch eines Überblickes:


\bbwCenterGraphic{17cm}{allg/daan/img/Datentypen.png}\index{Ordinale
  Daten}\index{Nominale
  Daten}\index{Verhältnis-Skala}\index{Kardinalskala}\index{Intervallskala}

\subsubsection{Stetig / Diskret}
Eine Unterscheidung in «stetig»\index{stetig} und
«diskret»\index{diskret} ergibt nur bei quanitativen Daten Sinn. Bei
\textbf{stetigen} Werten kann ein neuer Wert zwischen jeden anderen
Werten liegen (m, kg, ...). Bei \textbf{diskreten} Werten gibt es
Lücken wie \zB bei der digitalen Uhrzeit liegt nichts zwischen 19:03
und 19:04.
\newpage

\subsubsection{Entscheidungshilfe zu Skalen}

\bbwCenterGraphic{14cm}{allg/daan/img/DatenanalyseSkalentypen.pdf}\index{Ordinale Daten}\index{Kardinal-Skala}\index{Intervall-Skala}


\GESO{\cite{marthaler21} S. 376}\TALS{\cite{frommenwiler17alg} S. 249}

Die «Datentypen» (\zB ganze Zahlen / Farbnamen / ...) werden in der deskriptiven
Statistik auch \textbf{Merkmalsausprägung}\index{Merkmalsausprägung} genannt.


\begin{tabular}{|p{4cm}|p{8cm}|p{4cm}|}
  \hline
  Skala & Beschreibung & Beispiele\\\hline
  Nominal-Skala & Werte haben ausschließlich Namen und können nicht
  geordnet werden. & Handy-Marken; Lieblingsfarbe; ...\\\hline
  Ordinal-Skala & Es gibt eine Natürliche Ordnung, jedoch sind die
  Abstände zwischen den Werten nicht messbar & Medallien
  (gold/silber/bronze); ...\\\hline
  Intervall-Skala & Intervalle können metrisch angegeben werden ---
  Differenzen können in kg, m, s, etc. berechnet werden & Grad
  Celsius, Östlicher Längengrad; ...\\\hline
  Verhältnis- oder Kardinal-Skala\index{Kardinalskala} & Es gibt einen
  natürlichen Nullpunkt und \textbf{Verhältnisse} (\zB «doppelt so groß»)
  können gebildet werden. & Anzahl; Gewicht in kg; Dicke in cm; ...\\\hline
  \end{tabular}
\newpage




\subsection*{Aufgaben}
\GESO{In Kapitel 21 (S. 363 im Buch) sind
  Beispiele aufgezeigt. Beantworten Sie folgende Fragen:
  \TRAINER{In Klassen mit etwas mehr Zeit: Wählt Euch in Teamarbeit
    ein Beispiel aus, die   ihr in max. 5 Minuten der Klasse
    erklärt. In Klassen mit weniger Zeit: Beantworten Sie die
    folgenden Fragen für die Beispiele 1-5}
  \begin{itemize}
  \item Welcher Datentyp liegt zu Grunde? 
  \item Welche Darstellungsart (Diagramm) ist sinnvoll? 
  \item Welche Fehler könnten sich einschleichen?
  \item (optional) Welche Erhebungsart wurde verwendet? 
  \end{itemize}
}%% END GESO
\TALSAadBMTA{249}{942. 943.}
\GESOAadBMTA{401}{6}

\aufgabenFarbe{Klassifizieren Sie die Daten aus dem eigenen Fragebogen
  in die folgenden Kategorien:\\
  \begin{tabular}{c|c|c|c}
    nominal & ordinal & Intervall & Verhältnis\\\hline
    Handymarke &\hspace{40mm} &\hspace{40mm} &Umfang Handgelenk\\
    ...&\hspace{40mm} &\hspace{40mm} &...\\
    \hspace{40mm}&\hspace{40mm} &\hspace{40mm} &\hspace{40mm}\\
    \hspace{40mm}&\hspace{40mm} &\hspace{40mm} &\hspace{40mm}\\
    \hspace{40mm}&\hspace{40mm} &\hspace{40mm} &\hspace{40mm}\\
    \hspace{40mm}&\hspace{40mm} &\hspace{40mm} &\hspace{40mm}\\
    \hspace{40mm}&\hspace{40mm} &\hspace{40mm} &\hspace{40mm}\\
    \hspace{40mm}&\hspace{40mm} &\hspace{40mm} &\hspace{40mm}\\
    \hspace{40mm}&\hspace{40mm} &\hspace{40mm} &\hspace{40mm}\\
    \end{tabular}}

\newpage
