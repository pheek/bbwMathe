%%
%% 2019 07 04 Ph. G. Freimann
%%
\section{Datenerhebung / -gewinnung}\index{Datenerhebung}\index{Datengewinnung}
\sectuntertitel{Traue keiner Statistik, die Du nicht selbst verfälscht hast}

\TadBMTA{371}{22} und \TadBMTA{373}{23}
%%\TALSTadBFWA{250}{4.2}
%%%%%%%%%%%%%%%%%%%%%%%%%%%%%%%%%%%%%%%%%%%%%%%%%%%%%%%%%%%%%%%%%%%%%%%%%%%%t%%%%%
\subsection*{Lernziele}

\begin{itemize}
\item Stichprobe vs. Grundgesamtheit
\item Methoden der Datengewinnung
\item Datenqualität
\item Fehlererkennung
\end{itemize}


\aufgabenFarbe{Füllen Sie die Umfrage «Eine Umfrage» aus.}

\newpage


\subsection{Grundbegriffe}
\GESO{Ab S. 373 im Buch \cite{marthaler21alg}\\%%
}%%

\begin{tabular}{p{5cm}|l}
  Grundgesamtheit         &  Als Grundgesamtheit bezeichnet man ....\\
  \\
  Stichprobe\index{Stichprobe}              & \\
  \\
  Stichprobenumfang       & \\
  \\
  Bias\index{Bias}                   & \\
  \\ 
  Repräsentativität       & \\
  \\
  Datensatz\index{Datensatz}               & \\
  \\
  Untersuchungseinheit\index{Untersuchungseinheit}    & \\
  \\
  Beobachtung             & \\
  \\
  Merkmal\index{Merkmal} / Variable      & \\
  \\
  Merkmalsträger\index{Merkmalsträger}          & \\
  \\
  Wert/Ausprägung         & \\
  \\
  nominal\index{nominal}                 & \\
  \\
  ordinal\index{ordinal}                 & \\
  \\
  diskret\index{diskret}                 & \\
  \\
  stetig (kontinuierlich) & \\
  \\
  Intervalle\index{Intervalle} sind bildbar & \\
  \\
  Verhältnisse\index{Verhältnisse!bildbar} (Quotienten) sind bildbar & \\
  \\
  relative Häufigkeit\index{Häufigkeit!relative}     & \\
\end{tabular}%%
\newpage

%%\TALS{Theorie \cite{frommenwiler17alg}: S.250 Kap. 4.2}

\subsection{Merkmale}
Bei einer Datenerhebung fallen diverse Merkmale mehr oder weniger ins
Gewicht. Wenn wir Personen untersuchen, so ist jede Person ein
sog. \textbf{Merkmalsträger}. Merkmale können Gewicht, Alter, Wohnort
etc. sein.

\subsubsection{Abgrenzung}\index{Abgrenzung}
Einstiegsbeispiel


Die berufstätige Bevölkerung von Winterthur soll im Jahr 2022 auf ihre
Altersstruktur untersucht werden.

\noTRAINER{\bbwCenterGraphic{14cm}{allg/daan/img/abgrenzung_leer.png}}
\TRAINER{\bbwCenterGraphic{14cm}{allg/daan/img/abgrenzung.png}}


\newpage
Um den Überblick zu behalten, unterscheiden wir die Merkmale in drei
Kategorien:
\begin{itemize}
\item \textbf{Untersuchungsmerkmal}\index{Untersuchungsmerkmal}:
\TNT{2.4}{Das Merkmal, worüber ich eine Aussage machen will.\vspace{12mm}}%% END
%% TNT
\item \textbf{Abgrenzungsmerkmal}\index{Abgrenzungsmerkmal}:
  \TNT{2.8}{Merkmale, die die Untersuchende Gruppe von allen anderen
    abgrenzt. Wenn ich eine
Untersuchung über die berufstätigen Winterthurer im Jahr 2022 will, so
sind weder Pensionierte, noch Personen aus Liechtenstein, noch
berufstätige im Jahr 2004 zu untersuchen.\vspace{12mm}
}%% END TNT
\item ... und der ganze Rest: 

  \TNT{1.6}{Alle übrigen Merkmale sind für die Untersuchung irrelevant.\vspace{12mm}}
\end{itemize}

Ein gutes Abgrenzungsmerkmal sollte die folgenden Eigenschaften
aufweisen:
\begin{itemize}
\item \textbf{Örtlich}: Nur Personen innerhalb einer geographischen
  Grenze (\zB Winterthur)
\item \textbf{Zeitlich}: Die Untersuchung untersucht nur einen vorgegebenen
Zeitraum
\item \textbf{Sachlich}: Beispiel: berufstätig, ...
\end{itemize}

Bei allgemeingültigen Experimenten\footnote{Allgemeingültige Aussagen
  werden insb. in Physik, Chemie, Biologie, aber auch anderen
  (Natur)wissenschaften gemacht.} sollten hingegen Zeit und Ort keine Rolle spielen.
\newpage

\subsection{Arten der Datengewinnung}
\begin{itemize}
 \item \TNT{1.2}{Experiment}
 \item \TNT{1.2}{Befragung}
 \item \TNT{1.2}{Beobachtungsstudie}
 \item \TNT{1.2}{Datensammlung (sekundärstatistische Erhebung)}
\end{itemize}

\subsection*{Aufgaben}
\AadBMTA{399}{3.: «Datengewinnung»}

\subsection{Vorgehen (Optional)}
Das Vorgehen von den Rohdaten bis zur Interpretation kann grob wie folgt zusammengefasst werden. Dabei wird eine Variable (ein Merkmal)
\begin{enumerate}
\item gemessen,
\item kategorisiert\footnote{nominal, ordinal, «Intervalle-bildbar»,
    «Verhältnisse-bildbar»},
\item allenfalls sortiert,
\item graphisch dargestellt und
\item interpretiert.
\end{enumerate}
\newpage


\subsection{Fehlerarten}\index{Fehlerarten!in der Datengewinnung}


Die \textbf{Datenqualität}\index{Datenqualität} kann durch minimieren
von Erfassungsfehlern verbessert werden.

Hier die häufigsten Fehlerquellen:


\begin{itemize}
  
\item \textbf{Übertragungsfehler} (der Werte, \zB falsche Maßeinheit,
  Legasthenie, Übermüdung beim Abschreiben, ...)

\item \textbf{Systematische} Fehler (Es wird etwas anderes gemessen, als das,
  wofür die Aussage gerade stehen muss.)

\item \textbf{Zufällige} Fehler, Messungenauigkeit

\item \textbf{Mutwilliger} oder \textbf{fahrlässiger} Fehler («Traue keiner Statistik...»)

\item \textbf{Bias}: \textbf{Stichprobenverzerrung}\index{Bias}\index{Stichprobenverzerrung}
\end{itemize}

\subsection*{Aufgaben}
\AadBMTA{399}{4. «Fehler», 5. «Bias»}
\newpage
