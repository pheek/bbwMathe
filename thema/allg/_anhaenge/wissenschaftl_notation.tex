%%%%%%%%%%%%%%%%%%%%%%%%%%%%%%%%

\subsection{Wissenschaftliche Notation}\index{Notation!wissenschaftliche}\label{wissenschaftlicheNotation}
Bei Zahlen größer als 10 können wir trotz korrektem Runden einer Zahl manchmal nicht ansehen, wie viel Stellen denn nun signifikant sind.

$$ 679\,946 \text{\ Einwohner} \approx  \LoesungsRaumLang{680\,000} \text{\ Einwohner}$$
$$ 680\,023 \text{\ Einwohner} \approx  \LoesungsRaumLang{680\,000} \text{\ Einwohner}$$

Daher bietet sich darüber hinaus die wissenschaftliche Notation an.\footnote{Die
\textbf{wissenschaftliche Notation} wird vorwiegend für sehr große
aber auch für Zahlen sehr nahe an Null verwendet.}
Bei der wissenschaftlichen Notation wird die erste signifikante Ziffer
vor das Komma geschrieben. Nach dem Komma stehen \textbf{alle} weiteren signifikanten Stellen.
Zuletzt wird die Zahl mit Zehnerpotenzen
($10^{n}: n \in \mathbb{Z}$) «an die richtige Stelle» gerückt:

$$64\,038.6  = \LoesungsRaumLang{6.40386 \cdot 10^{ 4}} \approx \LoesungsRaumLang{6.40 \cdot 10^{ 4}}$$
$$0.00463640 = \LoesungsRaumLang{4.63640 \cdot 10^{-3}} \approx \LoesungsRaumLang{4.64 \cdot 10^{-3}}$$

Dabei bezeichnen negative Exponenten die Zehntel, Hundertstel, etc.
Erst in der wissenschaftlichen Notation können wir die signifikanten Stellen auch bei gerundeten Zahlen größer als 10 effektiv ablesen.


Stellen Sie in Wissenschaftlicher Notation dar:

0.00479 $= \LoesungsRaum{4.79 \cdot{}10^{-3}}$

0.08 Milliarden $= \LoesungsRaum{8\cdot{}10^7}$

\GESO{
Eine Aufgabe dazu finden Sie auch im «Kompendium»:

\olatLinkGESOKompendium{1.2}{7}{7.}
}%% END GESO


%%\TALS{S. \cite{frommenwiler17alg} S. 40 Kap. 1.5.5}
\newpage

\paragraph{Taschenrechner} Auf Taschenrechnern oder in
Programmiersprachen wird die Exponentialschreibweise i.\,d.\,R. mit dem
Buchstaben «e» angegeben. Also «e$\color{red}n$» anstelle von «$\cdot10^{\color{red}n}$». Beispiele:

$5\,000 = 5\, \cdot 10^{\color{red}3} = 5\mathrm{e}\color{red}3$

$0.063 = 6.3\, \cdot 10^{-2} = 6.3\mathrm{e-}2$

Berechnen Sie mit dem Taschenrechner und interpretieren Sie:

$$5.7^{19}  \approx  \LoesungsRaumLang{2.299440931\small{E}14}  = \LoesungsRaumLang{2.299440931 \cdot{} 10^{14}}$$
$$0.44^{36} \approx  \LoesungsRaumLang{1.459810064\small{E}-13} = \LoesungsRaumLang{1.459810064\cdot{} 10^{-13}}$$


(und überprüfen Sie mit dem Taschenrechner):

\begin{rezept}{EE}{}
    Um 5.77 Millionen auf Ihrem Taschenrechner \textbf{einzugeben} tippen Sie:

\begin{center}  \TRAINER{5.77 \GESO{\tiprobutton{EE}}\TALS{\nspirebutton{EE}} 6}\noTRAINER{\vspace{5mm}} \end{center}

\end{rezept}

\GESO{\begin{rezept}{SCI}{}
   Der Taschenrechner kann auch direkt die \textit{Wissenschaftliche Notation}
   anzeigen, wenn er so eingestellt ist:

   Beispiel 0.087 in \textit{Wissenschaftlicher Notation}:

   0.087 \tiprobutton{mode}\texttt{SCI}\tiprobutton{enter}\tiprobutton{2nd}\tiprobutton{mode_quit}\tiprobutton{enter}

   \texttt{8.7E-2} (Dies bedeutet $8.7 \cdot{} 10^{-2}$.)
\end{rezept}
}%% END GESO

\GESO{\olatLinkArbeitsblatt{A1rnd}{https://olat.bms-w.ch/auth/RepositoryEntry/6029794/CourseNode/106261461706500}{2. - 8.}}
\TALS{\olatLinkArbeitsblatt{A1rnd}{https://olat.bms-w.ch/auth/RepositoryEntry/6029786/CourseNode/106261461797588}{2. - 8.}}

\newpage
