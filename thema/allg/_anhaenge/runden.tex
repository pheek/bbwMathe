\subsection{Runden}\index{runden}
In der Regel sind wir bei Dezimalbrüchen nicht an allen auftretenden
Stellen interessiert, sondern begnügen uns mit einer Näherung.

\subsubsection{Dezimalen (Nachkommastellen)}\index{Dezimale}\index{Nachkommastelle}
Als Dezimalen, Dezimalstellen oder Nachkommastellen werden die Stellen
\textbf{nach} dem Komma bezeichnet.

\begin{gesetz}{Runden}{}
  Beim \textbf{Runden} auf die $n$-te Stelle, wird die $n+1$-te Stelle
  betrachtet. Ist diese $\ge 5$, so wird \textbf{aufgerundet}, ansonsten \textbf{abgerundet}.
\end{gesetz}

Runden auf die vierte \textbf{Dezimale} (= vierte \textbf{Nachkommastelle}):
$$ 36.4699432 \approx  \LoesungsRaum{36.4699}$$
$$ 36.4699618 \approx  \LoesungsRaum{36.4700}$$

\textbf{Vorsicht} bei Zahlen nahe an Null. So wird die relativ präzise
Zahl

$$0.002468$$ beim Runden auf drei Dezimalen wie folgt gerundet:
$$0.002$$

Runden Sie auf {\color{ForestGreen}vier} Dezimalen:

\begin{bbwFillInTabular}{rcl}
  $55.55555$      &$\approx$& \LoesungsRaum{$55.{\color{ForestGreen}\mathbf{5556}}$}\\
  $8.55695$       &$\approx$& \LoesungsRaum{$8.{\color{ForestGreen}\mathbf{5570}}$}\\
%%  $3.3339499$     &$\approx$& \LoesungsRaum{$3.{\color{ForestGreen}\mathbf{3339}}$}\\
%%  $1000.0001$     &$\approx$& \LoesungsRaum{$1000.{\color{ForestGreen}\mathbf{0001}}$}\\
%%  $10\,000.00001$ &$\approx$& \LoesungsRaum{$10\,000.{\color{ForestGreen}\mathbf{0000}}$}\\
%%  $-6.99999$      &$\approx$& \LoesungsRaum{$-7.{\color{ForestGreen}\mathbf{0000}}$}\\
  $0.000040447$   &$\approx$& \LoesungsRaum{$0.{\color{ForestGreen}\mathbf{0000}}$}\\
\end{bbwFillInTabular}
\newpage

%%\TRAINER{Das letzte Beispiel zeigt, dass so oft wichtige Information
%%  verloren geht. Daher wird sinnvollerweise meist auf signifikante Stellen, und nicht
%%  auf Dezimalen gerundet.}

\GESO{\olatLinkArbeitsblatt{A1rnd}{https://olat.bbw.ch/auth/RepositoryEntry/572162163/CourseNode/106261461706500}{1.}}
\TALS{\olatLinkArbeitsblatt{A1rnd}{https://olat.bbw.ch/auth/RepositoryEntry/572162090/CourseNode/106261461797588}{1.}}

\GESO{
Eine Aufgabe dazu finden Sie auch im «Kompendium»:

\olatLinkGESOKompendium{1.2}{6}{5.}
}%% END GESO

\newpage

\subsubsection{Signifikante Stellen\GESO{ (optional)}}\index{signifikante Stellen}
\begin{rezept}{Auf signifikante Ziffern runden}{}
  
  Beim Runden auf \textbf{vier signifikante Ziffern} wird
  \begin{itemize}
  \item  von links nach rechts die erste von Null verschiedene Ziffer gesucht. Dies ist die
    erste signifikante Ziffer.
  \item
    Danach werden die nächsten drei Ziffern
  genommen, egal ob sie Null sind oder nicht. 
\item   Mit diesen drei Ziffern bilden die Ziffern zusammen die vier
  signifikanten Ziffern.
\item  Die 5. Ziffer wird nur noch zum Auf- oder Abrunden verwendet.
  \end{itemize}
\end{rezept}

Runden Sie zweimal dieselbe Zahl auf \textbf{vier Dezimalen}:
 
$$0.000040447 [km] \approx \LoesungsRaum{0.0000 [km] ???}$$
$$40.447      [mm] \approx \LoesungsRaum{40.4470 [mm] ???}$$

Geben Sie {\color{ForestGreen}\textbf{vier}} \textbf{signifikante} Ziffern an und runden Sie wenn nötig:
 
$$0.000040447 [km] \approx \LoesungsRaum{0.0000{\color{ForestGreen}\mathbf{4045}} [km]}$$
$$40.447      [mm] \approx \LoesungsRaum{{\color{ForestGreen}\mathbf{40.45}} [mm]}$$

%%$$36.4699432 \approx \LoesungsRaum{{\color{ForestGreen}\mathbf{36.47}}}$$
%%$$36.9952831 \approx \LoesungsRaum{{\color{ForestGreen}\mathbf{37.00}}}$$
%%$$30009.78   \approx \LoesungsRaum{{\color{ForestGreen}\mathbf{3001}}0}$$
%%$$0.0439899  \approx \LoesungsRaum{0.0{\color{ForestGreen}\mathbf{4399}}}$$
%%$$1\,000\,000 \approx \LoesungsRaum{{\color{ForestGreen}\mathbf{1\,000}}\,000}$$
%%$$0.00001    \approx \LoesungsRaum{0.0000{\color{ForestGreen}\mathbf{1000}}}$$


\GESO{
Aufgaben zu signifikanten Ziffern finden Sie im «Kompendium»:

\olatLinkGESOKompendium{1.2}{7}{6.}
}%% END GESO

\TNTeop{Rechnen Sie nie mit bereits gerundeten Zwischenresultaten. Der Fehler kann sich dadurch vergrößern!}%% END TNT
\newpage

%%%%%%%%%%%%%%%%%%%%%%%%%%%%%%%%
  
\subsubsection{Wissenschaftliche Notation}\index{Notation!wissenschaftliche}\label{wissenschaftlicheNotation}
Bei Zahlen größer als 10 können wir trotz korrektem Runden einer Zahl manchmal nicht ansehen, wie viel Stellen denn nun signifikant sind.

$$ 679\,946 \text{\ Einwohner} \approx  \LoesungsRaumLang{680\,000} \text{\ Einwohner}$$
$$ 680\,023 \text{\ Einwohner} \approx  \LoesungsRaumLang{680\,000} \text{\ Einwohner}$$

Daher bietet sich darüber hinaus die wissenschaftliche Notation an.\footnote{Die
\textbf{wissenschaftliche Notation} wird vorwiegend für sehr große
aber auch für Zahlen sehr nahe an Null verwendet.}
Bei der wissenschaftlichen Notation wird die erste signifikante Ziffer
vor das Komma geschrieben. Nach dem Komma stehen \textbf{alle} weiteren signifikanten Stellen.
Zuletzt wird die Zahl mit Zehnerpotenzen
($10^{n}: n \in \mathbb{Z}$) «an die richtige Stelle» gerückt:

$$64\,038.6  = \LoesungsRaumLang{6.40386 \cdot 10^{ 4}} \approx \LoesungsRaumLang{6.40 \cdot 10^{ 4}}$$
$$0.00463640 = \LoesungsRaumLang{4.63640 \cdot 10^{-3}} \approx \LoesungsRaumLang{4.64 \cdot 10^{-3}}$$

Dabei bezeichnen negative Exponenten die Zehntel, Hundertstel, etc.
Erst in der wissenschaftlichen Notation können wir die signifikanten Stellen auch bei gerundeten Zahlen größer als 10 effektiv ablesen.


Stellen Sie in Wissenschaftlicher Notation dar:

0.00479 $= \LoesungsRaum{4.79 \cdot{}10^{-3}}$

0.08 Milliarden $= \LoesungsRaum{8\cdot{}10^7}$

\GESO{
Eine Aufgabe dazu finden Sie auch im «Kompendium»:

\olatLinkGESOKompendium{1.2}{7}{7.}
}%% END GESO


%%\TALS{S. \cite{frommenwiler17alg} S. 40 Kap. 1.5.5}
\newpage

\paragraph{Taschenrechner} Auf Taschenrechnern oder in
Programmiersprachen wird die Exponentialschreibweise i.\,d.\,R. mit dem
Buchstaben «e» angegeben. Also «e$\color{red}n$» anstelle von «$\cdot10^{\color{red}n}$». Beispiele:

$5\,000 = 5\, \cdot 10^{\color{red}3} = 5\mathrm{e}\color{red}3$

$0.063 = 6.3\, \cdot 10^{-2} = 6.3\mathrm{e-}2$

Berechnen Sie mit dem Taschenrechner und interpretieren Sie:

$$5.7^{19}  \approx  \LoesungsRaumLang{2.299440931\small{E}14}  = \LoesungsRaumLang{2.299440931 \cdot{} 10^{14}}$$
$$0.44^{36} \approx  \LoesungsRaumLang{1.459810064\small{E}-13} = \LoesungsRaumLang{1.459810064\cdot{} 10^{-13}}$$


(und überprüfen Sie mit dem Taschenrechner):

\begin{rezept}{EE}{}
    Um 5.77 Millionen auf Ihrem Taschenrechner \textbf{einzugeben} tippen Sie:

\begin{center}  \TRAINER{5.77 \GESO{\tiprobutton{EE}}\TALS{\nspirebutton{EE}} 6}\noTRAINER{\vspace{5mm}} \end{center}

\end{rezept}

\GESO{\begin{rezept}{SCI}{}
   Der Taschenrechner kann auch direkt die \textit{Wissenschaftliche Notation}
   anzeigen, wenn er so eingestellt ist:

   Beispiel 0.087 in \textit{Wissenschaftlicher Notation}:

   0.087 \tiprobutton{mode}\texttt{SCI}\tiprobutton{enter}\tiprobutton{2nd}\tiprobutton{mode_quit}\tiprobutton{enter}

   \texttt{8.7E-2} (Dies bedeutet $8.7 \cdot{} 10^{-2}$.)
\end{rezept}
}%% END GESO

\GESO{\olatLinkArbeitsblatt{A1rnd}{https://olat.bbw.ch/auth/RepositoryEntry/572162163/CourseNode/106261461706500}{2. - 8.}}
\TALS{\olatLinkArbeitsblatt{A1rnd}{https://olat.bbw.ch/auth/RepositoryEntry/572162090/CourseNode/106261461797588}{2. - 8.}}


\newpage
