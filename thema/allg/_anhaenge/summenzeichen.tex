%
% Das Summenzeichken SIGMA
%

\subsection{Das Summenzeichen}\label{Summenzeichen}\index{Sigma}\index{Sigma@$\Sigma$}\index{$\Sigma$ s. Sigma}\index{Summe}
«Der Mathematiker zählt lieber, als dass er glaubt.»
\vspace{4mm}


\GESO{\matheNinjaLink{Summenzeichen}{https://olat.bms-w.ch/auth/RepositoryEntry/6029794/CourseNode/106261490058869}}%% END GESO
\TALS{\matheNinjaLink{Summenzeichen}{https://olat.bms-w.ch/auth/RepositoryEntry/6029786/CourseNode/106261490112745}}%% END TALS

%% Summenzeichen für Mittelwert bei Datenreihen.
\newpage


\subsubsection{Einführung}

\bbwCenterGraphic{5cm}{allg/_anhaenge/img/GaussPhi.jpg}
\begin{center}\textit{Carl Friedrich Gauss (1777-1855)}\end{center}

%\bbwCenterGraphic{5cm}{allg/_anhaenge/img/Carl_Friedrich_Gauss.jpg}
%\begin{center}\textit{Carl Friedrich Gauss (1777-1855) (Gemälde von G. Biermann
%    1887)}\end{center}

Zählen Sie alle Zahlen von 1 bis 100 zusammen:

$$1+2+3+4+...+99+100 = ?$$

\TNT{4}{Lösung Gauss: 100+99+...+2+1 unter 1+2+...+100
  schreiben. 101*100 / 2 = 10\,100 / 2 = 5\,050}

\begin{gesetz}{kleiner Gauß}{}\index{Gau\ss!kleiner}
\textit{«Der kleine Gauß»}: Formel

$$1+2+3+...+n = \noTRAINER{\hspace{4cm}}\TRAINER{\frac{n\cdot{}(n+1)}2}$$
\end{gesetz}

\newpage

\subsubsection{Notation}
Um Summen mit vielen Summanden abzukürzen, wird das mathematische
Summenzeichen  $\sum{}$ (das griechische Sigma) benutzt:

\begin{definition}{}{}
  $$1 + 2 + 3 + 4 + 5 =: \sum_{z=1}^{5}{z}$$
\end{definition}

  Sprich «Summe über alle $z$; von
$z$ gleich eins bis fünf».

\subsubsection{Laufvariable}
Es spielt keine Rolle, mit welchem Variablennamen der Laufindex abgekürzt wird. Üblich sind Buchstaben wie $i$, $j$, $k$, $m$, $n$:

$$\sum_{n=1}^{5}{n} = \sum_{i=1}^5{i} = \sum_{k=1}^5{k} = \LoesungsRaum{15}$$

Die Laufvariable erhöht ihren Wert für jeden Summanden um 1 (eins).
\newpage

\subsubsection{Beispiele}
\begin{beispiel}{Summenzeichen}{beispiele_summenzeichen}
  $${\color{ForestGreen}\sum_{{\color{orange}n}{{\color{ForestGreen}=\color{blue}3}}}^{\color{blue}5\color{ForestGreen}}}  {\color{orange}n}^7 = \noTRAINER{\hspace{90mm}}\TRAINER{{\color{blue}3}^7 {\color{ForestGreen}+} {\color{blue}4}^7 {\color{ForestGreen}+} {\color{blue}5}^7=96\,696}$$
\end{beispiel}

\begin{beispiel}{}{}
$${\color{ForestGreen}\sum_{{\color{orange}s}={\color{blue}5}}^{{\color{blue}8}}{\color{black}(3{\color{orange}s}-4)}} = \noTRAINER{\hspace{105mm}}\TRAINER{(3 \cdot{\color{blue}5} - 4) {\color{ForestGreen}+} (3 \cdot{\color{blue}6} - 4) {\color{ForestGreen}+} (3 \cdot {\color{blue}7} - 4) {\color{ForestGreen}+} (3 \cdot {\color{blue}8} - 4)=62}$$
\end{beispiel}

\TALS{
\textbf{Aber:}

\begin{beispiel}{}{}
  $$\sum_{s=5}^{8}{3s-4} = \noTRAINER{???}\TRAINER{(3\cdot 5) +
    (3\cdot 6) + (3\cdot 7) + (3\cdot 8) - 4}$$
  \TNT{4}{}
\end{beispiel}
}

\newpage
\subsubsection{Übungsaufgaben}
Berechnen Sie
$$\sum_{x=3}^5{x^2} = \LoesungsRaum{50}$$
$$\sum_{x=2}^4{(x+1)^2} = \LoesungsRaum{50}$$
$$\sum_{n=-40}^{60}{(n-10)} = \LoesungsRaum{0}$$
$$\sum_{i=2}^{4}{\frac{1}{i}} = \LoesungsRaum{\frac{13}{12} = 1.08\overline{3}}$$
$$\sum_{x=3}^6{(2x-4)^2} = \LoesungsRaum{120}$$


Für Spezialisten (Teleskopformeln):

\textit{Tipp: Schreiben Sie die ersten drei und die letzten beiden Summanden
  jeweils explizit hin.}

$$\sum_{i=3}^{10}\left(i^6 - ( i-1 )^6\right) = \LoesungsRaum{1\,000\,000 - 2^6 = 999\,936}$$
  
\TNT{2.8}{$$(3^6 - 2^6) + (4^6 - 3^6) + (5^6 - 4^6) + ... + (9^6 - 8^6) + (10^6
- 9^6) = 10^6 - 2^6$$}


$$\sum_{i=0}^{n-1}x^i(x-1)=\LoesungsRaum{}$$
\TNT{2.8}{$= x^0(x-1) + x^1(x-1) ... + x^{n-1}(x-1) =$

$=x-1+x^2-x+x^3-x^2+ ...+ x^n-x^{n-1} = x^n-1$}


\newpage

\GESO{Taschenrechner: \tiprobutton{math} danach \texttt{5: sum(}\\
  Die Laufvariable lautet im Rechner immer $x$.}
\TALS{Taschenrechner: Die Taste mit den mathematischen Symbolen 
  (Brüche etc.; die Taste neben der Ziffer 9) beinhaltet auf der zweiten Zeile etwa in der Mitte
  auch das Summenzeichen.}

Geben Sie mit Hilfe des Summenzeichens an und berechnen Sie:

$$(-7)^5 + (-6)^5 + (-5)^5 + (-4)^5 + ... + 5^5+6^5 = \noTRAINER{\sum_{}^{}\hspace{7mm}}\TRAINER{\sum_{x=-7}^{6}x^5} = \LoesungsRaum{-16\,807}$$


\olatLinkGESOKompendium{1.3}{7}{8.}

\newpage


%%\subsubsection{Verdeutlichung}
%%Die folgende Schreibweise rechts verdeutlicht, von wo bis wo der Wert
%%der Variable $z$ wandert, wenn auch die rechte Schreibweise eher unüblich ist:
%%$$\sum_{z=1}^{5}{z} = \sum_{z=1}^{z=5}{z}$$

\subsubsection{Mittelwert, Indizes\GESO{ (optional)}}
Im Zusammenhang mit Indizes ($i$) (\zB bei Datenreihen) wird meist die
folgende Notation benutzt.

Seien also $x_1=5$, $x_2=8$, $x_3=6$ und $x_4=5$ vier
Messwerte. Betrachten wir die Summe über
alle $x_i$ von $i$ gleich eins bis vier:

$$5+8+6+5 = x_1 + x_2 + x_3 + x_4 =: \sum_{i=1}^4{x_i}=24$$


Dies wird \zB beim Mittelwert (arithmetisches Mittel)
verwendet. Seien wieder $x_1=5$, $x_2=8$, $x_3=6$ und $x_4=5$. So ist
$\mittelwert{x}$ wie folgt berechnet:

$$\mittelwert{x} = \LoesungsRaumLang{\frac{1}{4}\cdot{}\sum_{i=1}^{4}{x_i}}$$

Dabei ist $$\frac{1}{4}\cdot{}\sum_{i=1}^{4}{x_i}=\frac{1}{4}\cdot{}\left( \sum_{i=1}^{4}{x_i}\right) =\frac{1}{4}\cdot{}(x_1 + x_2 + x_3 + x_4)$$

oder ganz allgemein für $n$ Datensätze. Der Mittelwert der $n$
Datensätze ist der $n$-te Teil der Summe über alle $x_i$ für $i$
gleich eins bis $n$:

\begin{gesetz}{Mittelwert}{}
$$\mittelwert{x} = \frac{1}{n}\cdot{}\sum_{i=1}^{n}{x_i}$$
\end{gesetz}

Ausgeschrieben:
$$\mittelwert{x} = \frac{1}{n}\cdot{}\sum_{i=1}^{n}{x_i} = \frac{1}{n}\cdot{}(x_1 + x_2 + x_3 + \dots + x_n)$$

\newpage
