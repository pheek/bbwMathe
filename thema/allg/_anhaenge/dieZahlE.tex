\subsection{Die Zahl $e$}


\subsubsection{Beispiel: Wissenschaftliche Publikation\GESO{ (optional)}}


\begin{bemerkung}{Konstante $\e$}{}
  Mit der Eulerschen Konstanten $\e$ als Basis wird lediglich ein einziger
  Parameter (meist $m$ oder $q$) verwendet, um die Form der
  Exponentialfunktion zu beschreiben.

  $$\text{Wir schreiben }\, b\cdot{}\e^{mt} \,\text{ anstelle von }\, b\cdot{}a^{\frac{t}{\tau}}$$

  Eine Abhängigkeit von $a$ zu $\tau$ fällt weg.

  Dabei ist $m$ die Tangentensteigung an die Funktion $$f(t) = \e^{mt}$$
  im Punkt $$\left(0\middle|1\right).$$
\end{bemerkung}

\subsubsection*{Rechenbeispiel}

In einem wissenschaftlichen Journal ist angegeben, dass eine Population annähernd folgendes Verhalten an den Tag lege [$t$ = Tage]:
$$f(t) = 3\,000\cdot{}\e^{1.504\cdot{}t}$$

Berechnen Sie die tägliche Zunahme $a$ und geben Sie somit die Funktionsgleichung ohne Bruch im Exponenten ($\tau = 1$) und mit dem neu berechneten $a$ in der Basis an:

$$f(t) = 3\,000\cdot{}\e^{1.504\cdot{}t} = b\cdot{}a^t \approx \LoesungsRaum{3\,000\cdot{}5.000^t}$$

\TNTeop{Tipp: Setzen Sie $t=0$, um den Startwert $b$ zu erhalten
  ($b=3\,000$) und danach $t=1$ ein:
 $$3000\cdot{}a^t = 3000\cdot{} e^{1.504t}$$
 $$a^t =  e^{1.504t}$$
  $$a = \e^{1.504} \approx 5.000$$
\vspace{3cm}}


\subsubsection*{Aufgaben}
\GESO{\olatLinkArbeitsblatt{Exponentialfunktionen}{https://olat.bms-w.ch/auth/RepositoryEntry/6029794/CourseNode/106029175831971}{Kap. 2.2:
    E-Funktion / Basiswechsel}}
\TALS{\olatLinkArbeitsblatt{Exponentialfunktionen}{https://olat.bms-w.ch/auth/RepositoryEntry/6029786/CourseNode/106029175777725}{Kap. 2.2:
    E-Funktion / Basiswechsel}}
\newpage


\subsubsection{Bedeutung der Zahl $\e$ (optional)}

\paragraph{$45\degre$ auf der $y$-Achse:}
Die E-Funktion $y=\e^x$ hat die $45\degre$-Steigung genau auf der $y$-Achse; sie beginnt im Gegensatz zu anderen Exponentialfunktionen genau bei $x=0$ so «richtig zu wachsen»\TALS{\footnote{Wenn wir in einem Punkt $x$ den Wert der $\e$-Funktion ($\e^x$) betrachten, so ist ihr Funktionswert exakt gleich der Steigung einer Tangente bei diesem $x$-Wert.}}.

\bbwGraph{-4}{3}{-1}{5}{
  \bbwFunc{pow(2.71828,\x)}{-3.5:1.5}
  \bbwFuncC{\x+1}{-2:2}{green}
}%% end graph

Dies ist ein weiterer Grund, warum die Zahl $\e$ eine solche Bedeutung bei
exponentiellen Prozessen einnimmt.


\TALS{
\paragraph{Tangente an die E-Funktion}\index{E-Funktion}
Als \textbf{E-Funktion} bezeichnen wir die Exponentialfunktion zur
Basis $\e \approx 2.71828172846$; also:
$$f(x) = \e^x$$
Diese Funktion ist von allen Exponentialfunktionen insofern speziell,
als dass für jedes $x$ der Funktionswert $y=\e^x$ genau die Steigung
der Tangente an $\e^x$ im Punkt $P_x=\left(x | \e^x\right)$ angibt.

So ist die Tangentensteigung im Punkt $(0|\e^0) = (0 | 1)$ genau 1 und die Steigung im
Punkt $(1|\e^1) = (1|\e)$ genau $\e$. 

}%% END TALS

\newpage
