%
% Legende der Farben und Logos
%

\subsection{Legende Farben Links}

\subsubsection{Farben}

\begin{definition}{Definition}{}
  Definitionen werden in diesem Stil definiert.
\end{definition}

\begin{gesetz}{Gesetz}{}
  Gesetze haben dieses Layout.
\end{gesetz}

\begin{beispiel}{Beispiel}{}
  Beispielhaft soll dieses Beispiel dienen.
\end{beispiel}

\begin{rezept}{Rezept}{}
  Rezepte kommen so daher.
  \begin{enumerate}
  \item tu das
  \item tu dies
    \end{enumerate}
\end{rezept}

Aufgaben
\aufgabenFarbe{So sind Ihre Hausaufgaben farblich hinterlegt.}
\newpage

\subsubsection{Links}

Generelle Web-Links:

\weblink{Mathe Mann/Mathe Frau}{https://www.youtube.com/watch?v=yiJTCL9I-aU}!


Youtube Link:

\youtubeLink{https://www.youtube.com/watch?v=yiJTCL9I-aU}{Mathe Mann/MatheFrau  }

Links auf Mathe-Ninja Seiten:

\matheNinjaLink{Betrag}{https://olat.bms-w.ch/auth/RepositoryEntry/6029794/CourseNode/105951761084932}

Links auf Arbeitsblätter:

\olatLinkArbeitsblatt{Betrag [A1B]}{https://olat.bms-w.ch/auth/RepositoryEntry/6029794/CourseNode/105796974541159}{1. a) bis e) und 2. a) bis g)}

Prüfung:

\olatLinkPruefung{Algebra Zahlmengen}{https://olat.bms-w.ch/auth/RepositoryEntry/6029794/CourseNode/104326181772668}

Kompendium bzw. Strukturaufgaben
(Kapitel Seite Aufgaben)

\olatLinkGESOKompendium{3}{4}{5}
\olatLinkTALSStrukturaufgabenGLF{Teil 1}{5}{3. o)}
\olatLinkTALSStrukturaufgabenSPF{}{}{}
\newpage
