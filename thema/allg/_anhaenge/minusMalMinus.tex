%% Anhang: minus x minus
\subsection{Minus mal Minus (optional)}
Wir können uns vorstellen, dass $3 \cdot (-4)$ dasselbe ist wie $(-4) + (-4) + (-4)$. Daher gilt
\begin{itemize}
\item $3 \cdot 4 = 12$
\item $3 \cdot (-4) = (-12)$
\item $(-3) \cdot 4 = (-12)$
\end{itemize}
Warum soll aber $(-3)\cdot(-4)$ gleich $+12$ sein?

Hier einige Erklärungsversuche:


\paragraph{Negativer Krankheits-Befund}
Wer negativ auf einen schlimmen Virus- oder Bakterienbefall getestet wurde, kann
die sich doch in einer positiven Situation sehen.

\paragraph{Vorzeichen:} Sehen wir die Zahl $(-4)$ als Gegenzahl von $4$, so können wir auch die Gegenzahl der Gegenzahl betrachten:
$4 = -(-4) = -(1\cdot(-4)) = (-1)\cdot(-4)$. Somit ist $(-1)\cdot(-4) = +4$.

\paragraph{Permanenzprinzip:} Wir versuchen den Rechengesetzen, die wir von den positiven Zahlen her kennen, Allgemeingültigkeit zu verleihen, dann müssen sie auch für die negativen Zahlen gelten.
Somit ist
$$(-4)          \cdot 0 = 0$$
Null anders schreiben:
$$(-4)    \cdot (3-3) = 0$$
Gegenzahl addieren:
$$(-4)\cdot(3 + (-3)) = 0$$
Distributivgesetz:
$$(-4)\cdot3 + (-4)\cdot (-3) = 0$$
Term $(-4)\cdot3$ ausrechnen:
$$-12 + (-4)\cdot (-3)= 0$$
Der Gleichung links und rechts 12 hinzufügen (addieren):
$$(-4)\cdot (-3) = 12$$
\newpage

\paragraph{Schuldscheine abgeben:}
Eine anschauliche, aus dem Leben gegriffene, Analogie ist das
«Verschenken von Schuldscheinen». Bezeichnen wir Banknoten als
Kreditscheine, so besitzt eine 50er Note einen Wert von $+50$. So hat
ein Schuldschein von $50.-$ Franken (oder Euros) den Wert $-50$
(sprich «minus fünfzig»).

\begin{tabular}{r@{}l|rl|r@{}l}
Geldwert & Gewinn/Verlust & Effekt\\
\hline\\
 + 50&.--   (Banknote)     &  3&  (erhalten)  & + 150&.-- (Gewinn)   \\
 - 50&.--   (Schuldschein) &  3&  (erhalten)  & - 150&.-- (Verlust)  \\
 + 50&.--   (Banknote)     & -3&  (abgeben)   & - 150&.-- (Verlust)  \\
 - 50&.--   (Schuldschein) & -3&  (abgeben)   & + 150&.-- (Gewinn)   \\
\end{tabular}
\newpage
