%%
%% 2020 graphische Interpretation linearer Gleichungen (TALS nSpire)
%%


\newpage
\subsection{Graphische Interpretation}
Graphische Interpretation der Grundform:

In der Form
$$ax+b=0$$
ist $a$ die Steigung\footnote{Steigung: Eine Einheit nach rechts: Um wie viele Einheiten steigt die Gerade an?} der Geraden und $b$ der Abschnitt\index{Achsenabschnitt} auf der $y$-Achse\footnote{$y$-Achsenabschnitt: Wo schneidet die Gerade die $y$-Achse.}:

\bbwCenterGraphic{6cm}{allg/gleichungen/img/LineareGleichungsfunktion.png}
%  \begin{center}
%   \raisebox{-1cm}{\includegraphics[width=6cm]{img/LineareGleichungsfunktion.png}}
%  \end{center}

  \paragraph{Charakteristische Punkte}
  Die charakteristischen Punkte (spezielle Punkte) der linearen
  Funktion in Grundform sind

  \begin{itemize}
  \item $b$ = $y$-Achsenabschnitt: Wo schneidet die Gerade die $y$-Achse?
  \item $a$ = Steigung der Geraden pro eine Einheit nach rechts (in $x$-Richtung)
  \item $\frac{-b}{a}$ = Lösung der Gleichung $ax+b=0$. Dies ist der $x$-Achsenabschnitt.
  \end{itemize}
\newpage

Mit diesem Gedanken können wir lineare Gleichungen auch lösen, wenn
diesen nicht in der Grundform stehen:

$$3x+1 = 2x$$


$\text{term1} := 3x+1$\\
  $\text{term2} := 2x$\\
  solve($\text{term1}=\text{term2}$, $x$)\\
  Diese letzte Variante hat den Vorteil, dass wir die Terme leicht als Funktionen in $x$ auffassen können und diese im Graph-Modus sofort anzeigen können:
  \begin{enumerate}
    \item Neue «Page» (1.2) erstellen (+page) als Graph. 
    \item Tippe $f1(x)=\text{term1}$
    \item Tippe \fbox{CTRL} \fbox{G} (oder mehrmals die \fbox{TAB}-Taste), um die Eingabezeile wieder zu aktivieren.
      \item $f2(x)=\text{term2}$
  \end{enumerate}

\bbwCenterGraphic{6cm}{tals/gl1/img/nspire_zwei_gleichungen.png}
%  \begin{center}
%   \raisebox{-1cm}{\includegraphics[width=6cm]{img/nspire_zwei_gleichungen.png}}
%  \end{center}
  Der Schnittpunkt der beiden Geraden zeigt die Variable $x$ (hier $-1$) und den Wert der beiden Terme (links bzw. rechts) des Gleichheitszeichens als Variable $y$ (hier $-2$) an.

