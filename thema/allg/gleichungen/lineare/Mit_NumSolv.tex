%%
%% (C) 2022 ph. freimann @ bbw.ch
%%

\subsection{Gleichungen mit dem Solver lösen}\index{Solver}\index{num-Solv}

«Die \textit{sanfte} Tour»

Lineare Gleichungen ohne Parameter können mit dem Taschenrechner mit dem numerischen
\textit{Solver} (num-solve) gelöst werden.

  Hierzu dient die Taste \tiprobutton{sin_num-solv}.

  Tippen Sie also \tiprobutton{2nd}, dann \tiprobutton{sin} und geben
  Sie die folgende lineare Gleichung ein:

$$\sqrt{2}(x+5) = 3x + \sqrt{7} - \frac{x}4$$

  Alle Optionen sind standardmäßig auf die Suche nach $x$
  eingestellt. Sie müssen also nur \textbf{sechs mal} \tiprobutton{enter}
  drücken um auf $x\approx \LoesungsRaum{3.31289}$ zu kommen!

  \begin{bemerkung}{Limitation}{}\index{Taschenrechner!Limitation}

    Berechnen Sie mit dem numerischen Solver
    \tiprobutton{sin_num-solv} das $x$ in der folgenden
    Bestimmungsgleichung:

    $$x+1=x+2$$
    
    \TNT{4}{Der \textbf{num-Solver} versucht das Resultat anzunähern
      und erhält

      $$x \approx 1.845 \cdot{} 10^{12}$$

      
    }% end TNT

    \vspace{10mm}
    
    \TNT{2}{\textbf{Resultate des Taschenrechners immer nachkontrollieren!}}
    
  \end{bemerkung}
\newpage
  
\subsection*{Aufgaben}
Lösen Sie mit dem Taschenrechner (num-solv):

\GESO{\olatLinkArbeitsblatt{Lineare
    Gleichungen}{https://olat.bms-w.ch/auth/RepositoryEntry/6029794/CourseNode/110662976644490}{Aufgabe
    1. b) c) d) f), 2. a) b)}
  \TRAINER{Solche Aufgaben später direkt als neue Nummer aufs Arbeitsblatt integrieren.}
}

\newpage
