
\subsection{Diskriminante}\index{Diskriminante}\label{diskriminante}
Die \textbf{Diskriminante} ist das, was in der $a$-$b$-$c-$Formel unter der Wurzel steht (= Radikand).

$D := b^2 - 4ac$

\TRAINER{«diskriminieren = unterscheiden»}

Mit der Diskriminante ist eine einfache \textbf{Fallunterscheidung} möglich:
\begin{itemize}
\item $D > 0 \Rightarrow $ zwei Lösungen:
  $$x_{1,2}  = \frac{-b \pm \sqrt{b^2 -  4ac}}{2a}= \frac{-b \pm \sqrt{D}}{2a}$$

\item $D = 0 \Rightarrow $ eine Lösung:
  $$ x = \frac{-b}{2a}$$

\item $D < 0 \Rightarrow $ keine Lösung in $\mathbb{R}$

\end{itemize}

\subsubsection{Anwendung}
Entscheiden Sie, wie viele Lösungen die gegebenen Gleichungen haben:

%%\renewcommand{\arraystretch}{2}
\begin{bbwFillInTabular}{c|c|p{5cm}}
  $x^2 + 3x +1 = 0$ & $b^2-4ac = \LoesungsRaumLang{9 -4 > 0}$ & \LoesungsRaumLang{zwei Lösungen} \noTRAINER{\phantom{xxxxxxxxxx}} \\
  \hline
  $x^2 + 2x +1 = 0$ & $b^2-4ac = \LoesungsRaumLang{4 -4 = 0}$ & \LoesungsRaumLang{eine Lösung} \\
  \hline
  $x^2 + 1x +1 = 0$ & $b^2-4ac = \LoesungsRaumLang{1 -4 < 0}$ & \LoesungsRaumLang{keine Lösungen} \\
\end{bbwFillInTabular}
%%\newpage

\subsection*{Aufgaben}

\GESO{\olatLinkArbeitsblatt{GlQuad}{https://olat.bms-w.ch/auth/RepositoryEntry/6029794/CourseNode/111365559999287}{Aufgabe 8.}
  }%% end GESO
\TALS{\olatLinkArbeitsblatt{GlQuad}{https://olat.bms-w.ch/auth/RepositoryEntry/6029786/CourseNode/111365560138736}{Aufgabe 8.}
}%% end TALS
\newpage
%\GESO{\AadBMTA{182ff}{11. c) d) e) und f) challenge: 16. a) b)}}
%\TALS{\AadBMTA{182ff}{11. c) d) e) und f)}}%% challenge 16. a) b) bei Tals
%% Pflicht, aber im nächsten Kapitel (s. unten)
