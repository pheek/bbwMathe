
\subsection{Diskriminante}\index{Diskriminante}\label{diskriminante}
Die \textbf{Diskriminante} ist das, was in der $a$-$b$-$c-$Formel unter der Wurzel steht (= Radikand).

$D := b^2 - 4ac$

«diskriminieren = unterscheiden»

Mit der Diskriminante ist eine einfache \textbf{Fallunterscheidung} möglich:
\begin{itemize}
\item $D > 0 \Rightarrow $ zwei Lösungen:
  $$x_{1,2}  = \frac{-b \pm \sqrt{b^2 -  4ac}}{2a}= \frac{-b \pm \sqrt{D}}{2a}$$

\item $D = 0 \Rightarrow $ eine Lösung:
  $$ x = \frac{-b}{2a}$$

\item $D < 0 \Rightarrow $ keine Lösung in $\mathbb{R}$

\end{itemize}

\subsubsection{Anwendung}
Entscheiden Sie, wie viele Lösungen die gegebenen Gleichungen haben:

%%\renewcommand{\arraystretch}{2}
\begin{bbwFillInTabular}{c|c|p{5cm}}
  $x^2 + 3x +1 = 0$ & $b^2-4ac = \LoesungsRaumLang{9 -4 > 0}$ & \LoesungsRaumLang{zwei Lösungen} \noTRAINER{\phantom{xxxxxxxxxx}} \\
  \hline
  $x^2 + 2x +1 = 0$ & $b^2-4ac = \LoesungsRaumLang{4 -4 = 0}$ & \LoesungsRaumLang{eine Lösung} \\
  \hline
  $x^2 + 1x +1 = 0$ & $b^2-4ac = \LoesungsRaumLang{1 -4 < 0}$ & \LoesungsRaumLang{keine Lösungen} \\
\end{bbwFillInTabular}
%%\newpage
\newpage
\subsection*{Aufgaben}
\AadBMTA{182ff}{11. c) f)}

\newpage

\TALS{
\subsection*{Typ «genau eine Lösung»}

Bestimmen Sie den Parameter $b$ so, dass die folgende Gleichung genau
eine Lösung hat:
$$x^2+b+2 = -2bx$$
\TNTeop{
  Vorzeigeaufgabe: S. 183.: 16. b):%%
  Grundform:
  $$x^2+2bx+b+2=0$$
  Diskriminante = 0 setzen:
  $$4b^2 - 4(b+2) = 0$$
  Grundform $b$:
  $$b^2-b-2 = 0$$
  $$\LoesungsMenge{}_b = \{-1; 2\}$$
}%% END TNT
\newpage

\subsubsection{Taschenrechner}
Das selbe Beispiel können wir ganz einfach auch mit dem Taschenrechner lösen:
\begin{beispiel}{Taschenrechner}{}
  Bestimmen Sie den Parameter $b$ so, dass die quadratische
  Gleichung $$x^2+b+2=-2bx$$
  genau eine Lösung hat:

  \TNT{2}{\texttt{solve($x^2+b+2=-2bx$, x)} liefert

  $$x_{1,2} = \pm\sqrt{b^2-b-2} - b$$}
  
  nun die Diskriminante suchen

  \TNT{2}{\texttt{D = $b^2-b-2$}}

  jetzt Diskriminante = 0 setzen:

  \TNT{2}{\texttt{solve($b^2-b-2$ = 0,b)}

    liefert

  $$b_{1,2} = -1 \text{ oder } 2$$}%% end tnt
\end{beispiel}%%  
  
\newpage
\subsection*{Aufgaben}
\AadBMTA{182ff}{16. a) c) d)}

}%% END TALS
\newpage
