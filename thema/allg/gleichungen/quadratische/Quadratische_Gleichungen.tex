%%
%% 2019 07 04 Ph. G. Freimann
%%
\section{Quadratische Gleichungen\TALS{ I}}\index{Gleichungen!quadratische}
\sectuntertitel{Psychiater zur Quadratischen: ``Auch für
  Dich finden wir die eine oder andere Lösung.''}

%%\GESOTadBMTA{165}{10}
%%\TALSTadBFWA{93}{2.3}
\TadBMTA{165}{10}

%%%%%%%%%%%%%%%%%%%%%%%%%%%%%%%%%%%%%%%%%%%%%%%%%%%%%%%%%%%%%%%%%%%%%%%%%%%%%%%%%
\subsection*{Lernziele}

\begin{itemize}
\item Grundform $ax^2+bx+c=0$
\item Lösungsmethoden 
\item Allgemeine Form, ABC-Formel (Mitternachtsformel) $x_{1,2} = \frac{-b \pm \sqrt{b^2-4ac}}{2a} $
%%   \item Rationale Gleichungen (<- die Bruchgleichungen haben ein eigenes Kapitel)
\item Diskriminante
\TALS{\item{Parameter}}
\TALS{\item{Fallunterscheidung}}
\item Taschenrechner: \GESO{Ohne Parameter} \TALS{Mit und ohne Parameter}
\TALS{\item Taschenrechner: Visualisierung, Interpretation}
\end{itemize}
\newpage
\TRAINER{Typische Hausaufgabe als Einstieg:}
\subsection{Einstieg}
\subsubsection{Einstiegsaufgabe Snapchat}\index{Snapchat}
\aufgabenFarbe{In einem Campus werden innerhalb eines W-LANs
  29\,501\,192 bestimmte Datenpakete der ganzen Schule registriert. Wir gehen davon aus, dass dabei jede Person
genau ein \textit{snap} an alle anderen versendet hat.\\Wie viele Personen waren
beteiligt?}

Lösung:
\TNT{0.8}{$5432$ Personen (waren beteiligt).}

Lösungsstrategien:

\TNTeop{
  Wie sieht es bei $3$ Personen aus?
  Alle $3$ senden es $2$ Personen weiter.

  $$3\cdot{}2 = 6\text{ snaps}$$
Wie sieht es bei 10 Personen aus?
  $$10\cdot{}9 = 90$$

  Es sollen aber $29\,501\,192$ sein.
  
  $n$ sei die Anzahl Schüler.

  Jeder sendet $n-1$ Post; d.\,h. es werden $n\cdot{}(n-1)$ Posts
  versendet.

  Gleichung:

  $$n\cdot{}(n-1) = 29501192 \Longrightarrow   n^2 - n = 295011292 \text{ =
    quadratische Gleichung}$$

  Die Lösung $5432$ kann \zB durch ein Ausprobieren ermittelt werden.
  
  Das Ausprobieren könnte noch durch $\sqrt{29501192}\approx 5431$  angenähert werden;
  als ersten Startwert.

  Hier ev. noch die Lösung des openai.com chats (GPT chatbot) zeigen.
}

\newpage

\GESO{  \subsubsection{Voraussetzungen}
  (Repetitionsaufgaben)
  
Berechnen Sie:

a) $7^2 = \TRAINER{49}$

b) $-7^2 = \TRAINER{-49}$

c) $(-7)^2 = \TRAINER{49}$

d) $(9-2)^2 = \TRAINER{49}$

e) $(-5 - 2)^2 = \TRAINER{49}$

\GESO{
  Challenge: Wer schon fertig ist, soll die beiden Gleichungen auf der
nächsten Seite versuchen.

}
\TNTeop{
}%% end TNTeop

}
\TALS{  \subsubsection{Voraussetzungen}
  (Repetitionsaufgaben)
  
Berechnen Sie:

a) $7^2 = \TRAINER{49}$

b) $-7^2 = \TRAINER{-49}$

c) $(-7)^2 = \TRAINER{49}$

d) $(9-2)^2 = \TRAINER{49}$

e) $(-5 - 2)^2 = \TRAINER{49}$

\GESO{
  Challenge: Wer schon fertig ist, soll die beiden Gleichungen auf der
nächsten Seite versuchen.

}
\TNTeop{
}%% end TNTeop

}

\newpage

\subsubsection{Erste quadratische Gleichungen}
\TRAINER{Einzelarbeit; danach geführt Lösungsweg gemeinsam festhalten.}
$$x^2=49$$

\TNT{6}{

  $$ \left(
  \begin{bbwFillInTabular}{c}
  $\sqrt{x^2} = \sqrt{49}$\\
     $\TALS{|x| = 7}\GESO{x=\pm\sqrt{49}}$\\
  $x=7\text{ oder } x=-7$
    \end{bbwFillInTabular}
  \right) \text{Notation: } | \pm \sqrt{} \, {\color{red}\pm! \text{ nicht vergessen!}}$$
  
  $$x=\pm 7$$
  
  $\lx = \{-7; 7\}$ \vspace{15mm}
}%% END TNT

\TRAINER{\hrule}

$$(x-2)^2=49$$

\TNTeop{
\TALS{
( Erinnert Sie das ursprüngliche Beispiel an $|x-2| = 7$ ??)
}%% end TALS

  Direkt Wurzel ziehen:

  $$\left(
  \begin{bbwFillInTabular}{c}
    $ \sqrt{(x-2)^2}= \sqrt{49}$\\
        $\TALS{|x-2|=7}\GESO{x-2 = \pm\sqrt{49}}$\\
    $(x-2) = 7 \text{ od. } (x-2)= -7$
  \end{bbwFillInTabular} 
  \right) |  {\color{red}(\pm)} \, \sqrt{}$$
  
 $$x-2=\pm 7$$
 $$x=2\pm7$$

 $$\lx = \{-5; 9\}$$


  
  danach unbedingt Probe vorzeigen:

  $$x_1 = -5: (-5-2)^2 = (-7)^2 = 49$$
  $$x_2 = 9: (9-2)^2 = (7)^2 = 49$$

}%% END TNT


\newpage
%%%%%%%%%%%%%%%%%%%%%%%%%%%%%%%%%%%%%%%%%%%%%%%%%%%%%%%%%%%%%%%%%%%

%%\TRAINER{Normalform $x^2 +5x +6 = 0 \rightarrow (x+2)(x+3)=0$}
%%\newpage

%% \TALS{%% Vorbereitungsaufgaben zur quadratischen Ergänzung
%%  Stil x² + 2bx + b² = Zahl
%%

\subsection*{Aufgaben}

\TRAINER{Zweireteams/Bankreihen}

Geben Sie jeweils die Lösungsmenge $\mathbb{L}$ an:

$$x^2 = 7$$

\TNT{4}{$\lx = \{-\sqrt{7}; \sqrt{7}\}$}


$$(x+6)^2 = 3$$

\TNT{3.2}{$x-6=\pm\sqrt{3}$\\
  $x=6\pm\sqrt{3}$\\
  $\lx = \{6-\sqrt{3}; 6+\sqrt{3}\}$}


$$(x-5)^2 = 0$$

\TNT{3.2}{
  $x-5=\pm\sqrt{0} = 0$\\
  $x=5$\\
  $\lx = \{5\}$}



$$(x-5)^2 = -9$$

\TNT{4}{
  $x-5=\pm\sqrt{-9} ???$\\
  $\lx = \{\}$}


\newpage
$$(x+3)^2 = 2$$

\TNT{4}{
$x+3 = \pm\sqrt{2}$\\
$x=-3\pm\sqrt{2}$\\
  $\lx = \{-3-\sqrt{2}; -3+\sqrt{2}\}$}


$$x^2 + 6x + 9 = 2$$

\TNT{4}{
  $(x+3)^2 = 2$\\
  $x+3 = \pm\sqrt{2}$\\
  $x=-3 \pm\sqrt{2}$\\
  $\lx = \{-3-\sqrt{2}; -3+\sqrt{2}\}$}





$$x^2 -12x + 36 = 5$$

\TNT{4}{
  $(x-6)^2 = 5$\\
  $x-6=\pm\sqrt{5}$\\
  $x=6\pm\sqrt{5}$\\
  $\lx = \{6-\sqrt{5}; 6+\sqrt{5}\}$}



$$x^2 -12x + 36 = -10$$

\TNT{4}{
  $(x-6)^2 = -10$\\
  $x-6 = \pm\sqrt{-10} ???$
  $\lx = \{\}$}
\newpage
\newpage}



\subsection*{Aufgaben}

\GESO{\olatLinkArbeitsblatt{GlQuad}{https://olat.bms-w.ch/auth/RepositoryEntry/6029794/CourseNode/111365559999287}{Aufgabe 1 von «a bis z» [bis und mit s)]}
  }%% end GESO


%\GESO{\olatLinkArbeitsblatt{Quadratische
%    Gleichungen}{https://olat.bms-w.ch/auth/RepositoryEntry/6029794/CourseNode/107315676111812}{Alle
%geraden Aufgaben}}

\olatLinkGESOKompendium{2.3.1.}{16}{42. bis 44.}

%\TALS{\olatLinkArbeitsblatt{Quadratische
%    Gleichungen erste Aufgaben}{https://olat.bms-w.ch/auth/RepositoryEntry/6029786/CourseNode/107315676197926}{Was ändert sich von Aufgabe zu Aufgabe?}}


\TALS{\olatLinkArbeitsblatt{GlQuad}{https://olat.bms-w.ch/auth/RepositoryEntry/6029786/CourseNode/111365560138736}{Aufgabe 1 «von a bis z» [y) und z): challenge]}
  }%% end TALS



%%\aufgabenFarbe{Aufgabenblatt im OLAT (Gallin) bis Aufgabe 7}

\TRAINER{Mögliche weitere Hausaufgaben: }

\TRAINER{\AadBMTA{181}{3. a) d) e), 4. e) [Bei Aufgabe 4. e) die Probe machen!]}}


\TNTeop{
  
Lernzielkontrolle: Kahoot: Quadratische Gleichungen.}%% end TNT

%%\newpage


%\TALS{
%  \subsubsection{Binome}

$$x^2+6x+9 = 20$$

\TNTeop{Schreibe links als ein Quadrat:

  $$(x+3)^2 = 20$$

  Ziehe die Wurzel und beachte die 2. Lösung:

  $$x+3 = \pm\sqrt{20}$$

  -3:

  $$x = -3 \pm \sqrt{20}$$

  $$\lx = \{-3-\sqrt{20}; -3 + \sqrt{20}\}$$
}


\subsection*{Aufgaben}

\olatLinkArbeitsblatt{Quadratische
    Gleichungen}{https://olat.bbw.ch/auth/RepositoryEntry/572162163/CourseNode/101586190756718}{weitere
    im Aufgabenblatt, auch weiter als Aufg. 7. möglich)}


\newpage
}%% END TALS 
%\newpage




\subsection{Grundform}\index{Grundform!quadratische Gleichung}\index{Quadratische Gleichung!Grundform}

\begin{definition}{Grundform der quadratischen Gleichung}{}\index{Grundform!quadratische Gleichung}
  Die Grundform der quadratischen Gleichung lautet

  $$ax^2+bx+c=0$$
  mit $a\in \mathbb{R}\backslash\{0\}$, $b, c \in \mathbb{R}$

  Eine Gleichung, die äquivalent zu obiger Grundform ist, heißt
  \textbf{quadratische Gleichung}.
\end{definition}

\subsubsection{Bemerkung}

\begin{bemerkung}{quadratische Gleichung}{}\index{quadratische Gleichung}\index{Gleichungen!quadratische!Grundform}
  Bei einer \textbf{quadratischen Gleichung} kommt die Gesuchte
  in der 2. Potenz vor; \zB $x^2$.
\end{bemerkung}

\textbf{Welche sind quadratisch?}

\begin{bbwFillInTabular}{cc}
$5^2=25x-6$ &
  \noTRAINER{\fbox{\,\vphantom{$X$}}}\TRAINER{\fbox{\color{red} X}
    falsch} \\
  
$5=x^2-6$ &
  \noTRAINER{\fbox{\,\vphantom{$X$}}}\TRAINER{\fbox{\color{green}
      \checkmark}} \\
  
$5x^3=18$ & \noTRAINER{\fbox{\,\vphantom{$X$}}}
  \TRAINER{\fbox{\color{red} X} falsch}\\
  
$\sqrt{b}-x^2=b^3$ &
  \noTRAINER{\fbox{\,\vphantom{$X$}}}\TRAINER{\fbox{\color{green}\checkmark} Ja,
    falls die Gesuchte das $x$ ist.} \\

$(x-7)^2 = x-5$ &
  \noTRAINER{\fbox{\,\vphantom{$X$}}}\TRAINER{\fbox{\color{green}\checkmark} Ja,
    dies sieht man, nach dem Ausmultiplizieren.} \\

  
$(x-5)(x-8)=(x-3)^2$ &
  \noTRAINER{\fbox{\,\vphantom{$X$}}}\TRAINER{\fbox{\color{red} X}
    scheinquadratisch} \\
  
\end{bbwFillInTabular}
\newpage


\subsection{Lösungsformel}
\subsubsection{Motivation zur Lösungsformel}
Wie lautet die Lösungsmenge der folgenden Gleichung?

$$x^2 - 6x = 7$$

\TNTeop{
  Dies können wir kaum ohne Trick oder allgemeines Vorgehen lösen.

  Trick

  Quadratische Egränzung = ``Wieder ganz machen'' = «al-gabr» (Sprich «tschebr»=flicken). 

  $$x^2 -6x + 9 = 16$$
  $$(x-3)^2 = 16$$
  $$\sqrt{(x-3)^2} = \pm 4$$

  $$x-3 = \pm 4$$
  $$\Longleftrightarrow x-3 = 4 \text{ oder } x-3 = -4$$

  $$\lx = \left\{1; 7\right\}$$
}


\newpage

\subsubsection{Al Chwarizmi (optional)}\index{Khwarizmi s. Al Chwarizmi}\index{Al Chwarizmi}\index{Chwarizmi (Al Chwarizmi)}

(auch Al Khwarizmi)

\noTRAINER{\bbwCenterGraphic{9cm}{allg/gleichungen/img/AlKhwarizmiBriefmarke.png}}
\TRAINER{\bbwCenterGraphic{4cm}{allg/gleichungen/img/AlKhwarizmiBriefmarke.png}}
Bildlegende: Al Chwarizmi auf einer russischen Briefmarke zum
1200jährigen Gedenken. Quelle: \texttt{de.wikipedia.org} 2021.

\TRAINER{Bemerkung Name: Abu Dscha'far Muhammad ibn Musa al-Chwarismi.}

Al-Chwarizmis Buch (
al-Kit\={a}b al-muhta\d{s}ar f\={i} \d{h}is\={a}b al-$\check{\text{g}}$abr
wa-\'{}l-muq\={a}bala)

\textit{«Kleines Büchlein über die Rechenverfahren durch Ergänzen und Ausgleichen»}

trägt zum Namen \LoesungsRaumLang{«Algebra»} bei.


\TRAINER{In seinem Buch wird die
quadratische Gleichung in sechs Varianten gelöst.}


\TRAINER{al-$\check{\text{g}}$abr = «Wieder herstellen»,
  «vervollständigen», «ganz machen», «flicken»}


Bemerkung:
\TNT{1.2}{ Mathematisches Verfahren = \textbf{Algorithmus}\index{Algorithmus}}%% END TNT


%%\newpage
%%\TALS{
\subsubsection{Quadratische Ergänzung (optional)}\index{quadratische Ergänzung}\index{Ergänzung!quadratische}
Grundform\TRAINER{ (Normalform ist, wenn $a=1$)}:
$$3x^2 - 11x - 4 = 0$$

\TNTeop{
$+4$, dann $:3 \Rightarrow$
%%
  $$x^2 - {\color{red}\frac{11}{3}}x  = \frac{4}{3}$$
  quadratisch ergänzen:
  $$x^2 - {\color{red}\frac{11}{3}}x + ({\color{ForestGreen}\frac{1}{2}}\cdot{\color{red}\frac{11}{3}})^2 = \frac{4}{3} + ({\color{ForestGreen}\frac{1}{2}}\cdot{\color{red}\frac{11}{3}})^2$$
%%
faktorisieren mit binomischer Formel:
  $$(x - \frac{1}{2}\cdot\frac{11}{3})^2 = \frac{4}{3}+ \frac{121}{36}$$
links Brüche multiplizieren, rechts ersten Bruch mit 12 erweitern:
  $$(x - \frac{11}{6})^2 = \frac{48}{36}+ \frac{121}{36}$$
%%
  $$(x - \frac{11}{6})^2 = \frac{169}{36}$$
%%
Wurzel ziehen (Achtung, es kann zwei Lösungen geben)
%%
   $$x - \frac{11}{6} = \pm\sqrt{ \frac{169}{36}}$$
%%
   $$x - \frac{11}{6} = \pm \frac{13}{6}$$
%%
   $$x  = \frac{11}{6} \pm \frac{13}{6}$$
%%
   Erste Lösung $x=\frac{24}{6} = +4$ und zweite Lösung $x =
   -\frac{2}{6} = -\frac{1}{3}$.
%%
   Probe: 4 und $-\frac{1}{3}$ einsetzen.
}%% END TNT eop

%%%%%%%%%%%%%%%%%%%%%%%%%%%%%%%%%%%

}

\newpage



\subsubsection{ABC-Formel}\index{ABR-Formel!quadratische Gleichung}\index{Mitternachtsformel}
(Auch Mitternachtsformel)

%%Einführendes Youtube Video: \texttt{youtu.be/ZywdPuXR0S0}
\youtubeLink{https://youtu.be/ZywdPuXR0S0}{Dorfuchs: Quadratische Gleichung}

\begin{gesetz}{ABC-Formel}{}
Ist eine quadratische Gleichung in Grundform ($a\ne 0$) gegeben,
$$ax^2 + bx + c = 0$$
so ist die Lösung durch die folgende Formel bestimmt:

$$x_{1,2}=\frac{-b\pm\sqrt{b^2-4ac}}{2a}$$
\end{gesetz}

\textbf{Etwas Geschichte}

In René Descartes Werk «la Geometrie (1637)» \cite{descartes1664} taucht zum ersten mal
eine Lösungsformel mit modernen Symbolen zu einer leicht anderen
Grundform ($z^2=az + b^2$) auf:

\bbwCenterGraphic{90mm}{allg/gleichungen/quadratische/img/decartesGeometrie.jpg}

\begin{center}\small{Das Original von 1634 ist schwer
    zugänglich. Jedoch im Nachtruck 1664 taucht die identische Formel
    von Descartes nochmals auf.}\end{center}

\TRAINER{$$z^2 = az + b^2$$ wird quadratisch ergänzt zu
  $$z^2 -az + \frac{a^2}{4} = \left(z-\frac{a}2\right)^2 = b^2 + \frac{a^2}4$$
  Dies wird zu
  $$z=\frac{a}2 \pm \sqrt{\frac{a^2}4 + b^2}$$
}

\newpage


\subsubsection{Beweis der ABC-Formel \GESO{(optional)}}\index{Mitternachtsformel!Beweis}\index{ABC-Formel!Beweis}
\TRAINER{nach dorfuchs: [$-c$] [$\cdot{}4a$] [$+b^2$] [TU:
    bin. Formel] [$\pm\sqrt{\,\,}$] [$-b$] [$/2a$]}
\TRAINER{oder hier einfach das VIDEO von Dorfuchs zeigen; ca. 3
  min.}%% END TRAINER

\TNTeop{%%

\begin{tabular}{c|c|c}
Zahlenbeispiel                    & Vorgehen                     &  Allgemein           \\
\hline
$2x^2-14=4x$                      &                              & Grundform:        \\
                                  & in Grundform bringen         &                     \\
$2x^2-4x-14 = 0$                  &                              &  $ax^2 + bx + c =0$  \\
                                  & $| -c $ ($x$ separieren)     &                      \\
$2x^2-4x=14$                      &                              &  $ax^2 + bx = -c$    \\
                                  & $| \cdot{} 4a$               &                      \\
$16x^2 - 32x=112$                 &                              &  $4a^2x^2+4abx=-4ac$ \\
                                  & $| + b^2$ (Binom Vorbereiten)&                        \\
$16x^2 - 32x + 16 = 128$          &                              &  $4a^2x^2+4abx+b^2 = -4ac+b^2$ \\
                                  & Binomische Formel            &                           \\
$(4x-4)^2=128$                    &                              &  $(2ax+b)^2 = b^2-4ac$ \\
                                  & $\pm\sqrt{\,}$ Wurzel ziehen &                           \\
$4x-4 = \pm\sqrt{128}$            &                              &  $2ax+b=\pm \sqrt{b^2-4ac}$ \\
                                  & nochmals $x$ separieren      &                              \\
$4x   = 4 \pm \sqrt{128}$         &                              &  $2ax=-b\pm\sqrt{b^2-4ac}$ \\
                                  & durch $2a$ teilen            &                              \\
$x     = 1 \pm \frac14\sqrt{128}$ &                              &  $x=\frac{-b\pm\sqrt{b^2-4ac}}{2a}$ \\
\hline
\end{tabular}

}%% END TRAINER TNT end of page


\newpage


\subsubsection{Anwenden der ABC-Formel}
\begin{rezept}{ABC-Formel}{rezept_abc_formel}
  Beispiel $$3x^2 = 9x - 6$$

  1. in Grundform bringen: 

\TNT{1.6}{  $$3x^2 -9x + 6 = 0$$\vspace{11mm}}%% end tnt

2. $a$, $b$ und $c$ ermitteln:

\TNT{2.8}{%%
  
  \begin{tabular}{c|c|c}
  a&b&c\\
  \hline
  3 & -9 & 6\\
  \end{tabular}
  \vspace{15mm}
}%% END TNT

3. Einsetzen in \large{ $\frac{-b \pm \sqrt{b^2-4ac}}{2a}$}

\TNT{8}{  $$\frac{-(-9) \pm
    \sqrt{(-9)^2-4\cdot{}3\cdot{}6}}{2\cdot{}3} = \frac{+9 \pm
    \sqrt{81 - 72}}{6} = \frac{9\pm 3}6$$

  $\lx = \{1; 2\}$
  \vspace{11mm}
}%% END TNT
\end{rezept}

\subsection*{Aufgaben}

%
%Mit der ABC-Formel:
%\GESOAadBMTA{181}{6. d), 8. a), b) d) 5. a) e) und  7. a)  e)}


\GESO{\olatLinkArbeitsblatt{GlQuad}{https://olat.bms-w.ch/auth/RepositoryEntry/6029794/CourseNode/111365559999287}{Aufgabe 2. und 3.}
  }%% end GESO
\TALS{\olatLinkArbeitsblatt{GlQuad}{https://olat.bms-w.ch/auth/RepositoryEntry/6029786/CourseNode/111365560138736}{Aufgabe 2. und 3.}
  }%% end GESO


\newpage


\subsubsection{$a$, $b$ und $c$ finden}

\GESO{\textbf{1. Beispiel}}%% TALS hat kein 2. Beispiel

Betrachten wir das Beispiel in der Grundform:
$$x^2 - 15 = 0$$
Dies ist gleich

$$ {\color{ForestGreen}\LoesungsRaum{1}  \cdot{} }  x^2 +
{\color{blue} \LoesungsRaum{0}\cdot{}} x {
  \color{red} \LoesungsRaum{- 15}} = 0.$$
\TNTeop{
Somit ist
\begin{tabular}{|c|c|c|}
    {\color{ForestGreen}a} & {\color{blue}b} &  {\color{red}  c} \\\hline
    {\color{ForestGreen}1} & {\color{blue}0} &  {\color{red}-15}
\end{tabular}.

Lösung:
$$x_{1,2} = \frac{-{\color{blue}0} \pm \sqrt{{\color{blue}0}^2 -
    4\cdot{}{\color{ForestGreen}1}\cdot{}{\color{red}{(-15)}}}}{2\cdot{}{\color{ForestGreen}1}}
= \pm \frac{\sqrt{4\cdot{}15}}{2} \stackrel{\text{TR}}{=} \pm \sqrt{15}$$
}

%%%%%%%%%%%%%%%%%%%%%%%%%%%%%%%5
\GESO{%% TALS hat ein solches Beispiel bei den Aufgaben mit Parametern
\textbf{2. Beispiel mit Parametern\GESO{ (optional)}}

$$x^2 - sx = 2s^2$$

\TNTeop{Grundform: $$x^2 -sx - 2s^2 = 0$$
  $A$, $B$ und $C$ finden:

\begin{tabular}{|c|c|c|}
    {\color{ForestGreen}A} & {\color{blue}B} &  {\color{red}  C} \\\hline
    {\color{ForestGreen}$1$} & {\color{blue}$-s$} &  {\color{red}$-2s^2$}
\end{tabular}

In Formel einsetzen

%%$$x_{1,2} = \abcd{a}{b}{c} = \abcd{1}{(-s)}{(-s^2)}$$
$$x_{1,2} = \frac{-b\pm\sqrt{b^2 -4ac}}{2a}$$
$a$, $b$ und $c$ einsetzen:
$$x_{1,2} = \frac{-(-s) \pm  \sqrt{(-s)^2 -4 \cdot{} 1 \cdot{} (-2s^2)}  }{2\cdot{} 1}  $$
vereinfachen:
$$x_{1,2} = \frac{s \pm  \sqrt{s^2 +8s^2}  }2 = \frac{ s \pm  \sqrt{9s^2}  }2 = \frac{s\pm 3s}2$$

$$\lx = \{-s; 2s  \}$$
}%% END TNTeop (implizites Seitenende)
}%% END GESO
%%%%%%%%%%%%%%%%%%%%%

\subsection*{Aufgaben}


\GESO{\olatLinkArbeitsblatt{GlQuad}{https://olat.bms-w.ch/auth/RepositoryEntry/6029794/CourseNode/111365559999287}{Aufgabe 4. und 5. a) b)}
  }%% end GESO
\TALS{\olatLinkArbeitsblatt{GlQuad}{https://olat.bms-w.ch/auth/RepositoryEntry/6029786/CourseNode/111365560138736}{Aufgabe 4. und 5.}
  }%% end GESO



%\GESO{\olatLinkArbeitsblatt{$a$-$b$-$c$-Finden}{https://olat.bms-w.ch/auth/RepositoryEntry/6029794/CourseNode/101937272610102}{(Alle Übungen)}}%% END olatLinkArbeitsblatt
%\TALS{\olatLinkArbeitsblatt{$a$-$b$-$c$-Finden}{https://olat.bms-w.ch/auth/RepositoryEntry/6029786/CourseNode/101937272612767}{(Alle Übungen)}}%% END olatLinkArbeitsblatt


%\GESOAadBMTA{182}{13. a) b)}

%Wer fertig ist:\\

%\GESO{\olatLinkArbeitsblatt{Quadratische
%    Gleichungen}{https://olat.bms-w.ch/auth/RepositoryEntry/6029794/CourseNode/107315676111812}{(Weiter ab derjenigen Nummer, wo Sie aufgehört hatten.)}}

%\TALS{\olatLinkArbeitsblatt{Quadratische Gleichungen}{https://olat.bms-w.ch/auth/RepositoryEntry/6029786/CourseNode/107315676197926}{(Weiter
%    ab derjenigen Nummer, wo Sie aufgehört hatten.)}}


%%\TALSAadBFWA{95ff}{269. a), 275. a) b) d), 276. a) b) f)}%%
%\TALSAadBMTA{181}{[Alle entweder mit ABC-Formel oder mit einfacherem Verfahren] 4. a) c) e), 5. a) c) e), 6. d),
% 7. a) e) und 9. a) c) e)}
\newpage



\TALS{%%
%% 2020 03 27 Ph. G. Freimann
%%

\subsubsection{Spezialfall $c=0$}

Beispiel
$$5x^2 + 3x = 0$$

\TNT{8}{
Faktorisieren
$$x(5x+3) = 0$$
$$\lx= \{-\frac35; 0\}$$
}%% END TNT


Wenn eine quadratische Gleichung der Form
$$ax^2 +bx = 0$$
gegeben ist, so kann man einfach ein $x$ ausklammern:

$$x(ax+b)=0$$
 Die Gleichung ist erfüllt, wenn nun entweder $x$ selbst oder aber
 der Klammerausdruck $(ax+b)$ Null werden. Somit haben wir sofort zwei
 Lösungen gefunden:
 $$\lx=\left\{0, \frac{-b}{a}\right\}$$

 %%\TALSAadBMTA{95ff}{265. a) b) c) e) g)}
 \newpage

 
 \subsubsection{Spezialfall $b=0$ (reinquadratische Gleichungen)}
\TadBMTA{166}{10.2.1}

Beispiel
$$5x^2 - 3 = 0$$

\TNT{8}{
Quadrat separieren:
$$5x^2 = 3$$
$$x^2 = \frac35$$
Wurzel ziehen:
$$\lx= \left\{-\sqrt{\frac35}; +\sqrt{\frac35}\right\}$$
}%% END TNT




Ist die quadratische Gleichung in der Form
 $$ax^2 + c = 0$$
 gegeben, so gibt es ebenfalls eine einfache Lösungsformel. Es folgt:
 $$ax^2 = -c$$
 und daraus:
 $$x^2 = \frac{-c}{a}$$

 Die Lösungsmenge ist also schlicht
 $$\lx=\left\{ + \sqrt{\frac{-c}{a}}, -\sqrt{\frac{-c}{a}} \right\}.$$

\newpage

 \subsubsection{Faktorisierte Form}

$$(x-7)(2x+8) = 0$$

\TNTeop{$\lx=\{-4; 7\}$, danach unbedingt Probe vorzeigen.\vspace{32mm}%%
}%% END TNTeop
%%\TALSAadBMTA{95ff}{266. b) c) d)}
\newpage
}%% END TALS
\newpage



\subsection{Diskriminante}\index{Diskriminante}\label{diskriminante}
Die \textbf{Diskriminante} ist das, was in der $a$-$b$-$c-$Formel unter der Wurzel steht (= Radikand).

$D := b^2 - 4ac$

\TRAINER{«diskriminieren = unterscheiden»}

Mit der Diskriminante ist eine einfache \textbf{Fallunterscheidung} möglich:
\begin{itemize}
\item $D > 0 \Rightarrow $ zwei Lösungen:
  $$x_{1,2}  = \frac{-b \pm \sqrt{b^2 -  4ac}}{2a}= \frac{-b \pm \sqrt{D}}{2a}$$

\item $D = 0 \Rightarrow $ eine Lösung:
  $$ x = \frac{-b}{2a}$$

\item $D < 0 \Rightarrow $ keine Lösung in $\mathbb{R}$

\end{itemize}

\subsubsection{Anwendung}
Entscheiden Sie, wie viele Lösungen die gegebenen Gleichungen haben:

%%\renewcommand{\arraystretch}{2}
\begin{bbwFillInTabular}{c|c|p{5cm}}
  $x^2 + 3x +1 = 0$ & $b^2-4ac = \LoesungsRaumLang{9 -4 > 0}$ & \LoesungsRaumLang{zwei Lösungen} \noTRAINER{\phantom{xxxxxxxxxx}} \\
  \hline
  $x^2 + 2x +1 = 0$ & $b^2-4ac = \LoesungsRaumLang{4 -4 = 0}$ & \LoesungsRaumLang{eine Lösung} \\
  \hline
  $x^2 + 1x +1 = 0$ & $b^2-4ac = \LoesungsRaumLang{1 -4 < 0}$ & \LoesungsRaumLang{keine Lösungen} \\
\end{bbwFillInTabular}
%%\newpage

\subsection*{Aufgaben}

\GESO{\olatLinkArbeitsblatt{GlQuad}{https://olat.bms-w.ch/auth/RepositoryEntry/6029794/CourseNode/111365559999287}{Aufgabe 8.}
  }%% end GESO
\TALS{\olatLinkArbeitsblatt{GlQuad}{https://olat.bms-w.ch/auth/RepositoryEntry/6029786/CourseNode/111365560138736}{Aufgabe 8.}
}%% end TALS
\newpage
%\GESO{\AadBMTA{182ff}{11. c) d) e) und f) challenge: 16. a) b)}}
%\TALS{\AadBMTA{182ff}{11. c) d) e) und f)}}%% challenge 16. a) b) bei Tals
%% Pflicht, aber im nächsten Kapitel (s. unten)



\TALS{%%
%% 2019 07 04 Ph. G. Freimann
%%

\section{Lineare Gleichungen mit Parametern}\index{Gleichungen!lineare mit Parametern}\index{Parameter}
\sectuntertitel{Alle Zahlen sind gleich; nur einige sind gleicher.}
%%\TALSTadBFWA{89}{2.2.2}
%%%%%%%%%%%%%%%%%%%%%%%%%%%%%%%%%%%%%%%%%%%%%%%%%%%%%%%%%%%%%%%%%%%%%%%%%%%%%%%%%
\subsection*{Lernziele}

\begin{itemize}
\item Lineare Gleichungen mit Parametern
\end{itemize}
\TadBMTA{116}{8.2}

Parameter = (wörtlich) «Neben-Maß»

Griechisch $\pi\alpha\rho\alpha$ (para), deutsch: ‚neben‘ und $\mu\epsilon\tau\rho\omega\nu$ (metron) deutsch: ‚Maß‘

\newpage
\subsection{Motivation}
Ein Ladenpreis von CHF 500.- setzt sich zusammen aus ursprünglichem
vom Verkäufer gewünschtem Verkaufspreis $x$ plus der Mehrwertsteuer von 7.7\%.

Wie groß war der ursprüngliche Verkaufspreis $x$?

\TNT{3.6}{
  $x + 7.7 \% \text{von} x = 500$

  $x+\frac{x}{100}\cdot{7.7} = 500$

  $x\cdot{}(1+\frac{7.7}{100})= 500$

  $x\cdot{}(1.077)= 500$

  $x = \frac{500}{1.077} \approx 464.25$
}

Ein Ladenpreis ist CHF 650.-. Darin sind 7.7\% Mehrwertsteuer
enthalten. Was war der Preis vor der Mehrwertsteuer?

\TNT{3.6}{
  $x + 7.7 \% \text{von} x = 650$

  $x+\frac{x}{100}\cdot{7.7} = 650$

  $x\cdot{}(1+\frac{7.7}{100})= 650$

  $x\cdot{}(1.077)= 650$


  $x = \frac{650}{1.077} \approx 603.53$
  
}

Damit die Rechnung nicht jedesmal gemacht werden muss, führen wir
einen Parameter ein:

Ein Ladenpreis ist CHF $p$. Darin sind 7.7\% Mehrwertsteuer
enthalten. Was war der Preis vorher?

\TNT{3.6}{
  $x + 7.7 \% \text{von} x = p$

  $x+\frac{x}{100}\cdot{7.7} = p$

  $x\cdot{}(1+\frac{7.7}{100})= p$

  $x\cdot{}(1.077)= p$


  $x = \frac{p}{1.077}$
}

Als Anwendung: Der Ladenpreis soll 99.- CHF betragen. Was soll der
Verkaufspreis vor der MwSt betragen?

\TNT{4}{Verkaufspreis $ = \frac{p}{1.077} = \frac{99}{1.077} \approx
  91.92$ CHF.

Ziel erreicht: Rechnung nur einmal machen, danach nur noch Werte in Parameter einsetzen.}

\newpage



\TALS{%% TALS Schieberegler im CAS-Rechner

\paragraph{Schieberegler}\index{Schieberegler} Der Taschenrechner \textit{TI-$n$Spire CX
  II-T CAS} kann Terme in der Variablen $x$ parametrisiert
darstellen. Erstellen Sie dazu in einem Dokument eine
\textit{notes}-Page und definieren Sie den Term $terma := a\cdot
x+0.5$. In einem neuen Grafik-Fenster (Page) zum selben Problem
definieren Sie die Funktion $f1(x):=terma$. Damit erseint die Frage
nach einem Schieberegler, welcher uns den Parameter $a$ verändern
lassen kann.

\bbwCenterGraphic{6cm}{allg/gleichungen/img/LineareFunktionTR_terma.png}
%  \begin{center}
%   \raisebox{-1cm}{\includegraphics[width=6cm]{img/LineareFunktionTR_terma.png}}
%  \end{center}


\paragraph{Frage 1} Bei welchem Parameter $b$ hat die folgende Gleichung für $x$ die Lösung $1.5$?
$$-2x + b = 0$$

\noTRAINER{$$.....................................$$}
\TRAINER{$$b = 3$$}
Dies lösen wir, indem wir die gesuchte Lösung für $x$ einsetzen und nach $b$ auf"|lösen. Graphisch kann dies auch mit dem CAS-Rechner \textit{gelöst} werden. Definieren Sie «termb := $-2\cdot{}x+b$» und zeichnen Sie den Graphen in einer Graph-Page.
Erstellen Sie einen «Slider»\index{Slider}\index{Schieberegler} für $b$ und ziehen Sie an
diesem \textit{Slider}, bis der Wert der Geraden auf der $x$-Achse den
Wert 1.5 (gesuchte Lösung) angenähert hat.


\paragraph{Frage 2} Bei welchem Parameter $b$ hat die folgende Gleichung für $x$ die Lösung $1.5$?
$$ax-3=0$$

\noTRAINER{$$.....................................$$}
\TRAINER{$$a=2$$}
Dies nähern wir auch mit dem CAS-Rechner an. Definieren Sie den Term
$terma$ neu ($terma := a\cdot{}x-3$) und zeichnen Sie den Graphen in einer Graph-Page.
Ziehen Sie am \textit{Slider}, bis der Wert der Geraden auf der
$x$-Achse dem Wert 1.5 (gesuchte Lösung) genug nahe kommt.
}




%%\TRAINER{Aufg 247 l aus \cite{frommenwiler17alg}}
\subsection{Vorzeigeaufgabe}

\TALS{
  $$\frac{x}{n} + a - \frac{x}{m} + b = mx + c$$
\TNTeop{
  1. Alle $x$ nach links, alle «nicht $x$» nach rechts:

  $$\frac{x}{d}-\frac{x}{m}-mx = c-a-b$$

  2. $x$ ausklamern:

  $$x\left(\frac1n-\frac1m-m\right)=c-a-b$$

  3. durch die Klammer teilen

  $$x=\frac{c-a-b}{\frac1n-\frac1m-m} = (\text{optional}) = \frac{mn(c-a-b)}{m-n-m^2n}$$

}%% END TNT
}%% END TALS

\GESO{
  $$x^2 - \sqrt{2} - c = (b-x)\cdot{}(-\sqrt{5}-x)$$

  \TNTeop{
  Termumformung: $x^2-\sqrt{2} - c = -\sqrt{5}b -bx +x\sqrt{5} + x^2$\\
  beidseitig $-x^2$: $\Rightarrow -\sqrt{2}-c=-\sqrt{5}b-bx+x\sqrt{5}$\\
  Alle $x$ nach links (Rest nach rechts): $\Rightarrow bx-\sqrt{5}x= -\sqrt{5}b+\sqrt{2} + c$\\
  $x$ ausklammern: $\Rightarrow x(b-\sqrt{5})=-\sqrt{5}b+\sqrt{2} + c$\\
  $: (b-\sqrt{5})$ : $\Rightarrow x = \frac{-\sqrt{5}b+\sqrt{2} + c}{b-\sqrt{5}}$\\%%

}%% END TNT
}%% END GESO
\newpage

\subsection*{Aufgaben}
%%\TALS{\aufgabenFarbe{Aufgaben: \cite{frommenwiler17alg} S. 89, Aufg. 246. a) 247. a) c) g) 248. a) d) i) 249.}}

%%\TALS{Textaufgaben: \cite{frommenwiler17alg} S. 89 Aufg. 253}

\GESO{\olatLinkArbeitsblatt{Lineare
    Gleichungen}{https://olat.bms-w.ch/auth/RepositoryEntry/6029794/CourseNode/110662976644490}{Aufgabe
    3. und 4.}}
\TALS{\olatLinkArbeitsblatt{Lineare Gleichungen}{https://olat.bms-w.ch/auth/RepositoryEntry/6029786/CourseNode/110662976730861
}{Aufgabe 3., 4. und 5.}}


%\AadBMTA{128ff}{ 7. b) c) a) e),
%  8. a) d) e) f),
%  9. a) b) c) e) f),
%  10. a) b)
%\TALS{, 11. b) und 12. a) d) e)}
%}%% end AadBMTA

\olatLinkGESOKompendium{2.1.3.}{11}{12. bis 16.}

Testen Sie Ihr Können:

\GESO{\olatLinkArbeitsblatt{Lineare
    Gleichungen}{https://olat.bms-w.ch/auth/RepositoryEntry/6029794/CourseNode/110662976644490}{Aufgabe 6.}}
\TALS{\olatLinkArbeitsblatt{Lineare Gleichungen}{https://olat.bms-w.ch/auth/RepositoryEntry/6029786/CourseNode/110662976730861
}{Aufgabe 6.}}



\newpage
}

\TALS{\subsubsection{Taschenrechner}

Was fällt bei folgender Gleichung auf?
$$72x^2 - 605 x + 848 = 0$$

  Der TI-30X Pro MathPrint kann quadratische Gleichungen mit Zahlen direkt auflösen.
  Doch dazu einige Tipps:
  
  \begin{tabular}{|c|p{13cm}|}
    \hline
    Tasten & Bemerkung \\
    \hline
    \tiprobutton{2nd}\tiprobutton{cos_poly-solv} & Starte das Lösen
    von quadratischen Gleichungen. Lösung obiger Gleichung: $\lx=\left\{\frac{53}8,\frac{16}9\right\}$\\
    \hline
    \tiprobutton{2nd}\tiprobutton{mode_quit}      & Beende den Poly-Solver.\\
    \hline
    \tiprobutton{neg} & Negative Zahlen nicht mit dem Subtraktionsoperator sondern mit der Vorzeichen-Taste eingeben.\\
    \hline
    \textbf{ACHTUNG} & Das Resultat lässt beim TI-30X Pro negative Wurzeln zu! Dies wird in der Lösung mit $i$ angegeben ($i$ steht für die imaginäre Einheit: $i=\sqrt{-1}$). Das $i$ kann aber je nach Modus erst gesehen werden, wenn mit der Pfeiltaste ganz nach rechts \textit{gescrollt} wurde.
    
    Steht ein $i$ in der Lösung des Taschenrechners, so gibt es keine reellen Lösungen und wir schreiben

    $\lx=\{\}$\\
    \hline
  \end{tabular}


\begin{rezept}{}{}
Mit dem Taschenrechner reduzieren sich Aufgaben (ohne Parameter) auf
das \textbf{Umformen} einer quadratischen Gleichung in die \textbf{Grundform}.
\end{rezept}


  \newpage
  Lösen Sie mit dem Taschenrechner:
$$5x^2 - 7x + 2 = 0$$
  \TNT{2}{$x_{1,2} = 1, 0.4$\vspace{16mm}}
  $$x^2 -6x  = - 11$$
  \TNT{2}{Keine Lösung $\lx=\{\}$}

Lösen Sie: $$4.1x^2+6x=-15$$
\TNT{2}{Auch hier verschwindet das $i$.}

  
$$1001x^2 + x  = -1$$
\TNT{2}{$x_{1,2} = \text{keine reelle Lsg. Das }i\text{ verschwindet
    aus dem Sichtfeld}$\vspace{16mm}}%% END TNT
\newpage

\subsection*{Aufgaben}

\olatLinkArbeitsblatt{GlQuad}{https://olat.bms-w.ch/auth/RepositoryEntry/6029794/CourseNode/111365559999287}{Aufgabe 12.}

%\GESOAadBMTA{182}{ 8. e) i), 9. c) und 10. b) e)}

\olatLinkGESOKompendium{2.3.2.}{16}{46. bis 49.}

und Textaufgaben:

\olatLinkGESOKompendium{2.3.4.}{17ff}{56. bis 58.}

\newpage

}
\GESO{\subsubsection{Taschenrechner}

Was fällt bei folgender Gleichung auf?
$$72x^2 - 605 x + 848 = 0$$

  Der TI-30X Pro MathPrint kann quadratische Gleichungen mit Zahlen direkt auflösen.
  Doch dazu einige Tipps:
  
  \begin{tabular}{|c|p{13cm}|}
    \hline
    Tasten & Bemerkung \\
    \hline
    \tiprobutton{2nd}\tiprobutton{cos_poly-solv} & Starte das Lösen
    von quadratischen Gleichungen. Lösung obiger Gleichung: $\lx=\left\{\frac{53}8,\frac{16}9\right\}$\\
    \hline
    \tiprobutton{2nd}\tiprobutton{mode_quit}      & Beende den Poly-Solver.\\
    \hline
    \tiprobutton{neg} & Negative Zahlen nicht mit dem Subtraktionsoperator sondern mit der Vorzeichen-Taste eingeben.\\
    \hline
    \textbf{ACHTUNG} & Das Resultat lässt beim TI-30X Pro negative Wurzeln zu! Dies wird in der Lösung mit $i$ angegeben ($i$ steht für die imaginäre Einheit: $i=\sqrt{-1}$). Das $i$ kann aber je nach Modus erst gesehen werden, wenn mit der Pfeiltaste ganz nach rechts \textit{gescrollt} wurde.
    
    Steht ein $i$ in der Lösung des Taschenrechners, so gibt es keine reellen Lösungen und wir schreiben

    $\lx=\{\}$\\
    \hline
  \end{tabular}


\begin{rezept}{}{}
Mit dem Taschenrechner reduzieren sich Aufgaben (ohne Parameter) auf
das \textbf{Umformen} einer quadratischen Gleichung in die \textbf{Grundform}.
\end{rezept}


  \newpage
  Lösen Sie mit dem Taschenrechner:
$$5x^2 - 7x + 2 = 0$$
  \TNT{2}{$x_{1,2} = 1, 0.4$\vspace{16mm}}
  $$x^2 -6x  = - 11$$
  \TNT{2}{Keine Lösung $\lx=\{\}$}

Lösen Sie: $$4.1x^2+6x=-15$$
\TNT{2}{Auch hier verschwindet das $i$.}

  
$$1001x^2 + x  = -1$$
\TNT{2}{$x_{1,2} = \text{keine reelle Lsg. Das }i\text{ verschwindet
    aus dem Sichtfeld}$\vspace{16mm}}%% END TNT
\newpage

\subsection*{Aufgaben}

\olatLinkArbeitsblatt{GlQuad}{https://olat.bms-w.ch/auth/RepositoryEntry/6029794/CourseNode/111365559999287}{Aufgabe 12.}

%\GESOAadBMTA{182}{ 8. e) i), 9. c) und 10. b) e)}

\olatLinkGESOKompendium{2.3.2.}{16}{46. bis 49.}

und Textaufgaben:

\olatLinkGESOKompendium{2.3.4.}{17ff}{56. bis 58.}

\newpage

}

\newpage


\TALS{\subsection{Fallunterscheidung}

Aus den Strukturaufgaben:


\aufgabenFarbe{Berechnen Sie die Lösung für $x\in\mathbb{R}$ in Abhängigkeit von $k$ mit einer vollständigen Fallunterscheidung für alle Parameterwerte $k\in\mathbb{R}$.
}

$$kx^2 - x^2 + 3x = 2$$

\TNTeop{

  \begin{tabular}{c|c|c}
    A & B & C  \\\hline
    $k-1$ & $3$ & $-2$ 
    \end{tabular}

    $$D = 9-4\cdot{} (k-1)\cdot{}(-2)$$
    $$D = 9 + 8(k-1) = 8k + 1$$
Allgemeine Lösung:
    $$x_{1,2} = \frac{-3 \pm \sqrt{8k+1}}{2k-2}$$

 1. Soderfall Nenner: $k=1$ ist nicht möglich.

    Mit $k-1$ reduziert sich die ursprüngliche Gleichung jedoch auf:
    $$ 3x = 2 \Longrightarrow x = \frac23$$

2. Sonderfall: Diskriminante = 0:
Das heißt $8k+1=0$ und somit ist $k=\frac{-1}8$.

Somit erhalten wir für $x$ eingesetzt in die allgemeine Lösung nur noch eine Lösung:

$$x = \frac{-3\pm\sqrt{0}}{2\cdot{}\left(\frac{-1}{8} - 1\right)} = \frac43$$

3. Sonderfall: Diskriminante ist kleiner als 0:

$$8k+1 < 0 \Longleftrightarrow k <\frac{-1}8  \Longrightarrow  \lx=\{\}$$

Abgesehen von den Sonderfällen $k=1$ und $k<\frac{-1}8$ gilt die allgemeine Lösung.
    

    
    
}%% end TNT eop
%% \newpage implicit  
}
\newpage

\subsection{Welches Verfahren (optional)}
Wann soll welches Verfahren eingesetzt werden? Dies ist eine
individuelle Fragestellung. Je mehr Erfahrung man hat, umso eher sieht
man die Spezialfälle. Die ABC-Formel funktioniert immer, doch
die anderen Verfahren (Binome, weitere Spezialfälle, Taschenrechner)
sind oft sinnvoller. Hier ein Versuch eines Überblicks:

\begin{rezept}{Welches Verfahren}{}

\GESO{
\begin{tabular}{|p{30mm}|p{53mm}|p{64mm}|}
	\hline
	Spezialfall $b=0$               & $5x^2 = 3$                   & Durch 5 dividieren, dann positive und negative Wurzel ziehen.\\
	\hline
	Spezialfall $c=0$               & $3x^2 = 5x$                   & Fallunterscheidung $x = 0$ und $x \ne 0$\\
	\hline
	quadratische Form            & $(x-2)^2 = 49$ & $\pm\sqrt{}$: \hspace{3mm}  $x-2=\pm 7$\\
	\hline
	faktorisiert       & $(x-4.6 + b)\cdot{}(x+a) = 0$ & Lösungen hier $x_1=4.6-b$ und $x_2 = -a$\\
	\hline
	Einfache Zahlen            & $x^2 -2x + 1= 0$           & ABC-Formel (Mitternachtsformel) oder faktorisieren $x^2-2x+1=(x-1)^2$ und danach jeden einzelnen Faktor $=0$ setzen.\\
	\hline
	Komplizierte Zahlen        & $7.3x^2 - 8x - 3.4 = 0$       & Taschenrechner \tiprobutton{cos_poly-solv}             \\
	\hline
%	Variable (Parameter)       & $7.3x^2 - cx + 2.6=0$         & ABC-Formel (Mitternachtsformel) \\
%	\hline
\end{tabular}
\newpage
}

\TALS{
\begin{tabular}{|p{35mm}|p{53mm}|p{60mm}|}
	\hline
	Spezialfall $b=0$               & $5x^2 = 3$                   & Durch 5 dividieren, dann positive und negative Wurzel ziehen.\\
	\hline
	Spezialfall $c=0$               & $3x^2 = 5x$                   & Fallunterscheidung $x = 0$ und $x \ne 0$\\
	\hline
	quadratische Form            & $(x-2)^2 = 49$ & $\pm\sqrt{}$: \hspace{3mm}  $x-2=\pm 7$\\
  \hline
	faktorisiert & $(x-4.6 + b)\cdot{}(x+a) = 0$ & Lösungen hier $x_1=4.6-b$ und $x_2 = -a$\\
	\hline
	Einfache Zahlen      & $6x^2 -3x - 18 = 0$               & ABC-Formel (Mitternachtsformel) \\
	\hline
	Komplizierte Zahlen  & $7.3x^2 - 8x - 3.4 = 0$           & Taschenrechner (solve())                \\
	\hline
	Variable (Parameter) & $7.3x^2 - cx + 2.6=0$         & Taschenrechner                          \\
	\hline
	Anzahl Lösungen auf eine beschränken & $7.3x^2 - cx + 2.6 = 	0$     & 	Diskriminante mit Taschenrechner Null setzen.\\
	\hline
\end{tabular} 
}

\end{rezept}

\subsection*{Aufgaben}

\GESO{\olatLinkArbeitsblatt{GlQuad}{https://olat.bms-w.ch/auth/RepositoryEntry/6029794/CourseNode/111365559999287}{Aufgabe
15. Bsp. a) c) e) g) j) l) n)}
}%% end GESO

\TALS{\olatLinkArbeitsblatt{GlQuad}{https://olat.bms-w.ch/auth/RepositoryEntry/6029786/CourseNode/111365560138736}{Aufgabe
15. Bsp. a) c) e) h) l) n) q)}
}%% end TALS


\newpage


