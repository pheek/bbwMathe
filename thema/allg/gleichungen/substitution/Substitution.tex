%%
%% 2020 02 fp@bbw.ch
%%

%% Quadratische Gleichungen: Substitution
\section{Substitution}\index{Substitution!quadratische Gleichungen}

\TNT{2.4}{
Zur Erinnerung beim Faktorisieren:\\
$a(2s-r) + b(2s-r) {\text{Subst. }T:=(2s-r) \over =} aT + bT = (a+b)T {\text{Rücksusbst. }2s-r=T\over = } (a+b)(2s-r)$
}

Oft können kompliziertere Gleichungen mittels einer geeigneten \textbf{Substitution} (= Ersetzung\footnote{lat. \textbf{substitu\=o} = an die Stelle \textit{jmds. o. einer Sache} setzen.
})
in eine einfachere verwandelt werden. Wie \zB hier:

\newcommand\tmpPart{\left(\frac{1}{3}(5x+0.5)\right)}

$$\tmpPart{}^2 - 10 = 3\cdot{}{\tmpPart{}}$$

%%
\TRAINER{«Mit Farben geht alles besser.»\\}%%
\TNT{2.4}{$${\color{ForestGreen}\tmpPart{}}^{\color{black}2} -10 = 3{\color{ForestGreen}{\tmpPart{}}}$$}

\textbf{Substitution}

\TNT{1.2}{Wir ersetzen: $u := {\color{ForestGreen}\tmpPart{}}$}

und die ursprüngliche Gleichung ist nun äquivalent zu
\TNT{1.6}{$${\color{ForestGreen}u}^2 - 10 = 3{\color{ForestGreen}u}$$
  $$u^2 -3u - 10 = 0$$
}

\begin{bbwFillInTabular}{c|c|c}
  a&b&c\\
  \LoesungsRaum{1}&\LoesungsRaum{-3}&\LoesungsRaum{-10}
  \end{bbwFillInTabular}

\GESO{Substituierte Gleichung \GESO{mit TR }lösen:}
\TALS{Substituierte Gleichung mit $abc$-Formel:

\TNT{3.2}{
  $u_{1,2} = \frac{3\pm\sqrt{9+40}}{2} = \frac{3\pm 7}{2}$}%% END TNT
}%% END TALS


$$\LoesungsMenge{}_u=\LoesungsRaumLang{\{\LoesungsRaum{-2}; \LoesungsRaum{5}\}}$$

\TRAINER{\textbf{Nomenklatur:}\\
  Substitutionsbasis: $\tmpPart{}^2 - 10 = 3\tmpPart{}$\\
  Substituendum: $\tmpPart{}$\\
  Substituens: $u$
}%%

\newpage


\textbf{Rücksubstitution}\index{Rücksubstitution}\\
Nun setzen wir die Lösungen anstelle der substituierten Variable ein:


\TNTeop{
$$1.: u_1 = -2 = \tmpPart{} \Rightarrow -6 = 5x+0.5 \Rightarrow x =
  -6.5 = 5x \Rightarrow -1.3 = x$$

$$2.: u_2 = 5 = \tmpPart{} \Rightarrow 15=5x+0.5 \Rightarrow 14.5 = 5x
  \Rightarrow x = 2.9$$

$$\lx=\{\LoesungsRaum{-1.3}; \LoesungsRaum{2.9}\}$$}%% END TNTeop

%%%%%%%%%%%%%%%%%%%%%%%%%%%%%%%%%%%55555
\newpage

\subsection*{Aufgaben}


\AadBMTA{184}{29. a) 30. a)}


\GESO{  \aufgabenFarbe{Berufsmaturitätsprüfung 2020: Aufgabe 4 von Serie 2:
  $$(x-5)^4 - \frac{(x-5)^2}{3} = 8$$}%% END Aufgabenfarbe
\TNT{14}{Lösung der Substituierten $u=(x-5)^2$ ist $u=3$
  bzw. $u=-\frac83$. Die zweite Lösung kommt jedoch nur als
  Scheinlösung vor, denn $(x-5)^2$ kann nicht negativ sein.

  Danach Rücksubstitution:

  $$u=3=(x-5)^2$$
  Wurzel ziehen und $\pm$ nicht vergessen:
  $$\pm\sqrt{3} = x-5$$
  
  $$5\pm\sqrt{3} = x$$
  \vspace{45mm}}%% END TNT
    $$\lx = \LoesungsRaumLang{\left\{5-\sqrt{3}; 5+\sqrt{3}\right\}}$$
}%% END GESO
  
  %% TALS hat hier ein einfacheres Beispiel, da TALS ohne TR lösen muss:
  \TALS{
\aufgabenFarbe{$$\left(3x-\frac15\right)^2-4\left(3x-\frac15\right)=21$$}%% end AufgabenFarbe
\TNTeop{
  Substitution:
  $$u:= 3x-\frac15$$
  $$u^2 -4u -21 = 0$$
  $$u_{1,2} = \frac{4\pm\sqrt{16+84}}{2}=\frac{4\pm 10}{2}\Rightarrow u_1 = 7; u_2 = -3$$

  Resub:

  $$ 7 = 3x-\frac15 \Rightarrow 7.2=3x \Rightarrow x = 2.4$$
  $$-3 = 3x-\frac15 \Rightarrow -2.8=3x \Rightarrow x =
  \frac{-28}{30} = \frac{-14}{15}$$
}
  }%% END TALS
\newpage
  
  \subsection{Nullserie}
  Die folgende Aufgabe stammt aus der \textit{Nullserie}-Maturaprüfung
  \TALS{Gesundheitlich-soziale Ausrichtung}:

  Lösen Sie mit Hilfe einer geeigneten Substitution \TRAINER{Tipp: $Z^{10} = \left(Z^5\right)^2$ }:
  $$\left(\frac{x}{3}-3.5 \right)^{10} -2\left( \frac{x}{3} - 3.5 \right)^5 =-1$$
  
  \TNTeop{Substituiere $u = \left(\frac{x}{3} - 3.5\right)^5$. Somit lautet die Gleichung
$$u^2 - 2u +1 = 0$$
    Was uns zu $u = 1$ bringt. Damit ist $1 = \left(\frac{x}{3} - 3.5 \right)^5$ und somit auch $1 = \frac{x}{3} - 3.5$. Auf beiden Seiten 3.5 addieren und danach mit 3 multiplizieren liefert $x = 13.5$.
 }%% End TNT
  
  %%%%%%%%%%%%%%%%%%%%%%%%%%%%%%%%%%%%
  

\TALS{\newpage
  \aufgabenFarbe{
    $$-6\sqrt{x-5} + (x-5)=-9$$
    }%% end Aufgabenfarbe
  \TNT{8}{
    $$u:=\sqrt{x-5}$$
    $$-6u+u^2=-9$$
    $$u^2-6u+9=0$$
    $$(u-3)^2=0$$
    $$\LoesungsMenge{}_u = \{3\}$$
    Resub:
    $$3=u=\sqrt{x-5}$$
    $$9=x-5$$
    $$\LoesungsMenge{}_x = \{14\}$$
  }%% END TNT
}%% END TALS


\AadBMTA{184}{28. a) d) 29. c) 30. b)}
%%\TALSAadBMTA{99ff (Substitution)}{283. a) b) f) 282. a) f) 284. a)}


\olatLinkGESOKompendium{2.3.3.}{17}{52. bis 54.}
\newpage
