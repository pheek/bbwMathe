%%  OLAT Arbeitsblatt
\GESO{\olatLinkArbeitsblatt{Potenz- und Wurzelgleichungen}{https://olat.bbw.ch/auth/RepositoryEntry/572162163/CourseNode/101937264586296}{Aufg. Kap. 2.2}}%% END olatLinkArbeitsblatt
\TALS{\olatLinkArbeitsblatt{Potenz- und Wurzelgleichungen}{https://olat.bbw.ch/auth/RepositoryEntry/572162090/CourseNode/105177901239296}{Aufg. Kap. 2.2}}%% END olatLinkArbeitsblatt
%%
%% 2019 07 04 Ph. G. Freimann
%%

\section{Potenz- und Wurzelgleichungen}\index{Gleichungen!mit Potenzen}\index{Potenzgleichungen}\index{Wurzelgleichungen}\index{Gleichungen!mit Wurzeln}

\GESOTadBMTA{191}{11}
%%\TALSTadBFWA{}{}
%%%%%%%%%%%%%%%%%%%%%%%%%%%%%%%%%%%%%%%%%%%%%%%%%%%%%%%%%%%%%%%%%%%%%%%%%%%%%%%%%
\subsection*{Lernziele}

\begin{itemize}
\item Potenzgleichungen
\item Wurzelgleichungen
\end{itemize}
\newpage

%%
%% 2020 05 07 Ph. G. Freimann
%%

%% Überblick über die Begriffe
%% Potenz, Potenzgleichung, Exponentialgleichung, Wurzelgleichung

\subsection{Überblick über Gleichungen mit Potenzen}

\begin{bbwFillInTabular}{|p{52mm}|p{52mm}|p{52mm}|}\hline

  Potenzwert               & Basis gesucht                       &  Exponent gesucht      \\
  (berechnen)              & \TRAINER{Potenzgleichung}           &  \TRAINER{Exponentialgleichung}       \\
  \hline
  \multicolumn{3}{c}{Zehnerpotenzen}\\
%%  \multicolumn{3}{c}{\,}\\ %% Generiere etwas Abstand

  \hline
  $10^3=x$                 & $x^3=1000$                           &  $10^x=1000$               \\
  \hline
  $x=\TRAINER{1000}$  & $ x=\TRAINER{\sqrt[3]{1000}=10}$          & $x   =  \TRAINER{\log_{10}(1000) = 3}$     \\\hline
%%  \multicolumn{3}{c}{\,}\\ %% Generiere etwas Abstand
  \multicolumn{3}{c}{Beispiel Zweierpotenzen}\\
  \hline
  $2^5=x$                  & $x^5=32$                             &  $2^x=32$                  \\
  \hline
  $x=\TRAINER{2\cdot{}2\cdot{}2\cdot{}2\cdot{2}=32}$ & $x=\TRAINER{\sqrt[5]{32}=2}$        & $x = \TRAINER{\log_{2}(32)=3}$        \\\hline

%%  \multicolumn{3}{c}{\,}\\ %% Generiere etwas Abstand
  \multicolumn{3}{c}{\GESO{(Optional)}\TALS{Erinnerung} --- Wurzeln sind rationale Exponenten:}\\
  \hline
  $\sqrt[3]{1000}=x$       & $\sqrt[3]{x}=10$                    &  $\sqrt[x]{1000}=10$               \\
  \hline
  $1000^{\frac{1}{3}}=x$     & $x^{\frac{1}{3}}=10$                   &  $1000^{\frac{1}{x}}=10$               \\
  $x=\TRAINER{\sqrt[3]{1000}=10}$       & $\TRAINER{\left(x^\frac13 \right)^3}=\TRAINER{10^3}$  & $\TRAINER{\frac{1}{x} }\noTRAINER{\hspace{2cm}}=\TRAINER{  \log_{1000}(10)}$      \\
                           &   $\TRAINER{x}\noTRAINER{\hspace{15mm}}=\TRAINER{1000}$                                  & $\TRAINER{\frac{1}{x}}\noTRAINER{\hspace{2cm}} = \TRAINER{ \frac{1}{3}}$         \\
                           &                                      & $x = \TRAINER{3}$                      \\\hline
\end{bbwFillInTabular}



\TALS{
  \begin{center}\TRAINER{Letzter Typ: }Logarithmische Gleichungen\end{center}

    Beispiel: $\log_2(x-7) = 5$

    Meist können logarithmische Gleichungen (wie obige) in die
    Potenzschreibeweise umgeschrieben werden und sind dann einfacher
    lösbar:

    \TNTeop{$$\log_2(x-7) = 5 \Longleftrightarrow 2^5 = x-7
      \Longleftrightarrow x=39$$}%% END TNTeop
}%% END TALS

\GESO{Bem. zum Logarithmus zur Basis 2 ($\log_{2}(32)$): Dazu müssen
  Sie die

  \tiprobutton{ln_log}-Taste
  auf Ihrem TI-30 Taschenrechner 3x drücken.
}%% END GESO
\newpage

\newpage


\subsection{Potenzgleichung}\index{Potenzgleichungen}
\begin{definition}{Potenzgleichung}{}
  Bei \textbf{Potenzgleichungen}\index{Potenzgleichung}\footnote{Potenzgleichungen sind nicht zu
  verwechseln mit Exponentialgleichungen, bei denen das $x$ im
  Exponenten steht: $5^{2x-1}=7$} ist die Gesuchte ($x$) in der Basis
  einer Potenz.
\end{definition}

Im folgenden Beispiel kommt $x$ in der 5-ten Potenz vor:

$$x^5 = 1024$$

Typischerweise löst man diese Gleichungen mit der $n$-ten Wurzel.

\TNT{4}{
\begin{tabular}{rccl}
  \             & $x^5$           &=& 1024             \\
  $\Rightarrow$ & $\sqrt[5]{x^5}$ &=& $\sqrt[5]{1024}$ \\
  $\Rightarrow$ & $x$             &=& $4$ 
\end{tabular}
}%% END TNT

Was ist die Lösungsmenge der folgenden Potenzgleichung?

$$x^2 = 100$$

\TNT{2.4}{%%
$$x^2 = 100 \textrm{ somit } x=\pm\sqrt{100} = \pm 10.$$}%% END TNT

Bestimmen Sie auch die Lösungsmenge der folgenden Gleichung:
$$x^4 = 81$$
\TNT{2.4}{%%
$x^4 = 81$ heißt, $x=\pm \sqrt[4]{81} = \pm 3$
}%% end TNT

\newpage
\subsubsection{Vorzeichen}
Vorsicht ist geboten bei negativen Zahlen.

\begin{gesetz}{}{}
  Gerade Exponenten (2, 4, 6,
8, ...) zwingen die Potenz immer dazu, positiv zu werden. Beispiel:

$$x^6 \ge 0$$

Der Potenzwert bei ungeraden Exponenten hat dasselbe Vorezeichen wie die
Basis.

Beispiel:

$$(-3)^5 = \LoesungsRaumLang{((-1)\cdot{}(3))^5} = \LoesungsRaum{(-1)^5 \cdot 3^5} = - (3^5)$$
\end{gesetz}


Beachten Sie die folgenden typischen Fälle:

 \renewcommand{\arraystretch}{2}

\begin{tabular}{|c|l|}
  \hline
  $x^2 = 4$ & $\lx=\LoesungsRaumLang{\{-2; +2\}}$ \\
  \hline
  $x^3 =  8$& $\lx=\LoesungsRaumLang{\{2\}}$ \\
  \hline
  $x^3 = -8$& $\lx=\LoesungsRaumLang{\{-2\}}$ \\
  \hline
  $x^6 = -5$& $\lx=\LoesungsRaumLang{\{\,\}}$ \\
  \hline
  $x^7 = -5$& $\lx=\LoesungsRaumLang{\{-\sqrt[7]{5}\}}$ \\
  \hline
  \end{tabular} 

 \renewcommand{\arraystretch}{1}

 \newpage
 
\begin{gesetz}{}{}

 \begin{center}\fbox{Gegeben $x^n=a$, $n\ne 0$}\end{center}
 \begin{center}\fbox{Gesucht $\lx$}\end{center}

 \renewcommand{\arraystretch}{2}

 \begin{tabular}{l|l|l}
                        & $a>0$                                  & $a<0$                      \\\hline
  $n$ gerade            & $\lx =\LoesungsRaum{\{-\sqrt[n]{a}; +\sqrt[n]{a}\}}$ & $\lx=\LoesungsRaum{\{\}}$               \\\hline
  $n$ \textbf{un}gerade & $\lx = \LoesungsRaum{\{+\sqrt[n]{a}\}}$               & $\lx = \LoesungsRaum{\{-\sqrt[n]{|a|}\}}$
 \end{tabular}  

 \renewcommand{\arraystretch}{1}

\end{gesetz}
Bemerkung zu $a<0$ und $n$ \textbf{un}gerade: Für den TI-Taschenrechner ist $-\sqrt[3]{5} = \sqrt[3]{-5}$.



\subsection*{Aufgaben}


%%  OLAT Arbeitsblatt
\GESO{\olatLinkArbeitsblatt{Potenz- und Wurzelgleichungen}{https://olat.bbw.ch/auth/RepositoryEntry/572162163/CourseNode/101937264586296}{Aufg. Kap. 1 (Bei 1.2 hilft evtl. die Substitution)}}%% END olatLinkArbeitsblatt
\TALS{\olatLinkArbeitsblatt{Potenz- und Wurzelgleichungen}{https://olat.bbw.ch/auth/RepositoryEntry/572162090/CourseNode/105177901239296}{Aufg. Kap. 1 (Bei 1.2 hilft evtl. die Substitution)}}%% END olatLinkArbeitsblatt

\newpage

\subsection{Wurzelgleichungen}\index{Wurzelgleichung}

\TALS{(\cite{frommenwiler17alg} S.111 (Kap. 2.4.2))}

\begin{definition}{Wurzelgleichung}{}
Kommt in der Gleichung die Gesuchte unter der Wurzel vor, so sprechen
wir von einer \textbf{Wurzelgleichung}.
\end{definition}
Einführungsbeispiel:
$$\sqrt[5]{x}=6$$


Dies löst man indem man beide Seiten der Gleichung potenziert:
\TNT{4}{
  
\begin{tabular}{rccl}
  \             & $\sqrt[5]{x}$   &=&     6      \\
  $\Rightarrow$ & $(\sqrt[5]x)^5$ &=&  $6^5$     \\
  $\Rightarrow$ & $x$             &=& $7\,776$ 
\end{tabular}
}%% END TNT

Lösen Sie das folgende Musterbeispiel:
$$\sqrt{x}+1=2x$$
\TRAINER{Das obige Beispiel von M. Rohner zeigt viele Sackgassen und Dinge, auf die man noch achten muss.}

\GESO{Lösungsverfahren im Buch \cite{marthaler21alg} Seite 192 im roten
  Kasten.}

\platzFuerBerechnungenBisEndeSeite{}

\TRAINER{1. Wurzel separieren
  $$\sqrt{x}=2x-1$$ quadrieren:
  $$x=4x^2-4x+1$$
  $$0=4x^2-5x+1$$
  $$x_{1,2}=\frac{+5 \pm \sqrt{25-16}}{8}$$
  $$\lx={1}$$, denn $\frac{1}{4}$ ist eine durchs Quadrieren erschienene Scheinlösung.}

\newpage
\begin{bemerkung}{Scheinlösung}{}
  
\textbf{Achtung}: Beim \textbf{Radizieren} auf beiden Seiten
einer Gleichung können \textbf{Lösungen} \textbf{verschwinden}.

Beim \textbf{Potenzieren} können (Schein)\textbf{Lösungen}
\textbf{hinzukommen} \GESO{Bsp. S.191 und roter Kasten S. 192
  im \cite{marthaler21alg}}.
\end{bemerkung}


\GESO{optional:}
\begin{rezept}{Wurzelgleichung lösen}{}
  \begin{enumerate}
  \item Definitionsbereich $\mathbb{D}$ festlegen
  \item Eine Wurzel isolieren (separieren)
  \item Quadrieren
  \item 2. und 3. Schritt wiederholen, bis keine Wurzeln mehr
    vorkommen
  \item Auflösen
  \item Probe:
    \begin{enumerate}
    \item Mit Definitionsbereich $\mathbb{D}$ abgleichen
    \item In ursprüngliche Gleichung einsetzen
    \end{enumerate}
  \end{enumerate}
\end{rezept}
\newpage


\GESO{Optionales}
Beispiel: $2\cdot{}\sqrt{x+1} + 2 = 5 + 2\cdot{}\sqrt{x-2}$

\TNT{16}{\bbwCenterGraphic{16cm}{allg/gleichungen/img/WurzelgleichungLoesungsweg.png}
  \vspace{3cm}
  $$\lx = \left\{\frac{33}{16}\right\}$$}

\subsection*{Aufgaben}
  \TALSAadBMTA{112ff}{339. d) e), 340. c), 342. d), 344. a) b) c) [Tipp
      bei c): Substitution]}

\GESOAadBMTA{
  196 (Wurzelgleichungen)}{2., 3. a)-f),  4. a) c) d) f),
  5. a) c), 6. b),  7. d), 8. c) d) und Textaufgabe 10. a) und b)}

\newpage


\subsection{Spezielle Exponenten}
\subsubsection{negative
  Exponenten}\index{Exponenten!negative}\index{negative Exponenten}
Bei Potenzgleichungen mit negativen Exponenten ist am Einfachsten
folgendes Potenzgesetz anzuwenden:

$$a^{-n} = \frac1{a^n}$$

Beachte, dass es manchmal zwei Lösungen, manchmal keine gibt.

Beispiel:

$$x^{-4} = 16$$

\TNT{6}{
  Potenzgesetz:   $$\frac1{x^4} = 16$$
  Kehrwert: $$x^4 = \frac1{16}$$
  4. Wurzel: $$x=\pm \sqrt[4] {\frac1{16}}=\frac1{\sqrt[4]{2}}$$
$$\lx = \{-\frac12; \frac12\}$$}%% END TNT

\subsection*{Aufgaben}

%%  OLAT Arbeitsblatt
\GESO{\olatLinkArbeitsblatt{Potenz- und Wurzelgleichungen}{https://olat.bbw.ch/auth/RepositoryEntry/572162163/CourseNode/101937264586296}{Aufg. Kap. 2.1}}%% END olatLinkArbeitsblatt
\TALS{\olatLinkArbeitsblatt{Potenz- und Wurzelgleichungen}{https://olat.bbw.ch/auth/RepositoryEntry/572162090/CourseNode/105177901239296}{Aufg. Kap. 2.1}}%% END olatLinkArbeitsblatt



\newpage

\subsubsection{Rationale Exponenten}\index{Exponenten!rationale}\index{Wurzeln}

 Eine Wurzelgleichung ist nichts anderes als eine
Potenzgleichung mit rationalen Exponenten.

\TRAINER{Anstelle von $\sqrt[n]{x}$ hatten wir auch  $x^{\frac{1}{n}}$
  geschrieben.}

Beispiel: $\sqrt[4]{x^3} = 125$

\TNT{5.2}{

\begin{tabular}{rccl}
%%                & $\sqrt[4]{x^3}$               &=&   125                \\
  $\Rightarrow$ & $x^{\frac{3}{4}}$               &=&   125                \\
  $\Rightarrow$ & $(x^{\frac{3}{4}})^\frac{4}{3}$  &=& $(125)^\frac{4}{3}$  \\
  $\Rightarrow$ & $x^{\frac{3}{4}\cdot\frac{4}{3}}$  &=& $(\sqrt[3]{125})^4$  \\
  $\Rightarrow$ & $x^1$                          &=& $5^4$  \\
  $\Rightarrow$ & $x$                            &=& $625$  
\end{tabular}

}%% END TNT

\subsection*{Aufgaben}

%%  OLAT Arbeitsblatt
\GESO{\olatLinkArbeitsblatt{Potenz- und Wurzelgleichungen}{https://olat.bbw.ch/auth/RepositoryEntry/572162163/CourseNode/101937264586296}{Aufg. Kap. 2.2}}%% END olatLinkArbeitsblatt
\TALS{\olatLinkArbeitsblatt{Potenz- und Wurzelgleichungen}{https://olat.bbw.ch/auth/RepositoryEntry/572162090/CourseNode/105177901239296}{Aufg. Kap. 2.2}}%% END olatLinkArbeitsblatt
