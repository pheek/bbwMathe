%%
%% 2019 07 04 Ph. G. Freimann
%%
\section{Textaufgaben}\index{Textaufgaben}
\sectuntertitel{Mathe ist der einzige Ort, wo Leute 144 Eier kaufen
  und sich keiner fragt wozu.}

%%%%%%%%%%%%%%%%%%%%%%%%%%%%%%%%%%%%%%%%%%%%%%%%%%%%%%%%%%%%%%%%%%%%%%%%%%%%%%%%%
\subsection*{Lernziele}

\begin{itemize}
\item Texte analysieren, Unbekannte bestimmen
\item Gleichung aufstellen
\item Probe, Lösungsmenge
\end{itemize}
(Beispiele: Zahlenrätsel, Geschwindigkeit, Arbeit, Leistung)

\TadBMTA{124}{8.5}


\newpage
\subsection{Vorzeigeaufgabe}
\bbwCenterGraphic{16cm}{allg/gleichungen/img/RockDesBedienten.jpg}
\cite{bardey1878} (Seite 135 Aufgabe 44): Ein Bedienter sollte von seinem Herrn für das Jahr 180 Mark und einen Rock haben. Als er fünf Monate im Dienst gestanden hatte, erhielt er 54 Mark und den Rock. Wie hoch wurde der Rock gerechnet?

\TNTeop{
  1. Der Rock hat einen gerechneten Preis \textbf{in} Mark. Jeder Monat sollte gleich viel verdient werden.

  2. $x$ = Preis des Rocks in Mark.

  3. $x+180$: Jahreslohn; $x+54$: 5 facher Monatslohn

  $\frac{x+180}{12}$: Monatslohn und $\frac{x+54}5$: Monatslohn

  4. Gleichung
  $$\frac{x+180}{12} = \frac{x+54}{5}$$

  5. Lösen (\zB TR): $x=$ 36 Mark.

  6. Probe: Monatslohn = 18 Mark (auf beide Arten gerechnet)

  7. Antwort: Der Rock wurde zu 36 Mark gerechnet.
}


\newpage

\begin{rezept}{systematisches Lösen von Textaufgaben}{}
\index{Textaufgaben!Gleichungen}
Die folgenden Vorgehensschritte\footnote{Entliehen aus dem Skript von Michael
  Rohner (BBW)} sind sehr hilfreich, auch wenn einige davon bei
  einfacheren Aufgaben gut übersprungen werden können:

\begin{enumerate}\label{textaufgaben_verfahren_in_sieben_schritten}
\item Aufgabe analysieren und verstehen:
  \begin{itemize}
  \item zuerst ganz durchlesen
  \item Situation in eigenen Worten formulieren / Aufgabe verstehen.\\
    (ev. eine Skizze / Tabelle machen)
  \item gegebene \& gesuchte Informationen herausschreiben
  \end {itemize}
  
\item Unbekannte Größe(n) eindeutig einführen
  (Variable immer möglichst präzise (inkl. Einheiten) definieren!)
  \\
  $x$ = .... \textbf{in} m / \textbf{in} kg / \textbf{in} Stunden  /
  ...
  \\
  Jede Variable hat...
  \begin{itemize}
  \item ... einen \textbf{Name}n (\zB $x$),
  \item eine präzise \textbf{Bedeutung} und
  \item eine Messgröße inkl. Maßeinheit.
  \end{itemize}
  
\item mit den gegebenen Informationen Terme bilden
  
\item mit den Termen Gleichung(en) aufstellen
  (Gesuchtes \& Gegebenes in eine Beziehung zueinander bringen.)

\item Gleichung lösen

\item Kontrolle: Ergibt mein Ergebnis Sinn? (Plausibilitätsüberlegung)

\item  Antwortsatz geben (d.\,h. die mathematische Lösung sprachlich
    beschreiben)
    
\end{enumerate}    
\end{rezept}

\newpage

\subsection*{Aufgaben}

\GESO{\olatLinkArbeitsblatt{Textaufgaben}{https://olat.bms-w.ch/auth/RepositoryEntry/6029794/CourseNode/111070380723460}{Aufgaben 1., (optional 2.) 3. und 4.}}
\TALS{\olatLinkArbeitsblatt{Textaufgaben}{https://olat.bms-w.ch/auth/RepositoryEntry/6029786/CourseNode/111070380764968}{Aufgaben 1., (optional 2.) 3. und 4.}}%% end tals


\AadBMTA{131ff}{24., 25., 27., 28., 29., 42., 43}
\TALS{\AadBMTA{133ff}{47.-51., 54., 59. und 62.}}

\olatLinkGESOKompendium{2.1.4.}{11}{18. und 21}

Schwierigere Aufgaben (Knacknüsse):
\AadBMTA{131ff}{26., 30. und 31.}

