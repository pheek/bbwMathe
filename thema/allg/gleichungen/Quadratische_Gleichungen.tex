%%
%% 2019 07 04 Ph. G. Freimann
%%
\section{Quadratische Gleichungen\TALS{ I}}\index{Gleichungen!quadratische}
\sectuntertitel{
al-Kit\={a}b al-muhta\d{s}ar f\={i} \d{h}is\={a}b al-gabr wa-\'{}l-muq\={a}bala
}

%%\GESOTadBMTA{165}{10}
%%\TALSTadBFWA{93}{2.3}
\TadBMTA{165}{10}

\newpage

%%%%%%%%%%%%%%%%%%%%%%%%%%%%%%%%%%%%%%%%%%%%%%%%%%%%%%%%%%%%%%%%%%%%%%%%%%%%%%%%%
\subsection*{Lernziele}

\begin{itemize}
\item Grundform
\item Lösungsmethoden
  \begin{itemize}
    \item{Fallunterscheidung}
    \item{Zerlegung in Linearfaktoren}
    \item{Substitution}
  \end{itemize}
\item Allgemeine Form, ABC-Formel (Mitternachtsformel)
%%   \item Rationale Gleichungen (<- die Bruchgleichungen haben ein eigenes Kapitel)
\item Definitionsbereich
\item Taschenrechner: Ohne Parameter
\item Mögliche Lösungsverfahren und sinnvolle Anwendung
\TALS{\item Taschenrechner: Visualisierung, Interpretation}
\TALS{\item Linearfaktoren}
\TALS{\item Substitution}
\end{itemize}
\newpage

\TALS{%%

%% Voraussetzungen fürs Verständnis der quadratischen Gleichungen

\subsection{Voraussetzungen / Repetition}
\TRAINER{selbständig\\}
A) Berechnen Sie:

$$(-5)\cdot{}(-5) = \LoesungsRaum{+25}$$

B) Faktorisieren Sie:

$$x^2-12x+36$$

\TNT{6}{
  Binomische Formeln

  $$x^2-12x+36 = (x-6)\cdot{}(x-6)=(x-6)^2$$
}


C) Finden Sie die Lösungsmenge für die Variable $x$:

$$|x-2| = 7$$

\TNTeop{a) $x-2 = 7$

  b) $x-2=-7 \Longrightarrow |x-2| = 7$

  $$\lx = \{-5; 9\}$$

}


\newpage
}

\subsection{Al Chwarizmi}\index{Al Chwarizme}\index{Chwarizmi (Al
  Chwarizmi)}

(auch Al Khwarizmi)

\bbwCenterGraphic{6cm}{allg/gleichungen/img/AlKhwarizmiBriefmarke.png}
Bildlegende: Al Chwarizmi auf einer russischen Briefmarke zum
1200jährigen Gedenken. Quelle: \texttt{de.wikipedia.org} 2021.

Al-Chwarizmis Buch (
al-Kit\={a}b al-muhta\d{s}ar f\={i} \d{h}is\={a}b al-gabr wa-\'{}l-muq\={a}bala) \textit{über die Rechenverfahren durch Ergänzen und
Ausgleichen} trägt zum Namen «Algebra» bei. In seinem Buch wird die
quadratische Gleichung in allen Varianten gelöst.

\subsubsection{Einstiegsaufgabe}
Auf \textit{Insta} werden von einer Klasse innerhalb eines W-LANs 650
\textit{Posts} getätigt. Wir gehen davon aus, dass dabei jedes
Profilbild allen anderen geschickt wurde. Wie viele Personen waren
beteiligt?

\TNT{5.2}{
  $n$ sei die Anzahl Schüler.

  Jeder sendet $n-1$ Post; d.\,h. es werden $n\cdot{}(n-1)$ Posts
  versendet.

  Gleichung:

  $n\cdot{}(n-1) = 650 \Longrightarrow   n^2 - n = 650$

  Lösung 26; dies kann \zB durch Ausprobieren ermittelt werden.
}


\begin{definition}{quadatische Gleichung}{}\index{quadratische Gleichung}\index{Gleichung!quadratische}
  Bei einer \textbf{quadratischen Gleichung} kommt die Gesuchte in der
  2. Potenz vor; \zB $x^2$.
\end{definition}

\newpage


\subsection{Einstiegsbeispiele}
$$x^3=343$$

\TNT{2}{$x=7$}

$$x^2=9$$

\TNT{4}{$\lx = \{-3; 3\}$ \vspace{15mm}}

$$(x+5)^2=50$$

\TNTeop{$x+5=\pm\sqrt{50} \Rightarrow x = \pm\sqrt{50} - 5$, danach unbedingt Probe vorzeigen.\vspace{32mm}}
\newpage
%%\TRAINER{Normalform $x^2 +5x +6 = 0 \rightarrow (x+2)(x+3)=0$}
%%\newpage

\subsection*{Aufgaben}

\GESO{\olatLinkArbeitsblatt{Quadratische Gleichungen}{https://olat.bbw.ch/auth/RepositoryEntry/572162163/CourseNode/101586190756718}{bis
    7. )}}

\olatLinkGESOKompendium{2.3.1.}{16}{42. bis 44.}

\TALS{\olatLinkArbeitsblatt{Quadratische Gleichungen}{https://olat.bbw.ch/auth/RepositoryEntry/572162090/CourseNode/102901170063995}{bis
    7. )}}


%%\aufgabenFarbe{Aufgabenblatt im OLAT (Gallin) bis Aufgabe 7}


\AadBMTA{181}{3. a) d) e), 4. e) [Bei Aufgabe 4. e) die Probe machen!]}

\TALS{\newpage{}
  \subsubsection{Binome}

$$x^2+6x+9 = 20$$

\TNTeop{Binomische Formel (!):
(Hinweis: Beidseitig -9 ist nicht falsch, führt hier jedoch in eine Sackgasse)

  $$(x+3)^2 = 20$$

  Ziehe die Wurzel und beachte die 2. Lösung:

  $$x+3 = \pm\sqrt{20}$$

  -3:

  $$x = -3 \pm \sqrt{20}$$

  $$\lx = \{-3-\sqrt{20}; -3 + \sqrt{20}\}$$
}
\newpage

Selbständig:

$$x^2 -10x + 25 = 17$$
\TNT{7.2}{
$$(x-5)^2 = 17$$

$$\Longrightarrow$$

$$x-5 = \pm \sqrt{17}$$

$$\Longrightarrow$$

$$x=5\pm\sqrt{17}$$

$$\Longrightarrow  \lx = \{5-\sqrt{17}; 5+\sqrt{17}\}$$


}


$$x^2 -4x + 4 = 0$$
\TNTeop{
$$(x-2)^2 = 0$$

$$\Longrightarrow$$

$$x-2 = 0$$

$$\Longrightarrow$$

$$x= 2$$

$$\Longrightarrow  \lx = \{2\}$$


}

%%%%%%%%%%%%%%%%%%%%%%%%%%%%%%%%%%%%%%%%%%%%%%%%%%%

\subsection*{Aufgaben}

\olatLinkArbeitsblatt{Quadratische
    Gleichungen}{https://olat.bbw.ch/auth/RepositoryEntry/572162163/CourseNode/101586190756718}{weitere
    im Aufgabenblatt, auch weiter als Aufg. 7. möglich)}

\TRAINER{Lernzielkontrolle: Kahoot: Quadratische Gleichungen.}
\newpage
}%% END TALS 

\TALS{\newpage
%%
%% 2020 03 27 Ph. G. Freimann
%%
\subsection{Spezialfälle}\index{Spezialfälle!quadratische Gleichungen}

%%\TALSTadBFWA{95 ff}{2.3.3.1}

\subsubsection{$c=0$}
Wenn eine quadratische Gleichung der Form
$$ax^2 +bx = 0$$
gegeben ist, so kann man einfach ein $x$ ausklammern:

$$x(ax+b)=0$$
 Die Gleichung ist erfüllt, wenn nun entweder $x$ selbst oder aber
 der Klammerausdruck $(ax+b)$ Null werden. Somit haben wir sofort zwei
 Lösungen gefunden:
 $$\lx=\left\{0, \frac{-b}{a}\right\}$$

 %%\TALSAadBMTA{95ff}{265. a) b) c) e) g)}
 \newpage

 
 \subsubsection{$b=0$ (reinquadratische Gleichungen)}
\TadBMTA{166}{10.2.1}
 Ist die quadratische Gleichung in der Form
 $$ax^2 + c = 0$$
 gegeben, so gibt es ebenfalls eine einfache Lösungsformel. Es folgt:
 $$ax^2 = -c$$
 und daraus:
 $$x^2 = \frac{-c}{a}$$

 Die Lösungsmenge ist also schlicht
 $$\lx=\left\{ + \sqrt{\frac{-c}{a}}, -\sqrt{\frac{-c}{a}} \right\}.$$

 \hrule

 \subsubsection{Faktorisierte Form}

$$(x-7)(2x+8) = 0$$

\TNTeop{$\lx=\{-4; 7\}$, danach unbedingt Probe vorzeigen.\vspace{32mm}%%
}%% END TNTeop
%%\TALSAadBMTA{95ff}{266. b) c) d)}
\newpage
}%% END TALS


\newpage
\subsection{Motivation zur Lösungsformel}
Wie lautet die Lösungsmenge der folgenden Gleichung?

$$10x^2+11x = 6$$

\TNTeop{
  Dies können wir nicht ohne Trick oder allgemeines Vorgehen lösen.

  Trick

  $$10x^2+11x=6 \Longleftrightarrow   10x^2+11x-6=0
  \Longleftrightarrow (5x-2)(2x+3) = 0$$

  $$\lx = \left\{-\frac32;\frac25\right\}$$
}


\newpage
\TALS{
\subsubsection{Quadratische Ergänzung (optional)}\index{quadratische Ergänzung}\index{Ergänzung!quadratische}
Grundform\TRAINER{ (Normalform ist, wenn $a=1$)}:
$$2x^2 - 12x - 14 = 0$$

\TNTeop{
$:2$

$$x^2 -6x - 7 = 0$$

$+ 16$

$$x^2 -6x + 9 = 16$$

Binom

$$(x-3)^2 = 16$$

Wurzel ziehen (Achtung: zwei Lösungen sind möglich):

$$x-3 = \pm 4$$

$+3$

$$x = 3 \pm 4 \Longrightarrow  \lx = \{-1; 7\}$$

   Probe: $-1$ und $7$ einsetzen.
}%% END TNT eop

%%%%%%%%%%%%%%%%%%%%%%%%%%%%%%%%%%%

}

\newpage



\subsection{$a$-$b$-$c$ -- Formel}\index{abc-Formel!quadratische Gleichung}\index{Mitternachtsformel}
(Auch Mitternachtsformel)

%%Einführendes Youtube Video: \texttt{youtu.be/ZywdPuXR0S0}
\youtubeLink{https://youtu.be/ZywdPuXR0S0}{Dorfuchs: Quadratische Gleichung}


\begin{gesetz}{$a$-$b$-$c$ --Formel}{}
Ist eine quadratische Gleichung in Grundform ($a\ne 0$) gegeben,
$$ax^2 + bx + c = 0$$
so ist die Lösung durch die folgende Formel bestimmt:

$$x_{1,2}=\frac{-b\pm\sqrt{b^2-4ac}}{2a}$$
\end{gesetz}
\newpage


\subsubsection{Anwenden der $a$-$b$-$c$ $-$ Formel}
\begin{rezept}{$a$-$b$-$c$-Formel}{rezept_abc_formel}
  Beispiel $$3x^2 = 9x - 6$$

  1. in Grundform bringen: 

\TNT{2}{  $$3x^2 -9x + 6 = 0$$\vspace{11mm}}%% end tnt

2. $a$, $b$ und $c$ ermitteln:

\TNT{3.6}{%%
  
  \begin{tabular}{c|c|c}
  a&b&c\\
  \hline
  3 & -9 & 6\\
  \end{tabular}
  \vspace{15mm}
}%% END TNT

3. Einsetzen in \large{ $\frac{-b \pm \sqrt{b^2-4ac}}{2a}$}

\TNT{4.4}{  $$\frac{-(-9) \pm \sqrt{(-9)^2-4\cdot{}3\cdot{}6}}{2\cdot{}3} = \frac{+9 \pm \sqrt{81 - 72}}{6}$$

  $\lx = \{1; 2\}$
  \vspace{11mm}
}%% END TNT
\end{rezept}

\subsection*{Aufgaben}
Mit der $a$-$b$-$c$ $-$ Formel:
\GESOAadBMTA{181}{5. a) e), 6. d) und  7. a)  e)}
\newpage

\newpage
\subsubsection{Beweis der abc-Formel \GESO{(optional)}}\index{Mitternachtsformel!Beweis}\index{abc-Formel!Beweis}
\TRAINER{nach dorfuchs: [$-c$] [$\cdot{}4a$] [$+b^2$] [TU:
    bin. Formel] [$\pm\sqrt{\,\,}$] [$-b$] [$/2a$]}
\TRAINER{oder hier einfach das VIDEO von Dorfuchs zeigen; ca. 3
  min.}%% END TRAINER

\TNTeop{%%

\begin{tabular}{c|c|c}
Zahlenbeispiel                    & Vorgehen                     &  Allgemein           \\
\hline
$2x^2-14=4x$                      &                              & Grundform:        \\
                                  & in Grundform bringen         &                     \\
$2x^2-4x-14 = 0$                  &                              &  $ax^2 + bx + c =0$  \\
                                  & $| -c $ ($x$ separieren)     &                      \\
$2x^2-4x=14$                      &                              &  $ax^2 + bx = -c$    \\
                                  & $| \cdot{} 4a$               &                      \\
$16x^2 - 32x=112$                 &                              &  $4a^2x^2+4abx=-4ac$ \\
                                  & $| + b^2$ (Binom Vorbereiten)&                        \\
$16x^2 - 32x + 16 = 128$          &                              &  $4a^2x^2+4abx+b^2 = -4ac+b^2$ \\
                                  & Binomische Formel            &                           \\
$(4x-4)^2=128$                    &                              &  $(2ax+b)^2 = b^2-4ac$ \\
                                  & $\pm\sqrt{\,}$ Wurzel ziehen &                           \\
$4x-4 = \pm\sqrt{128}$            &                              &  $2ax+b=\pm \sqrt{b^2-4ac}$ \\
                                  & nochmals $x$ separieren      &                              \\
$4x   = 4 \pm \sqrt{128}$         &                              &  $2ax=-b\pm\sqrt{b^2-4ac}$ \\
                                  & durch $2a$ teilen            &                              \\
$x     = 1 \pm \frac14\sqrt{128}$ &                              &  $x=\frac{-b\pm\sqrt{b^2-4ac}}{2a}$ \\
\hline
\end{tabular}

}%% END TRAINER TNT end of page



\newpage
\subsubsection{$a$, $b$ und $c$ finden}
Betrachten wir das Beispiel in der Grundform:
$$x^2 - 15 = 0$$
Dies ist gleich

$$ {\color{ForestGreen}\LoesungsRaum{1}  \cdot{} }  x^2 +
{\color{blue} \LoesungsRaum{0}\cdot{}} x {
  \color{red} \LoesungsRaum{- 15}} = 0.$$
\TNTeop{
Somit ist
\begin{tabular}{|c|c|c|}
    {\color{ForestGreen}a} & {\color{blue}b} &  {\color{red}  c} \\\hline
    {\color{ForestGreen}1} & {\color{blue}0} &  {\color{red}-15}
\end{tabular}.

Lösung:
$$x_{1,2} = \frac{-{\color{blue}0} \pm \sqrt{{\color{blue}0}^2 -
    4\cdot{}{\color{ForestGreen}1}\cdot{}{\color{red}{(-15)}}}}{2\cdot{}{\color{ForestGreen}1}}
= \pm \frac{\sqrt{4\cdot{}15}}{2} \stackrel{\text{TR}}{=} \pm \sqrt{15}$$
}

%%%%%%%%%%%%%%%%%%%%%%%%%%%%%%%5

\textbf{2. Beispiel\GESO{ (optional)}}

Hier kommt alles zusammen:

$$x^2+3x = rx + 3r$$


\TNTeop{Grundform: $$x^2+3x-rx-3r=0$$
  $x$ ausklammern:
  $$x^2 + (3-r)x -3r = 0$$
  $a$, $b$ und $c$ finden:

\begin{tabular}{|c|c|c|}
    {\color{ForestGreen}a} & {\color{blue}b} &  {\color{red}  c} \\\hline
    {\color{ForestGreen}1} & {\color{blue}3-r} &  {\color{red}-3r}
\end{tabular}.

In Formel einsetzen

$$x_{1,2} = \frac{-(3-r)\pm\sqrt{(3-r)^2 -4\cdot{}1\cdot{}(-3r)}}{2}$$
vereinfachen:

$$x_{1,2} = \frac{-3+r\pm\sqrt{9-6r+r^2+12r}}{2}$$

$$x_{1,2} = \frac{-3+r\pm\sqrt{9+6r+r^2}}{2}$$

$$x_{1,2} = \frac{-3+r\pm\sqrt{(3+r)^2}}{2}$$

$$x_{1,2} = \frac{-3+r\pm(3+r)}{2}$$

$$x_1 = r \text{ und } x_2 = -3$$
$$\lx = \{-3, r\}$$
}

%%%%%%%%%%%%%%%%%%%%%

\subsection*{Aufgaben}

\GESO{\olatLinkArbeitsblatt{$a$-$b$-$c$-Finden}{https://olat.bbw.ch/auth/RepositoryEntry/572162163/CourseNode/101937272610102}{(Alle Übungen)}}%% END olatLinkArbeitsblatt
\TALS{\olatLinkArbeitsblatt{$a$-$b$-$c$-Finden}{https://olat.bbw.ch/auth/RepositoryEntry/572162090/CourseNode/101937272612767}{(Alle Übungen)}}%% END olatLinkArbeitsblatt


\GESOAadBMTA{182}{13. a) b)}
Wer fertig ist:\\


\GESO{\olatLinkArbeitsblatt{Quadratische
    Gleichungen}{https://olat.bbw.ch/auth/RepositoryEntry/572162163/CourseNode/101586190756718}{(Weiter
    ab derjenigen Nummer, wo Sie aufgehört hatten.)}}

\TALS{\olatLinkArbeitsblatt{Quadratische Gleichungen}{https://olat.bbw.ch/auth/RepositoryEntry/572162090/CourseNode/102901170063995}{(Weiter
    ab derjenigen Nummer, wo Sie aufgehört hatten.)}}


%%\TALSAadBFWA{95ff}{269. a), 275. a) b) d), 276. a) b) f)}%%
\TALSAadBMTA{181}{4. a) c) e), 5. a) c) e)}
\newpage

\TALS{%%
%% 2019 07 04 Ph. G. Freimann
%% Ergänzung für TALS zu Quadratischen Gleichungen I (Ohne Parameter)
%%

\newpage
\section{Quadratische Gleichungen II (mit Parametern)}\index{Gleichungen!quadratische mit Parametern}
\sectuntertitel{Welches ist denn nun die Gesuchte?}

\TadBMTA{172}{10.4}
%%%%%%%%%%%%%%%%%%%%%%%%%%%%%%%%%%%%%%%%%%%%%%%%%%%%%%%%%%%%%%%%%%%%%%%%%%%%%%%%%
\subsection*{Lernziele}

\begin{itemize}
\item Taschenrechner: Mit Parameter(n)
\item Taschenrechner: Visualisierung, Interpretation
\end{itemize}
\newpage

\subsection{Vorzeigeaufgabe}
Lösen Sie die folgende Gleichung ohne Berücksichtigung von
Spezialfällen, die durch die Wahl des Parameters $\lambda$ auftreten
könnten:

$$bx^2 + \sqrt{2} x = \lambda b x + \lambda \sqrt{2}$$

\TNTeop{
Alles nach links nehmen:
$$bx^2 + \sqrt{2}x - \lambda b x - \lambda \sqrt{2} = 0$$
$A$, $B$ und $C$ bestimmen, dazu zuerst $x$ ausklammern:
\\
$$b\cdot{}x^2 + (\sqrt{2} - \lambda b) \cdot x - \lambda \sqrt{2} = 0$$
$$\Longrightarrow  A = b; B = (\sqrt{2} - \lambda{} b); C = -\lambda \sqrt{2}$$
Einsetzen in $\frac{-B \pm \sqrt{B^2-4AC}}{2A} $:
$$x_{1,2} = \frac{\lambda b - \sqrt{2} \pm \sqrt{(2-2\sqrt{2}\lambda b) + \lambda^2 b^2 + 4b\lambda\sqrt{2}}}{2b}$$
$$  = \frac{\lambda b - \sqrt{2} \pm \sqrt{2+2\sqrt{2}\lambda b+ \lambda^2 b^2 }}{2b} $$
binomische Formel: 
  $$  = \frac{\lambda b - \sqrt{2} \pm \sqrt{(\sqrt{2} + \lambda b)^2 }}{2b} $$
  $$  = \frac{\lambda b - \sqrt{2} \pm (\sqrt{2} + \lambda b)}{2b}$$
  $$x_1 = \frac{-\sqrt{2}}{b}; x_2 = \frac{2\lambda b}{2b} = \lambda $$
}%% END TNT
%%\newpage

\subsection{Aufgaben}
\AadBMTA{182ff}{13. a) b)}%%\TALSAadBMTA{99}{ab 279}
\newpage

\newpage}

\subsubsection{Taschenrechner Tipps}

\TALS{Der CAS\footnote{CAS = Computer Algebra System}-Rechner kann
  quadratische Gleichungen einfach mit der \texttt{solve}-Funktion
  lösen:

  \texttt{solve(} $5x^2 - 6x = 3x -5$ \texttt{,} $x$ \texttt{)}

  Natürlich können Sie dies auch veranschaulichen:

  \texttt{f(x) := 5x${ }^2$ - 6x}  

  \texttt{g(x) := 3x - 5}

  Danach mit dem ``Graph'' Werkzeug beide Funktionen aufzeichnen.
}%% END TALS


\GESO{Was fällt bei folgender Gleichung auf?
$$72x^2 - 605 x + 848 = 0$$
}%% END GESO
  

\GESO{
  Der TI-30X Pro MathPrint kann quadratische Gleichungen mit Zahlen direkt auflösen.
  Doch dazu einige Tipps:
  
  \begin{tabular}{|c|p{13cm}|}
    \hline
    Tasten & Bemerkung \\
    \hline
    \tiprobutton{2nd}\tiprobutton{cos_poly-solv} & Starte das Lösen
    von quadratischen Gleichungen. Lösung obiger Gleichung: $\lx=\left\{\frac{53}8,\frac{16}9\right\}$\\
    \hline
    \tiprobutton{2nd}\tiprobutton{mode_quit}      & Beende den Poly-Solver.\\
    \hline
    \tiprobutton{neg} & Negative Zahlen nicht mit dem Subtraktionsoperator sondern mit der Vorzeichen-Taste eingeben.\\
    \hline
    \textbf{ACHTUNG} & Das Resultat lässt beim TI-30X Pro negative Wurzeln zu! Dies wird in der Lösung mit $i$ angegeben ($i$ steht für die imaginäre Einheit: $i=\sqrt{-1}$). Das $i$ kann aber je nach Modus erst gesehen werden, wenn mit der Pfeiltaste ganz nach rechts \textit{gescrollt} wurde.
    
    Steht ein $i$ in der Lösung des Taschenrechners, so gibt es keine reellen Lösungen und wir schreiben

    $\lx=\{\}$\\
    \hline
  \end{tabular}
  
  Lösen Sie mit dem Taschenrechner:
$$5x^2 - 7x + 2 = 0$$
\TNT{2}{$x_{1,2} = 1, 0.4$\vspace{16mm}}
$$x^2 +x  = -1$$
\TNT{2}{$x_{1,2} = \text{keine reelle Lsg. mit} \approx
  \text{verschwindet das }i \text{ aus dem Sichtfeld}$\vspace{16mm}}%% END TNT
}%% END GESO

\GESO{%%
\newpage
\begin{rezept}{}{}
Mit dem Taschenrechner reduzieren sich Aufgaben (ohne Parameter) auf
das \textbf{Umformen} einer quadratischen Gleichung in die \textbf{Grundform}.
\end{rezept}

  \subsection*{Aufgaben mit Taschenrechner}
  \GESOAadBMTA{182}{ 8. a) d) e) i), 9. c) und 10. b) e)}
}%% END GESO

\newpage


\subsection{Diskriminante}\index{Diskriminante}\label{diskriminante}
Die \textbf{Diskriminante} ist das, was in der $a$-$b$-$c-$Formel unter der Wurzel steht (= Radikand).

$D := b^2 - 4ac$

«diskriminieren = unterscheiden»

Mit der Diskriminante ist eine einfache \textbf{Fallunterscheidung} möglich:
\begin{itemize}
\item $D > 0 \Rightarrow $ zwei Lösungen:
  $$x_{1,2} = \frac{-b \pm \sqrt{D}}{2a} = \frac{-b \pm \sqrt{b^2 -
    4ac}}{2a}$$

\item $D = 0 \Rightarrow $ eine Lösung:
  $$ x = \frac{-b}{2a}$$

\item $D < 0 \Rightarrow $ keine Lösung in $\mathbb{R}$

\end{itemize}

\subsubsection{Anwendung}
Entscheiden Sie, wie viele Lösungen die gegebenen Gleichungen haben:

%%\renewcommand{\arraystretch}{2}
\begin{bbwFillInTabular}{c|c|p{5cm}}
  $x^2 + 3x +1 = 0$ & $b^2-4ac = \LoesungsRaumLang{9 -4 > 0}$ & \LoesungsRaumLang{zwei Lösungen} \noTRAINER{\phantom{xxxxxxxxxx}} \\
  \hline
  $x^2 + 2x +1 = 0$ & $b^2-4ac = \LoesungsRaumLang{4 -4 = 0}$ & \LoesungsRaumLang{eine Lösung} \\
  \hline
  $x^2 + 1x +1 = 0$ & $b^2-4ac = \LoesungsRaumLang{1 -4 < 0}$ & \LoesungsRaumLang{keine Lösungen} \\
\end{bbwFillInTabular}

\newpage

\subsection*{Aufgaben}
  Vorzeigeaufgabe: S. 183.: 16. b)%%
\AadBMTA{182ff}{11. c) f), und 16. \TALS{a) } c) \TALS{ d)}}

  
\olatLinkGESOKompendium{2.3.2.}{16}{46. bis 51.}

\GESO{und Textaufgaben:}

\olatLinkGESOKompendium{2.3.4.}{17ff}{55. bis 59.}

\newpage


%%
%% 2020 02 fp@bbw.ch
%%

%% Quadratische Gleichungen: Substitution
\subsection{Substitution}\index{Substitution!quadratische Gleichungen}

\TRAINER{
Zur Erinnerung beim Faktorisieren:\\
$a(2s-r) + b(2s-r) {\textrm{Subst. }T:=(2s-r) \over =} aT + bT = (a+b)T {\textrm{Rücksusbst. }2s-r=T\over = } (a+b)(2s-r)$
}

Oft können kompliziertere Gleichungen mittels einer geeigneten \textbf{Substitution} (= Ersetzung\footnote{lat. \textbf{substitu\=o} = an die Stelle \textit{jmds. o. einer Sache} setzen.
})
in eine quadratische oder lineare Gleichung verwandelt werden. Wie \zB hier:

\newcommand\tmpPart{\left(\frac{1}{3}(5x+0.5)\right)}

$$-6\tmpPart{} + 8.75 = -{\tmpPart{}}^2$$

%%
\TRAINER{«Mit Farben geht alles besser.»\\}%%
\TNT{2.4}{$$-6{\color{green}\tmpPart{}} + 8.75 = -{\color{green}{\tmpPart{}}^{\color{black}2}}$$}

\textbf{Substitution}

\TNT{1.6}{Wir ersetzne: $y := {\color{green}\tmpPart{}}$}

und die ursprüngliche Gleichung ist nun äquivalent zu
\TNT{1.6}{$$-6{\color{green}y} + 8.75 = -{\color{green}y}^2.$$}

Substituierte Gleichung lösen:


$$\mathbb{L}_y=\LoesungsRaumLang{\{2.5; 3.5\}}$$

\platzFuerBerechnungen{7.2}%% Bis Ende Seite nicht möglich wegen Fussnote

\TRAINER{
  Erst in Grundform bringen $$y^2 - 6y + 8.75 = 0$$
  und danach mit der $a$-$b$-$c$ -- Formel auflösen ($a=1, b=-6, c=8.75$).
}%%

\TRAINER{\textbf{Nomenklatur:}\\
  Substitutionsbasis: $-6\tmpPart{} + 8.75 = -\tmpPart{}^2$\\
  Substituendum: $\tmpPart{}$\\
  Substituens: $y$
}%%

\newpage


\textbf{Rücksubstitution}\index{Rücksubstitution}\\
Nun setzen wir die Lösungen anstelle der substituierten Variable ein:


\TNTeop{
$$1.: y_1 = 2.5 = \tmpPart{} \Rightarrow 7.5 = 5x+0.5 \Rightarrow x = \frac{7}{5} = 1.4$$

$$2.: y_2 = 3.5 = \tmpPart{} \Rightarrow 10.5=5x+0.5 \Rightarrow x = 2$$

$$\lx=\{1.4; 2\}$$}%% END TNTeop

%%%%%%%%%%%%%%%%%%%%%%%%%%%%%%%%%%%55555


\subsection*{Aufgaben}

\GESO{
  \subsubsection{Nullserie}
  Die folgende Aufgabe stammt aus der \textit{Nullserie}-Maturaprüfung:

  Lösen Sie mit Hilfe einer geeigneten Substitution \TRAINER{Tipp: $Z^{10} = \left(Z^5\right)^2$ }:
  $$\left(\frac{x}{3}-3.5 \right)^{10} -2\left( \frac{x}{3} - 3.5 \right)^5 =-1$$
  
  \TNTeop{Substituiere $z = \left(\frac{x}{3} - 3.5\right)^5$. Somit lautet die Gleichung
$$z^2 - 2z +1 = 0$$
    Was uns zu $z = 1$ bringt. Damit ist $1 = \left(\frac{x}{3} - 3.5 \right)^5$ und somit auch $1 = \frac{x}{3} - 3.5$. Auf beiden Seiten 3.5 addieren und danach mit 3 multiplizieren liefert $x = 13.5$.
 }%% End TNT

  %%%%%%%%%%%%%%%%%%%%%%%%%%%%%%%%%%%%
  
\aufgabenFarbe{Berufsmaturitätsprüfung 2020: Aufgabe 4 von Serie 2:
  $$(x-5)^4 - \frac{(x-5)^2}{3} = 8$$}%% END Aufgabenfarbe
\TNT{8}{Lösung der Substituierten $y=(x-5)^2)$ ist $y=3$ bzw. $y=-\frac83$. Die zweite Lösung kommt jedoch nur als Scheinlösung vor.\vspace{45mm}}%% END TNT
    $$\lx = \LoesungsRaumLang{\left\{5-\sqrt{3}; 5+\sqrt{3}\right\}}$$
  
}%% END GESO

\AadBMTA{184}{28. a) d) 29. a) c) 30. a) b)}
%%\TALSAadBMTA{99ff (Substitution)}{283. a) b) f) 282. a) f) 284. a)}
\newpage


\newpage


\subsection{Welches Verfahren (optional)}
Wann soll welches Verfahren eingesetzt werden? Dies ist eine
individuelle Fragestellung. Je mehr Erfahrung man hat, umso eher sieht
man die Spezialfälle. Die $a$-$b$-$c$-Formel funktioniert immer, doch
die anderen Verfahren (Binome, weitere Spezialfälle, Taschenrechner)
sind oft sinnvoller. Hier ein Versuch eines Überblicks:

\GESO{
\begin{tabular}{|p{44mm}|p{53mm}|p{64mm}|}
	\hline
	Spezialfall $b=0$               & $5x^2 = 3$                   & Durch 5 dividieren, dann positive und negative Wurzel ziehen.\\
	\hline
	Spezialfall $c=0$               & $3x^2 = 5x$                   & Fallunterscheidung $x$ = 0 und $x \ne 0$\\
	\hline
	Bereits faktorisiert       & $(x-4.6 + b)\cdot{}(x+a) = 0$ & Lösungen hier $x_1=4.6-b$ und $x_2 = -a$\\
	\hline
	Einfache Zahlen            & $x^2 -2x + 1= 0$           & $a$-$b$-$c$--Formel (Mitternachtsformel) oder faktorisieren $x^2-2x+1=(x-1)^2$ und danach jeden einzelnen Faktor $=0$ setzen.\\
	\hline
	Komplizierte Zahlen        & $7.3x^2 - 8x - 3.4 = 0$       & Taschenrechner \tiprobutton{cos_poly-solv}             \\
	\hline
	Variable (Parameter)       & $7.3x^2 - cx + 2.6=0$         & $a$-$b$-$c$-Formel (Mitternachtsformel) \\
	\hline
	Komplexe Terme im Quadrat  & $3(x-6)^2 - 17(x-6)  = 0$     & Substitution                            \\
	\hline
\end{tabular}
\newpage
}

\TALS{
\begin{tabular}{|p{49mm}|p{53mm}|p{60mm}|}
	\hline
	Spezialfall $b=0$               & $5x^2 = 3$                   & Durch 5 dividieren, dann positive und negative Wurzel ziehen.\\
	\hline
	Spezialfall $c=0$               & $3x^2 = 5x$                   & Fallunterscheidung $x$ = 0 und $x \ne 0$\\
  \hline
	Bereits faktorisiert & $(x-4.6 + b)\cdot{}(x+a) = 0$ & Lösungen hier $x_1=4.6-b$ und $x_2 = -a$\\
	\hline
	Einfache Zahlen      & $6x^2 -3x - 18 = 0$               & $a$-$b$-$c$-Formel (Mitternachtsformel) \\
	\hline
	Komplizierte Zahlen  & $7.3x^2 - 8x - 3.4 = 0$           & Taschenrechner (solve())                \\
	\hline
	Variable             & $7.3x^2 - cx + 2.6=0$         & Taschenrechner                          \\
	\hline
	Gesucht Anzahl Lösungen (auf 1)
           beschränken & $7.3x^2 - cx + 2.6 = 	0$     & 	Diskriminante mit Taschenrechner Null setzen.\\
	\hline
\end{tabular} 
}
\newpage
