%%
%% 2019 07 04 Ph. G. Freimann
%%
\section{Bruchgleichungen I, ...}\index{Gleichungen!Bruchgleichungen}\index{Bruchgleichungen}

\subsection{... lineare}

\sectuntertitel{125\% der Leute können nicht Bruchrechnen. Das ist
jeder vierte, nein mehr noch: Jeder fünfte!}



\theorieGESO{118}{8.3}

\theorieGESO{121}{8.4}
\theorieTALS{105}{2.4.1}
%%%%%%%%%%%%%%%%%%%%%%%%%%%%%%%%%%%%%%%%%%%%%%%%%%%%%%%%%%%%%%%%%%%%%%%%%%%%%%%%%
\subsection*{Lernziele}

\begin{itemize}
	\item ggT\index{ggT}, kgV\index{kgV}
	\item gleichnamig machen
	\item Hauptnenner\index{Hauptnenner}
\end{itemize}


\begin{definition}{Bruchgleichung}{definition_bruchgleichung}
  Unter einer Bruchgleichung verstehen wir eine Gleichung, bei der die
  gesuchte Variable (mindestens einmal) im Nenner vorkommt.
\end{definition}

\begin{beispiel}{Bruchgleichung}{beispiel_beispiel_einer_bruchgleichung}
$$\frac{1+x}{x}=\frac{3+x}{2x}$$
\end{beispiel}
\newpage


Vorzeigebeispiel:

\TNT{16.4}{
$$\frac{1+x}{x}=\frac{3+x}{2x}$$
Der Definitionsbereich enthält die Null (0) nicht:
$$\mathbb{D} = \mathbb{R}\backslash \{0\}$$

Wir multiplizieren beide Seiten mit $x$. Dies ist hier möglich, denn das Multiplizieren mit Null wäre keine Äquivalenzumformung; doch hier ist Null explizit vom Definitionsbereich ausgeschlossen.

$$1+x=\frac{3+x}{2}$$

Das Auflösen der Gleichung liefert:

$$x=1$$

Das Multiplizieren beider Seiten mit der unbekannten Größe ist heikel. Es können Scheinlösungen hinzukommen. Daher muss am Schluss die Probe gemacht werden:

$$\frac{1+1}{1}=\frac{3+1}{2\cdot{}1}$$

Die Probe passt. Ergo:
$$\mathbb{L}_x = \{1\}$$

}
\newpage

  
  \subsection{Grundmenge, Definitionsmenge}\index{Grundmenge}\index{Definitionsmenge}
  Die Menge aller Zahlen, welche für die Lösung in Frage kommen,
  nennen wir die Grundmenge oder Definitionsmenge. Diese ist üblicherweise $\mathbb{R}$, die
  Menge der reellen Zahlen und wir schreiben:
  $$\mathbb{G}=\mathbb{R}$$
  Teilweise sind als Lösungen auch nur positive ($\mathbb{G}=\mathbb{R}^+$) oder
  beispielsweise nur ganze Zahlen ($\mathbb{G}=\mathbb{Z}$)
  zugelassen.
  Insbesondere ist die Menge der Lösungen jedoch eingeschränkt durch
  die Terme, welche die Gleichung definieren. So besteht die folgende
  Gleichung aus zwei Termen ($T1=\frac{4}{x-2}$ und $T2=\frac{\sqrt{x}}{x-3}$) mit den unten angegebenen
  Einschränkungen:
  $$\frac{4}{x-2}=\frac{\sqrt{x}}{x-3}$$
  Die Definitionsmengen von $T1$ und $T2$ schränken automatisch die
  Grundmenge der Gleichung ein. In obigem Beispiel gilt:
  
  $$\mathbb{D}_1=\mathbb{D}(T1)=\LoesungsRaum{\mathbb{R}\backslash\{2\}}$$

  $$\mathbb{D}_2=\mathbb{D}(T2)=\LoesungsRaum{\mathbb{R}^+_0\backslash\{3\}}$$

  $$\mathbb{D}=\mathbb{D}_1\cap\mathbb{D}_2=\LoesungsRaumLang{\mathbb{R}^{+}_{0}\backslash\{2;3\}}$$

  \newpage
  
\begin{rezept}{Bruchgleichungen}{}
  \begin{itemize}
    \item Definitionsmenge $\mathbb{D}$ festlegen: Division durch null ausschließen
  \item Einzelbrüche kürzen: Dazu Zähler und insbesondere Nenner \textit{faktorisieren}
  \item Brüche \textit{wegschaffen}: Alle Brüche auf
    \textit{Haupnenner} (kgV) erweitern.
  \item Beidseitig die Gleichung mit dem Hauptnenner
    multiplizieren. Nenner wegkürzen.
  \item Lineare Gleichung lösen und provisorische Lösungsmenge $\mathbb{L}$
    bestimmen
  \item Definitionsmenge $\mathbb{D}$ mit Lösungsmenge $\mathbb{L}$
    vergleichen: Scheinlösungen müssen ausgeschlossen werden.
  \item Definitive Lösungsmenge $\mathbb{L}$ angeben
    \end{itemize}
  Quelle: \cite{marthaler17} Seite 120
\end{rezept}

\GESO{\subsection*{Aufgaben}}
\GESOAadB{129}{14. a) e) g) 15. a) c) e) 16. c) e) 20. a) b) d)}
\newpage
