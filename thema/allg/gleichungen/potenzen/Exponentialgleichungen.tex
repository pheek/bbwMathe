%%
%% 2019 07 04 Ph. G. Freimann
%%

\section{Exponentialgleichungen}\index{Gleichungen!Exponentialgleichungen}
\sectuntertitel{Exponenten? Und warum haben sie aufgehört Ponenten zu sein?}


%%\TALS{Theorie: (\cite{frommenwiler17alg} S.117 (Kap. 2.4.4))}

\TALS{\bbwCenterGraphic{12cm}{allg/gleichungen/img/Ritter.jpg}}
\GESO{\bbwCenterGraphic{13cm}{allg/gleichungen/img/Ritter.jpg}}


\newpage
\TadBMTA{199}{12}
%%%%%%%%%%%%%%%%%%%%%%%%%%%%%%%%%%%%%%%%%%%%%%%%%%%%%%%%%%%%%%%%%%%%%%%%%%%%%%%%%
\subsection*{Lernziele}

\begin{itemize}
\item Exponentialgleichungen
\TALS{\item logarithmische Gleichungen (\cite{marthaler21alg} S. 203 Kap. 12.2)}
\TALS{\item Exponentialgleichungen mit Parametern}
\end{itemize}

\GESO{\matheNinjaLink{Exponentialgleichungen}{https://olat.bbw.ch/auth/RepositoryEntry/572162163/CourseNode/106029166783298}}
\TALS{\matheNinjaLink{Exponentialgleichungen}{https://olat.bbw.ch/auth/RepositoryEntry/572162090/CourseNode/106261459129069}}

\newpage

\subsection{Exponentialgleichungen\TALS{...}}\index{Exponentialgleichungen}
\begin{definition}{Exponentialgleichung}{}
  Bei einer \textbf{Exponentialgleichung} kommt die gesuchte Größe im
  Exponenten (von Potenzen) vor.
\end{definition}

Beispiel:

$$5^{x+1} = 34$$

Typischerweise werden diese Gleichungen gelöst, indem die Definition
des Logarithmus angewendet wird.


\vspace{4mm}
\textbf{Typ Ia:} Exponentenvergleich\\

\begin{rezept}{Exponentenvergleich}{}
Bei Exponentialgleichungen der Form $$5^{x+1} =
5^{2x-1}\TRAINER{\hspace{30mm}| \log_5(\Box{})}$$ können bei
gleicher Basis einfach die Exponenten verglichen werden:

$$\LoesungsRaumLang{\Longleftrightarrow x+1=2x-1}$$

\TNT{2.4}{$1=x-1$\\
  $2=x$
\vspace{12mm}}

Es entsteht eine lineare Gleichung mit der Lösung $\LoesungsRaum{\lx = \{2\}}$
\end{rezept}

\begin{bemerkung}{Voraussetzungen}{}
  \begin{itemize}
  \item auf beiden Seiten der Gleichung steht eine Potenz
  \item beide Potenzen haben dieselbe Basis
    \end{itemize}
\end{bemerkung}
\newpage


\subsection*{Aufgaben}

\GESO{\olatLinkArbeitsblatt{Exponentialgleichungen
  [GL\_Ex]}{https://olat.bbw.ch/auth/RepositoryEntry/572162163/CourseNode/106029166708108}{1.}}
\TALS{\olatLinkArbeitsblatt{Exponentialgleichungen
  [GL\_Ex]}{https://olat.bbw.ch/auth/RepositoryEntry/572162090/CourseNode/106029166654550}{1.}}
\newpage
\textbf{Typ II a:} «Zahl hoch $x$ gleich Zahl»


\begin{rezept}{Logarithmieren}{}
$$5^x=32 \Longleftrightarrow x=\LoesungsRaum{\log_5(32)} \Longrightarrow x \approx \LoesungsRaum{2.15338}$$

Die Lösung kann dann auch mit Logarithmen zur Basis 10 angegeben
werden:
$$\LoesungsRaumLang{x = \log_5(32) = \frac{\lg(32)}{\lg(5)}}$$
\end{rezept}


\subsection*{Aufgaben}

\GESO{\olatLinkArbeitsblatt{Exponentialgleichungen
  [GL\_Ex]}{https://olat.bbw.ch/auth/RepositoryEntry/572162163/CourseNode/106029166708108}{2. a)
bis g) }}
\TALS{\olatLinkArbeitsblatt{Exponentialgleichungen
  [GL\_Ex]}{https://olat.bbw.ch/auth/RepositoryEntry/572162090/CourseNode/106029166654550}{2. a)
bis g) }}


\newpage

\textbf{Typ Ib:} Typischer Fall\\
\TRAINER{Der folgende Typ tritt insb. bei Exponentialfunktionen auf.}

Lösen Sie und kontrollieren Sie mit dem Taschenrechner.
$$6 = 14 - 3\cdot{} 2^{\frac{x}4}$$

\TRAINER{Bemerkung: Erst TR-Lösung, dann Konntrolle von Hand! Dies ist
die richtige Reihenfolge!}

\TNT{5.2}{
$$-8 =  - 3\cdot{} 2^{\frac{x}4}$$
$$ 8 =    3\cdot{} 2^{\frac{x}4}$$
$$ \frac83 = 2^{\frac{x}4}$$

  So haben wir wieder den Fall «Zahl = Zahl hoch Gesuchtes».

  $$\log_2(\frac83) = \frac{x}4$$

  $$x = 4\cdot{} \log_2(\frac83) \approx 5.660$$
}
\subsection*{Aufgaben}
\GESO{\olatLinkArbeitsblatt{Exponentialgleichungen
  [GL\_Ex]}{https://olat.bbw.ch/auth/RepositoryEntry/572162163/CourseNode/106029166708108}{2. h) bis j) }}
\TALS{\olatLinkArbeitsblatt{Exponentialgleichungen
  [GL\_Ex]}{https://olat.bbw.ch/auth/RepositoryEntry/572162090/CourseNode/106029166654550}{2. h) bis j) }}


\newpage

\textbf{Typ III:} Allgemeiner Fall\\

\begin{rezept}{Exponentialgleichung lösen}{rezept_allgemeine_exponentialgleichung}
$$8^{x-1} = 5\cdot{}7^{x+2} + 4\cdot{}7^{x-3}$$
\end{rezept}

Siehe auch \cite{marthaler21alg} Seite 200 im roten Kasten.

\TNTeop{
  1. Zuerst die Summen in den Exponenten wegbringen:
      $$8^x\cdot{}8^{-1} = 5\cdot{}7^x\cdot{}7^2 + 4\cdot{}7^x\cdot{}7^{-3}$$

  2. $x$-Potenzen ausklammern. Links und rechts des
  Gleichheitszeichens stehen nur noch Produkte, Quotienten oder
  Potenzen:
       $$8^x\cdot{}8^{-1} = 7^x\cdot{}(5\cdot{}7^2 + 4\cdot{}7^{-3})$$

      3. Jetzt alle Potenzen mit $x$ auf eine Seite bringen (den Rest
      nach rechts):
      $$8^x : 7^x = (5\cdot{}7^2 + 4\cdot{}7^{-3}) : 8^{-1}$$

      4.Vereinfachen und links als Potenz in $x$ schreiben:
      $$\left(\frac87\right)^x = (...) : 8^{-1}$$
      $$\left(\frac87\right)^x = (...) \cdot{} 8$$

      (dies ist wieder Typ II:)\\
      5. Definition Logarithmus anwenden ($a^x=b \Leftrightarrow x=\log_a(b)$):
      $$x = \log_{\left(\frac87\right)}\left((5\cdot{}7^2 + 4\cdot{}7^{-3})\cdot{}8\right)$$
$$x \approx 56.77$$
      (6. optional) Dies kann (falls gefragt) mit Zehnerlogarithmen geschrieben
      werden:
      $$x = \frac{\lg\left((5\cdot{}7^2 + 4\cdot{}7^{-3})\cdot{}8\right)}{\lg\left(\frac87\right)} \approx
      56.77$$
}%% END TNT

%%%%%%%%%%%%%%%%%%%%%%%%%%%%%%%%%%%%%%%%%%%%%%%%%%%%%%%%%%%%%%%%%%%%%%%%%%%

\TALS{%% Methode mit Logarithmieren von Anfang an

\begin{rezept}{... durch direktes Logarithmieren.}{}
$$8^{x-1}=5\cdot{}7^{x+2}$$
\end{rezept}

  \TNTeop{
  Erst mal auf beiden Seiten logarithmieren, mit einem Logarithmus zu beliebiger Basis:
  $$\log(8^{x-1}) = \log(5\cdot{}7^{x+2})$$
  Log Gesetz (Produktregel):
  $$\log(8^{x-1}) =\log(5)+ \log(7^{x+2})$$
  Log Gesetz (Potenzregel)\TRAINER{\footnote{Ab hier ist es eine
      lineare Gleichung}}:
  $$(x-1)\cdot{}\log(8) = \log(5) + (x+2)\cdot{}\log(7)$$
  ausmultiplizieren
  $$\log(8)\cdot{}x-\log(8) = \log(5) + \log(7)\cdot{}x+2\cdot{}\log(7)$$
  und $x$ auf eine Seite bringen
  $$\log(8)\cdot{}x-\log(7)\cdot{}x =\log(5)+ 2\cdot{}\log(7)+\log(8)$$
  $x$ ausklammern
  $$x\cdot{}(\log(8)-\log(7)) = \log(5) + 2\cdot{}\log(7)+\log(8)$$
  und durch die Klammer $(\log(8)-\log(7))$ teilen
  $$x = \frac{\log(5) + 2\cdot{}\log(7)+\log(8)}{\log(8)-\log(7)} \approx 56.77$$
}%% END TNTeop
%%\newpage
}%% END TALS


\GESO{\newpage}

\subsection*{Aufgaben}

 Logarithmieren Sie (Vorzeigeaufgabe) ...
 
$$3.4^{\frac{x}2} = 16 \Longrightarrow x = \LoesungsRaum{\frac{2\cdot{}\lg(16)}{\lg(3.4)}}$$
 
 \TNT{4}{%%
   $$\log_{3.4}(16)  = \frac{x}2$$
   $$2 \cdot{} \log_{3.4}(16)  = x$$
   $$ x = 2 \cdot{} \log_{3.4}(16) = 2\cdot{} \frac{\lg(16)}{\lg(3.4)}$$
 }%% end TNT

 ... selbständig:

$$6^{\frac{x}3} = 14 \Longrightarrow x = \LoesungsRaum{\frac{3\cdot{}\lg(14)}{\lg(6)}}$$
 
 \TNT{4}{
   $$\log_{6}(14) = \frac{x}3$$
   $$x = 3\cdot{} \log_{6}(14) = 3\frac{\lg(14)}{\lg(6)} = \frac{3\cdot{}\lg(14)}{\lg(6)}$$
 }%% end TNT

$$1.8^{\frac{x}5} = 100 \Longrightarrow x = \LoesungsRaum{\frac{10}{\lg(1.8)}}$$
 
 \TNTeop{
   $$\log_{1.8}(100) = \frac{x}5$$
   $$x = 5\cdot{} \log_{1.8}(100) = 5\frac{\lg(100)}{\lg(1.8)} =
   \frac{5\cdot{}\lg(10^2)}{\lg(1.8) = \frac{5\cdot{}2}{\lg(1.8)} = \frac{10}{\lg(1.8)}}$$
 }%% end TNT

\GESO{\olatLinkArbeitsblatt{Exponentialgleichungen
  [GL\_Ex]}{https://olat.bbw.ch/auth/RepositoryEntry/572162163/CourseNode/106029166708108}{3.}}
\TALS{\olatLinkArbeitsblatt{Exponentialgleichungen
  [GL\_Ex]}{https://olat.bbw.ch/auth/RepositoryEntry/572162090/CourseNode/106029166654550}{3. bis  5. }}


Textaufgaben:

\GESO{\olatLinkArbeitsblatt{Exponentialgleichungen
    [GL\_Ex]}{https://olat.bbw.ch/auth/RepositoryEntry/572162163/CourseNode/106029166708108}{Kap. 6: Aufg. 6. - 10.}}
\TALS{\olatLinkArbeitsblatt{Exponentialgleichungen
    [GL\_Ex]}{https://olat.bbw.ch/auth/RepositoryEntry/572162090/CourseNode/106029166654550}{Kap. 6: Aufg. 6. - 10.}}

\olatLinkGESOKompendium{2.5}{19}{66. bis 73.}

%%\TALSAadBMTA{118}{359. a) d), 360. a) d), 361. a) d), 362. a) b) c), 363. a) b) d)}
%%\GESOAadBMTA{71 (Exponentenvergleich)}{44}
%%\GESOAadBMTA{206}{2. a) d) g) h), 3. a) b) c) d) f), 4. a) d) e) g) h),
%%  9. und 11.}
