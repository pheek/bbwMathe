%%
%% 2020 05 07 Ph. G. Freimann
%%

%% Überblick über die Begriffe
%% Potenz, Potenzgleichung, Exponentialgleichung, Wurzelgleichung

\subsection{Überblick über Gleichungen mit Potenzen}

\begin{bbwFillInTabular}{|p{52mm}|p{52mm}|p{52mm}|}
  \multicolumn{3}{c}{Beispiel 1: Zehnerpotenzen}\\
%%  \multicolumn{3}{c}{\,}\\ %% Generiere etwas Abstand

  \hline
  $10^3=x$                 & $x^3=1000$                           &  $10^x=1000$               \\
  \hline
  $x=\TRAINER{1000}$  & $ x=\TRAINER{\sqrt[3]{1000}=10}$          & $x   =  \TRAINER{\log_{10}(1000) = 3}$     \\\hline
%%  \multicolumn{3}{c}{\,}\\ %% Generiere etwas Abstand
  \multicolumn{3}{c}{Beispiel 2: Zweierpotenzen}\\
  \hline
  $2^5=x$                  & $x^5=32$                             &  $2^x=32$                  \\
  \hline
  $x=\TRAINER{2\cdot{}2\cdot{}2\cdot{}2\cdot{2}=32}$ & $x=\TRAINER{\sqrt[5]{32}=2}$        & $x = \TRAINER{\log_{2}(32)=3}$        \\\hline

    \multicolumn{3}{c}{Begriffe}\\\hline

  Potenzwert               & Basis gesucht                       &  Exponent gesucht      \\
  (berechnen)              & \TRAINER{Potenzgleichung}           &  \TRAINER{Exponentialgleichung}       \\
  \hline

%%  \multicolumn{3}{c}{\,}\\ %% Generiere etwas Abstand
  \multicolumn{3}{c}{\GESO{(optional)}\TALS{Erinnerung} --- Wurzeln sind rationale Exponenten:}\\
  \hline
  $\sqrt[3]{1000}=x$       & $\sqrt[3]{x}=10$                    &  $\sqrt[x]{1000}=10$               \\
  \hline
  $1000^{\frac{1}{3}}=x$     & $x^{\frac{1}{3}}=10$                   &  $1000^{\frac{1}{x}}=10$               \\
  $x=\TRAINER{\sqrt[3]{1000}=10}$       & \noTRAINER{\hspace{15mm}}$\TRAINER{\left(x^\frac13 \right)^3}=\TRAINER{10^3}$  & $\TRAINER{\frac{1}{x} }\noTRAINER{\hspace{2cm}}=\TRAINER{  \log_{1000}(10)}$      \\
                           &   $\TRAINER{x}\noTRAINER{\hspace{15mm}}=\TRAINER{1000}$                                  & $\TRAINER{\frac{1}{x}}\noTRAINER{\hspace{2cm}} = \TRAINER{ \frac{1}{3}}$         \\
                           &                                      & $x = \TRAINER{3}$                      \\\hline
\end{bbwFillInTabular}

\TALS{
  \newpage
  \subsubsection{Spezialfall «logarithmische Gleichungen»}
  Beispiel: $\log_2(x-7) = 5$

    Meist können logarithmische Gleichungen (wie obige) in die
    Potenzschreibeweise umgeschrieben werden und sind dann einfacher
    lösbar:

    \TNTeop{$$\log_2(x-7) = 5 \Longleftrightarrow 2^5 = x-7
      \Longleftrightarrow x=39$$}%% END TNTeop
}%% END TALS

\GESO{Bem. zum Logarithmus zur Basis 2 ($\log_{2}(32)$): Dazu müssen
  Sie die

  \tiprobutton{ln_log}-Taste
  auf Ihrem TI-30 Taschenrechner 3x drücken.
}%% END GESO
\newpage
