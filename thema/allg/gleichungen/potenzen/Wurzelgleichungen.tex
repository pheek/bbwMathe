
\subsection{Wurzelgleichungen}\index{Wurzelgleichung}

\TadBMTA{191}{11}

\begin{definition}{Wurzelgleichung}{}
Kommt in der Gleichung die Gesuchte unter der Wurzel vor, so sprechen
wir von einer \textbf{Wurzelgleichung}.
\end{definition}
Einführungsbeispiel:
$$\sqrt[5]{x}=6$$


Dies löst man indem man beide Seiten der Gleichung potenziert:
\TNT{4}{
  
\begin{tabular}{rccl}
  \             & $\sqrt[5]{x}$   &=&     6      \\
  $\Rightarrow$ & $(\sqrt[5]x)^5$ &=&  $6^5$     \\
  $\Rightarrow$ & $x$             &=& $7\,776$ 
\end{tabular}
}%% END TNT

Lösen Sie das folgende Musterbeispiel:
$$\sqrt{x}+1=2x$$
\TRAINER{Das obige Beispiel von M. Rohner zeigt viele Sackgassen und Dinge, auf die man noch achten muss.}

\GESO{Lösungsverfahren im Buch \cite{marthaler21alg} Seite 192 im roten
  Kasten.}

\platzFuerBerechnungenBisEndeSeite{}

\TRAINER{1. Wurzel separieren
  $$\sqrt{x}=2x-1$$ quadrieren:
  $$x=4x^2-4x+1$$
  $$0=4x^2-5x+1$$
  $$x_{1,2}=\frac{+5 \pm \sqrt{25-16}}{8}$$
  $$\lx={1}$$, denn $\frac{1}{4}$ ist eine durchs Quadrieren erschienene Scheinlösung.}

\newpage
\begin{bemerkung}{Scheinlösung}{}
  
\textbf{Achtung}: Beim \textbf{Radizieren} auf beiden Seiten
einer Gleichung können \textbf{Lösungen} \textbf{verschwinden}.

Beim \textbf{Potenzieren} können (Schein)\textbf{Lösungen}
\textbf{hinzukommen} \GESO{Bsp. S.191 und roter Kasten S. 192
  im \cite{marthaler21alg}}.
\end{bemerkung}


\GESO{optional:}
\begin{rezept}{Wurzelgleichung lösen}{}
  \begin{enumerate}
  \item Definitionsbereich $\mathcal{D}$ festlegen
  \item Eine Wurzel isolieren (separieren)
  \item Quadrieren
  \item 2. und 3. Schritt wiederholen, bis keine Wurzeln mehr
    vorkommen
  \item Auflösen
  \item Probe:
    \begin{enumerate}
    \item Mit Definitionsbereich $\mathcal{D}$ abgleichen
    \item In ursprüngliche Gleichung einsetzen
    \end{enumerate}
  \end{enumerate}
\end{rezept}
\newpage


Beispiel: $2\cdot{}\sqrt{x+1} + 2 = 5 + 2\cdot{}\sqrt{x-2}$

\TNT{16}{\bbwCenterGraphic{16cm}{allg/gleichungen/img/WurzelgleichungLoesungsweg.png}
  \vspace{3cm}
  $$\lx = \left\{\frac{33}{16}\right\}$$}

\subsection*{Aufgaben}
%%  \TALSAadBMTA{112ff}{339. d) e), 340. c), 342. d), 344. a) b) c) [Tipp
%%      bei c): Substitution]}

\AadBMTA{
  196 (Wurzelgleichungen)}{2., 3. a)-f),  4. a) c) d) f),
  5. a) c), 6. b),  7. d), 8. c) d) und Textaufgabe 10. a) und b)}

\newpage
