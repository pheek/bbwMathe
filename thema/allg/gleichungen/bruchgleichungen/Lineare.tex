%%
%% 2019 07 04 Ph. G. Freimann
%%

\subsection{... lineare ...}

\sectuntertitel{125\% der Leute können nicht Bruchrechnen. Das ist
jeder vierte, nein mehr noch: Jeder fünfte!}



\TadBMTA{118}{8.3}

\TALSTadBMTA{121}{8.4}

%%\TALSTadBFWA{105}{2.4.1}
%%%%%%%%%%%%%%%%%%%%%%%%%%%%%%%%%%%%%%%%%%%%%%%%%%%%%%%%%%%%%%%%%%%%%%%%%%%%%%%%%
\subsection*{Lernziele}

\begin{itemize}
	\item gleichnamig machen
  \item Definitionsmenge
\end{itemize}

\begin{definition}{Bruchgleichung}{definition_bruchgleichung}\index{Bruchgleichungen}
  Unter einer Bruchgleichung verstehen wir eine Gleichung, bei der die
  gesuchte Variable (mindestens einmal) \textbf{im Nenner} vorkommt.
\end{definition}

\begin{beispiel}{Bruchgleichung}{beispiel_beispiel_einer_bruchgleichung}
$$\frac{1+x}{x}=\frac{x+3}{x-3}$$
\end{beispiel}

\begin{bemerkung}{\textbf{Keine} Bruchgleichung!}{}
  $$\frac{1+x}7=\frac{x+3}{\sqrt{2}}$$
  \TNT{2}{$$\frac17 + \frac17x = \frac1{\sqrt{2}}x+\frac3{\sqrt{2}}$$}
  \end{bemerkung}
\newpage
\subsubsection{Referenzaufgabe}

$$\frac{1+x}{x}=\frac{x+3}{x-3}$$
  
\TNTeop{
  1. Definitionsmenge ist $\mathcal{D}_x = \mathbb{R}\backslash\{0, 3\}$, denn sowohl $0$, wie auch $3$ dürfen nicht
  für $x$ eingesetzt werden.

  2. Nun alles auf eine Seite bringen:

  $$\frac{1+x}x - \frac{x+3}{3-x} = 0$$

  3. Bruchterm vereinfachen \ifisALLINONE{(Vgl. \totalref{bruchterme})}\fi

  $$\frac{(1+x)(x-3)}{x(x-3)} - \frac{x(x+3)}{x(x-3)} = 0$$

  $$\frac{(1+x)(x-3) - x(x+3)}{x(x-3)} = 0$$

  $$\frac{-5x-3}{x(x-3)} = 0$$

  4. Wir betrachten nur den Zähler!

  $$-5x-3 \stackrel{!}{=} 0$$

  und somit:

  $$x = \frac{-3}5$$

  Probe a) stimmt die Lösung in der ursprünglichen Gleichung? (ja)\\

  Probe b) liegt die Lösung auch im Definitionsbereich\index{Definitionsmenge}\index{Definitionsbereich|see{Definitionsmenge}} $\mathcal{D}$? (ja)

}%% END TNT
%% implicit endofpage

%%%%%%%%%%%%%%%%%%%%%%%%%%%%%%%%%%%%%%%%%%%%%555
  
\subsection{Definitionsmenge einer
    Gleichung}\index{Definitionsmenge!bei Gleichungen}

\begin{definition}{Definitionsmenge}{}  Die Menge aller Zahlen, welche für die Lösung einer Gleichnug in Frage kommen,
  nennen wir die \textbf{Definitionsmenge}.
\end{definition}

  Insbesondere ist die Menge der Lösungen eingeschränkt durch
  die Terme, welche die Gleichung definieren. So besteht die folgende
  Gleichung aus zwei Termen ($T1=\frac{4}{x-2}$ und $T2=\frac{\sqrt{x}}{x-3}$) mit den unten angegebenen
  Einschränkungen:
  $$\frac{4}{x-2}=\frac{\sqrt{x}}{x-3}$$

$$T1=\frac{4}{x-2} \text{ und } T2=\frac{\sqrt{x}}{x-3}$$
  
  Die Definitionsmengen von $T1$ und $T2$ schränken automatisch die
  Grundmenge der Gleichung ein. In obigem Beispiel gilt:
  
  $$\mathcal{D}_1=\mathcal{D}(T1)=\LoesungsRaum{\mathbb{R}\backslash\{2\}}$$

  $$\mathcal{D}_2=\mathcal{D}(T2)=\LoesungsRaum{\mathbb{R}^+_0\backslash\{3\}}$$

  $$\mathcal{D}=\mathcal{D}_1\cap\mathcal{D}_2=\LoesungsRaumLang{\mathbb{R}^{+}_{0}\backslash\{2;3\}}$$

  \begin{bemerkung}{Schnittmenge}{}
    Dabei bedeutet das $\cap$-Zeichen «Schnittmenge».

    $$\mathcal{D}_1\cap\mathcal{D}_2$$
    heißt also, dass nur Zahlen genommen werden, die \textbf{sowohl}
    in $\mathcal{D}_1$ \textbf{als auch} in $\mathcal{D}_2$ liegen.
    \end{bemerkung}

  \TRAINER{\GESO{GESO: $R_0^+$ müssen Sie nicht im Deteil können, aber die 2
  und die 3 müssen Sie ausschließen!}}
  \newpage
  
\begin{rezept}{Bruchgleichungen}{}
  \begin{itemize}
  \item Alle Terme nach links nehmen (wie bei den quadratischen Gleichungen)
  \item links: Definitionsmenge $\mathcal{D}$ bestimmen und
    Termvereinfachung wie im Rezept über Addition/Subtraktion von
    Brüchen\ifisALLINONE{\totalref{bruchtermeRezept}}\fi{}; so, dass nur noch ein vereinfachter Bruch dasteht.
  \item Zähler muss gleich Null sein; Lösungsmenge bestimmen
   \item Definitionsmenge $\mathcal{D}$ mit Lösungsmenge $\mathbb{L}$
     vergleichen: Scheinlösungen müssen ausgeschlossen werden
  \end{itemize}
  
\end{rezept}

\subsection*{Aufgaben}


\GESO{\olatLinkArbeitsblatt{Bruchgleichungen}{https://olat.bbw.ch/auth/RepositoryEntry/572162163/CourseNode/105951755115452}{1.), 2.) und 3.)}}

\olatLinkGESOKompendium{2.1.2.}{10}{7. bis 11.}

\TALS{\olatLinkArbeitsblatt{Bruchgleichungen}{https://olat.bbw.ch/auth/RepositoryEntry/572162090/CourseNode/105951754967029}{1.), 2.) und 3.)}}

\newpage
