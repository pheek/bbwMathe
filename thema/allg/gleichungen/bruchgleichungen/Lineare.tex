%%
%% 2019 07 04 Ph. G. Freimann
%%

\subsection{... (linear werdende) ...}

\sectuntertitel{125\% der Leute können nicht Bruchrechnen. Das ist
jeder vierte, nein mehr noch: Jeder fünfte!}



\TadBMTA{118}{8.3}

\TALSTadBMTA{121}{8.4}

%%\TALSTadBFWA{105}{2.4.1}
%%%%%%%%%%%%%%%%%%%%%%%%%%%%%%%%%%%%%%%%%%%%%%%%%%%%%%%%%%%%%%%%%%%%%%%%%%%%%%%%%
\subsection*{Lernziele}

\begin{itemize}
	\item gleichnamig machen
\end{itemize}

\begin{definition}{Bruchgleichung}{definition_bruchgleichung}\index{Bruchgleichungen}
  Unter einer Bruchgleichung verstehen wir eine Gleichung, bei der die
  gesuchte Variable (mindestens einmal) \textbf{im Nenner} vorkommt.
\end{definition}

\begin{beispiel}{Bruchgleichung}{beispiel_beispiel_einer_bruchgleichung}
$$\frac{1+x}{x}=\frac{x+3}{x-3}$$
\end{beispiel}

\begin{bemerkung}{\textbf{Keine} Bruchgleichung!}{}
  $$\frac{1+x}7=\frac{x+3}{\sqrt{2}}$$
  \TNT{2}{$$\frac17 + \frac17x = \frac1{\sqrt{2}}x+\frac3{\sqrt{2}}$$}
  \end{bemerkung}
\newpage
\subsubsection{Referenzaufgabe}

$$\frac{1+x}x=\frac{12}{2x}$$
  
\TNTeop{
  1. Alles auf eine Seite bringen (Grundform):

  $$\frac{1+x}x - \frac{12}{2x} = 0$$

  2. Bruchterm vereinfachen \ifisALLINONE{(Vgl. \totalref{bruchterme})}\fi

  Als einzigen Bruchterm schreiben:
  
  $$\frac{2(1+x)}{2\cdot{}x} - \frac{12}{2x} = 0$$

  $$\frac{2(1+x) - 12}{2x} = 0$$

  3. Wir betrachten nur den Zähler!

  Begründung I: «Ein Bruch ist genau dann = 0, wenn der Zähler = 0 ist»;

  Begründung II: Beidseitig multiplizieren mit dem Nenner: Rechts bleibt die Null stehen.

  $$2(1+x) - 12 = 0 \stackrel{!}{=} 0$$

  $$  2 + 2x - 12 = 0$$
  $$2x=10$$
  und somit:

  $$x = 5$$

  Probe a) stimmt die Lösung in der ursprünglichen Gleichung? $\frac65\stackrel{?}{=}\frac{12}{10}$ (ja)\\

}%% END TNT
%% implicit endofpage

\newpage

\subsection{Definitionsmenge}\index{Definitionsmenge!Bruchterme}\index{Definitionsmenge@Definitionsbereich}
Einstiegsbeispiel:
$$\frac{3(x-2)}{x-5} = \frac{x+23}{2x-10}$$

\TNTeop{
  1.  Grundform (alles nach links)

  $$\frac{3(x-2)}{x-5} - \frac{x+23}{2x-10} = 0$$

  2. Bruchterm links vereinfachen (auf einen Bruchstrich):

  $$\frac{2\cdot{}3(x-2)}{2x - 10} - \frac{x+23}{2x-10} = 0$$


    $$\frac{2\cdot{}3(x-2) - (x+23)}{2x - 10} = 0$$

  3. Nur Zähler betrachten:

    $$2\cdot{}3(x-2) - (x+23) = 0$$

  4. Auflösen
    $$6x-12 -x - 23 = 0$$
  
    $$5x = 25$$

  $$x=5$$
  5. Probe:

  $$\frac{3({\color{red}5}-2)}{{\color{red}5}-5} \stackrel{?}{=}
  \frac{{\color{red}5}+23}{2\cdot{}{\color{red}5}-10}$$

  $$\frac90 \stackrel{?}{=} \frac{28}0$$

  Nicht möglich!
}%% end TNT eop
%% implicit nepage
%%%%%%%%%%%%%%%%%%%%%%%%%%%%%%%%%%%%%%%%%%%%%%%%%%%%%%%%%%%%%%%%%%%%%%%%%
Noch ein Beispiel:

$$\frac0x = 3.84$$
\TNT{2}{$$\cdot x \Longrightarrow   0 = x\cdot{} 3.84 \Longrightarrow
  x = 0$$

  Null ist aber keine Lösung!
}



\begin{definition}{Definitionsmenge}{}
Die Menge aller Zahlen, die für die Variable in einen Term (\zB einen
Bruch) eingesetzt werden darf, nennen wir
die \textbf{Definitionsmenge} $\DefinitionsMenge{}$ oder den \textbf{Definitionsbereich}.
\end{definition}

%%\renewcommand{\arraystretch}3
\begin{bbwFillInTabular}{c|c|l}%%
Bruch                  & Was darf nicht eingesetzt werden? & $\DefinitionsMenge{}$ Definitionsmenge \\\hline
$\frac1x$              & \TRAINER{0}                       & \TRAINER{$\DefinitionsMenge{}=\mathbb{R}\backslash\{0\}$}\\\hline  
$\frac1{2-x}$          & \TRAINER{2}                       & \TRAINER{$\DefinitionsMenge{}=\mathbb{R}\backslash\{2\}$}\\\hline
$\frac{x}{(x-5)(x+3)}$ & \TRAINER{-3; 5}                   & \TRAINER{$\DefinitionsMenge{}=\mathbb{R}\backslash\{-3; 5\}$}\\\hline
$\frac{11}{5x^2-5x-60}$ & \TRAINER{-3; 4}                   & \TRAINER{$\DefinitionsMenge{}=\mathbb{R}\backslash\{-3; 4\}$}\\\hline
\end{bbwFillInTabular}


\TNT{4}{$$5x^2 - 5x - 60 $$   $$= 5(x^2-x-12) $$   $$= 5(x-4)(x+3)$$}


\newpage
Ein letztes Beispiel:

%%\renewcommand{\arraystretch}3
\begin{bbwFillInTabular}{c|c|l}%%
Bruch                  & Was darf nicht eingesetzt werden? & $\DefinitionsMenge{}$ Definitionsmenge \\\hline
$\frac{7x^4-4x+11}{3x^2-5x}$ & \TRAINER{$0; \frac53$}                   & \TRAINER{$\DefinitionsMenge{}=\mathbb{R}\backslash\{0;\frac53\}$}\\\hline
\end{bbwFillInTabular}

\TNT{10}{Wir müssen nur den Nenner (unten) betrachten:
$$3x^2-5x\stackrel{?}{=} 0$$
$$x\cdot{}(3x-5)\stackrel{?}{=} 0$$
Erster Fall: $x=0$

Zweiter Fall: $$3x-5 = 0$$

d. h.:

$$3x=5$$
$$x=\frac53$$

}
\newpage
%\subsection*{Aufgaben}

%\AadBMTA{48ff}{4. a) b) und 3. c) d)}%


%%%%%%%%%%%%%%%%%%%%%%%%%%%%%%%%%%%%%%%%%%%%%555

\subsection*{Aufgaben}

%\TRAINER{Ev. Arbeitsblatt Steve}


\textbf{Definitionsbereich: }

\GESO{\olatLinkArbeitsblatt{Bruchgleichungen (Definitionsbreich)}{https://olat.bbw.ch/auth/RepositoryEntry/572162163/CourseNode/108794294060339}{1.),
    2.) und Kap. 2: Tabelle Definitionsbereich}}


\TALS{\olatLinkArbeitsblatt{Bruchgleichungen (Definitionsbereich)}{https://olat.bbw.ch/auth/RepositoryEntry/572162090/CourseNode/105951754967029}{1.),
    2.) und Kap. 2: Tabelle Definitionsbereich}}

\vspace{5mm}
\textbf{Bruchgleichungen, die auf lineare Gleichungen führen}


\GESO{\olatLinkArbeitsblatt{Bruchgleichungen}{https://olat.bbw.ch/auth/RepositoryEntry/572162163/CourseNode/108794294060339}{3.,
    4. und 5.}}

\olatLinkGESOKompendium{2.1.2.}{10}{7. bis 11.}

\TALS{\olatLinkArbeitsblatt{Bruchgleichungen}{https://olat.bbw.ch/auth/RepositoryEntry/572162090/CourseNode/105951754967029}{3.,
    4. und 5.}}




\newpage
