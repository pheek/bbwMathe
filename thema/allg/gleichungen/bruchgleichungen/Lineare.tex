%%
%% 2019 07 04 Ph. G. Freimann
%%

\subsection{... lineare ...}

\sectuntertitel{125\% der Leute können nicht Bruchrechnen. Das ist
jeder vierte, nein mehr noch: Jeder fünfte!}



\TadBMTA{118}{8.3}

\TALSTadBMTA{121}{8.4}

%%\TALSTadBFWA{105}{2.4.1}
%%%%%%%%%%%%%%%%%%%%%%%%%%%%%%%%%%%%%%%%%%%%%%%%%%%%%%%%%%%%%%%%%%%%%%%%%%%%%%%%%
\subsection*{Lernziele}

\begin{itemize}
	\item gleichnamig machen
\end{itemize}

\begin{definition}{Bruchgleichung}{definition_bruchgleichung}\index{Bruchgleichungen}
  Unter einer Bruchgleichung verstehen wir eine Gleichung, bei der die
  gesuchte Variable (mindestens einmal) \textbf{im Nenner} vorkommt.
\end{definition}

\begin{beispiel}{Bruchgleichung}{beispiel_beispiel_einer_bruchgleichung}
$$\frac{1+x}{x}=\frac{x+3}{x-3}$$
\end{beispiel}

\begin{bemerkung}{\textbf{Keine} Bruchgleichung!}{}
  $$\frac{1+x}7=\frac{x+3}{\sqrt{2}}$$
  \TNT{2}{$$\frac17 + \frac17x = \frac1{\sqrt{2}}x+\frac3{\sqrt{2}}$$}
  \end{bemerkung}
\newpage
\subsubsection{Referenzaufgabe}

$$\frac{1+x}{x}=\frac{x+3}{x-3}$$
  
\TNTeop{
  1. Alles auf eine Seite bringen (Grundform):

  $$\frac{1+x}x - \frac{x+3}{3-x} = 0$$

  2. Bruchterm vereinfachen \ifisALLINONE{(Vgl. \totalref{bruchterme})}\fi

  $$\frac{(1+x)(x-3)}{x(x-3)} - \frac{x(x+3)}{x(x-3)} = 0$$

  $$\frac{(1+x)(x-3) - x(x+3)}{x(x-3)} = 0$$

  $$\frac{-5x-3}{x(x-3)} = 0$$

  3. Wir betrachten nur den Zähler!

  Begründung I: «Ein Bruch
  ist genau dann = 0, wenn der Zähler = 0 ist»; Begründung II:
  Beidseitig multiplizieren mit dem Nenner: Rechts bleibt die Null stehen.

  $$-5x-3 \stackrel{!}{=} 0$$

  und somit:

  $$x = \frac{-3}5$$

  Probe a) stimmt die Lösung in der ursprünglichen Gleichung? (ja)\\

  Probe b) liegt die Lösung auch im Definitionsbereich\index{Definitionsmenge}\index{Definitionsbereich|see{Definitionsmenge}} $\DefinitionsMenge{}$? (ja)

}%% END TNT
%% implicit endofpage

%%%%%%%%%%%%%%%%%%%%%%%%%%%%%%%%%%%%%%%%%%%%%555

\subsection*{Aufgaben}


\GESO{\olatLinkArbeitsblatt{Bruchgleichungen (Definitionsbreich wie
    bei Termen)}{https://olat.bbw.ch/auth/RepositoryEntry/572162163/CourseNode/105951755115452}{1.), 2.) und 3.)}}

\olatLinkGESOKompendium{2.1.2.}{10}{7. bis 11.}

\TALS{\olatLinkArbeitsblatt{Bruchgleichungen (Definitionsbereich wie
    bei Termen)}{https://olat.bbw.ch/auth/RepositoryEntry/572162090/CourseNode/105951754967029}{1.), 2.) und 3.)}}

\newpage
