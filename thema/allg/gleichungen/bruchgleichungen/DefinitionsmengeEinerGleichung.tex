\subsection{Definitionsmenge einer
    Gleichung}\index{Definitionsmenge!bei Gleichungen}

\begin{definition}{Definitionsmenge}{}  Die Menge aller Zahlen, welche für die Lösung einer Gleichung in Frage kommen,
  nennen wir die \textbf{Definitionsmenge}.
\end{definition}

  Insbesondere ist die Menge der Lösungen eingeschränkt durch
  die Terme, welche die Gleichung definieren. So besteht die folgende
  Gleichung aus zwei Termen ($T1=\frac{4}{x-2}$ und $T2=\frac{\TALS{\sqrt{x}}\GESO{x+7}}{x-3}$) mit den unten angegebenen
  Einschränkungen:
  $$\frac{4}{x-2}=\frac{\TALS{\sqrt{x}}\GESO{x+7}}{x-3}$$

$$T1=\frac{4}{x-2} \text{ und } T2=\frac{\TALS{\sqrt{x}}\GESO{x+7}}{x-3}$$
  
  Die Definitionsmengen von $T1$ und $T2$ schränken automatisch die
  Grundmenge der Gleichung ein. In obigem Beispiel gilt:
  
  $$\DefinitionsMenge{}_1=\DefinitionsMenge{}(T1)=\LoesungsRaum{\mathbb{R}\backslash\{2\}}$$

  $$\DefinitionsMenge{}_2=\DefinitionsMenge{}(T2)=\LoesungsRaum{\TALS{\mathbb{R}^+_0\backslash\{3\}}\GESO{\mathbb{R}\backslash{}\{3\}}}$$

  \TALS{  $$\DefinitionsMenge{}=\DefinitionsMenge{}_1\cap\DefinitionsMenge{}_2=\LoesungsRaumLang{\mathbb{R}^{+}_{0}\backslash\{2;3\}}$$}
  \GESO{  $$\DefinitionsMenge{}=\DefinitionsMenge{}_1\cap\DefinitionsMenge{}_2=\LoesungsRaumLang{\mathbb{R}\backslash\{2;3\}}$$}
  

  \begin{bemerkung}{Schnittmenge}{}
    Dabei bedeutet das $\cap$-Zeichen «Schnittmenge».

    $$\DefinitionsMenge{}_1\cap\DefinitionsMenge{}_2$$
    heißt also, dass nur Zahlen genommen werden, die \textbf{sowohl}
    in $\DefinitionsMenge{}_1$ \textbf{als auch} in $\DefinitionsMenge{}_2$ liegen.
    \end{bemerkung}

  \TRAINER{\GESO{GESO: $R_0^+$ müssen Sie nicht im Deteil können, aber die 2
  und die 3 müssen Sie ausschließen!}}
  \newpage
