%%
%% 2019 07 04 Ph. G. Freimann
%%
\subsection{... quadratische}\index{Gleichungen!Bruchgleichungen, quadratische}\index{quadratische Bruchgleichungen}
%%\sectuntertitel{...}

\TadBMTA{118}{8.3}

\TALSTadBMTA{121}{8.4}
%%\TALSTadBFWA{105}{2.4.1}
%%%%%%%%%%%%%%%%%%%%%%%%%%%%%%%%%%%%%%%%%%%%%%%%%%%%%%%%%%%%%%%%%%%%%%%%%%%%%%%%%
\subsection*{Lernziele}

\begin{itemize}
\item Bruchgleichungen, die in quadratischen Gleichungen münden
\item Definitionsbereich
%%\TALS{  \item \textit{Bruch\textbf{un}gleichungen (halbgrafische Methode)}}
\end{itemize}

Bruchgleichungen können beim Lösen auch auf quadratische Gleichungen führen:
\begin{beispiel}{quadratische Bruchgleichung}{beispiel_quadratische_bruchgleichung}
$$\frac{6x-24}{3-x} + x - 2 = \frac{6}{x-3}$$
\end{beispiel}

\newpage
Vorzeigebeispiel:

$$\frac{6x-24}{3-x} + x - 2 = \frac{6}{x-3}$$
\TNTeop{
Definitionsbereich:
  $\mathcal{D} = \mathbb{Q}\backslash \left\{3\right\}$\\

Alle Terme nach links bringen:

$$\frac{6x-24}{3-x} + x - 2 - \frac{6}{x-3} = 0$$

Termvereinfachung (alles gleichnamig); 3. Bruch mit -1 erweitern: 


$$\frac{6x-24}{3-x} + \frac{(x-2)(3-x)}{3-x} + \frac{6}{3-x} = 0$$


gemeinsamer Bruchstrich:

$$\frac{6x-24+(-x^2+5x-6)+6}{3-x} = 0$$

Zähler vereinfachen:

$$\frac{-x^2+11x-24}{3-x} = 0$$

Zähler muss Null sein!

$$-x^2 +11x-24 = 0 \stackrel{\text{TR}}{\Longrightarrow} x_1 = 8; x_2=3$$
Nur Definitionsbereich in der Lösungsmenge:

$$\mathbb{L} = \{8\}$$
\vspace{20mm}
}%% END TNT eop

%%%%%%%%%%%%%%%%%%%%%%%%%%%%%%%%%%%%%%%%%%%%%%
\newpage


\subsection*{Aufgaben}

\olatLinkGESOKompendium{2.3.1.}{16}{45., 50. und 51.}


\GESO{\olatLinkArbeitsblatt{Bruchgleichungen}{https://olat.bbw.ch/auth/RepositoryEntry/572162163/CourseNode/105951755115452}{4.), 5.) und 6.)}}

\TALS{\olatLinkArbeitsblatt{Bruchgleichungen}{https://olat.bbw.ch/auth/RepositoryEntry/572162090/CourseNode/105951754967029}{4.), 5.) und 6.)}}



\GESO{\olatLinkArbeitsblatt{Bruchgleichungen alte Maturaaufgaben}{https://olat.bbw.ch/auth/RepositoryEntry/572162163/CourseNode/102611886537294}{[2017/2018]}}
\newpage
