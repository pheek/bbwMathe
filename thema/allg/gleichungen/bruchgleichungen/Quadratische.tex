%%
%% 2019 07 04 Ph. G. Freimann
%%
\subsection{... quadratisch werdende}\index{Gleichungen!Bruchgleichungen, quadratische}\index{quadratische Bruchgleichungen}
%%\sectuntertitel{...}

\TadBMTA{118}{8.3}

\TALSTadBMTA{121}{8.4}
%%\TALSTadBFWA{105}{2.4.1}
%%%%%%%%%%%%%%%%%%%%%%%%%%%%%%%%%%%%%%%%%%%%%%%%%%%%%%%%%%%%%%%%%%%%%%%%%%%%%%%%%


Bruchgleichungen können beim Lösen auch auf quadratische Gleichungen führen:
\begin{beispiel}{quadratische Bruchgleichung}{beispiel_quadratische_bruchgleichung}
$$\frac{6x-24}{3-x} + x - 2 = \frac{6}{x-3}$$
\end{beispiel}

\newpage
Vorzeigebeispiel\TRAINER{ = Maturaaufgabe 2018 Serie I Aufg 5.}:

$$\frac{6x-24}{3-x} + x - 2 = \frac{6}{x-3}$$

\TNT{2}{
 (erst am Schluss ausfüllen, da der Definitionsbereich für Gleichungen
  noch nicht definiert worden ist)
  
  $\DefinitionsMenge{} = \mathbb{Q}\backslash \left\{3\right\}$\\
}
\vspace{5mm}

\TNTeop{

Alle Terme nach links bringen:

$$\frac{6x-24}{3-x} + x - 2 - \frac{6}{x-3} = 0$$

Termvereinfachung (alles gleichnamig); 3. Bruch mit -1 erweitern: 


$$\frac{6x-24}{3-x} + \frac{(x-2)(3-x)}{3-x} + \frac{6}{3-x} = 0$$


gemeinsamer Bruchstrich:

$$\frac{6x-24+(-x^2+5x-6)+6}{3-x} = 0$$

Zähler vereinfachen:

$$\frac{-x^2+11x-24}{3-x} = 0$$

Zähler muss Null sein!

$$-x^2 +11x-24 = 0 \stackrel{\text{TR}}{\Longrightarrow} x_1 = 8; x_2=3$$
Nur Definitionsbereich in der Lösungsmenge:

$$\LoesungsMenge{} = \{8\}$$
\vspace{20mm}
}%% END TNT eop

%% Optional, da bereits bei linearen behandelt:
%% \subsection{Definitionsmenge einer
    Gleichung}\index{Definitionsmenge!bei Gleichungen}

\begin{definition}{Definitionsmenge}{}  Die Menge aller Zahlen, welche für die Lösung einer Gleichung in Frage kommen,
  nennen wir die \textbf{Definitionsmenge}.
\end{definition}

  Insbesondere ist die Menge der Lösungen eingeschränkt durch
  die Terme, welche die Gleichung definieren. So besteht die folgende
  Gleichung aus zwei Termen ($T1=\frac{4}{x-2}$ und $T2=\frac{\TALS{\sqrt{x}}\GESO{x+7}}{x-3}$) mit den unten angegebenen
  Einschränkungen:
  $$\frac{4}{x-2}=\frac{\TALS{\sqrt{x}}\GESO{x+7}}{x-3}$$

$$T1=\frac{4}{x-2} \text{ und } T2=\frac{\TALS{\sqrt{x}}\GESO{x+7}}{x-3}$$
  
  Die Definitionsmengen von $T1$ und $T2$ schränken automatisch die
  Grundmenge der Gleichung ein. In obigem Beispiel gilt:
  
  $$\DefinitionsMenge{}_1=\DefinitionsMenge{}(T1)=\LoesungsRaum{\mathbb{R}\backslash\{2\}}$$

  $$\DefinitionsMenge{}_2=\DefinitionsMenge{}(T2)=\LoesungsRaum{\TALS{\mathbb{R}^+_0\backslash\{3\}}\GESO{\mathbb{R}\backslash{}\{3\}}}$$

  \TALS{  $$\DefinitionsMenge{}=\DefinitionsMenge{}_1\cap\DefinitionsMenge{}_2=\LoesungsRaumLang{\mathbb{R}^{+}_{0}\backslash\{2;3\}}$$}
  \GESO{  $$\DefinitionsMenge{}=\DefinitionsMenge{}_1\cap\DefinitionsMenge{}_2=\LoesungsRaumLang{\mathbb{R}\backslash\{2;3\}}$$}
  

  \begin{bemerkung}{Schnittmenge}{}
    Dabei bedeutet das $\cap$-Zeichen «Schnittmenge».

    $$\DefinitionsMenge{}_1\cap\DefinitionsMenge{}_2$$
    heißt also, dass nur Zahlen genommen werden, die \textbf{sowohl}
    in $\DefinitionsMenge{}_1$ \textbf{als auch} in $\DefinitionsMenge{}_2$ liegen.
    \end{bemerkung}

  \TRAINER{\GESO{GESO: $R_0^+$ müssen Sie nicht im Deteil können, aber die 2
  und die 3 müssen Sie ausschließen!}}
  \newpage

  %%%%%%%%%%%%%%%%%%%%%%%%%%%%%%%%%%%%%%%%%%%%%%%%%%%%%%%%%%%%%%%%%%%%%%%%%%%%%%%%%%%%%%%%%%%%%%%
    
\begin{rezept}{Bruchgleichungen}{}
  \begin{itemize}
  \item Alle Terme nach links
  \item Definitionsmenge $\DefinitionsMenge{}$ bestimmen
  \item Termvereinfachung: Als einzigen Bruch schreiben
  \item Zähler gleich Null setzen
  \item Lösungsmenge\GESO{ (linear / qudaratisch) }bestimmen
  \item Definitionsmenge $\DefinitionsMenge{}$ mit Lösungsmenge $\LoesungsMenge{}$
     vergleichen: Scheinlösungen ausschließen
  \end{itemize}
  
\end{rezept}

\subsection*{Aufgaben}

\olatLinkGESOKompendium{2.3.1.}{16}{45., 50. und 51.}

\GESO{\olatLinkArbeitsblatt{Bruchgleichungen}{https://olat.bms-w.ch/auth/RepositoryEntry/6029794/CourseNode/105951755115452}{6.)
    und 8.) (7. nur Technische BM)}}

\TALS{\olatLinkArbeitsblatt{Bruchgleichungen}{https://olat.bms-w.ch/auth/RepositoryEntry/6029786/CourseNode/105951754967029}{6.), 7.) und 8.)}}


\GESO{\olatLinkArbeitsblatt{Bruchgleichungen alte Maturaaufgaben}{https://olat.bms-w.ch/auth/RepositoryEntry/6029794/CourseNode/102611886537294}{[2017/2018]}}
\newpage
