%%
%% 2019 07 04 Ph. G. Freimann
%%

\section{Exponentialgleichungen}\index{Gleichungen!Exponentialgleichungen}
\sectuntertitel{Exponenten? Und warum haben sie aufgehört Ponenten zu sein?}

\GESOTadBMTA{199}{12}
%%%%%%%%%%%%%%%%%%%%%%%%%%%%%%%%%%%%%%%%%%%%%%%%%%%%%%%%%%%%%%%%%%%%%%%%%%%%%%%%%
\subsection*{Lernziele}

\begin{itemize}
\item Exponentialgleichungen
\item logarithmische Gleichungen \GESO{(\cite{marthaler21alg} S. 203
    Kap. 12.2)}
\TALS{\item Exponentialgleichungen mit Parametern}
\end{itemize}

\TALS{Theorie: (\cite{frommenwiler17alg} S.117 (Kap. 2.4.4))}

\bbwCenterGraphic{8cm}{allg/gleichungen/img/Ritter.jpg}

\GESO{\matheNinjaLink{Exponentialgleichungen}{https://olat.bbw.ch/auth/RepositoryEntry/667320356/CourseNode/105951756546053}}

\newpage

\subsection{Exponentialgleichung\TALS{...}}\index{Exponentialgleichungen}
\begin{definition}{Exponentialgleichung}{}
  Bei einer \textbf{Exponentialgleichung} kommt die gesuchte Größe im
  Exponenten (von Potenzen) vor.
\end{definition}

Beispiel:

$$5^{x+1} = 34$$

Typischerweise werden diese Gleichungen gelöst, indem die Definition
des Logarithmus angewendet wird.

\textbf{Typ I:} Exponentenvergleich\\

\begin{rezept}{Exponentenvergleich}{}
Bei Exponentialgleichungen der Form $$5^{x+1} = 5^{2x-1}$$ können bei
gleicher Basis einfach die Exponenten verglichen werden:
$$\LoesungsRaumLang{\Longleftrightarrow x+1=2x-1}$$

Es entsteht eine lineare Gleichung mit der Lösung $\LoesungsRaum{\lx = \{2\}}$
\end{rezept}

\textbf{Typ II:} Lösen durch Logarithmieren\\

\begin{rezept}{Logarithmieren}{}
$$5^x=32 \Longleftrightarrow x=\LoesungsRaum{\log_5(32)} \Longrightarrow x \approx \LoesungsRaum{2.15338}$$

Die Lösung kann dann auch mit Logarithmen zur Basis 10 angegeben
werden:
$$\LoesungsRaumLang{x = \log_5(32) = \frac{\lg(32)}{\lg(5)}}$$
\end{rezept}
\newpage

\textbf{Typ III:} Allgemeiner Fall\\

\begin{rezept}{Exponentialgleichung lösen}{rezept_allgemeine_exponentialgleichung}
$$8^{x-1}=5\cdot{}7^{x+2}$$
\end{rezept}

Siehe auch \cite{marthaler21alg} Seite 200 im roten Kasten.

\TNTeop{
      1. Zuerst die Summen in den Exponenten wegbringen:
      $$8^x\cdot{}8^{-1} = 5\cdot{}7^x\cdot{}7^2$$
      2. Jetzt alle Potenzen mit $x$ auf eine Seite bringen:
      $$\frac{8^x}{7^x} = 5\cdot{}\frac{7^2}{8^{-1}}$$
      3. Vereinfachen und als Potenz in $x$ schreiben:
      $$\left(\frac87\right)^x = 5\cdot{}7^2\cdot{}8$$
      4. Definition Logarithmus anwenden ($a^x=b \Leftrightarrow x=\log_a(b)$):
      $$x = \log_{\frac87}(5\cdot{}7^2\cdot{}8) \approx{} 56.77$$
      (5.) Dies kann (falls gefragt) mit Zehnerlogarithmen geschrieben
      werden:
      $$x = \frac{\lg(5\cdot{}7^2\cdot{}8)}{\lg(\frac87)} \approx
      56.77$$
}%% END TNT

%%%%%%%%%%%%%%%%%%%%%%%%%%%%%%%%%%%%%%%%%%%%%%%%%%%%%%%%%%%%%%%%%%%%%%%%%%%

\TALS{%% Methode mit Logarithmieren von Anfang an

\begin{rezept}{... durch direktes Logarithmieren.}{}
$$8^{x-1}=5\cdot{}7^{x+2}$$
\end{rezept}

  \TNTeop{
  Erst mal auf beiden Seiten logarithmieren, mit einem Logarithmus zu beliebiger Basis:
  $$\log(8^{x-1}) = \log(5\cdot{}7^{x+2})$$
  Log Gesetz (Produktregel):
  $$\log(8^{x-1}) =\log(5)+ \log(7^{x+2})$$
  Log Gesetz (Potenzregel)\TRAINER{\footnote{Ab hier ist es eine
      lineare Gleichung}}:
  $$(x-1)\cdot{}\log(8) = \log(5) + (x+2)\cdot{}\log(7)$$
  ausmultiplizieren
  $$\log(8)\cdot{}x-\log(8) = \log(5) + \log(7)\cdot{}x+2\cdot{}\log(7)$$
  und $x$ auf eine Seite bringen
  $$\log(8)\cdot{}x-\log(7)\cdot{}x =\log(5)+ 2\cdot{}\log(7)+\log(8)$$
  $x$ ausklammern
  $$x\cdot{}(\log(8)-\log(7)) = \log(5) + 2\cdot{}\log(7)+\log(8)$$
  und durch die Klammer $(\log(8)-\log(7))$ teilen
  $$x = \frac{\log(5) + 2\cdot{}\log(7)+\log(8)}{\log(8)-\log(7)} \approx 56.77$$
}%% END TNTeop
%%\newpage
}%% END TALS
\GESO{\newpage}

\subsection*{Aufgaben}

\GESO{\olatLinkArbeitsblatt{Exponentialgleichungen
  [GL\_Ex]}{https://olat.bbw.ch/auth/RepositoryEntry/572162163/CourseNode/106029166708108}{1. bis
  \GESO{3.}\TALS{5.}}}
\TALS{\olatLinkArbeitsblatt{Exponentialgleichungen
  [GL\_Ex]}{https://olat.bbw.ch/auth/RepositoryEntry/572162090/CourseNode/106029166654550}{1. bis
  \GESO{3.}\TALS{5.}}}


Textaufgaben:

\GESO{\olatLinkArbeitsblatt{Exponentialgleichungen  [GL\_Ex]}{https://olat.bbw.ch/auth/RepositoryEntry/572162163/CourseNode/106029166708108}{6. - 9.}}
\TALS{\olatLinkArbeitsblatt{Exponentialgleichungen  [GL\_Ex]}{https://olat.bbw.ch/auth/RepositoryEntry/572162090/CourseNode/106029166654550}{6. - 9.}}

\olatLinkGESOKompendium{2.5}{19}{66. bis 73.}

%%\TALSAadBMTA{118}{359. a) d), 360. a) d), 361. a) d), 362. a) b) c), 363. a) b) d)}
%%\GESOAadBMTA{71 (Exponentenvergleich)}{44}
%%\GESOAadBMTA{206}{2. a) d) g) h), 3. a) b) c) d) f), 4. a) d) e) g) h),
%%  9. und 11.}
