%%
%% 2019 07 04 Ph. G. Freimann
%%

\section{Lineare Gleichungen mit Parametern}\index{Gleichungen!lineare mit Parametern}\index{Parameter}
\sectuntertitel{Alle Zahlen sind gleich; nur einige sind gleicher.}
%%\TALSTadBFWA{89}{2.2.2}
%%%%%%%%%%%%%%%%%%%%%%%%%%%%%%%%%%%%%%%%%%%%%%%%%%%%%%%%%%%%%%%%%%%%%%%%%%%%%%%%%
\subsection*{Lernziele}

\begin{itemize}
\item Lineare Gleichungen mit Parametern
\end{itemize}
\TadBMTA{116}{8.2}

Parameter = (wörtlich) «Neben-Maß»

Griechisch $\pi\alpha\rho\alpha$ (para), deutsch: ‚neben‘ und $\mu\epsilon\tau\rho\omega\nu$ (metron) deutsch: ‚Maß‘


\TALS{%% TALS Schieberegler im CAS-Rechner

\paragraph{Schieberegler}\index{Schieberegler} Der Taschenrechner \textit{TI-$n$Spire CX
  II-T CAS} kann Terme in der Variablen $x$ parametrisiert
darstellen. Erstellen Sie dazu in einem Dokument eine
\textit{notes}-Page und definieren Sie den Term $terma := a\cdot
x+0.5$. In einem neuen Grafik-Fenster (Page) zum selben Problem
definieren Sie die Funktion $f1(x):=terma$. Damit erseint die Frage
nach einem Schieberegler, welcher uns den Parameter $a$ verändern
lassen kann.

\bbwCenterGraphic{6cm}{allg/gleichungen/img/LineareFunktionTR_terma.png}
%  \begin{center}
%   \raisebox{-1cm}{\includegraphics[width=6cm]{img/LineareFunktionTR_terma.png}}
%  \end{center}


\paragraph{Frage 1} Bei welchem Parameter $b$ hat die folgende Gleichung für $x$ die Lösung $1.5$?
$$-2x + b = 0$$

\noTRAINER{$$.....................................$$}
\TRAINER{$$b = 3$$}
Dies lösen wir, indem wir die gesuchte Lösung für $x$ einsetzen und nach $b$ auf"|lösen. Graphisch kann dies auch mit dem CAS-Rechner \textit{gelöst} werden. Definieren Sie «termb := $-2\cdot{}x+b$» und zeichnen Sie den Graphen in einer Graph-Page.
Erstellen Sie einen «Slider»\index{Slider}\index{Schieberegler} für $b$ und ziehen Sie an
diesem \textit{Slider}, bis der Wert der Geraden auf der $x$-Achse den
Wert 1.5 (gesuchte Lösung) angenähert hat.


\paragraph{Frage 2} Bei welchem Parameter $b$ hat die folgende Gleichung für $x$ die Lösung $1.5$?
$$ax-3=0$$

\noTRAINER{$$.....................................$$}
\TRAINER{$$a=2$$}
Dies nähern wir auch mit dem CAS-Rechner an. Definieren Sie den Term
$terma$ neu ($terma := a\cdot{}x-3$) und zeichnen Sie den Graphen in einer Graph-Page.
Ziehen Sie am \textit{Slider}, bis der Wert der Geraden auf der
$x$-Achse dem Wert 1.5 (gesuchte Lösung) genug nahe kommt.
}


\TRAINER{Aufg 247 l aus \cite{frommenwiler17alg}}

\TALS{Vorzeigeaufgabe:
  $$\frac{x}{n} + a - \frac{x}{m} + b = mx + c$$
\TNTeop{
<<<<<<< HEAD
  Lösungsweg:
  $$ \cdot n      \Rightarrow x + an - \frac{nx}{m} + bn = mnx + cn$$
  $$ \cdot m      \Rightarrow mx + anm - nx + bnm = nm^2x + cnm$$
  $$ x \text{\, nach links}  \Rightarrow mx - nx -nm^2x = cnm -anm - bnm$$
  $$ x \text{\, ausklammern} \Rightarrow x(m - n -nm^2) = cnm -anm - bnm$$
  $$ \text{\, dividieren}    \Rightarrow x = \frac{cnm - anm - bnm}{m-n-nm^2} = \frac{nm(c-a-b)}{m-n-nm^2}$$
=======
  1. Alle $x$ nach links, alle «nicht $x$» nach rechts:

  $$\frac{x}{d}-\frac{x}{m}-mx = c-a-b$$

  2. $x$ ausklamern:

  $$x\left(\frac1n-\frac1m-m\right)=c-a-b$$

  3. durch die Klammer teilen

  $$x=\frac{c-a-b}{\frac1n-\frac1m-m} = (\text{optional}) = \frac{mn(c-a-b)}{m-n-m^2n}$$
  
>>>>>>> ef8c20d164776cde33a4cb6509f72042f98a75c8
}%% END TNT
}%% END TALS

\GESO{Vorzeigeaufgabe:
  $$x^2 - \sqrt{2} - c = (b-x)\cdot{}(a-x)$$

  \TNTeop{
  Termumformung: $x^2-\sqrt{2} - c = ab -bx -xa + x^2$\\
  beidseitig $-x^2$: $\Rightarrow -\sqrt{2}-c=ab-bx-xa$\\
  Alle $x$ nach links (Rest nach rechts): $\Rightarrow bx+ax= ab+\sqrt{2} + c$\\
  $x$ ausklammern: $\Rightarrow x(a+b)=ab+\sqrt{2} + c$\\
  $: (a+b)$ : $\Rightarrow x = \frac{ab+\sqrt{2} + c}{a+b}$\\%%

}%% END TNT
}%% END GESO
\newpage

\subsection*{Aufgaben}
%%\TALS{\aufgabenFarbe{Aufgaben: \cite{frommenwiler17alg} S. 89, Aufg. 246. a) 247. a) c) g) 248. a) d) i) 249.}}

%%\TALS{Textaufgaben: \cite{frommenwiler17alg} S. 89 Aufg. 253}

\AadBMTA{128}{ 7. a) b) c) e),
  8.  a) d) e) f),
  9. a) b) c) e) f),
  10. a) b)
\TALS{, 11. b) und 12. a) d) e)}
}%% end AadBMTA

\olatLinkGESOKompendium{2.1.3.}{11}{12. bis 16.}

\newpage
