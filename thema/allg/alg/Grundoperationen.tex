%%
%%
\newpage
\section{Grundoperationen}\index{Grundoperationen}
\renewcommand{\bbwAufgabenBlockID}{A1G}
\sectuntertitel{Im Grunde ganz einfach?}

%%%%%%%%%%%%%%%%%%%%%%%%%%%%%%%%%%%%%%%%%%%%%%%%%%%%%%%%%%%%%%%%%%%%%%%%%%%%%%%%%
\subsection*{Lernziele}
\begin{itemize}
\item Addition\index{Addition}, Subtraktion\index{Subtraktion}
\item Multiplikation\index{Multiplikation}, Ausmultiplizieren\index{ausmultiplizieren}
\item Distributivgesetz\index{Distributivgesetz},
  Assoziativgesetz\index{Assoziativgesetz},
  Kommutativgesetz\index{Kommutativgesetz}
\end{itemize}

%%\TALSTadBFWA{15}{1.2}
\TadBMTA{28}{2}
%%\TALSTadBMTA{28}{2}
\newpage

\subsection{Addition und Subtraktion}\index{Addition}\index{Subtraktion}
\begin{beispiel}{}{}
%%  $$x^2-((x^3-x^2)-(-(-x+x^2)-(x^2-x^3)))$$
$$-bx-(-bx^2+((xb-2ax)-(2bx-x^2b))+3ax)$$
\end{beispiel}

\TNT{10}{\bbwCenterGraphic{17cm}{allg/alg/img/KlammernLoesen.png}}%% END TNT

Regeln:
\begin{itemize}
\item Klammern werden von innen nach außen aufgelöst.
\item Negative Vorzeichen vor Klammern wechseln die Vorzeichen der
      Summanden innerhalb der Klammer.
\item Es können nur gleiche Variable (bzw. Produkte von Faktoren)
      addiert (bzw. subtrahiert) werden: $ab^2 + a^2b + b^2a = 2ab^2+a^2b$.
%%\item Gleichwertige Operationen\footnote{Zum Beispiel alles Minus und Plus oder aber zum Beispiel alles \textit{Punkt}-Operationen (Produkt/Quotient).} werden von links nach rechts \textit{geklammert}:
%%  $$10-4+5 = (10-4) + 5 \ne 10-(4+5)$$
\end{itemize}


\subsection*{Aufgaben}
%%\GESOAadBMTA{34}{1. a) c) e) g), 2. a) e), 3. b), 5. a) f) und 6. d) h)}
%%\TALSAadBMTA{15}{19.}
\GESO{\olatLinkArbeitsblatt{Grundoperationen [A1G]}{https://olat.bbw.ch/auth/RepositoryEntry/572162163/CourseNode/105796961619595}{1. und 2.}}
\TALS{\olatLinkArbeitsblatt{Grundoperationen [A1G]}{https://olat.bbw.ch/auth/RepositoryEntry/572162090/CourseNode/105796961688601}{1. und 2.}}


\newpage
\subsection{Multiplikation}\index{Multiplikation}
\TadBMTA{29}{2.2}
%%\TALS{Theorie im Buch \cite{frommenwiler17alg} S. 16 Kap. 1.3}

Für die Addition und die Multiplikation gelten die folgenden
Gesetze:



\begin{gesetz}{}{}
\begin{itemize}
\item Assoziativgesetz: $a+(b+c) = (a+b)+c$ und $a\cdot(b\cdot{}c) = (a\cdot b)\cdot c$
\item Kommutativgesetz:  $a\cdot b = b \cdot a$
\item Distributivgesetz\footnote{lat. \textit{\textbf{distribuere}} = verteilen}:\\
  $a\cdot (b+c) = a\cdot b + a\cdot c$\\
  $a\cdot (b-c) = a\cdot b - a\cdot c$\\
%%  $(a+b)\cdot c = ac + bc$\\
%%  $(a-b)\cdot c = ac - bc$\\
%%  $(a+b):c = a:c + b:c$\\
%%  $(a-b):c = a:c - b:c$\\
  
  \end{itemize}
\end{gesetz}

\subsubsection{Distributivgesetz}

\youtubeLink{https://www.youtube.com/watch?v=HVk43UypIdY}{Distributivgesetz}
\TNTeop{Platz für graphischen Beweis: a(x+y) = ax + ay}
\newpage

\subsubsection{Ausmultiplizieren}\index{ausmultiplizieren}
Beim Ausmultiplizieren wird jeder Summand in der Klammer mit dem
Faktor vor (bzw. nach) der Klammer multipliziert.
\begin{beispiel}{}{}

  $4\cdot (x + 5) =\LoesungsRaumLang{4\cdot x + 4\cdot 5 = 4x + 20}$
\end{beispiel}

\begin{beispiel}{}{}
%%  \noTRAINER{$$(x + 7)\cdot(8-y) = x\cdot(8-y) + 7\cdot(8-y) =
%%  8x-xy+56-7y$$}\TRAINER{\bbwCenterGraphic{15cm}{allg/alg/img/DistributivRechts.png}}
%%oder
\TRAINER{\\}
  $$(x + 7)\cdot(8-y) = \LoesungsRaumLang{8x+56-xy-7y}$$

  \TRAINER{\par Jeden
  Summanden der ersten Klammer mit jedem Summanden der zweiten
  Klammer multiplizieren}
\end{beispiel}

\newpage

\subsubsection{Achtung}
Auch wenn die folgenden Ausdrücke sehr ähnlich aussehen, so
handelt es sich beim ersten um eine \textbf{Differenz} und bei den anderen um
ein \textbf{Produkt}\TRAINER{ (Genau genommen um die Gegenzahl eines
Produktes) }!

$$10-(x-4) =\LoesungsRaum{14-x}$$

$$-10(x-4) = \LoesungsRaum{-10x + 40}$$

$$-10(x\cdot{}(-4)) = \LoesungsRaum{40x}$$

Diese Unterschied wird auf dem Taschenrechner besonders gut deutlich:

\tiprobutton{7}\tiprobutton{minus}\tiprobutton{3}  \tiprobutton{enter}  \LoesungsRaumLang{$ 7 - 3 = 4$}

\vspace{3mm}
\tiprobutton{7}\tiprobutton{neg}\tiprobutton{3} \tiprobutton{enter}  \LoesungsRaumLang{$ 7\cdot{}(-3) = -21$}


\subsubsection{Referenzaufgabe}
$$3ab-(x-a(2-b))\cdot{}3$$

\TNTeop{$$3ab-(x-2a+ab)\cdot{}3$$
$$3ab - (3x -6a +3ab)$$
$$3ab - 3x +6a -3ab$$
$$6a-3x$$}
\newpage

\subsection*{Aufgaben}
%%\olatLinkArbeitsblatt{Grundoperationen [A1G]}{\GESO{https://olat.bbw.ch/auth/RepositoryEntry/572162163/CourseNode/105796961619595}\TALS{https://olat.bbw.ch/auth/RepositoryEntry/572162090/CourseNode/105796961688601}}{3., 4. und 5.}
\GESO{\olatLinkArbeitsblatt{Grundoperationen [A1G]}{https://olat.bbw.ch/auth/RepositoryEntry/572162163/CourseNode/105796961619595}{3., 4. und 5.}}
\TALS{\olatLinkArbeitsblatt{Grundoperationen [A1G]}{https://olat.bbw.ch/auth/RepositoryEntry/572162090/CourseNode/105796961688601}{3., 4. und 5.}}
%%\aufgabenFarbe{Aufgaben zu Grundoperationen im OLAT Aufg. [A1G] 3., 4. und 5.}
\newpage
