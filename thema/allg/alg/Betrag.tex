%%
%% 2019 07 04 Ph. G. Freimann
%%

\section{Betrag}\index{Betrag}\index{Absolutbetrag}
%%\sectuntertitel{Absolut? Abstand?}
\sectuntertitel{Wo ist negativ positiv? Beim Alkohol-Test!}
\TRAINER{Einstiegsvideos: Daniel Jung und Mathe-Mann}


%%%%%%%%%%%%%%%%%%%%%%%%%%%%%%%%%%%%%%%%%%%%%%%%%%%%%%%%%%%%%%%%%%%%%%%%%%%%%%%%%
\subsection*{Lernziele}

\begin{itemize}
  \item Symbol
  \item Bedeutung als Abstand
  \TALS{\item Gleichungen mit Betrag lösen}
\end{itemize}


\GESOTadBMTA{15}{1.2}%
\TALSTadBMTA{15}{1.2}%

\GESO{\matheNinjaLink{Betrag}{https://olat.bbw.ch/auth/RepositoryEntry/572162163/CourseNode/105951761084932}}
\TALS{\matheNinjaLink{Betrag}{https://olat.bbw.ch/auth/RepositoryEntry/572162090/CourseNode/106261459444683}}

\youtubeLink{https://www.youtube.com/watch?v=iNqygRTf2Ek}{Elementare Betragsgleichungen}

\youtubeLink{https://www.youtube.com/watch?v=yiJTCL9I-aU}{Mathe Mann/Mathe Frau: Betrag}


\begin{definition}{Betrag}{definition_betrag}\index{Betrag}\index{Absolutbetrag}
  Unter dem \textbf{Betrag} oder \textbf{Absolutbetrag} einer Zahl versteht man deren (positiven)
  \textbf{Abstand}\index{Abstand} zum Nullpunkt. Das Symbol zum Betrag sind zwei
  senkrechte Striche:\index{$\mid\cdot \mid$ s. Betrag(-striche)}\\
  $$\begin{array}{rcl}
    |-7| & := & +7\\
    | 5| & := & +5  \text{ (positive Zahl bleibt positiv)}\\
      \end{array}$$
\end{definition}
\newpage


\begin{bemerkung}{}{}
Mit $|a - b|$ wird der Abstand der Zahlen $a$ und $b$ berechnet. Ist
nämlich $b > a$, so ist die Differenz negativ und wird mit dem
$| \cdot{} |$-Symbol ins Positive gekehrt.
\end{bemerkung}

\begin{bemerkung}{}{}
  Einfach zum Merken: Dem Abstand zwischen zwei Punkten ist egal, in
  welche Richtung er gemessen wird. So liegt Zürich genauso weit von
  Bern entfernt, wie Bern von Zürich entfernt ist. Somit gilt $|a - b| = |b - a|$.
\end{bemerkung}

\GESO{%% TR GESO
  \begin{bemerkung}{}{}
  Auf dem Taschenrechner kann der Absolutbetrag mit \tiprobutton{math}
  «NUM» «abs(...» eingegeben werden.\\
  Tippen Sie:

  \tiprobutton{math} (Pfeil nach rechts) \tiprobutton{enter} 10 - 22 \tiprobutton{enter}

  Sie erhalten $|10-22| = 12$
  \end{bemerkung}
  }%% END GESO


\subsubsection{Beispiele}


\TALS{
Für welche Zahlen $x \in \mathbb{R}$ gilt

$|4| = x$ \TRAINER{$x=4$}%

$|-4| = x$ \TRAINER{$x=4$}

$|4| = -x$ \TRAINER{$x=-4$}%

$|-4| = -x$ \TRAINER{$x=-4$}

$|x| = 4$ \TRAINER{$\lx=\{4, -4\}$}%

$|x| = -4$ \TRAINER{$\lx=\{\}$}%

$|-x| = 4$ \TRAINER{$\lx=\{4, -4\}$}%

$|-x| = -4$ \TRAINER{$\lx=\{\}$}%
}%% END TALS

\GESO{
$|4| = \LoesungsRaum{4}$%

$|-4| = \LoesungsRaum{4}$%

  aber:
  
$-|4| = \LoesungsRaum{-4}$%

$-|-4| = \LoesungsRaum{-4}$%

$|8-5| = \LoesungsRaum{3}$%

$|5 -8| = \LoesungsRaum{3}$%

Achtung:
  
  $|-5-8| = \LoesungsRaum{13}$

  $|5+8| = \LoesungsRaum{13}$

  
${\big|}|6| - |-10|{\big|} = \LoesungsRaum{4}$%

}%% END GESO

Theorieaufgabe:
$$\big\vert 7 - \left\vert -3 \right\vert \big\vert - |-7-3|$$

\TNT{2.4}{%%
  $ = | 7 - (3) | - | (-7 - 3)|$ \\
  $ =| 4 |       - | -10 |$ \\
  $ = 4 - (+10)$ \\
  $= -6$
}%% END TNT


\subsection*{Aufgaben}
\GESO{\olatLinkArbeitsblatt{Betrag [A1B]}{https://olat.bbw.ch/auth/RepositoryEntry/572162163/CourseNode/105796974541159}{1.}}%% END olatLinkArbeitsblatt
\TALS{\olatLinkArbeitsblatt{Betrag [A1B]}{https://olat.bbw.ch/auth/RepositoryEntry/572162090/CourseNode/105796974602793}{1.}}%% END olatLinkArbeitsblatt


\subsubsection{Betragsgleichungen \GESO{ (Optional)}}

  Für welche $x$ gilt folgendes:
  $$|x - 3| = 8$$

\TNT{6}{%%
  Erste Lösung: Welche Zahlen haben von 3 den Abstand 8? Lsg.:
  11 und -5. Formal: 1. Fall $x-3 > 0$, dann ist $x-3=8$ und somit
  $x_1=11$; 2. Fall $x-3 <=0$, dann ist $-(x-3)=8$ (i.\,a.\,W.: $(x-3)=-8$) und somit $x_2=-5$
}% END TNT


\subsection*{Aufgaben}
\GESO{\olatLinkArbeitsblatt{Betrag
    [A1B]}{https://olat.bbw.ch/auth/RepositoryEntry/572162163/CourseNode/105796974541159}{2.,
3. }}%% END olatLinkArbeitsblatt
\TALS{\olatLinkArbeitsblatt{Betrag
    [A1B]}{https://olat.bbw.ch/auth/RepositoryEntry/572162090/CourseNode/105796974602793}{2.,
3.}}%% END olatLinkArbeitsblatt

\GESO{\subsubsection{Taschenrechner}
  Unter \tiprobutton{math} verbrirgt sich die Funktion \texttt{abs()},
  mit welcher Betragsterme berechnet werden können.
  }%% END GESO
%%Weitere Aufgabe im Buch:
%%\TALSAadBMTA{9}{5. - 7., 14., 16.-18.}
%%\GESOAadBMTA{23}{14. a) b) c) f), 15. a), 17. a) e) und 18. a)}

\olatLinkGESOKompendium{1.2}{6}{3., 4.}

\newpage
