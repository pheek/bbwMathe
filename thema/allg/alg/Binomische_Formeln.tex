%%
%% 2019 07 04 Ph. G. Freimann
%%
\newpage
\section{Binomische Formeln}\index{Formeln!binomische|textbf}\index{Binomische Formeln|textbf}

\sectuntertitel{2: Zwei, bi-, di-, zwie-, doppel, sekund-, duo, dual,
sekund-, paar, Boole'sch, stereo, binär, ...\footnote{Zürich unterscheidet die Zahl
2 je nach Geschlecht: «zwoo Fraue», «zwéé Mane» und «zwäi Chinde»}}

\youtubeLink{https://www.youtube.com/watch?v=nSmlfe-ftTo}{Binomische
Formeln (10:37)}
\youtubeLink{https://www.youtube.com/watch?v=zYVY0nmGnbE}{Binomische
Formeln Daniel Jung (4:07)}
        

%%%%%%%%%%%%%%%%%%%%%%%%%%%%%%%%%%%%%%%%%%%%%%%%%%%%%%%%%%%%%%%%%%%%%%%%%%%%%%%%%
\subsection*{Lernziele}

\begin{itemize}
\item Drei binomische Formeln
\item Wichtiger Verterter: $1-x^2 = \LoesungsRaumLang{(1+x)\cdot{}(1-x)}$

%%  \item Polynomdivision

\end{itemize}

%% \TALSTadBFWA{16}{1.3.1}
\TadBMTA{29}{2.2.1}
%%\TALSTadBMTA{29}{2.2.1}


\newpage


\subsection{Binomische Formeln}\index{Formeln!binomische}

\TRAINER{selbständig:}
$$(a-b)^2 = $$
\TNT{4}{
$$= (a-b)\cdot{}(a-b) = a^2 + a\cdot{}(-b) - ba - b\cdot{}(-b) = a^2 - 2ab + b^2$$
  \vspace{2cm}
}%% END TNT

Daraus folgen die drei binomischen Formeln
\begin{gesetz}{Binomische Formeln}{}
$$(a+b)^2 = a^2 + 2ab +b^2$$
\vspace{0.01mm}
$$(a-b)^2 = a^2 - 2ab +b^2$$
\vspace{0.2mm}
$$(a+b)(a-b) = a^2 - b^2$$
\end{gesetz}


Graphischer Beweis der 1. binomischen Formel:

%% Für Millimeterpapier siehe auch hier:
%%     http://www.texample.net/tikz/examples/graph-paper/


\TNT{5.2}{%
\raisebox{-1cm}{\includegraphics[width=12cm]{allg/alg/img/a_plus_b_zum_quadrat.png}}%
}%% END TNT

\newpage

\begin{beispiel}{}{}

Typische Anwendungen der binomischen Formeln:

$$(x+5)\cdot(x-5)=\LoesungsRaum{x^2 -25}$$
$$(x+3)^2=\LoesungsRaum{x^2 +6x + 9}$$
$$(t+2s)\cdot(t-2s)=\LoesungsRaum{t^2 - 4s^2}$$
$$(x-2)\cdot(x-2)=\LoesungsRaum{x^2 -4x + 4}$$
\end{beispiel}

\newpage

%Mit diesem Wissen lassen sich einige Summen einfacher ausklammern:
%\vspace{4cm}
%\TRAINER{
%  $$49 - x^2 = (7-x)\cdot(7+x)$$
%  $$a^2 -16a + 64 = (a-8)^2$$
%}

\TALS{
\subsection{Pascalsches Dreieck (optional)}\index{Pascalsches
  Dreieck}\index{Dreieck!Pascalsches}

Berechnen Sie $(a+b)^4$:

\TNT{6}{$$[a^2 + 2ab + b^2]^2$$ $$a^4 + 4a^3b + 6a^2b^2 + 4ab^3 + b^4$$
\vspace{5cm}}%% END TNT

Zeichnen Sie das Pascalsche Dreieck:
\begin{center}
\begin{tabular}{cccccccccccc}\\
  &   &     &     &     & 1   &     &     &     &     &    & \\
  &   &     &     & 1   &     & 1   &     &     &     &    & \\  
  &   &     & 1   &     & 2   &     & 1   &     &     &    & \\
  &   &  1  &     & 3   &     & 3   &     & 1   &     &    & \\
  & 1 &     & ... &     & ... &     & ... &     & ... &    & \\ 
1 &   & ... &     & ... &     & ... &     & ... &     & ...&  
\end{tabular}
...
%\noTRAINER{\vspace{3cm}}
\end{center}
%%\newpage
}
\newpage

\subsection*{Aufgaben}
%% \TALSAadBMTA{16}{22a) b) e)}
%% \TALSAadBMTA{17}{23a) 24a)}
%% \GESOAadBMTA{36ff}{ 19. b) d) h) 20. a) b) c) d) e) 21. a) 23. a)
%% 24. a) 25. b) e) Optional: 27. c)}
\GESO{\olatLinkArbeitsblatt{Binomische Formeln [A1Bi]}{https://olat.bms-w.ch/auth/RepositoryEntry/6029794/CourseNode/105796980058302}{1.) und 2.)}}

\TALS{\olatLinkArbeitsblatt{Binomische Formeln [A1Bi]}{https://olat.bms-w.ch/auth/RepositoryEntry/6029786/CourseNode/105796978456639}{1.) und 2.)}}

%%\aufgabenFarbe{Aufgabenblatt im OLAT zu Algebra 1: Binomische
%%Formeln. Aufgaben A1Bi: 1) und 2)}

\olatLinkGESOKompendium{1.3}{7}{9}


\TNTeop{Falls Zeit, hier Platz für den graphischen Beweis der

\textbf{3. Binomischen Formel:}

\bbwCenterGraphic{17cm}{allg/alg/img/DritteBinomische.png}
}%% end TNT EOP
