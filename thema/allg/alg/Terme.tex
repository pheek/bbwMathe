%%
%% 2019 07 04 Ph. G. Freimann
%%

\section{Terme}\index{Terme}
\sectuntertitel{Römische Bäder?}

%%%%%%%%%%%%%%%%%%%%%%%%%%%%%%%%%%%%%%%%%%%%%%%%%%%%%%%%%%%%%%%%%%%%%%%%%%%%%%%%%
\subsection*{Lernziele}

\begin{itemize}
 \item Termanalyse (Summe, Differenz, Produkt, Quotient, Potenz)
 \item Hierarchie der Terme (Vorrangregeln)
 \item Termumformungen
\end{itemize}

%%\TALSTadBFWA{10}{1.1.3}
\TadBMTA{17}{1.3}
%%\TALSTadBMTA{17}{1.3}

\newpage

\subsection{Term-Definition}\index{Term}


Generell werden die fünf wichtigsten Termarten in die folgenden drei Kategorien eingeteilt:
\begin{itemize}
\item \LoesungsRaumLang{Potenz}
\item \LoesungsRaumLang{Produkt} und \LoesungsRaumLang{Quotient}
\item \LoesungsRaumLang{Summe} und \LoesungsRaumLang{Differenz}
\end{itemize}




\begin{definition}{Term}{definition_term}
  Ein \textbf{Term} ist entweder
  \begin{itemize}
  \item ein Atom: eine Zahl (\zB{} $4.86$), eine Konstante\index{Konstante}
    ($\pi$, $\e$, ...) oder  eine Variable (\zB{}   $x$, $r$)
  \item ein Klammerausdruck \fbox{({\color{ForestGreen}T})} 
%%  \item weitere \TALS{Monome\index{Monom}}\GESO{Summanden (Teile einer Summe)}:
%%    \begin{itemize}
   \item eine \textbf{Potenz}\index{Potenz}\footnote{Als Exponent darf
     ein beliebiger Term eingesetzt werden, wohingegen als Basis
     lediglich Atome, Klammerausdrücke und Wurzelterme verwendet
     werden dürfen, wegen der Verwechslungsgefahr. Bei
     ${\frac{a}{b}}^c$ ist nämlich nicht klar, wohin das $c$ denn
     gehört.}
     \fbox{${\color{ForestGreen}T_1}^{\color{ForestGreen}T_2}$} \zB{}
     $a^3$, $(2a - 4)^{x+2}$ (inkl.  Wurzeln \fbox{$\sqrt{\color{ForestGreen}T}$})\footnote{Dabei wird der
  horizontale Strich wie eine Klammer aufgefasst.}
    \item  ein Bruchterm\footnote{Wie bei der Wurzel, dient der Bruchstrich als Klammerpaar: $\frac{U}{V}=(U):(V)$} (\fbox{$\frac{\color{ForestGreen}T_1}{\color{ForestGreen}T_2}$} \zB{} $\frac{2^x}{x^2}$)
    \item ein implizites \textbf{Produkt}\footnote{Ein implizites
      Produkt ist mit Koeffizienten angereicherter Ausdruck
      \textbf{ohne} Multiplikationszeichen.} (\zB{}
      ${\color{red}4a}{\color{ForestGreen}T}$)\footnote{Tritt eine Zahl auf,
      so ist diese immer ganz links zu schreiben. Zahlen rechts von
      Ausdrücken werden mit einem Multiplikationszeichen ($\cdot$) versehen: $5x$, aber $x\cdot{}5$.}
    \item ein explizites \textbf{Produkt}\index{Produkt} ($\cdot$;  ${\color{ForestGreen}T_1}\cdot {\color{ForestGreen}T_2}$ \zB{} $a\cdot(-1)$) bzw. ein expliziter \textbf{Quotient}\index{Quotient} ($:$, $/$, $\div$; $6a\cdot3b$ bzw. ${\color{ForestGreen}T_1}:{\color{ForestGreen}T_2}$ \zB{} $36m^2:12m^2$)
%%    \end{itemize}
  \item eine \textbf{Summe}\index{Summe} (bzw. \textbf{Differenz}\index{Differenz}) von \TALS{Monomen}\GESO{Summanden\index{Summand}}
  (Zum Beispiel bilden die folgenden
  vier «Pakete» eine Differenz):\\
  $-4x^2 + \frac{3a+b}{x} + \sqrt{5y^2-6} - 5t:2t$

    \end{itemize}
(In obiger Aufzählung hat der am höchsten stehende Term die größte
«Bindungskraft». Beispiel «Punkt vor Strich».)
\end{definition}

\textbf{Gegenbeispiele}
Keine Terme sind \zB{}: $x \cdot{}-8$, $\sqrt{+^2}$, $\frac{7}{\mathbb{N}}$, $\frac{(a+b}{-c-d)}$.


\newpage

\subsection{Vorrangregeln}\index{Vorrangregeln}

Es gilt Punkt vor Strich. Daneben bindet ein Exponent (\zB $5^8$) noch
stärker. Am stärksten binden Klammern oder horizontale Linien
(Bruchstrich, Wurzelzeichen).

\bbwCenterGraphic{8cm}{allg/alg/img/Klapopustri.png}
\begin{center}
  Das \textit{Klapopustri}\index{Klapopustri} meint dazu:

  \textbf{Klammern} vor \textbf{Potenzen} vor \textbf{Punkt} vor
  \textbf{Strich}
  
\end{center}




Beispiel:
$$-10^4 = \LoesungsRaumLang{-(10^4) = -
  (10\cdot{}10\cdot{}10\cdot{}10) = -10\,000}$$


%%\subsubsection{Terme benennen und berechnen}\index{Terme!berechnen}\index{Terme!benennen}
%%Nicht jeder Term, der ein Pluszeichen enthält, ist automatisch eine Summe.


%Teilen Sie die folgenden Terme in die Kategorie «Summe/Differenz»,
%«Produkt/Quotient» und «Potenz/Wurzel» ein. Tipp: Setzen Sie vorab
%«unnötige» Klammern und Multiplikationspunkte:


%%\renewcommand{\arraystretch}{2}
%\begin{bbwFillInTabular}{c|c|c}
%  Term                       & mit Klammern                        & Zuordnung\\\hline
%  $3x^{4-a} - y^{b+2}$ & \LoesungsRaumLang{$\left(3\cdot{}\left(x^{(4-a)}\right)\right) - \left(y^{(b+2)}\right)$}  &  \LoesungsRaum{Differenz} \\\hline
%  $ax + 2b$ & \TRAINER{$(a\cdot{}x) + (2\cdot{}b)$} & \TRAINER{Summe}\\\hline
%  $\frac{5+x}{5x}$ & \TRAINER{$\frac{(5+x)}{(5\cdot{}x)}$} & \TRAINER{Quotient}\\\hline
%  $\sqrt{2x^3+5}$ & \TRAINER{$\sqrt{((2\cdot{}(x^3)) + 5)}$} & \TRAINER{Wurzelterm}\\\hline
%  $(a+b)^{c+d}$ & \TRAINER{$(a+b)^{(c+d)}$} & \TRAINER{Potenz}\\\hline
%  $(5-3y)c^8$ & \TRAINER{$(5-3\cdot{}y)\cdot{}(c^8)$} & \TRAINER{Produkt}\\\hline
%  $(\sqrt{x-3}+\sqrt{8-b})^2$ & \TRAINER{$((\sqrt{(x-3)})+(\sqrt{(8-b)}))^2$} & \TRAINER{Potenz}\\\hline
%  $2\cdot{}7^{3-y}$ & \TRAINER{$2\cdot{}\left(7^{(3-y)}\right)$} & \TRAINER{Produkt}\\\hline
%$\frac15-2\cdot{}4^{x+1}$ & \TRAINER{$\frac15-\left(2\cdot{}\left(4^{(x+1)}\right)\right)$} & \TRAINER{Differenz}\\\hline
%\end{bbwFillInTabular}



\subsection*{Aufgaben}
%\AadBMTA{23ff}{21. und 22.}

\GESO{\olatLinkGESOKompendium{1.1}{6}{1}}

\TALS{\olatLinkArbeitsblatt{Terme
    [A1Te]}{https://olat.bbw.ch/auth/RepositoryEntry/572162090/CourseNode/106261488967281}{1. und
    2.}}%% TALS
\GESO{\olatLinkArbeitsblatt{Terme
    [A1Te]}{https://olat.bbw.ch/auth/RepositoryEntry/572162163/CourseNode/106261488881900}{1. und
2.}}%% GESO



%%\renewcommand{\arraystretch}{2}
\newpage
%%
\TALS{Compilerbau für Applikationsentwickler (optional)}
\TALS{
\TNTeop{
Im Compilerbau (Schreiben einer Programmiersprache) werden in etwa die
folgenden Vorrangregeln verwendet:

\begin{itemize}
\item Term :== Summand \{'+'|'-' Summand\}*\\
\TRAINER{$4a^3 - 6\cdot az^{(7+b)} : \sin(30)$}

\item Summand :== ExpilziterFaktor \{'$\cdot$'|'/' ExpliziterFaktor\}*\\
\TRAINER{$4a^3$, $6 \cdot az^{(7+b)} : \sin(30)$}

\item ExpliziterFaktor :== Faktor \{Faktor\}*\\
\TRAINER{$4a^3$, $6$, $az^{(7+b)}$, $\sin(30)$}

\item Faktor :== SkalarOderKlammerausdruck \{${\,}^{Term}$\}?\\
\TRAINER{$4$, $a^3$, $6$, $a$, $z^{(7+b)}$, $\sin(30)$}

\item SkalarOderKlammerausdruck :== Zahl |
           Variable |
           $\sqrt[Term]{Term)}$ | 
           '(' Term ')' |
           $\frac{Term}{Term}$|
           \textit{Funktionsname} '(' Term ')'\\
\TRAINER{$4$, $a$, $3$, $6$, $a$, $z$, $(7+b)$, $\sin(30)$}

\item \textit{Funktionsname} := 'sin', 'cos', 'tan', 'log', 'lg', 'ln', ....
\end{itemize}
}%% END TNTeop
}%% END TALS
%%

\TALS{\newpage
\textbf{Achtung} Bei zusammengeschriebenen Faktoren (\zB $ab$) bindet
           die Multiplikation stärker als beim expliziten verwenden
           des Multiplikationszeichens (\zB $a\cdot{}b$). Beispiel
           $a\cdot bm = a\cdot (b\cdot m)$.

Gleich ein Beispiel, wo dies eine Rolle spielt:
$$111x : 37x = (111x) : (37x) = 3$$
Aber
$$111\cdot x : 37\cdot x = ((111 \cdot x) : 37) \cdot x = 3x^2$$
\newpage
}%% END TALS

%%%%%%%%%%%%%%%%%%%%%%%%%%%%%%%%%%%%%%%%%%%%%%%%%

\subsection{Terme mit Namen}\index{Terme!mit Namen}\index{Werte in Terme einsetzen}\index{Terme!Werte einsetzen}
Oft gibt man Termen Namen, um sie einfacher identifizieren und
bezeichnen zu können. So könnte \zB die Oberfläche einer
Konservendose mit $A$ (Area) wie folgt bezeichnet werden, wenn $r$ den
Radius bzw. $h$ die Höhe bezeichnen:

$$A(r; h) = r^2\pi + r^2\pi + 2r\pi{}h$$

Dabei ist $A$ der Name des Terms und $r$ bzw. $h$ sind die Parameter.
\vspace{3mm}
\begin{beispiel}{Werte einsetzen}{beispiel_terme_werte_einsetzen}
  Wir betrachten den Term

  $$T({\color{red}a}; {\color{blue}x}) = 5{\color{red}a}{\color{blue}x} - {\color{red}a} + 7.$$

  Nun gilt, dass für jeden Parameter im Term (hier ${\color{red}a}$
  bzw. ${\color{blue}x}$) jede Zahl eingesetzt
  werden kann.\leserluft{}

  $$T({\color{red}2}; {\color{blue}-3}) = \LoesungsRaumLang{5\cdot{}{\color{red}2}\cdot{\color{blue}(-3)} - {\color{red}2} + 7}$$

  Es können auch Terme anstelle der Parameter eingesetzt
  werden:\leserluft{}

  $$T({\color{red}z-4}; {\color{blue}2y}) =
  \LoesungsRaumLang{5 \cdot{} {\color{red}(z-4)} \cdot {\color{blue}(2y)} - {\color{red}(z-4)} + 7}$$
\end{beispiel}

\begin{gesetz}{Einsetzen}{}
  Beim Einsetzen eines Terms in einen
  Parameter (= Variable des Terms) sind \textbf{immer} Klammern zu setzen!

  Ist
  
  $$ T(x) = x^4$$
  so ist
  
  $$ T(3-a) = (3-a)^4.$$
  
Die Klammern dürfen nur dann weggelassen werden, wenn sich der Wert
des Terms beim Weglassen der Klammern nicht ändert!
\end{gesetz}
\newpage

\subsubsection{Übungsbeispiel}
$$T(b; y) = 7y^2 - 4by$$

Berechnen Sie:


$T(\underbrace{{\color{blue}-2}}_b; \underbrace{{\color{ForestGreen}-3}}_y) = $%%
\noTRAINER{.......................................................}%%
\TRAINER{$7\cdot (\underbrace{{\color{ForestGreen}-3}}_{y})^2 - 4\cdot (\underbrace{{\color{blue}-2}}_b) \cdot (\underbrace{{\color{ForestGreen}-3}}_y) = 39$}




$T(\underbrace{{\color{blue}s}}_b; \underbrace{{\color{ForestGreen}-t}}_y) = $%%
\noTRAINER{.......................................................}%%
\TRAINER{$7\cdot (\underbrace{{\color{ForestGreen}-t}}_{y})^2 - 4\cdot (\underbrace{{\color{blue}s}}_b) \cdot (\underbrace{{\color{ForestGreen}-t}}_y) = 7t^2+4ts$}


$T(x; 2b) = $ \noTRAINER{.........................................................}\TRAINER{$7(2b)^2 - 4\cdot (x) \cdot (2b) = 28b^2 - 8bx =
  4b (7b-2x)$}



\subsection*{Aufgaben}
\TALS{\olatLinkArbeitsblatt{Terme
    [A1Te]}{https://olat.bbw.ch/auth/RepositoryEntry/572162090/CourseNode/106261488967281}{3.,
    4., 5. und 6.}}%% TALS
\GESO{\olatLinkArbeitsblatt{Terme
    [A1Te]}{https://olat.bbw.ch/auth/RepositoryEntry/572162163/CourseNode/106261488881900}{3.,
    4. und 5. \TRAINER{Aufgabe 6. nur TALS oder als Option}}}%% GESO

\newpage
