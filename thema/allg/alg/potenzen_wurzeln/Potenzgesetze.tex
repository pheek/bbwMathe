\newcommand{\aaaa}{a\cdot a \cdot a \cdot{} ... \cdot a}
\newcommand{\bbbb}{b\cdot b \cdot b \cdot{} ... \cdot b}

\newpage
\subsection{Potenzen, Definitionen und Gesetze}\index{Potenz}

\subsubsection{Einführungsbeispiele}

Berechen Sie
$$5^3$$

\TNT{4}{125\vspace{16mm}}

\hrule

Vereinfachen Sie:

$$ca^5\cdot(ab)^3\cdot{}(2c)^{1+3}$$

\TNT{4}{$$16a^8b^3c^5$$\vspace{22mm}}

\hrule

Die sieben verschiedenen schweizer Münzen werden nacheinander geworfen
und es wird notiert, welcher Wurf Zahl oder «Kopf» aufweist. Dieses
Experiment wird einige Male wiederholt, bis man sich die Frage stellt:
Wie viele mögliche Ausgänge hat das Experiment?

\TNTeop{$$2^7 = 128$$\vspace{22mm}}


%%%%%%%%%%%%%%%%%%%%%%%%%%%%%%%%%%%%%%%%%%%%%%%%%%%%%%%%%

\begin{definition}{Potenz}{}\index{Potenz}
Unter der $n$-ten \textbf{Potenz} verstehen wir eine $n$-malige Multiplikaiton mit demselben Faktor:
$$a^2 = a\cdot a$$
$$a^n = 1\cdot{} \underbrace{\aaaa}_{n\, \text{Faktoren}}$$
\end{definition}
Dabei bezeichnen

\bbwCenterGraphic{9cm}{allg/alg/potenzen_wurzeln/img/Potenzbegriff.png}

\begin{gesetz}{$a$ hoch 1}{}
  Es gilt:
  $$a^1 = a$$
\end{gesetz}


%\begin{itemize}
% \item Potenz\index{Potenz}: $a^n$
% \item Exponent\index{Exponent}: $n$
% \item Basis\index{Basis}: $a$
%\end{itemize}

\vspace{40mm}

\bbwCenterGraphic{10cm}{allg/alg/potenzen_wurzeln/img/Ivegotthepower.png}


\newpage

\subsubsection{gleiche Basis}\index{Potenz!Gesetze}


\begin{gesetz}{Produkt}{}
  $$a^m\cdot a^n = a^{m+n}$$
\end{gesetz}

   Begründung:

   \TNT{2.4}{
   \TALS{$a^m\cdot a^n =
    \underbrace{{\underbrace{\aaaa}_{m\text{-mal}}}\cdot{\underbrace{\aaaa}_{n\text{-mal}}}}_{m+n\text{-mal}}
    = a^{m+n}$}
   \GESO{$7^3\cdot{} 7^5 = (7\cdot{}7\cdot{}7)\cdot{}(7\cdot{}7\cdot{}7\cdot{}7\cdot{}7)=7^8$}%%
}%% END TNT

   \begin{gesetz}{Quotient}{}
   $$a^m : a^n = \frac{a^m}{a^n}= a^{m-n}$$
   \end{gesetz}
   
     Begründung:

     \TNT{2.4}{%
     \TALS{$a^m :  a^n =
    \underbrace{{\underbrace{\aaaa}_{m\text{-mal}}} : {\underbrace{\aaaa}_{n\text{-mal}}}}_{m-n\text{-mal}}
    = a^{m-n}$}
     \GESO{
  Kürzen: 
   $7^5 :  7^3 = \frac{7^5}{7^3} =
  \frac{7\cdot{}7\cdot{}7\cdot{}7\cdot{}7}{7\cdot{}7\cdot{}7} = 7^2 =
  7^{5-3}$}
     }%% END TNT

\begin{gesetz}{Potenzen von Potenzen}{}
$$(a^n)^m = a^{n\cdot{}m} = (a^m)^n$$
\end{gesetz}
  Begründung:

  \TNT{2.4}{Beispiel $(a^4)^3$ vorrechnen:
    $(a^4)^3 = a^4\cdot{}a^4\cdot{}a^4 =
    aaaa\cdot{}aaaa\cdot{}aaaa = a^{12} = a^{4\cdot{}3}$\vspace{12mm}}%%
  


\newpage
\subsection*{Aufgaben}

%%\AadBMTA{66ff}{5. a) c) und 6. a) b)}


Gleiche Basis:

%%\AadBMTA{69ff}{25. b) f), 29. b) f), 30. b) c) d) e), 33. e) f)}


%%Potenzen von Potenzen:

%%\AadBMTA{70ff}{39. b) d) e)}

%%  OLAT Arbeitsblatt
\GESO{\olatLinkArbeitsblatt{Potenzgesetze}{https://olat.bms-w.ch/auth/RepositoryEntry/6029794/CourseNode/102690264435484}{Kapitel
    2 (inkl. 2.1 gleiche Basis) (Aufgaben 17. - 42.) }}%% END olatLinkArbeitsblatt
\TALS{\olatLinkArbeitsblatt{Potenzgesetze}{https://olat.bms-w.ch/auth/RepositoryEntry/6029786/CourseNode/104915210426569}{Kapitel
    2 (inkl. 2.1 gleiche Basis) (Aufgaben 17. - 42)}}%% END olatLinkArbeitsblatt



\newpage

%%%%%%%%%%%%%%%%%%%%%%%%%%%%%%%%%%%%%%%%%%%%%%%%%%%%%%%%%%%

\subsubsection{gleiche Exponenten}

\begin{gesetz}{Produkt}{}
$$a^n \cdot{} b^n  = (ab)^n$$
\end{gesetz}


  Begründung

  \TNT{2.4}{
    \TALS{$a^n\cdot b^n
= \underbrace{\aaaa}_{n\text{-mal}}\underbrace{\bbbb}_{n\text{-mal}}
= \underbrace{ab\cdot ab\cdot ... \cdot ab}_{n\text{-mal}} =
(ab)^n$}%%
   \GESO{\zB $7^3 \cdot 5^3 = (7\cdot{}7\cdot{}7) \cdot{}
     (5\cdot{}5\cdot{}5) = (7\cdot{}5)\cdot{} (7\cdot{}5)\cdot{} (7\cdot{}5) =(7\cdot{}5)^3$}
}%% end TRAINER
   
\begin{gesetz}{Division}{}
  $$a^n : b^n = (a:b)^n$$
  bzw.
  $$\frac{a^n}{b^n} = \left(\frac{a}{b}\right)^n$$
\end{gesetz}
\TNT{2}{(Gilt ganz analog.)}


%%%%%%%%%%%%%%%%%%%%%%%%%%%%%%%%%%%%%%%%%%%%%%%%%%%%%%%%%%%%%%

\textbf{Aber Vorsicht}

\begin{bemerkung}{Achtung}{}
$$a^3 \cdot{}  b^4 \ne \, ???$$ \TRAINER{(\text{Keine gleiche Basis, kein
  gleicher Exponent})}
$$a^3 +        b^3 \ne \, ???$$ \TRAINER{ (\text{keine
      Punkt-Rechunung!})}
\end{bemerkung}

\newpage

\subsubsection{Theorieaufgaben}

Lösen Sie:

%\TRAINER{
%$a^4 \cdot a^5 = (a\cdot a\cdot a\cdot a) \cdot (a\cdot a\cdot a\cdot a\cdot a) = a^9 = a^{4+5}$}%%

 $-a^4\cdot (-a)^5 =\LoesungsRaum{-(a\cdot a\cdot a\cdot a) \cdot ((-a)\cdot (-a)\cdot (-a)\cdot (-a)\cdot (-a)) = +a^9}$

 $r^6 : r^4 =\LoesungsRaum{r^{6-4}=r^2}$

 $a^4\cdot b^4 =\LoesungsRaum{(a\cdot a\cdot a\cdot a) \cdot (b\cdot b\cdot b\cdot b) = (a\cdot{}b)^4 = (ab)^4}$

$p^5 : q^5 = \LoesungsRaum{(p:q)^5=\left(\frac{p}{q}\right)^5}$

$(-s^4)^3 = \LoesungsRaum{-s^{4\cdot{}3} = -s^{12}}$

Exponentenvergleich: Finde $x$:
$(r^x)^{10}\cdot{}r^{22} = r^{72}$, $x=\LoesungsRaum{5}$
\newpage

\subsection*{Aufgaben}

Gleiche Exponenten:

\GESO{\olatLinkArbeitsblatt{Potenzgesetze}{https://olat.bms-w.ch/auth/RepositoryEntry/6029794/CourseNode/102690264435484}{Kapitel
    2.2 (Aufg. 43. - 51.)  }}%% END olatLinkArbeitsblatt
\TALS{\olatLinkArbeitsblatt{Potenzgesetze}{https://olat.bms-w.ch/auth/RepositoryEntry/6029786/CourseNode/104915210426569}{Kapitel
    2.2 (Aufg. 43. - 51.)}}%% END olatLinkArbeitsblatt


Erste Exponentialgleichungen:

\GESO{\olatLinkArbeitsblatt{Potenzgesetze}{https://olat.bms-w.ch/auth/RepositoryEntry/6029794/CourseNode/102690264435484}{Kapitel
    2.3 (Aufg. 52. - 54.)  }}%% END olatLinkArbeitsblatt
\TALS{\olatLinkArbeitsblatt{Potenzgesetze}{https://olat.bms-w.ch/auth/RepositoryEntry/6029786/CourseNode/104915210426569}{Kapitel
    2.3 (Aufg. 42. - 54.)}}%% END olatLinkArbeitsblatt


%Gleiche Exponenten:

%\AadBMTA{70ff}{40. a) c) f), 42. f) g) h) i)}


%Erste Exponentialgleichungen:

%\AadBMTA{71}{44. a) b) c)}


%\olatLinkGESOKompendium{1.4}{8ff}{14., bis 18.}

%%\TALS{Aufgaben noch ohne negative Exponenten:}
%%\TALSAadBMTA{32ff}{79. a), 90. d) l), 91. l), 92. a) d), 93. j), 94. c) f), 95. a),
%%  101. a), 103. a) b) c), 105. a) g) h) und 106. f)}
\newpage
