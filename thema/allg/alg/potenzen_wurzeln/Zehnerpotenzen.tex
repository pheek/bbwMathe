\subsection{Zehnerpotenzen}\index{Zehnerpotenzen}
\TadBMTA{61}{4.5}

\bbwCenterGraphic{8cm}{allg/alg/potenzen_wurzeln/img/one_in_a_mellon.jpg}

Neben den im Buch (\cite{marthaler21alg} S. 62) angegebenen SI-Einheiten (Kilo, Mega, ...) sind die
Namen der positiven Zehnerpotenzen im Englischen anders als im
Deutschen. Hier zur Vollständigkeit:

\begin{tabular}{lrclll}
Potenz    & Zahl & SI-Kürzel & SI-Vorsätze & Deutsch & Englisch\\
\hline\\
$10^{2}$   & 100  & h         & \LoesungsRaumLen{35mm}{Hekto}       & Hundert   & hundred\\
$10^{3}$   & 1000 & k         & \LoesungsRaumLen{35mm}{Kilo}        & Tausend   & thousand\\
$10^{6}$   & ...  & M         & \LoesungsRaumLen{35mm}{Mega}        & Million   & million\\
$10^{9}$   & ...  & G         & \LoesungsRaumLen{35mm}{Giga}        & \LoesungsRaumLen{35mm}{Milliarde} & billion\\
$10^{12}$  & ...  & T         & \LoesungsRaumLen{35mm}{Tera}        & \LoesungsRaumLen{35mm}{Billion}   & trillion\\
$10^{15}$  & ...  & P         & Peta        & Billiarde & quadrillion\\
$10^{18}$  & ...  & E         & Exa         & Trillion  & quintillion\\
\end{tabular}

\begin{tabular}{lrclll}
Potenz     & Zahl & SI-Kürzel & SI-Vorsätze & Deutsch\\
\hline\\
$10^{-1}$  & 0.1   & d         & \LoesungsRaumLen{35mm}{Dezi}        & Zehntel\\
$10^{-2}$  & 0.01  & c         & \LoesungsRaumLen{35mm}{Centi}       & Hundertstel\\
$10^{-3}$  & 0.001 & m         & \LoesungsRaumLen{35mm}{Milli}       & Tausendstel\\
$10^{-6}$  & ...   & $\mu$     & \LoesungsRaumLen{35mm}{Mikro}       & Millionstel\\
$10^{-9}$  & ...   & n         & \LoesungsRaumLen{35mm}{Nano}        & Milliardstel\\
\end{tabular}

\subsection*{Aufgaben}
\GESO{\olatLinkArbeitsblatt{Potenzgesetze}{https://olat.bms-w.ch/auth/RepositoryEntry/6029794/CourseNode/102690264435484}{Kapitel 1 (Aufgabe 1.)}}%% END olatLinkArbeitsblatt
\TALS{\olatLinkArbeitsblatt{Potenzgesetze}{https://olat.bms-w.ch/auth/RepositoryEntry/6029786/CourseNode/104915210426569}{Kapitel 1 (Aufgabe 1.)}}%% END olatLinkArbeitsblatt

%\newpage
\subsubsection{Zehnerpotenzen mit Taschenrechner}
Geben Sie im Taschenrechner die Zahl $0.37$ Milliarden ein.

\TNT{4}{
  Lösung:
  
Mit dem Taschenrechner können große Zehnerpotenzen mit der
  \GESO{\tiprobutton{EE}}\TALS{\nspirebutton{EE}}-Taste eingegben
  werden: 0.37 Milliarden:

  \GESO{\tiprobutton{0}\tiprobutton{dot}\tiprobutton{3}\tiprobutton{7}\tiprobutton{EE}\tiprobutton{9}}%% END GESO
  \TALS{\nspirebutton{0}\nspirebutton{dot}\nspirebutton{3}\nspirebutton{7}\nspirebutton{EE}\nspirebutton{9}}%% END TALS

  0.37 Mia = 370\,000\,000
}%% end TNT

\subsection*{Aufgaben}
Geben Sie die Größen in der gegebenen Einheit (m, l, s, Byte) an:

\begin{bbwFillInTabular}{|c|c|}\hline
 $207$        $\mu$m    & \LoesungsRaumLen{40mm}{$0.000207$}  m\\\hline
 $0.026$          ml    & \LoesungsRaumLen{40mm}{$0.000026$}  l\\\hline
 $172$            hl    & \LoesungsRaumLen{40mm}{$17\,200 $}  l\\\hline
 $50\,000\,000$   cl    & \LoesungsRaumLen{40mm}{$500\,000$}  l\\\hline
 $1\,800$         TByte & \LoesungsRaumLen{40mm}{$1.8\cdot{}10^{15} $}  Byte\\\hline
 $0.027$          ns    & \LoesungsRaumLen{40mm}{$2.7\cdot{}10^{-11} $} s\\\hline
  
  \end{bbwFillInTabular} 
\newpage


\subsubsection{Ausklammern von Zehnerpotenzen\GESO{ (optional)}}
Gegeben ist die folgende Summe. Leider etwas mühsam zum Lesen wegen der vielen Nullen. Klammern Sie tausend (= $10^3$) aus:

$$400\,000 + 5\,000 + 3\,000\,000 + 70$$
\TNT{3.2}{
$$ = 4\cdot{} 10^5 + 5\cdot{} 10^3 + 3\cdot{} 10^6 + 7\cdot{} 10^1$$
$$ = 10^3 \cdot{} (4\cdot{}10^2 + 5 \cdot{} 1 + 3\cdot{} 10^3 + 7
  \cdot{} 0.01)$$
  $$=10^3\cdot{} (400 + 5 + 3000 + 0.07)  $$
  $$= (3405.07 \cdot{} 10^3) = 3405.07 k$$
}%% END TNT


\paragraph{Ausklammern negativer Zehnerpotenzen:}
\,

\vspace{1mm}

Genauso, wie man positive Zehnerpotenzen ausklammern kann, kann man auch negative Zehnerpotenzen ausklammern. Dies ist insofern praktisch, um sich einen Überblick zu verschaffen, bei sehr kleinen positiven Größen:

Klammern Sie einen Millionstel (=$10^{-6}$) aus:


$$a\cdot{}10^{-6} + b\cdot{}10^{-2} + c\cdot{}10^{-5} +
d\cdot{}10^{-1} = \LoesungsRaumLang{(a + b\cdot{}10^4 + 10c + d\cdot{}10^{5}) \cdot{} 10^{-6}}$$

\subsubsection{Negative Basis}

Berechnen Sie

$$(-10)^6 = \LoesungsRaumLen{60mm}{ (-10)(-10)(-10)(-10)(-10)(-10) = 100 \cdot{} 100 \cdot{} 100 = 1\,000\,000 }$$
$$-10^6   = \LoesungsRaumLen{60mm}{ -(10^6) = - (1\,000\,000) = -1\,000\,000 }$$

\subsection*{Aufgaben}

%%\TALS{Zehnerpotenzen:}
%%\TALSAadBMTA{41ff}{110. a) b) c) d), 111. c) f), 112. a) d) e) f) m) 114. und 116.}
%%\TALS{Zehnerpotenzen ausklammern:}
%%\TALSAadBMTA{34}{88. e) und h)}

%%  \AadBMTA{65ff}{3. b) c), 4. c), 12. und 16. a) und c)}
%%  \AadBMTA{68ff}{24. c), 28. c)}
%%  \AadBMTA{74ff}{52. a) c) d), \GESO{ (optional) }53. a) b) d) h) i), 57. a) b) e)}
%%  \AadBMTA{75}{60., 63. und 64.}


\GESO{\olatLinkArbeitsblatt{Potenzgesetze}{https://olat.bms-w.ch/auth/RepositoryEntry/6029794/CourseNode/102690264435484}{Kapitel 1 (Aufgaben 2. - 9. und 10. - 15.)}}%% END olatLinkArbeitsblatt
\TALS{\olatLinkArbeitsblatt{Potenzgesetze}{https://olat.bms-w.ch/auth/RepositoryEntry/6029786/CourseNode/104915210426569}{Kapitel 1 (Aufgaben 2. - 9. und 10. - 14. und 16.)}}%% END olatLinkArbeitsblatt

\newpage
