%%
%% 2019 07 04 Ph. G. Freimann
%%

\section{Potenzen}\index{Potenzen}
\sectuntertitel{Ein dreifaches Hoch auf die Basen}

\GESOTadBMTA{57-59}{4.1, 4.2, 4.3 und 4.4}

%%%%%%%%%%%%%%%%%%%%%%%%%%%%%%%%%%%%%%%%%%%%%%%%%%%%%%%%%%%%%%%%%%%%%%%%%%%%%%%%%
\subsection*{Lernziele}

\begin{itemize}
\item Namen der Zehnerpotenzen
\item Definition: Exponent, Basis, Potenz
\item Rechengesetze
\item negative Exponenten
\item Hoch 0
\item Potenzen von Bruchtermen
\end{itemize}
\TALS{(\cite{frommenwiler17alg} S. 32 (Kap. 1.5))}
\newpage
%%\subsection{Definition}
%%$$a^n = \underbrace{a\cdot a \cdot a \cdot ... \cdot a}_{\textrm{n Faktoren}}$$

\subsection{Zehnerpotenzen}\index{Zehnerpotenzen}
\TadBMTA{61}{4.5}

\bbwCenterGraphic{8cm}{allg/alg/potenzen_wurzeln/img/one_in_a_mellon.jpg}

Neben den im Buch (\cite{marthaler21alg} S. 62) angegebenen SI-Einheiten (Kilo, Mega, ...) sind die
Namen der positiven Zehnerpotenzen im Englischen anders als im
Deutschen. Hier zur Vollständigkeit:

\begin{tabular}{lrclll}
Potenz    & Zahl & SI-Kürzel & SI-Vorsätze & Deutsch & Englisch\\
\hline\\
$10^{2}$   & 100  & h         & Hekto       & Hundert   & hundred\\
$10^{3}$   & 1000 & k         & Kilo        & Tausend   & thousand\\
$10^{6}$   & ...  & M         & Mega        & Million   & million\\
$10^{9}$   & ...  & G         & Giga        & Milliarde & billion\\
$10^{12}$  & ...  & T         & Tera        & Billion   & trillion\\
$10^{15}$  & ...  & P         & Peta        & Billiarde & quadrillion\\
$10^{18}$  & ...  & E         & Exa         & Trillion  & quintillion\\
\end{tabular}

\begin{tabular}{lrclll}
Potenz     & Zahl & SI-Kürzel & SI-Vorsätze & Deutsch\\
\hline\\
$10^{-1}$  & 0.1   & d         & Dezi        & Zehntel\\
$10^{-2}$  & 0.01  & c         & Centi       & Hundertstel\\
$10^{-3}$  & 0.001 & m         & Milli       & Tausendstel\\
$10^{-6}$  & ...   & $\mu$     & Mikro       & Millionstel\\
$10^{-9}$  & ...   & n         & Nano        & Milliardstel\\
\end{tabular}

Mit dem Taschenrechner können große Zehnerpotenzen mit der
  \GESO{\tiprobutton{EE}}\TALS{\nspirebutton{EE}}-Taste eingegben
  werden: 0.37 Milliarden:

  \GESO{\tiprobutton{0}\tiprobutton{dot}\tiprobutton{3}\tiprobutton{7}\tiprobutton{EE}\tiprobutton{9}}%% END GESO
  \TALS{\nspirebutton{0}\nspirebutton{dot}\nspirebutton{3}\nspirebutton{7}\nspirebutton{EE}\nspirebutton{9}}%% END TALS


\newpage


\subsubsection{Ausklammern von Zehnerpotenzen}
Gegeben ist die folgende Summe. Leider etwas mühsam zum Lesen wegen der vielen Nullen. Klammern Sie tausend (= $10^3$) aus:

$$400\,000 + 5\,000 + 3\,000\,000 + 70$$
\TNT{3.2}{
$$ = 4\cdot{} 10^5 + 5\cdot{} 10^3 + 3\cdot{} 10^6 + 7\cdot{} 10^1$$
$$ = 10^3 \cdot{} (4\cdot{}10^2 + 5 \cdot{} 1 + 3\cdot{} 10^3 + 7
  \cdot{} 0.01)$$
  $$=10^3\cdot{} (400 + 5 + 3000 + 0.07)  $$
  $$= (3405.07 \cdot{} 10^3) = 3405.07 k$$
}%% END TNT


\paragraph{Ausklammern negativer Zehnerpotenzen:}
\,

\vspace{1mm}

Genauso, wie man positive Zehnerpotenzen ausklammern kann, kann man auch negative Zehnerpotenzen ausklammern. Dies ist insofern praktisch, um sich einen Überblick zu verschaffen, bei sehr kleinen positiven Größen:

Klammern Sie einen Millionstel (=$10^{-6}$) aus:


$$a\cdot{}10^{-6} + b\cdot{}10^{-2} + c\cdot{}10^{-5} +
d\cdot{}10^{-1} = \LoesungsRaumLang{(a + b\cdot{}10^4 + 10c + d\cdot{}10^{5}) \cdot{} 10^{-6}}$$

\subsection*{Aufgaben}

%%\TALS{Zehnerpotenzen:}
%%\TALSAadBMTA{41ff}{110. a) b) c) d), 111. c) f), 112. a) d) e) f) m) 114. und 116.}
%%\TALS{Zehnerpotenzen ausklammern:}
%%\TALSAadBMTA{34}{88. e) und h)}

%%  \AadBMTA{65ff}{3. b) c), 4. c), 12. und 16. a) und c)}
%%  \AadBMTA{68ff}{24. c), 28. c)}
%%  \AadBMTA{74ff}{52. a) c) d), \GESO{ (optional) }53. a) b) d) h) i), 57. a) b) e)}
%%  \AadBMTA{75}{60., 63. und 64.}


\GESO{\olatLinkArbeitsblatt{Potenzgesetze}{https://olat.bbw.ch/auth/RepositoryEntry/572162163/CourseNode/102690264435484}{Kapitel
    1 (Aufgaben 1. - 16.)}}%% END olatLinkArbeitsblatt
\TALS{\olatLinkArbeitsblatt{Potenzgesetze}{https://olat.bbw.ch/auth/RepositoryEntry/572162090/CourseNode/104915210426569}{Kapitel
    1 (Aufgaben 1. - 16.)}}%% END olatLinkArbeitsblatt

\newpage


%% Philipp G Freimann Juli 2019 für die BBW
%% Phi BBW-Vorlage für Arbeitsblätter (LaTeX)
%% 2019 - 08 - 18

%% %% %% %%
\documentclass[twoside,12pt,a4paper]{article}%%
\usepackage[paper=a4paper,margin=17mm]{geometry}%%


%% Zentralisiert
\input{inputs/bbwUsePackages}
\input{inputs/bbwLayout}
\input{inputs/bbwMakros}
\input{inputs/matheMakros}
%%

\usepackage{bbwLayoutPageSty}


%%%%%%%%%%%%%%%  H E A D E R   &   F O O T E R %%%%%%%%%%%%%%%%%%%%
%% Headers
\fancyhf[HL]{\makebox{\includegraphics[width=30mm]{logos/bbw.pdf}}}%%
\fancyhf[HC]{\metaHeaderLine{}}%%
\fancyhf[FR]{\tiny{\shortAuthor{} (\today{})}}%%

\newcommand{\arbeitsblattHeader}{%%
  \begin{center}%%
    {\Large \fontfamily{qhv}\selectfont \arbeitsblattTitel{}}%%
\end{center}}%%

\renewcommand{\bbwAufgabenBlockID}{APot}


%%%%%%%%%%%%%%%%%%%%%%%%%%%%%%%%%%%%%%%%%%%%%%%%%%%%%%%%%%%%%%%%%%

\usepackage{amssymb} %% für \blacktriangleright
\renewcommand{\metaHeaderLine}{Potenzgesetze}
\renewcommand{\arbeitsblattTitel}{(BMS Version 1.1)}

\begin{document}%%
\arbeitsblattHeader{}


%\newcounter{aufgabennummer}
%\setcounter{aufgabennummer}{1}

\newcommand\aufgabeML[3]{

Aufgabe \arabic{bbwAufgabenNummerCounter}. :\,\,
$${#2}$$ = $\LoesungsRaum{{#3}}$

\abplz{#1}

\stepcounter{bbwAufgabenNummerCounter}
}

{\huge{Vermischte Aufgaben zu Potenzgesetzen}}


\section{Zehnerpotenzen}

Berechnen Sie Zehnerpotenzen und vergleichen Sie:

\aufgabeML{6}{(-10)^4}{10\,000}
\aufgabeML{6}{-10^4}{-10\,000}


\aufgabeML{6}{(-10)^5}{-100\,000}
\aufgabeML{6}{-10^7}{-10\,000\,000}
\aufgabeML{6}{(-10)^8}{+100\,000\,000}
\aufgabeML{6}{-(10^6)}{-1\,000\,000}


\aufgabeML{6}{0.1^1}{0.1}
\aufgabeML{6}{0.1^2}{0.01}
\aufgabeML{6}{-0.1^4}{-(0.1^4) = -0.0001}
\newpage

Aufgabe \arabic{bbwAufgabenNummerCounter}.: 
Füllen Sie die Tabelle aus. Tipp: Welche Operation wird in den
hinteren drei Spalten konsequent von einer Zeile zur nächsten ausgeführt?

\begin{bbwFillInTabular}{|r|r|r|l|}\hline
Exponent       & Zehnerpotenz & Potenzwert & Name \\\hline
$4$            &  $10^4$           &  $10\,000$        & Zehntausend         \\\hline
$3$            &  \TRAINER{$10^3$} &  \TRAINER{$1000$} & \TRAINER{Tausend}   \\\hline
$2$            &  \TRAINER{$10^2$} &  \TRAINER{$100$}  & \TRAINER{Hundert}   \\\hline
$\TRAINER{1}$  &  \TRAINER{$10^1$} &  \TRAINER{$10$}   & \TRAINER{Zehn}      \\\hline
$\TRAINER{0}$  &  \TRAINER{$10^0$} &  \TRAINER{$1$}    & \TRAINER{Eins}      \\\hline
$\TRAINER{-1}$ &  \TRAINER{$10^{-1}$} &  \TRAINER{$0.1$}  & \TRAINER{Ein Zehntel}      \\\hline
$-2$           &  \TRAINER{$10^{-2}$} &  \TRAINER{$0.01$}	 & \TRAINER{Ein Hundertstel}      \\\hline
\end{bbwFillInTabular}
\stepcounter{bbwAufgabenNummerCounter}


Schreiben Sie ohne Klammern und ohne Bruchstrich:
\aufgabeML{6}{\frac1{10\,000}}{10^{-4}}

Fassen Sie die Zehnerpotenzen zusammen und schreiben Sie
wissenschaftlich (genau eine Ziffer vor dem Dezimalpunkt):

\aufgabeML{6}{9\cdot{}10^3 + 4.7 \cdot{}10^4}{5.6\cdot{} 10^4}
\aufgabeML{6}{2\cdot{}10^{-4} + 220.3 \cdot{} 10^-5}{2.403 \cdot{} 10^{-3}}



Multiplizieren Sie die Zehnerpotenzen (gleiche Basis):
\aufgabeML{6}{-0.1^4\cdot{} 0.1^5}{-0.1^{9} = -0.000\,000\,001}

\arabic{bbwAufgabenNummerCounter}.:
Ein rotes Blutkörperchen hat ein Gewicht von $3\cdot{}10^{-11}$ g.
Beim Menschen liegt die Konzentration im Blut bei $4\,000$ bis $5\,900$ Stück pro Nanoliter.

\begin{bbwAufgabenBlock}
\item Geben Sie die Anzahl der roten Blutkörperchen je im Minimum und
im Maximum pro Liter Blut mit Hilfe von Zehnerpotenzen
an.\TRAINER{Minimal: $4000\cdot{}10^9 = 4\cdot{}10^{12}$, maximal:
$5.9\cdot{}10^{12}$}

\item Wie viele solcher Blutkörperchen hat ein Mensch, wenn wir von
einer Konzentration von 5\,000 Stück pro Nanoliter ausgehen und von
einer Blutmenge von 6 Litern? Wie heißt die Zahl in Worten?
\TRAINER{$6\cdot{} 5.0\cdot{}10^{}12 = 3.0 \cdot{} 10^{13}$ = 30
Billionen}

\item Wie groß ist die Masse aller roten Blutkörperchen einer Person,
wenn wir von 6 Liter Blut und 5\,000 roten Blutkörperchen pro
Nanoliter ausgehen?
\TRAINER{$3.0\cdot{}10^{13} \cdot{} 3\cdot{} 10^{-11} \text{ g } = 9\cdot{}10^2 \text{ g } = 900 \text{ g }$}
\end{bbwAufgabenBlock}

%%%%%%%%%%%%%%%%%%%%%%%%%%%%%%%%%%%%%%%%%%%%%%%%%%%%%%%%%%%%%%%%%%%%%%%%%%%%%%%%%%%%%%%%%%%%%%%%%%

\section{Ganze positive Exponenten}

\aufgabeML{6}{-\left((-a)^3\right)^{10}}{-a^{30}}\noTRAINER{\newpage}

\aufgabeML{6}{\left(-(-x^2)\right)^7}{x^{14}}


\aufgabeML{6}{\left(-x^3\right)^4}{x^{12}}
%%\platzFuerBerechnungenBisEndeSeite{}

\newpage
\section{Division}

\aufgabeML{6}{c^9 : c^4}{c^5}

\aufgabeML{6}{\frac{a^2}{a^3}}{\frac{1}{a}=a^{-1}}\noTRAINER{\newpage}

\aufgabeML{6}{(-a)^4:(-a^{10})}{-\frac{1}{a^6}}

\aufgabeML{6}{a\cdot{} a^{2x+3} : a^{1-x}}{a^{3x+3}}\noTRAINER{\newpage}

\aufgabeML{6}{\left(\frac{x}{y}\right)^7:\left(\frac{-x}{y}\right)^3}{-\left(\frac{x}{y}\right)^4}\noTRAINER{\newpage}

\newpage
\section{negative Exponenten und die Null}

\aufgabeML{6}{-\left(-(-a)^2\right)^0}{-1}

\aufgabeML{6}{-\left( -a^3\cdot{} (-a)^2 \right)^{-4}}{\frac{-1}{a^{20}} = -a^{-20}}\noTRAINER{\newpage}

\aufgabeML{6}{(-b)^{-6} \cdot (-b^8)}{-b^2}

\aufgabeML{6}{\left(-(a^{-1})^{-2}\right)^6}{a^{12}}\noTRAINER{\newpage}

\aufgabeML{6}{-\left((a^3)\cdot{}a^{-1}\right)^2}{-a^4}
\noTRAINER{\newpage}

Schreiben Sie ohne negative Exponenten:

\aufgabeML{6}{2\cdot{}x^{-3}}{\frac{2}{x^3}}

\aufgabeML{6}{ab^{-5}}{\frac{a}{b^5}=a\cdot{}\frac{1}{b^5}}\noTRAINER{\newpage}

\aufgabeML{6}{\frac{1}{81}\cdot{}c^{-2}\cdot{}a^{-3}\cdot{}c^2\cdot{} a^3 \cdot 3^4}{1}
%%\platzFuerBerechnungenBisEndeSeite{}

\newpage
\section{negative Exponenten mit Brüchen}
\aufgabeML{6}{\frac{6\cdot{} a^3 \cdot{} b^7 \cdot{} 3}{(ab)^4\cdot 9 \cdot a^{-1}}}{2b^3}

\aufgabeML{6}{\left(\frac{a^2 b^{-3} c^3}{a\cdot b}\right)^{-3}\cdot \left(\frac{c^5}{ab}\right)^2}{\frac{b^{10}c}{a^5}}\noTRAINER{\newpage}

\aufgabeML{6}{\left(\frac{ab^{-2}c^3}{a^{-2}\cdot{}b}\right)^{-2} \cdot{} \left(\frac{a^3\cdot{} c^{2}}{b^2} \right)^3}{a^3}


\aufgabeML{6}{\left(\frac{a^3 \cdot{}c^{-1}}{b^{-2}}\right)^{-4} \cdot \left( \frac{a^4\cdot{}b^3}{c^{-2}} \right)^3}{bc^{10}}\noTRAINER{\newpage}

\aufgabeML{6}{\frac{a^{-5}\cdot{} (-a)^3}{a^7} : \frac{a^{-10}}{a^4} }{-a^5}

\aufgabeML{6}{\frac{ab\cdot{}b^{-2}\cdot{}b^4\cdot{}b^{-3}}{b^3\cdot{}a\cdot{}b^{-4} \cdot{} b}}{1}\noTRAINER{\newpage}


\aufgabeML{6}{\left(\frac{3x^{-2}y^2}{4x^{-4}\cdot{}y^3}\right)^{-2} : \left(\frac{2x^{-1}}{3xy^{-2}}\right)^3}{\frac{6x^2}{y^4}}

\aufgabeML{6}{4^k\cdot{} \left(\frac{1}{2}\right)^k \cdot{} \left(\frac{1}{3}\right)^{-k}}{6^k}\noTRAINER{\newpage}

\aufgabeML{6}{\frac{a^{-2}}{a^{-3}}}{a}

\aufgabeML{6}{\left(\frac{a^4\cdot{}b^{-2}\cdot{}c}{a^2\cdot{}c^{-3}}\right)^{-2} \cdot \left(\frac{c^2\cdot{}b}{a^{-1}}\right)^4}{b^8}\noTRAINER{\newpage}


\aufgabeML{6}{\left(\frac{3a^{-1}\cdot{} b^2}{2ac^{-1}}\right)^{-2} \cdot{} \frac{\left(3\cdot{}b^2\right)^2\cdot{}c^2}{4}}{a^4}

\aufgabeML{6}{a^{-1}\cdot{} a^{-2} : a^{-3} \cdot{} a^7}{a^7}\noTRAINER{\newpage}

\newpage
\section{Wurzeln}

\aufgabeML{6}{\sqrt{a^2}}{a}

\aufgabeML{6}{\left(\sqrt[2]{x}\right)^2}{x}\noTRAINER{\newpage}

\aufgabeML{6}{\sqrt[3]{b^3}}{b}

\aufgabeML{6}{\sqrt[3]{z^9}}{z^3}\noTRAINER{\newpage}

\aufgabeML{6}{\sqrt[5]{b^{10}r^5}}{b^2r}

\aufgabeML{6}{\sqrt[4]{r^8n^{12}}}{r^2n^3}\noTRAINER{\newpage}

\aufgabeML{6}{\sqrt[5]{m^4\cdot{m}}}{m}

\aufgabeML{6}{\sqrt{m^2+m^2 + m^2}}{\sqrt{3}\cdot{}m}\noTRAINER{\newpage}

\aufgabeML{6}{\sqrt{\sqrt[2]{x^4}}}{x}

\aufgabeML{6}{\frac{\sqrt[3]{a^7}}{\sqrt[3]{a}}}{a^2}\noTRAINER{\newpage}

\aufgabeML{6}{\sqrt[8]{r^{24}}}{r^3}
\noTRAINER{\newpage}

Und als Überleitung ins nächste Thema:

\aufgabeML{6}{\sqrt[6]{v^3}}{v^{\frac12} = \sqrt{v}}



\newpage
\section{Rationale Exponenten}

\aufgabeML{6}{a^4\cdot{}a^{\frac{1}{4}}\cdot a^5\cdot a^{\frac{-1}{5}}}{a^\frac{181}{20}}

\aufgabeML{6}{\sqrt[4]{a^3}}{a^{\frac{3}{4}}}\noTRAINER{\newpage}

\aufgabeML{6}{\sqrt[3]{r^{\frac{3}{4}}}}{\sqrt[4]{r} = r^{\frac{1}{4}}}

\aufgabeML{6}{\sqrt{b^\frac12}}{\sqrt[4]b = b^\frac14}\noTRAINER{\newpage}

\aufgabeML{6}{\sqrt{a^\frac{1}{2}} \cdot \sqrt[4]{a\cdot{} \sqrt[5]{a^{10}}}}{a}

\aufgabeML{6}{\sqrt{x^{10}} \cdot \sqrt[4]{x^3} \cdot x^{\frac{1}{4}}}{x^6}\noTRAINER{\newpage}

\aufgabeML{6}{\sqrt[3]{a} \cdot \sqrt{a^3 \cdot \sqrt[3]{a}}}{a^2}

\aufgabeML{6}{\sqrt[3]{a^2} \cdot \sqrt[4]{a^3 \cdot \sqrt[3]{a}}}{a^{\frac{9}{6}} = \sqrt[6]{a^9} = a^\frac32 = \sqrt{a^3}}\noTRAINER{\newpage}

\aufgabeML{6}{\sqrt[3]{b\cdot \sqrt{b^3\cdot \sqrt[3]b}}}{\sqrt[9]{b^8}}

\aufgabeML{6}{\sqrt[4]{b\cdot{} \sqrt[3]{b^2\cdot \sqrt{b}}}}{b^\frac{11}{24} = \sqrt[24]{b^{11}}}\noTRAINER{\newpage}

\aufgabeML{6}{3a\cdot \sqrt[3]{9a^2}}{\sqrt[3]{3^5 a^5} = \left(3a\right)^\frac53}

\aufgabeML{6}{\sqrt{x} \cdot \sqrt[3]{x^4} \cdot \sqrt[6]{x^3}}{x^\frac73 = \sqrt[3]{x^7}}\noTRAINER{\newpage}

\aufgabeML{6}{\sqrt{a^3}\cdot \sqrt[3]{a^2}}{a^\frac{13}{6} = \sqrt[6]{a^{13}}}

\aufgabeML{6}{\sqrt{a^2 \sqrt{a}}}{a^\frac54 = \sqrt[4]{a^5}}\noTRAINER{\newpage}

\aufgabeML{6}{\sqrt{a^{-2}}}{\frac{1}{a} = a^{-1}}

\aufgabeML{6}{\sqrt[3]{a^2 \cdot b \cdot \sqrt{a\cdot b^{-1}}}}{a^\frac56 \cdot b^\frac16 = \sqrt[6]{a^5\cdot b}}\noTRAINER{\newpage}

\aufgabeML{6}{\sqrt{x^3} \cdot \left(\sqrt{x}\right)^{-3}}{1}

\aufgabeML{6}{\frac{\sqrt{a^3}}{\sqrt{a^{-1}}}}{a^2}\noTRAINER{\newpage}

\aufgabeML{6}{\left(\frac{1}{a}\right)^{-\frac14}}{\sqrt[4]a = a^\frac14}

\aufgabeML{6}{\frac{a^4}{\sqrt{a}}}{\sqrt{a^7} = (\sqrt{a})^7 = a^\frac72}\noTRAINER{\newpage}

\aufgabeML{6}{\frac{\sqrt[3]{a^2}}{\sqrt{a}}}{a^{\frac16} = \sqrt[6]{a}}

\aufgabeML{6}{\frac{-\sqrt{a^3}}{-\left(\sqrt{a}\right)^3}}{1}\noTRAINER{\newpage}

\aufgabeML{6}{\frac{\sqrt[3]{a^{13}}}{a^4}}{a^{\frac13} = \sqrt[3]{a}}


\aufgabeML{6}{\sqrt[3]{26\cdot{} \sqrt[4]{a^3} + \sqrt[8]{a^6}}}{3\cdot{}\sqrt[4]{a}}
%%\platzFuerBerechnungenBisEndeSeite{}

\end{document}


\subsubsection{Negative Exponenten}
\sectuntertitel{Nicht für alle ist die Potenzrechnung \textit{positiv}.}
Wir kennen bereits das Rechengesetz für positive Exponenten:

$$a^5\cdot{}a^2 = a^{5+2} = a^7$$

Sinnvoll wäre die folgende Erweiterung auf negative
Exponenten:\\\TRAINER{Damit die Rechengesetze weiterhin gelten.}
$$a^5\cdot{}a^{-2} = a^{5+(-2)} = a^3$$

Dividieren wir obige Gleichung beidseitig durch $a^5$, so erhalten wir
folgende sinnvolle Definition:

\TNT{6}{

$$a^5\cdot{} a^{-2} = a^3$$

Wir dividieren beidseitig durch $a^5$:

$$\frac{a^5\cdot{}a^{-2}}{a^5} = \frac{a^3}{a^5}$$

$$a^{-2} = \frac{a^3}{a^5} = \frac{1}{a^2}$$

Damit ist die folgende Definition sinnvoll: 
}%% END TNT

\begin{definition}{}{}
$$a^{-n} := \frac{1}{a^n}$$
\end{definition}

\begin{bemerkung}{}{}
$$ a^{-n} =\frac1{a^n}= 1 : \underbrace{a : a : a : ... : a}_{n \text{\ Divisoren}}$$
\end{bemerkung}

\begin{gesetz}{}{}
$a^{-n} = \left(\frac1a\right)^n$
\end{gesetz}
Begründung:

\TNTeop{$a^{-n} = \frac1{a^n}
= \frac1{\underbrace{a\cdot{}a\cdot{}a\cdot{}...\cdot{}a}_{n \text{\ Faktoren}}}
= \underbrace{\frac1a\cdot{}\frac1a\cdot{}\frac1a\cdot{}...\cdot{}\frac1a}_{n \text{\ Faktoren}}
= \left(\frac1a\right)^n$
}%% END TNT

%%%%%%%%%%%%%%%%%%%%%%%%%%%%%%%%%%%%%%%%%%%%%%%%%%%%%%

Ganz analog gilt:
\TRAINER{Lieblingsgesetz von $\varphi$: Wer mag negative Exponenten?
Wer mag Brüche?}

\begin{gesetz}{}{}
$\frac{1}{a^{-n}} = \LoesungsRaumLen{30mm}{a^n}$
\end{gesetz}
Begründung:
\TNT{8}{

Beispiel:

$$\frac1{10^{-3}} = \frac1{0.001} = 1000 = 10^3$$

\TALS{TALS:}
Beweis: Definition hinschreiben und auf beiden Seiten den Kehrwert bilden:

$$a^{-n} = \frac1{a^n}$$

$$\frac1{a^{-n}} = a^n$$

\vspace{2cm}
}%% END TNT



\begin{gesetz}{}{}
$\left(\frac{1}{a}\right)^{-n}=\LoesungsRaumLen{30mm}{a^n}$
\end{gesetz}
Begründung:
\TNTeop{
Zahlenbeispiel (erster Schritt nach Definition):
$$\left(\frac1{10}\right)^{-3}  = \frac1{\left(\frac1{10}\right)^3}  = \frac1{0.001} = 1000 = 10^3$$

\TALS{TALS:}

Beweis: Nach Definition gilt:

$$x^{-n} = \frac1{x^n}$$

Somit gilt es auch, wenn wir anstelle von $x$ den Term $\frac1a$ einsetzen:

$$\left(\frac1a\right)^{-n} = \frac{1}{\left(\frac{1}{a}\right)^n}
= \frac1{\frac1{a^n}} = 1 : \frac{1}{a^n}= 1\cdot{}\frac{a^n}{1}=a^n$$


\vspace{2cm}

}%% END TNT

%%%%%%%%%%%%%%%%%%%%%%%%%%%%%%%%%%%%%%


Rechenbeispiel\GESO{ (optional)}:

Wurm «Wurli» schaft 3 cm pro Sekunde (= $3 \cdot{} 10^{-2} $ m pro Sekunde). Wie lange braucht «Wurli» für 12 m?
\TNT{4}{
$$t = \frac{s}v
= \frac{12[ \text{m}]}{3 \frac{[\text{cm}]}{[\text{s}]}}
= \frac{12[ \text{m}]}{3\cdot{}10^{-2}\frac{[\text{m}]}{[\text{s}]}}= \frac{4[ \text{m}]}{10^{-2}\frac{[\text{m}]}{[\text{s}]}}
= 4\cdot{} 10^2 [\text{s}]
= 400 [\text{s}]
\approx 6-7 \text{ Min.}$$

\vspace{28mm}
}%% END TNT




Und ebenso für beliebige Brüche:

\begin{gesetz}{}{}
$\left(\frac{a}{b}\right)^{-n} = \left(\frac{b}{a}\right)^{+n}$
\end{gesetz}

    
    \TALS{ Begründung

      \TNT{2.4}{
$\left( \frac{a}{b} \right)^{-n}  =
 \left(a \cdot{} \frac{1}{b} \right)^{-n} =
 b^{-n} \cdot \left(\frac{1}{a}\right)^{-n} =
 \left(\frac{1}{b}\right)^{n} \cdot{} a^{n} =
 \left(\frac{1}{b}\cdot{}a\right)^{n} =
 \left(\frac{a}{b}\right)^{n} $}} %% END TNT END TALS

    \GESO{ Begründung

      \TNT{2.4}{$\left( \frac{5}{2} \right)^3  =
       \left(5 \cdot{} \frac{1}{2} \right)^3 =
       5^3 \cdot \left(\frac{1}{2}\right)^3 =
       \left(\frac{1}{5}\right)^{-3} \cdot{} 2^{-3} =
       \left(\frac{1}{5}\cdot{}2\right)^{-3} =
       \left(\frac{2}{5}\right)^{-3} 
       $}} %% END TNT END GESO

%%$$\left(\frac{1}{a}\right)^{-n} = \frac{1}{\left(\frac{1}{a}\right)^n} = a^n$$
\newpage



\subsubsection{Null}

Idee:
\TNT{3.2}{$$a^5\cdot{}a^0 \stackrel{!}{=} a^{5+0} = a^5$$
$$a^5 \cdot{} a^0 \stackrel{!}{=} a^5   \hspace{5mm} | : a^5$$
$$a^0 = 1$$
}%% end TNT

Oder so:
\TNT{3.2}{
$$a^0 = a^{1-1} = a^1 : a^1 = a:a = 1$$
}%% end TNT

\begin{definition}{Exponent Null}{} Für alle Basen
$a \in \mathbb{R}\backslash\{0\}$ definieren wir:
\begin{center}
\fbox{$a^0 := 1$}
\end{center}
\end{definition}



%\textbf{Rechengesetze zusammengefasst:}

%\begin{itemize}
%\item  $\frac{a^m}{a^n} = a^{m-n}$ (Dies gilt auch wenn $n > m$.)
 
%\item $a^{-n} := a^{0-n}=\frac{a^0}{a^n} = \frac{1}{a^n} = \left(\frac{1}{a}\right)^n$

%\item
%$\left(\frac{1}{a}\right)^{-n} = \frac{1}{\left(\frac{1}{a}\right)^n} = a^n$


%\item $\left(\frac{a}{b}\right)^{-n} = \left(\frac{b}{a}\right)^{+n}$ gilt daher auch. 
%\end{itemize}


\subsection*{Aufgaben}
\GESO{\olatLinkArbeitsblatt{Potenzgesetze}{https://olat.bms-w.ch/auth/RepositoryEntry/6029794/CourseNode/102690264435484}{Kapitel
    3 Aufg. 56. - 59., 61., 63., 64. - 68.  }}%% END olatLinkArbeitsblatt
\TALS{\olatLinkArbeitsblatt{Potenzgesetze}{https://olat.bms-w.ch/auth/RepositoryEntry/6029786/CourseNode/104915210426569}{Kapitel
    3 Aufg. 56. - 59., 61., 63., 64. - 68.}}%% END olatLinkArbeitsblatt

\newpage

\begin{rezept*}{«Kielholen»}{}{}
Exponenten vertauschen ihr Vorzeichen beim Übertreten des Bruchstrichs:
$$\frac{a^{-3}b^2}{c^5b^{-6}} = \LoesungsRaumLang{\frac{b^2\cdot{}b^{+6}}{a^{+3}c^5}}= \LoesungsRaumLang{\frac{b^8}{a^3c^5}}$$
\end{rezept*}



\subsection{Aufgaben}

%%\TALS{Potenzen:}\TALSAadBMTA{32ff}{Von Hand: 79. c), 82. a), 83. b),
%%86. b) c),
%%91. a), 92. j), 94. k), 96. b) f) h) und 105. i)\\
%%Prüfen Sie die folgenden Aufgaben auch mit dem TR (Training):\\
%%103. a) b) c), und 106. h)}

%%\AadBMTA{67ff}{15., 18. c), 19. b), 20. h), 26. b), 31. b),
%%  38. c) e), 41. e), 43. c), 44. d) e) f) h) i), 48. a) b), 49. a) c)}

\AadBMTA{72ff}{(optional) 51. (Koch)}

Mit Brüchen:

\GESO{\olatLinkArbeitsblatt{Potenzgesetze}{https://olat.bms-w.ch/auth/RepositoryEntry/6029794/CourseNode/102690264435484}{Kapitel
    4 Aufg. 77., 78., 81., 82., 83., 89.  }}%% END olatLinkArbeitsblatt
\TALS{\olatLinkArbeitsblatt{Potenzgesetze}{https://olat.bms-w.ch/auth/RepositoryEntry/6029786/CourseNode/104915210426569}{Kapitel
    4 Aufg. 77., 78., 81., 82., 83., 89. }}%% END olatLinkArbeitsblatt


vermischte Exponentialgleichungen: 

\GESO{\olatLinkArbeitsblatt{Potenzgesetze}{https://olat.bms-w.ch/auth/RepositoryEntry/6029794/CourseNode/102690264435484}{Kapitel
    4.1 Aufg. 91. - 94.  }}%% END olatLinkArbeitsblatt
\TALS{\olatLinkArbeitsblatt{Potenzgesetze}{https://olat.bms-w.ch/auth/RepositoryEntry/6029786/CourseNode/104915210426569}{Kapitel
    4.1 Aufg. 91. - 94. }}%% END olatLinkArbeitsblatt

\TRAINER{Alle anderen Aufgaben vom Blatt der Nummern 56. - 90. optional als Training}
\newpage

