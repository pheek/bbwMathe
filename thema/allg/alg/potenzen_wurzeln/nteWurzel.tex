%%
%% 2019 07 04 Ph. G. Freimann
%%

\section{$n$-te Wurzel}\index{n-te@$n$-te Wurzel}
\sectuntertitel{Was soll's denn sein: Radi, Radieschen, Rande, Rübe, Räbe, Rettich, ...?}

\TadBMTA{78}{5}
%%\TALSTadBMTA{78}{5}
\subsection*{Lernziele}

\begin{itemize}
\item Quadratwurzeln, Kubikwurzeln
\item allgemeine $n$-te Wurzeln
\item Rechengesetze
\item Rationale (gebrochene) Exponenten
\end{itemize}
\newpage


\begin{verse}
\textit{«En-te Wurzel»}:
\bbwCenterGraphic{35mm}{allg/alg/potenzen_wurzeln/img/Ente_Wurzel}
\end{verse}

Wir wissen bereits, dass
$$3^2 = 9 \text{ heißt } \sqrt{9} = 3.$$



Doch wenn wir nun
$$x^3 = 1000$$
vor uns haben? Wie kommen wir auf das $x$?

\TNT{1.6}{Dazu gibt es die dritte Wurzel:
$$\sqrt[3]{1000} = 10$$
}


Was bedeutet $\sqrt[3]{8}$? Schätzen Sie und prüfen Sie nach (allenfalls auch mit dem Taschenrechner):

\TNT{1.6}{Verwenden Sie die n-te Wurzel Funktion des Taschenrechners.}



\newpage
\begin{definition}{$n$-te Wurzel}{definition_n_te_wurzel}
Mit $\sqrt[n]{a}$ bezeichnen wir die $n$\textbf{-te Wurzel} aus $a$; das heißt
$$\sqrt[n]{a} = x \Rightarrow x^n = a$$
\end{definition}

\begin{beispiel}{Dritte Wurzel}{}
\TNT{2}{$$\sqrt[3]{1000} = 10 \text{ , denn } 10^3 = 1000$$}
\end{beispiel}

\begin{gesetz}{}{}
$$\sqrt[n]{a^n} = \left(\sqrt[n]a\right)^n = a$$
\end{gesetz}


\begin{definition}{}{}
Den Wurzelexponenten $n=2$ lassen wir üblicherweise weg:

$$\sqrt{a} := \sqrt[2]{a}$$
\end{definition}

\subsection*{Aufgaben}
\AadBMTA{85}{8. f) und 10. g)}

%%  OLAT Arbeitsblatt
\GESO{\olatLinkArbeitsblatt{Potenzgesetze}{https://olat.bbw.ch/auth/RepositoryEntry/572162163/CourseNode/102690264435484}{Kapitel 5 Wurzeln}}%% END olatLinkArbeitsblatt
\TALS{\olatLinkArbeitsblatt{Potenzgesetze}{https://olat.bbw.ch/auth/RepositoryEntry/572162090/CourseNode/104915210426569}{Kapitel 5 Wurzeln}}%% END olatLinkArbeitsblatt


%%\TALSAadBMTA{52}{146. «geometrisches Mittel»}

\newpage


\subsection{Rationale Exponenten}\index{rationale Exponenten}

Wir repetieren:

\TALS{
 $$(x^m)^n = \LoesungsRaum{x^{m\cdot n}}\ \ \text{,
  }\ \ \left(\sqrt[n]{a}\right)^n = \LoesungsRaum{a} \text{ und }
  a^{m+n} = \LoesungsRaum{a^m\cdot{}a^n}.$$
}%% END TALS
\GESO{$$a^m\cdot{}a^n = \LoesungsRaum{a^{m+n}}$$}


\begin{beispiel}{}{}
 Es gilt
\TNT{2}{ $$\sqrt[3]{a} \cdot{}\sqrt[3]{a} \cdot{}\sqrt[3]{a} = a = a^1
  = a^{\frac13 +\frac13 +\frac13} = a^\frac13 \cdot{} a^\frac13
  \cdot{}a^\frac13$$
(Bei $a$ in der Mitte beginnen erst nach links und dann nach rechts gehen.)}
 Somit ist
\TNT{1.6}{ $$\sqrt[3]{a} = a^\frac13$$}
\end{beispiel}

Für $a > 0$ gilt somit allgemein:

\begin{definition}{}{}
  $$a^{\frac{1}{n}} := \sqrt[n]{a}$$
\end{definition}

\TALS{
 Denn: $\sqrt[n]{a} = \left(\sqrt[n]{a}\right)^1 =
 \left(\sqrt[n]{a}\right)^{n\cdot{}\frac{1}{n}} =
 \left(\left(\sqrt[n]{a}\right)^n\right)^\frac{1}{n} = a^\frac{1}{n}$
}%% END TALS

 Es gelten die üblichen Potenzgesetze\GESO{\footnote{S. 81 \cite{marthaler21alg}}} nun auch für die $n$-ten Wurzeln.


\begin{gesetz}{Wurzeln von Wurzeln}{}
  $$\sqrt[m\,]{\sqrt[n]{a}} = \sqrt[m\cdot n]{a}=a^\frac1{mn} =
  \sqrt[n\,]{\sqrt[m]{a}}$$
\end{gesetz}

$$\sqrt[m]{\sqrt[n]{a}} = \sqrt[m]{a^\frac{1}{n}} = (a^{\frac{1}{n}})^{\frac{1}{m}} = a^{\frac{1}{n}\cdot\frac{1}{m}} = a^{\frac{1}{m\cdot n}} = \sqrt[m\cdot n]{a}$$

\begin{gesetz}{Wurzelschreibweise}{}
  $$a^\frac{m}n = \sqrt[n]{a^m} = \left(\sqrt[n]{a}\right)^m$$
\end{gesetz}
\newpage


\begin{beispiel}{}{}
 $\sqrt[3]{\sqrt[2]{64}} = \sqrt[3]{8} = 2 = \sqrt[6]{64} = \sqrt[3\cdot2]{64}$
\end{beispiel}

\GESO{Die Rechengesetze für $n$-te Wurzeln sind im Buch \cite{marthaler21alg} auf Seite 82 zu finden.}

Rechenbeispiel. Schreiben Sie unter eine Wurzel:
$$ab^\frac{-3}4$$

\TNT{10}{
 $$ab^\frac{-3}4 = a\cdot{}b^\frac{-3}4
                = a\cdot{} \left(\frac1b\right)^\frac34
                = a\cdot{} \left(\left(\frac1b\right)^3\right)^\frac14
                = a\cdot{}\sqrt[4\,\,]{\left(\frac1b\right)^3}
                = \sqrt[4\,\,]{a^4 \left(\frac1b\right)^3}
                = \sqrt[4\,\,]{\frac{a^4}{b^3}}$$
}%% END TNT


\newpage

\subsection{Aufgaben}
%%\TALSAadBMTA{51ff}{141. f) g) i), 142. b) d) i), 143. a) d) j), 144. b) i),
%%                149. a) d) h) i), 150. h) und 157. a) b) k)}
\AadBMTA{85 ff}{[Wenn möglich immer \GESO{mit } \TALS{ohne } Taschenrechner] 9. g), 11. a) e) g) i), 12. a) b) d), 13. a) b), 16. a) c) d), 17. a) b), 19. a) d) i),
                 20. a) c), 21. a) c), 22. a) c) e), 24. a)}


%%  OLAT Arbeitsblatt
\GESO{\olatLinkArbeitsblatt{Potenzgesetze}{https://olat.bbw.ch/auth/RepositoryEntry/572162163/CourseNode/102690264435484}{Kapitel 6 Rationale Exponenten}}%% END olatLinkArbeitsblatt

\olatLinkGESOKompendium{1.4}{9}{19}


\TALS{\olatLinkArbeitsblatt{Potenzgesetze}{https://olat.bbw.ch/auth/RepositoryEntry/572162090/CourseNode/104915210426569}{Kapitel 6 Rationale Exponenten}}%% END olatLinkArbeitsblatt


\newpage
