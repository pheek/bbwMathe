%%
%% 2019 07 11 Ph. G. Freimann
%%

\section{Zehnerlogarithmus}\index{Logarithmus!Basis
  10}\index{Zehnerlogarithmus}
\sectuntertitel{Der Logarithmus erzählt nie über seine Vergangenheit,
  weil er Angst hat vor Negativem.}

\begin{center}
  \textit{logos}: \LoesungsRaumLen{50mm}{Das Verhältnis}, \textit{arithmos}: \LoesungsRaumLen{50mm}{Die Zahl}
\end{center}

\GESOTadBMTA{97}{6.1}

%%%%%%%%%%%%%%%%%%%%%%%%%%%%%%%%%%%%%%%%%%%%%%%%%%%%%%%%%%%%%%%%%%%%%%%%%%%%%%%%%
\subsection*{Lernziele}

\begin{itemize}
\item Zehnerpotenzen
\item Zehnerlogarithmus
\item Notation
\TALS{\item Logarithmengesetze I}
\item Beispiele: Richterskala (Erdbeben), dB (Lautstärke), pH («Säuregehalt»)
\end{itemize}

\ifisALLINONE{
\subsubsection*{wissenschaftliche Notation}
Die wissenschaftliche Notation wurde bereits
eingeführt. Zur Erinnerung:\totalref{wissenschaftlicheNotation}.
}\fi{}%% END ALL IN ONE
\newpage

\subsection{Definition Zehnerlogarithmus}\index{Logarithmus!Definition Zehnerlogarithmus}

%%\renewcommand{\arraystretch}{2}
\begin{bbwFillInTabular}{cc|cc}
  \hline
  Potenz               & ${\color{blue}10}^{\color{red}3} = {\color{ForestGreen}x}$     & ${\color{ForestGreen}x}=\LoesungsRaumLang{{\color{blue}10}\cdot{}{\color{blue}10}\cdot{}{\color{blue}10} = {\color{ForestGreen}1000}}$ & \LoesungsRaum{Potenzwert} \\\hline
  Potenzgleichung      & ${\color{blue}x}^{\color{red}3}  = {\color{ForestGreen}1000}$  & ${\color{blue}x}=\LoesungsRaumLang{\sqrt[{\color{red}3}]{{\color{ForestGreen}1000}}       = {\color{blue}10}}$   & \LoesungsRaum{Wurzel}  \\\hline
  Exponentialgleichung & ${\color{blue}10}^{\color{red}x} = {\color{ForestGreen}1000}$  & ${\color{red}x}=\LoesungsRaumLang{\lg(1000) = {\color{red}3}}$                                  & \LoesungsRaum{Logarithmus} \\\hline
  \end{bbwFillInTabular} 
%%\renewcommand{\arraystretch}{1}

\vspace{5mm}

Allgemein für ${\color{red}z}\in \mathbb{Z}$:


Definition für $x\in\mathbb{R}, p \in \mathbb{R}^{+}\backslash\{0\}$:
\begin{definition}{Logarithmus zur Basis 10}{}
  \begin{center}
    ${\color{blue}10}^{\color{red}x}={\color{ForestGreen}p}$
    $\Longleftrightarrow$
    ${\color{red}x} = {\color{blue}\lg}({\color{ForestGreen}p})$
    \end{center}
\end{definition}


\begin{gesetz}{Logarithmus = Exponent}{}
  $$\TRAINER{\lg({\color{ForestGreen} 10^z})} = \TRAINER{{\color{red} z}}$$
\end{gesetz}



\begin{bemerkung}{}{}\index{Logarithmus!Definition}
  Der \textbf{Logarithmus} (hier $\color{red}x$) eines Potenzwertes
  (hier $\color{ForestGreen}p$) ist der Exponent ($\color{red}x$) in der
  Potenzschreibweise (${\color{blue}10}^{\color{red}x}$); und somit die \textbf{Umkehrung des Potenzierens}.
\end{bemerkung}

\textbf{Beispiele}:\\
\leserluft{}

$\lg(100 \cdot{} 1000) = \LoesungsRaum{\lg(10^5)=5}$
\leserluft{}

$\lg(10^{-8}) = \LoesungsRaum{-8}$
\leserluft{}

$\lg(0.0001) = \LoesungsRaum{\lg(10^{-4})=-4}$

$\lg\left(\frac1{100}\right) =\LoesungsRaumLang{\lg(10^{-2}) } = \LoesungsRaum{-2}$
\leserluft{}

$\lg\left(\frac1{1000}\cdot{} 1\,000\,000 \cdot{} 0.01\right)
=\LoesungsRaumLen{50mm}{\lg(10^{-3} \cdot{} 10^6 \cdot{} 10^{-2}) } =
\LoesungsRaumLen{30mm}{-3 + 6 - 2} = \LoesungsRaum{1}$

$\lg\left(\frac1{100}\cdot{}10^{-3}\cdot{}2\cdot{} 5\,000 \cdot{} \frac1{10^{-4}}\cdot{} 0.1\right) = \LoesungsRaumLen{40mm}{+2}$

\newpage



\begin{gesetz}{}{}
$$\lg(10) = \LoesungsRaumLen{10mm}{1}$$
\end{gesetz}

\begin{gesetz}{}{}
$$\lg(1) = \LoesungsRaumLen{10mm}{0}$$
\end{gesetz}


\subsubsection{Logarithmus der Wurzel}
Wir erinnern uns: $$\sqrt{10} =\LoesungsRaum{ 10^\frac12}$$
\begin{beispiel}{Logarithmus der
    Wurzel}{beispiel_logarithmus_der_wurzel}

  Daraus folgt direkt $\lg(\sqrt{10}) = \LoesungsRaum{\lg(10^\frac12)}  = \LoesungsRaum{0.5}$.
\end{beispiel}


\textbf{Beispiel} $\lg(316.23)$ :\\


$\sqrt{10} = 10^{0.5} \approx \LoesungsRaum{3.16227766}$
\leserluft{}

$10^{2.5} \approx \LoesungsRaum{316.227766}$\\
\TRAINER{, denn $10^{2.5} = 10^{2+0.5} = 10^2 \cdot{} 10^{0.5} = 100\cdot{}\sqrt{10}$.}
\leserluft{}

$\lg(10^{2.5}) = \LoesungsRaum{2.5}$

\leserluft{}

$\lg(316.23) \approx \LoesungsRaum{2.5000}$


\begin{bemerkung}{Taschenrechner}{}
  Auf vielen amerikanischen Taschenrechnern steht $\log$ für den Zehnerlogarithmus $\lg$.

  \TALS{Beim «nSpire» muss die Basis 10 jedoch explizit angegeben werden.}
  
  \GESO{  \tiprobutton{ln_log}}
  \TALS{  \nspirebutton{10xlog}}
  \end{bemerkung}

\newpage
\subsection*{Aufgaben}

\GESO{
  \olatLinkArbeitsblatt{A2Log}{https://olat.bms-w.ch/auth/RepositoryEntry/6029794/CourseNode/110436259154906}{1., 2. und 3.}
}%% end GESO
\TALS{
  \olatLinkArbeitsblatt{A2Log}{https://olat.bms-w.ch/auth/RepositoryEntry/6029786/CourseNode/110436259219039}{1., 2. und 3.}
}%% ent TALS




\newpage

\subsection{Rechenregeln}
\subsubsection{Umkehrung}

Der Logarithmus ist die Umkehrung des Exponenzierens.

\begin{gesetz}{}{}
  $$10^{\lg(p)} = \LoesungsRaumLen{10mm}{p}$$
\end{gesetz}

Begründung:
\TNT{2.4}{

  \TALS{$$10^x = p \Longleftrightarrow  \lg(p)=x$$
    nun das $x$ von $x=\lg(p)$ in die linke Seite einsetzen.
    $$10^{\lg(p)} = p$$
  }%% end TALS

  \GESO{Beispiel $$10^{\lg(1000)} = 1000$$
\TRAINER{\vspace{12mm}}
  }

}%% END TNT

\begin{bemerkung}{negative Exponenten}{}
  Weil $10^x$ für alle $x$ immer positiv ausfällt, so gibt es keine
  Logarithmen von negativen Zahlen:

  $$\lg(a) \text{ ist definiert für } a\in\mathbb{R}^+\backslash\{0\}$$
  \end{bemerkung}
\newpage



\subsection*{Aufgaben}
Denken Sie immer daran, dass der Logarithmus als Potenz geschrieben
werden kann:
$${\color{red}x}=\lg({\color{ForestGreen}p}) \Longleftrightarrow  10^{\color{red}x}={\color{ForestGreen}p}$$

Beispiel:

\TNT{2.4}{
  Berechnen Sie $x$: $\lg(x-4) = 2$
  Dies heißt aber$$10^2 = x-4$$ und somit $$104 = x$$
}


\GESO{
  \olatLinkArbeitsblatt{A2Log}{https://olat.bms-w.ch/auth/RepositoryEntry/6029794/CourseNode/110436259154906}{4. und 5.}
}%% end GESO
\TALS{
  \olatLinkArbeitsblatt{A2Log}{https://olat.bms-w.ch/auth/RepositoryEntry/6029786/CourseNode/110436259219039}{4. und 5.}
}%% ent TALS



\newpage
\subsubsection{Produktregel\GESO{ (optional)}}
\begin{gesetz}{}{}
  $\lg(r\cdot s) = \lg(r) + \lg(s)$
\end{gesetz}

Begründung:

\TNT{6.0}{

  \GESO{
$$\lg(10^n \cdot 10^m) = \lg(10^{(n+m)}) = n+m = \lg(10^n) + \lg(10^m)$$
  Setzen wir nun $r:=10^n$ und $s:=10^m$, so erhalten wir\footnote{\TRAINER{Wobei hier $n$ und $m$ nicht notwendigerweise in
    $\mathbb{N}$}}:

$$\lg(r\cdot{}s) = \lg(10^n \cdot 10^m) = ... = \lg(10^n) + \lg(10^m)  = \lg(r) + \lg(s)$$
  }%% end GESO

  \TALS{Es gilt $$10^x=p \Longleftrightarrow \lg(p) = x$$ Setzen wir
    $x$ aus $x=\lg(p)$ in $10^x=p$ ein, so erhalten wir
    $$10^{\lg(p)} = p $$

    Somit gilt (beim Vorzeigen mit $ab$ in der Mitte beginnen:

$$10^{\lg(ab)} = a\cdot{}b = 10^{\lg(a)} \cdot{} 10^{\lg(b)} =
    10^{\lg(a) + \lg(b)}  $$
    Exponentenvergleich liefert
    $$\lg(ab) = \lg(a) + \lg(b)$$
  }%% end TALS
  
  \vspace{15mm}
}%% END TNT

Beispiel: \TRAINER{Beim Vorzeigen mit den beiden 8ern von links und
  rechts beginnen.}

\TNT{2}{$${\color{gray}8=\lg(10^8)=\lg(10^{3+5})}=\lg(10^3\cdot{}10^5) =
\lg(10^3)+\lg(10^5){\color{gray}=3+5=8}$$}%% END TNT

\TALS{
  \textbf{Multiplizieren durch Addieren}\\

  Nach dem ebene gezeigten Gesetz folgt:

\TNT{1.2}{  $$a\cdot{}b = 10^{\lg(a\cdot{}b)} = 10^{\lg(a) +  \lg(b)}$$}

  Und somit können wir zwei Zahlen multiplizieren, indem wir ihre
  Logarithmen addieren:

  \aufgabenFarbe{Lösen Sie die Multiplikation auf dem Blatt im OLAT,
    indem Sie die Logarithmen der Zahlen addieren.}
}
\newpage

%%\AadBMTA{102ff}{2. a) c) d) g) h) i), 3. a) c) e), 4. a), 12. a) c)
%5  und 18. a) b) c) e) \GESO{Aufg. 18. ohne Definitionsbereich}} 
%%\AadBMTA{108}{38.}
%%\AadBMTA{107ff}{Optional: 36., 37. und 39.}


% Frommenwiler:
%\TALSAadBMTA{60ff}{172. a) b) d), 173. f) h) k), 174. a) b) c), 175. b) c),
%%  176. a) b) f) und 179. a) f) h)}
\newpage


