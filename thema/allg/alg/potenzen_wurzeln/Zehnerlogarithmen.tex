%%
%% 2019 07 11 Ph. G. Freimann
%%

\section{Zehnerlogarithmus}\index{Logarithmus!Basis 10}\index{Zehnerlogarithmus}
\sectuntertitel{\textit{logos}: Das Verhältnis, \textit{arithmos}: Die Zahl}

\theorieGESO{97}{6.1}

%%%%%%%%%%%%%%%%%%%%%%%%%%%%%%%%%%%%%%%%%%%%%%%%%%%%%%%%%%%%%%%%%%%%%%%%%%%%%%%%%
\subsection*{Lernziele}

\begin{itemize}
\item Zehnerpotenzen
\item wissenschaftliche Notation (Repetition)
\item Zehnerlogarithmus
\item Notation
\item Logarithmengesetze I
\item Beispiele: Richterskala (Erdbeben), dB (Lautstärke), pH («Säuregehalt»)
\end{itemize}

\ifisALLINONE{
\subsubsection*{wissenschaftliche Notation}
Die wissenschaftliche Notation wurde bereits
eingeführt. Zur Erinnerung:\totalref{wissenschaftlicheNotation}.
}\fi{}%% END ALL IN ONE
\newpage

\subsection{Definition Zehnerlogarithmus}\index{Logarithmus!Definition
  Zehnerlogarithmus}

\renewcommand{\arraystretch}{2}
\begin{tabular}{cc|cc}
  \hline
  Potenz               & ${\color{blue}10}^{\color{red}3} = {\color{ForestGreen}x}$     & ${\color{ForestGreen}x}=\LoesungsRaumLang{{\color{blue}10}\cdot{}{\color{blue}10}\cdot{}{\color{blue}10} = {\color{ForestGreen}1000}}$ & Potenzwert \\\hline
  Potenzgleichung      & ${\color{blue}x}^{\color{red}3}  = {\color{ForestGreen}1000}$  & ${\color{blue}x}=\LoesungsRaumLang{\sqrt[{\color{red}3}]{{\color{ForestGreen}1000}}       = {\color{blue}10}}$   & 3. Wurzel  \\\hline
  Exponentialgleichung & ${\color{blue}10}^{\color{red}x} = {\color{ForestGreen}1000}$  & ${\color{red}x}=\LoesungsRaumLang{\lg(1000) = {\color{red}3}}$                                  & Logarithmus \\\hline
  \end{tabular} 
\renewcommand{\arraystretch}{1}

\vspace{5mm}

Allgemein für ${\color{red}z}\in \mathbb{Z}$:

\begin{definition}{Logarithmus = Exponent}{}
  $$\lg({\color{green} 10^z}) = {\color{red} z}$$
\end{definition}


Definition für $r\in\mathbb{R}, p \in \mathbb{R}^{+}\backslash\{0\}$:
\begin{definition}{Logarithmus zur Basis 10}{}
  \begin{center}
    ${\color{red}r} = {\color{blue}\lg}({\color{ForestGreen}p})$
    $\Longleftrightarrow$
    ${\color{blue}10}^{\color{red}r}={\color{ForestGreen}p}$
    \end{center}
\end{definition}

\begin{bemerkung}{}{}\index{Logarithmus!Definition}
  Der \textbf{Logarithmus} (hier $\color{red}r$) eines Potenzwertes
  (hier $\color{ForestGreen}p$) ist der Exponent ($\color{red}r$) in der
  Potenzschreibweise (${\color{blue}10}^{\color{red}r}$); und somit die \textbf{Umkehrung des Potenzierens}.
\end{bemerkung}

\newpage
\subsubsection{Logarithmus der Wurzel}
War erinnern uns: $$\sqrt{10} = 10^\frac12$$
\begin{beispiel}{Logarithmus der
    Wurzel}{beispiel_logarithmus_der_wurzel}

  Daraus folgt direkt $\lg(\sqrt{10}) =\lg(10^\frac12)  = 0.5$.
\end{beispiel}

\subsubsection{Notation}
Der Zehnerlogarithmus $\lg()$ wird oft auch explizit mit der Basis 10
angegeben; dann wird $\log_{10}(\,\,)$ anstelle von $\lg(\,\,)$ geschrieben:

\begin{center}
\fbox{$\lg = \log_{10}$ }
\end{center}

\begin{bemerkung}{Taschenrechner}{}
  Auf vielen Taschenrechnern steht einfach $\log$ anstatt $\lg$.

  \tiprobutton{ln_log}
  \end{bemerkung}
\newpage

\subsection{Rechengesetze}
Per Definition gilt:

\begin{gesetz}{}{}
$\lg(10) = 1$
\end{gesetz}

\begin{gesetz}{}{}
$\lg(1) = 0$
\end{gesetz}

\subsubsection{Multiplikatiosgesetz\GESO{ (optional)}}
\begin{gesetz}{}{}
  $\lg(r\cdot s) = \lg(r) + \lg(s)$
\end{gesetz}

Begründung:
\TNT{6.0}{
$10^n * 10^m = 10^{n+m}.$
\\
Zahlenbeispiel:
$$\log(10^2\cdot{}10^4) = \log(10^{2+4}) = 2+4
= \log(10^2) + \log(10^4) $$

Oder allgemein (für $r = 10^n$ und $s = 10^m$)\footnote{\TRAINER{Wobei hier $n$ und $m$ nicht notwendigerweise in $\mathbb{N}$}}:
$$\lg(r \cdot s) = \lg(10^n \cdot 10^m) = \lg(10^{(n+m)}) = n+m = \lg(10^n) + \lg(10^m) = \lg(r) + \lg(s).$$
\vspace{3cm}
}%% END TNT

Beispiel: \TRAINER{Beim Vorzeigen mit den beiden 8ern von links und
  rechts beginnen.}

$${\color{gray}8=\lg(10^8)=\lg(10^{3+5})}=\lg(10^3\cdot{}10^5) = \lg(10^3)+\lg(10^5){\color{gray}=3+5=8}$$
\newpage


\subsection*{Aufgaben}
\GESOAadB{102ff}{2. a) c) d) g) h) i), 3. a) c) e), 4. a), 12. a) c)
  und 18. a) b) c) e)} 
\GESOAadB{107ff}{36., 37., 38. und 39.}

\TALSAadB{???}{???}
\newpage

