%% 2024_12_28 umgestellt 
\subsubsection{Null}

Ein Blatt Papier wird drei mal über seine Mitte gefaltet. Somit
sind \LoesungsRaumLen{20mm}{8} Schichten übereinander.

Wie viele Schichten sind es bei fünf mal falten? Bei fünfmaligem Falten werden
es

$$\LoesungsRaumLen{20mm}{32 = 2^5} \text{ Schichten.}$$

Die Formel bei $n$ maligem Falten lautet:

$$n \text{ Faltungen} \mapsto \LoesungsRaumLen{30mm}{2^n}\text{ Schichten}$$


Wie viele Schichten sind da, wenn nicht, also null mal, gefaltet wird?
Richtig:

$\LoesungsRaumLen{20mm}{2^0 = 1 \text{ Schicht}}$

Idee:
\TNT{2}{
$$a^0 = a^{n-n} = a^n : a^n = 1$$
}%% end TNT

Oder so:
\TNT{4}{
$$a^5\cdot{}a^0 \stackrel{!}{=} a^{5+0} = a^5$$
$$a^5 \cdot{} a^0 \stackrel{!}{=} a^5   \hspace{5mm} | : a^5$$
$$a^0 = 1$$
}%% end TNT

\begin{definition}{Exponent Null}{} Für alle Basen
$a \in \mathbb{R}\backslash\{0\}$ definieren wir:
\begin{center}
\fbox{$a^0 := 1$}
\end{center}
\end{definition}


%\textbf{Rechengesetze zusammengefasst:}

%\begin{itemize}
%\item  $\frac{a^m}{a^n} = a^{m-n}$ (Dies gilt auch wenn $n > m$.)
 
%\item $a^{-n} := a^{0-n}=\frac{a^0}{a^n} = \frac{1}{a^n} = \left(\frac{1}{a}\right)^n$

%\item
%$\left(\frac{1}{a}\right)^{-n} = \frac{1}{\left(\frac{1}{a}\right)^n} = a^n$


%\item $\left(\frac{a}{b}\right)^{-n} = \left(\frac{b}{a}\right)^{+n}$ gilt daher auch. 
%\end{itemize}

%% TODO: Nur Aufgaben mit 0
\subsection*{Aufgaben}
\GESO{\olatLinkArbeitsblatt{Potenzgesetze}{https://olat.bms-w.ch/auth/RepositoryEntry/6029794/CourseNode/102690264435484}{Kapitel
    3.1 Aufg. 56. - 59.}}%% END olatLinkArbeitsblatt
\TALS{\olatLinkArbeitsblatt{Potenzgesetze}{https://olat.bms-w.ch/auth/RepositoryEntry/6029786/CourseNode/104915210426569}{Kapitel
    3.1 Aufg. 56. - 59.}}%% END olatLinkArbeitsblatt


\newpage


\subsubsection{Negative Exponenten}
\sectuntertitel{Nicht für alle ist die Potenzrechnung \textit{positiv}.}

\subsubsection*{Permanenzprinzip (optional)}\index{Permanentsprnzip}

\subsubsection*{Repetition: Zehnerpotenzen mit negativen Exponenten}

\begin{gesetz}{negative Exponenten bei Zehnerpotenzen}{}
$$10^{-n} = \frac{1}{10^n}$$
\end{gesetz}

\begin{beispiel}{negative Exponenten bei Zehnerpotenzen}{}
$$10^{-3} = 0.001 = \frac1{1000} = \frac1{10^3}$$
\end{beispiel}


\subsection*{Aufgaben}
\GESO{\olatLinkArbeitsblatt{Potenzgesetze}{https://olat.bms-w.ch/auth/RepositoryEntry/6029794/CourseNode/102690264435484}{Kapitel
    3.2 Aufg. 60. - 63.}}%% END olatLinkArbeitsblatt
\TALS{\olatLinkArbeitsblatt{Potenzgesetze}{https://olat.bms-w.ch/auth/RepositoryEntry/6029786/CourseNode/104915210426569}{Kapitel
    3.2 Aufg. 60. - 63.}}%% END olatLinkArbeitsblatt

\newpage


(optional)
Michael(is) Sti(e)fel («Arithmetica Integra» \cite{stiefel1544} 1544)
Buch I Seite 31 über die «geometrische Progression»:
%\noTRAINER{\vspace{14mm}}
\bbwCenterGraphic{140mm}{allg/alg/img/permanentsprinzip/01GeometrischeProgression.png}

%\TRAINER{... Terme lieferten immer positive Werte. Buch II S. 123. Hier findet eine Subtraktion statt. Resultate sind immer positiv...}
%             \noTRAINER{\vspace{14mm}}
Termwerte waren ausschließlich positiv, Buch II \TRAINER{S. 123, da geometrisch motiviert}:
\bbwCenterGraphic{140mm}{allg/alg/img/permanentsprinzip/02NurPositiveWerte.png}

... erstes Auftreten von «negativen Zahlen». Buch III S. 249. $(-)\cdot{}(-) = (+)$ ...
\bbwCenterGraphic{140mm}{allg/alg/img/permanentsprinzip/03Permanenzprinzip.png}
\TNT{4}{Permanentsprinzip kann auch via Distributivgesetz motiviert
werden:

$$0= (-6)\cdot{}0 = (-6)\cdot{}((-4) + 4) = (-6)\cdot{}(-4) +
(-6)\cdot{}(4) = (-6)\cdot{}(-4) - 6\cdot{}4$$ ergo ($6\cdot{}4$ auf
die andere Seite nehmen)
$$6\cdot{}4 = (-6)\cdot{}(-4)$$

Schon auf der nächsten Seite benutzt Michael Stifl negative Zahlen
(bis dahin «numeri absurdi»):}%% end TNT

\bbwCenterGraphic{140mm}{allg/alg/img/permanentsprinzip/04NegativeExponenten.png}
\TRAINER{\begin{center}\tiny{Lat.: «Aber gezeigt soll werden diese Spekulation per Beispiel.»}\end{center}}

\newpage

\subsubsection*{Definition negativer Exponenten}


Wir kennen bereits das Rechengesetz für positive Exponenten:

$$a^5\cdot{}a^2 = a^{5+2} = a^7$$

Sinnvoll wäre die folgende Erweiterung auf negative
Exponenten:\\\TRAINER{Damit die Rechengesetze weiterhin gelten.}
$$a^5\cdot{}a^{-2} = a^{5+(-2)} = a^3$$

\TRAINER{Dividieren wir obige Gleichung beidseitig durch $a^5$, so erhalten wir

folgende sinnvolle Definition:}

\TNT{6}{

$$a^5\cdot{} a^{-2} = a^3$$

Wir dividieren beidseitig durch $a^5$:

$$\frac{a^5\cdot{}a^{-2}}{a^5} = \frac{a^3}{a^5}$$

$$a^{-2} = \frac{a^3}{a^5} = \frac{1}{a^2}$$

Damit ist die folgende Definition sinnvoll: 
}%% END TNT

\begin{definition}{}{}
$$a^{-n} := \frac{1}{a^n}$$
\end{definition}

\begin{bemerkung}{}{}
$$ a^{-n} =\frac1{a^n}= 1 : \underbrace{a : a : a : ... : a}_{n \text{\ Divisoren}}$$
\end{bemerkung}

\begin{gesetz}{}{}
$a^{-n} = \left(\frac1a\right)^n$
\end{gesetz}
Begründung:

\TNTeop{$a^{-n} = \frac1{a^n}
= \frac1{\underbrace{a\cdot{}a\cdot{}a\cdot{}...\cdot{}a}_{n \text{\ Faktoren}}}
= \underbrace{\frac1a\cdot{}\frac1a\cdot{}\frac1a\cdot{}...\cdot{}\frac1a}_{n \text{\ Faktoren}}
= \left(\frac1a\right)^n$
}%% END TNT

%%%%%%%%%%%%%%%%%%%%%%%%%%%%%%%%%%%%%%%%%%%%%%%%%%%%%%

Ganz analog gilt:
\TRAINER{Lieblingsgesetz von $\varphi$: Wer mag negative Exponenten?
Wer mag Brüche?}

\begin{gesetz}{}{}
$\frac{1}{a^{-n}} = \LoesungsRaumLen{30mm}{a^n}$
\end{gesetz}
Begründung:
\TNT{8}{

Beispiel:

$$\frac1{10^{-3}} = \frac1{0.001} = 1000 = 10^3$$

\TALS{TALS:}
Beweis: Definition hinschreiben und auf beiden Seiten den Kehrwert bilden:

$$a^{-n} = \frac1{a^n}$$

$$\frac1{a^{-n}} = a^n$$

\vspace{2cm}
}%% END TNT



\begin{gesetz}{}{}
$\left(\frac{1}{a}\right)^{-n}=\LoesungsRaumLen{30mm}{a^n}$
\end{gesetz}
Begründung:
\TNTeop{
Zahlenbeispiel (erster Schritt nach Definition):
$$\left(\frac1{10}\right)^{-3}  = \frac1{\left(\frac1{10}\right)^3}  = \frac1{0.001} = 1000 = 10^3$$

\TALS{TALS:}

Beweis: Nach Definition gilt:

$$x^{-n} = \frac1{x^n}$$

Somit gilt es auch, wenn wir anstelle von $x$ den Term $\frac1a$ einsetzen:

$$\left(\frac1a\right)^{-n} = \frac{1}{\left(\frac{1}{a}\right)^n}
= \frac1{\frac1{a^n}} = 1 : \frac{1}{a^n}= 1\cdot{}\frac{a^n}{1}=a^n$$


\vspace{2cm}

}%% END TNT

%%%%%%%%%%%%%%%%%%%%%%%%%%%%%%%%%%%%%%


Rechenbeispiel\GESO{ (optional)}:

Wurm «Wurli» schafft 3 cm pro Sekunde (= $3 \cdot{} 10^{-2} $ m pro Sekunde). Wie lange braucht «Wurli» für 12 m?
\TNT{4}{
$$t = \frac{s}v
= \frac{12[ \text{m}]}{3 \frac{[\text{cm}]}{[\text{s}]}}
= \frac{12[ \text{m}]}{3\cdot{}10^{-2}\frac{[\text{m}]}{[\text{s}]}}= \frac{4[ \text{m}]}{10^{-2}\frac{[\text{m}]}{[\text{s}]}}
= 4\cdot{} 10^2 [\text{s}]
= 400 [\text{s}]
\approx 6-7 \text{ Min.}$$

\vspace{28mm}
}%% END TNT




Und ebenso für beliebige Brüche:

\begin{gesetz}{}{}
$\left(\frac{a}{b}\right)^{-n} = \left(\frac{b}{a}\right)^{+n}$
\end{gesetz}

    \TALS{ Begründung

      \TNT{2.4}{
$\left( \frac{a}{b} \right)^{-n}  =
 \left(a \cdot{} \frac{1}{b} \right)^{-n} =
 b^{-n} \cdot \left(\frac{1}{a}\right)^{-n} =
 \left(\frac{1}{b}\right)^{n} \cdot{} a^{n} =
 \left(\frac{1}{b}\cdot{}a\right)^{n} =
 \left(\frac{a}{b}\right)^{n} $}} %% END TNT END TALS

    \GESO{ Begründung

      \TNT{2.4}{$\left( \frac{5}{2} \right)^{-3}  =
       \left(5 \cdot{} \frac{1}{2} \right)^{-3} =
       5^{-3} \cdot \left(\frac{1}{2}\right)^{-3} =
       \left(\frac{1}{5}\right)^{3} \cdot{} 2^{3} =
       \left(\frac{1}{5}\cdot{}2\right)^{3} =
       \left(\frac{2}{5}\right)^{3}
       $}} %% END TNT END GESO

%%$$\left(\frac{1}{a}\right)^{-n} = \frac{1}{\left(\frac{1}{a}\right)^n} = a^n$$



\subsection*{Aufgaben}
\GESO{\olatLinkArbeitsblatt{Potenzgesetze}{https://olat.bms-w.ch/auth/RepositoryEntry/6029794/CourseNode/102690264435484}{Kapitel
    3.3 Aufg. 64. - 79.  }}%% END olatLinkArbeitsblatt
\TALS{\olatLinkArbeitsblatt{Potenzgesetze}{https://olat.bms-w.ch/auth/RepositoryEntry/6029786/CourseNode/104915210426569}{Kapitel
    3.3 Aufg. 64. - 79.}}%% END olatLinkArbeitsblatt

\newpage




\subsection{Negative Exponenten in Brüchen}

\begin{rezept*}{«Kielholen»}{}{}
Exponenten vertauschen ihr Vorzeichen beim Übertreten des Bruchstrichs:
$$\frac{a^{-3}b^2}{c^5b^{-6}} = \LoesungsRaumLang{\frac{b^2\cdot{}b^{+6}}{a^{+3}c^5}}= \LoesungsRaumLang{\frac{b^8}{a^3c^5}}$$
\end{rezept*}



\subsection{Aufgaben}

%%\TALS{Potenzen:}\TALSAadBMTA{32ff}{Von Hand: 79. c), 82. a), 83. b),
%%86. b) c),
%%91. a), 92. j), 94. k), 96. b) f) h) und 105. i)\\
%%Prüfen Sie die folgenden Aufgaben auch mit dem TR (Training):\\
%%103. a) b) c), und 106. h)}

%%\AadBMTA{67ff}{15., 18. c), 19. b), 20. h), 26. b), 31. b),
%%  38. c) e), 41. e), 43. c), 44. d) e) f) h) i), 48. a) b), 49. a) c)}

\AadBMTA{72ff}{(optional) 51. (Koch)}

Mit Brüchen:

\GESO{\olatLinkArbeitsblatt{Potenzgesetze}{https://olat.bms-w.ch/auth/RepositoryEntry/6029794/CourseNode/102690264435484}{Kapitel
    4.1 Aufg. 80., 81., 84., 85., 86., 92.  }}%% END olatLinkArbeitsblatt
\TALS{\olatLinkArbeitsblatt{Potenzgesetze}{https://olat.bms-w.ch/auth/RepositoryEntry/6029786/CourseNode/104915210426569}{Kapitel
    4.1 Aufg.  80., 81., 84., 85., 86., 92. }}%% END olatLinkArbeitsblatt

vermischte Exponentialgleichungen:

\GESO{\olatLinkArbeitsblatt{Potenzgesetze}{https://olat.bms-w.ch/auth/RepositoryEntry/6029794/CourseNode/102690264435484}{Kapitel
    4.2 Aufg. 94. - 97.  }}%% END olatLinkArbeitsblatt
\TALS{\olatLinkArbeitsblatt{Potenzgesetze}{https://olat.bms-w.ch/auth/RepositoryEntry/6029786/CourseNode/104915210426569}{Kapitel
    4.2 Aufg. 94. - 97. }}%% END olatLinkArbeitsblatt

\TRAINER{Alle anderen Aufgaben vom Blatt der Nummern 56. - 97. optional als Training}
\newpage
