
\section{Quadratwurzeln}\index{Quadratwurzel}\index{$\sqrt{\mathstrut{}\,}$}
\sectuntertitel{Was machen Zahnarzt, Mathematiker und Gärtner? Wurzeln
  ziehen!}

\subsection*{Lernziele}

\begin{itemize}
\item Definition Quadratwurzel
\item Wurzelgesetze
\item Multiplikation und Divison von Wurzeln
\TRAINER{\item \textit{Optional: Normalform}}
\end{itemize}


\TadBMTA{78}{5.1}

\TRAINER{
\bbwCenterGraphic{35mm}{allg/alg/potenzen_wurzeln/img/rasiidli.png}
}
\noTRAINER{\vspace{35mm}}

\subsection{Wortherkunft und Definition}

\begin{bemerkung}{Symbol}{}
  
Das Symbol $\sqrt{\vphantom{b} \,\,}$
bezeichnet ein stilisiertes {\textit{kleines}} {\huge{r}} und steht für
  (lat.) \textit{radix}, die Wurzel\footnote{Eingeführt 1525 vom
  Mathematiker Christoph Rudolff; die Verlängerung wurde erst 1637 von
René Descartes eingeführt.}.
\end{bemerkung}


\begin{definition}{Quadratwurzel}{definition_quadratwurzel}
Mit $\sqrt{\vphantom{b} \,\,}$ bezeichnen wir die
Quadratwurzel\index{Quadratwurzel}.
\end{definition}

\newpage

\subsection{Gesetze}

\begin{definition}{Quadratwurzel}{}
Für $a$ in $\mathbb{R}_0^{+}$, also für $a\ge 0$ gilt

$\sqrt{a} \cdot \sqrt{a} = a$

$\sqrt{a} \ge 0$
\end{definition}



\begin{gesetz}{Quadratwurzel}{}
Für $a$ in $\mathbb{R}_0^{+}$ gilt
$$\sqrt{a^2} = a$$
$$(\sqrt{a})^2 = a$$
\end{gesetz}


Die «Wurzel aus $a$ Quadrat» kann also auf zwei Arten aufgefasst werden:

\TNT{3.2}{$$a = \sqrt{a^2} = \sqrt{a}^2 = a$$
Ob der Exponent innerhalb oder außerhalb der Wurzel steht spielt für positive Zahlen keine Rolle.
}

\begin{beispiel}{}{}
$$\sqrt{9^2}=9$$
$$\sqrt{9}^2 = 9$$
\end{beispiel}


\TALS{
  \begin{bemerkung}{}{}
    Für $a$ in $\mathbb{R}$ gilt

  $$\sqrt{a\cdot a} = |a|$$  
  $$\left(\sqrt{|a|}\right)^2 = |a|$$
  \end{bemerkung}
}



\newpage
\subsubsection{Multiplikation}

\begin{gesetz}{Multiplikation}{}
  Für $a\ge 0$, $b\ge 0$ gilt:
 
$$\sqrt{\vphantom{b}a}\cdot{\sqrt{b}} = \sqrt{\vphantom{b}a\cdot{b}}$$

\end{gesetz}


$\sqrt{\vphantom{b}a}\cdot\sqrt{b} =$

\TNT{3.2}{

%\TALS{
${\color{blue}\sqrt{\mathstrut a}\cdot\sqrt{b}} = \sqrt{({\color{blue}\sqrt{a}\cdot\sqrt{b}})^2} =
  \sqrt{\sqrt{a}\cdot\sqrt{b}\cdot\sqrt{a}\cdot\sqrt{b}} =
  \sqrt{\sqrt{a}\cdot\sqrt{a}\cdot\sqrt{b}\cdot\sqrt{b}} =
  \sqrt{a \cdot b}$

Die erste Gleichung gilt, wenn ich den ganzen Ausdruck links quadriere
und danach davon die Wurzel ziehe.
%}%% END TALS

%\GESO{
%Es gilt (für positive Zahlen):
%$$A^2\cdot B^2 = (A\cdot B)^2$$
%%
%$\Rightarrow$
%
%\begin{center}
%$\sqrt{\mathstrut{}A^2\cdot B^2}=\sqrt{(AB)^2}=AB=\sqrt{A^2}\cdot\sqrt{B^2}$
%\end{center}

%$\Rightarrow$

%\begin{center}
%\fbox{$\sqrt{\mathstrut{}a\cdot b} = \sqrt{\mathstrut{}a}\cdot\sqrt{\mathstrut{}b}$}
%\end{center}

%}%% END GESO

}%% end TNT


\subsubsection{Division}
Ganz analog kann gezeigt werden, dass gilt: 
\begin{gesetz}{Division}{}
  Für $a\ge 0$, $b\ge 0$ gilt:
  \begin{tabular}{c@{ = }c}
$\frac{\sqrt{\vphantom{b}a}}{\sqrt{b}}$  &  $\sqrt{\frac{\vphantom{b}a}{b}}$\\
    \end{tabular}
\end{gesetz}

$\sqrt{a} : \sqrt{b} =$

\TNT{2}{
  $\frac{\sqrt{a}}{\sqrt{b}}
  = \sqrt{a} : \sqrt{b}
  = \sqrt{(\sqrt{a}:\sqrt{b})^2}
  = \sqrt{\frac{\sqrt{a}}{\sqrt{b}} \cdot \frac{\sqrt{a}}{\sqrt{b}}}
  = \sqrt{\frac{a}{b}}$%%
}%% END TNT

\subsubsection{Addition/Subtraktion}
\begin{bemerkung}{Addition}{}
  Aber \textbf{Achtung:}
  
$$\sqrt{a^2+b^2} = \noTRAINER{?}\TRAINER{\sqrt{a^2+b^2}}$$
  $$\sqrt{a\mathstrut{}} + \sqrt{b\mathstrut{}} = \noTRAINER{?}\TRAINER{\sqrt{a\mathstrut{}} + \sqrt{b\mathstrut{}}}$$
\TNT{2}{Hierzu gibt es keine Gesetzmäßigkeiten!}
\end{bemerkung}

\newpage

\subsection{Partielles Radizieren}

\begin{beispiel}{Partielles
Radizieren\index{radizieren!partielles}\index{partielles Radizieren}}{}
  $$\sqrt{12\cdot{} a^3} = ...$$

\TNT{3.2}{$$\sqrt{12\cdot{} a^3} = \sqrt{3\cdot{} 4 \cdot{} a \cdot{} a^2}
= \sqrt{3} \cdot{} \sqrt{4} \cdot{} \sqrt{a} \cdot{} \sqrt{a^2} = 2a\cdot{}\sqrt{3a}$$
}%% END TNT
\end{beispiel}


\subsection*{Aufgaben}
\AadBMTA{84ff}{1. g), 2. a) d), 3. a) c) e) f) g) h), 4. c) b) d)
  und 6. d)\GESO{*}\TALS{ i)}}

Beachten Sie im Buch die Tipps auf Seite 79.

\newpage
