%%
%% 2019 07 04 Ph. G. Freimann
%%

\newpage
\section{Zins}\index{Textaufgaben zur Zinsrechnung}\index{Zins}
\sectuntertitel{``Klar hab' ich Probleme --- ich bin Mathelehrer.''}
%%%%%%%%%%%%%%%%%%%%%%%%%%%%%%%%%%%%%%%%%%%%%%%%%%%%%%%%%%%%%%%%%%%%%%%%%%%%%%%%%
\subsection*{Lernziele}

\begin{itemize}
  \item Der Zins ist eine  Multiplikation, der Zinseszins ist eine Potenz
\item Textaufgaben mit Zins und Zinseszins
\end{itemize}

\subsection{Einstiegsaufgabe}
Einstiegsaufgabe:
Ein Händler gibt Ihnen auf eine Ware von 234.50 CHF einen Jugend-Rabatt von
5\%, wenn Sie ihren BMS-Ausweis zeigen. Wenn Sie in bar bezahlen, erhalten Sie
einen weiteren Rabatt von 3\% auf den bereits reduzierten Preis.
Ist es für Sie nun schlauer, zuerst den
Jugend-Rabatt (5\%) und danach den Bar-Rabatt(\%) einzufordern oder
ist die umgekehrte Reihenfolge schlauer?


\TNTeop{
Minus $3\%$ bedeutet 97\% vom Originalpreis, also $100\% - 3\%$

Der Preis nach Abzug der 3\% berechnet sich wie folgt:

$234.50 - \frac{234.50}{100}\cdot{}3 = \frac{234.50}{100}\cdot{}100
- \frac{234.50}{100}\cdot{3}$

$= \frac{234.50}{100}\cdot{100-3} = \frac{234.50}{100}\cdot{}97 =
0.97\cdot{} 234.50$

\vspace{12mm}

Analog: $... -5\% = ... \cdot{} 0.95$

Ergo Endpreis = $0.95\cdot{}0.97\cdot{}234.50\approx 216.10$
}

\newpage



\subsection{Zins als Faktor / Zinsformel}\index{Zins}
Beispiel $2\%$ Zins auf Kapital CHF $1\,000.-$:\\

\begin{tabular}{l|l|l}
  \textit{Variable}      &   \textit{Bedeutung} & \textit{Beispiel}\\%%
\hline%%
 $K_0$                    & Startkapital        & 1\,000.-  [CHF]\\\hline

 $p$                      & Zinsfuß             & 2  [\%]\\\hline

$K_0\cdot{}\frac{p}{100}$ & Zins von $K_0$      & \TRAINER{$\frac{K_0}{100[\%]}\cdot{}2[\%] =$}\\
%%  &                     & \TRAINER{$\frac{K_0\cdot{}2[\%]}{100[\%]} =$}\\
                          &                     & \TRAINER{$K_0 \cdot{} \frac2{100} = $}\\
                          &                     & \TRAINER{$K_0\cdot 0.02$}\\\hline

$K_1$                     & neues Kapital       & \TRAINER{  $K_0+$ Zins von  $K_0=$} \\
                          &                     & \TRAINER{  $K_0+K_0\cdot{}0.02 = $ }\\
                          &                     & \TRAINER{  $K_0\cdot(1+0.02) = $}\\
                          &                     & \TRAINER{  $K_0\cdot{}1.02$}\\\hline
$r$                       & Aufzinsungsfaktor   & \TRAINER{$r=1+\frac{p}{100}$}\\
\end{tabular}

\begin{bemerkung}{Faktor}{}

\textbf{100\% = \text{ (Faktor)  }1} und  \textbf{2\% = \text{
                          (Faktor) } 0.02}.

Somit gilt \textbf{100\% + 2\% = \text{ (Faktor) } 1.02}

\end{bemerkung}

\begin{gesetz}{Zinsformel}{}
Bei einem Zinssatz\index{Zinssatz} von $p$\% wird das neue Kapital $K_1$ aus dem
Anfangskapital $K_0$ wie folgt berechnet:
\begin{center}\fbox{$K_1 = K_0\cdot{} r$}\end{center}
\textbf{Aufzinsungsfaktor}\index{Aufzinsungsfaktor}:
$$r = 1 + \frac{p}{100}$$
\end{gesetz}
\newpage


\subsubsection{Rate vs. Faktor I}\index{Rate}\index{Faktor}\index{Zunahmerate}\index{Zunahmefaktor}
\label{RateZins1}

\begin{tabular}{r|r}
Zinssatz $p$ (= Zuwachs\textbf{rate}\index{Zuwachsrate})& Aufzinsungsfaktor $r$ (= Zunahme\textbf{faktor})\\\hline
2\%              & \TRAINER{1.02}\\\hline
\TRAINER{0.25\%} & 1.0025        \\\hline
-3\%             & \TRAINER{0.97}\\\hline
100\%            & \TRAINER{2}   \\\hline
\TRAINER{15\%}   & 1.15          \\\hline
\TRAINER{50\%}   & 1.5           \\\hline
-18\%            & \TRAINER{0.82}\\\hline
-50\%            & \TRAINER{0.5} \\\hline
\TRAINER{200\%}  &  3            \\\hline
\TRAINER{-90\%}  & 0.1           \\\hline
-100\%           &\TRAINER{0}    \\\hline
\end{tabular}

\hrule

Die BBW-Mütze kostet CHF 80.-. Dazu müssen Sie 7.7\%
Mehrwertsteuer bezahlen. Als Schüler erhalten Sie danach einen Rabatt von
30\%. Wie teuer ist die BBW-Mütze letztlich?
\TNT{4}{$$80 \cdot{} 1.077 \cdot{} 0.7 = 60.31,2 CHF \approx{} 60.30 CHF$$}


\hrule

Das BBW-Shirt kostet Sie CHF 60.20. Dabei ist ihr Schülerrabatt von
20\% und auch eine Mehrwertsteuer von 7.5\% auf den reduzierten Preis bereits enthalten.
Was war der ursprüngliche Preis des BBW-Shirts, wenn nach Zuschlag von
7.5\% und einem nachträglichen Abzug von 20\% noch ein Preis von 60.20 CHF zu bezahlen
ist?

\TNT{4}{
$$60.20 / 1.075 / 0.8 = 70.- CHF$$
}
\aufgabenFarbe{Machen Sie zu obigen beiden Aufgaben je ein eigenes Beispiel, das Sie
mit ihrer Banknachbarin/ihrem Banknachbarn überprüfen.}
\newpage


%Um von einer Ausgangsgröße 100\% zu berechnen, kann einfach der Faktor
%1.0 genommen werden. Ebenso kann der Faktor 0.03 genommen werden, um
%3\% der Größe zu berechnen. Ein Anwachsen eines Kapitals um 3\% ist
%also nichts anderes als das multiplizieren mit dem Faktor 1.03.

\subsection{Zinseszins}\index{Zinseszins}

\TRAINER{EinstiegsComics Donald Duck im OLAT}

Beim Zinseszins, wird bei jeder weiteren Verzinsung der bereits

erhaltene Zins weiter verzinst. Beispiel 20.- Taler zu 2\%:

CHF 20.- $\stackrel{\cdot{}1.02}{\longrightarrow}$ \LoesungsRaum{20.40} $\stackrel{\cdot{}1.02}{\longrightarrow}$ \LoesungsRaum{20.808}

$$K_2 = {\color{ForestGreen}K_1}\cdot{}1.02 = ({\color{ForestGreen}K_0\cdot{}1.02})\cdot{}1.02 = K_0\cdot{}1.02^2$$

Bei 2\% kann also jedesmal mit einem \textbf{Aufzinsungsfaktor} von
1.02 multipliziert werden.


\TNT{4}{
Nach 100 Jahren wachsen die 20 Taler von Tick Trick und Track auf
$20\cdot{}1.02^{100}\approx 144.89$ an\TRAINER{ (runden auf 0.01 Taler
miteinbezogen jedes Jahr -> 144.99,000 Taler)}.


Nach 1000 Jahren wachsen unsere CHF 20.-- also auf $20 \cdot{}
1.02^{1000}$ an ($\approx 8.0 Mia. Taler$)\footnote{Dies entspricht einem Faktor von fast 400 Millionen}. 
}%% END TNT

\subsubsection{Zinseszinsformel}\index{Zins und Zinseszins}
\TRAINER{Hinweis an die Lehrperson: Insbesondere BM2 gut behandeln, denn der Stoff ist ev. in der Sekundarschule nicht behandelt worden (Sek B) oder es liegt generell zu weit zurück.}

\begin{gesetz}{Zinseszins}{}
Bei gegebenem Startkapital $K_0$ und gegebenem Zinsfuß $p$ (in \%) kann das Endkapital $K_n$ nach $n$ Jahren wie folgt berechnet werden:

\begin{center}\fbox{$K_n = K_0 \cdot{} r^n$}\end{center}

Mit \textbf{Aufzinsungsfaktor}:

$$r = 1 + \frac{p}{100}$$
\end{gesetz}

\TRAINER{Bemerkung: Die Zinseszinsformel beschreibt ein exponentielles Wachstum.}
\newpage

\subsubsection{Zinsaufgaben}

Berechnen Sie:

\begin{itemize}
  \item Berechnen Sie das Endkapital nach 20 Jahren bei einem
  Startkapital von CHF 15\,000.- und einem Zins von jährlich
  2.5\%.\\%%

\TNT{2.4}{Endkapital = $15\,000\cdot{} 1.025^{20}\approx 24\,579.25$ CHF\vspace{2cm}}%%
\item In einem Wald werden 200 Fichten für Weihnachten
  gepflanzt. Wegen der hohen Nachfrage wird der Bestand jedes Jahr um
  3\% ausgeweitet (vergrößert).
  Wie viele Fichten werden nach fünf Jahren gepflanzt?


  \TNT{2.4}{End\textit{kapital} = $200\cdot{} (1.03)^{5} \approx
  231 \text{ bis } 232$
  Fichten.\vspace{2cm}}%%


\item Ein Auto hat einen Neupreis von CHF 40\,000.-. Jedes Jahr müssen
wegen Abnutzung ein Wertverlust von dreizehn \% in Kauf
  genommen werden. Wie viel Wert hat das Auto noch nach fünfzehn Jahren? (Achtung, hier ist der Zins negativ!)

\TNT{2.4}{End\textit{kapital} = $40\,000\cdot{} (1-0.13)^{15} = 40\,000\cdot{} 0.87^{15}\approx 4\, 953.-$ CHF\vspace{2cm}}
\end{itemize}


\GESO{\subsection*{Aufgaben}}
\GESOAadBMTA{355}{13.}
%%\GESOAadBMTA{207}{10. a)} %% die anderen setzen Lograithmen voraus! und 12.}
%%\GESOAadBMTA{207}{9. a) und 11. a)}
\newpage


\subsubsection{Momentanverzinsung\GESO{ (optional)}}\index{Momentanverzinsung}\index{stetige Verzinsung}\index{Verzinsung!stetige}
(auch stetige Verzinsung)\index{Verzinsung!stetige}
Wird ein Kapital zu 100\% verzinst, so wächst unser Kapital auf 200\%
an. Wenn wir das Kapital jedoch zweimal zu 50\%
verzinsen\footnote{Sprich: Wir lassen uns nach 6 Monaten den Zins
auszahlen, heben das Konto auf und bezahlen sofort wieder mit Zins in
ein neues Konto ein.}, erhalten wir den folgenden, besseren Verzinsungsfaktor:

$1.50^2  = (1 + \frac12)^2 = 2.25$

Bei 4-maliger Verzinsung steigt der Faktor weiter an:
$1.25^4 = (1 + \frac14)^4 \approx 2.44 $

Füllen Sie die folgende Tabelle aus:

\begin{tabular}{c|c|c|c} 
  Anzahl Teile  & Faktor                        & Formel          & Endkapital \\ \hline
  $1$           & $2^1$                         & $(1+1)^1$ & $= K_0 \cdot{} 2 $ \\ \hline
  $2$           & $1.5^2$                       & $(1+\frac12)^2$ & $= K_0 \cdot{} 2.25 $ \\ \hline
  $4$           & $1.25^4$                      & $(1+\frac14)^4$ & \LoesungsRaum{$\approx K_0 \cdot{} 2.4414 $} \\ \hline
  $5$           & $\LoesungsRaum{1.2^5}$         & $\LoesungsRaum{(1+\frac15)^5}$ & $\LoesungsRaum{\approx K_0 \cdot{} 2.48832} $ \\ \hline
  $10$          & $\LoesungsRaum{1.1^{10}}$      & $\LoesungsRaum{(1+\frac{1}{10})^{10}}$ & $\LoesungsRaum{\approx K_0 \cdot{} 2.5937} $ \\ \hline
  $100$         & $\LoesungsRaum{1.01^{100}}$    & $\LoesungsRaum{(1+\frac{1}{100})^{100}}$ & $\LoesungsRaum{\approx K_0 \cdot{} 2.7048 }$ \\ \hline
  $1000$        & $\LoesungsRaum{1.001^{1000}}$  & $\LoesungsRaum{(1+\frac{1}{1000})^{1000}}$ & $\LoesungsRaum{\approx K_0 \cdot{} 2.7169 }$ \\ \hline
  Großes $n$    & ****  & $\LoesungsRaum{(1+\frac{1}{n})^{n}}$ & $\LoesungsRaum{\approx K_0 \cdot{} e }$ \\ \hline
\end{tabular} 

Dieses maximal erreichbare Kapital entspricht etwa dem Faktor \noTRAINER{\hspace{33mm}}\TRAINER{2.71828} und
wird als Eulersche Konstante\index{Eulersche Zahl}\index{$\e$!Eulersche Konstante} mit $\e$ bezeichnet.

\bbwCenterGraphic{140mm}{allg/img/euler_banknote.jpg}
Bildlegende: Leonhard Euler (1707-1783) auf der Schweizer Zehnernote (1976-1995).

\begin{definition}{Eulersche Zahl}{}\index{Eulersche Zahl}
$$\e \approx 2.7182818284590$$
\end{definition}
\newpage
