\subsection{Sonderfälle (Fallunterscheidung)}\index{Fallunterscheidung!Gleichungssysteme}\index{Sonderfälle!Gleichungssysteme}\index{lineare Gleichungssysteme!Sonderfälle, Fallunterscheidung}

Berechnen Sie die Lösung für $(x,y)$ in Abhängigkeit von $a$:

\gleichungZZ{ax+ay}{1}{x-ay}{-1}

\TNTeop{
  Additionsverfahren:
  
  $$ax+x=0$$
  $$x(a+1)=0$$
  $$\Longrightarrow x=0, y=\frac1a$$


  $$\LoesungsMenge{}_{(x;y)} = \left\{\left( 0 ; \frac1a \right)\right\}$$

  Sonderfälle (SF)

  \textbf{SF1}: Lösung für $a=0$ kann die Lösung für $y$ nicht
  stimmen (Division durch 0).
  Setzen wir $a=0$ in die erste Gleichung ein, so entsteht eine
  Falschaussage:

  Für $a=0$ gilt also $\LoesungsMenge{}_{(x;y)} = \{\}$
  
  \textbf{SF2}: Lösung für $a+1=0$ gilt: $x$ ist beliebig und $a=-1$

  Aus der 2. Gleichung folgt mit $a=-1$: $x+y=-1$ und somit $y=-x-1$

}%% END TNTeop

%%%%%%%%%%%%% \ newpage %%%%%%%%%%%%%%%%%%

\subsection*{Aufgabe}
Berechnen Sie die Lösung für $(x,y)$ in Abhängigkeit von $m$ und $n$:

\gleichungZZ{2mx+y}{3}{-x-y}{n}

\TNTeop{
  Nach dem Additionsverfahren steht da:
  $$2mx-x=3+n$$
  $$x(2m-1) = 3+n$$
  
  Lösung: 
  $$(x,y) = \left(\frac{n+3}{2m-1} ,\frac{3+2nm}{1-2m} \right)$$

  Sonderfälle (SF):

   \textbf{SF 1}: \\
   Aus $x(2m-1)=3+n$ folgt: Ist $3+n$ nicht null, aber $2m-1=0$, so
   gibt es keine mögliche Lösung. 

   \textbf{SF 2}:\\
   Ist hingegen in $x(2m-1)=3+n$ der Term $3+n = 0$ (also $n=-3$), so
   muss entweder $x=0$ sein und es gilt $\LoesungsMenge{}_{(x;y)}=
   \{(0;3)\}$

   oder aber es gilt dass $2m-1 = 0$ ist, und somit gäbe es für $x$
   beliebig viele Lösungen: $\LoesungsMenge{}_{(x;y)} = \{(x;y) |
   x\in\mathbb{R} \land y=3-x\}$
}%% END TNT
\newpage

\subsection*{Aufgaben}
\TALSAadBMTA{153}{20. a) c) e)}
