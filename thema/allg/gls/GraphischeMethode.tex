\subsection{Graphische Methode}\index{graphische Methode!Gleichungssysteme}

\subsubsection{Gleichsetzungsverfahren (optional)}\label{lin_gl_gleichsetzungsverfahren}\index{Gleichsetzungsverfahren}
Beim Gleichsetzungsverfahren werden zwei Gleichungen nach der selben Variable \textit{aufgelöst} und die beiden Gleichungen der Form $a = ...$ einander gleichgesetzt.

\gleichungZZ{a}{{\color{orange}\frac{3.10 -  2b}{3}}}{a}{{\color{ForestGreen}\frac{6.00 - 5b}{4}}}%%
\vspace{22mm}

Setzen wir nun die beiden Terme gleich, so erhalten wir die
folgende Gleichung:

\TNT{7.2}{$${\color{orange}\frac{3.10 - 2b}{3}} =
  {\color{ForestGreen}\frac{6.00 - 5b}{4}}$$
  \vspace{10mm}

  Links mit 4, rechts mit 3 erweitern
  $$4\cdot{}(3.10-2b) = 3\cdot{}(6.00-5b)$$
  $$12.40 - 8b = 18 - 15b$$
  $$7b = 5.60$$
  
}%% END TNT

%%Nach Auf"|lösung erhalten wir $b=\LoesungsRaum{0.80}$ (CHF) und das $a$ erhalten wir danach durch Einsetzen in eine der beiden bereits nach $a$ aufgelösten Gleichungen: $a=\LoesungsRaum{0.50}$ (CHF).
Nach Auf"|lösung erhalten wir $b=\LoesungsRaum{0.80}$ (CHF) und das
$a$ erhalten wir danach durch Einsetzen in eine der beiden bereits
nach $a$ aufgelösten Gleichungen:

\TNT{4}{
$$a=\frac{3.10 -2b}{3} =\frac{3.10 - 2\cdot{}0.80}{3} =  \frac{3.10 -
    1.60}{3} =  \frac{1.50}{3} = 0.50 $$
}
Das Gleichsetzungsverfahren ist ein Spezialfall des
Einsetzungsverfahrens\totalref{einsetzungsverfahren}.
\newpage



%%%%%%%%%%%%%%%%%%%%%%%%%%%%%%%%%%%%%%%%%%%%%%%%%%%%%%%%%%%%%%%%%%%%%%%%%%%%%%%%%%%%%

\subsubsection{Ablesen}
Das Gleichsetzungsverfahren\totalref{lin_gl_gleichsetzungsverfahren} kann auch als Schnittpunkt zweier linearer Funktionen\totalref{lineare_funktionen}
aufgefasst werden. Betrachten wir folgende beiden Gleichungen:

\gleichungZZ{2x - 10y}{-10}{6x+15y}{60}

Lösen wir beide Gleichungen nach $y$ auf, so erhalten wir zwei Funktionsgleichungen:

\gleichungZZ{f: y}{\noTRAINER{...............}\TRAINER{0.2x + 1}}{g:y}{\noTRAINER{...............}\TRAINER{-\frac{2}{5}x + 4}}


Zeichnen Sie die beiden linearen Funktionen als Graph ins folgende
Koordinatensystem ein und lesen Sie die Lösung \TRAINER{bei $(5|2)$} ab:

\noTRAINER{
  \bbwGraph{-6}{8}{-1}{6}{}
}
\TRAINER{
  \bbwGraph{-6}{8}{-1}{6}{
    \bbwFuncC{\x * 0.2 + 1}{-5.5:8}{green}
    \bbwLetter{7.5,3}{f}{green}
    \bbwFuncC{-0.4*\x + 4}{-1:7}{blue}
    \bbwLetter{7.5,1}{g}{blue}
  }
}
\newpage


\subsection*{Aufgabe Graphisch}
\GESO{\olatLinkArbeitsblatt{Gleichungssysteme}{https://olat.bms-w.ch/auth/RepositoryEntry/6029794/CourseNode/112603866030292}{Kap. 5: a) b) c)}}
\TALS{\olatLinkArbeitsblatt{Gleichungssysteme}{https://olat.bms-w.ch/auth/RepositoryEntry/6029786/CourseNode/112603866140343}{Kap. 5: a) b) c)}}

%%  Lösen Sie die folgende Aufgabe graphisch. Bringen Sie die beiden Gleichungen erst in die Form $y=ax+b$ der linearen Funktionen. Zeichnen Sie danach beide Funktionen in ein Koordinatensystem ($x$ von -6 bis 1 und $y$ von -3 bis 5). Schätzen Sie die Lösung, bevor Sie diese berechnen.
%%\AadBMTA{151}{8. g)}
%\olatLinkGESOKompendium{2.2.4.}{15}{39. bis 41.}

\newpage
