\subsubsection{Spezialfälle}
\textbf{TYP A:} Keine Lösung

Bestimmen Sie die Lösungsmenge des folgenden linearen Gleichungssystems:

\gleichungZZ{\frac{-9}{5}x + \frac{11}{5} y}{0.8}{-2.7 x + \frac{33}{10}y}{1}

Die Gleichung hat keine Lösung. Geometrisch kann man sich das so vorstellen, dass es sich um zwei Geradengleichungen handelt, welche zwei Parallele darstellen.

$$\LoesungsMenge{}_{(x;y)} = \LoesungsRaumLang{\{\}}$$

\textbf{TYP B:} Beliebig viele Lösungen (lineare Abhängigkeit)

Bestimmen Sie die Lösungsmenge des folgenden linearen Gleichungssystems:

\gleichungZZ{\frac{9}{2}x - 2.4y}{2.1}{3x-\frac{8}{5}y}{1.4}

Hier gibt es unendlich viele Lösungen. Für jedes $x$ kann ich aus der ersten Gleichung ein $y$ berechnen. Doch beim Einsetzen in die 2. Gleichung erhalte ich keine neue Information.
Geometrisch handelt es sich bei beiden Gleichungen um ein und dieselbe Gerade: Jeder Punkt auf der Geraden löst also beide Gleichungen.

$$\LoesungsMenge{}_{(x;y)} = \LoesungsRaumLang{
\left\{(x;y)| x\in \mathbb{R} \land{} y = \frac{15}{8} \cdot{} x - \frac78\right\} =
\left\{(x;y)| y\in \mathbb{R} \land{} x = \frac8{15}y + \frac7{15}\right\}}$$

%%\TRAINER{$y = \frac{9x-4.2}{4.8}$ ist die Funktionsgleichung beider Geraden.}
\newpage


\subsection*{Aufgaben}
%%\TALSAadBMTA{125ff}{382. 383., 389. a) 410. a), 411. a) b)}
\AadBMTA{150ff}{Falls nötig zuerst in Grundform bringen. Schreiben Sie $x$, $y$ und $z$ streng untereinander. Danach mit Taschenrechner lösen: 18. a) b) c) und e) 21. b) c) e)}

Optionale Aufgaben zu linearen Funktionen mit Taschenrechner:
\AadBMTA{253}{24. b) c) d)  25. a)}

\newpage
