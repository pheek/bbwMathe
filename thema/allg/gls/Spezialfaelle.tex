%
% Spezialfälle linearer Gleichungssysteme (leere menge/unendlich viele Lösungen)
%
\subsubsection{Spezialfälle}
\textbf{TYP A:} Keine Lösung

Bestimmen Sie die Lösungsmenge des folgenden linearen Gleichungssystems:

\gleichungZZ{2x-y}{1}{4x-2y}{3}           

\TNT{2}{Additionsverfahren liefert $$(4x-2y)-2\cdot{}(2x-y) = 3 - 2\cdot{}(1) \Longrightarrow 0 = 1$$}

Die Gleichung hat keine Lösung. Geometrisch: Die Geraden sind parallel.

$$\LoesungsMenge{}_{(x;y)} = \LoesungsRaumLang{\{\}}$$

\textbf{TYP B:} Beliebig viele Lösungen (lineare Abhängigkeit)

Bestimmen Sie die Lösungsmenge des folgenden linearen Gleichungssystems:

\gleichungZZ{2x-y}{1}{4x-2y}{2}           

\TNT{2}{Additionsverfahren liefert $$(4x-2y)-2\cdot{}(2x-y) = 2 - 2\cdot{}(1) \Longrightarrow 0 = 0$$}

Hier gibt es unendlich viele Lösungen. Die entsprechenden Geraden sind zusammefallend.

$$\LoesungsMenge{}_{(x;y)} = \LoesungsRaumLen{50mm}{ \left\{(x;y) \text{ mit } y = 2x-1 \Longleftrightarrow x = \frac{y+1}2 \right\} }$$

%%\TRAINER{$y = \frac{9x-4.2}{4.8}$ ist die Funktionsgleichung beider Geraden.}
\newpage

\subsection*{Repetition: Lösungsmenge}\index{Lösungsmenge}
Auch graphisch kann die Lösungsmenge ermittelt werden. Die Angaben
sind ananolg zu den Gleichungen mit einer Variable.
\begin{center}
\begin{tabular}{c|c|c}
schneidend                 & parallel      &
zusammenfallend\\
 & & \\
\vspace{0.5mm}
$\gleichungZZ{y}{x}{y}{8}$ & $\gleichungZZ{y}{x+5}{y}{x+7}$ & $\gleichungZZ{y}{x+3}{2y}{2x+6}$   \\
 & & \\
$\times$                   & //            &   \textbf{/}      \\
 & & \\
$x=8$                      & $5=7$         &  $3=3$            \\
 & & \\
$\lx=\LoesungsRaumLen{20mm}{\{8\}}$                & $\lx=\LoesungsRaumLen{20mm}{\{\}}$    &  $\lx=\LoesungsRaumLen{20mm}{\mathbb{R}}$ \\
\end{tabular} 
\end{center}

%%%%%%%%%%%%%%%%%%%%%%%%%%%%%%%
%\subsection*{Aufgaben}
%%\TALSAadBMTA{125ff}{382. 383., 389. a) 410. a), 411. a) b)}
%\GESO{\olatLinkArbeitsblatt{Gleichungssysteme}{https://olat.bms-w.ch/auth/RepositoryEntry/6029794/CourseNode/112603866030292}{Kap. 1: a) b) c) g) h)}}
%\TALS{\olatLinkArbeitsblatt{Gleichungssysteme}{https://olat.bms-w.ch/auth/RepositoryEntry/6029786/CourseNode/112603866140343}{Kap. 1: a) b) c) g) h)}}
%\AadBMTA{153ff}{Falls nötig zuerst in Grundform bringen. Schreiben Sie $x$, $y$ und $z$ streng untereinander. Danach mit Taschenrechner lösen: 18. a) b) c) und e)}

%Optionale Aufgaben zu linearen Funktionen mit Taschenrechner:
%\AadBMTA{253}{24. b) c) d)  25. a)}

\newpage
