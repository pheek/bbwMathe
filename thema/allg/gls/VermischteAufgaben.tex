
\subsection{Vermischte Aufgaben}\index{Bruchgleichungen!Gleichungssysteme}
Oft kommen auch in Gleichungssystemen Bruchgleichungen vor. Tipp:
bevor Sie die nachfolgenden Aufgaben lösen, betrachten Sie die folgende
Abkürzung: Um die folgende Bruchgleichung zu vereinfachen...

\gleichungZZ{\frac{\color{ForestGreen}2-x}{\color{red}2x-3}}{\frac{\color{red}5-y}{\color{ForestGreen}2y-4}}{8}{5x-3y}
... wird die erste Gleichung mit «übers Kreuz multiplizieren» zu:

$$({\color{ForestGreen}2-x})({\color{ForestGreen}2y-4}) =
({\color{red}2x-3})({\color{red}5-y})$$
Nun wie gewohnt auflösen:
\TNT{3.2}{
  $$4y-8-2xy+4x=10x-2xy-15+3y$$
  $$\Rightarrow y+7=6x$$
}
Grundform:
\TNT{3.2}{
  \gleichungZZ{6x-y}{7}{5x-3y}{8}
  Additionsverfahren: Obere Gleichung mit 3 multiplizieren:
  \gleichungZZ{18x-3y}{21}{5x-3y}{8}
}%% END TNT

$$\LoesungsMenge{}_{(x;y)} = \LoesungsRaumLang{\{(1; -1)\}}$$

Lösen Sie die folgenden Aufgaben mit der für Sie am besten geeigneten Methode, indem Sie die
Gleichungen in der Regel zunächst in die Grundform bringen.
  \AadBMTA{151}{9. b) d) e) und f)}
\TALS{  \AadBMTA{158}{58. (Quader) mit Taschenrechner}}

\olatLinkGESOKompendium{2.2.1}{13}{31. bis 32.}

\newpage
