\subsection{Lineare Abhängigkeit}\index{abhängig!linear}\index{lineare Abhängigkeit}

Lösen Sie das folgende Gleichungssystem vorerst mit dem
Taschenrechner:

\gleichungZZ{9x-6y}{18}{30x-20y}{60}

Die Lösung ist nicht vielsagend.

Durch die Additionsmethode erhalten wir folgendes Gleichungssystem:

\TNT{3.2}{\gleichungZZ{90x-60y}{180}{90x-60y}{180}}

oder nach der Subtraktion:

\TNT{1.6}{$$0=0.$$}

Dies bedeutet, wir haben durch Äquivalenzumformungen nun zweimal die selbe
Gleichung da stehen. Wir können $x$ nur in Abhängigkeit von $y$
berechnen (mehr nicht):
\TNT{4}{(I) $\Longrightarrow$ $9x=18+6y$ $\Longrightarrow$ $x=\frac{6y+18}{9}$}
$$x=\noTRAINER{\hspace{20mm}}\TRAINER{\frac{2y+6}{3}}$$

Oder wir können $y$ in Abhängigkeit von $x$ berechnen:
$$y=\noTRAINER{\hspace{20mm}}\TRAINER{\frac{3x-6}{2}}$$

Wichtig ist vor allem, dass Sie auch die Antwort des Taschenrechners verstehen!
\newpage
