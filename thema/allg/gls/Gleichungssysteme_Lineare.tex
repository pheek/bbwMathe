\section{Lineare Gleichungssysteme}\index{Gleichungssysteme}\index{lineare Gleichungssysteme}

\subsection*{Lernziele}
\begin{itemize}
	\item{Lineare Gleichung mit mehreren unbekannten}
	\item{Grundform}
	\item{Lösungsmethoden}
	\begin{itemize}
		\item{Einsetzungsverfahren}
		\item{Gleichsetzungsverfahren}
		\item{Additionsverfahren}
		\item{Substitution}
		\item{Graphische Methode}
		\item{Taschenrechner}
      \TALS{\item 2x2 Gleichungssysteme mit Parametern}
      \TALS{\item 3x3 Gleichungssysteme}
    \item Textaufgaben
	\end{itemize}

\end{itemize}

%%\TALSTadBFWA{124}{2.5.1}
\TadBMTA{135}{9}
\newpage


\subsection{Einstiegsbeispiel}
Ein Apfel koste CHF $a$, eine Birne CHF $b$. Wenn ich drei Äpfel und zwei Birnen kaufe, bezahle ich CHF 3.10. Wenn ich hingegen vier Äpfel und fünf Birnen kaufe, so muss ich CHF 6.00 bezahlen. Wie viel kostet ein Apfel, wie viel eine Birne?


\TNT{5.2}{$a$ = Preis eines Apfels in CHF

  $b$ = Preis einer Birne \textbf{in} CHF

  $3a$ = Totalpreis von drei Äpfeln \textbf{in} CHF

  $2b$ = Totalpreis von zwei Birnen \textbf{in} CHF

  etc.
  
  \vspace{22mm}
}


Wir schreiben dies üblicherweise als sortiertes Gleichungssystem auf:

\TNTeop{\gleichungZZ{3a + 2b}{3.10}{4a + 5b}{6.00}}

%%%%%%%%%%%%%%%%%%%%%%%%%%%%%%%%%%%%%%%%%%%%%%%%%%%%%%%%%%%%%%%%%%%%%%%%%%

\subsection{Einsetzungsverfahren}\label{einsetzungsverfahren}\index{Einsetzungsverfahren!Gleichungssysteme}
Beim Einsetzungsverfahren wird aus einer der beiden Gleichungen eine der beiden Variablen alleine gestellt. Zum Beispiel hier aus der zweiten Gleichung $b = ...$; danach wird dieser $b$-Term in die \textbf{andere} Gleichung eingesetzt, und so erhalten wir eine Gleichung mit einer Unbekannten.

\gleichungZZ{3a + 2b}{3.10\TRAINER{ (I)}}{4a + 5b}{6.00\TRAINER{ (II)}}

Aus der zweiten Gleichung erhalten wir für $b$:

\TNT{3.2}{$$5b= 6 - 4a \Longrightarrow b = {\color{orange}\frac{6.00-4a}{5}}. (III)$$}

Diesen Term setzen wir nun für $b$ in die \textbf{andere} (hier die obere) Gleichung ein:
$$3a+2b = 3.10$$

Dies ergibt für $a$:

\TNT{1.2}{$$3a+2\cdot{}{\color{orange}\frac{6.00 - 4a}{5}} = 3.10.$$}

Wenn wir diese Gleichung nun nach $a$ auf"|lösen, ...

\TNT{4}{
  $$ 3a+\frac{12-8a}{5} = 3.1$$
  $$ 3a+\frac{12}{5} - \frac{8a}{5} = 3.1$$
  $$ 15a+12 - 8a = 15.50$$
  $$ 7a = 3.50$$
}

so erhalten wir $a = \LoesungsRaum{0.50}$ (CHF).


Nun setzen wir das gefundene $a$ in (III) ein:

\TNT{2}{$b=\frac{6-4a}{5} = \frac{6-4\cdot{} 0.5}{5} = \frac{6-2}{5} =
  \frac{4}{5} = 0.8$}%% end TNT

...und erhalten $b = \LoesungsRaum{0.80}$ (CHF).


\newpage
\begin{rezept}{Einsetzungsverfahren}{}
  \begin{enumerate}

  \item Eine (beliebige) Gleichung nach einer (beliebigen) Variable
    auf\/lösen (separieren), falls nicht schon geschehen
  \item Den Term (Wert) der Variable in die \textbf{andere(n)}
    Gleichung(en) einsetzen
  \item Die neue(n) Gleichung(en) wie gewohnt auf\/lösen
  \item Das Resultat in die Gleichung aus Schritt 1 einsetzen,
    um die dortige Variable zu erhalten
  \end{enumerate}
\end{rezept}


\subsection*{Aufgaben}
  \AadBMTA{151}{8. a) b) d) f)}

\olatLinkGESOKompendium{2.2.1}{13}{27. bis 32.}
\newpage
  
\subsubsection{Ein nicht lineares Beispiel}
Mit dem Einsetzungsverfahren lassen sich auch nicht-lineare
Gleichungssysteme lösen:
\gleichungZZ{a\cdot{}b}{72}{a^4\cdot{}\frac{b}2}{288}

Lösungsweg:
  \TNT{10.8}{
    Erstens aus der ersten Gleichung das $b$ separieren:
    $$b = \frac{72}a$$
    Und dieses $b$ nun in die \textbf{andere} Gleichung einsetzen:
    $$a^4\cdot{}\frac12\cdot{}\frac{72}{a} = 288$$
    
    Auf\/lösen nach $a$:

    $$a^3 \cdot{} 36 = 288$$

    $$a^3 = 8$$

    ergibt $a=2$. Dies nun in die «separierte
    Gleichung» einsetzen: $b=\frac{72}{a} = \frac{72}2 = 36$.

    Hinweis: Hier hätte man gleich von vornherein die zweite Gleichung
    durch die erste dividieren können (dies wäre dann aber nicht mehr
    das Einsetzungsverfahren).
  }%%

  \begin{gesetz}{Lösungsmenge}{}
    Die Lösung eines Gleichungssystems ist typischerweise ein
    geordnetes Zahlenpaar. Wir schreiben 

\TNT{2}{    $$\mathbb{L}_{(a; b)} = \left\{(2; 36)\right\}$$ } 

  als Beispiel für obige Gleichung.
    
    \end{gesetz}

  \newpage
  %%%%%%%%%%%

\subsection{Abkürzende Verfahren}
\subsubsection{Additions- bzw. Subtraktionsverfahren}\index{Additionsverfahren!lineare Gleichungen}\index{Subtraktionsverfahren!Gleichungssysteme}
Eine weitere Möglichkeit Gleichungen mit mehreren Unbekannten zu lösen ist das Verfahren, die linke und die rechte Seite miteinander zu addieren (bzw. voneinander zu subtrahieren), sodass eine Variable wegfällt. Dazu müssen die Gleichungen vorher so \textit{präpariert} werden, dass in einer Variablen (hier \zB $a$) die Koeffizienten übereinstimmen:

\gleichungZZ{3a + 2b}{3.10}{4a + 5b}{6.00}

  \begin{rezept}{Rechengesetz Additionsverfahren}{gesetz_additionsverfahren}
    Eine beliebige Variable wird gewählt und danach beide Gleichungen so multipliziert,
    dass die Zahlen vor dieser Variable in beiden Gleichungen gleich werden.

    Dies funktioniert
    immer mittels kleinstem gemeinsamen Vielfachem (kgV); kann aber auch mit
    einem beliebigen gemeinsamen Vielfachen beider Zahlen erreicht werden.
  \end{rezept}

Wir wählen \zB in obigem Beispiel die Variable $a$ und multiplizieren die obere Gleichung mit 4 und die untere mit 3 und
erhalten:

\TNT{2.8}{\gleichungZZ{12a + 8b}{12.40 (I)}{12a + 15b}{18.00 (II)}}


Nun können wir die obere Gleichung von der unteren Gleichung abziehen (damit die Variable $a$ eliminiert wird)
und erhalten:
  
\TNTeop{
 (II) - (I)
$$0a  + (15b - 8b)  = 18.00 - 12.40$$
Also
$$7b = 5.60,$$ und somit
$$b = 0.80.$$
Nun setzen wir dieses $b$ in eine der beiden obigen Gleichungen ein
und erhalten nach dem Auf"|lösen $a=0.50$ (CHF).
}
\newpage


\subsection*{Aufgaben}
\AadBMTA{150}{7. b) c) e) g)}

\newpage


\subsubsection{Gleichsetzungsverfahren (optional)}\label{lin_gl_gleichsetzungsverfahren}\index{Gleichsetzungsverfahren}
Beim Gleichsetzungsverfahren werden zwei Gleichungen nach der selben Variable \textit{aufgelöst} und die beiden Gleichungen der Form $a = ...$ einander gleichgesetzt.

\gleichungZZ{a}{{\color{orange}\frac{3.10 -  2b}{3}}}{a}{{\color{ForestGreen}\frac{6.00 - 5b}{4}}}%%
\vspace{22mm}

Setzen wir nun die beiden Terme gleich, so erhalten wir die
folgende Gleichung:

\TNT{8}{$${\color{orange}\frac{3.10 - 2b}{3}} =
  {\color{ForestGreen}\frac{6.00 - 5b}{4}}$$
  \vspace{10mm}

  Links mit 4, rechts mit 3 erweitern
  $$4\cdot{}(3.10-2b) = 3\cdot{}(6.00-5b)$$
  $$12.40 - 8b = 18 - 15b$$
  $$7b = 5.40$$
  
}%% END TNT

%%Nach Auf"|lösung erhalten wir $b=\LoesungsRaum{0.80}$ (CHF) und das $a$ erhalten wir danach durch Einsetzen in eine der beiden bereits nach $a$ aufgelösten Gleichungen: $a=\LoesungsRaum{0.50}$ (CHF).
Nach Auf"|lösung erhalten wir $b=\LoesungsRaum{0.80}$ (CHF) und das
$a$ erhalten wir danach durch Einsetzen in eine der beiden bereits
nach $a$ aufgelösten Gleichungen:

\TNT{4}{
$$a=\frac{3.10 -2b}{3} =\frac{3.10 - 2\cdot{}0.80}{3} =  \frac{3.10 -
    1.60}{3} =  \frac{1.50}{3} = 0.50 $$
}
Das Gleichsetzungsverfahren ist ein Spezialfall des
Einsetzungsverfahrens\totalref{einsetzungsverfahren}.
\newpage


\subsubsection*{Spezielle Bruchgleichungen}\index{Bruchgleichungen!Gleichungssysteme}
Oft kommen auch in Gleichungssystemen Bruchgleichungen vor. Tipp:
bevor Sie die folgenden Aufgaben lösen, betrachten Sie die folgende
Abkürzung, um eine Bruchgleichung zu vereinfachen...

\gleichungZZ{\frac{\color{ForestGreen}2-x}{\color{red}2x-3}}{\frac{\color{red}5-y}{\color{ForestGreen}2y-4}}{8}{5x-3y}
... wird mit «übers Kreuz multiplizieren» zu:

$$({\color{ForestGreen}2-x})({\color{ForestGreen}2y-4}) =
({\color{red}2x-3})({\color{red}5-y})$$
Nun wie gewohnt auflösen:
\TNT{6.4}{
  $$4y-8-2xy+4x=10x-2xy-15+3y$$
  $$\Rightarrow y+7=6x$$
  \gleichungZZ{7}{6x-y}{8}{5x-3y}
  Additionsverfahren: Obere Gleichung mit 3 multiplizieren:
  \gleichungZZ{21}{18x-3y}{8}{5x-3y}
}%% END TNT

$$\mathbb{L}_{(x;y)} = \LoesungsRaumLang{\{(1; -1)\}}$$

Lösen Sie die folgenden Aufgaben mit der für Sie am besten geeigneten Methode, indem Sie die
Gleichungen in der Regel zunächst in die Grundform bringen.
  \AadBMTA{151}{9. b), d), e) und f)}

\newpage

%%%%%%%%%%%%%%%%%%%%%%%%%%%%%%%%%%%%%%%%%%%%%%%%%%%%%%%%%%%%%%%%%%%%%%%%%%%%%%%%%%%%%

\subsection{Graphische Methode}\index{graphische Methode!Gleichungssysteme}
Das Gleichsetzungsverfahren\totalref{lin_gl_gleichsetzungsverfahren} kann auch als Schnittpunkt zweier linearer Funktionen\totalref{lineare_funktionen}
aufgefasst werden. Betrachten wir folgende beiden Gleichungen:

\gleichungZZ{2x - 10y}{-10}{6x+15y}{60}

Lösen wir beide Gleichungen nach $y$ auf, so erhalten wir zwei Funktionsgleichungen:

\gleichungZZ{f: y}{\noTRAINER{...............}\TRAINER{0.2x + 1}}{g:y}{\noTRAINER{...............}\TRAINER{-\frac{2}{5}x + 4}}


Zeichnen Sie die beiden linearen Funktionen als Graph ins folgende
Koordinatensystem ein und lesen Sie die Lösung \TRAINER{bei $(5|2)$} ab:

\noTRAINER{
  \bbwGraph{-6}{8}{-1}{6}{}
}
\TRAINER{
  \bbwGraph{-6}{8}{-1}{6}{
    \bbwFuncC{\x * 0.2 + 1}{-5.5:8}{green}
    \bbwLetter{7.5,3}{f}{green}
    \bbwFuncC{-0.4*\x + 4}{-1:7}{blue}
    \bbwLetter{7.5,1}{g}{blue}
  }
}
\newpage


\GESO{\subsection*{Aufgabe Graphisch}
  Lösen Sie die folgende Aufgabe graphisch. Bringen Sie die beiden Gleichungen erst in die Form $y=ax+b$ der linearen Funktionen. Zeichnen Sie danach beide Funktionen in ein Koordinatensystem ($x$ von -6 bis 1 und $y$ von -3 bis 5). Schätzen Sie die Lösung, bevor Sie diese berechnen.
\GESOAadBMTA{151}{8. g)}
}

\olatLinkGESOKompendium{2.2.4.}{15}{39. bis 41.}

\newpage


\GESO{\subsection{Taschenrechner}\index{Taschenrechner!Gleichungssysteme}
Lineare Gleichungssysteme sind so zentral, dass viele heutige
Taschenrechner diese lösen kann. Hier nochmals Äpfel und Bananen:

\gleichungZZ{3a + 2b}{3.10}{4a + 5b}{6.00}

\GESO{Suchen Sie \tiprobutton{2nd}\tiprobutton{tan_sys-solv} und geben Sie die Zahlen 3, 2, 3.10, 4, 5 bzw. 6.00 in die entsprechenden Felder ein.}
\TALS{Definieren Sie das Gleichungssystem gls:=\{3a+2b=3.1, 4a+5b=6.0\}. Dies können Sie nun einfach mit solve(gls,\{a, b\}) auflösen lassen.}

\GESO{
\begin{bemerkung}{}{}
  Die Variable beim TI-30 PRO müssen bei 2x2-Gleichungssystemen $x$ und $y$  heißen.
\end{bemerkung}
}

\GESO{
  \subsubsection{Eingabe negativer Zahlen}
  Lösen Sie das folgende Gleichungssystem mit dem Taschenrechner.

  \gleichungZZ{-4x + (-8)y}{16}{3x-5y}{32}

  Beachten Sie die Eingabe negativer Zahlen auf dem TI 30 Pro
  MathPrint. Das negative Vorzeichen wird mit \tiprobutton{neg} eingegeben,
  wohingegen die Subtraktion mit dem einfachen \tiprobutton{minus} eingegeben wird.

  \TNT{2.4}{Lösung: $x=4$, $y=-4$\vspace{22mm}}

}

\GESO{
\aufgabenFarbe{  
  Prüfen Sie mit diesem Wissen von Seite 150ff die Resultate von Aufgabe 7. g) [$x=\frac{42}{61}$ und $y=\frac{60}{61}$], 7. b) [$x=\frac52$ und $y=-\frac{15}{2}$] 8. a) [$x=2$ und $y=6$] und 8. b) [$x=-2$ und $y=2$]
}}

\olatLinkGESOKompendium{2.2.1}{13}{27. bis 30.}
\newpage
}
\newpage

\GESO{%
% Spezialfälle linearer Gleichungssysteme (leere menge/unendlich viele Lösungen)
%
\subsubsection{Spezialfälle}
\textbf{TYP A:} Keine Lösung

Bestimmen Sie die Lösungsmenge des folgenden linearen Gleichungssystems:

\gleichungZZ{2x-y}{1}{4x-2y}{3}           

\TNT{2}{Additionsverfahren liefert $$(4x-2y)-2\cdot{}(2x-y) = 3 - 2\cdot{}(1) \Longrightarrow 0 = 1$$}

Die Gleichung hat keine Lösung. Geometrisch: Die Geraden sind parallel.

$$\LoesungsMenge{}_{(x;y)} = \LoesungsRaumLang{\{\}}$$

\textbf{TYP B:} Beliebig viele Lösungen (lineare Abhängigkeit)

Bestimmen Sie die Lösungsmenge des folgenden linearen Gleichungssystems:

\gleichungZZ{2x-y}{1}{4x-2y}{2}           

\TNT{2}{Additionsverfahren liefert $$(4x-2y)-2\cdot{}(2x-y) = 2 - 2\cdot{}(1) \Longrightarrow 0 = 0$$}

Hier gibt es unendlich viele Lösungen. Die entsprechenden Geraden sind zusammefallend.

$$\LoesungsMenge{}_{(x;y)} = \LoesungsRaumLen{50mm}{ \left\{(x;y) \text{ mit } y = 2x-1 \Longleftrightarrow x = \frac{y+1}2 \right\} }$$

%%\TRAINER{$y = \frac{9x-4.2}{4.8}$ ist die Funktionsgleichung beider Geraden.}
\newpage

\subsection*{Repetition: Lösungsmenge}\index{Lösungsmenge}
Auch graphisch kann die Lösungsmenge ermittelt werden. Die Angaben
sind ananolg zu den Gleichungen mit einer Variable.
\begin{center}
\begin{tabular}{c|c|c}
schneidend                 & parallel      &
zusammenfallend\\
 & & \\
\vspace{0.5mm}
$\gleichungZZ{y}{x}{y}{8}$ & $\gleichungZZ{y}{x+5}{y}{x+7}$ & $\gleichungZZ{y}{x+3}{2y}{2x+6}$   \\
 & & \\
$\times$                   & //            &   \textbf{/}      \\
 & & \\
$x=8$                      & $5=7$         &  $3=3$            \\
 & & \\
$\lx=\LoesungsRaumLen{20mm}{\{8\}}$                & $\lx=\LoesungsRaumLen{20mm}{\{\}}$    &  $\lx=\LoesungsRaumLen{20mm}{\mathbb{R}}$ \\
\end{tabular} 
\end{center}

%%%%%%%%%%%%%%%%%%%%%%%%%%%%%%%
%\subsection*{Aufgaben}
%%\TALSAadBMTA{125ff}{382. 383., 389. a) 410. a), 411. a) b)}
%\GESO{\olatLinkArbeitsblatt{Gleichungssysteme}{https://olat.bms-w.ch/auth/RepositoryEntry/6029794/CourseNode/112603866030292}{Kap. 1: a) b) c) g) h)}}
%\TALS{\olatLinkArbeitsblatt{Gleichungssysteme}{https://olat.bms-w.ch/auth/RepositoryEntry/6029786/CourseNode/112603866140343}{Kap. 1: a) b) c) g) h)}}
%\AadBMTA{153ff}{Falls nötig zuerst in Grundform bringen. Schreiben Sie $x$, $y$ und $z$ streng untereinander. Danach mit Taschenrechner lösen: 18. a) b) c) und e)}

%Optionale Aufgaben zu linearen Funktionen mit Taschenrechner:
%\AadBMTA{253}{24. b) c) d)  25. a)}

\newpage
}
\newpage

\TALS{\subsection{Lineare Abhängigkeit}\index{abhängig!linear}\index{Lineare
  Abhängigkeit}

Lösen Sie das folgende Gleichungssystem vorerst mit dem
Taschenrechner:

\gleichungZZ{9x-6y}{18}{30x-20y}{60}

Die Lösung ist nicht vielsagend.

Durch die Additionsmethode erhalten wir folgendes Gleichungssystem:

\TNT{3.2}{\gleichungZZ{90x-60y}{180}{90x-60y}{180}}

oder nach der Subtraktion:

\TNT{2}{$$0=0.$$}

Dies bedeutet, wir haben durch Äquivalenzumformungen nun zweimal die selbe
Gleichung da stehen. Wir können $x$ nur in Abhängigkeit von $y$
berechnen (mehr nicht):

$$x=\noTRAINER{\hspace{20mm}}\TRAINER{\frac{6+2y}{3}}$$

Oder wir können $y$ in Abhängigkeit von $x$ berechnen:
$$y=\noTRAINER{\hspace{20mm}}\TRAINER{\frac{3x-6}{2}}$$

Wichtig ist vor allem, dass Sie auch die Antwort des Taschenrechners verstehen!

\newpage}

\subsection{Substitution}\index{Substitution!Gleichungssysteme}\index{Lineare Gleichungssysteme!mit Substitutionsmethode}
Manchmal gibt es Situationen, in denen ein Gleichungssystem besser mit
einer Ersetzung (Substitution) als mit sturem Ausmultiplizieren gelöst
werden kann.

Betrachten Sie einmal das folgende Gleichungssystem:

\gleichungZZ{\frac{2a}{3+b} - \frac{b}{5-a}}{1}{\frac{3a}{b+3} + \frac{2b}{5-a}}{19}

Es ist offensichtlich, dass die Terme $\frac{a}{3+b}$ und
$\frac{b}{5-a}$ mehrfach vorkommen.

Hier bietet sich eine Ersetzung (Substitution) an:

$$X := \LoesungsRaum{\frac{a}{3+b}}$$

und

$$Y := \LoesungsRaum{\frac{b}{5-a}}$$

Das neue entstandene Gleichungssystem ist viel übersichtlicher und
auch einfacher zu lösen:

\TNT{2.4}{\gleichungZZ{2X - Y}{1}{3X+2Y}{19}}

Nach dem Auf"|lösen (\zB Taschenrechner) erhalten wir $X=\LoesungsRaum{3}$ bzw. $Y=\LoesungsRaum{5}$. Mit diesen Werten
können wir $a$ bzw. $b$ bestimmen.
\newpage

\textbf{Rücksubstitution}\index{Rücksubstitution}\,\\

\vspace{1mm}

\TNT{10.8}{

  \gleichungZZ{3}{\frac{a}{3+b}}{5}{\frac{b}{5-a}}

    und somit:

\gleichungZZ{9+3b}{a}{25-5a}{b}

...sortieren...

\gleichungZZ{a-3b}{9}{5a+b}{25}

Nach dem Auf"|lösen erhalten wir:

$$\mathbb{L}_{(a;b)} = \left\{ \left(\frac{21}{4} ;  -\frac{5}{4} \right)  \right\}$$
}%% END TNT


\subsection*{Aufgaben}
\TALSAadBMTA{127ff}{390. b), 391.}
\GESOAadBMTA{152}{12. c), 13. a)}
\olatLinkGESOKompendium{2.2.2}{14}{33. bis 34.}

\newpage

\TALS{\subsection{Gleichungssysteme mit Parametern}\index{Parameter!Gleichungssysteme}\index{Lineare Gleichungssysteme!mit Parametern}
Das folgende Gleichungssystem hat offensichtlich vier und nicht wie
üblich zwei Variable. Wenn wir das System hingegen nach $x$ und $y$
auf"|lösen, so können wir diese beiden Größen
\textbf{in Abhängigkeit} der beiden Parameter\footnote{Parameter =
  Bei- oder Nebenmaß} ($p$ und $q$) ausdrücken.

\gleichungZZ{3x-5y}{p-5q}{5x+6y}{16p+6q}

Zum Lösen können wir wieder mit der Additionsmethode vorgehen, indem
wir die erste Gleichung mit 6 und die zweite Gleichung mit 5 multiplizieren\footnote{Danke Melisa für den Tipp: Natürlich könnte man auch zuerst die erste Gleichung mit 5 und die zweite Gleichung mit 3 multiplizieren, um so das $x$ zu eliminieren. Doch mit Melisas Trick, fällt auch das $q$ direkt weg.
}:

\TNT{4}{\gleichungZZ{18x-30y}{6p-30q}{25x+30y}{80p+30q}}

Nach Addition der beiden Gleichungen erhalten wir

\TNT{2}{$43x=86p$,}

was uns zu $x=\LoesungsRaum{2p}$ bringt.
Einsetzen in eine der beiden ursprünglichen Gleichungen liefert $y=\LoesungsRaum{p+q}$.
Somit ist die Lösungsmenge


$$\mathbb{L}_{(x;y)} = \LoesungsRaumLang{\left\{(2p; p+q)\right\}}$$


\subsection*{Aufgaben}
%%\TALSAadBMTA{127ff}{392. a) b), 394. a)}
\AadBMTA{151}{10. a) b) c) und d)}

  \newpage}

\TALS{\subsection{Sonderfälle (Fallunterscheidung)}\index{Fallunterscheidung!Gleichungssysteme}\index{Sonderfälle!Gleichungssysteme}\index{Lineare Gleichungssysteme!Sonderfälle, Fallunterscheidung}

Berechnen Sie die Lösung für $(x,y)$ in Abhängigkeit von $a$:

\gleichungZZ{ax+ay}{1}{x-ay}{-1}

\TNTeop{
  Additionsverfahren:
  
  $$ax+x=0$$
  $$x(a+1)=0$$
  $$\Longrightarrow x=0, y=\frac1a$$


  $$\mathcal{L}_{(x;y)} = \left\{\left( 0 ; \frac1a \right)\right\}$$

  Sonderfälle (SF)

  \textbf{SF1}: Lösung für $a=0$ kann die Lösung für $y$ nicht
  stimmen (Division durch 0).
  Setzen wir $a=0$ in die erste Gleichung ein, so entsteht eine
  Falschaussage:

  Für $a=0$ gilt also $\mathcal{L}_{(x;y)} = \{\}$
  
  \textbf{SF2}: Lösung für $a+1=0$ gilt: $x$ ist beliebig und $a=-1$

  Aus der 2. Gleichung folgt mit $a=-1$: $x+y=-1$ und somit $y=-x-1$

}%% END TNTeop

%%%%%%%%%%%%% \ newpage %%%%%%%%%%%%%%%%%%

\subsection*{Aufgabe}
Berechnen Sie die Lösung für $(x,y)$ in Abhängigkeit von $m$ und $n$:

\gleichungZZ{2mx+y}{3}{-x-y}{n}

\TNTeop{
  Nach dem Additionsverfahren steht da:
  $$2mx-x=3+n$$
  $$x(2m-1) = 3+n$$
  
  Lösung: 
  $$(x,y) = \left(\frac{n+3}{2m-1} ,\frac{3+2nm}{1-2m} \right)$$

  Sonderfälle (SF):

   \textbf{SF 1}: \\
   Aus $x(2m-1)=3+n$ folgt: Ist $3+n$ nicht null, aber $2m-1=0$, so
   gibt es keine mögliche Lösung. 

   \textbf{SF 2}:\\
   Ist hingegen in $x(2m-1)=3+n$ der Term $3+n = 0$ (also $n=-3$), so
   muss entweder $x=0$ sein und es gilt $\mathcal{L}_{(x;y)}=
   \{(0;3)\}$

   oder aber es gilt dass $2m-1 = 0$ ist, und somit gäbe es für $x$
   beliebig viele Lösungen: $\mathcal{L}_{(x;y)} = \{(x;y) |
   x\in\mathbb{R} \land y=3-x\}$
}%% END TNT
\newpage

\subsection*{Aufgaben}
\TALSAadBMTA{153}{20. a) c) e)}

  \newpage}

\TALS{\subsection{Drei Unbekannte}\index{Drei unbekannte!Gleichungssysteme}\index{lineare Gleichungssysteme!mit drei Unbekannten}\index{Gleichungssysteme!lineare mit drei Unbekannten}
Das Additionsverfahren funktioniert auch mit mehr als zwei unbekannten. Betrachten wir dazu das folgende lineare Gleichungssystem:

\gleichungDD{2x -3y +4z}{33}{3x+2y-z}{-5}{5x-y-5z}{-12}

Zunächst eliminieren wir die Variable $x$. Dazu erzeugen wir die Gleichung (IV), indem wir die erste Gleichung mit 3 und die zweite Gleichung mit 2 multiplizieren und danach die Gleichungen voneinander abziehen.

(IV)

\TNT{3.6}{\gleichungZZ{6x-9y+12z}{99}{6x + 4y -2z}{-10}
  Daraus ergibt sich (IV):
  $$-13y + 14z = 109$$
}%%

Analog mit Gleichung (II)$\cdot{}5$ und (III)$\cdot{}3$:

(V)

\TNT{3.6}{\gleichungZZ{15x+10y-5z}{-25}{15x-3y-15z}{-36}
  Daraus ergibt sich (V):
  $$13y + 10z = 11$$
}
\newpage
Gleichung (IV) und (V) enthalten nur noch zwei Variable und können nach einem gewohnten Verfahren gelöst werden:

\TNT{12.0}{\gleichungZZ{-13y + 14z}{109}{13y+10z}{11}
  und somit $24z = 120$ und schließlich $z=5$. Dieses $z$ setzen wir in Gleichung (V) ein:
  $$-13y + 10\cdot{}(5) = 11$$
  und wir erhalten $y=-3$

  Zuletzt $z=5$ und $y=-3$ einsetzen in die Gleichung (I):
  $$2\cdot{}x -3\cdot{}(-3) + 4\cdot{}(5) = 33$$
  was uns zu $x=2$ bringt.

  $$\LoesungsMenge{}_{(x;y;z)} = \{(2; -3; 5)\}$$
}%% END TNT

\newpage}


\subsection{Textaufgaben zu Gleichungssystemen}
Typische Textaufgaben, die auf Gleichungssysteme führen werden in den folgenden Kapiteln angeschaut.


Verwenden Sie zum Lösen jeweils das \textit{Verfahren in sieben
Schritten}, das Sie bereits aus dem Kapitel zu linearen Gleichungen kennen\totalref{textaufgaben_verfahren_in_sieben_schritten}.

\subsubsection{Zahlen-Aufgaben}
  \subsubsection*{Aufgaben}
  \AadBMTA{155}{27. und 28.}
  \newpage
  
\subsubsection{Mischaufgaben}
\GESO{\TNT{8}{Vorzeigen Aufgabe «Farben mischen» aus dem Kompendium Seite
    14 Aufgabe 35\vspace{80mm}%%
}%% END TNT
}%% END GESO


\subsubsection*{Aufgaben zum Misch-Typ}
\AadBMTA{156}{37., 39., 38., 40., 41.}

\newpage


\subsubsection{Zinsaufgaben}
\GESO{Aufg 36 aus dem Kompendium Seite 14:}

Frau Gross hat ein Kapital in zwei Posten angelegt, einen zu 4\%
und einen zu 5\%. Nach ihrer Rechnung beträgt die Summe der
Jahreszinsen CHF 2’560. Das sind aber CHF 80 zu viel; sie hat
nämlich die Zinssätze verwechselt. Welche Posten hat sie zu welchem
Zinssatz angelegt?

\TNT{14}{$x$ = erster Posten (IN CHF)vor der Verzinsung zu 4\%. $y$ = zweiter
  Posten (in CHF) vor der Verzinsung zu 5\%.\vspace{140mm}}%% END TNT


  \subsubsection*{Aufgaben zum Zins-Typ}
  \AadBMTA{157}{44., 45.}

\newpage



\subsubsection{Ziffern-Aufgaben}

\paragraph{Ziffern-Tauschen}: Eine zweistellige Zahl $z$ ist
gesucht. Die Quersumme sei 9. Wenn ich hingegen zur Zahl 45 addiere,
so erhalte ich dasselbe, wie wenn ich der Zahl ihre Ziffern tausche.

\TNT{10.4}{
  Die Variable sind die Ziffern! Sinnvoll: $x$ = Zehnerziffer, $y$= Einerziffer.
  Somit ist die Quersumme = $x+y$.
  Die Zahl ist aber nicht(!) $xy$, das wäre ja $x\cdot{}y$. Die Zahl ist
  $10\cdot{}x + y$. Die Ziffern-vertauschte Zahl ist nun $10\cdot{}y + x$.

  Damit lassen sich die Gleichungen aufstellen:

  \gleichungZZ{10x+y+45}{10y+x}{x+y}{9}
  Ordnen:
  \gleichungZZ{9x-9y}{-45}{x+y}{9}
  Taschenrechner: $x=2$ und $y=7$. Somit ist die Zahl $z=27$.
  \vspace{30mm}
}%% END TNT


\subsection*{Aufgaben zum «Ziffern-Typ»}
\AadBMTA{155ff}{31. und 32.}

\newpage

\subsubsection{Rest-Aufgaben}\index{Rest!Textaufgaben mit Divisionsrest}\index{Divisionsrest}\index{teilen mit Rest}

\TRAINER{17:5 = 3 Rest 2 Was heißt das? Das heißt: 3*5 + 2 = 17!}

Was bedeutet «Teilen mit Rest»?

\TNT{7.2}{

  $$x : 5 = 3 \text{ Rest } 2 \Longleftrightarrow   3\cdot{} 5 + 2 = x $$
  Ergo $x=17$.\vspace{40mm}
}%% end TNT

\begin{gesetz}{Teilen mit Rest}{}
  Für $R<B$ (Rest $R$ kleiner als Divisor $B$) gilt:
  $$A : B = C \text{ Rest } R \Longleftrightarrow \frac{A - R}{B} = C$$
  $$A : B = C \text{ Rest } R \Longleftrightarrow B\cdot{}C + R = A $$
  \end{gesetz}

  Referenzaufgabe: Wenn wir eine Zahl durch 15 teilen, so erhalten wir
  8 Rest 4.
  \TNT{3.2}{$$z : 15 = 8 \text{ Rest } 4 \Longleftrightarrow
    15\cdot{}8 + 4 = z \Longrightarrow z = 124$$\vspace{20mm}}


%%\TALS{\subsection*{Aufgaben}
%%\TALSAadBMTA{133ff}{422. 432.}
%%}%% END TALS

  \subsubsection*{Aufgaben zum «Rest-Typ»}
\olatLinkGESOKompendium{2.2.3}{14}{37}
\AadBMTA{155}{30.}

\newpage

\subsubsection{Arbeit/Leistung-Aufgaben}\index{Arbeit}\index{Leistung}

\begin{gesetz}{Arbeit/Leistung}{}
  \begin{center}
    $\text{Leistung} = \frac{\mathrm{Arbeit}}{\mathrm{Zeit}}$
  \end{center}
  
  $$\Longleftrightarrow$$
  \begin{center}
    $\mathrm{Leistung} \cdot{} \text{Zeit} = \text{Arbeit}$
  \end{center}

  $$\Longleftrightarrow$$
  \begin{center}
    $\text{Zeit} = \frac{\mathrm{Arbeit}}{\mathrm{Leistung}}$
  \end{center}
 
\end{gesetz}

Normalerweise wird die Zeit in Sekunden, Stunden, Tagen etc. angegeben. Die
Arbeit hingegen kann irgend eine Verrichtung sein: Eine Wand
streichen, einen Quadratmeter schleifen, eine Schachtel herstellen,
einen Baum pflanzen, einen Auftrag erledigen, einen Kilometer zurücklegen\footnote{Ja, km/h\index{Kilometer pro Stunde}\index{Geschwindigkeit} ist auch eine Leistung: Somit sind Geschwindigkeitsaufgaben nur ein Spezialfall der Arbeit/Leistungs-Aufgaben.}, ...

\begin{beispiel}{Arbeit/Leistung}{}
  Eine Leistung könnte  $\left[\frac{\text{Auftrag}}{\text{Minute}}\right]$ sein.

  Eine Arbeitskraft (Person/Maschine) schafft einen Auftrag innerhalb von 20 Minuten, also

  $$\text{Leistung} = \frac1{20} \left[\frac{\text{Auftrag}}{\text{Minute}}\right]$$

  Wie lange braucht die Arbeitskraft für vier Aufträge?

  $$\text{Zeit} [\text{Min.}] = \frac{4[\text{Aufträge}]}{\frac{1}{20} \left[\frac{\text{Aufträge}}{\text{Minuten}}\right]} = 4\cdot{} 20 \text{ Minuten} = 80 \text{ Minuten}$$
\end{beispiel}


\newpage

Im folgenden Beispiel ist sowohl die Einheit der Arbeit, wie auch die
Einheit der Zeit klar gegeben.

\begin{beispiel}{Arbeit/Leistungs-Aufgabe}{}
  Um Desinfektionsmittel (Fläschchen) rasch herzustellen, hilft Tony mit.
  Tony und die Maschine schaffen zusammen in acht Stunden 800
  Desinfektionsfläschchen (=800 Stück).
  Heute fällt die Maschine nach fünf Stunden aus und Tony arbeitet noch
  drei Stunden alleine weiter. Am Abend sind 608 Fläschchen fertig.

  a) Wie viel Zeit braucht die Maschine alleine für 800 Fläschchen?

  b) Wie viele Fläschchen schafft Tony in einer Stunde?
\end{beispiel}

\TNT{8}{
  $m$ = Leistung Maschine $\left[\frac{\text{Fläschchen}}{\text{Stunde}}\right]$. Das heißt $m$ mal
  Anzahl Stunden = Anzahl Fläschchen.

  Analog $k$ = Leistung Tony.

  
  \gleichungZZ{(m+k) \cdot{} 8}{800}{(m+k)\cdot{}5 + k\cdot{}3}{608}

  $$m=64, k=36$$

  a) Die Maschine braucht alleine 12.5 Stunden für 800 Fläschchen, denn $[h] = \left[\frac{\text{Fläschchen}}{\frac{\text{Fläschchen}}{h}}\right]$.

  b) Tony schafft 36 Fläschchen pro Stunde.
}%% END TNT

\subsubsection*{Aufgaben Leistung/Arbeit}
Matura Niveau:
\AadBMTA{160}{72.}
\AadBMTA{190}{78., 79.}

\newpage
\GESO{
  \subsubsection*{vermischte Aufgaben}
  \olatLinkGESOKompendium{2.2.3.}{14}{35. bis 38.}
  \newpage
}%% END GESO
