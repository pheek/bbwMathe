
\subsection{Textaufgaben zu Gleichungssystemen}
Typische Textaufgaben, die auf Gleichungssysteme führen werden in den folgenden Kapiteln angeschaut.


Verwenden Sie zum Lösen jeweils das \textit{Verfahren in sieben
Schritten}, das Sie bereits aus dem Kapitel zu linearen Gleichungen kennen\totalref{textaufgaben_verfahren_in_sieben_schritten}.

\subsubsection{Zahlen-Aufgaben}
  \subsubsection*{Aufgaben}
  \AadBMTA{155}{27. und 28.}
  \newpage
  
  \subsubsection{Mischaufgaben}

  Referenzaufgabe:

  $100$ g der Kaffeesorte I kosten CHF $4.30$ und $50$ g der Kaffeesorte
  II kosten CHF $3.15$.

  Eine Mischung aus Sorte I und II soll $200$ g wiegen und die $200$ g
  sollen auf einen Preis von CHF $9.90$ kommen.

  Wie viel g von jeder Sorte sind in der Mischung vorhanden?

  \TNTeop{
    Sei $x$ die Anzahl g der Sorte I in der $200$ g Mischung.

    Sei $y$ die Anzahl g der Sorte II in der $200$ g Mischung.

    Somit gilt $x+y$ = $200$ g.

    Der Grammpreis der Sorte I ist CHF $\frac{4.30}{100}$ und der
    Grammpreis der Sorte II ist CHF $\frac{3.15}{50}$.

    Kosten aus der Sorte I: $\frac{4.30}{100}\cdot{}x$ CHF.

    Kosten aus der Sorte II: $\frac{3.15}{50}\cdot{}y$ CHF.

    Somit haben wir folgendes Gleichungssystem:

    \gleichungZZ{x+y}{200}{\frac{4.30}{100}\cdot{}x +
      \frac{3.15}{50}\cdot{}y}{9.90}

    Dies \zB mit TR lösen: $135$ g (=$x$) in der Mischung von Sorte I
    und $65$ g (=$y$) in der Mischung von Sorte II.
    
  }%% end TNT eop

  
  %%%%%%%%%%%%%%%%%% \newpage %%%%%%%%%%%%%%%%%%%%%%5
  
\subsubsection*{Aufgaben zum Misch-Typ}
\AadBMTA{156}{37., 39., 38., 40., 41.}

\newpage


\subsubsection{Zinsaufgaben}
\GESO{Aufg 36 aus dem Kompendium Seite 14:}

Frau Gross hat ein Kapital in zwei Posten angelegt, einen zu 4\%
und einen zu 5\%. Nach ihrer Rechnung beträgt die Summe der
Jahreszinsen CHF 2\,560. Das sind aber CHF 80 zu viel; sie hat
nämlich die Zinssätze verwechselt. Welche Posten hat sie zu welchem
Zinssatz angelegt?

\TNTeop{$x$ = erster Posten (IN CHF)vor der Verzinsung zu 4\%. $y$ = zweiter
  Posten (in CHF) vor der Verzinsung zu 5\%.

  (I) = Ihre Rechnung ; (II) korrekte Rechnung
%%  \gleichungZZ{0.04x + 0.05y}{2560 - 80}{0.05x+0.04y}{2560}
  \gleichungZZ{0.05x+0.04y}{2560}{0.04x + 0.05y}{2560 - 80}

  TR:

  $$x=24000; y=32000$$

  Sie hat effektiv den ersten Posten a CHF 32\,000.- zu 4\% angelegt und den
  zweiten Posten a CHF 24\,000.- zu 5\% angelegt. 

}%% END TNTeop
%% implicit \newpage

  \subsection*{Aufgaben zum Zins-Typ}
  \AadBMTA{157}{44., 45.}

\newpage



\subsubsection{Ziffern-Aufgaben}

\paragraph{Ziffern-Tauschen}: Eine zweistellige Zahl $z$ ist
gesucht. Die Quersumme sei 9. Wenn ich hingegen zur Zahl 45 addiere,
so erhalte ich dasselbe, wie wenn ich der Zahl ihre Ziffern tausche.

\TNT{10.4}{
  Die Variable sind die Ziffern! Sinnvoll: $x$ = Zehnerziffer, $y$= Einerziffer.
  Somit ist die Quersumme = $x+y$.
  Die Zahl ist aber nicht(!) $xy$, das wäre ja $x\cdot{}y$. Die Zahl ist
  $10\cdot{}x + y$. Die Ziffern-vertauschte Zahl ist nun $10\cdot{}y + x$.

  Damit lassen sich die Gleichungen aufstellen:

  \gleichungZZ{10x+y+45}{10y+x}{x+y}{9}
  Ordnen:
  \gleichungZZ{9x-9y}{-45}{x+y}{9}
  Taschenrechner: $x=2$ und $y=7$. Somit ist die Zahl $z=27$.
  \vspace{30mm}
}%% END TNT


\subsection*{Aufgaben zum «Ziffern-Typ»}
\AadBMTA{155ff}{31. und 32.}

\newpage

\subsubsection{Rest-Aufgaben}\index{Rest!Textaufgaben mit Divisionsrest}\index{Divisionsrest}\index{teilen mit Rest}

\TRAINER{17:5 = 3 Rest 2 Was heißt das? Das heißt: 3*5 + 2 = 17!}

Was bedeutet «Teilen mit Rest»?

\TNT{7.2}{

  $$x : 5 = 3 \text{ Rest } 2 \Longleftrightarrow   3\cdot{} 5 + 2 = x $$
  Ergo $x=17$.\vspace{40mm}
}%% end TNT

\begin{gesetz}{Teilen mit Rest}{}
  Für $R<B$ (Rest $R$ kleiner als Divisor $B$) gilt:
  $$A : B = C \text{ Rest } R \Longleftrightarrow \frac{A - R}{B} = C$$
  $$A : B = C \text{ Rest } R \Longleftrightarrow B\cdot{}C + R = A $$
  \end{gesetz}

  Referenzaufgabe: Wenn wir eine Zahl durch 15 teilen, so erhalten wir
  8 Rest 4.
  \TNT{3.2}{$$z : 15 = 8 \text{ Rest } 4 \Longleftrightarrow
    15\cdot{}8 + 4 = z \Longrightarrow z = 124$$\vspace{20mm}}


%%\TALS{\subsection*{Aufgaben}
%%\TALSAadBMTA{133ff}{422. 432.}
%%}%% END TALS

  \subsubsection*{Aufgaben zum «Rest-Typ»}
\olatLinkGESOKompendium{2.2.3}{14}{37}
\AadBMTA{155}{30.}

\newpage

\subsubsection{Arbeit/Leistung-Aufgaben}\index{Arbeit}\index{Leistung}

\begin{gesetz}{Arbeit/Leistung}{}
  \begin{center}
    $\text{Leistung} = \frac{\text{Arbeit}}{\text{Zeit}}$
  \end{center}
  
  $$\Longleftrightarrow$$
  \begin{center}
    $\text{Leistung} \cdot{} \text{Zeit} = \text{Arbeit}$
  \end{center}

  $$\Longleftrightarrow$$
  \begin{center}
    $\text{Zeit} = \frac{\text{Arbeit}}{\text{Leistung}}$
  \end{center}
 
\end{gesetz}

Normalerweise wird die Zeit in Sekunden, Stunden, Tagen etc. angegeben. Die
Arbeit hingegen kann irgend eine Verrichtung sein: Eine Wand
streichen, einen Quadratmeter schleifen, eine Schachtel herstellen,
einen Baum pflanzen, einen Auftrag erledigen, einen Kilometer zurücklegen\footnote{Ja, km/h\index{Kilometer pro Stunde}\index{Geschwindigkeit} ist auch eine Leistung: Somit sind Geschwindigkeitsaufgaben nur ein Spezialfall der Arbeit/Leistungs-Aufgaben.}, ...

\begin{beispiel}{Arbeit/Leistung}{}
  Eine Leistung könnte  $\left[\frac{\text{Auftrag}}{\text{Minute}}\right]$ sein.

  Eine Arbeitskraft (Person/Maschine) schafft einen Auftrag innerhalb von 20 Minuten, also

  $$\text{Leistung} = \frac1{20} \left[\frac{\text{Auftrag}}{\text{Minute}}\right]$$

  Wie lange braucht die Arbeitskraft für vier Aufträge?

  $$\text{Zeit} [\text{Min.}] = \frac{4[\text{Aufträge}]}{\frac{1}{20} \left[\frac{\text{Aufträge}}{\text{Minuten}}\right]} = 4\cdot{} 20 \text{ Minuten} = 80 \text{ Minuten}$$
\end{beispiel}
%%
\TALS{\newpage
  Beispiel aus der Physik:
  \TNTeop{$$kW = \frac{kWh}{h}; kW\cdot{}h = kWh; h = \frac{kWh}{kW}$$}%%
}%%
%%
\newpage

Im folgenden Beispiel ist sowohl die Einheit der Arbeit, wie auch die
Einheit der Zeit klar gegeben.

\begin{beispiel}{Arbeit/Leistungs-Aufgabe}{}
  Um Desinfektionsmittel (Fläschchen) rasch herzustellen, hilft Kim mit.
  Kim und die Maschine schaffen zusammen in acht Stunden 800
  Desinfektionsfläschchen (=800 Stück).
  Heute fällt die Maschine nach fünf Stunden aus und Kim arbeitet noch
  drei Stunden alleine weiter. Am Abend sind 608 Fläschchen fertig.

  a) Wie viel Zeit braucht die Maschine alleine für 800 Fläschchen?

  b) Wie viele Fläschchen schafft Kim in einer Stunde?
\end{beispiel}

\TNTeop{
Einheiten: $$\left[\frac{\text{Fläschchen}}{\text{Stunde}}\right] =
\frac{[\text{Fläschchen}]}{[\text{Stunde}]}, \text{ denn Leistung} = \frac{\text{Arbeit}}{\text{Zeit}}$$
  
  $m$ = Leistung Maschine $\left[\frac{\text{Fläschchen}}{\text{Stunde}}\right]$. Das heißt $m$ mal
  Anzahl Stunden = Anzahl Fläschchen, welche die Maschine prozuziert.

  Analog $k$ = Leistung Kim.

  
  \gleichungZZ{(m+k) \cdot{} 8}{800}{(m+k)\cdot{}5 + k\cdot{}3}{608}

  $$m=64, k=36$$

  a) Die Maschine braucht alleine 12.5 Stunden für 800 Fläschchen, denn $[h] = \left[\frac{\text{Fläschchen}}{\frac{\text{Fläschchen}}{h}}\right]$.

  b) Kim schafft 36 Fläschchen pro Stunde.
}%% END TNTeop

%%%%%%%%%%%%%%%%%%%%%%%%%%%%%%%%%%%%%%%%%%%%%%%%%%%%%%%%%%%%%%%%%%%%%%%%%

\subsubsection*{Aufgaben Leistung/Arbeit}
Matura Niveau:
\AadBMTA{160}{72.}
\AadBMTA{190}{78., 79.}

\GESO{
  \subsubsection*{vermischte Aufgaben}
  \olatLinkGESOKompendium{2.2.3.}{14}{35. bis 38.}
  \newpage
}%% END GESO
