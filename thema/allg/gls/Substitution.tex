\subsection{Substitution}\index{Substitution!Gleichungssysteme}\index{lineare Gleichungssysteme!mit Substitutionsmethode}
Manchmal gibt es Situationen, in denen ein Gleichungssystem besser mit
einer Ersetzung (Substitution) als mit sturem Ausmultiplizieren gelöst
werden kann.

Betrachten Sie einmal das folgende Gleichungssystem:

\gleichungZZ{\frac{2a}{3+b} - \frac{b}{5-a}}{1}{\frac{3a}{b+3} + \frac{2b}{5-a}}{19}

Es ist offensichtlich, dass die Terme $\frac{a}{3+b}$ und
$\frac{b}{5-a}$ mehrfach vorkommen.

Hier bietet sich eine Ersetzung (Substitution) an:

$$X := \LoesungsRaum{\frac{a}{3+b}}$$

und

$$Y := \LoesungsRaum{\frac{b}{5-a}}$$

Das neue entstandene Gleichungssystem ist viel übersichtlicher und
auch einfacher zu lösen:

\TNT{2.4}{\gleichungZZ{2X - Y}{1}{3X+2Y}{19}}

Nach dem Auf"|lösen (\zB Taschenrechner) erhalten wir $X=\LoesungsRaum{3}$ bzw. $Y=\LoesungsRaum{5}$. Mit diesen Werten
können wir $a$ bzw. $b$ bestimmen.
\newpage

\textbf{Rücksubstitution}\index{Rücksubstitution!Gleichungssysteme}\,\\

\vspace{1mm}

\TNT{14.4}{

  \gleichungZZ{3}{\frac{a}{3+b}}{5}{\frac{b}{5-a}}

    und somit:

\gleichungZZ{9+3b}{a}{25-5a}{b}

...sortieren...

\gleichungZZ{a-3b}{9}{5a+b}{25}

\GESO{... und mit Taschenrechner lösen:}

\TALS{Nach dem Auf"|lösen erhalten wir:

(2. Gleichung mit 3 Multiplizieren:)

\gleichungZZ{a-3b}{9}{15a+3b}{75}, dann addieren:

$16a = 84$, und somit $a = \frac{21}4$

$b$: $b=25-5a = 25 - 5\frac{21}4 = \frac{100 - 5\cdot{}21}4 =
\frac{100-105}4 = \frac{-5}4$
}%% end TALS


$$\LoesungsMenge{}_{(a;b)} = \left\{ \left(\frac{21}{4} ;  -\frac{5}{4}
\right)  \right\}$$
\vspace{38mm}
}%% END TNT
\newpage


\subsection*{Aufgaben}
%%\TALSAadBMTA{127ff}{390. b), 391.}

\aufgabenFarbe{Lösen Sie das folgende lineare Gleichungssystem mit
  einer geeigneten Substitution:
  \gleichungZZ{2\cdot{}\frac{a}{a-b} + 3\cdot{}\frac{b}{a-2} }{18}{\frac{3a}{a-b} +
  \frac{2b}{2-a}}{1}
}
\TNT{15.6}{Substituiere:
  $x=\frac{a}{a-b}$ und $y=\frac{b}{a-2}$
  Löse
  \gleichungZZ{2x+3y}{18}{3x-2y}{1} $\Longrightarrow x=3; y=4$
  Resub:
  $3=\frac{a}{a-b}$ und $4 = \frac{b}{a-2}$
  Löse
  \gleichungZZ{2a-3b}{0}{4a-b}{8}
  Lösung: $a= \frac{12}5$ und $b=\frac85$
  \vspace{96mm}
} %% end TNT

%\AadBMTA{152}{12. c), 13. a)}
%\olatLinkGESOKompendium{2.2.2}{14}{33. bis 34.}
\newpage
\subsection*{Aufgaben}
%%\TALSAadBMTA{125ff}{382. 383., 389. a) 410. a), 411. a) b)}
\GESO{\olatLinkArbeitsblatt{Gleichungssysteme}{https://olat.bms-w.ch/auth/RepositoryEntry/6029794/CourseNode/112603866030292}{Kap. 8: a) b) c)}}
\TALS{\olatLinkArbeitsblatt{Gleichungssysteme}{https://olat.bms-w.ch/auth/RepositoryEntry/6029786/CourseNode/112603866140343}{Kap. 8: a) b) c) d)}}
%\AadBMTA{153ff}{Falls nötig zuerst in Grundform bringen. Schreiben Sie $x$, $y$ und $z$ streng untereinander. Danach mit Taschenrechner lösen: 18. a) b) c) 
\newpage
