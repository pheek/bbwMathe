%%
%% 2019 07 04 Ph. G. Freimann
%%


\section{Grundlagen Arithmetik und Algebra}

\textbf{Kompendium}\footnote{Auszug aus dem Original-``Kompendium
  Mathematik'' \cite{kompendium18} von Susanne Wagner und Urs Vonesch. Rückmeldungen und Inputs zu dieser Kopie an \texttt{philipp.freimann@bbw.ch}}

Erlaubt ist ein Taschenrechner \textbf{ohne} Computeralgebrasystem
(CAS).


\subsection{Grundlagen}
Benennen Sie die folgenden Terme mit dem richtigen Begriff
(Jeweils einer aus: \textit{Summe}, \textit{Differenz}, \textit{Produkt},
\textit{Quotient}, \textit{Potenz}):


\begin{multicols}{3}
\begin{enumerate}[label=\alph*)]
 \item $(3+x)\cdot{}2 + y$ \TRAINER{Summe}
 \item $(4a)^x$ \TRAINER{Potenz}
 \item$4a^x$ \TRAINER{Produkt}
 \item$(x-2y^4z)^2$ \TRAINER{Potenz}
 \item$2a^7 - 4bc(z-1)^3$ \TRAINER{Differenz}
 \item$(2a -b) : x + z : 4$ \TRAINER{Summe}
 \item$(a-4)(z+3)$ \TRAINER{Produkt}
 \item$(2a-b):(x+z:4)$ \TRAINER{Quotient}
\end{enumerate}
\end{multicols}

\subsubsection{Zahlmengen und zugehörige Grundoperationen}
Ordnen Sie die folgenden Zahlen der am weitesten links stehenden
Zahlmenge zu:

$$\mathbb{N} \subset \mathbb{Z} \subset \mathbb{Q} \subset \mathbb{R}$$

\begin{multicols}{2}
\begin{enumerate}[label=\alph*)]
 \item$-\sqrt{\frac{16}{2}}$ \TRAINER{$\mathbb{R}$}
 \item$|-\pi|$ \TRAINER{$\mathbb{R}$}
 \item$1.\overline{25}$ \TRAINER{$\mathbb{Q}$}
 \item$\frac{\sqrt{8}}{\sqrt{2}}$ \TRAINER{$\mathbb{N}$}
\end{enumerate}
\end{multicols}


\subsubsection{Ordnungsrelationen}
Setzen Sie zwischen die Terme (an die Stelle der drei Punkte) die
richtigen Relationszeichen ($<$, $=$, $>$):

\begin{multicols}{2}
\begin{enumerate}[label=\alph*)]
 \item$|5-2| ... |2-5|$ \TRAINER{$=$}
 \item$-\sqrt{3} ... -\sqrt{2}$ \TRAINER{$<$}
 \item$-3 ... |-3|$ \TRAINER{$<$}
 \item$\frac{1}{3} ... \frac{1}{4}$ \TRAINER{$>$}
 \item$\frac{2}{3} ... \frac{3}{4}$ \TRAINER{$<$}
 \item$2.\overline{9}$ ... 3\TRAINER{$=$}
 \item$-\frac{1}{3} ... -\frac{1}{4}$ \TRAINER{$<$}
\end{enumerate}
\end{multicols}


\subsubsection{Absolutbetrag}
Lösen Sie in der Grundmenge $\mathbb{Z}$:

\begin{multicols}{2}
\begin{enumerate}[label=\alph*)]
 \item$|x+4| = 5$ \TRAINER{$\LoesungsMenge{}=\{1,-9\}$}
 \item$|2x-1| = 0$ \TRAINER{$\LoesungsMenge{}=\{0.5\}$}
 \item$|-x + 7| = 10$ \TRAINER{$\LoesungsMenge{}=\{-3, 17\}$}
 \item$|-x|=-1$ \TRAINER{$\LoesungsMenge{}=\{\}$}
\end{enumerate}
\end{multicols}

\subsubsection{Runden}\index{Runden!Aufgaben}
Runden Sie auf 2 Dezimalen (Nachkommastellen):

\begin{multicols}{3}
\begin{enumerate}[label=\alph*)]
 \item$2.245$ \TRAINER{2.25}
 \item$2.995$ \TRAINER{3.00}
 \item$\sqrt{3}$ \TRAINER{1.73}
\end{enumerate}
\end{multicols}

Runden Sie auf vier signifikante Stellen:

\begin{multicols}{2}
\begin{enumerate}[label=\alph*)]
 \item 12.05649  \TRAINER{12.06}
 \item 0.028249 \TRAINER{0.02825}
 \item 2\,468\,900  \TRAINER{2\,469\,000}
 \item 1\,370.5   \TRAINER{1\,371}
\end{enumerate}
\end{multicols}

Runden Sie auf 2 Dezimalstellen in der wissenschaftlichen Notation:
\begin{multicols}{2}
\begin{enumerate}[label=\alph*)]
 \item 12.05649  \TRAINER{1.21 $\cdot 10^1$}
 \item 0.028249 \TRAINER{2.82 $\cdot 10^{-2}$}
 \item 2\,468\,900  \TRAINER{2.47 $\cdot 10^{6}$}
\end{enumerate}
\end{multicols}


%%%%%%%%%%%%%%%%%%%%%%%%%%%%%%%%%%%%%%%%%%%%%%%%%%
%% Summenzeichen
\subsection{Grundoperationen mit algebraischen Termen}
Schreiben Sie die Summanden hin und berechnen Sie den Wert des Ausdrucks:
\begin{enumerate}[label=\alph*)]
 \item $$\sum_{k=2}^{k=4}{(3k+2^k)}$$ \TRAINER{(6+4) + (9+8) + (12+16) = 10+17+28=55}
 \item $$\sum_{i=4}^{i=7}(3i) + 6$$   \TRAINER{((12) + (15) + (18) +(21) )+6=72}
\end{enumerate}


Schreiben Sie als Summen:

\begin{multicols}{2}
\begin{enumerate}[label=\alph*)]
\item $\left(5m-\frac{1}{2}n\right)^2$
\item $(t-9s^2)\cdot(t+9s^2)$
  \item $(3a + b^4)^2$
  \end{enumerate}
\end{multicols}

Faktorzerlegung mit Hilfe der Binomischen Formeln\index{Binomische
  Formeln!Aufgaben}

\begin{multicols}{2}
\begin{enumerate}[label=\alph*)]
\item $4x^2-y^2$
\item $36a^2 + 36ab + 9b^2$
\item $4x^4-28x^2y+49y^2$
\item $a^3-a$
\item $x^3-6x^2y+9xy^2$
  \item $z^4 -3^2$
\end{enumerate}
\end{multicols}

Faktorzerlegung durch planmäßiges Probieren

\begin{multicols}{2}
\begin{enumerate}[label=\alph*)]
\item $a^2 + 2a -15$
\item $a^3-a^2-6a$
\item $a^2 - 20a + 75$
  \item $a^2 -19a + 48$
\end{enumerate}
\end{multicols}

Faktorzerlegung durch teilweises Ausklammern und Ausklammern von
Klammern (Zerlegen Sie in möglichst viele Faktoren):

\begin{multicols}{2}
\begin{enumerate}[label=\alph*)]
\item $ab(x-2y)-b(x-2y)$
\item $x(y-2z)-(y-2z)$
\item $(a-2b)(m+n)+(a-2b)(3m+n)$
\item $3a-6ab+c-2bc$
  \item $ac-ad+bc-bd$
\end{enumerate}
\end{multicols}

Bruchterme umformen

\begin{multicols}{2}
\begin{enumerate}[label=\alph*)]
\item $\frac{st-3t^2}{s^2-st} + \frac{3s^2-3st-st+t^2}{s^2-2st+t^2}$
\item $\frac{\frac{a^2-16c^2}{8a^2}}{\frac{a-4c}{4a}}$
\item $\left(\frac{1}{a^2} - \frac{1}{b^2}\right) \cdot
    \left(\frac{a}{a+b} + \frac{b}{a-b}\right)$
\item $\left( \frac{a+4}{a} - \frac{b+4}{b}\right) : \frac{a-b}{a}$
    
\end{enumerate}
\end{multicols}


\subsection{Potenzen}

Schreiben Sie als Bruch ohne negative Exponenten:

\begin{multicols}{2}
\begin{enumerate}[label=\alph*)]
\item $2x^{-3}$
  \item $ab^{-5}$
\end{enumerate}
\end{multicols}

Schreiben Sie folgende Bruchzahlen als Dreierpotenzen:

\begin{multicols}{2}
\begin{enumerate}[label=\alph*)]
  \item $\frac{1}{9}$
  \item $\frac{1}{27}$
    \item $\frac{1}{81}$
\end{enumerate}
\end{multicols}

Lösen Sie die Exponentialgleichungen durch Erzeugen gleicher Basis:

\begin{multicols}{3}
\begin{enumerate}[label=\alph*)]
\item $2^x=\frac{1}{8}$
\item $2^{2x}=\frac{1}{8}$
\item $4^x=\frac{1}{32}$
\item $3^x=\frac{1}{81}$
  \item $9^x=\frac{1}{3}$
  \item $27^{2x}=\frac{1}{9}$
  \item $\left(3^x\right)^6 = \frac{1}{81}$
    \item $50\cdot 5^{n-1} + 3\cdot 5^{n+1} = 5^x$
    \item $\frac{25^k}{125^3} = 5^x$
      \item $2^x=\frac{8^4}{4^8}$
\end{enumerate}
\end{multicols}


Wenden Sie die Potenzgesetze an und schreiben Sie so einfach wie
möglich:
\begin{multicols}{3}
\begin{enumerate}[label=\alph*)]
\item $a\cdot a^{2x+3} : a^{1-x}$
\item $\left(\frac{x}{y}\right)^7 : \left(\frac{-x}{y}\right)^3$
\item $(-b)^6 \cdot (-b^8)$
\item $(-x^3)^4$
\item $\left( -(a^{-1})^{-2} \right)^6$
\item $(-a)^4 : (-a^{10})$
  \item $4^k \cdot \left( \frac{1}{2} \right)^k \cdot \left(
    \frac{1}{3} \right)^{-k}$
\end{enumerate}
\end{multicols}


Vereinfachen Sie:
$$\left(\frac{3x^{-2}y^2}{4x^{-4}y^3}\right)^{-2} : \left(\frac{2x^{-1}}{3xy^{-2}}\right)^3$$


Vereinfachen Sie folgende Terme. Resultat in Wurzelschreibweise:

\begin{multicols}{2}
\begin{enumerate}[label=\alph*)]
\item $\left(\frac{1}{a}\right)^{-\frac{1}{4}}$
\item $\sqrt{x}\cdot \sqrt[3]{x^4} \cdot \sqrt[6]{x^3}$
  \item $3a\cdot \sqrt[3]{9a^2}$
\item $\sqrt[4]{b\cdot \sqrt[3]{b^2\cdot \sqrt{b}}}$
\end{enumerate}
\end{multicols}

%%%%%%%%%%%%%%%%%%%%%%%%%%%%%%%%%%%%%%%%%%%%%%%%%%%%%%%%% 
\subsection{Zehnerlogarithmen}

Berechnen Sie die Exponenten. Resultate exakt, wenn nötig mit Hilfe von
Zehnerlogarithmen:

\begin{multicols}{2}
\begin{enumerate}[label=\alph*)]
\item $a^x=b$
\item $4^x=12$
  \item $4\cdot 3^{x+1}=6$
\end{enumerate}
\end{multicols}x
