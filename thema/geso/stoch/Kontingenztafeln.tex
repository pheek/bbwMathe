%%
%% Stochastik Grundlagen
%% 2020 - 08 - 03 φ
%%

\section{Kontingenztafeln}\index{Kontingenztafel}\index{Vierfeld Tafel}
Auch \textbf{Vier-Feld-Tafeln} genannt.


Das folgende Beispiel zählt die (fiktive) Haarfarbe in zwei BMS-Klassen:

\begin{tabular}{l|c|c|c|c|c}
Geschlecht\, $\backslash$ Haarfarbe  &  blond           & schwarz            & andere           & abs. Häufigkeit    & rel. Häufigkeit \\ \hline
weiblich                             &           6      &          2       &          10      &       \TRAINER{18} & \TRAINER{56.3\%}\\ \hline 
männlich                             &           7      &          1       &           6      &       \TRAINER{14} & \TRAINER{43.8\%}\\ \hline
absolute Häufigkeit                  & \TRAINER{13}     & \TRAINER{3}      & \TRAINER{16}     &       \TRAINER{32} &  *****          \\ \hline
relative Häufigkeit                  & \TRAINER{40.6\%} & \TRAINER{9.38\%} & \TRAINER{50.0\%} &   *****            &  100\%          \\ \hline
\end{tabular}


Kontingenztafeln werden meist auch Vierfeld-Tafeln genannt, denn meist sind in der Mitte bei den absoluten (noch nicht summierten) Häufigkeiten
vier Felder vorhanden. Es können jedoch (wie in obigem Beispiel) natürlich mehrere Spalten bzw. Zeilen als zwei vorkommen.
\newpage

\subsection{Referenzaufgabe: Krankheitstest}
Es werden in einem Spital 123 Personen mit Symptomen auf eine Krankheit getestet. Bei 99 Personen fiel der Test positiv aus. Trotzdem sind vier von diesen positiv getesteten Personen gesund und der Test hat falsch angegeben (\textit{false positive}\index{false positive}). Zweiundzwanzig (von allen 123) sind jedoch gesund und auch negativ getestet worden.

Wie viele Personen wurden negativ getestet, sind aber dennoch krank (\textit{false negative})?\index{false negative}

Zeichnen Sie eine Kontingenztafel (Vier-Feld-Tafel) und geben Sie die
Anzahlen absolut, sowie auch relativ an.

\TNT{6}{

  \begin{tabular}{|l|r|r|r|r|}\hline
                 & Test positiv & Test negativ & Total & relativ (\%) \\\hline
    krank        & 95           & 2            & 97    & 78.9 \%      \\\hline    
    gesund       & 4            & 22           & 26    & 21.1 \%      \\\hline
    Total        & 99           & 24           & 123   &  --- \%      \\\hline
    relativ (\%) & 80.5\%       &19.5\%        & ---   &   100\%      \\\hline

    \end{tabular} 
}%% END TNT

a) Absolut: \LoesungsRaum{2}
b) In \%  : \LoesungsRaum{1.6\%}

\subsection*{Aufgaben}
\olatLinkGESOKompendium{5.6}{51}{31. und 33.}
\newpage
