%%
%% Stochastik Grundlagen
%% 2020 - 08 - 03 φ
%%


\subsection{Einstiegsbeispiel}
Als Einstiegsbeispiel dient die folgende Wanderung:

\vspace{5mm}

\bbwGraphic{12cm}{geso/stoch/img/wanderung.png}

Auf wie viele Arten kann der Wanderer von Wesen (im Westen) nach Obertal (im Osten) gelangen, wenn er ausschließlich von West nach Ost wandern will?

Die Antwort kann durch Abzählen (oder eine Kombination von Abzählen und Multiplizieren) gefunden werden:

\TNTeop{$3\cdot{}(1+(2\cdot3)) + 3 = 24$ Möglichkeiten. Mögliche
  Erklärung: Bis «L» sind 3 Wege offen. Bis «R» sind total 3*2 + 1,
  also sieben Wege möglich. Um nach «O» zu gelangen sind nun 7*3 + die
  drei Wege ab Lisibach und dann direkt, also total 24 Wege möglich. Parallele Wege werden addiert;
  serielle Wegmöglichkeiten multiplizieren sich.}%% END TNT eop
%% implicit \newpage
%%%%%%%%%%%%%%%%%%%%%%%%%%%%%%%%%%%%%%%%%%%%%%%%%%%%%%%%%%%%%%%%%%%%5
\subsection{Überblick}
\subsubsection{Produkt- und Summenregel}

\textbf{Produktregel}: Müssen zwei Ereignisse unabhängig voneinander
(quasi hintereinander) ausgeführt werden müssen, so sprechen wir von
«\textbf{und}» oder der Produktregel (siehe Seite \pageref{experimentePfadUndSummenregel}).

\textbf{Summenregel}: Ist von zwei alternativen Ergebnissen die Rede,
welche «\textbf{nur entweder / oder»} eintreffen, so sprechen wir von
Summenregel (siehe Seite \pageref{experimentePfadUndSummenregel}).

\subsubsection{Taschenrechner}
Taschenrechner Abkürzungen \tiprobutton{ncrnpr}.


\begin{bbwFillInTabular}{l|c||c|c|}
            & mit Wiederholung                & \multicolumn{2}{c}{ohne Wiederholung}     \\\hline
            &   $n$ unabhängig von $k$         &  $n > k$ \noTRAINER{\,\,\,\,\,\,}  &  $n=k$ \\\hline
Variation   &   \TRAINER{\tiprobutton{xhoch}}  & \TRAINER{\texttt{nPr}}          & \TRAINER{\texttt{!}} \\\hline
Kombination &   ($n+k-1$) \texttt{nCr} $k$     & \TRAINER{\texttt{nCr}} & \TRAINER{1} \\\hline
\end{bbwFillInTabular}

Dabei bedeuten:

\textbf{Mit Wiederholung} = \textbf{mit Zurücklegen}: Die ausgewählten $n$ Obejkte können
mehrfach verwendet werden (Beispiel Zahlenschloss) (Seite \pageref{kombiVariation}).

\textbf{Ohne Wiederholung} = \textbf{ohne Zurücklegen}: Die ausgewählten $n$ Objekte kommen
in der Stichprobe genau einmal vor (Beispiel: Personen auswählen)
(Seite \pageref{kombiVariationOhneWiederholung}).

\textbf{Variation} = \textbf{Reihenfolge wesentlich}: Die ausgewählten $n$ Objekte werden in
einer Reihe aufgelistet (Seite \pageref{kombiVariation}).


\textbf{Kombination} = \textbf{Reihenfolge} \textbf{un}wesentlich: Es geht nur darum,
welche $n$ Objekte ausgewählt wurden, nicht an welcher Position (Seite \pageref{kombiKombination}).

\newpage
\subsubsection{Entscheidungshilfe}

\bbwCenterGraphic{15cm}{geso/stoch/img/EntscheidungsHilfeBaum.png}

$$\hspace{4mm}
n^k; \hspace{24mm}
n!; \hspace{19mm}
\frac{n!}{(n-k)!}; \hspace{2mm}
{n+k-1 \choose k};\hspace{22mm}
{n\choose k}$$

{\hspace{18mm}
s. \pageref{kombiVariationMitZuruecklegen}\hspace{19mm}
s. \pageref{kombiVariationOhneWiederholung}\hspace{19mm}
s. \pageref{kombiVariationEinerTeilmenge}\hspace{17mm}
 ----- \hspace{30mm}
s. \pageref{kombiKombination}
}
\newpage
