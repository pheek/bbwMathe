%%
%% Stochastik Grundlagen
%% 2020 - 08 - 03 φ
%%


\subsection{Einstiegsbeispiel Produktregel (optional)}
Als Einstiegsbeispiel dient die folgende Wanderung

\vspace{5mm}

\bbwGraphic{12cm}{geso/stoch/img/wanderung.png}

Auf wie viele Arten kann der Wanderer von Wesen (im Westen) nach Obertal (im Osten) gelangen, wenn er ausschließlich von West nach Ost wandern will?

Die Antwort kann durch Abzählen (oder eine Kombination von Abzählen und Multiplizieren) gefunden werden:

\TNTeop{$3\cdot{}(1+(2\cdot3)) + 3 = 24$ Möglichkeiten. Mögliche
  Erklärung: Bis «L» sind 3 Wege offen. Bis «R» sind total 3*2 + 1,
  also sieben Wege möglich. Um nach «O» zu gelangen sind nun 7*3 + die
  drei Wege ab Lisibach und dann direkt, also total 24 Wege möglich. Parallele Wege werden addiert;
  serielle Wegmöglichkeiten multiplizieren sich.}%% END TNT eop
%% implicit \newpage
%%%%%%%%%%%%%%%%%%%%%%%%%%%%%%%%%%%%%%%%%%%%%%%%%%%%%%%%%%%%%%%%%%%%5
