\section{Grundlagen}\index{Grundlagen!Stochastik}

\sectuntertitel{Mit an Sicherheit grenzender Wahrscheinlichkeit...}

\TRAINER{Video ``Quatematik'' \texttt{https://www.youtube.com/watch?v=kmXgtZJNSVQ}}


\subsection{Ergebnis}\index{Ergebnis}

Mögliche \textbf{Ergebnisse} beim dreimaligen Münzwurf wären zum
Beispiel:

\TNT{5.2}{
  \begin{itemize}
  \item Kopf-Kopf-Zahl
  \item Zahl-Zahl-Kopf
  \item Zahl-Zahl-Zahl
  \item Kopf-Zahl-Kopf
  \end{itemize}
}

Diese vier Resultate oder Ausgänge unseres Experimentes nennen wir \textbf{Ergebnisse}, womit wir bereits
bei den ersten Begriffen ankommen:

\begin{definition}{Ergebnis}{}
Ein \textbf{Ergebnis}\index{Ergebnis} ist ein möglicher Ausgang eines
Zufallsexperiments.
\end{definition}


\subsubsection{Ergebnismenge oder Ergebnisraum}\index{Ergebnismenge}\index{Ergebnisraum}

und bezeichnen diese als $\Omega$\index{Omega!Ergebnismenge}\index{$\Omega$ s. Omega}.

\bbwCenterGraphic{5cm}{geso/stoch/img/omega.jpg}
\begin{center}{\small{(Foto 2023) Zürich}}\end{center}
Zeichnen Sie drei schöne {\huge $\Omega$}-Symbole

\begin{definition}{Ergebnismenge}{}
  Unter \textbf{Ergebnismenge} oder \textbf{Ergebnisraum} bezeichnen
  wir die Menge $\Omega$, welche alle möglichen Ausgänge eines
  Zufallsexperiments enthält.
\end{definition}

\newpage
