\section{Grundlagen}\index{Grundlagen!Stochastik}

\sectuntertitel{Mit an Sicherheit grenzender Wahrscheinlichkeit...}

\TRAINER{Video ``Quatematik'' \texttt{https://www.youtube.com/watch?v=kmXgtZJNSVQ}}


\subsection*{Einstieg}
Für die Begriffe beginnen wir mit einem einfachen Münzwurf-Experiment.

Werfen Sie eine Münze dreimal hintereinander (oder werfen Sie drei
Münzen, eine erste, eine zweite und eine dritte). Jede Münze hat die
Möglichkeit auf Kopf oder Zahl (alles andere heißt: nochmals werfen).

Notieren Sie den Ausgang Ihres Wurfexperimentes

\TNT{2.4}{%
\zB Kopf, Kopf, Zahl}%%

\TRAINER{\vspace{24mm}}

Wiederholen Sie das Experiment nun noch weitere drei Male:

\noTRAINER{\mmPapier{1.2}}

\noTRAINER{\mmPapier{1.2}}

\noTRAINER{\mmPapier{1.2}}
\newpage

\subsection{Ergebnis}

Jeder mögliche Ausgang dieses Experiments wird Ergebnis genannt:

\begin{definition}{Ergebnis}{}
Ein \textbf{Ergebnis}\index{Ergebnis} ist ein möglicher Ausgang eines
Zufallsexperiments.
\end{definition}


\subsubsection{Ergebnismenge oder Ergebnisraum}\index{Ergebnismenge}\index{Ergebnisraum}\label{ergebnisraum}

Die Menge aller möglichen Ausgänge nennen wir Ergebnismenge (oder Ergebnisraum)
und bezeichnen diese als $\Omega$\index{Omega!Ergebnismenge}\index{$\Omega$ s. Omega}.

\bbwCenterGraphic{5cm}{geso/stoch/img/omega.jpg}
\begin{center}{\small{(Foto 2023) Zürich}}\end{center}
Zeichnen Sie drei schöne {\huge $\Omega$}-Symbole

\TNT{2}{}

\begin{definition}{Ergebnismenge}{}
  Unter \textbf{Ergebnismenge} oder \textbf{Ergebnisraum} bezeichnen
  wir die Menge $\Omega$, welche alle möglichen Ausgänge eines
  Zufallsexperiments enthält.
\end{definition}

\newpage

\subsection*{Aufgabe}

\aufgabenFarbe{Notieren Sie die Ergebnismenge $\Omega$ zu obigem
Zufallsexperiment «Münze dreimal werfen». Notiert wird dies üblich in der Mengennotation:}

$$\Omega = \{ (Kopf-Kopf-Kopf), (Kopf-Kopf-Zahl), ...\}$$

$\Omega = ...$\TRAINER{$\{(Kp.-Kp.-Kp.); (Kp.-Kp.-Zl.); (Kp.-Zl.-Kp);
  (Kp.-Zl.-Zl.);$
  
  $(Zl.-Kp.-Kp.); (Zl.-Kp.-Zl.); (Zl.-Zl.-Kp); (Zl.-Zl.-Zl.)\}$}

\TNT{2.4}{\vspace{30mm}}



\aufgabenFarbe{(optional) Notieren Sie die Ergebnismenge $\Omega$ zum Wurf nacheinander mit zwei
Spielwürfeln.}

$\Omega=...$\footnote{Achtung: Es gibt 36 mögliche Ergebnisse!}

\TRAINER{$...= \{$

  $  (1,1); (1,2); (1,3); (1, 4); (1,5); (1,6);$

  $  (2,1); (2,2); (2,3); (2, 4); (2,5); (2,6);$

  $  (3,1); (3,2); (3,3); (3, 4); (3,5); (3,6);$

  $  (4,1); (4,2); (4,3); (4, 4); (4,5); (4,6);$

  $  (5,1); (5,2); (5,3); (5, 4); (5,5); (5,6);$

  $  (6,1); (6,2); (6,3); (6, 4); (6,5); (6,6)$

  $\}$}

\TNTeop{}

\newpage

\subsection{Ereignis}\index{Ereignis}
\begin{definition}{Ereignis}{}
Der Begriff \textbf{Ereignis} bezeichnet eine Menge möglicher
Ergebnisse.
\end{definition}

Da die Begriffe nahe beieinander liegen, schauen wir
nochmals das Münzwurf-Experiment an. Wir haben acht mögliche
Ergebnisse. 

Das \textbf{Ereignis} «Es wurde genau zweimal Kopf geworfen» besteht
aus den Ergebnissen

\begin{itemize}
\item Kopf-Kopf-Zahl
\item Kopf-Zahl-Kopf
\item Zahl-Kopf-Kopf
\end{itemize}

\subsection*{Aufgabe(n)}

\aufgabenFarbe{
Notieren Sie für obigen dreimaligen Münzwurf das Ereignis $E$ in Mengennotation «Es wurde mindestens
zweimal Zahl geworfen»:
}
$E = \{\noTRAINER{...}\TRAINER{ZZZ,ZZK, ZKZ, KZZ\}}$

\TNT{2.8}{Es gibt vier mögliche Ergebnisse mit besagter Eigenschaft,
  wobei das Ergebnis dreimal Zahl beim Ereignis mit dabei sein
  muss.
\vspace{1cm}
}

\newpage


\subsubsection{Gegenereignis}\index{Gegenereignis}
\begin{definition}{Gegenereignis}{}
  Das \textbf{Gegenereignis} zu einem bestimmten Ereignis $E$ sind alle möglichen Ergebnisse,
  die im Ereignis $E$ \textbf{nicht} vorkommen.
\end{definition}

\begin{beispiel}{Gegenereignis}{}
So ist das Gegenereignis beim dreimaligen Münzwurf zu «\textit{Es wurde mindestens zwei mal Zahl
geworfen}» wäre folglich «\textit{Es wurde keinmal oder einmal Zahl geworfen}».
\end{beispiel}

\vspace{2mm}


\begin{definition}{}{}
Das Gegenereignis wird mit einem Strich über dem Ereignis angegeben:

$$\bar{E} = \Omega \backslash E$$
\end{definition}

\subsection*{Aufgabe(n)}

\textbf{Beispiel drei Münzwürfe}

\aufgabenFarbe{Notieren Sie das \textbf{Ereignis} und das \textbf{Gegenereignis} von «\textit{Der erste von drei Münzwürfen
  ist Zahl.}» in der Mengenschreibweise, indem Sie alle
möglichen Ergebnisse in Mengenklammern angeben:}

$$E = \{
\noTRAINER{.........................................................}\TRAINER{ZZZ,
ZKZ, ZZK, ZKK\}}$$

$$\bar{E} = \{
\noTRAINER{.........................................................}\TRAINER{KZZ,
KKZ, KZK, KKK\}}$$


Sprich: Das Gegenereignis $E$-Strich ist gleich Omega($\Omega$) ohne die Elemente
aus $E$.
\newpage
\textbf{Beispiel mit zwei Würfeln (optional)}


\aufgabenFarbe{Gehen wir zurück zum Würfelexperiment. Diesmal werden
\textbf{zwei} Spielwürfeln nacheinander geworfen:
\\
Notieren Sie das Gegenereignis zu folgendem
Würfelereignis (einerseits in Worten, andererseits in der
Mengennotation)}%% end Aufgabenfarbe

$E=$ «Bei mindestens einem der beiden Würfel liegen
mehr als zwei Augen.».

$\bar{E} = ...$

\TNTeop{$...= \{(1,1); (1,2); (2,1); (2,2)\}$
  
In Worten: \begin{itemize}
  \item «Bei beiden Würfeln liegt höchstens eine Zwei»
  \item «Bei jedem Wurf zeigt sich maximal die Zwei»
  \item «Auf beiden Würfeln liegt weniger als eine Drei»
  \item «Bei keinem der Würfel liegen mehr als zwei Augen»»
\end{itemize}}%% END TNTeop
%%\newpage
