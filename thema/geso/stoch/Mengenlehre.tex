%%
%% Stochastik Grundlagen
%% 2020 - 07 - 24 φ
%%

\section{Mengenlehre in der Stochastik}\index{Stochastik}\index{Wahrscheinlichkeitsrechnung}


Sie kennen bereits den Ergebnisraum $\Omega$ (s.\ref{ergebnisraum} auf
Seite \pageref{ergebnisraum}).


\subsection{Elementarereignis}
\begin{bemerkung}{}{}
  Ein Ereignis kann natürlich auch aus genau einem Ergebnis
  bestehen. So ist bei einem einmaligen Wurf mit einem Würfel die Zahl
  6 sowohl ein Ergebnis, nämlich «6», als auch ein Ereignis, nämlich
  $\{6\}$. Das Ereignis «die höchstmögliche Augenzahl» besteht aus
  einem einzigen Ergebnis, der «6».
  \end{bemerkung}

\begin{definition}{Elementarereignis}{}
Besteht ein Ereignis genau aus einem Ergebnis, so sprechen wir von einem \textbf{Elementarereignis}\index{Elementarereignis}.
\end{definition}

\begin{beispiel}{Elementarereignis}{}
  Bei dreimaligem Münzwurf ist das Ereignis\\
  «Es wird keinmal Kopf geworfen»\\
  ein Elementarereignis:\\

  $$E = \{ZZZ\}$$
\end{beispiel}
\newpage

\subsection{Sicheres ...}\index{Ereignis!unmögliches}\index{Ereignis!sicheres}
Unter dem sicheren Ereignis verstehen wir nichts anderes als die Menge
$\Omega$ selbst. Ein Ergebnis muss ja zutreffen, und da in $\Omega$
\textbf{alle} möglichen Ergebnisse zusammengefasst sind, bezeichnen
wir mit $\Omega$ das \textbf{sichere Ereignis}.

\subsubsection{... und unmögliches Ereignis}
Da ein Zufallsexperiment mindestens ein Ergebnis (also einen Ausgang
oder ein Resultat) aufweisen wird, so bezeichnen wir mit der leeren
Menge $\{\}$ das unmögliche Ereignis. Mit anderen Worten:

Es ist nicht möglich, dass das Experiment keinen Ausgang hat.


\begin{bemerkung}{}{}
Die Gesamtheit aller Elemente aus $\Omega$ bezeichnet selbst eine
\textit{Teil}menge von $\Omega$.

Der Begriff «Teil» wird hier in der Mengenlehre etwas \textit{überstrapaziert}; in dem Sinne, dass jede Menge auch
  Teilmenge ihrer selbst ist.
\end{bemerkung}

\begin{definition}{}{}
Mit Betragsstrichen $|.|$\index{Betragsstriche}\index{$\mid\cdot \mid$ s. Betrag(-striche)} wird die Anzahl Elemente in einer Menge angegeben.
\end{definition}

\begin{bemerkung}{Ergebnisraum}{}
  Es gilt: $$|E| + |{\bar{E}}| = |\Omega|$$
  Die Anzahl Ergebnisse im Ereignis $|E|$ plus die Anzahl Ergebnisse im Gegenereignis ($|\bar{E}|$) ergeben zusammen die Gesamtheit aller Ergebnisse ($=|\Omega|$).
\end{bemerkung}

\subsection*{Aufgaben}
\olatLinkGESOKompendium{5.3}{43ff}{5. und 7.}

\newpage
