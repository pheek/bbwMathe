\subsection{Gegenereignis}

\begin{definition}{Gegenereignis}{}
  Das \textbf{Gegenereignis} zu einem bestimmten Ereignis $E$ sind alle möglichen Ergebnisse,
  die im Ereignis $E$ \textbf{nicht} vorkommen.
\end{definition}

\begin{beispiel}{Gegenereignis}{}
So ist das Gegenereignis beim dreimaligen Münzwurf zu «\textit{Es wurde mindestens zwei mal Zahl
geworfen}»: 

\vspace{3mm}
\LoesungsRaumLen{120mm}{«\textit{Es wurde keinmal oder einmal Zahl geworfen}»}.
\end{beispiel}

\vspace{2mm}


\begin{definition}{}{}
Das Gegenereignis wird mit einem Strich über dem Ereignis angegeben:

$$\bar{E} = \Omega \backslash E$$

Sprich «Omega ohne E»
\end{definition}

\newpage
\textbf{Beispiel drei Münzwürfe}

\aufgabenFarbe{Notieren Sie das \textbf{Ereignis} und das \textbf{Gegenereignis} von «\textit{Der erste von drei Münzwürfen
  ist Zahl.}» in der Mengenschreibweise, indem Sie alle
möglichen Ergebnisse in Mengenklammern angeben:}

$$E = \{
\noTRAINER{.........................................................}\TRAINER{ZZZ,
ZKZ, ZZK, ZKK\}}$$

$$\bar{E} = \{
\noTRAINER{.........................................................}\TRAINER{KZZ,
KKZ, KZK, KKK\}}$$


Sprich: Das Gegenereignis $E$-Strich ist gleich Omega($\Omega$) ohne die Elemente
aus $E$.
\newpage
\textbf{Beispiel mit zwei Würfeln (optional)}


\aufgabenFarbe{Gehen wir zurück zum Würfelexperiment. Diesmal werden
\textbf{zwei} Spielwürfeln nacheinander geworfen:
\\
Notieren Sie das Gegenereignis zu folgendem
Würfelereignis (einerseits in Worten, andererseits in der
Mengennotation)}%% end Aufgabenfarbe

$E=$ «Bei mindestens einem der beiden Würfel liegen
mehr als zwei Augen.».

$\bar{E} = ...$

\TNTeop{$...= \{(1,1); (1,2); (2,1); (2,2)\}$
  
In Worten: \begin{itemize}
  \item «Bei beiden Würfeln liegt höchstens eine Zwei»
  \item «Bei jedem Wurf zeigt sich maximal die Zwei»
  \item «Auf beiden Würfeln liegt weniger als eine Drei»
  \item «Bei keinem der Würfel liegen mehr als zwei Augen»»
\end{itemize}}%% END TNTeop

\newpage




Es gelten folgende Gesetze, Kolmogorow-Axiome genannt\footnote{Andrei Nikolajewitsch
  Kolmogorow 1903-1987}:

\begin{gesetz}{}{}
\begin{itemize}

\item $P(E) \ge 0$: Die Wahrscheinlichkeit, dass ein bestimmtes Ereignis eintritt, ist nie kleiner als 0.

\item $P(\Omega) = 1$: Die Wahrscheinlichkeit, dass irgend etwas eintritt, ist immer 100\%, also 1.

\item Bei Ereignissen $A$ und $B$, die keine gemeinsamen Ergebnisse aufweisen, gilt die Summenregel:   $$P(A\cup B) = P(A) + P(B)$$

\end{itemize}
\end{gesetz}

Aus den obigen Gesetzen kann die folgende nützliche Rechenregel abgeleitet werden:

\begin{gesetz}{}{}
  Es gilt für das Gegenereignis $\overline{E}$

  $$P(\overline{E}) = 1-P(E).$$

  Mit anderen Worten: Die
  Wahrscheinlichkeit, dass ein Ereignis \textbf{nicht} eintritt ist
  gleich 100\% \textbf{minus} die Wahrscheinlichkeit, dass das
  Ereignis eintritt.\TRAINER{\footnote{Diese Tatsache gehört nicht zu
      Kolmogorows Axiomen, kann jedoch einfach daraus abgeleitet
      werden und wird häufig verwendet. Herleitung: $E$ und
      $\overline{E}$ sind unvereinbar und $\Omega = E \cup
      \overline{E}.$ Und somit ist $1 = P(E) + P(\overline{E})$.}}
\end{gesetz}

\TNTeop{
  Beweis optional:

  $P(\Omega) = 1$ und $\Omega = E \cup{} \overline{E}$

  $$\Longrightarrow  1 = P(\Omega) = P(E\cup{} \overline{E}) = P(E) +
  P(\overline{E} )$$
  $$\Longrightarrow$$
  $$P(\overline{E}) = 1 - P(E)$$
}%%
%%%%%%%%%%%%%%%%%%%%%%%%%%%%%%%%%%%%%%%%%%%%%%%%%%%%%%%%%%%%%%%%%%%%%%%%%%%%%%%%%%%%%%%%%%%%%%%%%%%%%%

\begin{beispiel}{}{}\textbf{Ziehen einer Kugel aus einer Urne}

Aus einer Urne mit 4 roten ($r$), 3 schwarzen ($s$) und 2 weißen ($w$) Kugeln wird eine Kugel gezogen.

Ergebnisraum: $\Omega = \{r, s, w\}$


Wahrscheinlichkeiten für die Elementarereignisse:

\noTRAINER{\leserluft\leserluft\leserluft}
$P(\{r\}) = \LoesungsRaum{\frac49}$

\noTRAINER{\leserluft\leserluft\leserluft}
$P(\{s\}) = \LoesungsRaum{\frac39}$

\noTRAINER{\leserluft\leserluft\leserluft}
$P(\{w\}) = \LoesungsRaum{\frac29}$

Wie groß ist nun die Wahrscheinlichkeit \textbf{eine rote oder eine schwarze} Kugel zu ziehen?
Wir definieren $E_1$ als das günstige Ereignis, dass eine rote oder eine schwarze Kugel gezogen wird.

$E_1 = \{r\} \cup \{s\} = \{r,s\}$

\noTRAINER{\leserluft\leserluft\leserluft}
$P(E_1) = \LoesungsRaumLen{50mm}{\frac49 + \frac39 = \frac79}$


Wie groß wäre dabei die Wahrscheinlichkeit $P(E_2)$, weder eine rote noch eine schwarze Kugel zu ziehen?

$E_2 := \{w\}$

$P(E_2) = \frac29$, denn $E_2 = \overline{E_1}$ und somit

\noTRAINER{\leserluft\leserluft}
$P(E_2) = P(\overline{E_1}) = \LoesungsRaumLen{90mm}{1 - P(E_1) = 1 - \frac79 = \frac29 . }$

\end{beispiel}
\newpage

\subsection*{Aufgaben}

\aufgabenFarbe{Wie groß ist die Wahrscheinlichkeit, mit drei Würfeln
  maximal zwei gleiche Augenzahlen zu erzielen?}
\TNT{7.6}{Gegenereignis = Alle drei Würfe sind gleich. Die
  Wahrscheinlichkeit von «alle drei sind gleich» liegt bei
  $\frac6{6^3}$ und somit ist die gesuchte Wahrscheinlichkeit

  $P=1-\frac6{6^3} =  \frac{210}{216}  = \frac{35}{36} \approx 97.2\%$.}

\olatLinkGESOKompendium{5.3}{44}{5., 6., 8., 9. und 10.}

\newpage



