\section{Grundbegriffe}\index{Grundbegriffe!Stochastik}

\sectuntertitel{Mit an Sicherheit grenzender Wahrscheinlichkeit...}

\TRAINER{Video ``Quatematik'' \texttt{https://www.youtube.com/watch?v=kmXgtZJNSVQ}}


\subsection*{Einstieg}
Für die Begriffe beginnen wir mit einem einfachen Münzwurf-Experiment.

Werfen Sie eine Münze dreimal hintereinander (oder werfen Sie drei
Münzen, eine erste, eine zweite und eine dritte). Jede Münze hat die
Möglichkeit auf Kopf oder Zahl (alles andere heißt: nochmals werfen).

Notieren Sie den Ausgang Ihres Wurfexperimentes

\TNT{2.4}{%
\zB Kopf, Kopf, Zahl}%%

\TRAINER{\vspace{24mm}}

Wiederholen Sie das Experiment nun noch weitere drei Male:

\noTRAINER{\mmPapier{1.2}}

\noTRAINER{\mmPapier{1.2}}

\noTRAINER{\mmPapier{1.2}}
\newpage

\subsection{Ergebnis}

Jeder mögliche Ausgang dieses Experiments wird Ergebnis genannt:

\begin{definition}{Ergebnis}{}
Ein \textbf{Ergebnis}\index{Ergebnis} ist ein möglicher Ausgang eines
Zufallsexperiments.
\end{definition}


\subsubsection{Ergebnismenge oder Ergebnisraum}\index{Ergebnismenge}\index{Ergebnisraum}\label{ergebnisraum}

\begin{definition}{Ergebnismenge}{}
  Unter \textbf{Ergebnismenge} oder \textbf{Ergebnisraum} bezeichnen
  wir die Menge $\Omega$\index{$\Omega$ s. Omega}, welche alle möglichen Ausgänge eines
  Zufallsexperiments enthält.
\end{definition}


\bbwCenterGraphic{5cm}{geso/stoch/img/omega.jpg}
\begin{center}{\small{(Foto 2023) Zürich}}\end{center}
Zeichnen Sie drei schöne {\huge $\Omega$}-Symbole

\TNT{2}{}


\newpage

\subsection*{Aufgabe}

\aufgabenFarbe{Notieren Sie die Ergebnismenge $\Omega$ zu obigem
Zufallsexperiment «Münze dreimal werfen». Notiert wird dies üblich in der Mengennotation:}

$$\Omega = \{ (Kopf-Kopf-Kopf), (Kopf-Kopf-Zahl), ...\}$$

$\Omega = ...$\TRAINER{$\{(Kp.-Kp.-Kp.); (Kp.-Kp.-Zl.); (Kp.-Zl.-Kp);
  (Kp.-Zl.-Zl.);$
  
  $(Zl.-Kp.-Kp.); (Zl.-Kp.-Zl.); (Zl.-Zl.-Kp); (Zl.-Zl.-Zl.)\}$}

\TNT{2.4}{\vspace{30mm}}



\aufgabenFarbe{(optional) Notieren Sie die Ergebnismenge $\Omega$ zum Wurf nacheinander mit zwei
Spielwürfeln.}

$\Omega=...$\footnote{Achtung: Es gibt 36 mögliche Ergebnisse!}

\TRAINER{$...= \{$

  $  (1,1); (1,2); (1,3); (1, 4); (1,5); (1,6);$

  $  (2,1); (2,2); (2,3); (2, 4); (2,5); (2,6);$

  $  (3,1); (3,2); (3,3); (3, 4); (3,5); (3,6);$

  $  (4,1); (4,2); (4,3); (4, 4); (4,5); (4,6);$

  $  (5,1); (5,2); (5,3); (5, 4); (5,5); (5,6);$

  $  (6,1); (6,2); (6,3); (6, 4); (6,5); (6,6)$

  $\}$}

\TNTeop{}

\newpage

\subsection{Ereignis}\index{Ereignis}
\begin{definition}{Ereignis}{}
Der Begriff \textbf{Ereignis} bezeichnet eine Menge möglicher
Ergebnisse.
\end{definition}

Da die Begriffe nahe beieinander liegen, schauen wir
nochmals das Münzwurf-Experiment an. Wir haben acht mögliche
Ergebnisse. 

Das \textbf{Ereignis} «Es wurde genau zweimal Kopf geworfen» besteht
aus den Ergebnissen

\begin{itemize}
\item Kopf-Kopf-Zahl
\item Kopf-Zahl-Kopf
\item Zahl-Kopf-Kopf
\end{itemize}

\subsection*{Aufgabe(n)}

\aufgabenFarbe{
Notieren Sie für obigen dreimaligen Münzwurf das Ereignis $E$ in Mengennotation «Es wurde mindestens
zweimal Zahl geworfen»:
}
$E = \{\noTRAINER{...}\TRAINER{ZZZ,ZZK, ZKZ, KZZ\}}$

\TNT{2.8}{Es gibt vier mögliche Ergebnisse mit besagter Eigenschaft,
  wobei das Ergebnis dreimal Zahl beim Ereignis mit dabei sein
  muss.
\vspace{1cm}
}

\olatLinkGESOKompendium{5.2}{44}{7.}

\newpage

\newpage

