\subsection{Zusammenfassung}\label{kombinatorikZusammenfassung}


%%\bbwCenterGraphic{16cm}{geso/stoch/img/KombinatorikRLP.pdf}
$N = $
Anzahl Möglichkeiten, um $\color{red}k$ Elemente aus total
$\color{blue}n$ Elementen auszuwählen. Von den $n$ zur Auswahl
stehenden Elementen werden typischerweise nicht alle ausgewählt.

\begin{tabular}{p{15mm}|p{75mm}|p{75mm}}
& $$\text{\textbf{mit} Wiederholung}$$ $$\text{(= mit
    Zurücklegen)}$$ & $$\text{\textbf{ohne}
    Wiederholung}$$ $$\text{(= ohne Zurücklegen)}$$ $$k\le n$$\\\hline

%% Zeilentitel Variationen
\rotatebox[origin=rT]{90}{\makecell{\textbf{Variation}\\Reihenfolge wesentlich}}
&
%% Formeln
 \begin{center}{\fbox{$N={\color{blue}n}^{\color{red}k}$}}\end{center}
 Bsp.: Anzahl Wörter der Länge
 {\color{red}4}, die man aus {\color{blue}10} vorgegebenen Buchstaben bilden kann: $${\color{blue}10}^{\color{red}4} = 10\,000$$
&
 \begin{center}{\fbox{$N = \frac{{\color{blue}n}!}{({\color{blue}n}-{\color{red}k})!}$}}\end{center}
 Bsp.: Anzahl Möglichkeiten, $\color{red}4$ Leute auf $\color{blue}
 10$ Sitzplätze zu verteilen.
 $$\frac{{\color{blue}10}!}{({\color{blue}10}-{\color{red}4})!} = 5040$$

 \\\hline

%% Zeilentitel Kombinationen
\rotatebox[origin=rT]{90}{\makecell{\textbf{Kombination}\\Reihenfolge irrelevant}}
&
%% Formeln
 \begin{center}$N = {{\color{blue}n}+{\color{red}k}-1 \choose {\color{red}k}}$\end{center}
 Bsp.: Anzahl Möglichkeiten, $\color{red}10$ Brote aus $\color{blue}4$ Sorten auszuwählen:
 $${{\color{blue}4}+{\color{red}10}-1 \choose {\color{red}10}} = 286$$
&
 \begin{center}{\fbox{$N={{\color{blue}n} \choose {\color{red}k}}$}}\end{center}
 Bsp.: Anzahl Möglichkeiten, $\color{red}4$ Karten aus {\color{blue}10} zu ziehen: $${{\color{blue}10} \choose {\color{red}4}} = 210$$
 \end{tabular}
\newpage


\subsubsection{Taschenrechner}
Taschenrechner Abkürzungen \tiprobutton{ncrnpr}.


\begin{bbwFillInTabular}{l|c||c|c|}
            & mit Wiederholung                & \multicolumn{2}{c}{ohne Wiederholung}     \\\hline
            &   $n$ unabhängig von $k$         &  $n > k$ \noTRAINER{\,\,\,\,\,\,}  &  $n=k$ \\\hline
Variation   &   \TRAINER{\tiprobutton{xhoch}}  & \TRAINER{\texttt{nPr}}          & \TRAINER{\texttt{!}} \\\hline
Kombination &   ($n+k-1$) \texttt{nCr} $k$     & \TRAINER{\texttt{nCr}} & \TRAINER{1} \\\hline
\end{bbwFillInTabular}

Dabei bedeuten:

\textbf{Mit Wiederholung} = \textbf{mit Zurücklegen}: Die ausgewählten $n$ Objekte können
mehrfach verwendet werden (Beispiel Zahlenschloss) (Seite \pageref{kombiVariation}).

\textbf{Ohne Wiederholung} = \textbf{ohne Zurücklegen}: Die ausgewählten $n$ Objekte kommen
in der Stichprobe genau einmal vor (Beispiel: Personen auswählen)
(Seite \pageref{kombiVariationOhneWiederholung}).

\textbf{Variation} = \textbf{Reihenfolge wesentlich}: Die ausgewählten $n$ Objekte werden in
einer Reihe aufgelistet (Seite \pageref{kombiVariation}).


\textbf{Kombination} = \textbf{Reihenfolge} \textbf{un}wesentlich: Es geht nur darum,
welche $n$ Objekte ausgewählt wurden, nicht an welcher Position (Seite \pageref{kombiKombination}).

\newpage
\subsubsection{Entscheidungshilfe}

\bbwCenterGraphic{15cm}{geso/stoch/img/EntscheidungsHilfeBaum.png}

$$\hspace{4mm}
n^k; \hspace{24mm}
n!; \hspace{19mm}
\frac{n!}{(n-k)!}; \hspace{2mm}
{n+k-1 \choose k};\hspace{22mm}
{n\choose k}$$

{\hspace{18mm}
s. \pageref{kombiVariationMitZuruecklegen}\hspace{19mm}
s. \pageref{kombiVariationOhneWiederholung}\hspace{19mm}
s. \pageref{kombiVariationEinerTeilmenge}\hspace{17mm}
 ----- \hspace{30mm}
s. \pageref{kombiKombination}
}

%%\newpage

\subsubsection{Produkt- und Summenregel}

\textbf{Produktregel}: Müssen zwei Ereignisse unabhängig voneinander
(quasi hintereinander) ausgeführt werden müssen, so sprechen wir von
«\textbf{und}» oder der Produktregel (siehe Seite \pageref{experimentePfadUndSummenregel}).

\textbf{Summenregel}: Ist von zwei alternativen Ergebnissen die Rede,
welche «\textbf{nur entweder / oder»} eintreffen, so sprechen wir von
Summenregel (siehe Seite \pageref{experimentePfadUndSummenregel}).

