%% Grundlagen
%% 2020 - 08 - 03 φ
%%

\subsection{Zufallsexperimente}\index{Experiment}\index{Zufallsexperimente}

\subsubsection{Laplace-Experiment}\index{Experiment!Laplace}\index{Laplace-Experiment}
Nach Pierre-Simon Laplace (1749-1827).

Ein fairer Spielwürfel wird geworfen. Jede Seite hat genau die selbe
Wahrscheinlichkeit. Wie groß ist nun die Wahrscheinlichkeit, mit einem
Wurf eine größere Zahl als eine 4 zu werfen?

\TNT{2.4}{Die günstigen Ergebnisse sind 5 und 6 (\textbf{zwei} Stück). Alle möglichen Ergebnisse: 1 bis 6 (\textbf{sechs} Stück). Wahrscheinlichkeit = $\frac{2}{6}=\frac13$}

Dabei handelt es sich um ein Laplace-Experiment:
\begin{itemize}
\item Jeder einzelne Ausgang (Ergebnis) hat die selbe Wahrscheinlichkeit
      (hier $p = \frac{1}{6}$).
\item Die Wahrscheinlichkeit, dass ein Ereignis (hier $E = \epsdice{5}$
oder $\epsdice{6}$) eintritt, wird berechnet mit $P(E)
= \frac{\text{Anzahl gewünschte Ergebnisse}}{\text{Anzahl
mögliche Ergebnisse}}$.

Hier $P(\{\epsdice{5}\}\cup\{\epsdice{6}\}) = \frac{\left|\left\{\epsdice{5}, \epsdice{6}\right\}\right|}{|\Omega|}= \frac{|E|}{|\Omega|}=\LoesungsRaum{\frac{2}{6}}$.
\end{itemize}

\begin{definition}{Laplace Experiment}{}
  Zufallsexperimente, deren Elementarereignisse alle gleich wahrscheinlich sind, werden als
  \textbf{Laplace-Experimente} bezeichnet. Die Wahrscheinlichkeit berechnet sich wie folgt:
  $$P(E) = \frac{|E|}{|\Omega|}$$
\end{definition}


Kein Laplace-Experiment ist \zB das Werfen eines gezinkten Würfels,
bei dem die Augenzahl 6 häufiger auftritt als die anderen Augenzahlen.

\subsection*{Aufgaben}
\olatLinkGESOKompendium{5.2}{42}{1. bis 3.}

\newpage


%%%%%%%%%%%%%%%%%%%%%%%%%%%%%%%%%%%%%%%%%%%%%%%%%%%%%%%%%%
\subsection{Baumdiagramme}\index{Baumdiagramm!Stochastik}
(Einstufige und mehrstufige Experimente)

\TRAINER{(S. Youtube Videos dazu: Einfach-Mathe!)}

\subsubsection{Einstiegsbeispiel: Glücksrad}
\bbwCenterGraphic{4cm}{geso/stoch/img/gluecksrad75deg}
An einem Glücksrad wird gedreht. $75\degre$ erzielen sofort einen Gewinn. Die anderen $285\degre$ geben einen Verlust an. Zum Glück darf ich am Rad drei mal drehen. Das Spiel endet entweder beim ersten Gewinn oder aber spätestens nach dreimaligem Drehen.

Wie groß ist die Wahrscheinlichkeit, bei diesem Spiel zu gewinnen?

Zeichnen Sie dazu das Baumdiagramm, das bei jedem ``Sieg'' sofort endet und
maximal drei Stufen tief geht.

\TNTeop{Einseitiger entarteter Baum mit Sieg = 75/360 und Verlust =
  285/360. Drei Mal Verlust bei $\left(\frac{285}{360}\right)^3
  \approx 49.6\%$; somit Gewinn beim Gegenereignis
  $1-\left(\frac{285}{360}\right)^3\approx{} 50.38\%$.
\bbwCenterGraphic{7cm}{geso/stoch/img/Rad75Baum.png}
}%% EN TNT
%%\newpage


\subsection{Pfad- und Summenregel}\label{experimentePfadUndSummenregel}
Um Wahrscheinlichkeiten aufzuzeichnen, bietet sich das Baumdiagramm an.
Dabei zeichnet man einen «Baum» von links-nach-rechts oder wie in der Informatik üblich von oben-nach-unten.

Zu den Ästen schreibt man deren Wahrscheinlichkeiten.

Man beginnt bei der Wurzel und für jeden möglichen Ausgang zeichnet man einen Ast. Dabei gelten die folgenden Gesetze:



\begin{gesetz}{Produktregel/Pfadregel}{}\index{Produktregel!Baumdiagramm}\index{Pfadregel!Baumdiagramm}
Alle Wahrscheinlichkeiten von der Wurzel bis zum Endknoten werden aufmultipliziert, um die Endwahrscheinlichkeit (am Endknoten) zu erhalten.
\end{gesetz}

\leserluft\leserluft

\begin{gesetz}{Summenregel I}{}\index{Summenregel!Baumdiagramm}
Alle von einem Knoten ausgehenden Äste haben in der Summe die Wahrscheinlichkeit 1 (= 100\%).
\end{gesetz}

\begin{gesetz}{Summenregel II}{}
Alle Endwahrscheinlichkeiten (bei den Endknoten\footnote{Endknoten werden auch als Blätter des Baumes bezeichnet.}) zusammen addiert, ergeben in der Summe die Wahrscheinlichkeit 1 (= 100\%).
\end{gesetz}

\begin{gesetz}{Summenregel III}{}
  Um die Wahrscheinlichkeit eines Ereignisses zu berechnen, werden die
  Wahrscheinlichkeiten all derjenigen Endknoten zusammengezählt, die zum gewünschten Ereignis gehören.
\end{gesetz}


\begin{bemerkung}{Und vs. Oder}{}
  Ein «Und» in einem Text entspricht in der Regel der Multiplikation
  während das «Oder» einer Addition entspricht:

  $$\text{ «... und ...» } \stackrel{{}^{}}{=}  \cdot{}$$
  $$\text{ «... oder ...» } \stackrel{{}^{}}{=}  +$$
\end{bemerkung}
\newpage


\subsubsection{Referenzaufgaben}

\paragraph{Urne ohne Zurücklegen} In einer Urne liegen drei grüne, zwei blaue und eine rote Kugel. Ich ziehe blind (Urne) zwei Kugeln hintereinander, ohne diese wieder zurückzulegen.

Wie groß ist die Wahrscheinlichkeit, dass genau eine blaue Kugel dabei ist?

Zeichnen Sie das zugehörige Baumdiagramm:

\TNTeop{
\bbwCenterGraphic{15cm}{geso/stoch/img/UrneBaum.png}
  
  Lösung: $P(E)=\frac8{15}=0.5333...$\vspace{13cm}}%% END TNTeop
%%\newpage

\paragraph{Ziege}\index{Ziege} Heidi und Peter spielen mit einem Würfel um eine Ziege.

Sie vereinbaren folgende Regeln:

\begin{enumerate}
\item Die Ziege erhält sofort, wer eine 5 oder eine 6 würfelt.
\item Zuerst würfelt Heidi, dann (falls Heidi noch nicht gewonnen hat) Peter, dann allenfalls noch einmal Heidi.
\item Ist nach insgesamt drei Würfen noch nichts entschieden, dann erhält Peter die Ziege.
\end{enumerate}


Zeichnen Sie das zugehörige Baumdiagramm und entscheiden Sie, ob es sich um ein faires Spiel handelt:

\TNTeop{\bbwCenterGraphic{15cm}{geso/stoch/img/HeidiPeter.png}

Peter gewinnt also mit $\frac29 + \frac8{27} = \frac{14}{27}$, was
etwas mehr als 50\% ausmacht. Das Spiel ist nicht fair, denn wer als
zweiter würfeln darf, hat eine leicht höhere Gewinnchance.}%% END TNT eop
%%\newpage
\newpage

\subsection*{Aufgaben}

Mehrstufige Zufallsexperimente:

\olatLinkGESOKompendium{5.2}{43ff}{Aufg. 4. (Glücksrad), 17. (Klasse), 19., 20. (Kugeln) und 21. (Würfel)}
\newpage

\textbf{Glücksrad (optional)}


Ein Glücksrad verspricht eine kleine
Gewinnwahrscheinlichkeit von $\frac1{32}$, jedoch einen hohen Gewinn.

Wie oft muss ich an diesem Rad drehen, damit die Wahrscheinlichkeit zu
gewinnen mindestens 99\% beträgt?

Tipp: Gegenwahrscheinlichkeit


\TNTeop{
Die Wahrscheinlichkeit zu verlieren ist somit $\frac{31}{32}$. Die
Wahrscheinlichkeit zweimal hintereinander zu verlieren ist damit
$\left(\frac{31}{32}\right)^2$. Dabei habe ich nie gewonnen. Dies kann
man gut im entarteten Baum mit einem einzigen Verlierer-Ast sehen.


$n$ mal hintereinander zu verlieren, ergibt somit eine Wahrscheinlichkeit
von $\left(\frac{31}{32}\right)^n$.

Dies soll aber mit einer Wahrscheinlichkeit von 0.01 (=1\%)
eintreten. Nun könnte ich so lange die Anzahl ausprobieren, bis die
Potenz $\left(\frac{31}{32}\right)^n$ unter 1\% sinkt. Das erhalte ich
aber auch, wenn ich die Potenz mit dem einen Prozent gleichsetze.

$$\left(\frac{31}{32}\right)^n = 0.01$$

Das Lösen dieser Exponentialgleichung liefert für $n$:

$$n = \log_{\frac{31}{32}}(0.01) =
\frac{\lg(0.01)}{\lg(\frac{31}{32})} = \frac{-2}{\lg(\frac{31}{32})} \approx  145.05$$

Somit muss ich 146 mal am Rad drehen, damit die
Gewinnwahrscheinlichkeit 99\% übersteigt.

Die Gegenwahrscheinlichkeit der Gewinnwahrscheinlichkeit von 99\%
mindestens einmal zu gewinnen ist ja
gerade die 1\% Verlustwahrscheinlichkeit aller Spiele.
}%% end TNTeop

\newpage

\textbf{Aufgabe Schweizer Münzen (optional)}

In zwei Urnen sind Münzen. In jeder Urne kommt jede Schweizer Münzart
genau einmal vor:

0.05, 0.10, ... 5.--.

Aus jeder Urne wird zufällig eine Münze gezogen.

Wie groß ist die Wahrscheinlichkeit:
\begin{itemize}
\item Einen ganzen Frankenbetrag zu ziehen?
\item Mehr als 3 Franken zu erwischen?
\end{itemize}

  Tipp: Baumdiagramm oder Tabelle (Zeilen = erste Urne; Spalten =
  zweite Urne)

  \TNTeop{
    Ganzer Frankenbetrag:

    Die erste Verzweigung im Baum hat drei Äste. Die Münzen 0.05, 0.10
    und 0.20 haben je 1/7 Wahrscheinlichkeit und daraus resultiert nie
    ein ganzer Frankenbetrag. Die Münze 0.50 hat 1/7
    Wahrscheinlichkeit, aber nur ein einziger weiterführender Ast
    liefert einen ganzen Frankenbetrag, nämlich nochmals 1/7.
    Die ganzen Frankenbeträge (1.-, 2.- und 5.-) haben zusammen auch
    3/7 Wahrscheinlichkeit, aber nur die ganzen Frankenbeträge aus der
    2. Urne liefern einen ganzen Frankenbetrag.

    Somit:

    $$P(X= \text{ ganzer Frankenbetrag } ) =
    \frac17\cdot{}\frac17+\frac37\cdot{}\frac37 = \frac{22}{49}
    \approx 44.90\%$$

    Mehr als CHF 3.-:

    Wenn der 5-Liber dabei ist, so sind immer mehr als 3 Franken
    gezogen. Anders ist mehr als 3.- nur möglich mit 2.- + 2.-. Somit
    sind Somit gibt es 14 von 49 Varianten.

    $$P(X= \text{ Mehr als CHF 3.- } ) = \frac{14}{49} \approx 28.57\%$$
  }
  

\newpage

%%%%%%%%%%%%%%%%%%%%%%%%%%%%%%%%%%%%%%%%%%%%%%%%%%%%%%%%%%%%%%%%%%%%%%%%

\subsection{Bernoulli-Ketten}\index{Verteilung!Bernoulli}\index{Bernoulli}
(Binomialverteilung)

\TRAINER{Einstiegsvideo: ``EinfachMathe!''\\
  \texttt{https://www.youtube.com/watch?v=WC5O317JzW0}}

\subsubsection{Einstiegsaufgabe}

\paragraph{Genau zwei Sechser} Ein Würfel wird \textbf{dreimal} hintereinander geworfen.

Wie groß ist die Wahrscheinlichkeit, dass genau zwei Sechser dabei sind?


Zeichnen Sie das zugehörige Baumdiagramm:

\TNT{18}{
\bbwCenterGraphic{15cm}{geso/stoch/img/genau2sechser.png}


  Lösung mit Baum oder  Bernoulli-Formel (Binomialverteilung).

  Genau zwei Sechser: $\left(\frac16\right)^2\cdot{}\frac56 +
  \left(\frac16\right)^2\cdot{}\frac56 +
  \left(\frac16\right)^2\cdot{}\frac56 =3\cdot{}
  \left(\frac16\right)^2\cdot{}\frac56= \frac{5}{72}\approx 0.06944$%%
%%
  \vspace{10cm}%%
}%% END TNT
\newpage


\subsubsection{Bernoulli-Experiment}\index{Experiment!Bernoulli}\index{Bernoulli-Experiment}
Obiges Würfelexperiment ist ein typisches
Bernoulli-Experiment\footnote{Jakob I Bernoulli, Schweizer Mathematiker,
  1654-1705}, d.\,h.:

\begin{definition}{Bernoulli-Experiment}{}
\begin{itemize}
\item Das Experiment wird $n$-mal durchgeführt\footnote{Daher wird dieser
Bernoulli-Prozess manchmal auch Bernoulli-Kette genannt. Quelle
\texttt{www.wikipedia.org} 2021-04-08}.
\item Das Einzelexperiment hat genau zwei Ausgänge: Erfolg/Misserfolg.
\item Jedes Einzelexperiment hat für die erfolgreichen Ausgänge die gleiche
      Wahrscheinlichkeit $p$ (somit ist die Wahrscheinlichkeit für den
      Misserfolg gleich $1-p$).
\item Die Einzelexperimente sind voneinander unabhängig.
\end{itemize}
\end{definition}

\begin{bemerkung}{Bernoulli-Experiment}{}
Typischerweise sind wir bei Bernoulli-Experimenten an der
Zufallsvariable $X$ interessiert, welche angibt, wie oft ein Erfolg
eingetreten ist: $X$ hat also die Werte 0, 1, 2, 3, 4, ..., $n$.
\end{bemerkung}

\textbf{Kein} Bernoulli-Experiment ist beispielsweise das Ziehen von zwei Bällen aus
einer Urne mit drei gelben und sechs orangefarbenen Bällen \textbf{ohne} diese
jeweils zurückzulegen; denn dabei verändern sich die
Wahrscheinlichkeiten der verbleibenden Bälle nach jedem Herausnehmen.
\newpage


\paragraph{Beispiele von Bernoulli-Experimenten}

\begin{itemize}
\item
In einer Urne liegen zehn Kugeln: vier blaue und sechs grüne. Ein
Einzelexperiment besteht darin, genau eine Kugel zu ziehen, die Farbe
zu notieren und die Kugel sogleich wieder zurückzulegen.
Das Gesamtexperiment besteht darin, fünf Einzelexperimente
durchzuführen. Jedesmal ist die Wahrscheinlichkeit, eine blaue Kugel
zu erwischen, gleich groß und wir sprechen von einem Bernoulli-Experiment.

\item Dreifaches Drehen am Glücksrad: Wie groß ist die Wahrscheinlichkeit, genau zweimal «Gewinn» zu erzielen?

\item Siebenfaches Würfeln mit einem Spielwürfel, bei dem wir
  interessiert sind, wie groß die Wahrscheinlichkeit ist, dass genau fünf mal ein kleinerer Wurf als eine \epsdice{3} gewürfelt wird.
  
\item
Es müssen aber nicht immer Würfel, Münzen oder Kugeln in Urnen sein:
Der Torwart Alex Calderoni\footnote{Alex Calderoni * 1976: Bekannter
  italienischer Torwart beim Fußballspiel (Quelle Wikipedia 2022).}
fängt womöglich einen Ball mit einer
Wahrscheinlichkeit von 38\%.
Wie groß ist die Wahrscheinlichkeit, dass er genau sechs von zehn Torschüssen
abfangen kann?
\end{itemize}
\newpage


\subsubsection{Formel von Bernoulli}\index{Bernoulli!Formel von}\index{Formel von Bernoulli}

\TRAINER{Einstiegs-Viedo: \texttt{https://www.youtube.com/watch?v=WC5O317JzW0}}


\begin{beispiel}{Sieben mal Würfeln}{}
Wie groß ist die Wahrscheinlichkeit, mit einem Würfel bei
siebenmaligem Werfen genau fünf mal eine der Zahlen \epsdice{1} oder
\epsdice{2} zu erreichen?
\end{beispiel}

Bei der sog. Bernoulli-Kette handelt es sich um ein
Bernoulli-Experiment
in einem mehrstufigen Baum ($n$ Versuche). Dabei interessiert
uns, wie groß die Wahrscheinlichkeit ist, dass von den beiden
möglichen Ausgängen (Treffer / Niete) einer davon $k$-mal auftritt.

Dabei legen wir fest (vgl. Beispiel oben: Siebenmaliges Würfeln):
\begin{itemize}

\item
  $n$ = Anzahl mögliche Durchführungen; hier 7

\item
  $p$ = Wahrscheinlichkeit, dass eine \epsdice{1} oder \epsdice{2}
  geworfen wird. Hier $p = \frac26$.\\
  Somit ist die Misserfolgsquote $= (1-p)$; hier $1-\frac26=\frac46$.


\item
  $k$ = Anzahl der gewünschten ``Treffer''; hier 5, denn wir wollen fünf
Mal eine \epsdice{1} oder eine \epsdice{2} erzielen.
\end{itemize}

Nun gilt

\begin{gesetz}{Bernoulli-Formel (Binomialverteilung)}{}
  $$P(X=k) = {{n}\choose {k}}\cdot{}p^k\cdot{}(1-p)^{n-k}$$


%%\renewcommand{\arraystretch}{2}
\begin{bbwFillInTabular}{rcl}
  $n$ &=& \TRAINER{Anzahl Durchführungen (Baumtiefe)}\\
  $p$ &=& \TRAINER{Gewinnwahrscheinlichkeit pro Durchführung}\\
  $k$ &=& \TRAINER{Anzahl geforderte Treffer}\\
\end{bbwFillInTabular}

  
\end{gesetz}

In obigem Beispiel ergibt sich:
$$P(X=5) = \LoesungsRaumLang{{7\choose 5} \cdot{} \left(\frac26\right)^5 \cdot{} \left(1-\frac26\right)^{7-5}} = \LoesungsRaumLang{{7\choose 5} \cdot{} \left(\frac26\right)^5 \cdot{} \left(\frac46\right)^2}$$

\newpage
\textbf{Taschenrechner}:

\leserluft

Dies kann mit dem Taschenrechner gelöst werden:
Entweder direkt ...

$$\LoesungsRaumLang{( 7 \text{\,\,nCr\,\,} 5 ) *\left(\frac26\right)^5 *
\left(1-\frac26\right)^{7-5}} \approx \LoesungsRaum{0.03841 = 3.841\%}$$

... oder aber mit der eingespeicherten Formel bei
\tiprobutton{data_stat-reg-distr} und dort unter \texttt{DISTR: 4
  Binomialpdf: SINGLE}

... dann ...

\begin{tabular}{c@{ $:$ }l}
  $n=7$   & ... bei \texttt{n} \\
  $p=2/6$ & bei \texttt{p(\text{success})} und \\
  $k=5$   & bei \texttt{x} eintragen.
\end{tabular} 



\subsection*{Aufgaben}

\olatLinkGESOKompendium{5.5.4}{50}{28. und 29}
\aufgabenFarbe{Maturaprüfungen: 2016 (GESO) Aufgabe 11 (Geburten), 2017 (GESO) Aufgabe 17. (borkenkäferresistent), 2018 (GESO) Serie 1 Aufg. 16. (Papayasendung), 2018 (GESO) Serie 2 Aufg. 15. (Sprössling), 2018 (GESO) Serie 3 Aufg. 14. (Ikosaeder)}


\newpage
\subsubsection{Kumulierte Wahrscheinlichkeit}

Basketballspieler «Basil Ballisti» trifft mit einer
Wahrscheinlichkeit von 85\%.

Die Wahrscheinlichkeit, dass er von 20 Würfen \textbf{genau} 17 Mal trifft, kennen wir
bereits von der Formel von Bernoulli (Bernoulli-Experiment/Binomialverteilung):

$$P(X=17) = \LoesungsRaumLang{{20 \choose 17}\cdot 0.85^{17}\cdot(1-0.85)^{20-17}}=$$
$$\LoesungsRaumLang{{20 \choose 17}\cdot 0.85^{17}\cdot (0.15)^{3}}\approx{}\LoesungsRaum{24.29\%}$$
Dabei bezeichnet $E$ das Ereignis ``siebzehn Treffer in den 20 Würfen''.

\begin{beispiel}{Kumulierte Wahrscheinlichkeit}{}
Wie groß ist nun aber die Wahrscheinlichkeit, dass er bei 20 Würfen
\textbf{höchstens} (= maximal)  17 mal trifft?
\end{beispiel}


Lösung: Wir summieren alle Wahrscheinlichkeiten auf, bei denen er
während seinen 20 Würfen
keinen Treffer, einen Treffer, zwei Treffer, ... bis zu 17 Treffer
erzielt. Dies nennen wir die kumulierte\footnote{«Kumulieren» kommt von
  lat. \texttt{cumulus} = Anhäufung.} Wahrscheinlichkeit\index{Wahrscheinlichkeit!kumulierte}:

$$P(X\le 17) = $$
\TNT{6}{$$P(X=0) + P(X=1) + P(X=2) + ... + P(X=16) + P(X=17)$$
$$={20\choose 0}\cdot 0.85^0 \cdot 0.15^{20} + {20\choose 1}\cdot 0.85^1 \cdot 0.15^{19} + {20\choose 2}\cdot 0.85^2 \cdot 0.15^{18} + ... $$

}%% END TNT

Dies können wir auch mit dem Summenzeichen ($\Sigma$) schreiben:
$$P(X\le 17) = \LoesungsRaumLang{\sum_{k=0}^{17} {20 \choose k} \cdot 0.85^k \cdot 0.15^{20-k}}$$

\newpage
\textbf{Taschenrechner}:

\leserluft

Für den Taschenrechner verwenden Sie folgende Notation: 

$$\LoesungsRaumLang{\sum_{x=0}^{17}\left(( 20 \text{\,\,nCr\,\,} x ) * 0.85^x *
0.15^{20-x}\right)} \approx \LoesungsRaum{0.5951}$$


Doch auch hierzu gibt es eine eingespeicherte Formel
bei
\tiprobutton{data_stat-reg-distr} und dort unter \texttt{DISTR: 5
  Binomial\textbf{\color{red} c}df: SINGLE}

dann

$n=20$ bei $n$,

$p=0.85$ bei $p(\text{success})$ und

$k=17$ bei $x$ eintragen.



\begin{gesetz}{Kumulierte Wahrscheinlichkeit}{}
  $$P(X \le k) = \sum_{i=0}^k {n \choose i} \cdot{} p^i \cdot{} (1-p)^{n-i}$$


\begin{tabular}{rcl}
  $n$ &=& Anzahl Durchführungen (Baumtiefe)\\
  $p$ &=& Trefferwahrscheinlichkeit pro Durchführung\\
  $k$ &=& \textbf{Maximal} gewünschte Treffer\\
\end{tabular}

\end{gesetz}


\subsection*{Aufgaben}

\olatLinkGESOKompendium{5.7}{53ff}{ 36. a) b)  (gerechter Würfel) und 37. (gezinkter Würfel)}
\newpage

\subsection*{Aufgaben}

Zeichnen Sie in folgender Aufgabe beide Diagramme: Die
Wahrscheinlichkeiten für die genaue Anzahl Treffern, aber auch die
Wahrscheinlichkeiten für die kumulativen Wahrscheinlichkeiten:

\olatLinkGESOKompendium{5.7}{53}{36. c) (gezinkter Würfel)}

\newpage

\subsection{Lotto-Wahrscheinlichkeit}\index{Verteilung!Hypergeometrische}\index{Hypergeometrische Verteilung}\index{Lotto-Wahrscheinlichkeit}
In einer Urne liegen sieben Kugeln. Drei davon sind Treffer (grün) und vier davon sind Nieten (schwarz).
Wir dürfen zwei Kugeln blind herausnehmen (ohne diese wieder zurückzulegen).

\bbwCenterGraphic{5cm}{geso/stoch/img/Urne3T4N.png}

Wie groß ist nun die Wahrscheinlichkeit
\begin{itemize}
\item beide Treffer zu erzielen ($t=2$),
\item genau einen Treffer zu erzielen ($t=1$) oder
\item gar keinen Treffer zu erzielen ($t=0$)?
\end{itemize}

Diese Wahrscheinlichkeiten können wir nun einerseits durch ein Baumdiagramm lösen, indem wir die Treffer mit \textbf{\color{ForestGreen}T} und die Nieten mit \textbf{\color{red}N} bezeichnen:

\noTRAINER{\mmPapier{7.2}}
\TRAINER{
\bbwCenterGraphic{7cm}{geso/stoch/img/Baum2aus3T4N.png}
}

Somit erhalten wir für zwei Treffer ($t=2$) eine Wahrscheinlichkeit von $\frac{6}{42}$; für einen Treffer ($t=1$) die Summe aus $\frac{12}{42}$ und nochmals $\frac{12}{42}$ also total $\frac{24}{42}$ und schlussendlich für keinen Treffer ($t=0$) die Wahrscheinlichkeit $\frac{12}{42}$.
Die Summe aller Wahrscheinlichkeiten muss dann immer gleich 1.0 (100\%) sein.
\newpage

Andererseits gibt es dazu auch eine Formel, die
\textbf{Lotto-Wahrscheinlichkeit} oder «\textit{hypergeometrische
  Verteilung}»\footnote{Die Folgeglieder der \textit{geometrischen
    Verteilung} $P(X=t) = p\cdot{}q^{t-1}$ verhalten sich wie eine
  geometrische Reihe (die Quotienten von zwei Folgengliedern $P(X=t) :
  P(X=t+1)$ sind jeweils konstant). Da die hier betrachtete Verteilung
  noch rascher ``abfällt'', wie die \textit{geometrische Verteilung},
  so sprechen wir hier von \textbf{hyper}geometrisch.}.

\begin{beispiel}{Lotto-Wahrscheinlichkeit}{}
  In obiger Urne mit \textbf{drei} Treffern und \textbf{vier} Nieten
  ziehen wir genau \textbf{zwei} Kugeln heraus. Wie groß ist die
  Wahrscheinlichkeit auf zwei verschiedene Kugeln?
\end{beispiel}



Es seien:
\begin{itemize}
\item $T$ die Anzahl der Treffer in der Urne. Hier die Anzahl der grünen Kugeln (also $T = 3$)
\item $N$ die Anzahl der Nieten in der Urne. Hier die Anzahl schwarze Kugeln ($N=4$)
\item $T+N$ die Anzahl Elemente (Treffer + Nieten) in der Urne (hier $T+N = 7$)

\item $t$ die Anzahl der gewünschten bzw. zu erreichenden Treffer. Hier \zB $t=1$.
  
\item $n$ die Anzahl «gewünschten» Nieten (hier z. B. $n = 1$)
\item $t+n$ die Anzahl total herausgezogener Kugeln (hier $t+n=1+1=2$)
\end{itemize}

\newpage


\begin{gesetz}{}{}
$$P(X=t) = \frac{ {T\choose t} \cdot{} {{N}\choose {n}}}{{{T+N} \choose {t+n}}}$$
\end{gesetz}

\textbf{Beispiele} mit obigen sieben Kugeln (drei Treffer und vier
Nieten)
\leserluft

$$T=\LoesungsRaum{3}, N=\LoesungsRaum{4}, T+N=\LoesungsRaum{7}$$


\textbf{Beispiel 1}: Ich will bei zwei gezogenen Kugeln zwei
verschiedene Kugeln, also \textbf{genau einen Treffer}:

($t=1$) mit zwei Ziehungen ($t+n=2$), somit eine Niete ($n=1$).
Wir erhalten:

$$P(X=1) = \frac{ {T\choose t} \cdot{} {{N}\choose {n}}}{{{T+N}
    \choose {t+n}}} = \LoesungsRaumLang{\frac{ {3\choose 1} \cdot{} {{4}\choose {1}}}{{7
    \choose 2}} = \frac{3\cdot{4}}{21} = \frac{12}{21} \approx 0.57}$$

\textbf{Beispiel 2}: Ich will bei zwei gezogenen Kugeln \textbf{je einen Treffer}:

($t=2$) mit zwei Ziehungen ($t+n=2$), somit keine Niete ($n=0$).
Wir erhalten:

$$P(X=2) = \frac{ {T\choose t} \cdot{} {{N}\choose {n}}}{{{T+N}
    \choose {t+n}}} = \LoesungsRaumLang{\frac{ {3\choose 2} \cdot{} {{4}\choose {0}}}{{7
    \choose 2}} = \frac{3\cdot{}1}{21} = \frac17 \approx 0.14}$$
\newpage

\subsection*{Aufgaben}

\aufgabenFarbe{
Robin kauft im Supermarkt 17 Artikel ein und scannt den Einkauf
selbständig. Robins fünfjähriges Kind «schmuggelt» nachträglich zwei
Tafeln Schokolade in die Einkaufstasche. Somit sind nun 19 Artikel in
der Einkaufstasche, wovon zwei nicht abgescannt wurden.
\\
Bei einer Kontrolle durch das Ladenpersonal werden jeweils fünf
Artikel wahllos aus den Einkaufstaschen herausgepickt und mit der
Abrechnung verglichen.
\\
a) Wie groß ist die Wahrscheinlichkeit, dass bei der Prüfung genau eine der nicht
abgescannten Schokoladentafeln dabei ist?
\\
b) Wie groß ist die Wahrscheinlichkeit, dass bei der Prüfung
mindestens eine der nicht abgescannten Schokoladentafeln dabei ist?
}%% end AufgabenFarbe

\TNT{8}{
  a) Lotto Wahrscheinlichkeit. $T=2; N=17; t=1; n=4$
  $$
        P(X=1) = \frac{{2 \choose 1} \cdot{} {17 \choose 4}}{{19 \choose 5}} =
\frac{70}{171}\approx  40.95\%$$


b) $$P(x > 1) = P(x=1) + P(x=2) = \frac{70}{171} +
    \frac{ {2 \choose 2} \cdot{} {17 \choose 3} }{{19 \choose 5}} = \frac{70}{171} + \frac{10}{171}
    = \frac{80}{171} \approx 46.78\% $$
}%% end TNT



\olatLinkGESOKompendium{5.5.3}{49}{25. (Herz-Karten), 26. (Smarties), 24. (Zahlenlotto) und optional
  27. (Euromillions)}

\newpage
