%% Grundlagen
%% 2020 - 08 - 03 φ
%%

\subsection{Laplace-Experiment}\index{Experiment!Laplace}\index{Laplace-Experiment}
Nach Pierre-Simon Laplace (1749-1827).

Ein fairer Spielwürfel wird geworfen. Jede Seite hat genau die selbe
Wahrscheinlichkeit. Wie groß ist nun die Wahrscheinlichkeit, mit einem
Wurf eine größere Zahl als eine 4 zu werfen?

\TNT{2.4}{Die günstigen Ergebnisse sind 5 und 6 (\textbf{zwei} Stück). Alle möglichen Ergebnisse: 1 bis 6 (\textbf{sechs} Stück). Wahrscheinlichkeit = $\frac{2}{6}=\frac13$}

Dabei handelt es sich um ein Laplace-Experiment:
\begin{itemize}
\item Jeder einzelne Ausgang (Ergebnis) hat die selbe Wahrscheinlichkeit
      (hier $p = \frac{1}{6}$).
\item Die Wahrscheinlichkeit, dass ein Ereignis (hier $E = \epsdice{5}$
oder $\epsdice{6}$) eintritt, wird berechnet mit $P(E)
= \frac{\text{Anzahl gewünschte Ergebnisse}}{\text{Anzahl
mögliche Ergebnisse}}$.

Hier $P(\{\epsdice{5}\}\cup\{\epsdice{6}\}) = \frac{\left|\left\{\epsdice{5}, \epsdice{6}\right\}\right|}{|\Omega|}= \frac{|E|}{|\Omega|}=\LoesungsRaum{\frac{2}{6}}$.
\end{itemize}

\begin{definition}{Laplace Experiment}{}
  Zufallsexperimente, deren Elementarereignisse \textbf{alle gleich wahrscheinlich} sind, werden als
  \textbf{Laplace-Experimente} bezeichnet. Die Wahrscheinlichkeit berechnet sich wie folgt:
  $$P(E) = \frac{|E|}{|\Omega|}$$
\end{definition}

\begin{definition}{}{}
Mit Betragsstrichen $|.|$\index{Betragsstriche}\index{$\mid\cdot \mid$ s. Betrag(-striche)} wird die Anzahl Elemente in einer Menge angegeben.
\end{definition}


Kein Laplace-Experiment ist \zB das Werfen eines gezinkten Würfels,
bei dem die Augenzahl 6 häufiger auftritt als die anderen Augenzahlen.
\newpage
\subsection*{Aufgaben}
\olatLinkGESOKompendium{5.2}{42}{1. bis 3.}

\newpage




