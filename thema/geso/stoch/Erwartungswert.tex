%%
%% 2020 - 08 -03 φ
%%
\section{Erwartungswert}\index{Erwartungswert}

Ich schlage Ihnen folgendes Würfelspiel vor.

Sie Würfeln und erhalten dabei die Quadratzahl der erzielten
Würfelpunkte in CHF ausbezahlt. Wenn Sie \zB eine \epsdice{5} werfen,
so erhalten Sie CHF 25.-.

Natürlich müssen Sie einen Grundeinsatz bezahlen, damit Sie bei meinem
Spiel überhaupt mitmachen dürfen.

Drehen wir nun die Rollen einmal um und betrachten Sie sich selbst als
Betreiber dieses «Casinos», also dieses Spiels.

Wie viel CHF soll der Spieler pro Wurf als Einsatz bezahlen, sodass
dieses Spiel mit minimaler Wahrscheinlichkeit besser für den
Casino-Betreiber, also für Sie ausgeht?

\TNT{10}{
  Dazu berechnen wir zuerst das, was wir erwarten, wenn dieses Spiel
  oft gespielt wird. Jedes Würfelresultat wird mit $\frac16$
  Wahrscheinlichkeit auftreten. Somit muss jeder Gewinn ja nur mit
  $\frac16$ Wahrscheinlichkeit ausbezahlt werden.

  Wenn wir nun alle Rückzahlungen mit $\frac16$ multiplizieren und
  diese CHF-Werte zusammenzählen, erhalten wir

  $$\frac16\cdot{} (1 + 4 + 9 + 16 + 25 + 36) = 15.166...$$

  Diese Zahl nennen wir den \textbf{Erwartungswert}.

  In obigem Beispiele ist $\mu = 15.166...$ und die Zufallsvariable
  ordnet jedem Ergebnis (1..6) seinen auszuzahlenden Wert zu.

  Zufallsvariable hier:

  $$X(\omega) = 1.- \textrm{ für } \epsdice{1}; 4.- \textrm{ für }
  \epsdice{2} ; ...$$
  
  Wenn wir als Einsatz CHF 16.- verlangen, wird sich dieses Spiel im
  Endeffekt positiv für das «Casino» auswirken. Wenn wir hingegen nur
  CHF 15.- als Einsatz verlangen, wird früher oder später der Spieler
  mehr auf dem Konto haben.
}%% END TNT


\begin{definition}{Zufallsvariable}{}
	Eine Zufallsvariable $X(\omega)$ ordnet jedem Ergebnis eine Zahl
	(typischerweise den Gewinn oder Verlust) zu.
\end{definition}


\begin{definition}{Erwartungswert}{}
	Mit $\mu$ wird der \textbf{Erwartungswert} bezeichnet.
	$\mu$ ist das arithmetische Mittel, das die Zufallsvariable
	annimmt.
\end{definition}
\newpage
